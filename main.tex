\documentclass[11pt, a4paper, BCOR=10mm, DIV=11]{scrbook}
\linespread{1.25}
\usepackage[utf8]{inputenc}
\usepackage{graphicx}
%\usepackage[a4paper, margin=2.5cm]{geometry}
\usepackage{hyperref}
\usepackage{amsmath}
\usepackage{enumitem}
%\usepackage{mdframed}
\usepackage{amsthm}
\usepackage{amssymb}
\usepackage{imakeidx}
\makeindex[columns=3, title=Index, intoc]
\usepackage{mathtools}
%\usepackage{fdsymbol}
\usepackage{cite}
\graphicspath{ {images/} }
\usepackage{xcolor}         % Extended colors
\usepackage{color}         % Color extended names
\usepackage{nomencl}
\usepackage[nottoc]{tocbibind}
\usepackage{lipsum}
\usepackage[tickmarkheight=0.1cm, colorinlistoftodos]{todonotes}
\makenomenclature
\renewcommand{\nomname}{Notation Index}
\mathsurround=2pt
\usepackage[framemethod=TikZ]{mdframed}
\mdfsetup{%
   middlelinecolor=black,
   middlelinewidth=.5pt,
   backgroundcolor=gray!2,
   roundcorner=5pt}
\newsavebox{\selvestebox}
\newenvironment{takeaways}
  {
   \begin{lrbox}{\selvestebox}%
   \begin{minipage}{12.4cm}
     \textbf{Takeaways}
   }
  {\end{minipage}\end{lrbox}%
   \begin{center}
\setlength\fboxsep{.5cm}
   \colorbox[HTML]{F8E0E0}{\usebox{\selvestebox}}
   \end{center}}
% Theorem handle

\newtheorem{gtheorem}{Theorem}[chapter]
\newenvironment{theorem}
  {\begin{mdframed}\begin{gtheorem}}
  {\end{gtheorem}\end{mdframed}}
\newtheorem*{theorem*}{Theorem}
\newtheorem*{definition*}{Definition}
\newtheorem*{lemma*}{Lemma}
\newenvironment{ftheorem}
  {\begin{mdframed}\begin{gtheorem}}
  {\end{gtheorem}\end{mdframed}}
\newenvironment{ftheorem*}
  {\begin{mdframed}\begin{theorem*}}
  {\end{theorem*}\end{mdframed}}
\newtheorem{fassumption}{Assumption}
\newenvironment{assumption}
{\begin{mdframed}[
   backgroundcolor=red!2,
    ]
    \begin{fassumption}}
  {\end{fassumption}\end{mdframed}}

\newtheorem*{assumptions*}{Assumptions}
\newtheorem{corollary}{Corollary}[gtheorem]

\theoremstyle{definition}
\newtheorem{fdefinition}{Definition}[chapter]
\newenvironment{definition}
{\begin{mdframed}[
   backgroundcolor=red!2,
    ]
    \begin{fdefinition}}
  {\end{fdefinition}\end{mdframed}}
\newtheorem*{remark/}{Remark}
\newenvironment{remark}
  {\renewcommand{\qedsymbol}{$\diamondsuit$}%
   \pushQED{\qed}\begin{remark/}}
  {\popQED\end{remark/}}
\newtheorem{example/}{Example}[chapter]
\newenvironment{example}
  {\renewcommand{\qedsymbol}{$\diamondsuit$}%
   \pushQED{\qed}\begin{example/}}
  {\popQED\end{example/}}
\newtheorem*{example*/}{Example}
\newenvironment{example*}
  {\renewcommand{\qedsymbol}{$\diamondsuit$}%
   \pushQED{\qed}\begin{example*/}}
  {\popQED\end{example*/}}
\newtheorem*{reflection*/}{Reflection}
\newenvironment{reflection*}
  {\renewcommand{\qedsymbol}{$\spadesuit$}%
   \pushQED{\qed}\begin{reflection*/}}
  {\popQED\end{reflection*/}}
\newtheorem{problem}{Problem}
\newenvironment{fproblem}
  {\begin{mdframed}\begin{problem}}
  {\end{problem}\end{mdframed}}
  %%%% plain %%%%
\theoremstyle{plain}
\newtheorem{flemma}{Lemma}[chapter]
\newenvironment{lemma}
{\begin{mdframed}\begin{flemma}}
  {\end{flemma}\end{mdframed}}
\newtheorem{proposition}{Proposition}[chapter]
\newtheorem*{proposition*}{Proposition}
%\newtheorem{subassumption}{}[assumption]

\renewcommand{\proofname}{\textbf{Proof}}
% definition of constants

\newcommand{\CP}{C_\P}
\newcommand{\Ctau}{C_\tau}
\newcommand{\LearnRate}{\varepsilon_n}

%
\DeclarePairedDelimiterX{\inner}[2]{\langle}{\rangle}{#1, #2}
\newcommand{\C}{\mathbb{C}}
\newcommand{\G}{\mathbb{G}}
\newcommand{\E}{\mathbf{E}}
\renewcommand{\P}{\mathbf{P}}
\newcommand{\R}{\mathbb{R}}
\newcommand\norm[1]{\left\lVert#1\right\rVert}


\setlength {\marginparwidth }{2cm}

\begin{document}

\begin{titlepage}
 \begin{center}
       \vspace*{1cm}
       \textbf{
    A Novel Weighted Mean Approach to Estimate the Distribution Function of Potential Outcomes in Observational Studies 
       }

       \vspace{0.5cm}
       Asymptotic Analysis
            
       \vspace{1.5cm}

       \textbf{Ioan Scheffel}

       \vfill
            
       A thesis presented for the degree of\\
       Master of Science Mathematics
            
       \vspace{0.8cm}
     
     %  \includegraphics[width=0.4\textwidth]{university}
      
       supervised by PD Dr. Jürgen Dippon
       \\
            Institute for Stochastics and Applications
            \\
       Faculty 8: Mathematics and Physics
       \\
University of Stuttgart
\\
submitted at 
   \end{center}
\end{titlepage}
%\maketitle
\frontmatter

\pagenumbering{roman}
\newpage
\begin{center}
\textbf{
Eigenständigkeitserklärung
}
\end{center}
Ich erkläre mit meiner Unterschrift,
dass ich diese Arbeit selbstständig verfasst habe und keine
anderen als die angegebenen Quellen benutzt habe.
Alle Stellen dieser Arbeit, die dem Wortlaut,
dem Sinn oder der Argumentation nach anderen Werken entnommen sind 
(einschließlich des World
Wide Web und anderer elektronischer Text- und Datensammlungen), habe ich unter Angabe der
Quellen vollständig kenntlich gemacht.
\newpage
  \begin{center}
  \textbf{Abstract} \\
  english
  \end{center}
  In this thesis I extend the balancing weights framework of \cite{Wang2019} to estimate the distribution function of potential outcomes.
  %
  I also suggest to balance basis functions of non-parametric partitioning estimates. 
  %
  This greatly simplifies the proofs and allows for rigorous mathematical treatment of the method.
  %
  The asymptotic analysis shows convergence of the error to a Gaussian process.
  %
  My findings allow to apply the functional delta method to plug-in estimators.
  This makes classical statistical methods such as quantile estimation, hypothesis testing or survival analysis accessible to causal inference in observational studies.
  %
  While the theoretical results are promising, this novel approach waits for testing in practice.
\newpage
  \begin{center}
  \textbf{Abstract}
  \\
  german
  \end{center}
  Dies ist ein Master Arbeit.

\tableofcontents 

\mainmatter
\pagenumbering{arabic}
\chapter{Introduction}
  How does action change an outcome?
How should I guide my actions towards a better outcome?
The first question is about causality, the second about ethics.

How do causality and ethics reflect on statistics?
If you have not spent much time thinking about study design, this is a good way to start: 
As an analyst, ask yourself “Who acted? Who assigned treatments?"
As researcher -- plan your study accurately. You can ask yourself “How do we act? How do we assign treatment? Can we act?"

Let's say, you gather a sample from a study population, assign treatment (but forget how you did it). Some units get the drug, others don't. Then the statistical analysis shows a strong correlation of treatment and outcome. You hurry to your supervisor. “How was treatment assigned", asks she. “I forgot", says you.
“How do you know your analysis is correct then?"
You show her the data and together find out, that all units that received treatment were significantly taller than the rest of the sample.
After all, is the drug or the height responsible for the change  in outcome?
You realise, that the data is worthless for answering this question.
But you are lucky: It is just grass and fertiliser you were studying.

You get a second chance. A new medication needs testing before it enters the market. 
A company shall recruit participants, but the board requires you to write an outline for the study.
You carefully  explain steps to minimize risks for participants. You include plans to meet other requirements of human research.
Then you have to decide how to assign treatment.
No hand waving this time. You talk to your supervisor.
“Last time, too many tall blades received fertiliser. The distribution of treatment was not really random..."
You decide to determine treatment status by the flip a fair coin.
You call the procedure 'randomization'.

Randomization does not work in important situations. Would you smoke if a coin tells you to? If you say yes - you are likely to be a smoker. The point is that forcing someone to smoke is unethical. But so is not investigating the risks of smoking.

Say, a professor is curious if the smoking habits of his students affect their grades. 
At a beginning break he observes the backyard through his field glasses.
His assistant gets to know his plans. He warns him. “Many students attend parties the night before exams. Maybe they are also more likely to smoke." “I shall see this for myself..." says the professor. He puts away the glasses. After a while, he visits the local club.
He talks to a few of his students. Some smoke, some don't. The chats are enjoyable. He thinks: “Some of best students celebrate \textbf{before} the exam."

I hope, by now it's clear that sometimes it's all about how treatment was assigned.
%You sit in a plain room. There are two chairs. You sit in one of them.
%A person enters the room. “Subject 3225057? The coin decided. You have to smoke!"
%On your way out, anguished cries hit you from another room.
%“Please! Don't make me smoke!"
%You feel mild anxiety and wake up. It was just a dream. But it recurs. You talk to a friend. “I dream about a randomized study about the effects of smoking. A cold, indifferent coin decides who smokes. The non-smokers suffer a lot." When you realise, that this is all non-sense, you feel silly. But your friend
%But you happen to be a professor at the university.
%One day, in your office, you lean out of the window.“Give me the field glasses please. I want to observe my students smoking in the backyard."“What for?" asks your assistant. “Let's see, if this affects their grades" you murmur and look through the eyepieces. The issue comes up again at lunch. “Did you consider that some of the students that smoke also go to a party the night before an exam."“Well, maybe I should. How about I see for myself?" The night before the next exam, you openly enter the place of the party. You find some of the students you deem smartest at the bar. You're having a conversation. But it's to loud in there and in the morning you have an headache.
%
%
%%randomized trials versus observational studies
%
%Is study design more important then statistical analysis?
%
%I think, they are at least equal. 
%
%But a bad analysis can be undone,
%whereas a bad design can not.
%
%You have to stick with the data.
%
%If you are not familiar with study design the distinction between randomized and observational study is helpful.
%
%If you read the literature and are unsure about the design of a study, ask for this terms.
%
%You are likely to find an answer.
%
%
%It is all about how we collect the data.
%
%Say, we want to test the effect of a drug in a study population.
%
%There usually are differences among the units of the study population.
%
%Some are more healthy than others.
%
%We form a treatment and control group, that is, one group takes the drug and the other doesn't.
%
%Then we compare the groups by their health. 
%
%Then a critcal review comes in. What do you mean by healthy.
%
%We mean this and that.
%
%It seems you did not consider this factor.
%
%Maybe the drug is not effective, but the effect we see in your analysis comes from something else.
%
%What do we answer to this?
%
%
%A good method to avoid this awkward situation is to randomize.
%
%For every unit of the population we toss a fair coin that decides if they get the drug.
%
%Now comes the critic.
%
%From the tables it seems there is an effect. But what about unknown influence?
%
%We answer: Does the coin now of them?
%
%It is not ideal, but this way you can prevent systematic damage to your analysis.
%
%
%What if we can't decide who gets treatment?
%
%Don't think treatment has to be something good, it should not carry any label of good or bad.
%
%But what about smoking?
%
%Would you smoke if a coin tells you to?
%
%So this is unethical.
%
%But it is also unethical not to investigate the effects of smoking on the health.
%
%Let's accept, that we sometimes (often?) can not control who gets treatment.
%
%Some smoke, some don't, and we mearly observe.
%
%This is typical example for an observational study.
%
%Honestly, this is an oversimplification, but I hope you get the point.
%
%Who still is insulted by the tone will maybe like\cite{Rubin2007}.
%
%%propensity score
%
%In \cite{Rubin2007} you will find the propensity score.
%
%The propensity score is the individual probability to receive treatment, that is,
%\begin{gather}
%  \P
%  [T=1|X]
%\end{gather}
%if $T$ is the random variable that decides abuot treatment and $X$ is the vector that carries your individual information.
%
%This concept goes back to \cite{Rosenbaum1983}.
%
%It is maybe worth to stop here and think about this definition and its connection to the two study designs.
%
%Discover it for yourself.
%
%\begin{reflection*}
%What is the propensity score in the above example.
%How does the propensity score behave in rs and os?
%\end{reflection*}
%
%
%
%
%
%

\chapter{The Optimization Problem behind the Weights}
  Let $(T_1,X_1),\ldots,(T_N,X_N)$ be independent and identically-distributed copies of $T$ and $X$ (see Introduction page). 
We gather them in the (random) data set 
\begin{gather*}
D_N:=\left\{ (T_i,X_i)\colon i\in \left\{ 1,\ldots,N \right\} \right\}
\,.
\end{gather*}
Furthermore, let
\begin{gather*}
  n
  \ 
  :=
  \ 
  \# 
  \left\{ 
    i\in \left\{ 1,\ldots,N \right\}
    \colon
    T_i=1
  \right\}
\end{gather*}
be the number of treated units. This is a random variable. We assume the order $T_i=1$ for all $i\le n$.

Let
$\# A\in\mathbb{N}_0$ be the number of elements of a finite set $A$, let
$\overline{\R}:=\R\cup \left\{ \infty \right\}$ 
be the extended real numbers, 
$\Sigma_\mathcal{X}$ be a $\sigma$-algebra on the covariate space $\mathcal{X}$,
and let $\mathcal{B}(\R)$ denote the Borel-$\sigma$-algebra on the real numbers.

For a convex function $\varphi\colon \R\to\overline{\R}$,
a finite set of measurable (basis-)functions on the covariate space
\begin{gather*}
  \mathfrak{B}
  :
  =
  \left\{ 
B_k\colon (\mathcal{X},\Sigma_\mathcal{X})\to(\R,\mathcal{B}(\R))
\ 
|
\ 
k\in \left\{ 1,\ldots,\# \mathfrak{B} \right\}
  \right\}
\end{gather*}
and a constraints vector $\delta=[\delta_1,\ldots,\delta_{\# \mathfrak{B}}]^\top>0$
we consider the following (random) convex optimization problem.
\newpage
\begin{fproblem}
  \label{bw:1:primal}
\begin{align*}
  %%%% objective %%%%
    &\underset{w_1, \ldots, w_n \in \R}
    {\text{minimize}}
    &&\qquad\qquad
    \sum_{i = 1}^{n} 
    \varphi(w_i)
    &&&
    \\
    %%%% w_i T_i >= 0 %%%%
    &\text{subject to}
    &&\qquad\qquad
    w_i 
    \ge
    0
    &&&
    \qquad
    \text{for all}\ 
    i \in \left\{ 1, \ldots, n \right\}
    \,,
    \\
    %%%% 1/n sum w = 1 %%%%
    & 
    &&\qquad\qquad
    \frac{1}{N}
    \sum_{i=1}^{n} 
    w_i
    =1
    \\
    %%%% box constraints %%%%
    & 
    &&\qquad
    \left| 
      \frac{1}{N} 
      \left( 
      \sum_{i = 1}^{n} 
      w_i
      B_k(X_i)
      -
      \sum_{i=1}^{N} 
      B_k(X_i)
      \right)
    \right|
    \ 
    \le 
    \ 
    \delta_k
    &&&
    \qquad
    \text{for all}\ 
    k \in \left\{ 1, \ldots, \# \mathfrak{B} \right\}
    \,.
\end{align*}
\end{fproblem}
What is random in Problem~\ref{bw:1:primal}?
First, the dimension of the search space $(w\in\R^n)$ depends on the random variable $n$. 
Thus, we only compute weights for the treated units (the ones with $T_i=1$).
Next consider the \textbf{objective function}
\begin{gather*}
    \sum_{i = 1}^{n} 
    \varphi(w_i)
    \,.
\end{gather*}
The number of summands is random (again $n$). We sometimes use the equivalent notation
\begin{gather*}
    \sum_{i = 1}^{N} 
    T_i
    \cdot
    \varphi(w_i)
    \,,
\end{gather*}
where we set the weights of the untreated (the ones with $T_i=0$) to some arbitrary value in the domain of $\varphi$.
Let's consider the constraints. There is no randomness in the first two constraints.
\begin{gather*}
    w_i 
    \ge
    0
    \qquad
    \text{for all}\ 
    i \in \left\{ 1, \ldots, n \right\}
    \quad
    \text{and}
    \quad
    \frac{1}{N}
    \sum_{i=1}^{n} 
    w_i
    =1
    \,.
\end{gather*}
They only make sure, that the weights (divided by $N$) form a convex combination.
If, for example, the outcome space $\mathcal{Y}$ is convex we make sure that a weighted-mean-estimate of $\E[Y(1)]$ satisfies
\begin{gather*}
  \widehat{Y}(1) 
  \ 
  :=
  \ 
  \frac{1}{N}
  \sum_{i=1}^{n} 
  w_i\cdot Y_i
  \ 
  \in
  \ 
  \mathcal{Y}
\end{gather*}
or that a weighted-mean-estimate of the distribution function of $Y(1)$ satisfies
\begin{gather*}
  \widehat{F}_{Y(1)} 
  \ 
  :=
  \ 
  \frac{1}{N}
  \sum_{i=1}^{n} 
  w_i\cdot \mathbf{1}\left\{ Y_i\le z \right\}
  \ 
  \in
  \ 
  [0,1]
  \,.
\end{gather*}
There remain the box constraints
\begin{gather*}
    \left| 
      \frac{1}{N} 
      \left( 
      \sum_{i = 1}^{n} 
      w_i
      B_k(X_i)
      -
      \sum_{i=1}^{N} 
      B_k(X_i)
      \right)
    \right|
    \ 
    \le 
    \ 
    \delta_k
    \qquad
    \text{for all}\ 
    k \in \left\{ 1, \ldots, \# \mathfrak{B} \right\}
    \,.
\end{gather*}
Here the number of summands in
\begin{gather*}
      \sum_{i = 1}^{n} 
      w_i
      B_k(X_i)
\end{gather*}
is again random, and we sometimes use the equivalent notation
\begin{gather*}
      \sum_{i = 1}^{N} 
      T_i
      \cdot
      w_i
      \cdot
      B_k(X_i)
      \,.
\end{gather*}
The (basis-)functions in $\mathfrak{B}$ can be random functions, for example, if they depend on the data $D_N$.
Also the constraint vector $\delta$ can depend on the data (see \cite[Algorithm~1 on page 11]{Wang2019}).


Next, we formulate and discuss assumptions on (the core of) the objective function $\varphi$.
\subsection{Objective Function}
\begin{assumption}
  \label{asu:objective_f}
  The objective function $\varphi\colon \R\to\overline{\R}$ of Problem~\ref{bw:1:primal} 
  satisfies the following conditions
  \begin{itemize}
    \item
      $\varphi(0)=0$ and $\varphi(x)=\infty$ for all $x<0$
    \item
      $\varphi$ is strictly convex and continuously differentiable on $(0,\infty)$ with derivative $\varphi^{'}$
    \item
      $\varphi^{'}(0,\infty)=\R$
    \item
      the inverse of the derivative $(\varphi^{'})^{-1}$ is Lipschitz continuous and continuously differentiable on $\R$.
  \end{itemize}
\end{assumption}
The next lemma provides a link to the assumptions on the objective function in Theorem~\ref{cv:ts:th}.
\newpage
\begin{lemma}
  \label{1065}
  Let Assumption\ref{asu:objective_f} hold true. Then the convex conjugate of $\varphi$ (see \eqref{def:convex_conjugate}) is
  \begin{gather*}
    \varphi^*
    \colon
    \R
    \ 
    \to
    \ 
    \R
    \,,
    \quad
    x^*
    \ 
    \mapsto
    \ 
    x^*
    \!
    \cdot
    (
    \varphi^{'}
    )^{-1}
    (x^*)
    \ 
    -
    \ 
    \varphi
    \left( 
      (
    \varphi^{'}
    )^{-1}
    (x^*)
    \right)
    \,.
  \end{gather*}
  Furthermore, $\varphi^*$ is strictly convex and continuously differentiable on $\R$.
\end{lemma}
\begin{proof}
We define
\begin{gather*}
 \phi
 \ 
 \colon
 [0,\infty)
 \times
 \R
 \ 
 \to
 \ 
 \R
 \,,
 \quad
 (x,x^*)
 \ 
 \mapsto
 \ 
 x\cdot x^*
 \ 
 -
 \ 
 \varphi(x)
 \,.
\end{gather*}
Let $x^*\in\R$.
Since
(by assumption)
      $\varphi$ is continuously differentiable on $(0,\infty)$ with derivative $\varphi^{'}$,
      so is $\phi(\cdot,x^*)$ with derivative
      satisfying 
      \begin{gather*}
        \frac{\partial}{\partial x}
        \phi(x,x^*)
        \ 
        =
 \ 
        x^*
        -
        \varphi^{'}(x)
        \qquad
        \text{for all}\ 
        x\in(0,\infty)
        \,.
      \end{gather*}
  It follows that 
  \begin{gather*}
    z
    \ 
    :=
    \ 
      (
    \varphi^{'}
    )^{-1}
    (x^*)
  \end{gather*}
  is an extreme value of $\phi(\cdot,x^*)$.
  Note, that $x^*\in\R$ is in the domain of 
  $
      (
    \varphi^{'}
    )^{-1}
  $
  by the assumption $\varphi^{'}(0,\infty)=\R$.
  Since $\varphi$ is strictly convex, $\phi(\cdot,x^*)$ is strictly concave. 
  Thus,
  $z>0$ is the unique maximum in $(0,1)$.
  By the continuity of $\phi(\cdot,x^*)$ on $[0,\infty)$ it follows, that $z$ is the unique maximum on $[0,\infty)$.
  Thus
  \begin{align*}
    \varphi^*(x^*)
    &
    \ 
    =
    \ 
    \sup_{x\in\R}
    x\cdot x^* - \varphi(x)
    \ 
    =
    \ 
    \sup_{x\in [0,\infty)}
    x\cdot x^* - \varphi(x)
    \ 
    =
    \ 
    \sup_{x\in [0,\infty)}
    \phi(x,x^*)
    \\
    &
    \ 
    =
    \ 
    \phi(z,x^*)
    \\
    &
    \ 
    =
    \ 
    x^*
    \!
    \cdot
    (
    \varphi^{'}
    )^{-1}
    (x^*)
    \ 
    -
    \ 
    \varphi
    \left( 
      (
    \varphi^{'}
    )^{-1}
    (x^*)
    \right)
    \qquad 
    \text{for all}\ 
    x^*\in\R
    \,.
  \end{align*}
  Now we proof the second statement.
  Since
  $
      (
    \varphi^{'}
    )^{-1}
  $
  is (by assumption) continuously differentiable, it holds
  \begin{align}
    \label{0098}
    \begin{split}
    \frac{\partial}{\partial x^*}
     \varphi^*(x^*)
    &
    \ 
    =
    \ 
    (
    \varphi^{'}
    )^{-1}
    (x^*)
    \ 
    +
    \ 
    x^*
    \!
    \cdot
    \frac{\partial}{\partial x^*}
    (
    \varphi^{'}
    )^{-1}
    (x^*)
    \ 
    -
    \ 
    \varphi^{'}
    \left( 
      (
    \varphi^{'}
    )^{-1}
    (x^*)
    \right)
    \cdot
    \frac{\partial}{\partial x^*}
    (
    \varphi^{'}
    )^{-1}
    (x^*)
    \\
    &
    \ 
    =
    \ 
    (
    \varphi^{'}
    )^{-1}
    (x^*)
    \ 
    +
    \ 
    x^*
    \!
    \cdot
    \frac{\partial}{\partial x^*}
    (
    \varphi^{'}
    )^{-1}
    (x^*)
    \ 
    -
    \ 
    x^*
    \cdot
    \frac{\partial}{\partial x^*}
    (
    \varphi^{'}
    )^{-1}
    (x^*)
    \\
    &
    \ 
    =
    \ 
    (
    \varphi^{'}
    )^{-1}
    (x^*)
    \qquad
    \text{for all}\ 
    x^*\in\R
    \,.
    \end{split}
  \end{align}
  Since $\varphi$ is strictly convex and continuously differentiable, 
  $\varphi^{'}$ is continuous and strictly non-decreasing.
  Thus 
  $
    (
    \varphi^{'}
    )^{-1}
  $
  is continuous and strictly non-decreasing.
  It follows from \eqref{0098} that $\varphi^*$ is strictly convex and continuously differentiable.
\end{proof}
The next lemma completes the link.
\begin{lemma}
  Let Assumption~\ref{asu:objective_f} hold true. Then 
\begin{gather*}
  \Phi
  \ 
  :
  \ 
  \R^n
  \to
  \ 
  \overline{\R}
  \,
  ,
  \qquad
  [w_1,\ldots,w_n]^\top
  \ 
  \mapsto
  \ 
  \sum_{i=1}^n \varphi(w_i)
  \,,
\end{gather*}
satisfies Assumption~\ref{cv:ts:asu}.
\end{lemma}
\begin{proof}
  By Example~\ref{cv:cc:ex}
  the convex conjugate of $\Phi$ is 
\begin{gather*}
  \Phi^*
  \ 
  :
  \ 
  \R^n
  \to
  \ 
  \overline{\R}
  \,
  ,
  \qquad
  [\lambda_1,\ldots,\lambda_n]^\top
  \ 
  \mapsto
  \ 
  \sum_{i=1}^n \varphi^*(\lambda_i)
  \,,
\end{gather*}
where $\varphi^*$ is the convex conjugate of $\varphi$.
By Assumption~\ref{asu:objective_f} $\varphi$ is strictly convex. Thus,
$\Phi$ is strictly convex. By Lemma~\ref{1065}, $\varphi^*$ continuously differentiable on $\R$. Thus,
$\Phi$ is continuously differentiable on $\R^n$.
It follows the statement of Assumption~\ref{cv:ts:asu} for $\Phi$.
\end{proof}


Next we discuss some concrete choices of $\varphi$

\begin{example}
  For two discrete distributions
  \begin{gather*}
    p:=[p_1,\ldots,p_N]
    \qquad
    \text{and}
    \qquad
    q:=[q_1,\ldots,p_N]
  \end{gather*}
  we consider the following distance measure
  \begin{gather*}
   D(p|q)
   \ 
   :=
   \ 
   \sum_{i=1}^{N} 
   p_i
   \cdot
   \log
   \left( 
     \frac{p_i}{q_i}
   \right)
   \,.
  \end{gather*}
  This is known as the Kullback-Leibler-Entropy.
  In \cite[§3.1]{Hainmueller2012} the author connects this concept to a convex optimization problem.
  The idea is, to optimize the Kullback-Leibler-Entropy of the distribution induced by the weights and some base weights.
  For example, if we choose
  \begin{gather*}
    w:=
    \frac{1}{N}[w_1\cdot T_1,\ldots,w_N\cdot T_N]
    \qquad
    \text{and}
    \qquad
    q:=\frac{1}{N}[1,\ldots,1]
  \end{gather*}
  we get
  \begin{align}
    \label{8926}
   D(w|q)
   \ 
   =
   \ 
   \frac{1}{N}
   \sum_{i=1}^{N} 
w_i\cdot T_i
   \cdot
   \log
   \left( 
     \frac{w_i\cdot T_i}{N}\cdot N
   \right)
   \ 
   =
   \ 
   \frac{1}{N}
   \sum_{i=1}^{n} 
w_i
   \cdot
   \log
   \left( 
     w_i
   \right)
   \,,
  \end{align}
  where we set $"0\cdot \log(0)"=0$. Thus, the optimization problem
  \begin{gather*}
    \underset{w_1, \ldots, w_n \in \R}
    {\text{minimize}}
    \qquad\qquad
    \sum_{i = 1}^{n} 
w_i
   \cdot
   \log
   \left( 
     w_i
   \right)
  \end{gather*}
  produces the same optimal solutions as minimizing the Kullback-Leibler-Entropy \eqref{8926} with respect to $w$.
Thus, we consider 
\begin{gather*}
  \varphi
  \ 
  \colon
  \R
  \ 
  \to
  \ 
  \overline{\R}
  \,,
  \qquad
  x
  \ 
  \mapsto
  \ 
  \begin{cases}
    x\cdot \log x\ &\text{if}\ x>0\,, \\
    0 \ &\text{if}\, x=0\,,\\
    \infty\ &\text{if}\ x<0\,.
  \end{cases}
\end{gather*}
We show, that this choices satisfies Assumption~\ref{asu:objective_f}.
By definition it holds \textit{(i)}.
For \textit{(ii)} note, that the second derivative on $(0,\infty)$ is $x\mapsto 1/x$. Thus (by the second derivative test), $\varphi$ is strictly convex.
Clearly it is also twice continuously differentiable.
For the continuity in $0$ note, that $\lim_{x\to 0} x\log x=0$. 
For \textit{(iii)} note, that $x\mapsto \log x$ is continuous and strictly non-decreasing on $\R$. 
Since 
\begin{align*}
  \lim_{x\to 0}\log x\ =\ -\infty
  \qquad
  \text{and}
  \qquad
  \lim_{x\to \infty}\log x\ =\ \infty
  \,,
\end{align*}
and $\varphi^{'}=(x\mapsto \log x +1)$ on $(0,\infty)$, it follows \textit{(iii)}. 
Finally, it holds $(\varphi^{'})^{-1}=(x\mapsto \exp(x-1))$ on $\R$. 
Thus, it follows \textit{(iv)}.
\end{example}
\begin{example}
  In a similar setting, the authors of \cite{Zubizarreta2015} choose the sample variance of the weights as objective function, that is,
\begin{gather*}
  \varphi
  \ 
  \colon
  \R
  \ 
  \to
  \ 
  \overline{\R}
  \,,
  \qquad
  x
  \ 
  \mapsto
  \ 
  \begin{cases}
    \left(
      x-\frac
      {1}
      {n}
    \right)^2\ &\text{if}\ x\ge0\,, \\
    \infty\ &\text{if}\ x<0\,.
  \end{cases}
\end{gather*}
\end{example}



Next, we derive the dual formulation of Problem~\ref{bw:1:primal}.
\subsection{Dual Problem}
\begin{lemma}
  \label{matrix_notation}
  A matrix formulation of Problem~\ref{bw:1:primal} is 
\begin{align}
  \label{cv:ts:primal}
  %%%% objective %%%%
    &\underset{w \in \R^n}
    {\mathrm{minimize}}
    &&\qquad\qquad
    \Phi(w)
    &&&
    \\
    %%%% Ax >= b %%%%
    \nonumber
    &\mathrm{subject}\ \mathrm{to} 
    &&\qquad\qquad
    \mathbf{U}w
    \ 
    \ge
    \ 
    d
    \,,
    \\
    \nonumber
    &
    &&\qquad\qquad
    \mathbf{A}w
    \ 
    =
    \ 
    a
    \,,
\end{align}
with objective function
\begin{gather*}
  \Phi
  \ 
  :
  \ 
  \R^n
  \to
  \ 
  \overline{\R}
  \,
  ,
  \qquad
  [w_1,\ldots,w_n]^\top
  \ 
  \mapsto
  \ 
  \sum_{i=1}^n \varphi(w_i)
  \,,
\end{gather*}
inequality matrix and vector
\begin{alignat*}{2}
    \mathbf{U}
    &
    \ 
    :=
    \ 
    \begin{bmatrix}
      \mathbf{I}_n
      \\
      \pm\,\mathbf{B}(\mathbf{X})
    \end{bmatrix}
    \in
    \R^{(n+  2 N)\times n}
        \qquad
    &&
d
    \ 
    :=
    \ 
    \begin{bmatrix}
      0_n
      \\
      -N\cdot\delta 
      \ 
      \pm\ 
      \sum_{i = 1}^{N} B(X_i)
    \end{bmatrix}
    \in
    \R^{n+  2 N}
    \,,
    \intertext{and equality matrix and vector}
    \mathbf{A}
    &
    \ 
    :=
    \ 
      \mathrm{1}_n
      ^\top
      \in\R^{1\times n}
      \qquad
    &&
    a
  \ 
    :=
    \ 
    N
    \in\mathbb{N}
    \,.
\end{alignat*}
\end{lemma}

\begin{proof}
  Recall that the box constraints of Problem~\ref{bw:1:primal} are
  \begin{gather*}
        \left| 
      \frac{1}{N} 
      \left( 
      \sum_{i = 1}^{n} 
      w_i
      B_k(X_i)
      -
      \sum_{i=1}^{N} 
      B_k(X_i)
      \right)
    \right|
    \ 
    \le 
    \ 
    \delta_k
    \qquad
    \text{for all}\ 
    k\in \left\{ 1,\ldots, N \right\}
    \,.
  \end{gather*}
  Put differently, it holds both
  \begin{align*}
    -
      \sum_{i = 1}^{n} 
      w_i
      B_k(X_i)
    \ge 
    -
    N
    \delta_k
      -
      \sum_{i=1}^{N} 
      B_k(X_i)
      \quad 
    \text{and}
      \quad
      \sum_{i = 1}^{n} 
      w_i
      B_k(X_i)
    \ge 
    -
    N
    \delta_k
      +
      \sum_{i=1}^{N} 
      B_k(X_i)
  \end{align*}
  for all 
  $
    k\in \left\{ 1,\ldots, N \right\}
  $. In matrix notation this is 
  \begin{gather*}
    \pm\mathbf{B}(\mathbf{X})w
    \ 
    \ge
    \ 
    [d_{n+1},\ldots, d_{n+  2 N}]^\top
    \,.
  \end{gather*}
  Proving the rest of the statements is straightforward. We omit the details.
\end{proof}
\begin{remark}
  The inequality constraints of
  Lemma~\ref{matrix_notation} differ from its counterpart
  \cite[Proof of Lemma~1]{Wang2019}.
  We don't transform the variable $w$, but shift to $d$ what prevents us from keeping $w$.
  Note, that the choice of
  \cite[Proof of Lemma~1]{Wang2019} leads to a mistake on page 21.
  The mistake is most obvious in the second display, where the first implication follows from dividing by 0.
  I discussed this with the authors and proposed a version of Lemma\ref{matrix_notation} to solve the problem. I think it's best not to transform variables, because the mistake comes from (wrongly) calculating the convex conjugate of the (more complicated) transformed version of the objective function. The subsequent analysis even simplifies with my version.

  I was surprised to find the (exact) same mistake in the earlier paper 
  \cite[page 35 second display]{Chan2016}. 
  There is no reference in
  \cite[Proof of Lemma~1]{Wang2019} 
  to
  \cite{Chan2016}. Yet the formulation and the mistake are the same.
  Did the authors of \cite{Wang2019} (inadvertently?) plagiarize
  the mathematical analysis of 
  \cite{Chan2016}
  ?
\end{remark}

\begin{lemma}
  Consider the optimization problem
\begin{align}
  \label{9993}
  \begin{split}
  \underset
  {\begin{smallmatrix}
\rho\,,\, \lambda^+,\,\lambda^-\ge 0 \\
\lambda_0\in\R
  \end{smallmatrix}}
  {
    \mathrm{maximize}
  }
  \quad
  &
  -
\sum_{i=1} 
  ^n
    \,
  \varphi^*
  \!
  \left( 
    \rho_i
    +
\lambda_0
+
\inner
{B(X_i)}
{
\lambda^+
-
\lambda^-
}
  \right)
  \\
  &
+
\ 
\sum_{i=1}^{N} 
  \left( 
\lambda_0
+
\inner
{B(X_i)}
{
\lambda^+
-
\lambda^-
}
  \right)
  \,
  \ 
-
\ 
\inner
{\delta}
{
\lambda^+
+
\lambda^-
}
  \,.
  \end{split}
\end{align}
If Assumption~\ref{asu:objective_f} holds true 
and there exists an optimal solution 
$
(\rho^\dagger,\lambda_0^\dagger,\lambda^{+,\dagger},\lambda^{-,\dagger})
$
then the unique optimal solutions to Problem~\ref{bw:1:primal} are 
\begin{gather*}
  w^\dagger_i
  \ 
  :=
  \ 
  (
  \varphi^{'}
  )^{-1}
  \left(
    \rho^\dagger_i
  \ 
    +
  \ 
\lambda_0^\dagger
  \ 
+
  \ 
\inner
{B(X_i)}
{
  \lambda^{+,\dagger}
-
\lambda^{-,\dagger}
}
  \right)
  \qquad
  \text{for all}\ 
  i\in
  \left\{ 1,\ldots,n \right\}
  \,.
\end{gather*}
\end{lemma}
\begin{proof}
  By Lemma~\ref{matrix_notation},
  Problem~\ref{bw:1:primal} has the form required in Theorem~\ref{cv:ts:th}.
  By Assumption~\ref{asu:objective_f} and Lemma~\ref{9991} the objective function $\Phi$ of Problem~\ref{bw:1:primal}
  satisfies Assumption~\ref{cv:ts:asu}.
  Thus we can apply
  Theorem~\ref{cv:ts:th} to Problem~\ref{bw:1:primal}.
  Calculations yield the result.
\end{proof}
The next Theorem aims at simplifying this result. 
\newpage
\begin{ftheorem}
  \label{dual_solution_th}
  Consider the optimization problem
\begin{align}
  \label{dual}
  \begin{split}
  \underset
  {\begin{smallmatrix}
      \rho&\in&&\R^N 
      \\
      \lambda_0 & \in&&\R
      \\
      \lambda&\in&&\R^{N}
  \end{smallmatrix}}
  {
    \mathrm{minimize}
  }
  \quad
  \frac{1}{N}
\sum_{i=1} 
  ^N
  &
  \Big[
  T_i
  \cdot
  \varphi^*
  \!
  \left( 
    \rho_i
    +
\lambda_0
+
\inner
{B(X_i)}
{
\lambda
}
  \right)
  \ 
  -
  \ 
\lambda_0
-
\inner
{B(X_i)}
{
\lambda
}
\Big]
  \ 
+
\ 
\inner
{\delta}
{
  |\lambda|
}
  \,,
  \\
  \mathrm{subject}\ \mathrm{to}
  \quad
  \qquad
  &
  \rho_i \ge 0 
  \quad 
  \mathrm{for}\ \mathrm{all}\ i\le n
  \qquad 
  \mathrm{and}
  \qquad
  \rho_i=0
  \quad 
  \mathrm{for}\ \mathrm{all}\ i>n
  \,.
\end{split}
\end{align}
If Assumption~\ref{asu:objective_f} holds true 
and there exists an optimal solution 
$
(\rho^\dagger,\lambda_0^\dagger,\lambda^\dagger)
$
then the unique optimal solutions to Problem~\ref{bw:1:primal} are 
\begin{gather*}
  w^\dagger_i
  \ 
  :=
  \ 
  (
  \varphi^{'}
  )^{-1}
  \left(
    \rho^\dagger_i
  \ 
    +
  \ 
\lambda_0^\dagger
  \ 
+
  \ 
\inner
{B(X_i)}
{
\lambda^{\dagger}
}
  \right)
  \qquad
  \text{for all}\ 
  i\in
  \left\{ 1,\ldots,n \right\}
  \,.
\end{gather*}
\end{ftheorem}

\begin{proof}
  Assume that
$
  (\rho^\dagger,\lambda_0^\dagger,\lambda^{+,\dagger},\lambda^{-,\dagger})
$
is an optimal solution to Problem~\ref{9993}.
We write
\begin{align*}
  G
  (\rho,\lambda_0,\lambda^+,\lambda^-)
  &
  \ 
  :=
  \ 
 -
\sum_{i=1} 
  ^n
    \,
  \varphi^*
  \!
  \left( 
    \rho_i
    +
\lambda_0
+
\inner
{B(X_i)}
{
\lambda^+
-
\lambda^-
}
  \right)
  \\
  &
  \qquad
+
\ 
\sum_{i=1}^{N} 
  \left( 
\lambda_0
+
\inner
{B(X_i)}
{
\lambda^+
-
\lambda^-
}
  \right)
  \,
  \ 
-
\ 
\inner
{\delta}
{
\lambda^+
+
\lambda^-
}
  \,.
\end{align*}
 To eliminate the remaining constraints, 
  we paraphrase \cite[pages~19-20]{Wang2019}.
  We show 
  for all $i \in \left\{ 1,\ldots,N \right\}$
\begin{alignat}{2}
  \notag
  \text{either}
  &
  &&
  \qquad
  \lambda_i^{+,\dagger} > 0
  \\
  \label{9992}
  \text{or}
  &
  &&
  \qquad
  \lambda_i^{-,\dagger} > 0
  \,.
\end{alignat}
Assume towards a contradiction that 
\begin{gather}
  \label{1232}
  \text{
there exists
  } 
  \ 
i \in \left\{ 1,\ldots,N \right\}
\ 
\text{such that}
\qquad
  \lambda_i^{+,\dagger} > 0
  \qquad 
  \text{and}
  \qquad
  \lambda_i^{-,\dagger} > 0
  \,.
\end{gather}
Consider
  \begin{align*}
    \tilde{\lambda}^{+,\dagger}
    &
    \ 
    :=
    \ 
    \begin{bmatrix}
      \ 
      \lambda_1^{+,\dagger}
      \ldots,
      \ 
      \lambda_i^{+,\dagger}
      \!
      \ 
      -
      \ 
      (
      \lambda_i^{+,\dagger}
      \!
      \land
      \lambda_i^{-,\dagger}
      )\,,
      \ 
      \ldots,
      \lambda_{N}^{+,\dagger}
    \end{bmatrix}
    ^\top
    \intertext{and}
    \tilde{\lambda}^{-,\dagger}
    &
    \ 
    :=
    \ 
    \begin{bmatrix}
      \ 
      \lambda_1^{-,\dagger}
      \ldots,
      \ 
      \lambda_i^{-,\dagger}
      \!
      \ 
      -
      \ 
      (
      \lambda_i^{+,\dagger}
      \!
      \land
      \lambda_i^{-,\dagger}
      )\,,
      \ 
      \ldots,
      \lambda_{N}^{-,\dagger}
    \end{bmatrix}
    ^\top
    \,.
  \end{align*}
  Since
  \begin{gather*}
      \lambda_i^{\pm,\dagger}
      \!
      \ 
      -
      \ 
      (
      \lambda_i^{+,\dagger}
      \!
      \land
      \lambda_i^{-,\dagger}
      )
      \ 
      \ge 
      \ 
      0
      \,,
  \end{gather*}
  the perturbed vectors $\tilde{\lambda}^{\pm,\dagger}$ are  in the domain of the 
  optimization problem.
  By Assumption~\eqref{1232} and $\delta>0$ it follows
  \begin{align*}
  G
  \left( 
  \rho^\dagger,\lambda_0^\dagger,\tilde{\lambda}^{+,\dagger},\tilde{\lambda}^{-,\dagger}
  \right)
  \ 
  -
  \ 
  G
  \left( 
  \rho^\dagger,\lambda_0^\dagger,\lambda^{+,\dagger},\lambda^{-,\dagger}
  \right)
  \ 
  =
  \ 
  2
  \cdot
  \delta_i
  \cdot
      (
      \lambda_i^{+,\dagger}
      \!
      \land
      \lambda_i^{-,\dagger}
      )
  \ 
  >
  \ 
  0
  \,,
  \end{align*}
  which contradicts the optimality of
$
  (\rho^\dagger,\lambda^{+,\dagger},\lambda^{-,\dagger},\lambda_0^\dagger)
$
(it is supposed to be a maximum in the domain of the optimization problem)
.
It follows \eqref{9992}.
But then 
$
\lambda^{\pm,\dagger}_i
\ge 0
$
collapses to
$
\lambda_i^\dagger\in \R
$ 
for all
$i\in \left\{ 0,\ldots,N \right\}$, that is, we set
\begin{gather*}
 \lambda_i^\dagger
 \ 
 =
 \ 
 \lambda_i^{+,\dagger}
 \ 
 -
 \ 
 \lambda_i^{-,\dagger}
 \qquad
 \text{and}
 \qquad
|\lambda_i^\dagger|
\ 
=
\ 
\lambda_i^{+,\dagger}
\ 
+
\ 
\lambda_i^{-,\dagger}
\,.
\end{gather*}
Thus, we can extend the domain of Problem~\ref{9993} to $\lambda\in\R^{N}$ and update the objective function in the following way
(without changing the optimal solution).
\begin{align*}
  G
  (\rho,\lambda_0,\lambda)
  &
  \ 
  :=
  \ 
 -
\sum_{i=1} 
  ^n
    \,
  \varphi^*
  \!
  \left( 
    \rho_i
    +
\lambda_0
+
\inner
{B(X_i)}
{
\lambda
}
  \right)
  \\
  &
  \qquad
+
\ 
\sum_{i=1}^{N} 
  \left( 
\lambda_0
+
\inner
{B(X_i)}
{
\lambda
}
  \right)
  \,
  \ 
-
\ 
\inner
{\delta}
{
  |\lambda|
}
  \,.
\end{align*}
Multiplying $G$ with $-1/N$ doesn't change the solution either
(if we search instead for a minimum).
To finish the proof, we choose the notation with $T_i$ instead of $n$. This extends the domain of $\rho$ to $\R^N_{\ge 0}$, but the 
new $\rho_i$ are not effective because of $T_i=0$ for all $i>n$. 
Thus we may set them to 0.
\end{proof}


%Bring optimal solutions together. with argmax meas. and Proposition for aSsumption and assumption.

%It is useful too define a weights function
%New chapter?
In the formulation of Theorem~\ref{dual} we encounter "If (...) there exists the optimal solution $(\rho^\dagger,\lambda_0^\dagger,\lambda)$ ... " .
To be able to study asymptotic properties of the solutions we have to become independent of this assumption.
For this we need some tools from functional analysis.
\subsection{Argmax Measurability Theorem}
We follow \cite{Aliprantis2007}.
A \textbf{correspondence} $\psi$ from a set $S_1$ to a set $S_2$ assigns to each $s_1\in S_1$ a subset $\psi(s_1)\subset S_2$.
To clarify that we map $s_1$ to a set, we use the double arrow, that is,
$
  \psi
  \colon
  S_1
  \twoheadrightarrow
  S_2
$.
  Let 
  $(\mathcal{Z},\Sigma_{\mathcal{Z}})$ be a measurable space and $\mathcal{S}$  a topological space.
  We say, that a correspondence 
  $
  \psi
  \colon
  \mathcal{Z}
  \twoheadrightarrow
  \mathcal{S}
  $
  is 
  \textbf{
  weakly measurable
  },
  if
  \begin{gather*}
    \left\{ 
      z\in \mathcal{Z}
      \ 
      |
      \ 
      \psi(z)
      \cap
      O
      \neq
      \emptyset
    \right\}
    \in
    \Sigma_{\mathcal{Z}}
    \qquad
    \text{for all open subsets}
    \ 
    O\subset \mathcal{S}
    \,.
  \end{gather*}
  A \textbf{selector} from a correspondence $\psi\colon \mathcal{Z}\twoheadrightarrow \mathcal{S}$ is a function $s\colon \mathcal{Z}\to \mathcal{S}$ that satisfies 
  \begin{align*}
s(z)\in\psi(z)
\qquad
\text{for all}\ 
z\in\mathcal{Z}
\,.
  \end{align*}
  

\begin{definition}
  Let 
  $(\mathcal{Z},\Sigma_{\mathcal{Z}})$ be a measurable space, and let $\mathcal{S}_1$ and $\mathcal{S}_2$  be topological space.
  A function 
  $f\colon \mathcal{Z}\times \mathcal{S}_1 \to \mathcal{S}_2$
  is a \textbf{Caratheodory function} if
  \begin{align*}
    f(\cdot,s_1)
    &
    \colon
    \mathcal{Z}\to \mathcal{S}_2
    \qquad
    \text{is}\ 
    (\Sigma_{\mathcal{Z}},\mathcal{B}(\mathcal{S}_2))-measurable
    \ 
    \text{for all}
    \ 
    s_1\in \mathcal{S}_1
    \,,
    \intertext{and}
    f(z,\cdot)
    &
    \colon
    \mathcal{Z}\to \mathcal{S}_2
    \qquad
    \text{is continuous for all}\ 
    z\in \mathcal{Z}
    \,.
  \end{align*}
\end{definition}
\begin{theorem}
  \label{th:argmax}
  Let $\mathcal{S}$ be a separable metrizable space and
  $
  (\mathcal{Z},\Sigma_{\mathcal{Z}})
  $
  a measurable space.
  Let $\psi\colon \mathcal{Z} \twoheadrightarrow \mathcal{S}$ be a weakly measurable correspondence with non-empty compact values, and suppose
  $f\colon \mathcal{Z}\times \mathcal{S} \to \R$
  is a Caratheodory function. Define the value function 
  $m\colon \mathcal{Z}\to \R$ by
  \begin{gather*}
    m(z):=\max_{s\in\psi(z)}f(z,s)
    \,,
  \end{gather*}
  and the correspondence 
  $\mu\colon \mathcal{Z}\twoheadrightarrow \mathcal{S}$ of maximizers by
  \begin{gather*}
    \mu(z):= \left\{ 
      s\in \psi(z)
      |
      f(z,s)=m(z)
    \right\}
    \,.
  \end{gather*}
  Then the value function $m$ is measurable, 
  the argmax correspondence $\mu$ has non-empty and compact values,
  is measurable and admits a measurable selector.
\end{theorem}
\begin{proof}
  \cite[Theorem~18.19]{Aliprantis2007}
\end{proof}


%\subsection{Propensity Score Function}

%\begin{definition}
  We define the \textbf{propensity score function} by
  \begin{gather*}
    \pi
    \ 
    \colon
    \R^d
    \ 
    \to
    \ 
    [0,1]
    \,,
    \qquad
    x
    \ 
    \mapsto
    \ 
    \P
    [
    T=1
    |
    X=x
    ]
    \,,
  \end{gather*}
  that is, 
  as the conditional probability of treatment given individual characteristics $x\in\R^d$.
\end{definition}
\begin{remark}
  We define $\pi$ on the whole $\R^d$, although $\mathcal{X}\subset \R^d$ may be a much smaller (possibly finite or countable) subset.
  The reason is that we want to assume continuity.
\end{remark}
\begin{assumption}
  \label{asu:ps}
  The propensity score function $\pi$ satisfies
  \begin{itemize}
    \item
      $\pi(x)\in (0,1)$ for all $x\in\R^d$
    \item
      $\pi$ is continuously differentiable on $\R^d$
  \end{itemize}
\end{assumption}

Next we give a standard example for a propensity score model

\begin{example}
  Logistic regression
\end{example}


%Next we formulate and discuss assumptions on the basis functions.
%\subsection{Basis Functions}
%In our setting basis functions of the covariates are stochastic processes indexed over the covariate space $\mathcal{X}$.
We consider a set $\mathfrak{B}$ of stochastic processes 
\begin{gather*}
  \left\{ 
      B_k(x)
      \ 
      \colon
      (
      \Omega,
      \mathcal{A}
      ,
      \P
      )
      \ 
      \to
      \ 
      (
      \R
      ,
      \mathcal{B}(\R)
      )
      \ 
      |
      \ 
      x\in\mathcal{X}
  \right\}
  \qquad
  \text{for}\ 
k\in \left\{ 1,\ldots,\# \mathfrak{B} \right\}
\end{gather*}
and
denote the vector of the basis functions (also a stochastic processes indexed over $\mathcal{X}$)
as
\begin{align*}
B(x)
  \ 
  :=
  \ 
  [B_1(x),\ldots,B_{\# \mathfrak{B}}(x)]^\top
  \ 
  \colon
  \ 
      (
      \Omega,
      \mathcal{A}
      ,
      \P
      )
      \to
      (
      \R^{\#\mathfrak{B}}
      ,
      \mathcal{B}
      (
      \R^{\#\mathfrak{B}}
      )
      )
    \qquad
  \text{for}\ 
  x\in\mathcal{X}
  \,.
\end{align*}
\begin{gather*}
\end{gather*}
\begin{assumption}
  \label{asu:basis}
  The set of basis functions 
  $
  \mathfrak{B}
  $
  satisfies
  \begin{enumerate}[label=(\roman*)]
    \item
      $\norm{B(x)}_2\lesssim 1$ for all $x\in \mathcal{X}$.
    \item
      There exist random vectors
      $
      \lambda^*_{\varphi^{'}\circ\, 1/\pi}
      $
      and
      $
      \lambda^*_{F_{Y(1)(z|\cdot)}}
      $,
      for all $z\in\R$,
      with values in $\R^{\#\mathfrak{B}}$
      such that
      \begin{align}
        \label{asu:basis:ii:1}
        \frac{1}{N}
        \sum_{i=1}^{N} 
        \left| 
        \inner{B(X_i)}
        {
      \lambda^*_{\varphi^{'}\circ\, 1/\pi}
      }
        -
        \varphi^{'}\left( \frac{1}{\pi(X_i)} \right)
        \right|
        &
        \ 
        \overset{\P}
        {
        \to 
        }
        \ 
        0
        \qquad
        \text{for}
        \ 
        N\to\infty
        \,,
        \intertext{and}
        \label{asu:basis:ii:2}
        \sqrt{N}
        \max_{i\in \left\{ 1,\ldots,N \right\}}
        \sup_{z\in\R}
        \left| 
        \inner{B(X_i)}
        {
      \lambda^*_{F_{Y(1)}(z|\cdot)}
      }
        -
        F_{Y(1)}(z|X_i)
        \right|
        &
        \ 
        \overset{\P}
        {
        \to 
        }
        \ 
        0
        \qquad
        \text{for}
        \ 
        N\to\infty
        \,.
      \end{align}
    \end{enumerate}
\end{assumption}

Next, we give some examples. 
We begin with a histogram basis \cite[§4]{Gyorfi2002}
\begin{example}
We consider a sequence of partitions
$
\left( 
  \mathcal{P}_N
  =
  \left\{ 
    A_{N,1}
    ,
    A_{N,2}
    ,
    \ldots
  \right\}
\right)
$
of $ \R^d $
and define
$ A_N(x) $ to be the cell of $ \mathcal{P}_N $ containing $x$.
We also assume
uniform partition width that decreases to 0, that is,
\begin{gather}
  \label{8881}
  \text{
for all $j\in\mathbb{N}$
it holds
  }
  \qquad
  \lambda^d(A_{N,j})
  \ 
  =
  \ 
  h_N^d
  \ 
  \to
  \ 
  0
  \qquad
  \text{for}
  \ 
  N\to\infty
  \,.
\end{gather}
We define $N$ basis functions $B_k$ of the covariates by
\begin{gather*}
  B_k(x)
  \ 
  :=
  \ 
  \frac{
  \mathbf{1}{\left\{ X_k \in A_N(x) \right\}}
  }{
  \sum_{j=1}^{N} 
  \mathbf{1}{\left\{ X_j \in A_N(x) \right\}}
  }
  \qquad
  \text{for}
  \,
  k\in
  \left\{ 
  1,\ldots,N
  \right\}
  \,,
\end{gather*}
where we keep to the convention $"0/0=0"$.
If at least one $B_k(x)>0$, the basis functions sum to 1. If $B_k(x)=0$ for all basis functions, the sum is 0.
Thus
\begin{align}
  \label{8882}
  \sum_{k=1}^{N}
  B_k(x)
  &
  \ 
  \in
  \ 
  \left\{ 0,1 \right\}
  \qquad
  \text{for all}
  \ 
  x\in\R^d
  \,.
  \notag
  \intertext{
Since $B_i(X_i)>0$, it holds
  }
  \sum_{k=1}^{N}
  B_k(X_i)
  &
  \ 
  =
 \  
  1
  \qquad
  \text{for all}\ 
  i\in
  \left\{ 1,\ldots,N \right\}
  \,.
\end{align}
We check \textit{(i)} in Assumption~\ref{asu:basis}. 
Since 
\begin{gather*}
B_k(x)
\ 
\in
\ 
[0,1]
\qquad
\text{for all}\ 
x\in\R^d
\ 
\text{and for all}\ 
k\in \left\{ 1,\ldots,N \right\}
\,,
\end{gather*}
it holds
\begin{gather*}
  \label{basis_l2_bdd}
  \norm{B(x)}_2^2
  \ 
  =
  \ 
  \sum_{k=1}^{N} 
  B_k(x)
  ^2
  \ 
  \le
  \ 
  \sum_{k=1}^{N} 
  B_k(x)
  \ 
  \in
  \ 
  \left\{ 0,1 \right\}
  \quad
  \text{
    for all
  }
x\in\R^d
\,
\,.
\end{gather*}
Next we check \eqref{asu:basis:ii:1} in \textit{(ii)} in Assumption~\ref{asu:basis}.
To this end we consider
\begin{gather*}
  \lambda^*_{\varphi^{'}\circ\,1/\pi}
  \ 
  :=
  \ 
  \left[ 
    \varphi^{'}
    \left( 
      \frac{1}{\pi(X_1)}
    \right)
    ,
    \ldots
    ,
    \varphi^{'}
    \left( 
      \frac{1}{\pi(X_N)}
    \right)
  \right]
  ^\top
  \,.
\end{gather*}
Since $f$ is continuous and the $X_k$ are random vectors,
$\lambda^*_f$ is also a random vector.
It follows 
\begin{align*}
  &
  \frac{1}{N}
  \sum_{i=1}^{N} 
        \left| 
        \inner{B(X_i)}
        {
  \lambda^*_{\varphi^{'}\circ\,1/\pi}
      }
        -
    \varphi^{'}
    \left( 
      \frac{1}{\pi(X_i)}
    \right)
        \right|
        \\
  &
  \ 
        =
  \ 
  \frac{1}{N}
  \sum_{i=1}^{N} 
        \left| 
        \sum_{k=1}^{N} 
        B_k(X_i)
        \cdot
    \varphi^{'}
    \left( 
      \frac{1}{\pi(X_k)}
    \right)
        -
    \varphi^{'}
    \left( 
      \frac{1}{\pi(X_i)}
    \right)
        \right|
        \\
  &
  \ 
        =
  \ 
  \frac{1}{N}
  \sum_{i=1}^{N} 
        \left| 
        \sum_{k=1}^{N} 
        B_k(X_i)
        \cdot
        \left( 
    \varphi^{'}
    \left( 
      \frac{1}{\pi(X_k)}
    \right)
        -
    \varphi^{'}
    \left( 
      \frac{1}{\pi(X_i)}
    \right)
        \right)
        \right|
\\
  &
  \ 
        =
  \ 
  \frac{1}{N}
  \sum_{i=1}^{N} 
        \left| 
        \sum_{k=1}^{N} 
        B_k(X_i)
        \cdot
        \mathbf{1}
        \left\{ X_k\in A_N(X_i) \right\}
        \cdot
        \left( 
    \varphi^{'}
    \left( 
      \frac{1}{\pi(X_k)}
    \right)
        -
    \varphi^{'}
    \left( 
      \frac{1}{\pi(X_i)}
    \right)
    \right)
        \right|
        \\
  &
  \ 
        \le
  \ 
  \frac{1}{N}
  \sum_{i=1}^{N} 
        \sum_{k=1}^{N} 
        B_k(X_i)
        \cdot
        \omega
        ( 
        \varphi^{'} 
        \circ
        (x\mapsto 1/x)
        \circ
        \pi
        ,
        \lambda^d(A_N(X_i))
        )
        \\
  &
  \ 
        =
  \ 
        \omega
        ( 
        \varphi^{'} 
        \circ
        (x\mapsto 1/x)
        \circ
        \pi
        ,
        h_N^d
        )
        \ 
        \to
        \ 
        0
        \qquad
        \text{for}
        \ 
        N\to\infty
        \,,
\end{align*}
where $\omega$ is the modulus of continuity.
The second equality is due to \eqref{8881}, the third equality follows from the definition of the basis functions, and 
the inequality follows from \eqref{8881}, the convexity of the absolute value and the continuity of $f$.
The convergence in due to \eqref{8882}.

Next, we check \eqref{asu:basis:ii:2} in \textit{(ii)} in Assumption~\ref{asu:basis}.
To this end, we consider
\begin{gather*}
  \lambda^*_{F_{Y(1)}(z|\cdot)}
  \ 
  :=
  \ 
  \left[ 
    F_{Y(1)}(z|X_1)
    ,
    \ldots
    ,
    F_{Y(1)}(z|X_N)
  \right]
  ^\top
  \qquad
  \text{for}\ 
  z\in\R
  \,.
\end{gather*}
To obtain the desired convergence, we need to assume some regularity for 
$
    F_{Y(1)}(z|\cdot)
$.
Let it be continuity for the moment.
With similar arguments as before we get
    \begin{align*}
      &
         \sqrt{N}
        \max_{i\in \left\{ 1,\ldots,N \right\}}
        \sup_{z\in\R}
        \left| 
        \inner{B(X_i)}
        {
      \lambda^*_{F_{Y(1)}(z|\cdot)}
      }
        -
        F_{Y(1)}(z|X_i)
        \right|
        \\
        &
        \ 
        =
        \ 
         \sqrt{N}
        \sup_{z\in\R}
        \omega
        \left( 
        F_{Y(1)}(z|\cdot)
        ,
        h_N^d
        \right)
    \end{align*}
    Thus, if the interplay of $F_{Y(1)}$ and $h_N$ is sufficiently good we get convergence.
    The most abstract assumption would be
    \begin{gather}
      \label{6674}
         \sqrt{N}
        \sup_{z\in\R}
        \omega
        \left( 
        F_{Y(1)}(z|\cdot)
        ,
        h_N^d
        \right)
        \to
        0
        \,.
    \end{gather}
    To be more concrete, if, for example,
$
    F_{Y(1)}(z|\cdot)
    $
    is $\alpha$-Hölder continuous for all $z\in\R$ with 
    $\alpha\in(0,1)$ and $\sqrt{N}h_N^{d\cdot \alpha}\to 0$ it holds
    \eqref{6674}.

\end{example}

%We continue with kernel bases \cite[§5]{Gyorfi2002}.
%\begin{example}
%Let
%$
%  K
%  \colon
%  \R^d
%  \to
%  [0,\infty)
%$
%be a measurable function with $K(0)>0$. 
%We call $K$ a kernel function.
%We assume that there exists
%$R>0$ such that 
%\begin{gather}
%  \label{8884}
%  K(x)
%  \ 
%  \le
%  \ 
%  \mathbf{1}\left\{ \norm{x}_2\le R\right\}
%  \,,
%\end{gather}
%that is, the kernel function has compact support.
%For a decreasing sequence (the bandwith of the kernel) $(h_N)\in (0,\infty)$ with $h_N\to 0$ for $N\to\infty$,
%we define $N$ basis functions $B_k$ of the covariates by
%\begin{gather*}
%  B_k(x)
%  \ 
%  :=
%  \ 
%  \frac{
%    K \left( \frac{\norm{x-X_k}_2}{h_N} \right)
%  }{
%  \sum_{j=1}^{N} 
%    K \left( \frac{\norm{x-X_j}_2}{h_N} \right)
%  }
%  \qquad
%  \text{for}
%  \,
%  k\in
%  \left\{ 
%  1,\ldots,N
%  \right\}
%  \,,
%\end{gather*}
%where (again) we keep to the convention $"0/0=0"$.
%Since $K(0)>0$, as in the previous example, it follows
%\begin{align}
%  \label{8883}
%  \sum_{k=1}^{N}
%  B_k(x)
%  &
%  \ 
%  \in
%  \ 
%  \left\{ 0,1 \right\}
%  \qquad
%  \text{for all}
%  \ 
%  x\in\R^d
%  \,,
%  \notag
%  \\
%  \sum_{k=1}^{N}
%  B_k(X_i)
%  &
%  \ 
%  =
% \  
%  1
%  \qquad
%  \text{for all}\ 
%  i\in
%  \left\{ 1,\ldots,N \right\}
%  \intertext{and}
%  \notag
%  \norm{B(x)}_2^2
%  &
%  \ 
%  \in
%  \ 
%  \left\{ 0,1 \right\}
%  \qquad
%  \text{for all}
%  \ 
%  x\in\R^d
%  \,.
%\end{align}
%This verifies the first condition of Assumption~\ref{asu:basis}.
%To verify the second condition, note, that
%\begin{align}
%  \label{8885}
%  \notag
%  K(x)
%  &
%  \ 
%  =
%  \ 
%  \mathbf{1}\left\{ \norm{x}_2\le R \right\} 
%  K(x)
%  \qquad
%  \text{for all}
%  \ 
%  x\in\R^d
%  \intertext{and thus}
%  B_k(x)
%  &
%  \ 
%  =
%  \ 
%  B_k(x)
%  \cdot
%  \mathbf{1}\left\{ \norm{x-X_k}_2\le R\cdot h_N \right\} 
%  \qquad
%  \text{for all}
%  \ 
%  x\in\R^d
%  \ 
%  \text{and for all}
%  \ 
%  k\in \left\{ 1,\ldots,N \right\}
%  \,.
%\end{align}
%Let 
%$f$ be a continuous function
%and consider (as in the previous example)
%\begin{gather*}
%  \lambda^*_f
%  \ 
%  :=
%  \ 
%  [f(X_1),\ldots,f(X_N)]^\top
%  \,.
%\end{gather*}
%It follows
%\begin{align*}
%  &
%        \left| 
%        \inner{B(X_i)}{\lambda^*_f}
%        -
%        f(X_i)
%        \right|
%        \\
%  &
%  \ 
%        =
%  \ 
%        \left| 
%        \sum_{k=1}^{N} 
%        B_k(X_i)
%        \cdot
%        f(X_k)
%        -
%        f(X_i)
%        \right|
%        \ 
%        =
%        \ 
%        \left| 
%        \sum_{k=1}^{N} 
%        B_k(X_i)
%        \cdot
%        \left( 
%        f(X_k)
%        -
%        f(X_i)
%        \right)
%        \right|
%\\
%  &
%  \ 
%        =
%  \ 
%        \left| 
%        \sum_{k=1}^{N} 
%        B_k(X_i)
%        \cdot
%  \mathbf{1}\left\{ \norm{x-X_k}_2\le R\cdot h_N \right\} 
%        \cdot
%        \left( 
%        f(X_k)
%        -
%        f(X_i)
%        \right)
%        \right|
%        \\
%  &
%  \ 
%        \le
%  \ 
%        \sum_{k=1}^{N} 
%        B_k(X_i)
%        \cdot
%        \omega
%        ( 
%        f,
%        R\cdot h_N
%        )
%        \\
%  &
%  \ 
%        =
%  \ 
%        \omega
%        ( 
%        f,
%        R\cdot h_N
%        )
%        \to
%        0
%        \qquad
%        \text{for}
%        \ 
%        N\to\infty
%        \,,
%\end{align*}
%The convergence in due to $h_N\to 0$ for $N\to \infty$. 
%\end{example}
We now have a large class of possible basis function. We may proceed.

\subsection{Pseudo Solutions}
There are two observations that help achieving our goal.
First, if we solve Problem~\ref{dual} on a compact search space, there always exists a measurable solution (if the objective function is Caratheodory).
Second (with hindsight), we want optimal solutions to estimate an oracle parameter $\lambda^*$ well. 
To this end, we will define a (random) compact parameter space that always contains the oracle parameter $\lambda^*$ and restrict Problem~\ref{dual} to it.
Then Theorem~\ref{th:argmax} gives us (under weak measurability conditions) a measurable solution of the restricted Problem~\ref{dual}.
In some cases, this is only a local solution. It becomes global if it is in the interior of the compact parameter space.
Thus we call it the restricted, or the pseudo solution.

Consider 
the random vector (the oracle parameter)
\begin{align*}
  \lambda^*
  \ 
  :=
  \ 
  \left(
  0_{N},0,
  \left[
  \varphi^{'}
  \left(
  \frac
  {1}
  {\pi(X_i)}
  \right)
\right]_{i\in \left\{
  1,\ldots,N
\right\}}
  \right)
\end{align*}
with values in 
$\R^N_{\ge 0}\times \R\times \R^N$.
Clearly,  
\begin{align*}
 \lambda^*
 \qquad
 \text{is}
 \qquad
 \left(
( 
\sigma
\left( 
  (T_i,X_i)_{i\in 
\left\{ 1,\ldots,N \right\}
  } 
\right)
,
\mathcal{B}
\left( 
\R^N_{\ge 0}\times \R\times \R^N
\right)
\right)
-\text{measurable}
\,.
\end{align*}
Next we define for $N\in\mathbb{N}$ the correspondence  
\begin{align*}
  \Theta_N\colon
  &
  \left( 
  \Omega,
\sigma
\left( 
  (T_i,X_i)_{i\in 
\left\{ 1,\ldots,N \right\}
  } 
\right)
  \right)
  \ 
\to
  \ 
\R^N_{\ge 0}\times \R\times \R^N
\\
&
\omega
  \ 
\mapsto
  \ 
\left\{ 
  \left( 
  \rho,\lambda_0,\lambda
  \right)
  \ 
  \in
  \ 
\R^N_{\ge 0}\times \R\times \R^N
  \ 
\colon
  \ 
\norm{
  \left( 
  \rho,\lambda_0,\lambda
  \right)
-\lambda^*}_2\le 1
\right\}
\,.
\end{align*}

\begin{lemma}
  \label{Theta_maes}
  For all $N\in\mathbb{N}$ it holds that 
  $\Theta_N$ is a weakly-measurable correspondence  with non-empty compact values.
\end{lemma}
\begin{proof}
  Let $N\in\mathbb{N}$. 
  Since 
  \begin{align*}
  \Theta_N(\omega)
  \ 
  \text{
  is the closed unit ball
  in
  }
  \ 
\R^N_{\ge 0}\times \R\times \R^N
\quad 
\text{around}
\quad 
  \lambda^*(\omega) 
  \,,
  \end{align*}
  it is non-empty and compact for all $\omega\in\Omega$.
  To prove that $\Theta_N$ is weakly-measurable,
  let 
  $
  O\subset
\R^N_{\ge 0}\times \R\times \R^N
  $
  be an open set and note, that
  \begin{align*}
    \left\{ 
      \Theta_N\cap O \neq \emptyset
    \right\}
    &
    \ 
    =
    \ 
    \left\{ 
      \exists
  \left( 
  \rho,\lambda_0,\lambda
  \right)
  \ 
  \in
  \ 
  O
\ 
\colon
\ 
\norm{
  \left( 
  \rho,\lambda_0,\lambda
  \right)
-\lambda^*}_2\le 1
    \right\}
    \\
    &
    \ 
    =
    \ 
    \left\{ 
      \exists
  \left( 
  \rho,\lambda_0,\lambda
  \right)
  \ 
  \in
  \ 
  O
  \cap
  \left( 
  \mathbb{Q}^N_{\ge 0}\times \mathbb{Q}\times \mathbb{Q}^N
  \right)
\ 
\colon
\ 
\norm{
  \left( 
  \rho,\lambda_0,\lambda
  \right)
-\lambda^*}_2\le 1
    \right\}
    \\
&
    \ 
    =
    \ 
    \bigcup_{
  \left( 
  \rho,\lambda_0,\lambda
  \right)
  \ 
  \in
  \ 
  O
  \cap
  \left( 
  \mathbb{Q}^N_{\ge 0}\times \mathbb{Q}\times \mathbb{Q}^N
  \right)
    }
    \left\{ 
\norm{
  \left( 
  \rho,\lambda_0,\lambda
  \right)
-\lambda^*}_2\le 1
    \right\}
    \,.
  \end{align*}
  Since
  \begin{align*}
    \left\{ 
\norm{
  \left( 
  \rho,\lambda_0,\lambda
  \right)
-\lambda^*}_2\le 1
    \right\}
    \in
\sigma
\left( 
  (T_i,X_i)_{i\in 
\left\{ 1,\ldots,N \right\}
  } 
\right)
\quad
\text{for all}\quad
  \left( 
  \rho,\lambda_0,\lambda
  \right)
  \in
\R^N_{\ge 0}\times \R\times \R^N
\,,
  \end{align*}
  and the union is countable, it follows
  $
    \left\{ 
      \Theta_N\cap O \neq \emptyset
    \right\}
    \in
\sigma
\left( 
  (T_i,X_i)_{i\in 
\left\{ 1,\ldots,N \right\}
  } 
\right)
  $.
  Thus $\Theta_N$ is weakly-measurable.
\end{proof}
We are ready to construct the measurable (pseudo) solution.
\begin{lemma}
  \label{lem:pseud_sol}
  For all $N\in\mathbb{N}$ there exists a (pseudo) solution
  \begin{align*}
    s_N
    \ 
    \colon
   \  
    \Omega
    \ 
    \to
    \ 
  \R^N_{\ge 0}
  \times
  \R
  \times
  \R^{N}
  \end{align*}
  to
  Problem~\ref{dual} restricted to $\Theta_N$ 
  that is
  \begin{align*}
  \left(
    \sigma \left( \left( T_i,X_i \right)_{i\in \left\{ 1,\ldots,N \right\}} \right),\mathcal{B}
  \left(
  \R^N_{\ge 0}
  \times
  \R
  \times
  \R^{N}
  \right)
  \right)
  -\text{measurable}
  \,.
  \end{align*}
  Furthermore,
if $s_N\in \mathrm{int}\, \Theta_N$, then it is the global solution
  $
  (\rho^\dagger,\lambda_0^\dagger,\lambda^\dagger)
  $.
\end{lemma}
\begin{proof}
  By Lemma~\ref{Theta_maes}
  the correspondence $\Theta_N$ satisfies the conditions of
  Theorem~\ref{th:argmax}.
Next, we consider the (random) objective function of (the maximize version of) Problem~\ref{dual}, that is,
  \begin{align*}
    &
  G\colon
  \left(
  \Omega,
\sigma
\left( 
  (T_i,X_i)_{i\in 
\left\{ 1,\ldots,N \right\}
  } 
  \right)
  \right)
  \times
  \left(
  \R^N_{\ge 0}
  \times
  \R
  \times
  \R^{N}
  \right)
  \to
  \R\cup \left\{
    -\infty
  \right\}
  \intertext{with}
    &
  G(\omega,(\rho,\lambda_0,\lambda))
  \ 
  =
  \ 
  -\infty
  \qquad 
  \text{if}\ 
  \rho_i\neq 0\  \text{for some}\ i>n
  \,,
  \\
  \intertext{and}\qquad
  &
  G(\omega,(\rho,\lambda_0,\lambda))
  \\
  &
  \ 
  =
  \ 
  \frac{1}{N}
\sum_{i=1} 
  ^N
  \Big[
    -
  T_i(\omega)
  \cdot
  \varphi^*
  \!
  \left( 
    \rho_i
    +
\lambda_0
+
\inner
{B(X_i)(\omega)}
{
\lambda
}
  \right)
  \ 
  +
  \ 
\lambda_0
+
\inner
{B(X_i)(\omega)}
{
\lambda
}
\Big]
  \ 
-
\ 
\inner
{\delta(\omega)}
{
  |\lambda|
}
\,.
  \end{align*}
  Clearly, the (random) objective function $G$  is a Caratheodory function
  . This follows from the (assumed) continuity of $\varphi^*$ and the measurability 
  of all random variables included.
  Since $G$ is also strictly concave, it has a unique argmax $s_N$  in $\Theta_N$. 
  By Theorem~\ref{th:argmax} this is $
  \left(
    \sigma
    \left( \left( T_i,X_i \right)_{i\in \left\{ 1,\ldots,N \right\}}\right)
    ,\mathcal{B}
  \left(
  \R^N_{\ge 0}
  \times
  \R
  \times
  \R^{N}
  \right)
  \right)
$-measurable.
Furthermore, by the strict concavity of $G$,
if $s_N\in \mathrm{int}\,\Theta_N$, then it's the global optimal solution. 
\end{proof}
We will prove later, that with probability going to 1 the latter will be the case.
Furthermore, we will prove, that the (measurable) solution converges to the random variable in probability.


Before we define the weights function we specify the basis $\mathfrak{B}$.
\subsection{basis}
Let $
\left(
\mathcal{P}_N
\right)
$
denote a sequence of countable, $\mathcal{B}$-measurable partitions 
\begin{align*}
\mathcal{P}_N= \left\{
  A_{N,1},
  A_{N,2},
  \ldots
\right\}
\subset \mathcal{B}(\R^d)
\end{align*}
of $\R^d$, that is, 
\begin{align*}
  A_{N,i}\cap A_{N,j}=\emptyset
  \qquad
  \text{if}\ i\neq j
  \qquad
  \text{and}
  \qquad 
  \bigcap_{i\in\mathbb{N}}A_{N,i}
  \ 
  =
  \ 
  \R^d
  \,.
\end{align*}
We define
$ A_N(x) $ to be the cell of $ \mathcal{P}_N $ containing $x$, that is,
\begin{align*}
  A_N
  \colon
  \R^d 
  \ 
  \twoheadrightarrow 
  \ 
  \R^d  
  \,,\qquad
  x
  \ 
  \mapsto
  \ 
  A_N(x)
  \,,
\end{align*}
where $A_N(x)$ is the only cell containing $x$. 

Next, we define the (empirical) basis vector
\begin{align*}
  B\colon
  \R^d\times \R^{d\cdot N}
  \to
  \R
  \,,
  \qquad
  (x,(x_1,\ldots,x_N))\mapsto
  \frac
  {
    \left[
    \mathbf{1}
    _{
      A_N(x)
    }
    (x_k)
    \right]
    _{k\in \left\{
        1,\ldots,N
    \right\}}
  }
  {
    \sum_{j=1}^N
    \mathbf{1}
    _{
      A_N(x)
    }
    (x_j)
    }
  \,,
\end{align*}
where we keep to the convention $"0/0=0"$.
\begin{lemma}
  \quad
  \begin{enumerate}[label=(\roman*)]
\item
  $B(\cdot,(X_1,\ldots,X_N))(\omega)$ is 
  $\left(
    \mathcal{B}(\R^d),\mathcal{B}(\R^N)
  \right)$-measurable
  and
  constant on each cell $A_N\in\mathcal{P}_N$
  for all $\omega\in\Omega$. 
\item
  $B(X,(X_1,\ldots,X_N))$ is $\left(
    \mathcal{A},\mathcal{B}(\R^N)
  \right)$-measurable. 
  \end{enumerate}
\end{lemma}

\begin{proof}
Consider
for $k\in \left\{
  1,\ldots,N
\right\}$
and $\omega\in\Omega$
the indicator function
\begin{align*}
  \mathbf{1}
  _
  {A_N(X_k(\omega))}
  \colon \R^d\to \left\{
    0,1
  \right\}
  \,.
\end{align*}
Since 
$
  {A_N(X_k(\omega))}
  \in\mathcal{B}(\R^d)
$
this is a 
  $\left(
    \mathcal{B}(\R^d),\mathcal{B}(\R^N)
  \right)$-measurable
  function.
  Since it is 1 if $
  x\in
  {A_N(X_k(\omega))}
  $
  and 0 else, it is also constant on each cell. It follows statement (i).
Since
\begin{align*}
  \mathbf{1}
  _
  {A_N(X_k(\omega))}(X(\omega))
  \ 
  =
  \ 
  \mathbf{1}
  \bigcup_{i\in\mathbb{N}}
  \left\{
    X,X_k \in A_{N,i}
  \right\}
  (\omega)
  \qquad
  \text{for all}\ 
  \omega\in\Omega
  \,,
\end{align*}
and
$
  \bigcup_{i\in\mathbb{N}}
  \left\{
    X,X_k \in A_{N,i}
  \right\}
  \in\mathcal{A}
$
it follows statement (ii).
\end{proof}

\subsection{Weights Function}
Theorem~\ref{dual_solution_th} tells us that if an optimal solution
$
(\rho^\dagger,\lambda_0^\dagger,\lambda^\dagger)
$
to Problem~\ref{dual} exists,
then the unique optimal solutions to Problem~\ref{bw:1:primal} are 
\begin{gather*}
  w^\dagger_i
  \ 
  :=
  \ 
  (
  \varphi^{'}
  )^{-1}
  \left(
    \rho^\dagger_i
  \ 
    +
  \ 
\lambda_0^\dagger
  \ 
+
  \ 
\inner
{B(X_i)}
{
\lambda^{\dagger}
}
  \right)
  \qquad
  \text{for all}\ 
  i\in
  \left\{ 1,\ldots,n \right\}
  \,.
\end{gather*}

This point of view is sufficient from a practical point of view
(when we are interested in actually computing the optimal weights).

For the subsequent analysis we need to view the weights as random quantities. To this end, we shall introduce the weights process. 

\begin{definition}
  \label{def:weights_function}
  We define the weights process to be a stochastic process indexed 
  over
  $
  \mathcal{X}
  \,
  \times
  \,
  \R^N_{\ge 0}
  \,
  \times
  \,
  \R
  \times
  \,
  \R^{N}
  $
  with values in $\R^N$ such that
\begin{align*}
  w
  \left( 
  x
  ,
  \rho
  ,
  \lambda_0
  ,
  \lambda
  \right)
  \ 
  :=
  \ 
  \left[ 
    (
    \varphi^{'}
    )^{-1}
    \left( 
     \rho_i 
     +
      \lambda_0
      +
      \inner{B(x)}{\lambda}
    \right)
  \right]_{i\in \left\{ 1, \ldots,N \right\}}
    \,.
\end{align*}
\end{definition}
By the measurability of the basis functions (or the basis processes),
the weights processes is also measurable 
(the quantities 
$
  w
  \left( 
  x
  ,
  \rho
  ,
  \lambda_0
  ,
  \lambda
  \right)
$
are random variables).
The next question is if there are (plausible) assumptions such that the weights processes at a random parameter
$
(\rho^\dagger,\lambda_0^\dagger,\lambda^\dagger)
$
is measurable.
To answer this in a good way we need the argmax measurability theorem \cite[Theorem~18.19]{Aliprantis2007}.
\subsection{Argmax Measurability Theorem}
We follow \cite{Aliprantis2007}
A \textbf{correspondence}$ \psi$ from a set $X$ to a set $Y$ assigns to each $x\in X$ a subset $\psi(x)\subset Y$.
To clarify that we map $x$ to a set, we use the double arrow, that is,
$
  \psi
  \colon
  X
  \twoheadrightarrow
  Y
$.
  Let 
  $(S,\Sigma_S)$ be a measurable space and $X$ a topological space.
  We say, that a correspondence 
  $
  \psi
  \colon
  S
  \twoheadrightarrow
  X
  $
  is 
  \textbf{
  weakly measurable
  },
  if
  \begin{gather*}
    \left\{ 
      s\in S
      \ 
      |
      \ 
      \psi(s)
      \cap
      O
      \neq
      \emptyset
    \right\}
    \in
    \Sigma_S
    \qquad
    \text{for all open subsets}
    \ 
    O\subset X
    \,.
  \end{gather*}

A selector from a correspondence $\psi\colon X\twoheadrightarrow Y$ is a function $s\colon X\to Y$ that satisfies 
$
s(x)\in\psi(x)
$
for all $x\in X$.

\begin{definition}
  Let 
  $(S,\Sigma_S)$ be a measurable space, and let $X$ and $Y$  be topological space.
  A function 
  $f\colon S\times X \to Y$
  is a \textbf{Caratheodory function} if
  \begin{align*}
    f(\cdot,x)
    &
    \colon
    S\to Y
    \qquad
    \text{is}\ 
    (\Sigma_{S},\mathcal{B}(Y))-measurable
    \ 
    \text{for all}
    \ 
    x\in X
    \,,
    \intertext{and}
    f(s,\cdot)
    &
    \colon
    X\to Y
    \qquad
    \text{is continuous for all}\ 
    s\in S
    \,.
  \end{align*}
\end{definition}
\begin{theorem}
  Let $X$ be a separable metrizable space and
  $
  (S,\Sigma_S)
  $
  a measurable space.
  Let $\psi\colon S \twoheadrightarrow X$ be a weakly measurable correspondence with non-empty compact values, and suppose
  $f\colon S\times X \to \R$
  is a Caratheodory function. Define the value function 
  $m\colon S\to \R$ by
  \begin{gather*}
    m(s):=\max_{\psi(s)}f(s,x)
    \,,
  \end{gather*}
  and the correspondence 
  $mu\colon S\twoheadrightarrow X$ of maximizers by
  \begin{gather*}
    \mu(s):= \left\{ 
      x\in \psi(s)
      |
      f(s,x)=m(s)
    \right\}
    \,.
  \end{gather*}
  Then the value function $m$ is measurable, 
  the argmax correspondence $\mu$ has non-empty and compact values,
  is measurable and admits a measurable selector.
\end{theorem}
\begin{proof}
  \cite[Theorem~18.19]{Aliprantis2007}
\end{proof}

Our goal is to find (plausible) assumptions under which we can measurably select optimal solutions of Problem~\ref{dual}.

\begin{assumption}
  There exists $\underline{N}\in\mathbb{N}$ such that 
  for all $N\ge \underline{N}$ 
  there exists a compact and deterministic parameter space
  $
  \Theta_N
  \subset
  \R^N_{\ge 0}
  \times
  \R
  \times
  \R^{N}
  $
  such that for all data sets $D_N$
  a sequence of optimal solution
  $(\rho^\dagger,\lambda_0^\dagger,\lambda^\dagger)$
  satisfying the Lemma
exist and it holds
\begin{gather*}
  (
  \rho^\dagger,
  0_{N-n},
  \lambda_0^\dagger,\lambda^\dagger)
  \in
  \Theta_N
  \,.
\end{gather*}
\end{assumption}

With this assumption we can construct the (constant) correspondence
\begin{align*}
  \psi
  \colon
  (\Omega,\mathcal{A},\P)
  \ 
  \twoheadrightarrow
  \ 
  \R^N_{\ge 0}
  \times
  \R
  \times
  \R^{N}
  \,,
  \qquad
  \omega
  \ 
  \to
  \ 
  \Theta_N
  \,.
\end{align*}
This is weakly measurable, because $\Theta_N$ is deterministic and thus
$\Theta_N\cap O$ is either true or false, that is,
  \begin{gather*}
    \left\{ 
      \omega\in \Omega
      \ 
      |
      \ 
      \psi(\omega)
      \cap
      O
      \neq
      \emptyset
    \right\}
    \ 
    \in
    \ 
    \left\{ \Omega,\emptyset \right\}
    \subset
    \Sigma_S
    \qquad
    \text{for all open subsets}
    \ 
    O
    \subset
  \R^N_{\ge 0}
  \times
  \R
  \times
  \R^{N}
    \,.
  \end{gather*}
  Furthermore the objective function is a Caratheodory function.
  Thus, by Theorem there exists the argmax correspondence  of Problem~\ref{dual} and a measurable selector that selects
  $(\rho^\dagger,0_{N-n},\lambda_0^\dagger,\lambda^\dagger)$ 
  of Assumption.
  With this, we can define the optimal weights processes
  \begin{gather*}
    w^\dagger(x)
    \ 
    :=
    \ 
    w
    \left( 
    x,\rho^\dagger,0_{N-n},\lambda_0^\dagger,\lambda^\dagger
    \right)
    \qquad
    \text{indexed over}\ 
    x\in\mathcal{X}\,.
  \end{gather*}
  Due to the definition of the (general) weights processes and the measurability of the argmax selector, the random variables $w^\dagger(x)$ are measurable for all $x\in\mathcal{X}$.
  To end the measurability discussion, note that $w^\dagger(X)$ is a random variable.


\begin{lemma}
  \label{weights_l_inf}
  Let Assumption~\ref{asu:existence_sol}, 
  Assumption~\ref{asu:basis},
  and Assumption~\ref{asu:objective_f} hold true.
  Then it holds 
  \begin{gather*}
    w_i^\dagger(X)
  \ 
  \in
  \ 
  L^\infty(\P)
  \qquad
  \text{for all}
  \ 
  i\in \left\{ 1,\ldots,N \right\}
  \ 
  \text{
    and
  for all 
  }\ 
  N\ge \underline{N}
  \,.
  \end{gather*}
\end{lemma}
\begin{proof}
  Let $N\ge \underline{N}$.
  By Assumption~\ref{asu:basis} it holds 
  $
    \text{for all}\ 
  $ 
  \begin{gather}
    \label{3967}
    \left| 
     \rho_i 
     +
      \lambda_0
      +
      \inner{B(x)}{\lambda}
    \right|
    \ 
    \lesssim
    \ 
    \norm{(\rho,\lambda_0,\lambda)}_2
    \,.
  \end{gather}
  $
    \text{for all}
    \ 
    x\in\mathcal{X}
    \
    \text{and for all}\ 
    (\rho,\lambda,\lambda_0)
    \in
  \R^N_{\ge 0}
  \times
  \R
  \times
  \R^{N}
  $.
  By Assumption~\ref{asu:existence_sol}, 
  there exists a (compact) parameter space
  $\Theta_N$ around the origin, with $\mathrm{diam}\,  \Theta_N<\infty$, 
  such that for all data sets $D_N$ it holds  $(\rho^\dagger,\lambda^\dagger,\lambda_0^\dagger)\in\Theta_N$.
  By Assumption~\ref{asu:objective_f}, $(\varphi^{'})^{-1}$ is non-decreasing and continuous. Thus
    by \eqref{3967} it holds
  \begin{align*}
    \left| 
    w_i^\dagger(x)
    \right|
    \ 
    \lesssim
    \ 
    \left| 
    (\varphi^{'})^{-1}
    \left( 
      \,
      -
    \norm{(\rho^\dagger,\lambda_0^\dagger,\lambda^\dagger)}_2
    \right)
    \right|
    \ 
    +
    \ 
    \left| 
    (\varphi^{'})^{-1}
    \left( 
      \,
    \norm{(\rho^\dagger,\lambda_0^\dagger,\lambda^\dagger)}_2
    \right)
    \right|
    \ 
    \ 
    <
    \ 
    \infty
  \end{align*}
  for all $x\in\mathcal{X}$
  and all $i\in \left\{ 1,\ldots,n \right\}$
  .
\end{proof}
\begin{lemma}
  \label{w.Z=0}
  Let Assumption~\ref{aa:assumption:2} and Assumption~\ref{aa:assumption:3} hold true.
  Furthermore, 
  let
  $N\ge\underline{N}$, and
  let
  $Z\in L^1(\P)$
  be a random variable that is independent of $D_N$ 
  with
  $
\E
\left[
  Z
  \,
  |
  \, 
  X
\right]
= 0
  $
  almost surely.
  It holds
  \begin{gather*}
  \E
  \left[
  w_i(X,\rho^\dagger,\lambda^\dagger,\lambda_0^\dagger)
  \cdot Z
  \right]
  =0
  \qquad
  \text{for all}
  \ 
  i\in \left\{ 1,\ldots,n \right\}
  \,.
  \end{gather*}
\end{lemma}
\begin{proof}
  Let
  $N\ge\underline{N}$.
  We write
  \begin{gather*}
  w_i(X,\rho^\dagger,\lambda^\dagger,\lambda_0^\dagger)
  \ 
  =
  \ 
  w^\dagger(X)
  \end{gather*}
  and ignore the index $i$.
  By Lemma~\ref{weights_l_inf} and 
  $Z\in L^1(\P)$
  it holds
  \begin{gather}
    \label{9876}
    \norm{
  w^\dagger(X)\cdot Z
    }_{L^1(\P)}
    \ 
  \le
    \ 
  \norm{w^\dagger(X)}_{L^\infty(\P)}
  \norm{Z}_{L^1(\P)}
  \ 
  <
  \ 
  \infty
  \,.
  \end{gather}
  By 
  \eqref{9876},
  $Z\perp D_N$
  and
  $
\E
\left[
  Z
  \,
  |
  \, 
  X
\right]
= 0
  $
  almost surely
  it holds 
  \begin{align*}
    \E
  \left[
  w^\dagger(X)
  \cdot
  Z
  \,
  |
  \,
  D_N,X
  \right]
  &
  \ 
  =
  \ 
  w^\dagger(X)
  \cdot
  \E
  \left[
  Z
  \,
  |
  \,
  D_N,X
  \right]
  \\
  &
  \ 
  =
  \ 
  w^\dagger(X)
  \cdot
  \E
  \left[
  Z
  \,
  |
  \,
  X
  \right]
  \
  =
  \ 
  0
  \qquad
  \text{almost surely.}
  \end{align*}
  Thus
  \begin{gather*}
    \E
    \left[
  w^\dagger(X)
  \cdot
  Z
  \,
    \right]
    \ 
    =
    \ 
    \E
    \left[
 \E
  \left[
  w^\dagger(X)
  \cdot
  Z
  \,
  |
  \,
  D_N,X
  \right]
    \right]
    \ 
    =
    \ 
    0
    \,.
     \end{gather*}
\end{proof}




\chapter{Constructing the Weights Process}
  In the formulation of Theorem~\ref{dual} we encounter "If (...) there exists the optimal solution $(\rho^\dagger,\lambda_0^\dagger,\lambda)$ ... " .
To be able to study asymptotic properties of the weights, we 
shall assume that Problem~\ref{dual} is feasible,
construct a measurable dual solution, and plug it in $(\varphi^{'})^{-1}$.
Before we formulate concrete assumptions, we provide tools from functional analysis
to obtain measurability. Afterwards, we shall tailor the assumptions to the capability of this tools.
\section{Argmax Measurability Theorem}
  We follow \cite{Aliprantis2007}.
A \textbf{correspondence} $\psi$ from a set $S_1$ to a set $S_2$ assigns to each $s_1\in S_1$ a subset $\psi(s_1)\subset S_2$.
To clarify that we map $s_1$ to a set, we use the double arrow, that is,
$
  \psi
  \colon
  S_1
  \twoheadrightarrow
  S_2
$.
  Let 
  $(\mathcal{Z},\Sigma_{\mathcal{Z}})$ be a measurable space and $\mathcal{S}$  a topological space.
  We say, that a correspondence 
  $
  \psi
  \colon
  \mathcal{Z}
  \twoheadrightarrow
  \mathcal{S}
  $
  is 
  \textbf{
  weakly measurable
  },
  if
  \begin{gather*}
    \left\{ 
      z\in \mathcal{Z}
      \ 
      |
      \ 
      \psi(z)
      \cap
      O
      \neq
      \emptyset
    \right\}
    \in
    \Sigma_{\mathcal{Z}}
    \qquad
    \text{for all open subsets}
    \ 
    O\subset \mathcal{S}
    \,.
  \end{gather*}
  A \textbf{selector} from a correspondence $\psi\colon \mathcal{Z}\twoheadrightarrow \mathcal{S}$ is a function $s\colon \mathcal{Z}\to \mathcal{S}$ that satisfies 
  \begin{align*}
s(z)\in\psi(z)
\qquad
\text{for all}\ 
z\in\mathcal{Z}
\,.
  \end{align*}
  

\begin{definition}
  Let 
  $(\mathcal{Z},\Sigma_{\mathcal{Z}})$ be a measurable space, and let $\mathcal{S}_1$ and $\mathcal{S}_2$  be topological space.
  A function 
  $f\colon \mathcal{Z}\times \mathcal{S}_1 \to \mathcal{S}_2$
  is a \textbf{Caratheodory function} if
  \begin{align*}
    f(\cdot,s_1)
    &
    \colon
    \mathcal{Z}\to \mathcal{S}_2
    \qquad
    \text{is}\ 
    (\Sigma_{\mathcal{Z}},\mathcal{B}(\mathcal{S}_2))-measurable
    \ 
    \text{for all}
    \ 
    s_1\in \mathcal{S}_1
    \,,
    \intertext{and}
    f(z,\cdot)
    &
    \colon
    \mathcal{Z}\to \mathcal{S}_2
    \qquad
    \text{is continuous for all}\ 
    z\in \mathcal{Z}
    \,.
  \end{align*}
\end{definition}
\begin{theorem}
  \label{th:argmax}
  Let $\mathcal{S}$ be a separable metrizable space and
  $
  (\mathcal{Z},\Sigma_{\mathcal{Z}})
  $
  a measurable space.
  Let $\psi\colon \mathcal{Z} \twoheadrightarrow \mathcal{S}$ be a weakly measurable correspondence with non-empty compact values, and suppose
  $f\colon \mathcal{Z}\times \mathcal{S} \to \R$
  is a Caratheodory function. Define the value function 
  $m\colon \mathcal{Z}\to \R$ by
  \begin{gather*}
    m(z):=\max_{s\in\psi(z)}f(z,s)
    \,,
  \end{gather*}
  and the correspondence 
  $\mu\colon \mathcal{Z}\twoheadrightarrow \mathcal{S}$ of maximizers by
  \begin{gather*}
    \mu(z):= \left\{ 
      s\in \psi(z)
      |
      f(z,s)=m(z)
    \right\}
    \,.
  \end{gather*}
  Then the value function $m$ is measurable, 
  the argmax correspondence $\mu$ has non-empty and compact values,
  is measurable and admits a measurable selector.
\end{theorem}
\begin{proof}
  \cite[Theorem~18.19]{Aliprantis2007}
\end{proof}
\begin{takeaways}
  Solving an optimization problem, that has a Caratheodory objective function, on a weakly-measurable, non-empty and compact search space, allows for measurable optimal solutions.
\end{takeaways}

\section{Measurable Dual Solution}
  Next, we formulate the feasibility assumption. The assumption is (asymptotically) justified by Theorem~\ref{th:cons_dual}.
Note that we assume compactness to be able to apply Theorem~\ref{th:argmax}.
\begin{assumption}
  \label{asu:feas_dual_sol}
  For all $N\in\mathbb{N}$ there exists a non-empty, compact, and deterministic 
  parameter space 
  $
  \Theta_N
  \subset
  \R^{N}_{\ge 0}
  \times
  \R
  \times
  \R^N
  $
  such that the optimal solution 
  $
  \left( \rho^\dagger,\lambda_0^\dagger,\lambda^\dagger \right)
  $
  of Problem~\ref{dual}
  are contained in $\Theta_N$.
\end{assumption}
Based on this assumption it is easy to derive measurability for the dual solutions 
  $
  \left( \rho^\dagger,\lambda_0^\dagger,\lambda^\dagger \right)
  $.
  To this end, we take a closer look at the objective function.
  \begin{definition}
    \label{def:rand_obj_f}
We define the (random) objective function of
Problem~\ref{dual} by
  \begin{align*}
    &
  G
  \ 
  \colon
  \ 
  \left(
  \Omega,
\sigma
(D_N)
  \right)
  \times
  \left(
  \R^N_{\ge 0}
  \times
  \R
  \times
  \R^{N}
  \right)
  \ 
  \to
  \ 
  \overline{\R}
  \intertext{with}
    &
  G(\omega,(\rho,\lambda_0,\lambda))
  \ 
  =
  \ 
  \infty
  \qquad 
  \text{if}\quad 
  \rho_i
  \neq
  \left[ 
  \varphi^{-1}
  (0)
  -
  \left( 
  \lambda_0 + \inner{B(X_i)}{\lambda}
  \right)
  \right]^+
  \
  \text{for some}\ i>n
  \,,
  \\
  \intertext{and else}\qquad
  &
  G(\omega,(\rho,\lambda_0,\lambda))
  \\
  &
  \ 
  =
  \ 
  \frac{1}{N}
\sum_{i=1} 
  ^N
  \Big[
  T_i(\omega)
  \cdot
  \varphi^*
  \!
  \left( 
    \rho_i
    +
\lambda_0
+
\inner
{B(X_i)(\omega)}
{
\lambda
}
  \right)
  \ 
  -
  \ 
\lambda_0
-
\inner
{B(X_i)(\omega)}
{
\lambda
}
\Big]
\\
&
  \qquad 
+
\ 
\inner
{\delta(\omega)}
{
  |\lambda|
}
\,.
  \end{align*}
  \end{definition}
  \begin{lemma}
    \label{lem:caratheo_G}
    The function $G$ of Definition~\ref{def:rand_obj_f}
    is Caratheodory.
  \end{lemma}
  \begin{proof}
    This follows from Lemma~\ref{1165}
    (continuity of $\varphi^*$) and the measurability 
  of all random variables included.
  \end{proof}
  In the proof of the next lemma we gather the arguments and apply Theorem~\ref{th:argmax}.
\begin{lemma}
  \label{lem:meas_dual_sol}
  Let Assumption~\ref{asu:feas_dual_sol} hold true.
  Then,
  for all $N\in\mathbb{N}$ the dual solution
  \begin{align*}
  \left( \rho^\dagger,\lambda_0^\dagger,\lambda^\dagger \right)
    \ 
    \colon
   \  
    \Omega
    \ 
    \to
    \ 
  \R^N_{\ge 0}
  \times
  \R
  \times
  \R^{N}
  \end{align*}
  to
  Problem~\ref{dual} 
  is
  \begin{align*}
  \left(
    \sigma \left( D_N \right),\mathcal{B}
  \left(
  \R^N_{\ge 0}
  \times
  \R
  \times
  \R^{N}
  \right)
  \right)
  -\text{measurable}
  \,.
  \end{align*}
\end{lemma}
\begin{proof}
  Since $\Theta_N$ is deterministic (by Assumption~\ref{asu:feas_dual_sol})
  we can define the (constant) correspondence
  $\omega \mapsto \Theta_N$.
  Clearly, this is weakly-measurable, non-empty and compact.
  Next, we consider the (random) objective function of (the maximize version of) Problem~\ref{dual}, that is, $-G$ (see Definition~\ref{def:rand_obj_f}).
  By Lemme~\ref{lem:caratheo_G}, $-G$  is a Caratheodory function.
  Since $-G$ is also strictly concave, it has a unique argmax in $\Theta_N$.
  By Assumption~\ref{asu:feas_dual_sol} this is 
  $
  \left( \rho^\dagger,\lambda_0^\dagger,\lambda^\dagger \right)
  $.
  By Theorem~\ref{th:argmax} this is
  \begin{align*}
  \left(
    \sigma
    (D_N)
    ,\mathcal{B}
  \left(
  \R^N_{\ge 0}
  \times
  \R
  \times
  \R^{N}
  \right)
  \right)
  -\text{measurable}
  \,.
  \end{align*}
\end{proof}

\begin{takeaways}
  With suitable assumptions on the feasibility of Problem~\ref{dual}, we can construct measurable dual solutions.
  An important tool to obtain measurability is the argmax measurability theorem (Theorem~\ref{th:argmax}).
\end{takeaways}

Before we can define the weights process (based on the dual solution), we specify the basis functions.
\section{Basis Functions}
  Let $
\left(
\mathcal{P}_N
\right)
$
denote a sequence of countable, $\mathcal{B}$-measurable partitions 
\begin{align*}
\mathcal{P}_N= \left\{
  A_{N,1},
  A_{N,2},
  \ldots
\right\}
\subset \mathcal{B}(\R^d)
\end{align*}
of $\R^d$, that is, 
\begin{align*}
  A_{N,i}\cap A_{N,j}=\emptyset
  \qquad
  \text{if}\ i\neq j
  \qquad
  \text{and}
  \qquad 
  \bigcap_{i\in\mathbb{N}}A_{N,i}
  \ 
  =
  \ 
  \R^d
  \,.
\end{align*}
We define
$ A_N(x) $ to be the cell of $ \mathcal{P}_N $ containing $x$, that is,
\begin{align*}
  A_N
  \colon
  \R^d 
  \ 
  \twoheadrightarrow 
  \ 
  \R^d  
  \,,\qquad
  x
  \ 
  \mapsto
  \ 
  A_N(x)
  \,,
\end{align*}
where $A_N(x)$ is the only cell containing $x$. 

\begin{lemma}
  \label{lem:basis_equiv_r}
  The relation
  \begin{align*}
    x\sim y
    \qquad
    :\Leftrightarrow
    \qquad
    x\in A_N(y)
  \end{align*}
  is an equivalence relation.
\end{lemma}
\begin{proof}
  The proof is simple. We omit it.
\end{proof}
Before we define the basis vector, we assume 
uniform partition width such that
\begin{align*}
  \lambda(A_N)
  \ 
  =:
  \ 
  h_N^d
  \ 
  \to
  \ 
  0 
  \qquad
  \text{for}\ N\to\infty
  \,.
\end{align*}
Next, we define the (empirical) basis functions vector
\begin{align}
  \label{def:basis}
  B\colon
  \R^d\times \R^{d\cdot N}
  \ 
  \to
  \ 
  \R
  \,,
  \qquad
  (x,(x_1,\ldots,x_N))
  \ 
  \mapsto
  \ 
  \frac
  {
    \left[
    \mathbf{1}
    _{
      A_N(x)
    }
    (x_k)
    \right]
    _{k\in \left\{
        1,\ldots,N
    \right\}}
  }
  {
    \sum_{j=1}^N
    \mathbf{1}
    _{
      A_N(x)
    }
    (x_j)
    }
  \,,
\end{align}
where we keep to the convention $"0/0=0"$.
We shall extend $B$ to depend on the random vectors
$X,X_1,\ldots,X_N$.
The next lemma studies the measurability of the extensions.
\begin{lemma}
  \label{lem:basis_meas}
  \quad
  \begin{enumerate}[label=(\roman*)]
\item
  $B(\cdot,(X_1,\ldots,X_N))(\omega)$ is 
  $\left(
    \mathcal{B}(\R^d),\mathcal{B}(\R^N)
  \right)$-measurable
  and
  constant on each cell 
  $A_N\in\mathcal{P}_N$
  for all $\omega\in\Omega$. 
\item
  $B(X,(X_1,\ldots,X_N))$ is $\left(
    \sigma(X,D_N),\mathcal{B}(\R^N)
  \right)$-measurable. 
  \end{enumerate}
\end{lemma}
%
\begin{proof}
Consider
for $k\in \left\{
  1,\ldots,N
\right\}$
and $\omega\in\Omega$
the indicator function
\begin{align}
  \label{33342}
  \mathbf{1}
  _
  {A_N(X_k(\omega))}
  \colon \R^d\to \left\{
    0,1
  \right\}
  \,.
\end{align}
Since 
$
  {A_N(X_k(\omega))}
  \in\mathcal{B}(\R^d)
$
this is a 
  $\left(
    \mathcal{B}(\R^d),\mathcal{B}(\R)
  \right)$-measurable
  function.
  From the definition of $B$ \eqref{def:basis} it follows the first part of (i).
  Since the indicator function in \eqref{33342} is 1 if $
  x\in
  {A_N(X_k(\omega))}
  $
  and 0 else, it is also constant on each cell
  $A_N\in\mathcal{P}_N$.
  It follows (i).
  To prove (ii), note that
\begin{align*}
  \mathbf{1}
  _
  {A_N(X_k(\omega))}(X(\omega))
  \ 
  =
  \ 
  \mathbf{1}
  \bigcup_{i\in\mathbb{N}}
  \left\{
    X,X_k \in A_{N,i}
  \right\}
  (\omega)
  \qquad
  \text{for all}\ 
  \omega\in\Omega
  \,,
\end{align*}
and
$
  \bigcup_{i\in\mathbb{N}}
  \left\{
    X,X_k \in A_{N,i}
  \right\}
  \in\sigma(X,D_N)
  $.
\end{proof}
%
Now we gather some useful properties of the (empirical) basis vector.
\begin{lemma}
  \label{lem:basis_sum}
  Let $(x,x_1,\ldots,x_N)\in\R^{d(N+1)}$.
  \begin{enumerate}[label=(\roman*)]
    \item
      $
      \sum_{k=1}^{N} 
      B_k(x,x_1,\ldots,x_N)
      \in
      \left\{ 0,1 \right\}
      $. 
      In particular,
      $
        x_1,\ldots,x_N\notin A_N(x)
      $
      is equivalent to
      $
      \sum_{k=1}^{N} 
      B_k(x,x_1,\ldots,x_N)
      =0
      $
    \item
      $
      \sum_{k=1}^{N} 
      B_k(x_i,x_1,\ldots,x_N)
      \ 
      =
      \ 
      1
      \qquad
      \text{for all}\ 
      i\in \left\{ 1,\ldots,N \right\}
      $.
      \item
        $
        \norm
        {
      B(x,x_1,\ldots,x_N)
        }_2
        \ 
        \le 1
        \ 
        $
      \item
        $
        B_k(x_i,x_1,\ldots,x_N)
        \ 
        =
        \ 
        B_i(x_k,x_1,\ldots,x_N)
        \qquad
        \text{for all}\ 
        i,k\in \left\{ 1,\ldots,N \right\}
        $
  \end{enumerate}
\end{lemma}
\begin{proof}
  Let $(x,x_1,\ldots,x_N)\in\R^{d(N+1)}$.
  We prove \textit{(i)}.
  Then \textit{(ii)} is a direct consequence of \textit{(i)}.
  If 
      $
        x_1,\ldots,x_N\notin A_N(x)
      $,
  then
  \begin{align*}
    B_k(
        x,x_1,\ldots,x_N
    )
    \ 
    =
    \ 
\frac
  {
    \mathbf{1}
    _{
      A_N(x)
    }
    (x_k)
  }
  {
    \sum_{j=1}^N
    \mathbf{1}
    _{
      A_N(x)
    }
    (x_j)
    }
    \ 
    =
    \ 
    0
    \qquad
    \text{for all}\ 
    k\in \left\{ 1,\ldots,N \right\}
    \,.
  \end{align*}
  On the other hand, if the sum is 0 it holds
  \begin{align*}
    \mathbf{1}
    _{
      A_N(x)
    }
    (x_k)
    \ 
  =
  \ 
  0
    \qquad
    \text{for all}\ 
    k\in \left\{ 1,\ldots,N \right\}
    \,.
  \end{align*}
  It follows the desired equivalence.
  If 
  \begin{align*}
    \mathbf{1}
    _{
      A_N(x)
    }
    (x_k)
    \ 
  =
  \ 
  1
    \qquad
    \text{for some}\ 
    k\in \left\{ 1,\ldots,N \right\}
    \,,
  \end{align*}
  then
  $
    \sum_{j=1}^N
    \mathbf{1}
    _{
      A_N(x)
    }
    (x_j)
    \ge 1
  $
  and thus "$0/0$" doesn't occure. 
  It follows
  \begin{align*}
      \sum_{k=1}^{N} 
      B_k(x,x_1,\ldots,x_N)
      \ 
      =
      \ 
\frac
  {
      \sum_{k=1}^{N} 
    \mathbf{1}
    _{
      A_N(x)
    }
    (x_k)
  }
  {
    \sum_{j=1}^N
    \mathbf{1}
    _{
      A_N(x)
    }
    (x_j)
    }
    \ 
    =
    \ 
    1
    \,.
  \end{align*}
  To prove \textit{(iii)}, note that by \textit{(i)}
  \begin{align*}
        \norm
        {
      B(x,x_1,\ldots,x_N)
        }_2^2
        \ 
        =
        \ 
      \sum_{k=1}^{N} 
      B_k(x,x_1,\ldots,x_N)^2
        \ 
      \le
        \ 
      \sum_{k=1}^{N} 
      B_k(x,x_1,\ldots,x_N)
        \ 
      \le
        \ 
      1
      \,.
  \end{align*}
  To prove \textit{(iv)}, note that by Lemma~\ref{lem:basis_equiv_r}
  and by symmetry and transitivity of the equivalence relation
  $x\in A_N(y)$
  it holds
  \begin{align*}
      B_k(x_i,x_1,\ldots,x_N)
      &
    \ 
      =
    \ 
 \frac
  {
    \mathbf{1}
    \left\{ 
      x_k
      \in
      A_N(x_i)
    \right\}
  }
  {
    \sum_{j=1}^N
    \mathbf{1}
    \left\{ 
      x_j
      ,
      x_k
      \in
      A_N(x_i)
    \right\}
    }
    \ 
    =
    \ 
 \frac
  {
    \mathbf{1}
    \left\{ 
      x_i
      \in
      A_N(x_k)
    \right\}
  }
  {
    \sum_{j=1}^N
    \mathbf{1}
    \left\{ 
      x_j
      \in
      A_N(x_k)
    \right\}
    }
    \\
    &
    \ 
    =
    \ 
      B_i(x_k,x_1,\ldots,x_N)
      \,.
  \end{align*}
\end{proof}
%
Now we show that the basis vector plays well with uniformly continuous functions. The result seems simple, yet the consequence are great. It allows us later on to specify an oracle parameter instead of assuming its existence (see \cite[Assumption~1.6]{Wang2019}). This greatly clarifies the proofs.
\begin{lemma}
  \label{lem:basis_approx_f}
  Let $(x,x_1,\ldots,x_N)\in\R^{d(N+1)}$.
  For all uniformly continuous functions $f\colon \R^d\to \R$ it holds
 \begin{align*}
   \begin{split}
   &
   \left|
  \sum_{k=1}^{N}
    B_k(x_i,x_1,\ldots,x_N)\cdot 
    f(x_k)
    -
    f(x_i)
   \right|
   \ 
   \le
   \ 
   \omega
   \left(
    f,h_N^d
   \right)
   \qquad
   \text{for all}\ 
   i\in \left\{ 1,\ldots,N \right\}
   \,,
   \end{split}
 \end{align*}
 where $\omega(f,\cdot)$ is the uniform modulus of continuity of $f$. 
\end{lemma}
\begin{proof}
  It follows from Lemma~\ref{lem:basis_sum}\textit{(ii)}
  \begin{align*}
   \begin{split}
   &
   \left|
  \sum_{k=1}^{N}
    B_k(x_i,x_1,\ldots,x_N)\cdot 
    f(x_k)
    -
    f(x_i)
   \right|
   \\
   &
   \ 
   \le
   \ 
   \left|
  \sum_{k=1}^{N}
    B_k(x_i,x_1,\ldots,x_N)
    \left(
    f(x_k)
    -
    f(x_i)
    \right)
   \right|
   \\
   &
   \ 
   \le
   \ 
  \sum_{k=1}^{N}
    B_k(x_i,x_1,\ldots,x_N)
    \cdot
    \mathbf{1}\left\{
      x_k\in A_N(x_i)
    \right\}
    \left|
    f(x_k)
    -
    f(x_i)
    \right|
   \\
   &
   \ 
   \le
   \ 
   \omega
   \left(
    f,h_N^d
   \right)
   \,.
   \end{split}
 \end{align*}
\end{proof}
Next, we bring forward the applications of Lemma~\ref{lem:basis_approx_f} that we need.
\begin{lemma}
  \label{lem:basis_2}
  Let $(x,x_1,\ldots,x_N)\in\mathcal{X}^{N+1}$.
  It holds
  for $N\to\infty$
  \begin{enumerate}[label=(\roman*)]
      \item
      \begin{align*}
        \frac
        {1}
        {N}
        \sum_{i,k=1}^{N}
            \left|
        B_k(x_i,x_1,\ldots,x_N)
        \cdot
        \varphi^{'}
            \left(
              \frac
              {1}
              {\pi(x_k)}
            \right)
            \ 
            -
            \ 
            \varphi^{'}
            \left(
              \frac
              {1}
              {\pi(x_i)}
            \right)
            \right|
            \ 
            \to
            \ 
            0
            \,,
          \end{align*}
\item
      \begin{align*}
        \sqrt{N}
        \sup_{z\in\R}
        \max_{i\in \left\{ 1,\ldots,N \right\}}
        \sum_{k=1}^{N}
            \left|
        B_k(x_i,x_1,\ldots,x_N)
        \cdot
        F_{Y(1)}(z|x_k)
            \ 
            -
            \ 
        F_{Y(1)}(z|x_i)
            \right|
            \ 
            \to
            \ 
            0
            \,.
      \end{align*}
\end{enumerate}
\end{lemma}
\begin{proof}
  By Lemma~\ref{lem:basis_approx_f},
  the uniform continuity of $\varphi^{'}$
  it holds
      \begin{align*}
        \frac
        {1}
        {N}
        \sum_{i,k=1}^{N}
            \left|
        B_k(x,x_1,\ldots,x_N)
        \cdot
            \varphi^{'}
            \left(
              \frac
              {1}
              {\pi(x_k)}
            \right)
            \ 
            -
            \ 
            \varphi^{'}
            \left(
              \frac
              {1}
              {\pi(x_i)}
            \right)
            \right|
            \ 
            \le
            \ 
   \omega
   \left(
     \varphi^{'},h_N^d
   \right)
            \ 
            \to
            \ 
            0
          \end{align*}
          for $N\to\infty$.
          Likewise
\begin{align*}
  &
        \sqrt{N}
        \sup_{z\in\R}
        \max_{i\in \left\{ 1,\ldots,N \right\}}
        \sum_{k=1}^{N}
            \left|
        B_k(x_i,x_1,\ldots,x_N)
        \cdot
        F_{Y(1)}(z|x_k)
            \ 
            -
            \ 
        F_{Y(1)}(z|x_i)
          \right|
             \\
            &
            \ 
            \le
            \ 
            \sqrt{N}
            \sup_{z\in\R}
            \omega
            \left(
        F_{Y(1)}(z|\cdot)
        ,
        h_N^d
            \right)
            \ 
            \to
            \ 
            0
            \qquad
            \text{for}
            \ 
            N\to\infty
            \,.
\end{align*}
        \end{proof}
        \begin{remark}
We want to comment on the assumption
\begin{gather*}
  \sqrt{N}
  \sup_{z\in\R}
  \omega
  \left( 
    F_{Y(1)}(z|\cdot)
    ,h_N^d
  \right)
  \to
  0
  \qquad
  \text{for}\ 
  N\to \infty
  \,,
\end{gather*}
I decided to keep this more general (and abstract) assumption, althogh
there are many (more concrete, yet stronger) assumptions on the regularity of
$
    F_{Y(1)}(z|\cdot)
$
and the convergence speed of $h_N$.
If for example 
$
    F_{Y(1)}(z|\cdot)
$
is $\alpha$-Hölder continuous with $\alpha\in(0,1]$ for all $z\in\R$, it suffices $\sqrt{N}h_N^{\alpha\cdot d}\to0$.

        \end{remark}
\begin{takeaways}
  Basis functions of non-parametric
  partitioning estimates are new to the framework of balancing weights.
  They play well with uniformly continuous functions and promise to simplify the analysis. 
  This choice of basis functions waits to be tested in practice.
\end{takeaways}

\section{Weights Process}
  Based on Theorem~\ref{dual_solution_th}
we want to use the dual 
solution 
$
\left( \rho^\dagger,\lambda_0^\dagger,\lambda^\dagger \right)
$
to construct weights.
To this end, we define the (empirical) weights function
\begin{align*}
 w\ \colon\
 &
 \left( 
  \R^d\times \R^{d\cdot N}
 \right)
  \times
  \left( 
\R^N_{\ge 0}\times \R\times \R^N
  \right)
  \to
  \R^N
  \\
 &
  \left( 
  (x,x_1,\ldots,x_N),(\rho,\lambda_0,\lambda)
  \right)
  \ 
  \mapsto
  \ 
  \left[ 
  (\varphi^{'})^{-1}
  \left( 
    \rho_i
    +
    \lambda_0
    +
    \inner
    {B(x,x_1,\ldots,x_N)}
    {\lambda}
  \right)
\right]_{i\in \left\{ 1,\ldots,N \right\}}
\,.
\end{align*}
\begin{definition}
  Let 
  $
\left( \rho^\dagger,\lambda_0^\dagger,\lambda^\dagger \right)
  $
  be the dual solution of Lemma~\ref{lem:meas_dual_sol}.
  We define the weights process 
  $\left\{ w^\dagger(x) | x\in\R^d\right\}$
  by
  \begin{align*}
    w^\dagger(x) 
    \ 
    :=
    \ 
    w
    \left( 
    \left( 
    x,X_1,\ldots,X_N,
    \right)
    ,
\left( \rho^\dagger,\lambda_0^\dagger,\lambda^\dagger \right)
    \right)
    \qquad
    \text{for all}\ 
    x\in\R^d
    \,.
  \end{align*}
\end{definition}
\begin{lemma}
  \label{lem:weights:meas}
  \quad
  \begin{enumerate}[label=(\roman*)]
\item
  $w^\dagger(\cdot)(\omega)$ is 
  $\left(
    \mathcal{B}(\R^d),\mathcal{B}(\R^N)
  \right)$-measurable
  and
  constant on each cell 
  $A_N\in\mathcal{P}_N$
  for all $\omega\in\Omega$. 
\item
  $w^\dagger(X)$ is $\left(
    \sigma(X,D_N),\mathcal{B}(\R^N)
  \right)$-measurable. 
  \end{enumerate}
\end{lemma}
\begin{proof}
  This is a direct consequence of Lemme~\ref{lem:basis_meas}, Lemma~\ref{lem:meas_dual_sol}
  and 
  the (assumed) continuity of $(\varphi^{'})^{-1}$.
\end{proof}


\begin{lemma}
  \label{weights_l_inf}
  It holds
  $w_i^\dagger(X)\in L^\infty(\P)$
  for all $i\in \left\{ 1,\ldots,N \right\}$.
\end{lemma}
\begin{proof}
  By Lemma~\ref{lem:basis_sum}.\textit{(iii)} it holds
  \begin{align*}
  \left| 
    \rho_i^\dagger
    +
    \lambda_0^\dagger
    +
    \inner
    {B(x,x_1,\ldots,x_N)}
    {\lambda^\dagger}
  \right|
  \ 
  \lesssim
  \ 
  \norm{
\left( \rho^\dagger,\lambda_0^\dagger,\lambda^\dagger \right)
  }_2
  \qquad
  \text{for all}\ 
  i \in \left\{ 1,\ldots,N \right\}
  \,,
  \end{align*}
  where $\lesssim$ denotes the lesser-or-equal-up-to-a-uniform-constant order, that is, we choose $C>1$ independent of $N$ large enough, such that $a\lesssim b$ means $a\le C\cdot b$.
  \index{$\lesssim$, 
lesser-or-equal-up-to-a-uniform-constant order
  }
  Since
  $
\left( \rho^\dagger,\lambda_0^\dagger,\lambda^\dagger \right)
  $ is contained in the deterministic and compact parameter space $\Theta_N$,
  it holds
  \begin{align*}
  \norm{
\left( \rho^\dagger,\lambda_0^\dagger,\lambda^\dagger \right)
  }_2
  \in 
  L^{\infty}(\P)
  \,.
  \end{align*}
  By the (assumed) uniform continuity of 
  $
  (\varphi^{'})^{-1}
  $
  on $\R$, it follows 
  $w_i^\dagger(X)\in L^\infty(\P)$
  for all $i\in \left\{ 1,\ldots,N \right\}$.
\end{proof}
Next, we want to simplify the weights process in the spirit of Lemma~\ref{lem:simple_weights}.
In other words, we want to become independent of the index $i$ in $w_i^\dagger$. This will be helpful in the subsequent analysis.
To this end, we define the (empirical) simplified weights function
\begin{align*}
 w_0\ \colon\
 &
 \left( 
  \R^d\times \R^{d\cdot N}
 \right)
  \times
  \left( 
    \R\times \R^N
  \right)
  \to
  [0,\infty)
  \\
 &
  \left( 
  (x,x_1,\ldots,x_N),(\lambda_0,\lambda)
  \right)
  \ 
  \mapsto
  \ 
  \left[ 
  (\varphi^{'})^{-1}
  \left( 
    \lambda_0
    +
    \inner
    {B(x,x_1,\ldots,x_N)}
    {\lambda}
  \right)
\right]^+
\,.
\end{align*}
\begin{definition}
  Let 
  $
\left( \rho^\dagger,\lambda_0^\dagger,\lambda^\dagger \right)
  $
  be the dual solution of Lemma~\ref{lem:meas_dual_sol}.
  We define the simplified weights process 
  $\left\{ w_0^\dagger(x) \,|\, x\in\R^d\right\}$
  by
  \begin{align*}
    w_0^\dagger(x) 
    \ 
    :=
    \ 
    w_0
    \left( 
    \left( 
    x,X_1,\ldots,X_N,
    \right)
    ,
\left( \lambda_0^\dagger,\lambda^\dagger \right)
    \right)
    \qquad
    \text{for all}\ 
    x\in\R^d
    \,.
  \end{align*}
\end{definition}
\begin{lemma}
  \label{lem:weights:meas:0}
  \quad
  \begin{enumerate}[label=(\roman*)]
\item
  $w_0^\dagger(\cdot)(\omega)$ is 
  $\left(
    \mathcal{B}(\R^d),\mathcal{B}(\R^N)
  \right)$-measurable
  and
  constant on each cell 
  $A_N\in\mathcal{P}_N$
  for all $\omega\in\Omega$. 
\item
  $w_0^\dagger(X)$ is $\left(
    \sigma(X,D_N),\mathcal{B}(\R^N)
  \right)$-measurable. 
  \end{enumerate}
\end{lemma}
\begin{proof}
The proof is as that of Lemma~\ref{lem:weights:meas}.
\end{proof}


\begin{lemma}
  \label{weights_0_l_inf}
  It holds $w_0^\dagger(X)\in L^\infty(\P)$.
\end{lemma}
\begin{proof}
  By Lemma~\ref{weights_l_inf},
  the monotonicity of 
  $
  (\varphi^{'})^{-1}
  $
  and $\rho_i\ge 0$ for $i\le n$,
  it holds
  \begin{align*}
    w_0^\dagger(X) 
    &
    \ 
    \le
    \ 
  \left[ 
  (\varphi^{'})^{-1}
  \left( 
    \lambda_0^\dagger
    +
    \inner
    {B(X)}
    {\lambda^\dagger}
  \right)
\right]^+
\\
&
\ 
\le
\ 
  \left[ 
  (\varphi^{'})^{-1}
  \left( 
    \rho_i^\dagger
    +
    \lambda_0^\dagger
    +
    \inner
    {B(X)}
    {\lambda^\dagger}
  \right)
\right]^+
\ 
\le
\ 
\left| 
    w_i^\dagger(X) 
\right|
\ 
\in
\ 
L^\infty(\P)
  \end{align*}
\end{proof}
\begin{lemma}
  \label{w.Z=0}
 Let 
 $Z\in L^1(\P)$
  be a random variable that is independent of $D_N=(T_i,X_i)_{i\in \left\{
    1,\ldots,N
  \right\}}$ 
  with
  $
\E
\left[
  Z
  \,
  |
  \, 
  X
\right]
= 0
  $
  almost surely.
  It holds
  \begin{gather*}
  \E
  \left[
    w_0^\dagger(X)
  \cdot Z
  \right]
  \ 
  =
  \ 
  0
  \,.
  \end{gather*}
\end{lemma}
\begin{proof}
  By Lemma~\ref{weights_0_l_inf} it holds
  \begin{gather}
    \label{9876}
    \norm{
  w_0^\dagger(X)\cdot Z
    }_{L^1(\P)}
    \ 
  \le
    \ 
  \norm{w_0^\dagger(X)}_{L^\infty(\P)}
  \norm{Z}_{L^1(\P)}
  \ 
  <
  \ 
  \infty
  \,.
  \end{gather}
  By 
  \eqref{9876},
  $Z\perp D_N$
  and
  $
\E
\left[
  Z
  \,
  |
  \, 
  X
\right]
= 0
  $
  almost surely
  it holds 
  \begin{align*}
    \E
  \left[
  w_0^\dagger(X)
  \cdot
  Z
  \,
  |
  \,
  D_N,X
  \right]
  &
  \ 
  =
  \ 
  w_0^\dagger(X)
  \cdot
  \E
  \left[
  Z
  \,
  |
  \,
  D_N,X
  \right]
  \\
  &
  \ 
  =
  \ 
  w_0^\dagger(X)
  \cdot
  \E
  \left[
  Z
  \,
  |
  \,
  X
  \right]
  \
  =
  \ 
  0
  \qquad
  \text{almost surely.}
  \end{align*}
  Note, that $w_0^\dagger(X)$ is 
  $
  \left(
  \sigma(D_N,X),\mathcal{B}(\R)
  \right)
  $-measurable.  
  Thus
  \begin{gather*}
    \E
    \left[
  w_0^\dagger(X)
  \cdot
  Z
  \,
    \right]
    \ 
    =
    \ 
    \E
    \left[
 \E
  \left[
  w_0^\dagger(X)
  \cdot
  Z
  \,
  |
  \,
  D_N,X
  \right]
    \right]
    \ 
    =
    \ 
    0
    \,.
     \end{gather*}
\end{proof}

\begin{theorem}
  \label{th:weights_constr}
  The simplified weights process satisfies the constraints
  of Problem~\ref{bw:1:primal}, that is,
  \begin{enumerate}[label=(\roman*)]
    \item
      $
      T_i\cdot w_0^\dagger(X_i)
      \ 
      \ge
      \ 
      0
      \qquad
      \text{for all}\ 
      i\in  \left\{ 1,\ldots,N \right\}
      $
    \item
      $
      \frac{1}{N}
      \sum_{i=1}^{N} 
      T_i\cdot w_0^\dagger(X_i)
      \ 
      =
      \ 
      1
      $
    \item
      For all $k\in \left\{ 1,\ldots,N \right\}$
      it holds
      \begin{align*}
      \left| 
      \frac{1}{N}
      \left( 
        \sum_{i=1}^{N} 
      T_i\cdot w_0^\dagger(X_i)
      \cdot
        B_k(X_i,X_1,\ldots,X_N)
        \
        -
        \
        \sum_{i=1}^{N} 
        B_k(X_i,X_1,\ldots,X_N)
      \right)
      \right|
      \ 
      \le
      \ 
      \delta_k
      \end{align*}
  \end{enumerate}
\end{theorem}
\begin{proof}
  This follows from Theorem~\ref{dual_solution_th},
  Lemma~\ref{lem:simple_weights}
  and the construction of the simplified weights process.
\end{proof}
To avoid notational overload, from now on we write
\begin{align*}
  B(x)
  \ 
  :=
  \ 
  B(x,X_1,\ldots,X_N)
  \qquad
  \text{for all}\ 
  x\in\R^d
  \,.
\end{align*}


  In the formulation of Theorem~\ref{dual} we encounter "If (...) there exists the optimal solution $(\rho^\dagger,\lambda_0^\dagger,\lambda)$ ... " .
To be able to study asymptotic properties of the solutions we have to become independent of this assumption.
For this we need some tools from functional analysis.
\section{Argmax Measurability Theorem}
We follow \cite{Aliprantis2007}.
A \textbf{correspondence} $\psi$ from a set $S_1$ to a set $S_2$ assigns to each $s_1\in S_1$ a subset $\psi(s_1)\subset S_2$.
To clarify that we map $s_1$ to a set, we use the double arrow, that is,
$
  \psi
  \colon
  S_1
  \twoheadrightarrow
  S_2
$.
  Let 
  $(\mathcal{Z},\Sigma_{\mathcal{Z}})$ be a measurable space and $\mathcal{S}$  a topological space.
  We say, that a correspondence 
  $
  \psi
  \colon
  \mathcal{Z}
  \twoheadrightarrow
  \mathcal{S}
  $
  is 
  \textbf{
  weakly measurable
  },
  if
  \begin{gather*}
    \left\{ 
      z\in \mathcal{Z}
      \ 
      |
      \ 
      \psi(z)
      \cap
      O
      \neq
      \emptyset
    \right\}
    \in
    \Sigma_{\mathcal{Z}}
    \qquad
    \text{for all open subsets}
    \ 
    O\subset \mathcal{S}
    \,.
  \end{gather*}
  A \textbf{selector} from a correspondence $\psi\colon \mathcal{Z}\twoheadrightarrow \mathcal{S}$ is a function $s\colon \mathcal{Z}\to \mathcal{S}$ that satisfies 
  \begin{align*}
s(z)\in\psi(z)
\qquad
\text{for all}\ 
z\in\mathcal{Z}
\,.
  \end{align*}
  

\begin{definition}
  Let 
  $(\mathcal{Z},\Sigma_{\mathcal{Z}})$ be a measurable space, and let $\mathcal{S}_1$ and $\mathcal{S}_2$  be topological space.
  A function 
  $f\colon \mathcal{Z}\times \mathcal{S}_1 \to \mathcal{S}_2$
  is a \textbf{Caratheodory function} if
  \begin{align*}
    f(\cdot,s_1)
    &
    \colon
    \mathcal{Z}\to \mathcal{S}_2
    \qquad
    \text{is}\ 
    (\Sigma_{\mathcal{Z}},\mathcal{B}(\mathcal{S}_2))-measurable
    \ 
    \text{for all}
    \ 
    s_1\in \mathcal{S}_1
    \,,
    \intertext{and}
    f(z,\cdot)
    &
    \colon
    \mathcal{Z}\to \mathcal{S}_2
    \qquad
    \text{is continuous for all}\ 
    z\in \mathcal{Z}
    \,.
  \end{align*}
\end{definition}
\begin{theorem}
  \label{th:argmax}
  Let $\mathcal{S}$ be a separable metrizable space and
  $
  (\mathcal{Z},\Sigma_{\mathcal{Z}})
  $
  a measurable space.
  Let $\psi\colon \mathcal{Z} \twoheadrightarrow \mathcal{S}$ be a weakly measurable correspondence with non-empty compact values, and suppose
  $f\colon \mathcal{Z}\times \mathcal{S} \to \R$
  is a Caratheodory function. Define the value function 
  $m\colon \mathcal{Z}\to \R$ by
  \begin{gather*}
    m(z):=\max_{s\in\psi(z)}f(z,s)
    \,,
  \end{gather*}
  and the correspondence 
  $\mu\colon \mathcal{Z}\twoheadrightarrow \mathcal{S}$ of maximizers by
  \begin{gather*}
    \mu(z):= \left\{ 
      s\in \psi(z)
      |
      f(z,s)=m(z)
    \right\}
    \,.
  \end{gather*}
  Then the value function $m$ is measurable, 
  the argmax correspondence $\mu$ has non-empty and compact values,
  is measurable and admits a measurable selector.
\end{theorem}
\begin{proof}
  \cite[Theorem~18.19]{Aliprantis2007}
\end{proof}


\section{Pseudo Dual Solution}
There are two observations that help achieving our goal.
First, if we solve Problem~\ref{dual} on a compact search space, there always exists a measurable solution (if the objective function is Caratheodory).
Second (with hindsight), we want optimal solutions to estimate an oracle parameter $\lambda^*$ well. 
To this end, we will define a (random) compact parameter space that always contains the oracle parameter $\lambda^*$ and restrict Problem~\ref{dual} to it.
Then Theorem~\ref{th:argmax} gives us (under weak measurability conditions) a measurable solution of the restricted Problem~\ref{dual}.
In some cases, this is only a local solution. It becomes global if it is in the interior of the compact parameter space.
Thus we call it the restricted, or the pseudo solution.

Consider 
the random vector (the oracle parameter)
\begin{align*}
  \lambda^*
  \ 
  :=
  \ 
  \left(
  0_{N},0,
  \left[
  \varphi^{'}
  \left(
  \frac
  {1}
  {\pi(X_i)}
  \right)
\right]_{i\in \left\{
  1,\ldots,N
\right\}}
  \right)
\end{align*}
with values in 
$\R^N_{\ge 0}\times \R\times \R^N$.
Clearly,  
\begin{align*}
 \lambda^*
 \qquad
 \text{is}
 \qquad
 \left(
( 
\sigma
\left( 
  (T_i,X_i)_{i\in 
\left\{ 1,\ldots,N \right\}
  } 
\right)
,
\mathcal{B}
\left( 
\R^N_{\ge 0}\times \R\times \R^N
\right)
\right)
-\text{measurable}
\,.
\end{align*}
Next we define for $N\in\mathbb{N}$ the correspondence  
\begin{align*}
  \Theta_N\colon
  &
  \left( 
  \Omega,
\sigma
\left( 
  (T_i,X_i)_{i\in 
\left\{ 1,\ldots,N \right\}
  } 
\right)
  \right)
  \ 
\to
  \ 
\R^N_{\ge 0}\times \R\times \R^N
\\
&
\omega
  \ 
\mapsto
  \ 
\left\{ 
  \left( 
  \rho,\lambda_0,\lambda
  \right)
  \ 
  \in
  \ 
\R^N_{\ge 0}\times \R\times \R^N
  \ 
\colon
  \ 
\norm{
  \left( 
  \rho,\lambda_0,\lambda
  \right)
-\lambda^*}_2\le 1
\right\}
\,.
\end{align*}

\begin{lemma}
  \label{Theta_maes}
  For all $N\in\mathbb{N}$ it holds that 
  $\Theta_N$ is a weakly-measurable correspondence  with non-empty compact values.
\end{lemma}
\begin{proof}
  Let $N\in\mathbb{N}$. 
  Since 
  \begin{align*}
  \Theta_N(\omega)
  \ 
  \text{
  is the closed unit ball
  in
  }
  \ 
\R^N_{\ge 0}\times \R\times \R^N
\quad 
\text{around}
\quad 
  \lambda^*(\omega) 
  \,,
  \end{align*}
  it is non-empty and compact for all $\omega\in\Omega$.
  To prove that $\Theta_N$ is weakly-measurable,
  let 
  $
  O\subset
\R^N_{\ge 0}\times \R\times \R^N
  $
  be an open set and note, that
  \begin{align*}
    \left\{ 
      \Theta_N\cap O \neq \emptyset
    \right\}
    &
    \ 
    =
    \ 
    \left\{ 
      \exists
  \left( 
  \rho,\lambda_0,\lambda
  \right)
  \ 
  \in
  \ 
  O
\ 
\colon
\ 
\norm{
  \left( 
  \rho,\lambda_0,\lambda
  \right)
-\lambda^*}_2\le 1
    \right\}
    \\
    &
    \ 
    =
    \ 
    \left\{ 
      \exists
  \left( 
  \rho,\lambda_0,\lambda
  \right)
  \ 
  \in
  \ 
  O
  \cap
  \left( 
  \mathbb{Q}^N_{\ge 0}\times \mathbb{Q}\times \mathbb{Q}^N
  \right)
\ 
\colon
\ 
\norm{
  \left( 
  \rho,\lambda_0,\lambda
  \right)
-\lambda^*}_2\le 1
    \right\}
    \\
&
    \ 
    =
    \ 
    \bigcup_{
  \left( 
  \rho,\lambda_0,\lambda
  \right)
  \ 
  \in
  \ 
  O
  \cap
  \left( 
  \mathbb{Q}^N_{\ge 0}\times \mathbb{Q}\times \mathbb{Q}^N
  \right)
    }
    \left\{ 
\norm{
  \left( 
  \rho,\lambda_0,\lambda
  \right)
-\lambda^*}_2\le 1
    \right\}
    \,.
  \end{align*}
  Since
  \begin{align*}
    \left\{ 
\norm{
  \left( 
  \rho,\lambda_0,\lambda
  \right)
-\lambda^*}_2\le 1
    \right\}
    \in
\sigma
\left( 
  (T_i,X_i)_{i\in 
\left\{ 1,\ldots,N \right\}
  } 
\right)
\quad
\text{for all}\quad
  \left( 
  \rho,\lambda_0,\lambda
  \right)
  \in
\R^N_{\ge 0}\times \R\times \R^N
\,,
  \end{align*}
  and the union is countable, it follows
  $
    \left\{ 
      \Theta_N\cap O \neq \emptyset
    \right\}
    \in
\sigma
\left( 
  (T_i,X_i)_{i\in 
\left\{ 1,\ldots,N \right\}
  } 
\right)
  $.
  Thus $\Theta_N$ is weakly-measurable.
\end{proof}
We are ready to construct the measurable (pseudo) solution.
\begin{lemma}
  \label{lem:pseud_sol}
  For all $N\in\mathbb{N}$ there exists a (pseudo) solution
  \begin{align*}
    s_N
    \ 
    \colon
   \  
    \Omega
    \ 
    \to
    \ 
  \R^N_{\ge 0}
  \times
  \R
  \times
  \R^{N}
  \end{align*}
  to
  Problem~\ref{dual} restricted to $\Theta_N$ 
  that is
  \begin{align*}
  \left(
    \sigma \left( \left( T_i,X_i \right)_{i\in \left\{ 1,\ldots,N \right\}} \right),\mathcal{B}
  \left(
  \R^N_{\ge 0}
  \times
  \R
  \times
  \R^{N}
  \right)
  \right)
  -\text{measurable}
  \,.
  \end{align*}
  Furthermore,
if $s_N\in \mathrm{int}\, \Theta_N$, then it is the global solution
  $
  (\rho^\dagger,\lambda_0^\dagger,\lambda^\dagger)
  $.
\end{lemma}
\begin{proof}
  By Lemma~\ref{Theta_maes}
  the correspondence $\Theta_N$ satisfies the conditions of
  Theorem~\ref{th:argmax}.
Next, we consider the (random) objective function of (the maximize version of) Problem~\ref{dual}, that is,
  \begin{align*}
    &
  G\colon
  \left(
  \Omega,
\sigma
\left( 
  (T_i,X_i)_{i\in 
\left\{ 1,\ldots,N \right\}
  } 
  \right)
  \right)
  \times
  \left(
  \R^N_{\ge 0}
  \times
  \R
  \times
  \R^{N}
  \right)
  \to
  \R\cup \left\{
    -\infty
  \right\}
  \intertext{with}
    &
  G(\omega,(\rho,\lambda_0,\lambda))
  \ 
  =
  \ 
  -\infty
  \qquad 
  \text{if}\ 
  \rho_i\neq 0\  \text{for some}\ i>n
  \,,
  \\
  \intertext{and}\qquad
  &
  G(\omega,(\rho,\lambda_0,\lambda))
  \\
  &
  \ 
  =
  \ 
  \frac{1}{N}
\sum_{i=1} 
  ^N
  \Big[
    -
  T_i(\omega)
  \cdot
  \varphi^*
  \!
  \left( 
    \rho_i
    +
\lambda_0
+
\inner
{B(X_i)(\omega)}
{
\lambda
}
  \right)
  \ 
  +
  \ 
\lambda_0
+
\inner
{B(X_i)(\omega)}
{
\lambda
}
\Big]
  \ 
-
\ 
\inner
{\delta(\omega)}
{
  |\lambda|
}
\,.
  \end{align*}
  Clearly, the (random) objective function $G$  is a Caratheodory function
  . This follows from the (assumed) continuity of $\varphi^*$ and the measurability 
  of all random variables included.
  Since $G$ is also strictly concave, it has a unique argmax $s_N$  in $\Theta_N$. 
  By Theorem~\ref{th:argmax} this is $
  \left(
    \sigma
    \left( \left( T_i,X_i \right)_{i\in \left\{ 1,\ldots,N \right\}}\right)
    ,\mathcal{B}
  \left(
  \R^N_{\ge 0}
  \times
  \R
  \times
  \R^{N}
  \right)
  \right)
$-measurable.
Furthermore, by the strict concavity of $G$,
if $s_N\in \mathrm{int}\,\Theta_N$, then it's the global optimal solution. 
\end{proof}
We will prove later, that with probability going to 1 the latter will be the case.
Furthermore, we will prove, that the (measurable) solution converges to the random variable in probability.

Before we can define the pseudo weights (based on the pseudo dual solution), we specify the basis functions.
\section{Basis Functions}
Let $
\left(
\mathcal{P}_N
\right)
$
denote a sequence of countable, $\mathcal{B}$-measurable partitions 
\begin{align*}
\mathcal{P}_N= \left\{
  A_{N,1},
  A_{N,2},
  \ldots
\right\}
\subset \mathcal{B}(\R^d)
\end{align*}
of $\R^d$, that is, 
\begin{align*}
  A_{N,i}\cap A_{N,j}=\emptyset
  \qquad
  \text{if}\ i\neq j
  \qquad
  \text{and}
  \qquad 
  \bigcap_{i\in\mathbb{N}}A_{N,i}
  \ 
  =
  \ 
  \R^d
  \,.
\end{align*}
We define
$ A_N(x) $ to be the cell of $ \mathcal{P}_N $ containing $x$, that is,
\begin{align*}
  A_N
  \colon
  \R^d 
  \ 
  \twoheadrightarrow 
  \ 
  \R^d  
  \,,\qquad
  x
  \ 
  \mapsto
  \ 
  A_N(x)
  \,,
\end{align*}
where $A_N(x)$ is the only cell containing $x$. 

Next, we define the (empirical) basis functions vector
\begin{align}
  \label{def:basis}
  B\colon
  \R^d\times \R^{d\cdot N}
  \ 
  \to
  \ 
  \R
  \,,
  \qquad
  (x,(x_1,\ldots,x_N))
  \ 
  \mapsto
  \ 
  \frac
  {
    \left[
    \mathbf{1}
    _{
      A_N(x)
    }
    (x_k)
    \right]
    _{k\in \left\{
        1,\ldots,N
    \right\}}
  }
  {
    \sum_{j=1}^N
    \mathbf{1}
    _{
      A_N(x)
    }
    (x_j)
    }
  \,,
\end{align}
where we keep to the convention $"0/0=0"$.
We shall extend $B$ to depend on the random vectors
$X,X_1,\ldots,X_N$.
The next lemma studies the measurability of the extensions.
\begin{lemma}
  \label{lem:basis_meas}
  \quad
  \begin{enumerate}[label=(\roman*)]
\item
  $B(\cdot,(X_1,\ldots,X_N))(\omega)$ is 
  $\left(
    \mathcal{B}(\R^d),\mathcal{B}(\R^N)
  \right)$-measurable
  and
  constant on each cell 
  $A_N\in\mathcal{P}_N$
  for all $\omega\in\Omega$. 
\item
  $B(X,(X_1,\ldots,X_N))$ is $\left(
    \mathcal{A},\mathcal{B}(\R^N)
  \right)$-measurable. 
  \end{enumerate}
\end{lemma}

\begin{proof}
Consider
for $k\in \left\{
  1,\ldots,N
\right\}$
and $\omega\in\Omega$
the indicator function
\begin{align}
  \label{33342}
  \mathbf{1}
  _
  {A_N(X_k(\omega))}
  \colon \R^d\to \left\{
    0,1
  \right\}
  \,.
\end{align}
Since 
$
  {A_N(X_k(\omega))}
  \in\mathcal{B}(\R^d)
$
this is a 
  $\left(
    \mathcal{B}(\R^d),\mathcal{B}(\R)
  \right)$-measurable
  function.
  From the definition of $B$ \eqref{def:basis} it follows the first part of (i).
  Since the indicator function in \eqref{33342} is 1 if $
  x\in
  {A_N(X_k(\omega))}
  $
  and 0 else, it is also constant on each cell
  $A_N\in\mathcal{P}_N$.
  It follows (i).
  To prove (ii), note that
\begin{align*}
  \mathbf{1}
  _
  {A_N(X_k(\omega))}(X(\omega))
  \ 
  =
  \ 
  \mathbf{1}
  \bigcup_{i\in\mathbb{N}}
  \left\{
    X,X_k \in A_{N,i}
  \right\}
  (\omega)
  \qquad
  \text{for all}\ 
  \omega\in\Omega
  \,,
\end{align*}
and
$
  \bigcup_{i\in\mathbb{N}}
  \left\{
    X,X_k \in A_{N,i}
  \right\}
  \in\mathcal{A}
$ by the 
$
\left( 
  \mathcal{A},
  \mathcal{B}(\R^d)
\right)
$-
measurability of $X$ and $X_k$.
\end{proof}

\begin{lemma}
  \label{lem:basis_sum}
  Let $(x,x_1,\ldots,x_N)\in\R^{d(N+1)}$.
  \begin{enumerate}[label=(\roman*)]
    \item
      $
      \sum_{k=1}^{N} 
      B_k(x,x_1,\ldots,x_N)
      \in
      \left\{ 0,1 \right\}
      $. 
      In particular,
      $
        x_1,\ldots,x_N\notin A_N(x)
      $
      is equivalent to
      $
      \sum_{k=1}^{N} 
      B_k(x,x_1,\ldots,x_N)
      =0
      $
    \item
      $
      \sum_{k=1}^{N} 
      B_k(x_i,x_1,\ldots,x_N)
      \ 
      =
      \ 
      1
      \qquad
      \text{for all}\ 
      i\in \left\{ 1,\ldots,N \right\}
      $.
      \item
        $
        \norm
        {
      B(x,x_1,\ldots,x_N)
        }_2
        \ 
        \le 1
        \ 
        $
  \end{enumerate}
\end{lemma}
\begin{proof}
  Let $(x,x_1,\ldots,x_N)\in\R^{d(N+1)}$.
  We prove \textit{(i)}.
  Then \textit{(ii)} is a direct consequence of \textit{(i)}.
  If 
      $
        x_1,\ldots,x_N\notin A_N(x)
      $,
  then
  \begin{align*}
    B_k(
        x,x_1,\ldots,x_N
    )
    \ 
    =
    \ 
\frac
  {
    \mathbf{1}
    _{
      A_N(x)
    }
    (x_k)
  }
  {
    \sum_{j=1}^N
    \mathbf{1}
    _{
      A_N(x)
    }
    (x_j)
    }
    \ 
    =
    \ 
    0
    \qquad
    \text{for all}\ 
    k\in \left\{ 1,\ldots,N \right\}
    \,.
  \end{align*}
  On the other hand, if the sum is 0 it holds
  \begin{align*}
    \mathbf{1}
    _{
      A_N(x)
    }
    (x_k)
    \ 
  =
  \ 
  0
    \qquad
    \text{for all}\ 
    k\in \left\{ 1,\ldots,N \right\}
    \,.
  \end{align*}
  It follows the desired equivalence.
  If 
  \begin{align*}
    \mathbf{1}
    _{
      A_N(x)
    }
    (x_k)
    \ 
  =
  \ 
  1
    \qquad
    \text{for some}\ 
    k\in \left\{ 1,\ldots,N \right\}
    \,,
  \end{align*}
  then
  $
    \sum_{j=1}^N
    \mathbf{1}
    _{
      A_N(x)
    }
    (x_j)
    \ge 1
  $
  and thus "$0/0$" doesn't occure. 
  It follows
  \begin{align*}
      \sum_{k=1}^{N} 
      B_k(x,x_1,\ldots,x_N)
      \ 
      =
      \ 
\frac
  {
      \sum_{k=1}^{N} 
    \mathbf{1}
    _{
      A_N(x)
    }
    (x_k)
  }
  {
    \sum_{j=1}^N
    \mathbf{1}
    _{
      A_N(x)
    }
    (x_j)
    }
    \ 
    =
    \ 
    1
    \,.
  \end{align*}
  To prove \textit{(iii)}, note that by \textit{(i)}
  \begin{align*}
        \norm
        {
      B(x,x_1,\ldots,x_N)
        }_2^2
        \ 
        =
        \ 
      \sum_{k=1}^{N} 
      B_k(x,x_1,\ldots,x_N)^2
        \ 
      \le
        \ 
      \sum_{k=1}^{N} 
      B_k(x,x_1,\ldots,x_N)
        \ 
      \le
        \ 
      1
      \,.
  \end{align*}
\end{proof}
\begin{lemma}
  \label{lem:basis_approx_f}
  Let $(x,x_1,\ldots,x_N)\in\R^{d(N+1)}$.
  For all continuous functions $f\colon \R^d\to \R$ it holds
 \begin{align*}
   \begin{split}
   &
   \left|
  \sum_{k=1}^{N}
    B_k(x_i,x_1,\ldots,x_N)\cdot 
    f(x_k)
    -
    f(x_i)
   \right|
   \ 
   \le
   \ 
   \omega
   \left(
    f,h_N^d
   \right)
   \qquad
   \text{for all}\ 
   i\in \left\{ 1,\ldots,N \right\}
   \,,
   \end{split}
 \end{align*}
 where $\omega(f,\cdot)$ is the modulus of continuity of $f$. 
\end{lemma}
\begin{proof}
  It follows from Lemma~\ref{lem:basis_sum}\textit{(ii)}
  \begin{align*}
   \begin{split}
   &
   \left|
  \sum_{k=1}^{N}
    B_k(x_i,x_1,\ldots,x_N)\cdot 
    f(x_k)
    -
    f(x_i)
   \right|
   \\
   &
   \ 
   \le
   \ 
   \left|
  \sum_{k=1}^{N}
    B_k(x_i,x_1,\ldots,x_N)
    \left(
    f(x_k)
    -
    f(x_i)
    \right)
   \right|
   \\
   &
   \ 
   \le
   \ 
  \sum_{k=1}^{N}
    B_k(x_i,x_1,\ldots,x_N)
    \cdot
    \mathbf{1}\left\{
      x_k\in A_N(x_i)
    \right\}
    \left|
    f(x_k)
    -
    f(x_i)
    \right|
   \\
   &
   \ 
   \le
   \ 
   \omega
   \left(
    f,h_N^d
   \right)
   \,.
   \end{split}
 \end{align*}
\end{proof}
\begin{lemma}
  Let $(x,x_1,\ldots,x_N)\in\R^{d(N+1)}$.
  It holds
  for $N\to\infty$
  \begin{enumerate}[label=(\roman*)]
      \item
      \begin{align*}
        \frac
        {1}
        {N}
        \sum_{i,k=1}^{N}
            \left|
        B_k(x_i,x_1,\ldots,x_N)
        \cdot
            \varphi
            \left(
              \frac
              {1}
              {\pi(x_k)}
            \right)
            \ 
            -
            \ 
            \varphi
            \left(
              \frac
              {1}
              {\pi(x_i)}
            \right)
            \right|
            \ 
            \to
            \ 
            0
            \,,
          \end{align*}
\item
      \begin{align*}
        \sqrt{N}
        \sup_{z\in\R}
        \max_{i\in \left\{ 1,\ldots,N \right\}}
        \sum_{k=1}^{N}
            \left|
        B_k(x_i,x_1,\ldots,x_N)
        \cdot
        F_{Y(1)}(z|x_k)
            \ 
            -
            \ 
        F_{Y(1)}(z|x_i)
            \right|
            \ 
            \to
            \ 
            0
            \,.
      \end{align*}
\end{enumerate}
\end{lemma}
\begin{proof}
  By Lemma~\ref{lem:basis_approx_f} it follows
      \begin{align*}
        \frac
        {1}
        {N}
        \sum_{i,k=1}^{N}
            \left|
        B_k(x,x_1,\ldots,x_N)
        \cdot
            \varphi^{'}
            \left(
              \frac
              {1}
              {\pi(x_k)}
            \right)
            \ 
            -
            \ 
            \varphi^{'}
            \left(
              \frac
              {1}
              {\pi(x_i)}
            \right)
            \right|
            \ 
            \le
            \ 
   \omega
   \left(
     \varphi^{'},h_N^d
   \right)
            \ 
            \to
            \ 
            0
          \end{align*}
          for $N\to\infty$.
          Likewise
\begin{align*}
  &
        \sqrt{N}
        \sup_{z\in\R}
        \max_{i\in \left\{ 1,\ldots,N \right\}}
        \sum_{k=1}^{N}
            \left|
        B_k(x_i,x_1,\ldots,x_N)
        \cdot
        F_{Y(1)}(z|x_k)
            \ 
            -
            \ 
        F_{Y(1)}(z|x_i)
          \right|
             \\
            &
            \ 
            \le
            \ 
            \sqrt{N}
            \sup_{z\in\R}
            \omega
            \left(
        F_{Y(1)}(z|\cdot)
        ,
        h_N^d
            \right)
            \ 
            \to
            \ 
            0
            \qquad
            \text{for}
            \ 
            N\to\infty
            \,.
\end{align*}
        \end{proof}

\section{Pseudo Weights Process}
Based on Theorem~\ref{dual_solution_th}
we want to use the dual (pseudo) solution $s_N$ to construct weights.
To this end, we define the (empirical) weights function
\begin{align*}
 w\ \colon\
 &
 \left( 
  \R^d\times \R^{d\cdot N}
 \right)
  \times
  \left( 
\R^N_{\ge 0}\times \R\times \R^N
  \right)
  \to
  \R^N
  \\
 &
  \left( 
  (x,x_1,\ldots,x_N),(\rho,\lambda_0,\lambda)
  \right)
  \ 
  \mapsto
  \ 
  \left[ 
  (\varphi^{'})^{-1}
  \left( 
    \rho_i
    +
    \lambda_0
    +
    \inner
    {B(x,x_1,\ldots,x_N)}
    {\lambda}
  \right)
\right]_{i\in \left\{ 1,\ldots,N \right\}}
\,.
\end{align*}
\begin{definition}
  Let $s_N$ be the (pseudo) solution of Lemma~\ref{lem:pseud_sol}.
  We define the (pseudo) weights process 
  $\left\{ w^\dagger(x) | x\in\R^d\right\}$
  by
  \begin{align*}
    w^\dagger(x) 
    \ 
    :=
    \ 
    w(x,X_1,\ldots,X_N,s_N)
    \qquad
    \text{for all}\ 
    x\in\R^d
    \,.
  \end{align*}
\end{definition}
\begin{proof}
  \begin{align*}
    \left| w^\dagger(x) \right|
    &
    \ 
    \le
    \ 
    \omega
    \left( 
      (
      \varphi^{'}
      )^{-1},
      \norm{s_N-\lambda^*}_2
    \right)
    \ 
    +
    \ 
    \left| 
    w(x,X_1,\ldots,X_N,\lambda^*)
    \right|
    \\
    &
    \ 
    \le
    \ 
    \omega
    \left( 
      (
      \varphi^{'}
      )^{-1}
      ,
      1
    \right)
    \ 
    +
    \ 
      (
      \varphi^{'}
      )^{-1}
      \left( 
        \sum_{k=1}^{N}
        B_k(x,X_1,\ldots,X_N)
        \cdot
            \varphi
            \left(
              \frac
              {1}
              {\pi(X_k)}
            \right)
      \right)
      \\
    &
    \ 
      \le
    \ 
    \omega
    \left( 
      (
      \varphi^{'}
      )^{-1}
      ,
      1
    \right)
    \\
    &
    \qquad 
    +
    \ 
    \left( 
 \omega
    \left( 
      (
      \varphi^{'}
      )^{-1}
      ,
      \omega
      \left( 
      \varphi^{'}
      ,
      h_N^d
      \right)
    \right)
    +
              \frac
              {1}
              {\pi(x)}
    \right)
    \cdot
    \mathbf{1}\bigcup_{k=1}^N\left\{ X_k=x\right\}
    \\
    &\qquad
    +
    \ 
      (
      \varphi^{'}
      )^{-1}
      (0)
      \cdot
    \mathbf{1}\bigcap_{k=1}^N\left\{ X_k\neq x\right\}
     \end{align*}
\end{proof}
\begin{proof}
  Since
  \begin{align*}
  (a+b)^r\
  \le \ 
  2^r\cdot (1+a)^r\cdot b^r
  \qquad
  \text{for all}\ 
  \left( a,b,r \right)
  \in
  \R_{\ge 0}\times\R_{\ge 1}\times [1,\infty)
  \end{align*}
 it holds
 \begin{align*}
   \left| w^\dagger(x) \right|
   \ 
   \lesssim
   \ 
   \left( 
     \frac{1}{\pi(x)}
   \right)^r
   \qquad
   \text{for all}\ 
   (x,r)\in\R^d\times [1,\infty)
   \,.
 \end{align*}
 Thus, if 
 $1/\pi(X)\in L^r(\P)$
 for some $r\in [1,\infty]$,
 it holds
 $w^\dagger(X)\in L^r(\P)$.
\end{proof}
\begin{lemma}
  \label{w.Z=0}
  Let 
 $1/\pi(X)\in L^r(\P)$
 for some $r\in [1,\infty]$.
 Furthermore, let 
 $Z\in L^{r^{'}}(\P)$,
 where
 $1/r + 1/r^{'}=1$,
  be a random variable that is independent of $D_N=(T_i,X_i)_{i\in \left\{
    1,\ldots,N
  \right\}}$ 
  with
  $
\E
\left[
  Z
  \,
  |
  \, 
  X
\right]
= 0
  $
  almost surely.
  It holds
  \begin{gather*}
  \E
  \left[
    w_i^\dagger(X)
  \cdot Z
  \right]
  =0
  \qquad
  \text{for all}
  \ 
  i\in \left\{ 1,\ldots,N \right\}
  \,.
  \end{gather*}
\end{lemma}
\begin{proof}
  We write
  \begin{gather*}
    w_i^\dagger(X)
  \ 
  =
  \ 
  w^\dagger(X)
  \end{gather*}
  and ignore the index $i$.
  By Lemma~\ref{weights_l_inf} and 
  Hölder's inequality it holds
  \begin{gather}
    \label{9876}
    \norm{
  w^\dagger(X)\cdot Z
    }_{L^1(\P)}
    \ 
  \le
    \ 
  \norm{w^\dagger(X)}_{L^r(\P)}
  \norm{Z}_{L^{r^{'}}(\P)}
  \ 
  <
  \ 
  \infty
  \,.
  \end{gather}
  By 
  \eqref{9876},
  $Z\perp D_N$
  and
  $
\E
\left[
  Z
  \,
  |
  \, 
  X
\right]
= 0
  $
  almost surely
  it holds 
  \begin{align*}
    \E
  \left[
  w^\dagger(X)
  \cdot
  Z
  \,
  |
  \,
  D_N,X
  \right]
  &
  \ 
  =
  \ 
  w^\dagger(X)
  \cdot
  \E
  \left[
  Z
  \,
  |
  \,
  D_N,X
  \right]
  \\
  &
  \ 
  =
  \ 
  w^\dagger(X)
  \cdot
  \E
  \left[
  Z
  \,
  |
  \,
  X
  \right]
  \
  =
  \ 
  0
  \qquad
  \text{almost surely.}
  \end{align*}
  Note, that $w^\dagger(X)$ is 
  $
  \left(
  \sigma(D_N,X),\mathcal{B}(\R)
  \right)
  $-measurable.  
  Thus
  \begin{gather*}
    \E
    \left[
  w^\dagger(X)
  \cdot
  Z
  \,
    \right]
    \ 
    =
    \ 
    \E
    \left[
 \E
  \left[
  w^\dagger(X)
  \cdot
  Z
  \,
  |
  \,
  D_N,X
  \right]
    \right]
    \ 
    =
    \ 
    0
    \,.
     \end{gather*}
\end{proof}

\begin{lemma}
  The arithmetic mean
  \begin{align*}
    \frac{1}{N}
    \sum_{i=1}^{N} 
    T_i
    \left| 
    w^\dagger(X_i)
    \right|
  \end{align*}
  is bounded above by a random variable that converges in probability to a constant.
\end{lemma}
\begin{proof}
  It holds
  \begin{align*}
    \frac{1}{N}
    \sum_{i=1}^{N} 
    T_i
    \left| 
    w^\dagger(X_i)
    \right|
    \ 
    \lesssim
    \ 
    \frac{1}{N}
    \sum_{i=1}^{N} 
    \frac{T_i}{\pi(X_i)}
    \ 
    \overset{\P}{\to}
    \ 
    1
    \qquad
    \text{for}\ 
    N \to \infty
    \,.
  \end{align*}
\end{proof}


\chapter{Consistency of the Weights Process}
  The goal of this section is to prove consistency of the optimal solutions to Problem~\ref{bw:1:primal}, that is, of the weights 
$(w^\dagger_i)_{i\le n}$,
for the inverse propensitiy score (see Theorem).
We first show that asymptotically there exists an optimal solution 
$(\rho^\dagger,\lambda^\dagger,\lambda_0^\dagger)$
to Problem~\ref{dual} that converges to the oracle parameter
$
\lambda^*_{\varphi^{'}\circ\,1/\pi}
$
(see Assumption~\ref{asu:basis}.(ii))
in probability (see Theorem).
Dual and primal optimal solutions
$(\rho^\dagger,\lambda^\dagger,\lambda_0^\dagger)$
and
$(w^\dagger_i)_{i\le n}$
are linked through Theorem~\ref{dual_solution_th}.
Thus consistency of the primal solutions follows readily from that of the dual solutions.

\subsection{Consistency of the Dual Solution}
  We get a grip by the following lemma.
The high-level idea is that the existence of the optimal dual solution and its proximity to the oracle parameter can be analysed by the objective function.
\begin{lemma}
  Let $m$,$N\in\mathbb{N}$ and let 
  $
  g \,:\, \R^N_{\ge 0}\times\R^m \to \overline{\R}
  $ 
  be a continuous and proper convex function.
  Consider 
  \begin{gather*}
    \tilde{S}(\varepsilon)
    :=
    \left\{ 
      (\Delta_\rho,\Delta)
      \in
      \R^N_{\ge 0}\times\R^m
      \ 
      \colon
      \ 
      \norm{
      (\Delta_\rho,\Delta)
      }_2
      =
      \varepsilon
    \right\}
    \qquad
    \text{for}
    \ 
    \varepsilon>0
    \,.
  \end{gather*}
Then 
  for all $y \in \R^m$ and $\varepsilon>0$ 
    \begin{gather}
      \label{696}
      \inf 
      \left\{ 
        g(\Delta_\rho,y+\Delta)
        -
        g(0,y)
      \ 
        \colon
      \ 
      (\Delta_\rho,\Delta)
      \in
    \tilde{S}(\varepsilon)
      \right\}
      \ 
      \ge
      \ 
      0
    \end{gather}
    implies
    the existence of  
    a global minimum
    \begin{gather*}
    (y^*_\rho,y^*)
    \in \,\R^N_{\ge 0}\times\R^m
    \quad
    \text{of $g$ such that}\quad
      \norm{
    (y^*_\rho,y^*)
      - (0,y)}_2 \le \varepsilon
      \,.
    \end{gather*}
\end{lemma}
\begin{proof}
  We start by defining the convex set
  \begin{gather*}
    \tilde{B}(\varepsilon)
    \ 
    :=
    \ 
    \left\{ 
      (\Delta_\rho,\Delta)
      \in
      \R^N_{\ge 0}\times\R^m
      \ 
      \colon
      \ 
      \norm{
      (\Delta_\rho,\Delta)
      }_2
      \le
      \varepsilon
    \right\}
    \qquad
    \text{for}
    \ 
    \varepsilon>0
    \,.
  \end{gather*}
  Then the translation 
  $
  (0,y)
  +
    \tilde{B}(\varepsilon)
  $
  is also convex.
  Assume towards a contradiction that it holds \eqref{696}
  and that there exists 
  \begin{gather}
    \label{698}
  (x^*_\rho,x^*)
  \ 
\in
  \ 
\R^N_{\ge 0}\times\R^m\setminus 
\left(
  (0,y)
  +
    \tilde{B}(\varepsilon)
\right)
\quad
\text{such that}\quad
g
  (x^*_\rho,x^*)
  \ 
  <
  \ 
  g(0,y)
  \,.
  \end{gather}
  Since 
  $
  (0,y)
  +
    \tilde{B}(\varepsilon)
  $
  is bounded, the line segment between 
  $
  (x^*_\rho,x^*)
  $
  and
  $
  (0,y)
  $
  crosses its boundary. The boundary consists of two disjoint sets
  \begin{gather*}
    S_0(\varepsilon)
    :=
    \left\{ 
      (0,y+\Delta)
      \ 
      \colon
      \ 
      \Delta\in\R^m\ 
      \text{and}\ 
      \norm{\Delta}_2<\varepsilon
    \right\}
    \qquad
    \text{and}
    \qquad
    \tilde{S}(\varepsilon)
    \,.
  \end{gather*}
  Clearly, if the line segment does not cross $\tilde{S}(\varepsilon)$ it leaves $\R^N_{\ge 0}\times\R^m$.
  But this is not possible.
  Thus, there exists $(\Delta_\rho,\Delta)\in \tilde{S}(\varepsilon)$ and $\theta\in(0,1)$ such that 
  \begin{gather}
    \label{697}
    \theta 
    \cdot
  (x^*_\rho,x^*)
  \ 
  +
  \ 
  (
  1
  -
\theta
  )
  \cdot
  (0,y)
  \ 
  =
  \ 
  (\Delta_\rho,y+\Delta)
  \,.
  \end{gather}
  It follows
 \begin{align*}
      \begin{split}
      g(0,y)
      \ 
      \le
      \ 
      g
  (\Delta_\rho,y+\Delta)
&
      \ 
      =
      \ 
      g
      \left( 
    \theta 
    \cdot
  (x^*_\rho,x^*)
  \ 
  +
  \ 
  (
  1
  -
\theta
  )
  \cdot
  (0,y)
      \right)
      \\
&
      \ 
      \le
      \ 
    \theta 
    \cdot
      g
  (x^*_\rho,x^*)
  \ 
  +
  \ 
  (
  1
  -
\theta
  )
  \cdot
  g
  (0,y)
  \ 
      <
  \ 
      g(0,y)
  \,    ,
      \end{split}
    \end{align*}
    which is a contradiction.
    The first inequality is due to \eqref{696}, the equality is due to \eqref{697}, the second inequality is due to the convexity of $g$,
    and the strict inequality is due to assumption \eqref{698}.
    Thus, all values outside 
    $(0,y)+\tilde{B}(\varepsilon)$
    are greater or equal $(0,y)$.
    Since 
    $(0,y)+\tilde{B}(\varepsilon)$
    is also compact, the continuous function $g$ has a local minimum
    \begin{gather*}
      (y^*_\rho,y^*)\in
    (0,y)+\tilde{B}(\varepsilon)
    \,.
    \end{gather*}
    But then it holds
    \begin{gather*}
      g
      (y^*_\rho,y^*)
      \ 
      \le
      \ 
      g(0,y)
      \ 
      \le
      \ 
      g
      (x_\rho,x)
      \qquad
      \text{for all}
      \qquad 
      (x_\rho,x)
      \in
\R^N_{\ge 0}\times\R^m\setminus 
\left(
  (0,y)
  +
    \tilde{B}(\varepsilon)
\right)
    \end{gather*}
    and
    \begin{gather*}
      g
      (y^*_\rho,y^*)
      \ 
      \le
      \ 
      g
      (z_\rho,z)
      \qquad
      \text{for all}
      \qquad 
      (z_\rho,z)
      \in
  (0,y)
  +
    \tilde{B}(\varepsilon)
    \,.
    \end{gather*}
    Thus,
    $
      (y^*_\rho,y^*)
    $
    is also a global minimum in
    $
\R^N_{\ge 0}\times\R^m
    $.
Since
    $
      (y^*_\rho,y^*)
      \in
  (0,y)
  +
    \tilde{B}(\varepsilon)
    $
    there exists
    $
    (\Delta_\rho,\Delta)\in
    \tilde{B}(\varepsilon)
    $
    such that 
    \begin{gather*}
      (y^*_\rho,y^*)
      =
      (\Delta_\rho,y+\Delta)
      \qquad
      \text{for some}\ 
      (\Delta_\rho,\Delta)
      \in
      \tilde{B}(\varepsilon)
      \,.
    \end{gather*}
  Thus
  \begin{gather*}
    \norm{
      (y^*_\rho,y^*)
      -
      (0,y)
    }_2
    =
    \norm{
      (\Delta_\rho,\Delta)
    }_2
    \le \varepsilon
    \,.
  \end{gather*}
  This finish the proof.
\end{proof}
\begin{remark}
  I learned of the high-level idea from \cite[page 22]{Wang2019}.
  I adapted it to the needs of the subsequent analysis and provided the details by myself.
  Note, that the hint in \cite[page 22]{Wang2019}
  uses strict inequality in the statement.
  I found out that this can be relaxed.
  It is crucial to my further approach that this holds (only) with inequality, because I use measurability properties to obtain convergence.
\end{remark}


To proceed we recall the objective function of Problem~\ref{dual}, that is,
\begin{align*}
  G\colon
  \R^{\#\mathfrak{B}}
  \times
  \R
  \times
  \R^n_{\ge 0}
  &
  \ 
  \to
  \ 
  \overline{\R}
  \\
  \left( 
    \lambda,
    \lambda_0,
    \rho
  \right)
  &
  \ 
  \mapsto
  \ 
    \frac{1}{N}
\sum_{i=1} 
  ^n
  \varphi^*
  \!
  \left( 
    \rho_i
    +
\lambda_0
+
\inner
{B(X_i)}
{
\lambda
}
  \right)
  \ 
  -
  \ 
  \left( 
\lambda_0
+
\inner
{B(X_i)}
{
\lambda
}
  \right)
  \\
  &
  \qquad 
+
\ 
\inner
{\delta}
{
  |\lambda|
}
\end{align*}
Note, that $G$ is continuous by the continuity of $\varphi^*$.
Next we define for $\varepsilon>0$ an auxiliary function:
   \begin{align*}
     \underline{
     \Delta G^*
     _\varepsilon
     }
     \colon
     &
     \left( \Omega,\mathcal{A},\P \right)
     \ 
     \to
     \ 
     \left( 
       \R,\mathcal{B}(\overline{\R})
     \right)
     \\
     &
     \omega
     \ 
     \mapsto
     \ 
   \inf
   \left\{ 
 G
   \left( 
\lambda^*_{\varphi^{'}\circ\,1/\pi}
(\omega)
+\Delta
,
\Delta_0
,
\Delta_\rho
   \right)
   -
   G
   \left(
\lambda^*_{\varphi^{'}\circ\,1/\pi}
(\omega)
,0,0
   \right)
   \Big\lvert
   \norm{
\Delta
,
\Delta_0
,
\Delta_\rho
   }_2
   =\varepsilon
   \right\}
   \end{align*}
\begin{lemma}
  \label{G_meas}
  For all $\varepsilon>0$ the function
  $
     \underline{
     \Delta G^*
     _\varepsilon
     }
  $
  is measurable.
\end{lemma}
\begin{proof}
  Since the function
  \begin{align*}
    \R^{\#\mathfrak{B}}
    \times
    \left( 
    \R^{\#\mathfrak{B}}
    \times
    \R
    \times
    \R^n_{\ge 0}
    \right)
    \ni
    \left( 
    \lambda,(\Delta,\Delta_0,\Delta_\rho)
    \right)
    \ 
    \mapsto
    \ 
 G
   \left( 
     \lambda
+\Delta
,
\Delta_0
,
\Delta_\rho
   \right)
   -
   G
   \left(
     \lambda
,0,0
   \right)
  \end{align*}
  is continuous and 
  $
  \left\{
   \norm{
\Delta
,
\Delta_0
,
\Delta_\rho
   }_2
   =\varepsilon
  \right\}
  $
  is compact in 
  $
    \R^{\#\mathfrak{B}}
    \times
    \R
    \times
    \R^n_{\ge 0}
    $
    for all $\varepsilon>0$,
  the function
\begin{align*}
    \R^{\#\mathfrak{B}}
    \ni
    \lambda
    \ 
    \mapsto
    \ 
    \inf
    \left\{ 
  G
   \left( 
     \lambda
+\Delta
,
\Delta_0
,
\Delta_\rho
   \right)
   -
   G
   \left(
     \lambda
,0,0
   \right)
   \lvert
   \norm{
\Delta
,
\Delta_0
,
\Delta_\rho
   }_2
   =\varepsilon
    \right\}
   \end{align*}
is continuous for all $\varepsilon>0$.
Since 
$
\lambda^*_{\varphi^{'}\circ\,1/\pi}
$
is measurable by assumption it follows the statement.
\end{proof}
 \begin{lemma}
   \label{bw:cd:lem2}
   Under conditions it holds
   for all $\varepsilon>0$
\begin{gather}
   \P
   \left[ 
     \underline{
     \Delta G^*
     _\varepsilon
     }
     \ge 
     0
   \right]
   \ 
   \to
   \ 
   1
   \qquad
   \text{for}
   \ 
   N\to\infty
   \,.
\end{gather}
 \end{lemma}
 \begin{proof}
   Let $\varepsilon>0$.
   We will show
\begin{gather}
   \P
   \left[ 
     \underline{
     \Delta G^*
     _\varepsilon
     }
     \ge 
     -\tilde{\varepsilon}
   \right]
   \ 
   \to
   \ 
   1
   \qquad
   \ 
   \text{for}
   \ 
   N\to\infty
   \ 
   \text{for all}\ 
   \tilde{\varepsilon}>0
   \,.
\end{gather}
Then the result follows from the measurability of 
$
     \underline{
     \Delta G^*
     _\varepsilon
     }
$
(see Lemma~\ref{G_meas}).
To this end, note, that
\begin{gather*}
  G(\lambda,\lambda_0,\rho)
  \ 
  =
  \ 
  g(\lambda,\lambda_0,\rho)
  \ 
  +
  \ 
  \inner{\delta}{|\lambda|}
  \qquad
  \text{for all}\ 
  (\lambda,\lambda_0,\rho)
  \in
  \R^{\#\mathfrak{B}}
  \times
  \R
  \times
  \R^n_{\ge 0}
  \,,
\end{gather*}
with
\begin{gather*}
  g
  \ 
  :=
  \ 
  (\lambda,\lambda_0,\rho)
  \ 
  \mapsto
  \ 
     \frac{1}{N}
     \left( 
\sum_{i=1} 
  ^n
  \varphi^*
  \!
  \left( 
    \rho_i
    +
\lambda_0
+
\inner
{B(X_i)}
{
\lambda
}
  \right)
  \ 
  -
\ 
\sum_{i=1}^{N} 
\lambda_0
+
\inner
{B(X_i)}
{
\lambda
}
     \right)
  \,.
\end{gather*}
Since we assume $\varphi^*$ to be continuously differentiable (it is always convex),
$g$ is a continuously differentiable convex function with gradient
\begin{align*}
  &
  (\lambda,\lambda_0,\rho)
  \\
  &
  \ 
  \mapsto
  \ 
     \frac{1}{N}
     \left( 
\sum_{i=1} 
  ^n
  (
  \varphi^{'}
  )
  ^{-1}
  \!
  \left( 
    \rho_i
    +
\lambda_0
+
\inner
{B(X_i)}
{
\lambda
}
  \right)
  \left[ 
    B(X_i)^\top,1,e_i^\top
  \right]^\top
  \ 
  -
  \ 
  \sum_{i=1}^{N} 
  \left[ 
    B(X_i)^\top,1,0_n^\top
  \right]^\top
     \right)
  \,.
\end{align*}
Thus, by \eqref{cv:primer:mvthe},
it holds
\begin{align*}
  &
  G
   \left( 
\lambda^*_{\varphi^{'}\circ\,1/\pi}
+\Delta
,
\Delta_0
,
\Delta_\rho
   \right)
   \ 
   -
   \ 
   G
   \left(
\lambda^*_{\varphi^{'}\circ\,1/\pi}
,0,0
   \right)
   \\
   &
   \ 
   \ge
   \ 
   \frac{1}{N}
  \left( 
\sum_{i=1} 
  ^n
  (
  \varphi^{'}
  )
  ^{-1}
  \!
  \left( 
\inner
{B(X_i)}
{
\lambda^*_{\varphi^{'}\circ\,1/\pi}
}
  \right)
  \left[ 
    B(X_i)^\top,1,e_i^\top
  \right]
  \ 
  -
  \ 
  \sum_{i=1}^{N} 
  \left[ 
    B(X_i)^\top,1,0_n^\top
  \right]
  \right)
  \begin{bmatrix}
    \Delta
    \\
    \Delta_0
    \\
    \Delta_\rho
  \end{bmatrix}
    \\
  &
  \qquad
  +
  \ 
  \inner{\delta}
  {|
\lambda^*_{\varphi^{'}\circ\,1/\pi}
+
\Delta
  |
  -
  |
\lambda^*_{\varphi^{'}\circ\,1/\pi}
  |
}
\\
   &
   \ 
   \ge
   \ 
   \frac{1}{N}
\sum_{i=1} 
  ^n
  \left( 
  (
  \varphi^{'}
  )
  ^{-1}
  \!
  \left( 
\inner
{B(X_i)}
{
\lambda^*_{\varphi^{'}\circ\,1/\pi}
}
  \right)
  \ 
  -
  \ 
  1
  \right)
  \left[ 
    B(X_i)^\top,1,e_i^\top
  \right]
  \cdot
  \begin{bmatrix}
    \Delta
    \\
    \Delta_0
    \\
    \Delta_\rho
  \end{bmatrix}
  +
  \inner{e_i}{\Delta_\rho}
  \\
  &
  \qquad
  +
  \ 
  \inner{\delta}
  {|
\lambda^*_{\varphi^{'}\circ\,1/\pi}
+
\Delta
  |
  -
  |
\lambda^*_{\varphi^{'}\circ\,1/\pi}
  |
}
\end{align*}
 \end{proof}



\begin{theorem}
  \label{th:cons_dual}
  With probability going to 1 Problem~\ref{dual} is feasible.
  Furthermore, if the solution 
$
(\rho^\dagger,\lambda^\dagger,\lambda_0^\dagger)
$
exists,
it
converges in probability to 
$
(0_N,0,\lambda^*)
$.
\end{theorem}
\begin{proof}
  By Lemma~\ref{lem:link_conv_p} and Lemma~\ref{lem:conv_dG}
  it holds for all $\varepsilon>0$
  \begin{align*}
    &
    \P
    \left[ 
      \text{Problem~\ref{dual} is feasible and }
      \norm{
(\rho^\dagger,\lambda^\dagger,\lambda_0^\dagger)
-
(0_N,0,\lambda^*)
      }_2
      \le \varepsilon
    \right]
    \\
    &
    \ 
    \ge
    \ 
    \P
    \left[ \underline{\Delta G^*_\varepsilon}\ge 0 \right]
    \ 
    \to
    \ 
    1
  \end{align*}
\end{proof}
\begin{corollary}
  If Problem~\ref{dual} is feasible it holds
  $\norm{(\rho^*,\lambda_0^*)}_2\overset{\P}{\to}0$
  and
  $\norm{\lambda^*-\lambda^\dagger}\overset{\P}{\to}0$.
\end{corollary}
\begin{proof}
  This follows from Theorem~\ref{th:cons_dual}
  and Slutzky's Theorem.
\end{proof}






\chapter{Convergence of the Weighted Mean}
  \section{Tools}
  For the subsequent analysis we need the theory of empirical processes.
For an introduction to empirical processes see \cite[§19]{Vaart2000}. For a thorough treatment see \cite[§2]{vaart2013}. 
\subsection{Empirical Processes - Definition}
  Let 
$
  \left( 
    \Omega,
    \mathcal{A},
    \P
  \right)
$
be a probability space,
$
  \left( 
    \mathcal{Z},
    \Sigma
  \right)
$
a measurable space, and 
\begin{gather*}
  \xi_1,\ldots,\xi_N
  :
  \left( 
    \Omega,
    \mathcal{A},
    \P
  \right)
  \to
  \left( 
    \mathcal{Z},
    \Sigma
  \right)
  \quad
  \text{independent and identically-distributed
  }
\end{gather*}
random variables
with probability distribution $\P_{\!\xi}$.
Let $\mathcal{F}$ be a class of measurable functions 
$
  f:
  \left( 
    \mathcal{Z},
    \Sigma
  \right)
    \to
  \left( 
    \R,
    \mathcal{B}(\R)
  \right)
$, where
$
    \mathcal{B}(\R)
$
is the Borel-$\sigma$-algebra on $\R$.
Then $\mathcal{F}$
induces a stochastic process by
\begin{gather}
  f
  \ 
  \mapsto
  \ 
  \G_N f 
  \ 
  :=
  \ 
  \frac{1}{\sqrt{n}}
  \sum_{i=1}^{N} 
  \left(
    f(\xi_i)
    -
    \E_\xi[f]
  \right)
  \,,
\end{gather}
where
$
    \E_\xi[f]
    :=
    \int_\mathcal{Z}
    f
    \,
    d\P_\xi
$.
We call
$\G_N$ the  \textbf{empirical process} indexed by $\mathcal{F}$.
The purpose of this construction is, to study the behaviour of a centered, scaled arithmetic mean uniformly over $\mathcal{F}$.
To this end, we define the (random) norm
\begin{gather}
  \norm{\G_n}_\mathcal{F}
  :=
  \sup_
        { f \in \mathcal{F}}
        \left|
          \G_N f
        \right|
        .
\end{gather}
We stress that 
$
  \norm{\G_n}_\mathcal{F}
$
often ceases to be measurable, even in simple situations~\cite[page 3]{vaart2013}.
To deal with this, we introduce the notion of \textbf{outer expectation} $\E^*$ (see \cite[page~6]{vaart2013})
\begin{gather*}
  \E^*[Z]
  \ 
  :=
  \ 
    \inf
  \left\{ 
    \E[U]
  \ 
  \lvert
  \ 
    U\ge Z,
    \ 
    U:
  \left( 
    \Omega,
    \mathcal{A},
    \P
  \right)
  \to 
  \left( 
    \overline{\R},
    \mathcal{B}(\overline{\R})
  \right)
  \text{measurable and}
  \ 
  \E[U]<\infty
  \right\}
  \,.
\end{gather*}
In our application the technical difficulties halt at this point, because we only consider $Z$ with $\E^*[Z]<\infty$. Then there exists a smallest measurable function $Z^*$ dominating $Z$ with
$\E^*[Z]=\E[Z^*]$ (see \cite[Lemma~1.2.1]{vaart2013}).

An \textbf{envelope function} $F$ of a class $\mathcal{F}$ satisfies 
\begin{align*}
|f(z)|
\ 
\le
\ 
F(z)< \infty 
\qquad
\text{for all}\ 
f\in\mathcal{F}
\ \text{and all}\ 
z\in\mathcal{Z}
\,.
\end{align*}


\subsection{Bracketing Numbers and Integral}
  To control empirical processes - apart from strong theorems - we need the notion of bracketing number and integral (see \cite[page 270]{Vaart2000}). 
Given two functions $\underline{f}\le \overline{f}$,
\begin{gather*}
  \text{
the bracket
  }\quad
[\underline{f},\overline{f}]
\quad 
\text{
is the set of all functions $f$ with 
}\quad 
\underline{f}\ \le\ f \ \le\  \overline{f}
\,.
\end{gather*}
For $\varepsilon>0$
we define a
\begin{gather*}
  \text{
$(\varepsilon, L^{r}(\P))$ -bracket
to be a bracket
  }
  \quad
[\underline{f},\overline{f}]
\quad
\text{with}
\quad
\norm{\overline{f}-\underline{f}}_{ L^r(\P)}
\ <\  \varepsilon
\,.
\end{gather*}
The \textbf{
bracketing number
} 
$
N_{[\,]}(\varepsilon, \mathcal{F}, L^r(\P))
$
is 
the minimum number of 
$(\varepsilon, L^{r}(\P))$-brackets needed to cover $\mathcal{F}$.

For most classes $\mathcal{F}$ the bracketing number grows to infinity for $\varepsilon\to 0$.
To measure the speed of growth we introduce 
for $\delta>0$
the
\textbf{bracketing integral}
\begin{gather*}
     J
    _{[\,]}
    (
    \delta
    ,
    \mathcal{F}
    ,
    L_r(\P)
    )
    \ 
    =
    \ 
  \int_0^{\delta}
      \sqrt{
        \log 
      N_{[\,]}
\left( \varepsilon, \mathcal{F}_N, L^r(\P) \right)
    }
    \,
    d\varepsilon
    \,.
\end{gather*}

Next we give a technical lemma to 
bound the bracketing numbers of products of two function classes, that is,
\begin{gather*}
  \mathcal{F}\cdot \mathcal{G}
  \ 
  :=
  \ 
  \left\{ 
    f\cdot g
    \ 
    \colon
  \ 
    f\in\mathcal{F},
    g\in\mathcal{G}
  \right\}\,.
\end{gather*}
\begin{lemma}
  \label{lem_prod_br}
  Let
  $\mathcal{F}$ and $\mathcal{G}$ be two function classes 
  with envelope functions $F$ and $G$ satisfying
  $\norm{F}_\infty,\norm{G}_\infty\le 1$.
  For all $\varepsilon>0$ and all $r\in [1,\infty)$ it holds
  \begin{gather*}
    N_{[\,]}(2\varepsilon,\mathcal{F}\cdot\mathcal{G},\mathrm{L}_r(\P))
    \
    \le
    \ 
    N_{[\,]}(\varepsilon,\mathcal{F},\mathrm{L}_r(\P))
    \cdot
    N_{[\,]}(\varepsilon,\mathcal{G},\mathrm{L}_r(\P))
    \,.
  \end{gather*}
\end{lemma}
\begin{proof}
  The proof is simple. We omit the details.
%  Let $f\in\mathcal{F}$ and $g\in\mathcal{G}$.
%  We can choose two 
%  $(\varepsilon,L^r(\P))$
%  brackets
%  $[\underline{f},\overline{f}]$
%  and
%  $[\underline{g},\overline{g}]$
%  containing $f$ and $g$ with 
%  $\norm{\underline{f}}_\infty,\norm{\overline{f}}_\infty\le\norm{F}_\infty\le 1$
%  and
%  $\norm{\underline{g}}_\infty,\norm{\overline{g}}_\infty\le\norm{G}_\infty\le 1$.
%  We the get an 
%  $(2\varepsilon,L^r(\P))$
%  $[\underline{h},\overline{h}]$
%  bracket, containing $f\cdot g$, by
\end{proof}
The following has the advantage of being both example (for the interested reader) and helpful for the subsequent analysis.

For $z\in\R$ we define the function
  \begin{align*}
    f_z
    \ 
    :
    \ 
      \left\{ 0,1 \right\}
      \times
      \mathcal{X}
      \times
      \mathcal{Y}
    &
    \ 
    \to
    \ 
    \R
    \\
      (t,x,y)
    &
      \ 
      \mapsto
      \ 
      t
      \left( 
        \mathbf{1}
        _{\left\{  y\,\le\,z \right\}}
        -
        F_{Y(1)}(z|x)
      \right)
      \,,
  \end{align*}
    Next we define the function classes
  \begin{gather}
    \label{F_g}
    \begin{split}
    \mathcal{F}
    &
    \ 
    :=
    \ 
    \left\{ 
      f_z
      \ 
      |
      \ 
      z\in\R\ 
    \right\}
    \\
    \mathcal{G}
    &
    \ 
    :=
    \ 
    \left\{ 
      \frac{f_z}{\pi(\cdot)}
      +
      F_{Y(1)}(z|\cdot)
      -
      F_{Y(1)}(z)
      \ 
      \colon
      \ 
      z\in\R\ 
    \right\}
    \,.
    \end{split}
  \end{gather}
Next, we provide bracketing numbers for these classes.
\begin{lemma}
  \label{aa:mean:l:br}
  The function class $\mathcal{F}$ and $\mathcal{G}$ defined in \eqref{F_g} are measurable.
  Furthermore, 
  \begin{gather*}
    N_{[\,]}
    (
    \varepsilon
    ,
    \mathcal{F}, L^2(\P))
    \ 
    \lesssim
    \ 
    \left( 
      \frac{1}{\varepsilon}
    \right)^2
    \qquad
    \text{for all}
    \ 
    \varepsilon>0
    \,.
  \end{gather*}
  If $1/\pi(X)\in L^2(\P)$, it also holds 
  \begin{gather*}
    N_{[\,]}
    (
    \varepsilon
    ,
    \mathcal{G}, L^2(\P))
    \ 
    \lesssim
    \ 
    \left( 
    \frac{
      1+
    \norm{1/\pi(X)}_{ L^2(\P)}
    }
    {\varepsilon}
    \right)^4
    \qquad
    \text{for all}
    \ 
    \varepsilon>0
    \,.
  \end{gather*}
%  Furthermore, 
%  consider the function class
%  \begin{gather*}
%    \mathcal{G}
%    :=
%    \left\{ 
%      f_{1/\pi}^z
%      +
%        F_{Y(1)}(z|\cdot)
%        -
%        F_{Y(1)}(z)
%      \ 
%      |
%      \ 
%      z\in\R
%    \right\}
%    \,.
%  \end{gather*}
%  If $1/\pi(X)\in  L^2(\P)$
%  it holds
%  \begin{gather}
%    N_{[\,]}(\varepsilon,\mathcal{G}, L^2(\P))
%    \le
%    ??
%    \qquad
%    \text{for all}
%    \ 
%    \varepsilon>0
%    \,.
%  \end{gather}
\end{lemma}
\begin{proof}
  As in \cite[Example~19.6]{Vaart2000}
  we choose for
  $\varepsilon>0$ and $m\in\mathbb{N}$
  \begin{gather*}
  -\infty=z_0\ <\ z_1\ <\ \cdots\ <\ z_{m-1}\ <\ z_m=\infty
  \,
  \end{gather*}
  such that
  \begin{gather}
    \label{size_z}
    \P
    \left[ 
      Y(1)\in \left[ z_{l-1},z_l \right]\,
    \right]
    \ 
    \le
    \ 
    \varepsilon
    \qquad
    \text{for all}\ 
    l\in \left\{ 1,\ldots,m \right\}
  \end{gather}
  and $m \le 2/\varepsilon$.
  Next, we define $m$ brackets by
\begin{align*}
  \overline{f_l}
  (t,x,y)
  &
  \ 
  :=
  \ 
      t
      \left( 
        \mathbf{1}
        _{\left\{  y\,\le\,z_{l} \right\}}
        -
        F_{Y(1)}(z_{l-1}|x)
      \right)
      \,,
      \\
  \underline{f_l}
  (t,x,y)
  &
  \ 
  :=
  \ 
      t
      \left( 
        \mathbf{1}
        _{\left\{  y\,\le\,z_{l-1} \right\}}
        -
        F_{Y(1)}(z_l|x)
      \right)
      \,,
\end{align*}
for $l\in \left\{ 1,\ldots,m \right\}$.
These brackets cover $\mathcal{F}$.
Indeed,
\begin{gather*}
  \text{for all}\ 
  z\in\R
  \ 
  \text{there exists} \ 
l\in \left\{ 1,\ldots,m \right\}
\qquad 
\text{such that}\qquad
z_{l-1}
\ 
\le
\ 
z
\ 
\le
\ 
z_l
\,.
\end{gather*}
By the monotonicity of 
$
        \mathbf{1}
      _{\left\{  y\,\le\,(\cdot) \right\}}
$
and
$
        F_{Y(1)}(\cdot|x)
$
and the non-negativity of $T$ it follows
\begin{gather*}
  \text{for all}\ 
  z\in\R
  \ 
  \text{there exists} \ 
l\in \left\{ 1,\ldots,m \right\}
\qquad 
\text{such that}\qquad
  \underline{f_l}
  \ 
  \le
  \ 
  f_z
  \ 
  \le
  \ 
  \overline{f_l}
  \,.
\end{gather*}
Thus, the $m$ brackets 
$
[
  \underline{f_l}
  ,
  \overline{f_l}
]
$
cover $\mathcal{F}$.

Let's calculate the size of the brackets.
It holds
\begin{align*}
  &
\E
\left[ 
      T
      \cdot
      \left( 
        \mathbf{1}
        _{\left\{  Y(T)\,\le\,z_{l} \right\}}
        -
        F_{Y(1)}(z_{l-1}|X)
        \ 
        -
        \ 
        \mathbf{1}
        _{\left\{  Y(T)\,\le\,z_{l-1} \right\}}
        +
        F_{Y(1)}(z_{l}|X)
      \right)
      \,
\right]
\\
  &
  \ 
=
  \ 
\E
\left[ 
      T
      \cdot
      \left( 
        \mathbf{1}
        _{\left\{
        Y(T)
        \,
        \in 
        \,
    [z_{l-1},z_l]
\right\}}
\ 
        +
\ 
        \P
        \left[ 
          Y(1)
          \in
    [z_{l-1},z_l]
        \,
    |
        \,
    X
        \right]
      \right)
      \,
\right]
\\
  &
  \ 
\le
  \ 
\E
\left[ 
  \,
  \pi(X)
  \cdot
        \P
        \left[ 
          Y(1)
          \in
    [z_{l-1},z_l]
          \,
    |
          \,
    X
        \right]
          \ 
\right]
\ 
+
\ 
\varepsilon
\\
  &
  \ 
\le
  \ 
2
\,
\varepsilon
\,.
\end{align*}
We used \eqref{size_z}, $0\le T,\pi(X)\le 1$ and Lemma~\ref{ps_weights_lemma}.
It follows
\begin{align*}
  &
  \norm{
    \left( 
  \overline{f_l}
-
  \underline{f_l}
    \right)
  (T,X,Y(T))
}_
{ L^2(\P)}
\\
&
\ 
\lesssim
\ 
\E
\left[ 
  \,
      T
      \cdot
      \left( 
        \mathbf{1}
        _{\left\{
        Y(T)
        \,
        \in 
        \,
    [z_{l-1},z_l]
\right\}}
\ 
        +
\ 
        \P
        \left[ 
          Y(1)
          \in
    [z_{l-1},z_l]
        \,
    |
        \,
    X
        \right]
      \right)
      \,
   \right]^{1/2}
\ 
\lesssim
\ 
\varepsilon^{1/2}
\,.
\end{align*}
Since $m\le 2/\varepsilon$ it holds
  \begin{align*}
    N_{[\,]}
    \left(
\varepsilon^{1/2}
    ,
    \,
    \mathcal{F}\,,\, L^2(\P)
    \right)
    &
    \ 
    \lesssim
    \ 
    \frac{1}{\varepsilon}
    \intertext{and thus}
    N_{[\,]}
    (
    \varepsilon
    ,
    \mathcal{F}, L^2(\P))
    &
    \ 
    \lesssim
    \ 
    \left( 
      \frac{1}{\varepsilon}
    \right)^2
    \,.
  \end{align*}
  Next, we look at $\mathcal{G}$. To this end, we define 
  $m$ brackets by
 \begin{align*}
    \overline{g_l}
    (t,x,y)
    \ 
    :=
    \ 
    \frac{t}{\pi(x)}
    \left( 
      \mathbf{1}{\left\{  y\,\le\,z_{l} \right\}}
      -
      F_{Y(1)}(z_{l-1}|x)
    \right)
    \ 
    +
    \ 
    F_{Y(1)}(z_{l}|x)
-
F_{Y(1)}(z_{l-1})
\,,
\\
    \underline{g_l}
    (t,x,y)
    \ 
    :=
    \ 
    \frac{t}{\pi(x)}
    \left( 
      \mathbf{1}{\left\{  y\,\le\,z_{l-1} \right\}}
      -
      F_{Y(1)}(z_l|x)
    \right)
    \ 
    +
    \ 
    F_{Y(1)}(z_{l-1}|x)
-
      F_{Y(1)}(z_l)
\,,
  \end{align*}
  for $l\in \left\{ 1,\ldots,m \right\}$.
  With the same arguments as before, we see that these brackets cover $\mathcal{G}$.
  Let's calculate the size.
  It holds
  \begin{align*}
    &
    \norm{
      \frac{T}{\pi(X)}
      \left( 
      \mathbf{1}{
      \left\{ 
      Y(T)\in [z_{l-1},z_l] 
    \right\}
    }
      +
      \P
      \left[ 
      Y(1)\in [z_{l-1},z_l] 
      \,
      |
      \,
      X
      \right]
      \right)
    }_{ L^2(\P)}
    \\
    &
    \
    \lesssim
    \
    \left( 
      \E
      \left[ 
        \frac{1}{\pi(X)}
        \frac{T}{\pi(X)}
      \left( 
      \mathbf{1}{
      \left\{ 
      Y(T)\in [z_{l-1},z_l] 
    \right\}
    }
      +
      \P
      \left[ 
      Y(1)\in [z_{l-1},z_l] 
      \,
      |
      \,
      X
      \right]
      \right)
      \right]
    \right)
    ^{1/2}
    \\
    &
    \
    \lesssim
    \
    \left( 
      \E
      \left[ 
        \frac{1}{\pi(X)}
      \P
      \left[ 
      Y(1)\in [z_{l-1},z_l] 
      \,
      |
      \,
      X
      \right]
      \right]
    \right)
    ^{1/2}
    \\
    &
    \ 
    \lesssim
    \ 
    \left( 
      \norm{1/\pi(X)}_{ L^2(\P)}
      \sqrt{\varepsilon}
    \right)
    ^{1/2}
    \ 
    =
    \ 
    \varepsilon^{1/4}
    \norm{1/\pi(X)}_{ L^2(\P)}^{1/2}
  \end{align*}
  and
  \begin{align*}
     \norm{
      \P
      \left[ 
      Y(1)\in [z_{l-1},z_l] 
      \,
      |
      \,
      X
      \right]
      \ 
     + 
      \ 
      \P
      \left[ 
      Y(1)\in [z_{l-1},z_l] \,
      \right]
    }_{ L^2(\P)}
    \ 
    \lesssim
    \ 
    \varepsilon^{1/2}
    \,.
  \end{align*}
  Thus
  \begin{align*}
  \norm{
    \left( 
  \overline{g_l}
-
  \underline{g_l}
    \right)
  (T,X,Y(T))
}_
{ L^2(\P)}
    &
\ 
\lesssim
\ 
\varepsilon^{1/4}
\left( 
  1
  +
    \norm{1/\pi(X)}_{ L^2(\P)}^{1/2}
\right)
\\
    &
\ 
\lesssim
\ 
\varepsilon^{1/4}
\left( 
  1
  +
    \norm{1/\pi(X)}_{ L^2(\P)}
\right)
\,.
  \end{align*}
As before, it follows
\begin{gather*}
    N_{[\,]}
    (
    \varepsilon
    ,
    \mathcal{G}, L^2(\P))
    \ 
    \lesssim
    \ 
    \left( 
    \frac{
      1+
    \norm{1/\pi(X)}_{ L^2(\P)}
    }
    {\varepsilon}
    \right)^4
    \,.
\end{gather*}
\end{proof}
Before we give another example, we fix some useful properties of $f_z$.
\begin{lemma}
    \label{lem:f_z}
    It holds
    $f_z(T,X,Y(T))\in L^1(\P)$
    and 
    $f_z(T,X,Y(T))\perp D_N$
    for all $z\in\R$.
    If also Assumption~\ref{aa:assumption:treatment_str_ign} holds,
    then 
    for all $z\in\R$
    \begin{gather*}
      \E
      \left[
        f_z
        \left( 
          T,
          X,
          Y(T)
        \right)
        \,
        |
        \,
        X
      \right]
      \ 
      =
      \ 
      0
      \qquad
      \text{almost surely.}
    \end{gather*}
  \end{lemma}
  \begin{proof}
    Since $f_z$ is bounded by 1,
    it holds
    $f_z(T,X,Y(T))\in L^1(\P)$.
    Since
    \begin{gather*}
    (T,X,Y(T))
    \ 
    \perp 
    \ 
    D_N
    \ 
    =
    \ 
    (T_i,X_i)_{i\in \left\{ 1,\ldots,N \right\}}
    \end{gather*}
    it holds
    $f_z(T,X,Y(T))\perp D_N$
    for all $z\in\R$.
    For the third statement, note that
    \begin{align*}
      \E
      \left[
        f_z
        \left( 
          T,
          X,
          Y(T)
        \right)
        \,
        |
        \,
        X
      \right]
      &
      \ 
      =
      \ 
      \E
      \left[
      T
      \left( 
        \mathbf{1}
        _{\left\{  Y(T)\,\le\,z \right\}}
        -
        F_{Y(1)}(z|X)
      \right)
        \,
        |
        \,
        X
      \right]
      \\
      &
      \ 
      =
      \ 
      \E
      \left[
        \mathbf{1}
        _{\left\{  Y(1)\,\le\,z \right\}}
        -
        F_{Y(1)}(z|X)
        \,
        |
        \,
        X
        ,
        T=1
      \right]
      \pi(X)
      \\
      &
      \ 
      =
      \ 
      \left( 
      \E
      \left[
        \mathbf{1}
        _{\left\{  Y(1)\,\le\,z \right\}}
        \,
        |
        \,
        X
      \right]
      \ 
        -
      \ 
        F_{Y(1)}(z|X)
      \right)
      \pi(X)
      \\
      &
      \ 
      =
      \ 
      0
      \qquad
      \text{almost surely.}
    \end{align*}
    The third equality is due to Assumption~\ref{aa:assumption:treatment_str_ign}.
  \end{proof}

Next, consider the stochastic process (indexed over $x\in\R^d$)
\begin{gather}
  \label{ghost_function}
  \mathbf{1}{
    \left\{ 
      \sup_{y\in A_N(x)}
      \left| 
      w_0^\dagger(y)
      -
      \frac{1}{\pi(y)}
      \right|
      \,
      \le
      \,
      \varepsilon_N
    \right\}
  }
  \left( 
    w_0^\dagger(x)
      -
      \frac{1}{\pi(x)}
  \right)
  \cdot
  \mathbf{1}
  \bigcup_{k=1}^n
  \left\{ 
    x=X_k
  \right\}
  \,,
\end{gather}
where $(\varepsilon_N)$ is the learning rate of Theorem~\ref{th:max_weight_cons}.
We show, 
that under mild regularity conditions on the inverse propensity score function all paths of \eqref{ghost_function} are contained in shrinking function classes $(\mathcal{F}_N)$ - and provide bracketing numbers. 
To be more precise, we need theory from \cite[§2.7.1]{vaart2013}.

Let for any vector $k\in\mathbb{N}_0^d \ (d\in\R)$
\begin{gather*}
  D^k
  \ :=\ 
  \frac
  {\partial^{\norm{k}_1}}
  {
    \partial^{k_1}x_1
    \cdots
    \partial^{k_d}x_d
  }
  \,,
\end{gather*}
and let $\lfloor a \rfloor$ be the greatest integer smaller than $a>0$.
For $\alpha>0$, a bounded set 
$\mathcal{Z}\subset\R^d\ (d\in\mathbb{N})$
and
$M>0$, we define $C^\alpha_M(\mathcal{Z})$ to be the space of all continuous functions $f\colon \mathcal{Z}\to\R$ with
\begin{gather*}
  \max_{\norm{k}_1\le \alpha}\sup_{x\in\mathcal{Z}}
  \left| D^k f(x) \right|
  \ 
  +
  \ 
  \max_{\norm{k}_1=\lfloor \alpha \rfloor}\sup_{x,y}
  \frac
  {
  \left|
  D^k f(x) 
  -
  D^k f(y) 
  \right|
  }
  {
    \norm{x-y}_2^{\alpha-\lfloor \alpha \rfloor}
  }
  \
  \le
  \ 
  M
  \,.
\end{gather*}
where the suprema in the second term are taken over all $x,y$ in the interior of $\mathcal{Z}$ with $x\neq y$.
Furthermore, let
\begin{gather*}
  \mathcal{Z}^1
  :=
  \left\{ 
    y\in\R^d
    \colon
    \norm{x-y}_2 <1
    \ 
    \text{for some}\ x\in\mathcal{Z}
  \right\}
  \,.
\end{gather*}
\begin{lemma}
  \label{vdv_coro}
  Let $\mathcal{P}=\left\{ A_1,A_2,\ldots \right\}$ be a partition of $\R^d$ into bounded, convex sets with non-empty interior, and let $\mathcal{F}$ be a class of functions $f\colon\R^d\to\R$ such that the restrictions $\mathcal{F}_{|A_j}$ belong to $C^\alpha_{M_j}(A_j)$
  for all $j\in\mathbb{N}$.
  Then there exists a constant $K$, depending only on $\alpha$, $V$, $r$ and $d$
  such that
  \begin{gather}
    \label{667}
    \log
    N_{[\,]}
    (
    \varepsilon
    ,
    \mathcal{F}
    ,
    L^r(\mathbf{Q})
    )
    \le
    K
    \left( \frac{1}{\varepsilon} \right)^V
    \left( 
      \sum_{j=1}^{\infty}
      \lambda(A_j^1)^{r/(V+r)}
      M_j^{Vr/(V+r)}
      \mathbf{Q}(A_j)^{V/(V+r)}
    \right)
    ^{(V+r)/r}
  \end{gather}
  for every $\varepsilon>0$, $V\ge d/\alpha$, and probability measure $\mathbf{Q}$.
\end{lemma}
\begin{proof}
  \emph{\cite[Corollary~2.7.4]{vaart2013}}
\end{proof}

The next lemma gives sufficient conditions on the regularity of the inverse propensity score function.
\begin{lemma}
  \label{lem:br_n_st}
  Let $(\mathcal{P}_N)$ denote a sequence of qubic partitions
  $\mathcal{P}_N=\left\{ A_{N,1},A_{N,2},\ldots \right\}$ 
  of $\R^d$ 
  with decreasing width $(h_N)\subset(0,1]$ such that $h_N\to 0$ for $N\to\infty$.
  Furthermore, assume that there exists
  $\alpha>d/2$, where $\mathcal{X}\subseteq \R^d$, such
  that for 
  $V:=d/\alpha$
 and for all 
$
(j,N)\in\mathbb{N}^2
$
there exists 
$M_{N,j}\ge 1$ such that 
\begin{gather}
  \label{0897}
  \frac{1}{\pi(\cdot)}
  \in C^\alpha_{M_{N,j}}(A_{N,j})
  \quad
  \text{and}
  \quad
  \sum_{j=1}^{\infty} 
  M_{N,j}^{2V/(V+2)}
  \P
  [
  X\in A_{N,j}
  ]^{V/(V+2)}
  \ 
  \lesssim
  \ 
  1
  \,.
\end{gather}
Then for any decreasing sequence
  $(\varepsilon_N)$ with $\varepsilon_N\to 0$ for $N\to\infty$,
there exists a sequence of (measurable) function classes
$(\mathcal{F}_N)$
with envelope functions
$(F_N)$,
satisfying 
for some $k<2$
\begin{gather*}
\norm{F_N}_{L^2(\P)}
\ 
\le
\ 
\varepsilon_N
\quad
\text{and}
\quad
  \log
  N_{[\,]}(\varepsilon,\mathcal{F}_N,\mathrm{L}_2(\P_{X}))
  \ 
  \lesssim
  \ 
  \left( 
  \frac{1}{\varepsilon}
  \right)^k
  \quad
  \text{for all}
  \ 
  N\in\mathbb{N}
  \,,
\end{gather*}
such that
for all $N\in\mathbb{N}$ the paths of the stochastic process
\begin{gather}
  \label{error_process}
  \mathbf{1}{
    \left\{ 
      \sup_{y\in A_N(x)}
      \left| 
      w_0^\dagger(y)
      -
      \frac{1}{\pi(y)}
      \right|
      \,
      \le
      \,
      \varepsilon_N
    \right\}
  }
  \left( 
    w_0^\dagger(x)
      -
      \frac{1}{\pi(x)}
  \right)
  \cdot
  \mathbf{1}
  \bigcup_{k=1}^N
  \left\{ x=X_k \right\}
  \,.
\end{gather}
are contained in $\mathcal{F}_N$.
\end{lemma}
\begin{proof}
  We want to employ Lemma~\ref{vdv_coro}. 
  To do this, the crucial observation is that by Lemma~\ref{lem:weights:meas}.\textit{(ii)}
  \begin{gather*}
    \text{the paths}\quad
    w^\dagger(\cdot)(\omega)
    \quad
    \text{are constant on each cell}\ 
    A_N\in\mathcal{P}_N
    \ \text{for all}\ \omega\in\Omega
    \,.
  \end{gather*}
  Thus, the regularity of a path of \eqref{error_process}
  on each cell $A_N\in\mathcal{P}_N$ is decided by 
  $1/\pi(\cdot)$.
  Indeed, a path of \eqref{error_process} is either 0 if the threshold of $\varepsilon_N$ is exceeded somewhere in the cell, or has the form constant-minus-smooth-function.
In any case, it is continuous and bounded by $\varepsilon_N$.
All its derivatives are 0 (if the threshold is exceeded) or are governed by $1/\pi(\cdot)$.
Thus, it follows from \eqref{0897}
\begin{gather}
  \eqref{error_process}(\cdot)(\omega)
  \in C^\alpha_{M_{N,j}}(A_{N,j})
  \quad
  \text{and}
  \quad
  \sum_{j=1}^{\infty} 
  M_{N,j}^{2V/(V+2)}
  \P
  [
  X\in A_{N,j}
  ]^{V/(V+2)}
  \ 
  \lesssim
  \ 
  1
  \,.
\end{gather}
To bound the right-hand-side in \eqref{667} we note
that 
$
\lambda(A_{N,j})=h^d_N
$ 
and thus
$
\lambda(A_{N,j}^1)\lesssim 1
$
for all $(j,N)\in\mathbb{N}^2$.
Thus
\begin{gather*}
  \sum_{j=1}^{\infty} 
  \lambda(A_{N,j}^1)^{2/(V+2)}
  M_{N,j}^{2V/(V+2)}
  \P
  [
  X\in A_{N,j}
  ]^{V/(V+2)}
  \ 
  \lesssim
  \ 
  1
  \,.
\end{gather*}
$
\eqref{error_process}(\cdot)(\omega)\in\mathcal{F}_N
$, 
where
$
\mathcal{F}_N
$
restricted to $A_{N,j}$ is 
$
  C^\alpha_{M_{N,j}}(A_{N,j})
$
and satisfies the requirements of Lemma~\ref{vdv_coro}.
Since $V=d/\alpha \in (0,2)$ by $\alpha>d/2$, 
applying Lemma~\ref{vdv_coro} finishes the proof.
\end{proof}
\begin{remark}
  Note, that we only get $L^2(\P_X)$ bracketing numbers in this way. If we assume, that all functions in $\mathcal{F}_N$ are independent of $(T,Y)$
  we readily obtain $L^2(\P)$ bracketing numbers. Note, that $w^\dagger(X)$ and $1/\pi(X)$ are independent of $(T,Y)$.  
\end{remark}
Next, we show, that a finite covariate space always meets the requirements of Lemma~\ref{lem:br_n_st} - and that a continuous distribution of $X$ never does so.
\begin{lemma}
  Consider the covariate space $\mathcal{X}$. 
  \begin{enumerate}[label=(\roman*)]
    \item
If $\#\mathcal{X}<\infty$, that is, $X$ can take only finitely many values with positive probability,
then 
\begin{align*}
  \sum_{j=1}^{\infty} 
  M_{N,j}^{2V/(V+2)}
  \P
  [
  X\in A_{N,j}
  ]^{V/(V+2)}
  \ 
  \lesssim
  \ 
  1
  \,.
\end{align*}
\item
  If $X$ is continuously distributed, then
  \begin{align*}
  \sum_{j=1}^{\infty} 
  \P
  [
  X\in A_{N,j}
  ]^{V/(V+2)}
  \ 
  \to
  \ 
  \infty
  \qquad
  \text{for}\ 
  N\to\infty
  \,.
  \end{align*}
  \end{enumerate}
\end{lemma}

\begin{proof}
 Assume $\#\mathcal{X}<\infty$, that is, $X$ can take only finitely many values with positive probability.
We write
\begin{gather*}
  J_N
  :=
  \left\{ 
    j\in\mathbb{N}
    \colon
    \P[X\in A_{N,j}]>0
  \right\}
  \,.
\end{gather*}
It holds
$\#J_N\le \# \mathcal{X}<\infty$.
Thus, the following maximum is attained
\begin{gather*}
  \max_{j\in J_N} M_{N,j}
  =:M^*_N
  \,.
\end{gather*}
But the partitions increasingly better fit the support of $X$. Thus
$M^*_N$ is decreasing in $N$, that is,  $\infty>M^*_1\ge M^*_N$.
It follows
\begin{align*}
  \sum_{j=1}^{\infty} 
  M_{N,j}
  ^{2V/(V+2)}
  \P
  [
  X\in A_{N,j}
  ]^{V/(V+2)}
  \ 
  \le
  \ 
  \left( 
  M^*_1
  \right)
  ^{2V/(V+2)}
  \cdot
  \# J_N
  \ 
  \lesssim
  \ 
  1
  \,.
\end{align*}
Now let $f_X$ be the probability density of $X$. 
  Then there exists a compact set $K\subset\mathcal{X}\subset \R^d$, such that
  $
  \inf_{x\in K}f_X(x)
  >0
  $. Since $\mathcal{P}_N$ are cubic partitions, it holds for 
  \begin{gather*}
    I_N
    :=
    \left\{ 
      i\in\mathbb{N}\colon
      A_{N,i}\subset K
    \right\}
    \qquad
    \text{that}
    \qquad
    \bigcup_{i\in I_N}A_{N,i}\nearrow K
    \,.
  \end{gather*}
Thus
\begin{align*}
  \sum_{i=1}^\infty
  \P
  [
  X\in A_{N,i}
  ]^{V/(V+2)}
  &
  \ 
  \ge
  \ 
  \sum_{i\in I_N}
  \P
  [
  X\in A_{N,i}
  ]^{V/(V+2)}
  \\
  &
  \ 
  \ge
  \ 
  \inf_{x\in K}f_X(x)
  ^{V/(V+2)}
  \cdot 
  h_N^{d\cdot(V/(V+2)-1)}
  \sum_{i\in I_N}
  \lambda
  \left( 
  A_{N,i}
  \right)
  \\
  &
  \ 
  \to
  \ 
  \infty
  \,.
\end{align*}
This follows from
$
  \sum_{i\in I_N}
  \lambda
  \left( 
  A_{N,i}
  \right)
  \to 
  \lambda(K)>0
$,
  $
  \inf_{x\in K}f_X(x)
  >0
  $,
  $V/(V+2)-1<0$ and $h_N\to 0$.
\end{proof}


\subsection{Maximal Inequality}
  In our application we need concentration inequalities for 
$
  \norm{\G_n}^*_\mathcal{F}
$.
One easy way to obtain this is, to use a maximal inequality (see Theorem~\ref{th:max_ineq}) to control the expectation,
together with Markov's inequality. There are also Bernstein-like inequalities for empirical processes (see \cite[§2.14.2]{vaart2013}). 
\begin{theorem}
  \label{th:max_ineq}
  \emph{(Maximal inequality)}
  For any class $\mathcal{F}$ of measurable functions with envelope function $F,$
  \begin{gather*}
    \E^*
    \norm{
      \G
      _n
      }
      _\mathcal{F}
    \ 
    \lesssim
    \ 
    J
    _{[\,]}
    (
    \norm{
      F
    }
    _{ L^2(\P)}
    ,
    \mathcal{F}
    ,
    L^2(\P)
    )
    \, 
    .
  \end{gather*}
\end{theorem}
\begin{proof}
  \cite[Corollary~19.35]{Vaart2000}
\end{proof}

\begin{lemma}
  \label{markov_max_lemma}
  Let $(\mathcal{H}_N)$ be a sequence of measurable function classes with envelope functions $(H_N)$.
  If
  \begin{gather*}
    J_{[\, ]}
    \left( 
    \norm{H_N}_{ L^2(\P)}
    ,
    \mathcal{H}_N
    ,
     L^2(\P)
    \right)
    \ 
    \to
    \ 
    0
    \qquad
    \text{for}
    \ 
    N
    \to
    \infty
    \,,
  \end{gather*}
  it holds 
  $
  \norm{\G_N}^*_{\mathcal{H}_N}\overset{\P}{\to}0
  $.
\end{lemma}
\begin{proof}
  By Markov's inequality and Theorem~\ref{th:max_ineq} it holds for all $\varepsilon>0$
  \begin{align*}
    \P
    [
  \norm{\G_N}^*_{\mathcal{H}_N}
  \ge
  \varepsilon
    ]
    &
    \ 
    \le
    \ 
    \varepsilon^{-1}
    \E
    [
  \norm{\G_N}^*_{\mathcal{H}_N}
    ]
    \ 
    =
    \ 
    \varepsilon^{-1}
    \E^*
    [
  \norm{\G_N}_{\mathcal{H}_N}
    ]
    \\
    &
    \ 
    \lesssim
    \ 
    \varepsilon^{-1}
    J_{[\, ]}
    \left( 
    \norm{H_N}_{ L^2(\P)}
    ,
    \mathcal{H}_N
    ,
     L^2(\P)
    \right)
    \\
    &
    \ 
    \to
    \ 
    0
    \qquad
    \text{for}
    \ 
    N\to\infty
    \,.
  \end{align*}
\end{proof}

\begin{lemma}
  \label{aa:r3:lemma:1}
  Let
$(\varepsilon_N)\subset(0,1]$
be 
a decreasing sequence
with $\varepsilon_N\to 0$ for $N\to\infty$ and
$(\mathcal{F}_N)$
a sequence of (measurable) function classes
with envelope functions
$(F_N)$,
satisfying 
for some $k<2$
\begin{gather*}
\norm{F_N}_{L^2(\P)}
\ 
\le
\ 
\varepsilon_N
\quad
\text{and}
\quad
  \log
  N_{[\,]}(\varepsilon,\mathcal{F}_N,\mathrm{L}_2(\P_{X}))
  \ 
  \lesssim
  \ 
  \left( 
  \frac{1}{\varepsilon}
  \right)^k
  \quad
  \text{for all}
  \ 
  N\in\mathbb{N}
  \,.
\end{gather*}
Then
\begin{gather*}
  J_{[\,]}(
\norm{F_N}_{L^2(\P)}
,\mathcal{F}_N\cdot\mathcal{F},\mathrm{L}_2(\P))
  \to 0
  \quad
  \text{and}
  \quad
  \norm{\G_N}^*_{\mathcal{F}_N\cdot\mathcal{F}}\overset{\P}{\to}0
  \qquad
  \text{for}\ 
  N\to\infty
  \,,
\end{gather*}
where $\mathcal{F}$ is defined in \eqref{F_g}.
\end{lemma}
\begin{proof}
  By assumption
  and Lemma~\ref{aa:mean:l:br} it holds
for some $k<2$
\begin{gather*}
\norm{F_N}_{L^2(\P)}
\ 
\le
\ 
\varepsilon_N
\quad
\text{and}
\quad
  \log
  N_{[\,]}(\varepsilon,\mathcal{F}_N,\mathrm{L}_2(\P))
  \ 
  \lesssim
  \ 
  \left( 
  \frac{1}{\varepsilon}
  \right)^k
  \quad
  \text{for all}
  \ 
  N\in\mathbb{N}
  \,,
\end{gather*}
and
  \begin{gather*}
    N_{[\,]}
    (
    \varepsilon
    ,
    \mathcal{F}, L^2(\P))
    \ 
    \lesssim
    \ 
    \left( 
      \frac{1}{\varepsilon}
    \right)^2
    \qquad
    \text{for all}
    \ 
    \varepsilon>0
    \,.
  \end{gather*}
  Since $\mathcal{F}_N$ and $\mathcal{F}$ have envelope functions smaller 1, we can apply Lemma~\ref{lem_prod_br} to get
  \begin{gather*}
  \log
  N_{[\,]}(\varepsilon,\mathcal{F}_N\cdot\mathcal{F},\mathrm{L}_2(\P))
  \ 
  \lesssim
  \ 
  \left( 
  \frac{1}{\varepsilon}
  \right)^k
  +
  \log
  (1/\varepsilon)
  \ 
  \lesssim
  \ 
  \left( 
  \frac{1}{\varepsilon}
  \right)^k
  \quad
  \text{for all}\ 
  \varepsilon>0
  \,.
  \end{gather*}
  Since 
  $k/2\in(0,1)$
  it holds
\begin{align*}
  J_{[\,]}(
\norm{F_N}_{L^2(\P)}
,\mathcal{F}_N\cdot\mathcal{F},\mathrm{L}_2(\P))
  &
  \ 
=
  \ 
\int_0^{
\norm{F_N}_{L^2(\P)}
}
\sqrt{
  \log
  N_{[\,]}(\varepsilon,\mathcal{F}_N\cdot\mathcal{F},\mathrm{L}_2(\P))
}
\,d\varepsilon
\\
&
\ 
\lesssim
\ 
\int_0^{
  \varepsilon_N
}
  \left( 
  \frac{1}{\varepsilon}
\right)^{k/2}
\,d\varepsilon
\\
&
\ 
=
\ 
\frac{
\varepsilon_N^{1-k/2}
}{1-k/2}
\ 
\to 0
\ 
\qquad
\text{for}
\ 
N\to\infty
\,.
\end{align*}
The second statement follows from Lemma~\ref{markov_max_lemma}
for 
$\mathcal{H}_N:=\mathcal{F}_N\cdot\mathcal{F}$ and
$H_N:=F_N$.
\end{proof}





\subsubsection{Donseker Theorem}
  There is a powerful theorem - a central limit theorem for $\G_N$ uniform in $\mathcal{F}$ - that we now introduce.
\begin{definition}
  We call a class 
  $\mathcal{F}$ of measurable functions 
$\P$-Donsker
if the sequence of processes 
$\left\{ \G_N f \colon f\in\mathcal{F}\right\}$
converges in
$l^\infty(\mathcal{F})$
to a tight limit process.
\end{definition}

\begin{theorem}
  Every class $\mathcal{F}$ of measurable functions 
  with
  \begin{gather*}
    J
    _{[\,]}
    (
    1
    ,
    \mathcal{F}
    ,
    L_2(\P)
    )
    <\infty
  \end{gather*}
  is
  $\P$-Donsker.
  Furthermore,
  the sequence of processes 
$\left\{ G_N f \colon f\in\mathcal{F}\right\}$
  converges 
  in
$l^\infty(\mathcal{F})$
to a Gaussian process with mean 0 and covariance function given by
\begin{gather*}
  \mathbf{Cov}(f,g)
  \ 
  :=
  \ 
  \E[fg]
  \ 
  -
  \ 
  \E[f]\E[g]
  \,.
\end{gather*}
\end{theorem}
\begin{proof}
  \cite[Theorem~19.5]{Vaart2000}
\end{proof}



\subsection{Propensity Score Weights}
  The next lemma shows what effect the 
\textbf{propensity score weights}
$T/\pi(X)$ have on other functions.
\begin{lemma}
  \label{lem:ps_weights}
  \label{ps_weights_lemma}
  Let
  $
  g_1\colon
  \mathcal{X}\to\R
  $
  and
  $
  g_2\colon
  \mathcal{Y}\to\R
  $
  be a measurable functions.
  \begin{enumerate}[label=(\roman*)]
    \item
  It holds
  \begin{gather*}
    \E
    \left[
    \frac{T}{\pi(X)}
    g_1(X)
    \right]
    \ 
    =
    \ 
    \E
    \left[
    g_1(X)
    \right]
    \,.
  \end{gather*}
  \item
 If Assumption~\ref{aa:assumption:treatment_str_ign} holds true, then
  \begin{gather*}
    \E
    \left[
    \frac{T}{\pi(X)}
    g_2(Y(T))
    \right]
    \ 
    =
    \ 
    \E
    \left[
    f(Y(1))
    \right]
    \,.
  \end{gather*}
  \end{enumerate}
 \end{lemma}




\section{Main Result}
  \begin{ftheorem}
  \label{th:main}
The stochastic process
\begin{gather}
    \sqrt{N}
    \left( 
  \frac{1}{N}
    \sum_{i=1}^{N} 
    T_i
    \cdot
    w_0^\dagger(X_i)
    \cdot
    \mathbf{1}
    \left\{ Y_i\,\le\, z \right\}
    \ 
    -
    \ 
    F_{Y(1)}(z)
    \right)
    _{z\in\R}
    \,
  \end{gather}
  converges in
  $l^\infty(\R)$
  to a Gaussian process with mean 0 and covariance function
  satisfying for all $z_1,z_2\in\R$
\begin{align}
  \label{cov:lp}
 \begin{split}
  &
  \mathbf{Cov}
  (z_1,z_2)
  \\
  &
  =\ 
  \E
  \left[ 
 \frac{
 F_{Y(1)}(z_1 \land z_2\,|\,X)
}{\pi(X)}
\ 
-
\ 
 \frac{1-\pi(X)}{\pi(X)}
 F_{Y(1)}(z_1|X)
 \cdot
 F_{Y(1)}(z_2|X)
  \right]
  \\
  &
  \qquad 
 -
 \ 
 F_{Y(1)}(z_1)
 \cdot
 F_{Y(1)}(z_2)
 \,.
 \end{split}
\end{align}
\end{ftheorem}



\section{Error Decomposition}
  \newpage
\begin{lemma}
  \label{aa:mean:lemma_decomp}
  It holds
  \begin{gather}
    \sqrt{N}
    \left( 
  \frac{1}{N}
    \sum_{i=1}^{N} 
    T_i
    \cdot
    w_i^\dagger(X_i)
    \cdot
    \mathbf{1}
    \left\{ Y_i\,\le\, z \right\}
    \ 
    -
    \ 
    F_{Y(1)}(z)
    \right)
    _{z\in\R}
    \ 
    =
    \ 
    R_1
    \ 
    +
    \ 
    R_2
    \ 
    +
    \ 
    R_3
    \ 
    +
    \ 
    R_4
  \end{gather}
  with
\begin{align*}
  R_1
  &
  \ 
  :=
  \ 
  \sqrt{N}
  \sum_{k=1}^{N} 
  \left[ 
  \frac{1}{N}
  \left( 
    \sum_{i=1}^{N} 
    T_i
    \cdot
    w^\dagger(X_i)
    \cdot
    B_k(X_i)
    \ 
    -
    \ 
    \sum_{i=1}^{N} 
    B_k(X_i)
  \right)
  \cdot
  F_{Y(1)}(z|X_k)
  \right]
  _{z\in\R}
  \,,
  %%%% 1 %%%%
  \\
  R_2
  &
  \
  :=
  \ 
  \sqrt{N}
    \sum_{i=1}^{N} 
    \frac{1}{N}
    \left[ 
      \left( 
    T_i\cdot w^\dagger(X_i) 
    \ 
    -
    \ 
    1 
      \right)
    \left( 
  F_{Y(1)}(z|X_i)
    \ 
    -
    \ 
    \sum_{k=1}^{N} 
    B_k(X_i)
    \cdot
  F_{Y(1)}(z|X_k)
    \right)
    \right]
  _{z\in\R}
  \,,
  %%%% 2 %%%%
  \\
  R_3
  &
  \
  :=
  \ 
  \sqrt{N}
  \left( 
  \frac{1}{N}
    \sum_{i=1}^{N} 
    \left[ 
    T_i
    \cdot
    \left( 
    w^\dagger(X_i) 
    \ 
    -
    \ 
    \frac{1}{\pi(X_i)}
    \right)
    \cdot
    \left( 
    \mathbf{1}{\left\{ Y_i \le z \right\}}
    \ 
    -
    \ 
  F_{Y(1)}(z|X_i)
    \right)
    \,
    \right]
  \right)
  _{z\in\R}
  \,,
  %%%% 3 %%%%
  \\
  R_4
  &
  \
  :=
  \ 
  \sqrt{N}
  \left( 
  \frac{1}{N}
    \sum_{i=1}^{N} 
    \frac{T_i}{\pi(X_i)}
    \left( 
    \mathbf{1}{\left\{ Y_i \le z \right\}}
    -
  F_{Y(1)}(z|X_i)
    \right)
    \ 
    +
    \ 
    \left( 
  F_{Y(1)}(z|X_i)
    -
  F_{Y(1)}(z)
    \right)
  \right)
  _{z\in\R}
  \,.
  \end{align*}
\end{lemma}
\nopagebreak
\begin{proof}
  We fix $z\in\R$.
  It holds
  \begin{align*}
    &
    \frac{1}{N}
    \sum_{i=1}^{N} 
    w^\dagger(X_i)
    \cdot
    T_i
    \cdot
    \mathbf{1}{\left\{ Y_i\, \le\, z \right\}}
    \\
    &
    \ 
    =
    \ 
    \frac{1}{N}
    \sum_{i=1}^{N} 
    \left( 
    w^\dagger(X_i)
    -
    \frac{1}{\pi(X_i)}
    \right)
    T_i
    \cdot
    \mathbf{1}{\left\{ Y_i\, \le\, z \right\}}
    \\
    &
    \quad 
    +
    \ 
    \frac{1}{N}
    \sum_{i=1}^{N} 
    \frac{T_i}{\pi(X_i)}
    \mathbf{1}{\left\{ Y_i\, \le\, z \right\}}
    \\
    &
    \ 
    =
    \ 
    \frac{1}{N}
    \sum_{i=1}^{N} 
    \left( 
    w^\dagger(X_i)
    -
    \frac{1}{\pi(X_i)}
    \right)
    T_i
    \left( 
    \mathbf{1}{\left\{ Y_i\, \le\, z \right\}}
    -
    F_{Y(1)}(z|X_i)
    \right)
    \\
    &
    \quad 
    +
    \ 
    \frac{1}{N}
    \sum_{i=1}^{N} 
    \frac{T_i}{\pi(X_i)}
    \left( 
    \mathbf{1}{\left\{ Y_i\, \le\, z \right\}}
    -
    F_{Y(1)}(z|X_i)
    \right)
    \\
    &
    \qquad 
    +
    \ 
    \frac{1}{N}
    \sum_{i=1}^{N} 
    w^\dagger(X_i)\cdot T_i\cdot
    F_{Y(1)}(z|X_i)
    \\
    &
    \ 
    =
    \ 
    R_3(z)/\sqrt{N}
    \\
    &
    \quad 
    +
    \ 
    \frac{1}{N}
    \sum_{i=1}^{N} 
    \frac{T_i}{\pi(X_i)}
    \left( 
    \mathbf{1}{\left\{ Y_i\, \le\, z \right\}}
    -
    F_{Y(1)}(z|X_i)
    \right)
    +
    \left( 
    F_{Y(1)}(z|X_i)
    -
    F_{Y(1)}(z)
    \right)
    \\
    &
    \qquad
    +
    \ 
    \frac{1}{N}
    \sum_{i=1}^{N} 
    \left( 
    w^\dagger(X_i)\cdot T_i
    \ 
    -
    \ 
    1
    \right)
    F_{Y(1)}(z|X_i)
    \\
    &
    \quad\qquad
    +
    \ 
    F_{Y(1)}(z)
    \\
    &
    \ 
    =
    \ 
    R_3(z)/\sqrt{N}
    \\
    &
    \quad
    +
    \ 
    R_4(z)/\sqrt{N}
    \\
    &
    \qquad
    +
    \ 
    \frac{1}{N}
    \sum_{i=1}^{N} 
    \left( 
    w^\dagger(X_i)\cdot T_i
    \ 
    -
    \ 
    1
    \right)
    \left( 
    F_{Y(1)}(z|X_i)
    -
    \sum_{k=1}^{N} 
    B_k(X_i)
    \cdot
  F_{Y(1)}(z|X_k)
    \right)
    \\
    &
    \quad\qquad
    +
    \ 
    \frac{1}{N}
    \sum_{i=1}^{N} 
    \left( 
    w^\dagger(X_i)\cdot T_i
    \ 
    -
    \ 
    1
    \right)
    \sum_{k=1}^{N} 
    B_k(X_i)
    \cdot
  F_{Y(1)}(z|X_k)
    \\
    &
    \qquad\qquad
    +
    \ 
    F_{Y(1)}(z)
\\
    &
    \ 
    =
    \ 
    R_3(z)/\sqrt{N}
    \\
    &
    \quad
    +
    \ 
    R_4(z)/\sqrt{N}
    \\
    &
    \qquad
    +
    \ 
    R_2(z)/\sqrt{N}
    \\
    &
    \quad\qquad
    +
    \ 
    \sum_{k=1}^{N} 
    \frac{1}{N}
    \sum_{i=1}^{N} 
    \left( 
    w^\dagger(X_i)\cdot T_i
    B_k(X_i)
    \ 
    -
    \ 
    B_k(X_i)
    \right)
    \cdot
  F_{Y(1)}(z|X_k)
    \\
    &
    \qquad\qquad
    +
    \ 
    F_{Y(1)}(z)
    \\
    &
    \ 
    =
    \ 
    \left( 
R_3(z)
    \ 
    +
    \ 
    R_4(z)
    \ 
    +
    \ 
    R_2(z)
    \ 
    +
    \ 
    R_1(z)
    \right)
    /\sqrt{N}
    \ 
    +
    \ 
    F_{Y(1)}(z)
    \,.
  \end{align*}
  This holds for all $z\in\R$.
  Multiplying with $\sqrt{N}$ yields the result.
\end{proof}


\section{Analysis of the Terms}
  \subsection{$R_1$}
    The convergence of this term is closely related to the box constraints of Problem~\ref{bw:1:primal}.
\begin{lemma}
  \label{aa:mean:l:r1}
  Let
$\sqrt{N}\norm{\delta}_1\overset{\P}{\to}0$.
Then it holds
$\sup_{z\in\R}|R_1(z)|\overset{\P}{\to}0$.
  \end{lemma}
\begin{proof}
  By Theorem~\ref{th:weights_constr}, 
  $(w^\dagger_0(X_i))$
satisfy the box constraints of Problem~\ref{bw:1:primal}.
Thus
  \begin{align}
    \label{R_1:1}
    \begin{split}
    \sup_{z\in\R}
    \left| 
    R_1(z)
    \right|
    &
    \ 
    =
    \ 
  \sqrt{N}
  \sup_{z\in\R}
  \sum_{k=1}^{N} 
  \left[ 
  \frac{1}{N}
  \left( 
    \sum_{i=1}^{N} 
    T_i
    \cdot
    w^\dagger_0(X_i)
    \cdot
    B_k(X_i)
    \ 
    -
    \ 
    \sum_{i=1}^{N} 
    B_k(X_i)
  \right)
  \cdot
  F_{Y(1)}(z|X_k)
  \right]
    \\
    &
    \ 
    \le
    \ 
  \sqrt{N}
  \sum_{k=1}^{N} 
  \left| 
  \frac{1}{N}
  \left( 
    \sum_{i=1}^{N} 
    T_i
    \cdot
    w^\dagger_0(X_i)
    \cdot
    B_k(X_i)
    \ 
    -
    \ 
    \sum_{i=1}^{N} 
    B_k(X_i)
  \right)
  \right|
  \cdot
    \sup_{z\in\R}
  F_{Y(1)}(z|X_k)
  \\
    &
    \ 
    \le
    \ 
  \sqrt{N}
  \norm{\delta}_1
    \end{split}
  \end{align}
  The last inequality is due to $F_{Y(1)}\in[0,1]$.
  Since we assume 
$\sqrt{N}\norm{\delta}_1\overset{\P}{\to}0$,
it follows
$\sup_{z\in\R}|R_1(z)|\overset{\P}{\to}0$
from \eqref{R_1:1}.
\end{proof}
\begin{remark}
  We want to comment on the box constraints of Problem~\ref{bw:1:primal}, that is,
 \begin{gather*}
      \left| 
      \frac{1}{N} 
      \left( 
      \sum_{i = 1}^{n} 
      w^\dagger_0(X_i)
      B_k(X_i)
      -
      \sum_{i=1}^{N} 
      B_k(X_i)
      \right)
    \right|
    \ 
    \le 
    \ 
    \delta_k
    \qquad
    \text{for all}\ 
    k \in \left\{ 1, \ldots, N \right\}
    \,.
  \end{gather*}
  Note, that the first sum goes over $\left\{ 1,\ldots,n \right\}$ while the second sum goes over $\left\{ 1,\ldots,N \right\}$.
  A second, equivalent version of the constraints is
  \begin{gather*}
      \left| 
      \frac{1}{N} 
      \left( 
      \sum_{i = 1}^{N} 
      T_i
      w^\dagger_0(X_i)
      B_k(X_i)
      -
      \sum_{i=1}^{N} 
      B_k(X_i)
      \right)
    \right|
    \ 
    \le 
    \ 
    \delta_k
    \qquad
    \text{for all}\ 
    k \in \left\{ 1, \ldots, N \right\}
    \,.
  \end{gather*}
  Now both sums go over $\left\{ 1,\ldots,N \right\}$ and the
  indicator of treatment $T_i$ takes care that in the first sum only the terms with $i\le n$ are effective. 
  Having this flexibility with the versions helps. I regard the first version as suitable for non-probabilistic computations as in Chapter~2, although $n$ is of course a random variable. On the other hand, the second version is more honest, exactly telling the dependence on the indicator of treatment. This version is useful in probabilistic computations. 

  Also we want to comment on the assumption on $\norm{\delta}$.
  Playing around with norm equivalences we discover that 
  $\sqrt{N}\norm{\delta}_1\overset{\P}{\to}0$ for $N\to \infty$ is the weakest
  (natural) assumption to
  control $R_1$.
  Indeed, other ways to continue the second row in \eqref{R_1:1} are
  \begin{gather*}
    (\,\cdots)
    \ 
  \le
    \ 
  \sqrt{N}
  \norm{\delta}_2
  \left( 
  \sum_{k=1}^{N} 
  \left( 
    \sup_{z\in\R}
  F_{Y(1)}(z|X_k)
  \right)^2
\right)^{1/2}
\ 
\le
\ 
N
  \norm{\delta}_2\,,
  \end{gather*}
  by the Cauchy-Schwarz inequality and
  $
  F_{Y(1)}\in [0,1]
  $,
or
\begin{gather*}
  (\,\cdots)
  \ 
  \le
  \ 
  \sqrt{N}
  \norm{\delta}_\infty
  \sum_{k=1}^{N} 
    \sup_{z\in\R}
  F_{Y(1)}(z|X_k)
  \ 
  \le
  \ 
  N^{3/2}
  \norm{\delta}_\infty
  \,.
\end{gather*}
Since $\delta\in \R^N$, however, it holds
\begin{gather*}
  \sqrt{N}\norm{\delta}_1
  \ 
  \le
  \ 
  N\norm{\delta}_2
  \ 
  \le
  \ 
  N^{3/2}\norm{\delta}_\infty
  \,.
\end{gather*}
With hindsight, the assumption 
$\sqrt{N}\norm{\delta}_1\overset{\P}{\to}0$ for $N\to \infty$ 
  also 
  suffices 
  to control the second (or first) occurrence of a term, that we control by assumptions on $\norm{\delta}$.
This is the term $I_2$ in \eqref{99909}, where we estimate
\begin{gather*}
  \inner{\delta}{\left| \Delta \right|}
  \ 
  =
  \ 
  \sum_{k=1}^{N} 
  \delta_k
  \left| \Delta_k \right|
  \ 
  \le
  \ 
  \norm{\delta}_1
  \norm{\Delta}_\infty
  \ 
  \le
  \ 
  \norm{\delta}_1
  \norm{\Delta}_2
  \ 
  \le
  \ 
  \norm{\delta}_1
  \varepsilon
  \ 
  \overset{\P}{\to}
  \ 
  0
  \quad
  \text{for}\ 
  N\to \infty
  \,.
\end{gather*}

\end{remark}


  \subsection{$R_2$}
    \begin{lemma}
  Assume
  \begin{align*}
    \sqrt{N}
    \sup_{z\in\R}
    \omega
    \left( 
      F_{Y(1)}(z|\cdot)
      ,h_N^d
    \right)
    \ 
    \to 
    \ 
    0
    \qquad
    \text{for}\ 
    N\to\infty
    \,.
  \end{align*}
  Then $\sup_{z\in\R}|R_2(z)|\overset{\P}{\to} 0$.
\end{lemma}
\begin{proof}
  \begin{align*}
    \sup_{z\in\R}
    \left| R_2(z) \right|
    &
    \  
    \le
    \  
        \sqrt{N}
        \sup_{z\in\R}
        \max_{i\in \left\{ 1,\ldots,N \right\}}
        \sum_{k=1}^{N}
            \left|
        B_k(X_i,X_1,\ldots,X_N)
        \cdot
        F_{Y(1)}(z|X_k)
            \ 
            -
            \ 
        F_{Y(1)}(z|X_i)
            \right|
            \\
            &
            \qquad
            \cdot
            \ 
    \frac{1}{N}
    \sum_{i=1}^{N} 
      \left| 
    T_i\cdot w^\dagger_i(X_i) 
    \ 
    -
    \ 
    1 
      \right|
  \end{align*}
  Note, that by Theorem~\ref{th:weights_constr}.\textit{(i)-(ii)}
  it holds
  \begin{align*}
    \frac{1}{N}
    \sum_{i=1}^{N} 
      \left| 
    T_i\cdot w^\dagger_i(X_i) 
    \ 
    -
    \ 
    1 
      \right|
      \ 
    \le
      \ 
    1
    \ 
    +
    \ 
    \frac{1}{N}
    \sum_{i=1}^{N} 
    T_i\cdot w^\dagger_i(X_i) 
    \ 
    =
    \ 
    2
    \,.
  \end{align*}
  The statement follows from Lemma~\ref{lem:basis_2}.\textit{(ii)}
\end{proof}
\begin{remark}
In the original paper \cite{Wang2019} the authors derive concrete learning rates for the weights and employ them in bounding this term. They obtain a multiplied learning rate, which is sufficiently fast. Their approach, however, calls for concrete learning rates of the weights. Arguably, the process of deriving such rates is the most complicated part of the paper. 
I found out, that we don't need concrete rates for the weights. 
Consistency of the weights is enough and gives us an (arbitrarily slow but sufficient) learning rate to establish the results.
We don't even need rates for the weights to control $R_2$.
They only play a role in bounding $R_3$. 
\end{remark}



  \subsection{$R_3$}
    \begin{lemma}
  \label{aa:mean:r3:lem:conv}
  It holds
  $\sup_{z\in\R}\left| R_3(z) \right|\overset{\P}{\to}0$.
\end{lemma}
\begin{proof}
  Let
  $z\in\R$.
    By Lemma~\ref{aa:mean:r3:lem:fz_E}
  it holds 
  \begin{align*}
    &
  f_z(T,X,Y(T))
  \ 
  \in
  \ 
  L^1(\P)\,, 
  \\
  &
  f_z(T,X,Y(T))
  \ 
  \perp
  \ 
  D_N
  \,,
  \\
  \E
  [
  &
  f_z(T,X,Y(T))
    |X
  ]
  \ 
  =
  \ 
  0
  \,.
  \end{align*}
  Thus, 
  it follows from Lemma~\ref{w.Z=0}
  \begin{gather*}
    \E
    \left[
      w_0^\dagger(X)
      \cdot
      f_z(T,X,Y(T))
    \right]
    \ 
    =
    \ 
    0
    \,.
  \end{gather*}
  Since
  \begin{gather*}
    \E
    \left[
      \frac{T}{\pi(X)}
      f_z(T,X,Y(T))
    \right]
    \ 
    =
    \ 
    \E
    \left[
      \frac{T}{\pi(X)}
      \left( 
        \mathbf{1}
        _{\left\{  Y(T)\,\le\,z \right\}}
        -
        F_{Y(1)}(z|X)
      \right)
    \right]
    \  
    =
    \ 
    0
  \end{gather*}
  by Lemma~\ref{ps_weights_lemma},
  it follows
  \begin{align*}
    \E
    \left[ 
      \left( 
      w_0^\dagger(X)
      -
      \frac{1}{\pi(X)}
      \right)
     \cdot
     f_z(T,X,Y(T))
    \right]
    \ 
    =
    \ 
  0
  \,.
  \end{align*}
  But then 
  \begin{align*}
    R_3(z)
    &
    \ 
    =
    \ 
  \frac{1}
  {
\sqrt{N}
  }
    \sum_{i=1}^{N} 
    \left[ 
    \left( 
    w_0^\dagger(X_i) 
    -
    \frac{1}{\pi(X_i)}
    \right)
    T_i
    \left( 
    \mathbf{1}{\left\{ Y_i \le z \right\}}
    -
  F_{Y(1)}(z|X_i)
    \right)
    \right]
    \ 
    =
    \ 
    \G_N
    \left( 
      \left( 
      w_0^\dagger
      -
      \frac{1}{\pi}
      \right)
     \cdot
     f_z
    \right)
    \,.
  \end{align*}
Let
  $g^\dagger$ 
  denote the stochastic process \eqref{ghost_function}, that is,
\begin{align*}
  g^\dagger
  (x)
  \ 
  :=
  \ 
  \mathbf{1}{
    \left\{ 
      \sup_{y\in A_N(x)}
      \left| 
      w_0^\dagger(y)
      -
      \frac{1}{\pi(y)}
      \right|
      \,
      \le
      \,
      \varepsilon_N
    \right\}
  }
  \left( 
    w_0^\dagger(x)
      -
      \frac{1}{\pi(x)}
  \right)
  \cdot
  \mathbf{1}
  \bigcup_{k=1}^n
  \left\{ 
    x=X_k
  \right\}
\end{align*}
for all $x\in\R^d$.
  If
  \begin{gather}
    \label{event}
    \left| 
    w_0^\dagger(X_i)- \frac{1}{\pi(X_i)} 
    \right|
    \ 
    \le
    \ 
    \varepsilon_N
    \qquad
    \text{for all}\ 
    i\in \left\{ 1,\ldots,n \right\}
  \end{gather}
  it holds
  for all $i\in \left\{ 1,\ldots,N \right\}$
  \begin{gather*}
    g^\dagger
    (X_i)
    \cdot
    f_z
    (T_i,X_i,Y_i(T_i))
    \ 
    =
    \ 
  \left( 
      w_0^\dagger(X_i)
      -
      \frac{1}{\pi(X_i)}
  \right)
    f_z
    (T_i,X_i,Y_i(T_i))
      \,.
  \end{gather*}
  Thus, if \eqref{event} holds, it follows
\begin{align*}
 R_3(z)
 \ 
 =
 \ 
 \G_N(g^\dagger \cdot f_z)
 \,.
\end{align*}
  It follows
\begin{align*}
    \P
    \left[ 
      \sup_{z\in\R}
     | 
    R_3(z)
    |
      \ge
      \varepsilon
    \right]
    &
    \ 
    \le
    \ 
    \P
    \left[ 
      \sup_{z\in\R}
     | 
    R_3(z)
    |
      \ge
      \varepsilon
      \ 
      \text{and}
      \ 
    | 
    w_0^\dagger(X_i)- 1/\pi(X_i)
    |
    \ 
    \le
    \ 
    \varepsilon_N
    \ \text{for all}\ i\in\{1,\ldots,n\}
    \right]
    \\
    &
    \quad
    +
    \ 
    \P
    \left[ 
    | 
    w_0^\dagger(X_i)- 1/\pi(X_i)
    |
    \ 
    >
    \ 
    \varepsilon_N
    \ 
    \ \text{for some}\ i\in\{1,\ldots,n\}
    \right]
    \\
    &
    \ 
    \le
    \ 
    \P
    \left[ 
      \norm{\G_N}^*_{\mathcal{F}_N\cdot\mathcal{F}}
      \ge
      \varepsilon
    \right]
    \ 
    +
    \ 
    \ 
    \P
    \left[ 
      \max_{i\in \left\{ 1,\ldots,n \right\}}
    | 
    w_0^\dagger(X_i)- 1/\pi(X_i)
    |
    \ 
    >
    \ 
    \varepsilon_N
    \right]
    \\
    &
    \ 
    \to
    \ 
    0
    \,.
\end{align*}
The convergence of the first term follows from
Lemma~\ref{aa:r3:lemma:1}.
The convergence of the second term follows from
Theorem~\ref{th:max_weight_cons}.

\end{proof}




\chapter{Discussion and Outlook}
  \section{Matching}
  \subsubsection*{Motivation}
The papers \cite{Wang2019,Wang2023} 
are closely related.
In \cite{Wang2019} - the paper this thesis is based on -
the authors study weighting methods.
In \cite{Wang2023} the authors propose (in similar style) a matching framework based a constrained convex optimization.
\subsubsection{Conjecture}
The extensions proposed in this thesis (or parts) can be applied to the matching framework of \cite{Wang2023}.
\subsubsection{Ideas/Brainstorming}
The constraints in the problem \cite[(2.1)]{Wang2023} are more complicated. Nevertheless, they rely on the notion of basis function.
While in \cite{Wang2019} the estimation objective is the expectation of potential outcomes, in \cite{Wang2023} it is the average treatment effect.
The structure of the proofs is similar - first reveal connection to the inverse propensity score and then employ it in the error analysis of the estimator.
\subsubsection{Organisation}
Get familiar with \cite{Wang2019,Wang2023}. Point out the differences and similarities.
Make the mathematical analysis of \cite{Wang2023} rigorous.
You can use ideas from this thesis.
Try to extend the matching framework - either with ides from this thesis or your own ideas.
\subsubsection{Next Step}
Read the introductions of \cite{Wang2019,Wang2023}. 



\section{Application of the Functional Delta Method}
  \subsubsection{Motivation}
Theorem~\ref{aa:mean:th}
immediately allows to apply the functional delta method \cite[§3.9]{vaart2013}, \cite[§20]{Vaart2000}.
This readily generates theoretic results for a large class of plug-in estimators. The plug-in estimators have not been tested before in practice.
\subsubsection{Conjecture}
A large class of plug-in estimators converges in distribution to a nice limiting process.
The estimators work well in practice.
\subsubsection{Ideas/Brainstorming}
A plethora of applications of the delta method to estimates of the distribution function are to be found in \cite{Vaart2000} and \cite{vaart2013}.
This includes Quantile estimation \cite[§21]{Vaart2000}\cite[§3.9.21/24]{vaart2013},
survival analysis via Nelson-Aalen and Kaplan-Meier estimator\cite[§3.9.19/31]{vaart2013},
Wilcoxon Test~\cite[§3.9.4.1]{vaart2013},
and much more.
\subsubsection{Organisation}
Understand the functional delta method.
Determine your interests. Pick an example and compute the limiting process.
If this is fun and successful try another example - or come up with a new plug-in estimator that works with the functional delta method.
Do a simulation study an evaluate the performance of the method in practice.
Start with a plain estimate of the distribution function.
If this works well, plug-in estimators might as well.
Apply the method to real world data.
\subsubsection{Next Step}
Read \cite[§20]{Vaart2000}.

\section{Bootstrapping} 
  \subsubsection{Motivation}
A very natural idea is to bootstrap from the weighted distribution
$
(w_i\cdot X_i)
$.
I discussed this with Jose Zubizaretta, one of the authors of \cite{Wang2019, Wang2023}.
He told me that testing in practice showed promising results.
To the best of my knowledge the theoretical properties of this particular weighted bootstrap wait to be studied.
\subsubsection{Conjecture}
Results similar to \cite[Theorem~23.5]{Vaart2000} holds for the weighted bootstrap.
\subsubsection{Ideas/Brainstorming}
A good starting point to become familiar with the asymptotic theory of bootstrap is \cite[§3.6]{vaart2013} and \cite[§23]{Vaart2000}.
For more details, a good starting point could be \cite{Barbe1995}.
The project seems to be challenging - maybe at PHD level.
One could get acquainted with the method of bootstrap by reading the (non-mathematical) introduction \cite{Efron1994}.

\section{Non-binary Treatment}
  In practice, there often exists multiple treatments.
For example, $T\in \left\{
  0,1,2
\right\}$, $T\in I\subset \mathbb{N}$ or even $T\in\R$. 
There exists a general notion of propensity score \cite{Hirano2005}.

\section{Different Basis Functions}
  \subsubsection{Motivation}
The introduction of partitioning estimates~\cite[§4]{Gyorfi2002} --- as done in this thesis --- was successful.
Thus the implementation of other local averaging regression techniques, such as kernel estimates~\cite[§5]{Gyorfi2002}
is promising. 
\subsubsection{Conjecture}
Similar results as of this thesis hold for basis functions of (boxed) kernel estimates~\cite[§5]{Gyorfi2002}.
They have good practical performance.
\subsubsection{Ideas/Brainstorming}
For boxed kernels it is likely easy to prove a lemma similar to Lemma~\ref{lem:basis_2}.
For kernels with unbounded support, such as gaussian kernels, this might be more difficult. 
Generally, the basis functions should approximate treatment and outcome model well (see \cite[Assumptions~1.6 \& 2.3]{Wang2019}).
Partitioning estimates work well in this thesis, because we can define concrete oracle parameters.
If concrete oracle parameters are not readily available, there might be theoretic results to rely on.


%  \section{Consistency of Optimal Solutions}
%  \subsection{Estimate of an Oracle Parameter by the Dual}
%  \begin{assumption}
  \label{aa:assumption:overlap}
  The propensity score function satisfies
  $\pi(x)\in(0,1)$ for all $x\in\mathcal{X}$.
\end{assumption}
\begin{assumption}
  \label{aa:assumption:ps_continuous}
  The propensity score function is continuous in $\mathcal{X}$.
\end{assumption}
\subsection*{My Contribution}
I found out, that consistency for the dual variable is enough to prove later results. This simplifies the proof. 
In \cite{Wang2019} they use a quadratic Taylor expansion to obtain learning rates. I found out, that a simpler mean value result for differentiable convex functions is sufficient to proof consistency. 
Since I work with partitioning estimates, I found a suitable oracle parameter.
I prove an (extended) lemma which is central but the details were omited.

Throughout this section we assume for all $N\in\mathbb{N}$ the existence of an 
optimal solution 
$(\lambda_0^\dagger,\lambda^\dagger)$
to Problem?
We define the oracle parameter $\lambda^*\in\R^N$ to be the vector with coordinates
\begin{gather}
  \label{oracle_1}
  \lambda^*_k
  \ 
  :=
  \ 
  f^{'}
  \left( 
    \frac{1}{\pi(X_k)}
  \right)
  \qquad
  \text{for all}\ 
  k\in \left\{ 1,\ldots,N \right\}
  \,,
\end{gather}
where $\pi(x)=\P[T=1|X=x]$ is the \textbf{propensity score} at $x\in\mathcal{X}$. Why this choice? 
%First, the $\lambda_0^\dagger$ part is unimportant. We need it to eliminate the same factor in
%\begin{gather}
%  w(x):=
%  (
%  f^{'}
%  )^{-1}
%  \left( \inner{B(x)}{\lambda^\dagger}+\lambda_0^\dagger \right)
%\end{gather}
%that is 
%\begin{gather}
%  \inner{B(x)}{\lambda^*}+\lambda_0^\dagger 
%  =
%\sum_{k=1}^{N} 
%B_k(x)
%  f^{'}
%  \left( 
%    \frac{1}{\pi(X_k)}
%  \right)
%  \,.
%\end{gather}
With hindsight, we need
for all $i\in \left\{ 1,\ldots,N \right\}$
\begin{gather}
\left| 
\inner{B(X_i)}
{\lambda^*}
  -
  f^{'}
  \left( 
    \frac{1}{\pi(X_i)}
  \right)
\right|
\ 
\le
\ 
      \omega
      \left( f^{'}\circ (x\mapsto1/x)\circ \pi,h_N \right)
      \,.
\end{gather}
Consequently, if $\pi$ is continuous and positive (not 0) on $\mathcal{X}$ and the width of the partition $h_N$ converges to 0, we get 
\begin{gather}
 \left| 
  \inner{B(X_i)}{\lambda^*}
  -
  f^{'}
  \left( 
    \frac{1}{\pi(X_k)}
  \right)
 \right| 
 \
 \to
 \ 
 0
 \qquad
 \text{almost surely.}
\end{gather}
We need this to bound the term $M_N$ in \eqref{c:first:1}.
The first big goal is to prove the following proposition.
\begin{proposition}
  \label{bw:cd:th}
Define the oracle parameter $\lambda^*$ as in \eqref{oracle_1}.
Furthermore, assume that the propensity score function is continuous and positive on $\mathcal{X}$.
Then for all $\varepsilon>0$ with probability going to 1 there exists an optimal solution
     $(\lambda^\dagger,\lambda_0^\dagger)$
     to Problem ? with
     $
     \norm
     {
       (\lambda^\dagger,\lambda_0^\dagger)
      -
      (\lambda^*,0)
     }_2
     \le
     \varepsilon
     $.
\end{proposition}
\begin{remark}
  In the analysis of the next sections will assume existence of an optimal solution to Problem?
  The purpose of Proposition~\ref{bw:cd:th} is therefore twofold. 
  First, it ensures the existence of an optimal solution with probability going to 1 as $N\to\infty$.
  This shows that the assumption of existence is (at least for large $N$) likely met.
  Nevertheless, it is beyond the scope of my thesis to investigate feasibility of Problem ? in more detail. 
  Second, if a solution exists it converges to the oracle parameter $\lambda^*$ in probability. This ensures the consistency of the weights for the inverse propensity score.
\end{remark}

We use a hint from the last display of~\cite[p.22]{Wang2019}.
The high-level idea is that the existence of the optimal solution and its proximity to the oracle parameter can be analysed by the objective function.
\begin{lemma}
  \label{bw:cd:lem}
  Let $m\in\mathbb{N}$ and
  $g \,:\, \R^m \to \overline{\R}$ 
  be convex.
  Then 
  for all $y \in \R^m$ and $\varepsilon>0$ 
    \begin{gather}
      \label{7060_0}
      \inf_{\norm{\Delta}=\varepsilon} g(y+\Delta) - g(y) \ge 0 \quad
    \end{gather}
    implies
    the existence of  
    a global minimum
    $
    y^* \in \,\R^m
    $
    of $g$
    satisfying
    $
      \norm{y^* - y}_2 \le \varepsilon
    $.
\end{lemma}
\begin{proof}
  Let $B$ be the euclidian ball in $\R^m$.
  Since 
  $
  y
  \,
  +
  \,
  \varepsilon
  B
  $
  is convex, it contains a 
  local minimum  
  of $g$.
  Suppose towards a contradiction that
  $
    y^* 
    \ 
    \in 
    \ 
  y
  \,
  +
  \,
  \varepsilon
  B
  $
  is a local minimum, but not a global one, and
  \eqref{7060_0} is true.
  Then it holds
  \begin{gather}
    \label{7060_3}
    g(x) < g(y^*)
    \quad
    \text{for some}\ 
    x 
    \in 
    \R^m 
    \setminus 
    \left( 
  y
  \,
  +
  \,
  \varepsilon
  B
    \right)
  \,.
  \end{gather}
  Furthermore, since 
  $
  y
  \,
  +
  \,
  \varepsilon
  B
  $ is compact and contains $y^*$,
  the line segment connecting 
  $y^*$ and $x$
  intersects the boundary of 
  $y + \mathcal{C}$, that is,
  there exist
  $
    \theta \in (0,1)
  $
  and 
  $
    \Delta_x
  $
  with 
  $
    \norm{\Delta_x}_2=\varepsilon
  $
  such that
  \begin{gather}
    \label{7060_4}
    \theta x + (1 - \theta) y^* = y + \Delta_x
    \,.
  \end{gather}
    It follows
    \begin{align}
      \label{7060_5}
      \begin{split}
      g(y^*)
      \le
      g(y)
      \le
      g(y + \Delta_x)
      &=
      g(
        \theta x + (1 - \theta) y^*
      )
      \\
      &\le
      \theta g(x)
      + 
      (1 - \theta)
      g(y^*)
      <
      g(y^*)
      ,
      \end{split}
    \end{align}
    which is a contradiction.
    The first inequality is due to
    $y^*$ being a local minimum of $g$ in
    $
  y
  \,
  +
  \,
  \varepsilon
  B
    $,
    the second inequality is due to  
    \eqref{7060_0} being true,
    the equality is due to \eqref{7060_4},
    the third inequality is due to the convexity of $g$
    and the strict inequality is due to \eqref{7060_3}.
    Thus every local minimum of $g$ in
    $
  y
  \,
  +
  \,
  \varepsilon
  B
    $
    is also a global minimum.
    %It follows the right-hand side of \eqref{7060_0}.
\end{proof}
\begin{remark}
  The hint from \cite[page 22]{Wang2019}
  states
  \eqref{7060_0}
  with strict inequality.
  They base their subsequent analysis on a quadratic Taylor expansion and aim to prove \eqref{7060_0} with strict inequality.
  I show, that this approach is inefficient.  
  To do that, I need Lemma~\ref{bw:cd:lem} exactly as stated - with $\ge$ in \eqref{7060_0}.
\end{remark}
Lemma~\ref{bw:cd:lem} motivates the following lemma.
 \begin{lemma}
   \label{bw:cd:lem2}
   Under the conditions of Theorem~\ref{bw:cd:th} it holds
   for all $\varepsilon>0$
\begin{gather}
   \P
   \left[ 
     \inf _ { 
       \norm{
         (
     \Delta
     ,
     \Delta_0
         )
 } 
= \varepsilon }
     G
     (
     \lambda^*
      +
      \Delta
      ,
     \Delta_0
     )
     -
     G
     (
     \lambda^*,
     0
     )
     \ge 
     0
   \right]
   \ 
   \to
   \ 
   1
   \qquad
   \text{for}
   \ 
   N\to\infty
   \,.
\end{gather}
 \end{lemma}
 \begin{proof}
  Recall the objective function $G$ of Problem? 
\begin{align*}
  G(\lambda,\lambda_0)
  &
  \ 
  =
  \ 
    \frac{1}{N}
\sum_{i=1} 
  ^N
  \left[ 
    \,
  T_i
  \cdot
  f^*
  \!
  \left( 
\lambda_0
+
\inner
{B(X_i)}
{\lambda}
  \right)
  \ 
-
\ 
  \left( 
\lambda_0
+
\inner
{B(X_i)}
{\lambda}
  \right)
  \,
  \right]
  \ 
+
\ 
\inner
{\delta}
{|\lambda|}
\,
\\
  &
\ 
=
\ 
g(\lambda,\lambda_0)
\ 
+
\ 
\inner
{\delta}
{|\lambda|}
\,
\end{align*}
with
  \begin{gather}
    g
    \ 
    := 
    \ 
    (\lambda,\lambda_0)
    \ 
    \mapsto
    \ 
    \frac{1}{N}
\sum_{i=1} 
  ^N
  \left[ 
    \,
  T_i
  \cdot
  f^*
  \!
  \left( 
\lambda_0
+
\inner
{B(X_i)}
{\lambda}
  \right)
  \ 
-
\ 
  \left( 
\lambda_0
+
\inner
{B(X_i)}
{\lambda}
  \right)
  \,
  \right]
  \,.
  \end{gather}
   Since we assume the convex conjugate $f^*$ of $f$ to be differentiable
   (it always convex),
   $g$ is differentiable convex function with gradient 
  \begin{gather*}
    \nabla
    g
    \ 
    =
    \ 
    (\lambda,\lambda_0)
    \ 
    \mapsto
    \ 
\frac{1}{N}
\sum_{i=1} 
  ^N
  \left[ 
    \,
  T_i
  \cdot
  (f^{'})^{-1}
  \!
  \left( 
\lambda_0
+
\inner
{B(X_i)}
{\lambda}
  \right)
  \ 
-
\ 
1
  \right]
  [
  B(X_i)^\top
  ,
  1
  ]^\top
  \,.
  \end{gather*}
It is well know that a differentiable convex functions $g$ satisfies
  \begin{gather*}
    g(x)
    \ 
    -
    \ 
    g(y)
    \ 
    \ge
    \ 
    \nabla
    g(y)^\top
    (x-y)
    \qquad 
    \text{for all}\ 
    x,y\,.
  \end{gather*}
Thus 
\begin{align}
  \label{c:1}
  \begin{split}
     &
   G
     (
     \lambda^*
      +
      \Delta
      ,
     \Delta_0
     )
     \ 
     -
     \ 
     G
     (
     \lambda^*,
     0
     )
         %%%%%%%%%%%%% 1 %%%%%%%%%%%%%%
     \\
     &
     \quad
     \ge
     -
     \frac{1}{N}
     \sum_{i=1}^{N} 
     \left[ 
       B(X_i)^\top,
       1
     \right]
     \cdot
     \begin{bmatrix}
       \Delta\\
       \Delta_0
     \end{bmatrix}
     \left( 
       1
       \ 
       -
       \ 
     T_i
     \cdot
     (f^{'})^{-1}
     \left( 
       \inner
       {B(X_i)}
       {\lambda^*}
     \right)
     \right)
     \\
     &
     \qquad
     +
     \ 
     \inner
     {\delta}
     {
       |\lambda^*+\Delta|
       -
       |\lambda^*|
     }
     \,.
   \end{split}
\end{align}

Next, we fix $
\tilde{\varepsilon}
>0
$
and establish in \eqref{c:1} the lower bound
$
-
\tilde{\varepsilon}
$
with probability going to $1$ as $N\to\infty$.
Then we conclude that this holds for all $\tilde{\varepsilon}>0$.
The measurability of
$
G
     (
     \lambda^*
      +
      \Delta
      ,
     \Delta_0
     )
     \ 
     -
     \ 
     G
     (
     \lambda^*,
     0
     )
$
will give us the lower bound 0 in \eqref{c:1} with probability going to $1$.

In \eqref{c:1} we control the \textbf{first term} by the law of large numbers 
and the \textbf{second term} by $\norm{\delta}_1\to 0$.
\subsection*{First Term}
We note, that by $\norm{B(x)}_2\le 1$ for all $x\in\mathcal{X}$ and the Cauchy-Schwarz inequality
it holds
\begin{gather}
  \label{c:first:0}
      \left[ 
       B(X_i)^\top,
       1
     \right]
     \cdot
     \begin{bmatrix}
       \Delta\\
       \Delta_0
     \end{bmatrix}
     \ 
     \lesssim
     \ 
     \norm{(\Delta,\Delta_0)}
     =\varepsilon
     \,.
\end{gather}
Next, we see that
\begin{align}
  \label{c:first:1}
  \begin{split}
  &
     \frac{1}{N}
     \sum_{i=1}^{N} 
     \left( 
       1
       \ 
       -
       \ 
     T_i
     \cdot
     (f^{'})^{-1}
     \left( 
       \inner
       {B(X_i)}
       {\lambda^*}
     \right)
     \right)
     \\
     &
     \ 
     \lesssim
     \ 
     \frac{1}{N}
     \sum_{i=1}^{N} 
     \left|
     1
     -
     \frac{T_i}{\pi(X_i)}
     \right|
      \ 
     +
      \ 
     \frac{1}{N}
     \sum_{i=1}^{N} 
     \left| 
        \inner
       {B(X_i)}
       {\lambda^*}
      \ 
        -
        \ 
        f^{'}
        \left( 
          \frac{1}{\pi(X_i)}
        \right)
     \right|
     \\
     &
     \ 
     =:
     \ 
     S_N
     \
     +
     \ 
     M_N
     \,.
\end{split}
\end{align}
With $
\tilde{\varepsilon}
>0
$
fixed previously,
we want to
establish the upper bound
$
\tilde{\varepsilon}
/
(2\varepsilon)
$
with probability going to $1$ as $N\to\infty$.
First, we bound $S_N$.
By the properties of conditional expectation it holds
\begin{gather*}
  \E
  \left[ 
    \frac{T}{\pi(X)}
  \right]
  =
  \E
  \left[ 
    \frac{\E[T|X]}{\pi(X)}
  \right]
  =1
  \,.
\end{gather*}
Also
\begin{gather}
  \E
  \left[ 
    \left| 
    1
    -
    \frac{T}{\pi(X)}
    \right|
  \right]
  \le
  1
  +
  \E
  \left[ 
    \frac{T}{\pi(X)}
  \right]
  =
  2
  \,.
\end{gather}
Thus Etemadi's ($\mathcal{L}_1$ version) strong law of large numbers (cf.\cite[Theorem~5.17]{Klenke2020}) applies
to $S_N$, that is,
$S_N \le 
\tilde{\varepsilon}
/
(4\varepsilon)
$
with probability going to 1.

Next, we bound $M_N$.
Recall that $\sum_{k=1}^{N}B_k(x)=1$ for all $x\in\mathcal{X}$. Thus
\begin{align}
  \label{part_trick}
  \begin{split}
  &
\left| 
        \inner
       {B(X_i)}
       {\lambda^*}
       \ 
        -
        \ 
        f^{'}
        \left( 
          \frac{1}{\pi(X_i)}
     \right)
\right|
      \\
      &
      \ 
      =
      \ 
      \left| 
      \sum_{k=1}^{N} 
      B_k(X_i)
      \cdot
        f^{'}
        \left( 
          \frac{1}{\pi(X_k)}
          \right)
      \ 
      -
      \ 
        f^{'}
        \left( 
          \frac{1}{\pi(X_i)}
     \right)
      \right|
      \\
      &
      \ 
      =
      \ 
      \left| 
      \sum_{k=1}^{N} 
      B_k(X_i)
      \left( 
        f^{'}
        \left( 
          \frac{1}{\pi(X_k)}
     \right)
     -
        f^{'}
        \left( 
          \frac{1}{\pi(X_i)}
     \right)
      \right)
      \right|
      \\
      &
      \ 
      \le
      \ 
      \sum_{k=1}^{N} 
      \frac{1_{\left\{ X_k\in A_N(X_i) \right\}}}
      {
        \sum_{j=1}^{N} 
1_{\left\{ X_j\in A_N(X_i) \right\}}
      }
      \left| 
        f^{'}
        \left( 
          \frac{1}{\pi(X_k)}
     \right)
     -
        f^{'}
        \left( 
          \frac{1}{\pi(X_i)}
     \right)
      \right|
      \\
      &
      \ 
      \le
      \ 
      \omega
      \left( f^{'}\circ (x\mapsto1/x)\circ \pi,h_N \right)
      \ 
      \to
      \ 
      0
      \,,
\end{split}
\end{align}
where $\omega$ is the modulus of continuity.
The convergence to 0 is due to the $f^{'}$ being continuous, $\pi(x)\in(0,1)$ for all $x\in\mathcal{X}$, the (assumed) continuity of $\pi$ and $h_N\to 0$ for $N\to\infty$.
We conclude, that $
M_N\le
\tilde{\varepsilon}
/
(4\varepsilon)
$
with probability going to 1.

This establishes the desired bound of 
$
\tilde{\varepsilon}/(2\varepsilon)
$
in \eqref{c:first:1}.
Together with \eqref{c:first:0}
we conclude that the \textbf{first term} 
in
\eqref{c:1}
is bounded below by
$
-
\tilde{\varepsilon}/2
$
with probability going to $1$ as $N\to\infty$.
\subsection*{Second Term}
It holds
\begin{gather*}
  |x+y|-|x|\ge
  -|y|
  \qquad
  \text{for all}\ 
  x,y\in\R
  \,.
\end{gather*}
Since
$\delta\ge 0$
we get
\begin{align*}
  &
     \inner
     {\delta}
     {
       |\lambda^*+\Delta|
       -
       |\lambda^*|
     }
     \\
     &
     \ 
     \ge
     \ 
     -
     \inner{\delta}
     {|\Delta|}
     \ 
     \ge
     \ 
     -
     \norm{\delta}_1
     \norm{\Delta}_\infty
     \ 
     \ge
     \ 
     -
     \norm{\delta}_1
     \norm{(\Delta,\Delta_0)}_2
     \ 
     \ge
     \ 
     -
     \norm{\delta}_1
     \varepsilon
     \ 
     \ge
     \ 
     -
     \tilde{\varepsilon}/2
     \,,
\end{align*}
with probability going to $1$ as $N\to \infty$.
The convergence is due to $\norm{\delta}_1$ converging to $0$ in probability.
\subsection*{Conclusion}
With the analysis of the \textbf{first} and \textbf{second term} in
\eqref{c:1} we conclude
\begin{gather}
  \label{c:7}
  G
     (
     \lambda^*
      +
      \Delta
      ,
     \Delta_0
     )
     \ 
     -
     \ 
     G
     (
     \lambda^*,
     0
     )
     \ge
     -
     \tilde{\varepsilon}
\end{gather}
with probability going to $1$ as $N\to \infty$.
But this holds for all $\tilde{\varepsilon}>0$.
A closer look reveals that
$
  G
     (
     \lambda^*
      +
      \Delta
      ,
     \Delta_0
     )
     \ 
     -
     \ 
     G
     (
     \lambda^*,
     0
     )
$ 
is measurable.
Indeed, this holds because $X$,$T$,$B(X)$ and $\lambda^*$ are measurable and $f^*$ is continuous.
Since \eqref{c:7} holds true for all $\tilde{\varepsilon}>0$, it thus follows 
\begin{gather}
  G
     (
     \lambda^*
      +
      \Delta
      ,
     \Delta_0
     )
     \ 
     -
     \ 
     G
     (
     \lambda^*,
     0
     )
     \ge
     0
\end{gather}
with probability going to $1$ as $N\to \infty$.
But this holds for all 
$
(\Delta,\Delta_0)
$
with 
$
\norm{
(\Delta,\Delta_0)
}
=\varepsilon
$. Thus
\begin{gather}
   \inf _ { 
       \norm{
         (
     \Delta
     ,
     \Delta_0
         )
 } 
= \varepsilon }
     G
     (
     \lambda^*
      +
      \Delta
      ,
     \Delta_0
     )
     -
     G
     (
     \lambda^*,
     0
     )
     \ge 
     0
\end{gather}
with probability going to $1$ as $N\to \infty$.
Finally, we see, that this holds for all $\varepsilon>0$. This finish the proof.
 \end{proof}

\begin{proof}
  \emph{(Proposition~\ref{bw:cd:th})}
An immediate consequence of Lemma~\ref{bw:cd:lem} is
that for all $\varepsilon>0$ it holds
\begin{gather*}
   \P
   \left[ 
     \text{There exists an optimal solution $(\lambda^\dagger,\lambda_0^\dagger)$ to Problem? with}
     \ 
     \norm
     {
       (\lambda^\dagger,\lambda_0^\dagger)
      -
      (\lambda^*,0)
     }_2
     \le
     \varepsilon
   \right]
   \\
   \
   \ge
   \ 
   \P
   \left[ 
     \inf _ { 
       \norm{
         (
     \Delta
     ,
     \Delta_0
         )
 } 
= \varepsilon }
     G
     (
     \lambda^*
      +
      \Delta
      ,
     \Delta_0
     )
     -
     G
     (
     \lambda^*,
     0
     )
     \ge 
     0
   \right]
   \,.
 \end{gather*}
 Applying Lemma~\ref{bw:cd:lem2} finishes the proof.
\end{proof}


%  \subsection{Estimate of the Inverse Propensitiy Score by the Weights}
%  The following theorem is an easy consequence of Theorem~\ref{bw:cd:th}.
\begin{ftheorem}
  Consider the weights function defined by
  \begin{gather}
    w(x)
    :=
    (
    f^{'}
    )^{-1}
    \left( 
      \inner{B(x)}{\lambda^\dagger}
      +
      \lambda_0^\dagger
    \right)
    \qquad
    \text{for all}\ 
    x\in\mathcal{X}
    \,.
  \end{gather}
  Under the conditions of Theorem~\ref{bw:cd:th} 
  it holds
  $w(X)\ 
  \overset{\P}{\to}
  \ 
  1/\pi(X)
  $
\end{ftheorem}
\begin{proof}
  For all $\varepsilon>0$ it holds
\begin{align}
  \begin{split}
  \left| 
  w(X)
  -
  \frac{1}{\pi(X)}
  \right|
  &
  \ 
  =
  \ 
  \left| 
  (f^{'})^{-1}
  \left( 
    \inner{B(X)}
    {\lambda^\dagger}
    +
    \lambda_0^\dagger
  \right)
  -
  \frac{1}{\pi(X)}
  \right|
  \\
  &
  \ 
  \lesssim
  \ 
  \left| 
  \inner
  {B(X)}
  {\lambda^\dagger-\lambda^*}
  \right|
  +
  \left| 
    \inner{B(X)}
    {\lambda^\dagger}
    +
    \lambda_0^\dagger
    -
    f^{'}
    \left( 
  \frac{1}{\pi(X)}
    \right)
  \right|
  \\
  &
  \ 
  \lesssim
  \ 
  \norm{\lambda^\dagger-\lambda^*}_2
  +
  \left| 
  \sum_{i=1}^{N} 
  B_k(X)
  \cdot
    f^{'}
    \left( 
  \frac{1}{\pi_k}
    \right)
    -
    f^{'}
    \left( 
  \frac{1}{\pi(X)}
    \right)
  \right|
  \\
  &
  \ 
  \le
  \ 
  \frac{\varepsilon}{2}
+
  \frac{\varepsilon}{2}
  \le
  \varepsilon
  \,,
\end{split}
\end{align}
with probability going to $1$ as $N\to\infty$.
\end{proof}

  \subsection*{Gaussian Bridge}
  \subsection{Tools and Assumptions}

To control the remaining terms $R_3$ and $R_4$ we use empirical processes. We introduce the concept and the results we need in the next paragraphs. 
For an introduction to empirical processes see \cite{Vaart2000}. More advanced techniques are in \cite{vaart2013}.

Let 
$
  \left( 
    \Omega,
    \mathcal{A},
    \P
  \right)
$
be a probability space,
$
  \left( 
    \mathcal{Z},
    \Sigma
  \right)
$
a measurable space, and 
$
  \xi_1,\ldots,\xi_N
  :
  \left( 
    \Omega,
    \mathcal{A},
    \P
  \right)
  \to
  \left( 
    \mathcal{Z},
    \Sigma
  \right)
$
a sample 
of independent and identically-distributed
random variables
with probability distribution $\P_{\!\xi}$.
A family $\mathcal{F}$ of measurable functions 
$
  f:
  \left( 
    \mathcal{Z},
    \Sigma
  \right)
    \to
  \left( 
    \R,
    \mathcal{B}(\R)
  \right)
$
induces a stochastic process by
\begin{gather}
  f
  \ 
  \mapsto
  \ 
  \G_N f 
  \ 
  :=
  \ 
  \frac{1}{\sqrt{n}}
  \sum_{i=1}^{N} 
  (
    f(\xi_i)
    -
    \E_\xi[f]
  \,.
\end{gather}
We call this the  \textbf{empirical process} $\G_N$ indexed by $\mathcal{F}$.
We define the (random) norm
\begin{gather}
  \norm{\G_n}_\mathcal{F}
  :=
  \sup_
        { f \in \mathcal{F}}
        \left|
          \G_N f
        \right|
        .
\end{gather}
\begin{remark}
We stress that 
$
  \norm{\G_n}_\mathcal{F}
$
often ceases to be measurable, even in simple situations~\cite[page 3]{vaart2013}.
To deal with this, we introduce the notion of \textbf{outer expectation} $\E^*$, that is,
\begin{gather}
  \E^*[Z]
  \ 
  :=
  \ 
    \inf
  \left\{ 
    \E[U]
  \ 
  \lvert
  \ 
    U\ge Z,
    \ 
    U:
  \left( 
    \Omega,
    \mathcal{A},
    \P
  \right)
  \to 
  \left( 
    \overline{\R},
    \mathcal{B}(\overline{\R})
  \right)
  \text{measurable and}
  \ 
  \E[U]<\infty
  \right\}
  \,.
\end{gather}
In our application the technical difficulties halt at this point, because we only consider $Z$ with $\E^*[Z]<\infty$. Then there exists a smallest measurable function $Z^*$ dominating $Z$ with
$\E^*[Z]=\E[Z^*]$. Thus, we may assume $Z$ to be measurable in this regard.
\begin{definition}
  We call a class 
  $\mathcal{F}$ of measurable functions 
$\P$-Donsker
if the sequence of processes 
$\left\{ G_N f \colon f\in\mathcal{F}\right\}$
converges in
$l^\infty(\mathcal{F})$
to a tight limit process.
\end{definition}

\begin{theorem}
  Every class $\mathcal{F}$ of measurable functions 
  with
  $
    J
    _{[]}
    (
    1
    ,
    \mathcal{F}
    ,
    L_2(\P)
    )
    <\infty
  $
  is

  $\P$-Donsker, that is,
  the sequence of processes 
$\left\{ G_N f \colon f\in\mathcal{F}\right\}$
  converges 
  in
$l^\infty(\mathcal{F})$
to a Gaussian process with mean 0 and covariance function given by
\begin{gather}
  \mathrm{Cov}(f,g)
  :=
  \E[fg]-\E[f]\E[g]
  \,.
\end{gather}
\end{theorem}
\begin{proof}
  \cite[Theorem~19.5]{Vaart2000}
\end{proof}



\end{remark}
In our application we need concentration inequalities for 
$
  \norm{\G_n}_\mathcal{F}
$.
One easy way is to use maximal inequalities for the expectation together with Markov's inequality. There are also Bernstein-like inequalities for empirical processes. We need to introduce more concepts.

Given two functions $\underline{f}$ and $\overline{f}$, the bracket
$[\underline{f},\overline{f}]$ 
is the set of all functions $f$ with 
$\underline{f}\le f \le \overline{f}$.
We define a
$(\varepsilon, L^{r}(\P))$-bracket
to be a bracket
$[\underline{f},\overline{f}]$ with
$\norm{\overline{f}-\underline{f}}_{ L^r(\P)}< \varepsilon$.
The bracketing number 
$
N_{[\,]}(\varepsilon, \mathcal{F}, L^r(\P))
$
is 
the minimum number of 
$(\varepsilon, L^{r}(\P))$-brackets needed to cover $\mathcal{F}$.
For most classes $\mathcal{F}$ the bracketing number grows to infinity for $\varepsilon\to 0$.
To measure the speed of convergence we introduce the bracketing integral
\begin{gather}
     J
    _{[]}
    (
    \delta
    ,
    \mathcal{F}
    ,
    L_r(\P)
    )
    =
  \int_0^{\delta}
      \sqrt{
        \log 
      N_{[\,]}
\left( \varepsilon, \mathcal{F}_N, L^r(\P) \right)
    }
    \,
    d\varepsilon
    \,.
\end{gather}
An envelope function $F$ of a class $\mathcal{F}$ satisfies 
$|f(x)|\le F(x)< \infty$ for all $f\in\mathcal{F}$ and all $x$.
\begin{theorem}
  \label{th:max_ineq}
  \emph{(Maximal inequality)}
  For any class $\mathcal{F}$ of measurable functions with envelope function $F,$
  \begin{gather}
    \E^*
    _\P
    [
    \norm{
      \G
      _n
      }
      _\mathcal{F}
    ]
    \lesssim
    J
    _{[]}
    (
    \norm{
      F
    }
    _{ L^2(\P)}
    ,
    \mathcal{F}
    ,
    L_2(\P)
    )
    .
  \end{gather}
\end{theorem}
\begin{proof}
  \cite[Corollary~19.35]{Vaart2000}
\end{proof}

\begin{lemma}
  \label{markov_max_lemma}
  Let $(\mathcal{H}_N)$ be a sequence of measurable function classes with envelope functions $(H_N)$.
  If
  \begin{gather}
    J_{[\, ]}
    \left( 
    \norm{F_N}_{ L^2(\P)}
    ,
    \mathcal{H}_N
    ,
     L^2(\P)
    \right)
    \ 
    \to
    \ 
    0
    \qquad
    \text{for}
    \ 
    N
    \to
    \infty
  \end{gather}
  it holds 
  $
  \norm{\G_N}^*_{\mathcal{H}_N}\overset{\P}{\to}0
  $.
\end{lemma}
\begin{proof}
  By Markov's inequality and Theorem~\ref{th:max_ineq} it holds for all $\varepsilon>0$
  \begin{align*}
    &
    \P
    [
  \norm{\G_N}^*_{\mathcal{H}_N}
  \ge
  \varepsilon
    ]
    \\
    &
    \ 
    \le
    \ 
    \varepsilon^{-1}
    \E
    [
  \norm{\G_N}^*_{\mathcal{H}_N}
    ]
    \ 
    =
    \ 
    \varepsilon^{-1}
    \E^*
    [
  \norm{\G_N}_{\mathcal{H}_N}
    ]
    \ 
    \lesssim
    \ 
    \varepsilon^{-1}
    J_{[\, ]}
    \left( 
    \norm{H_N}_{ L^2(\P)}
    ,
    \mathcal{H}_N
    ,
     L^2(\P)
    \right)
    \ 
    \to
    \ 
    0
  \end{align*}
  for $N\to\infty$. 
\end{proof}
A technical lemma for products of function classes.

Define the product of two function classes
\begin{gather*}
  \mathcal{F}\cdot \mathcal{G}
  :=
  \left\{ 
    f\cdot g
    \colon
    f\in\mathcal{F},
    g\in\mathcal{G}
  \right\}\,.
\end{gather*}
\begin{lemma}
  \label{lem_prod_br}
  Let
  $\mathcal{F}$ and $\mathcal{G}$ be two function classes 
  with envelope functions $F$ and $G$ satisfying
  $\norm{F}_\infty,\norm{G}_\infty\le 1$.
  For all $\varepsilon>0$ and all $r\in [1,\infty)$ it holds
  \begin{gather*}
    N_{[\,]}(2\varepsilon,\mathcal{F}\cdot\mathcal{G},\mathrm{L}_r(\P))
    \
    \le
    \ 
    N_{[\,]}(\varepsilon,\mathcal{F},\mathrm{L}_r(\P))
    \cdot
    N_{[\,]}(\varepsilon,\mathcal{G},\mathrm{L}_r(\P))
    \,.
  \end{gather*}
\end{lemma}
\begin{proof}
  Let $f\in\mathcal{F}$ and $g\in\mathcal{G}$.
  We can choose two 
  $(\varepsilon,L^r(\P))$
  brackets
  $[\underline{f},\overline{f}]$
  and
  $[\underline{g},\overline{g}]$
  containing $f$ and $g$ with 
  $\norm{\underline{f}}_\infty,\norm{\overline{f}}_\infty\le\norm{F}_\infty\le 1$
  and
  $\norm{\underline{g}}_\infty,\norm{\overline{g}}_\infty\le\norm{G}_\infty\le 1$.
  We the get an 
  $(2\varepsilon,L^r(\P))$
  $[\underline{h},\overline{h}]$
  bracket, containing $f\cdot g$, by
\end{proof}

\begin{lemma}
  \label{aa:r3:lemma:1}
  Let
$(\varepsilon_N)\subset(0,1]$
be 
a decreasing sequence
with $\varepsilon_N\to 0$ for $N\to\infty$ 
and let Assumption~\ref{aa:assumption:1} hold true
for a sequence of function classes $\mathcal{F}_N$.
Then
\begin{gather*}
  J_{[\,]}(
\norm{F_N}_{L^2(\P)}
,\mathcal{F}_N\cdot\mathcal{F},\mathrm{L}_2(\P))
  \to 0
  \quad
  \text{and}
  \quad
  \norm{\G_N}^*_{\mathcal{F}_N\cdot\mathcal{F}}\overset{\P}{\to}0
  \qquad
  \text{for}\ 
  N\to\infty
  \,.
\end{gather*}
\end{lemma}
\begin{proof}
  By Assumption~\ref{aa:assumption:1}
  and Lemma~\ref{aa:mean:l:br} it holds
for some $k<2$
\begin{gather*}
\norm{F_N}_{L^2(\P)}
\ 
\le
\ 
\varepsilon_N
\quad
\text{and}
\quad
  \log
  N_{[\,]}(\varepsilon,\mathcal{F}_N,\mathrm{L}_2(\P))
  \ 
  \lesssim
  \ 
  \left( 
  \frac{1}{\varepsilon}
  \right)^k
  \quad
  \text{for all}
  \ 
  N\in\mathbb{N}
  \,,
\end{gather*}
and
  \begin{gather*}
    N_{[\,]}
    (
    \varepsilon
    ,
    \mathcal{F}, L^2(\P))
    \ 
    \lesssim
    \ 
    \left( 
      \frac{1}{\varepsilon}
    \right)^2
    \qquad
    \text{for all}
    \ 
    \varepsilon>0
    \,.
  \end{gather*}
  Since $\mathcal{F}_N$ and $\mathcal{F}$ have envelope function smaller 1, we can apply Lemma~\ref{lem_prod_br} to get
  \begin{gather*}
  \log
  N_{[\,]}(\varepsilon,\mathcal{F}_N\cdot\mathcal{F},\mathrm{L}_2(\P))
  \ 
  \lesssim
  \ 
  \left( 
  \frac{1}{\varepsilon}
  \right)^k
  +
  \log
  (1/\varepsilon)
  \ 
  \lesssim
  \ 
  \left( 
  \frac{1}{\varepsilon}
  \right)^k
  \quad
  \text{for all}\ 
  \varepsilon>0
  \,.
  \end{gather*}
  Since 
  $k/2\in(0,1)$
  it holds
\begin{align*}
  J_{[\,]}(
\norm{F_N}_{L^2(\P)}
,\mathcal{F}_N\cdot\mathcal{F},\mathrm{L}_2(\P))
  &
  \ 
=
  \ 
\int_0^{
\norm{F_N}_{L^2(\P)}
}
\sqrt{
  \log
  N_{[\,]}(\varepsilon,\mathcal{F}_N\cdot\mathcal{F},\mathrm{L}_2(\P))
}
\,d\varepsilon
\\
&
\ 
\lesssim
\ 
\int_0^{
  \varepsilon_N
}
  \left( 
  \frac{1}{\varepsilon}
\right)^{k/2}
\,d\varepsilon
\\
&
\ 
=
\ 
\frac{
\varepsilon_N^{1-k/2}
}{1-k/2}
\ 
\to 0
\ 
\qquad
\text{for}
\ 
N\to\infty
\,.
\end{align*}
The second statement follows from Lemma~\ref{markov_max_lemma}
for 
$\mathcal{H}_N:=\mathcal{F}_N\cdot\mathcal{F}$ and
$H_N:=F_N$.
\end{proof}

\begin{lemma}
  \label{ps_weights_lemma}
  Let
  $
  g_1\colon
  \mathcal{X}\to\R
  $
  and
  $
  g_2\colon
  \mathcal{Y}\to\R
  $
  be a measurable functions.
  It holds
  \begin{gather*}
    \E
    \left[
    \frac{T}{\pi(X)}
    g_1(X)
    \right]
    \ 
    =
    \ 
    \E
    \left[
    g_1(X)
    \right]
    \,.
  \end{gather*}
  If Assumption~\ref{aa:assumption:treatment_str_ign} holds true, then
  \begin{gather*}
    \E
    \left[
    \frac{T}{\pi(X)}
    g_2(Y(T))
    \right]
    \ 
    =
    \ 
    \E
    \left[
    f(Y(1))
    \right]
    \,.
  \end{gather*}
\end{lemma}

\begin{assumption}
  \label{aa:assumption:1}
  For any decreasing sequence
  $(\varepsilon_N)$ with $\varepsilon_N\to 0$ for $N\to\infty$,
there exists a sequence of (measurable) function classes
$(\mathcal{F}_N)$
with envelope functions
$(F_N)$,
satisfying 
for some $k<2$
\begin{gather*}
\norm{F_N}_{L^2(\P)}
\ 
\le
\ 
\varepsilon_N
\quad
\text{and}
\quad
  \log
  N_{[\,]}(\varepsilon,\mathcal{F}_N,\mathrm{L}_2(\P))
  \ 
  \lesssim
  \ 
  \left( 
  \frac{1}{\varepsilon}
  \right)^k
  \quad
  \text{for all}
  \ 
  N\in\mathbb{N}
  \,,
\end{gather*}
such that for all $(\lambda,\lambda_0)\in\R^{N+1}$ and for all $N\in\mathbb{N}$ the function
\begin{gather}
  \label{ghost_function}
  \mathcal{X}
  \ 
  \to
  \ 
  \R
  \,,
  \qquad
  x
  \ 
  \mapsto
  \ 
  \mathbf{1}{
    \left\{ 
      \sup_{y\in A_N(x)}
      \left| 
      w(y,\lambda,\lambda_0)
      -
      \frac{1}{\pi(y)}
      \right|
      \,
      \le
      \,
      \varepsilon_N
    \right\}
  }
  \left( 
      w(x,\lambda,\lambda_0)
      -
      \frac{1}{\pi(x)}
  \right)
\end{gather}
is contained in $\mathcal{F}_N$.
\end{assumption}
The next Lemma shows that mild assumptions on the regularity of the inverse propensity score implies Assumption~\ref{aa:assumption:1}.
To this end, we need notation from \cite[§2.7.1]{vaart2013}.
To this end, let for any vector $k\in\mathbb{N}_0^d \ (d\in\R)$
\begin{gather*}
  D^k
  \ :=\ 
  \frac
  {\partial^{\norm{k}_1}}
  {
    \partial^{k_1}x_1
    \cdots
    \partial^{k_d}x_d
  }
  \,,
\end{gather*}
and let $\lfloor a \rfloor$ be the greatest integer smaller than $a>0$.
For $\alpha>0$, a bounded set 
$\mathcal{X}\subset\R^d\ (d\in\mathbb{N})$
and
$M>0$, we define $C^\alpha_M(\mathcal{X})$ to be the space of all continuous functions $f\colon \mathcal{X}\to\R$ with
\begin{gather*}
  \max_{\norm{k}_1\le \alpha}\sup_{x\in\mathcal{X}}
  \left| D^k f(x) \right|
  \ 
  +
  \ 
  \max_{\norm{k}_1=\lfloor \alpha \rfloor}\sup_{x,y}
  \frac
  {
  \left|
  D^k f(x) 
  -
  D^k f(y) 
  \right|
  }
  {
    \norm{x-y}_2^{\alpha-\lfloor \alpha \rfloor}
  }
  \
  \le
  \ 
  M
  \,.
\end{gather*}
where the suprema in the second term are taken over all $x,y$ in the interior of $\mathcal{X}$ with $x\neq y$.
Furthermore, let
\begin{gather*}
  \mathcal{X}^1
  :=
  \left\{ 
    y\in\R^d
    \colon
    \norm{x-y}_2 <1
    \ 
    \text{for some}\ x\in\mathcal{X}
  \right\}
  \,.
\end{gather*}
We will employ the following result.
\begin{lemma}
  \label{vdv_coro}
  Let $\mathcal{P}=\left\{ A_1,A_2,\ldots \right\}$ be a partition of $\R^d$ into bounded, convex sets with non-empty interior, and let $\mathcal{F}$ be a class of functions $f\colon\R^d\to\R$ such that the restrictions $\mathcal{F}_{|A_j}$ belong to $C^\alpha_{M_j}(A_j)$
  for all $j\in\mathbb{N}$.
  Then there exists a constant $K$, depending only on $\alpha$, $V$, $r$ and $d$
  such that
  \begin{gather}
    \label{667}
    \log
    N_{[\,]}
    (
    \varepsilon
    ,
    \mathcal{F}
    ,
    L^r(\mathbf{Q})
    )
    \le
    K
    \left( \frac{1}{\varepsilon} \right)^V
    \left( 
      \sum_{j=1}^{\infty}
      \lambda(A_j^1)^{r/(V+r)}
      M_j^{Vr/(V+r)}
      \mathbf{Q}(A_j)^{V/(V+r)}
    \right)
    ^{(V+r)/r}
  \end{gather}
  for every $\varepsilon>0$, $V\ge d/\alpha$, and probability measure $\mathbf{Q}$.
\end{lemma}
\begin{proof}
  \emph{\cite[Corollary~2.7.4]{vaart2013}}
\end{proof}
\pagebreak
\begin{lemma}
  Let $(\mathcal{P}_N)$ denote a sequence of qubic partitions
  $\mathcal{P}_N=\left\{ A_{N,1},A_{N,2},\ldots \right\}$ 
  of $\R^d$ 
  with decreasing width $(h_N)\subset(0,1]$ such that $h_N\to 0$ for $N\to\infty$.
  Furthermore, assume that there exists
  $\alpha>d/2$, where $\mathcal{X}\subseteq \R^d$, such
  that for 
  $V:=d/\alpha$
 and for all 
$
(j,N)\in\mathbb{N}^2
$
there exists 
$M_{N,j}\ge 1$ such that 
\begin{gather}
  \label{897}
  \frac{1}{\pi(\cdot)}
  \in C^\alpha_{M_{N,j}}(A_{N,j})
  \quad
  \text{and}
  \quad
  \sum_{j=1}^{\infty} 
  M_{N,j}^{2V/(V+2)}
  \P
  [
  X\in A_{N,j}
  ]^{V/(V+2)}
  \ 
  \lesssim
  \ 
  1
  \,.
\end{gather}
It then holds the statement of Assumption~\ref{aa:assumption:1}.
\end{lemma}
\begin{proof}
  We want to employ Lemma~\ref{vdv_coro}. 
  To do this, the crucial observation is, that
  \begin{gather*}
    w(\cdot,\lambda,\lambda_0)
    \quad
    \text{is constant on each cell}\ 
    A_N\in\mathcal{P}_N
    \,.
  \end{gather*}
  Thus, the regularity of the function \eqref{ghost_function}
  on each cell $A_N\in\mathcal{P}_N$ is decided by 
  $1/\pi(\cdot)$.
  Indeed, \eqref{ghost_function} is either 0 if the threshold of $\varepsilon_N$ is exceeded somewhere in the cell, or has the form constant-minus-smooth-function.
In any case, it is continuous and bounded by $\varepsilon_N$.
All its derivatives are 0 (if the threshold is exceeded) or are governed by $1/\pi(\cdot)$.
Thus, it follows from \eqref{897}
\begin{gather}
  \eqref{ghost_function}
  \in C^\alpha_{M_{N,j}}(A_{N,j})
  \quad
  \text{and}
  \quad
  \sum_{j=1}^{\infty} 
  M_{N,j}^{2V/(V+2)}
  \P
  [
  X\in A_{N,j}
  ]^{V/(V+2)}
  \ 
  \lesssim
  \ 
  1
  \,.
\end{gather}
To bound the right-hand-side in \eqref{667} we note
that 
$
\lambda(A_{N,j})=h^d_N
$ 
and thus
$
\lambda(A_{N,j}^1)\lesssim 1
$
for all $(j,N)\in\mathbb{N}^2$.
Thus
\begin{gather*}
  \sum_{j=1}^{\infty} 
  \lambda(A_{N,j}^1)^{2/(V+2)}
  M_{N,j}^{2V/(V+2)}
  \P
  [
  X\in A_{N,j}
  ]^{V/(V+2)}
  \ 
  \lesssim
  \ 
  1
  \,.
\end{gather*}
We get $\eqref{ghost_function}\in\mathcal{F}_N$, where 
$
\mathcal{F}_N
$
restricted to $A_{N,j}$ is 
$
  C^\alpha_{M_{N,j}}(A_{N,j})
$
and satisfies the requirements of Lemma~\ref{vdv_coro}.
Since $V=d/\alpha \in (0,2)$ by $\alpha>d/2$, 
applying Lemma~\ref{vdv_coro} finishes the proof.
\end{proof}
In the next examples we show the concrete applications of this Lemma.
\begin{example}
Let $d\in\R$ and assume $\pi(\cdot)$ follows a logistic regression model. Then there exist $(\beta_0,\beta)\in\R^{d+1}$ such that
\begin{gather*}
  \pi(x)
  \ 
  =
  \ 
  \frac{1}
  {
    1
    +
    \exp
    \left( 
      -
        \beta_0
        -
        \inner{\beta}{x}
    \right)
  }
  \quad
  \text{and}
  \quad
  \frac
  {1}
  {
  \pi(x)
  }
  \ 
  =
  \ 
    1
    +
    \exp
    \left( 
      -
        \beta_0
        -
        \inner{\beta}{x}
    \right)
    \quad
    \text{for all}\ 
    x\in\R^d
    \,.
\end{gather*}
Clearly, by the smoothness of the exponential function, for all 
$\alpha>0$ there exist $(M_{N,j})\ge 1$ such that
$
  1/\pi(\cdot)\in C^\alpha_{M_{N,j}}(A_{N,j})
$.
Assume $\#\mathcal{X}<\infty$, that is, $X$ can take only finitely many values with positive probability.
We write
\begin{gather*}
  J_N
  :=
  \left\{ 
    j\in\mathbb{N}
    \colon
    \P[X\in A_{N,j}]>0
  \right\}
  \,.
\end{gather*}
It holds
$\#J_N\le \# \mathcal{X}<\infty$.
Thus, the following maximum is attained
\begin{gather*}
  \max_{j\in J_N} M_{N,j}
  =:M^*_N
  \,.
\end{gather*}
But the partitions increasingly better fit the support of $X$. Thus
$M^*_N$ is decreasing in $N$, that is,  $\infty>M^*_1\ge M^*_N$.
It follows
\begin{align*}
  \sum_{j=1}^{\infty} 
  M_{N,j}
  ^{2V/(V+2)}
  \P
  [
  X\in A_{N,j}
  ]^{V/(V+2)}
  \ 
  \le
  \ 
  \left( 
  M^*_1
  \right)
  ^{2V/(V+2)}
  \cdot
  \# J_N
  \ 
  \lesssim
  \ 
  1
  \,.
\end{align*}
\end{example}
The next question is if we can extend this to countable sets
$\mathcal{X}$. To focus our attention, we assume (the best case)
$M_{N,j}=1$.
\begin{example}
  We give three examples. Let $\propto$ denote the equal-up-to-a-constant order.
  Without loss of generality we assume $\mathcal{X}=\mathbb{N}$.
  First consider
  for some $\varepsilon\in(0,1)$
  \begin{gather*}
    \P[X=k]
    \propto
    k^{-(1+\varepsilon)}
    \,.
  \end{gather*}
  This (barely) is a distribution. But scaling with $V/(V+2)\in(0,1)$ we cause trouble. Indeed, by $V<2$ it holds
  \begin{gather*}
    (
    1+\varepsilon
    )
    \cdot
    \frac{V}{V+2}
    <1
    \,,
  \end{gather*}
  and thus
  \begin{align*}
  \sum_{j=1}^\infty
  \P
  [
  X\in A_{N,j}
  ]^{V/(V+2)}
  &
  \ 
  =
  \ 
  \lim_{N\to\infty}
  \sum_{j\in J_N}
  \P
  [
  X\in A_{N,j}
  ]^{V/(V+2)}
  \\
  &
  \ 
  =
  \ 
  \sum_{k=1}^{\infty} 
  \P
  [
  X=k
  ]^{V/(V+2)}
  \\
  &
  \ 
  \propto
  \ 
  \sum_{k=1}^{\infty} 
  \left( 
  \frac{1}{k}
  \right)
  ^{(1+\varepsilon)\cdot V/(V+2)}
  =\infty
  \,.
  \end{align*}
  On the other hand, the same arguments applied for
  \begin{gather*}
    \P[X=k]
    \propto
    k^{-((V+2)/V+\varepsilon)}
    \,,
  \end{gather*}
  with $\varepsilon>0$ yield
\begin{gather*}
  \sum_{j=1}^\infty
  \P
  [
  X\in A_{N,j}
  ]^{V/(V+2)}
  \lesssim 1
  \,.
\end{gather*}
Finally, let $X$ be Poisson distributed with parameter $\lambda\in(0,1)$.
Then
\begin{align*}
  \sum_{j=1}^\infty
  \P
  [
  X\in A_{N,j}
  ]^{V/(V+2)}
  &
  \ 
  =
  \ 
  \sum_{k=0}^{\infty} 
  \P
  [
  X=k
  ]^{V/(V+2)}
  \\
  &
  \ 
  =
  \ 
  e^{-\lambda\cdot V/(V+2)}
  \sum_{k=0}^{\infty} 
  \left( 
    \frac{\lambda^k}{k\text{!}}
  \right)
  ^{V/(V+2)}
  \\
  &
  \ 
  \le
  \ 
  \sum_{k=0}^{\infty} 
\left( 
    \lambda
  ^{V/(V+2)}
\right)^k
  =
  \frac{1}
  {
    1-
    \lambda
  ^{V/(V+2)}
  }
  \lesssim 1
  \,.
  \end{align*}

\end{example}
In the next example we show, that 
\begin{gather*}
  \sum_{j=1}^\infty
  \P
  [
  X\in A_{N,j}
  ]^{V/(V+2)}
  \to
  \infty
\end{gather*}
for all continuous distributions of $X$.
\begin{example}
  Let $f_X$ be the probability density of $X$. 
  Then there exists a compact set $K\subset\mathcal{X}\subset \R^d$, such that
  $
  \inf_{x\in K}f_X(x)
  >0
  $. Since $\mathcal{P}_N$ are cubic partitions, it holds for 
  \begin{gather*}
    I_N
    :=
    \left\{ 
      i\in\mathbb{N}\colon
      A_{N,i}\subset K
    \right\}
    \qquad
    \text{that}
    \qquad
    \bigcup_{i\in I_N}A_{N,i}\nearrow K
    \,.
  \end{gather*}
Thus
\begin{align*}
  \sum_{i=1}^\infty
  \P
  [
  X\in A_{N,i}
  ]^{V/(V+2)}
  &
  \ 
  \ge
  \ 
  \sum_{i\in I_N}
  \P
  [
  X\in A_{N,i}
  ]^{V/(V+2)}
  \\
  &
  \ 
  \ge
  \ 
  \inf_{x\in K}f_X(x)
  ^{V/(V+2)}
  \cdot 
  h_N^{d\cdot(V/(V+2)-1)}
  \sum_{i\in I_N}
  \lambda
  \left( 
  A_{N,i}
  \right)
  \\
  &
  \ 
  \to
  \ 
  \infty
  \,.
\end{align*}
This follows from
$
  \sum_{i\in I_N}
  \lambda
  \left( 
  A_{N,i}
  \right)
  \to 
  \lambda(K)>0
$,
  $
  \inf_{x\in K}f_X(x)
  >0
  $,
  $V/(V+2)-1<0$ and $h_N\to 0$.

\end{example}


  $
(\lambda^\dagger,\lambda_0^\dagger)
$
We adapt the error decomposition in \cite[page 27]{Wang2019} 
to estimates of the distribution function $F_{Y(1)}$ of $Y(1)$, 
that is,
\begin{gather}
  F_{Y(1)}
  \ 
  \colon
  \ 
  \R
  \ 
  \to
  \ 
  [0,1]
  \, 
  , 
  \qquad
  z
  \ 
  \mapsto
  \ 
  \P
  [
  Y(1)
  \le
  z
  ]
  \,.
\end{gather}
\begin{assumption}
  \label{aa:assumption:1}
  For any decreasing sequence
  $(\varepsilon_N)$ with $\varepsilon_N\to 0$ for $N\to\infty$,
there exists a sequence of (measurable) function classes
$(\mathcal{F}_N)$
with envelope functions
$(F_N)$,
satisfying 
for some $k<2$
\begin{gather*}
\norm{F_N}_{L^2(\P)}
\ 
\le
\ 
\varepsilon_N
\quad
\text{and}
\quad
  \log
  N_{[\,]}(\varepsilon,\mathcal{F}_N,\mathrm{L}_2(\P))
  \ 
  \lesssim
  \ 
  \left( 
  \frac{1}{\varepsilon}
  \right)^k
  \quad
  \text{for all}
  \ 
  N\in\mathbb{N}
  \,,
\end{gather*}
such that for all $(\lambda,\lambda_0)\in\R^{N+1}$ and for all $N\in\mathbb{N}$ the function
\begin{gather}
  \label{ghost_function}
  \mathcal{X}
  \ 
  \to
  \ 
  \R
  \,,
  \qquad
  x
  \ 
  \mapsto
  \ 
  \mathbf{1}_{
    \left\{ 
      \sup_{y\in A_N(x)}
      \left| 
      w(y,\lambda,\lambda_0)
      -
      \frac{1}{\pi(y)}
      \right|
      \,
      \le
      \,
      \varepsilon_N
    \right\}
  }
  \left( 
      w(x,\lambda,\lambda_0)
      -
      \frac{1}{\pi(x)}
  \right)
\end{gather}
is contained in $\mathcal{F}_N$.
\end{assumption}
\begin{remark}
  We can derive Assumption~\ref{aa:assumption:1} from regularity assumptions on the inverse propensity score 
  $1/\pi(\cdot)$ and the distribution of $X$.
  To this end, we want to employ \cite[Corollary~2.7.4]{vaart2013}.
  To do this, the crucial observation is, that
  \begin{gather*}
    w(\cdot,\lambda,\lambda_0)
    \quad
    \text{is constant on each cell}\ 
    A_N\in\mathcal{P}_N
    \,.
  \end{gather*}
  Thus, the regularity of the function \eqref{ghost_function}
  on each cell $A_N\in\mathcal{P}_N$ is decided by 
  $1/\pi(\cdot)$.
  Thus, we assume that there exists
  $\alpha>d/2$, where $\mathcal{X}\subseteq \R^d$, such
  that for 
  $V:=d/\alpha$.
 and for all 
$
(j,N)\in\mathbb{N}^2
$
there exists 
$M_{N,j}\ge \varepsilon_N$ such that 
\begin{gather*}
  \frac{1}{\pi(\cdot)}
  \in C^\alpha_{M_{N,j}}(A_{N,j})
  \quad
  \text{and}
  \quad
  \sum_{j=1}^{\infty} 
  M_{N,j}^{2V/(V+2)}
  \P
  [
  X\in A_{N,j}
  ]^{V/(V+2)}
  \ 
  \lesssim
  \ 
  1
  \,.
\end{gather*}
Then the function \ref{ghost_function}, restricted to $A_{N,j}$, is in 
$
  C^\alpha_{M_{N,j}}(A_{N,j})
$
for all 
$
(j,N)\in\mathbb{N}^2
$.
Then \cite[Corollary~2.7.4]{vaart2013} gives us the statement 
in
Assumption~\ref{aa:assumption:1}.
\end{remark}
\begin{ftheorem}
  \label{aa:mean:th}
  Under conditions 
the stochastic process
\begin{gather}
    \sqrt{N}
    \left( 
  \frac{1}{N}
    \sum_{i=1}^{n} 
    w_i^\dagger
    \mathbf{1}
    _{\left\{ Y_i\,\le\, z \right\}}
    \ 
    -
    \ 
    F_{Y(1)}(z)
    \right)
    _{z\in\R}
    \,
  \end{gather}
  converges in
  $l^\infty(\R)$
  to a Gaussian process with mean 0 and covariance
\begin{align}
  \label{cov:lp}
 \begin{split}
  &
  \mathbf{Cov}
  (z_1,z_2)
  \\
  &
  =\ 
  \E
  \left[ 
 \frac{
 F_{Y(1)}(z_1 \land z_2\,|\,X)
}{\pi(X)}
\ 
-
\ 
 \frac{1-\pi(X)}{\pi(X)}
 F_{Y(1)}(z_1|X)
 \cdot
 F_{Y(1)}(z_2|X)
  \right]
  \\
  &
  \qquad 
 -
 \ 
 F_{Y(1)}(z_1)
 \cdot
 F_{Y(1)}(z_2)
 \end{split}
\end{align}
\end{ftheorem}

\pagebreak
\begin{lemma}
  \label{aa:mean:lemma_decomp}
  It holds
  \begin{gather}
  \sqrt{N}
\left( 
    \frac{1}{N}
    \sum_{i=1}^{n} 
    w(X_i)
    \mathbf{1}_{\left\{ Y_i \le z \right\}}
    \ 
    -
    \ 
    F_{Y(1)}(z)
    \right)
    _{z\in\R}
    \ 
    =
    \ 
    R_1
    \ 
    +
    \ 
    R_2
    \ 
    +
    \ 
    R_3
    \ 
    +
    \ 
    R_4
  \end{gather}
  with
\begin{align*}
  R_1
  &
  \ 
  :=
  \ 
  \sqrt{N}
  \sum_{k=1}^{N} 
  \left[ 
  \frac{1}{N}
  \left( 
    \sum_{i=1}^{n} 
    w(X_i)
    B_k(X_i)
    -
    \sum_{i=1}^{N} 
    B_k(X_i)
  \right)
  F_{Y(1)}(z|X_k)
  \right]
  _{z\in\R}
  \,,
  %%%% 1 %%%%
  \\
  R_2
  &
  \
  :=
  \ 
  \sqrt{N}
    \sum_{i=1}^{N} 
    \left[ 
  \frac{
    T_i\cdot w(X_i) -1 }{N}
    \left( 
  F_{Y(1)}(z|X_i)
    \ 
    -
    \ 
    \sum_{k=1}^{N} 
    B_k(X_i)
    \cdot
  F_{Y(1)}(z|X_k)
    \right)
    \right]
  _{z\in\R}
  \,,
  %%%% 2 %%%%
  \\
  R_3
  &
  \
  :=
  \ 
  \sqrt{N}
  \left( 
  \frac{1}{N}
    \sum_{i=1}^{N} 
    \left[ 
    T_i
    \left( 
    w(X_i) 
    -
    \frac{1}{\pi(X_i)}
    \right)
    \left( 
    \mathbf{1}_{\left\{ Y_i \le z \right\}}
    -
  F_{Y(1)}(z|X_i)
    \right)
    \right]
  \right)
  _{z\in\R}
  \,,
  %%%% 3 %%%%
  \\
  R_4
  &
  \
  :=
  \ 
  \sqrt{N}
  \left( 
  \frac{1}{N}
    \sum_{i=1}^{N} 
    \frac{T_i}{\pi(X_i)}
    \left( 
    \mathbf{1}_{\left\{ Y_i \le z \right\}}
    -
  F_{Y(1)}(z|X_i)
    \right)
    \ 
    +
    \ 
    \left( 
  F_{Y(1)}(z|X_i)
    -
  F_{Y(1)}(z)
    \right)
  \right)
  _{z\in\R}
  \,.
  \end{align*}
\end{lemma}
\nopagebreak
\begin{proof}
  We fix $z\in\R$.
  It holds
  \begin{align*}
    &
    \frac{1}{N}
    \sum_{i=1}^{N} 
    w^\dagger(X_i)
    \cdot
    T_i
    \cdot
    \mathbf{1}_{\left\{ Y_i(T_i)\, \le\, z \right\}}
    \\
    &
    \ 
    =
    \ 
    \frac{1}{N}
    \sum_{i=1}^{N} 
    \left( 
    w^\dagger(X_i)
    -
    \frac{1}{\pi(X_i)}
    \right)
    T_i
    \cdot
    \mathbf{1}_{\left\{ Y_i(T_i)\, \le\, z \right\}}
    \\
    &
    \quad 
    +
    \ 
    \frac{1}{N}
    \sum_{i=1}^{N} 
    \frac{T_i}{\pi(X_i)}
    \mathbf{1}_{\left\{ Y_i(T_i)\, \le\, z \right\}}
    \\
    &
    \ 
    =
    \ 
    \frac{1}{N}
    \sum_{i=1}^{N} 
    \left( 
    w^\dagger(X_i)
    -
    \frac{1}{\pi(X_i)}
    \right)
    T_i
    \left( 
    \mathbf{1}_{\left\{ Y_i(T_i)\, \le\, z \right\}}
    -
    F_{Y(1)}(z|X_i)
    \right)
    \\
    &
    \quad 
    +
    \ 
    \frac{1}{N}
    \sum_{i=1}^{N} 
    \frac{T_i}{\pi(X_i)}
    \left( 
    \mathbf{1}_{\left\{ Y_i(T_i)\, \le\, z \right\}}
    -
    F_{Y(1)}(z|X_i)
    \right)
    \\
    &
    \qquad 
    +
    \ 
    \frac{1}{N}
    \sum_{i=1}^{N} 
    w^\dagger(X_i)\cdot T_i\cdot
    F_{Y(1)}(z|X_i)
    \\
    &
    \ 
    =
    \ 
    R_3(z)/\sqrt{N}
    \\
    &
    \quad 
    +
    \ 
    \frac{1}{N}
    \sum_{i=1}^{N} 
    \frac{T_i}{\pi(X_i)}
    \left( 
    \mathbf{1}_{\left\{ Y_i(T_i)\, \le\, z \right\}}
    -
    F_{Y(1)}(z|X_i)
    \right)
    +
    \left( 
    F_{Y(1)}(z|X_i)
    -
    F_{Y(1)}(z)
    \right)
    \\
    &
    \qquad
    +
    \ 
    \frac{1}{N}
    \sum_{i=1}^{N} 
    \left( 
    w^\dagger(X_i)\cdot T_i
    \ 
    -
    \ 
    1
    \right)
    F_{Y(1)}(z|X_i)
    \\
    &
    \quad\qquad
    +
    \ 
    F_{Y(1)}(z)
    \\
    &
    \ 
    =
    \ 
    R_3(z)/\sqrt{N}
    \\
    &
    \quad
    +
    \ 
    R_4(z)/\sqrt{N}
    \\
    &
    \qquad
    +
    \ 
    \frac{1}{N}
    \sum_{i=1}^{N} 
    \left( 
    w^\dagger(X_i)\cdot T_i
    \ 
    -
    \ 
    1
    \right)
    \left( 
    F_{Y(1)}(z|X_i)
    -
    \sum_{k=1}^{N} 
    B_k(X_i)
    \cdot
  F_{Y(1)}(z|X_k)
    \right)
    \\
    &
    \quad\qquad
    +
    \ 
    \frac{1}{N}
    \sum_{i=1}^{N} 
    \left( 
    w^\dagger(X_i)\cdot T_i
    \ 
    -
    \ 
    1
    \right)
    \sum_{k=1}^{N} 
    B_k(X_i)
    \cdot
  F_{Y(1)}(z|X_k)
    \\
    &
    \qquad\qquad
    +
    \ 
    F_{Y(1)}(z)
\\
    &
    \ 
    =
    \ 
    R_3(z)/\sqrt{N}
    \\
    &
    \quad
    +
    \ 
    R_4(z)/\sqrt{N}
    \\
    &
    \qquad
    +
    \ 
    R_2(z)/\sqrt{N}
    \\
    &
    \quad\qquad
    +
    \ 
    \sum_{k=1}^{N} 
    \frac{1}{N}
    \sum_{i=1}^{N} 
    \left( 
    w^\dagger(X_i)\cdot T_i
    B_k(X_i)
    \ 
    -
    \ 
    B_k(X_i)
    \right)
    \cdot
  F_{Y(1)}(z|X_k)
    \\
    &
    \qquad\qquad
    +
    \ 
    F_{Y(1)}(z)
    \\
    &
    \ 
    =
    \ 
    \left( 
R_3(z)
    \ 
    +
    \ 
    R_4(z)
    \ 
    +
    \ 
    R_2(z)
    \ 
    +
    \ 
    R_1(z)
    \right)
    /\sqrt{N}
    \ 
    +
    \ 
    F_{Y(1)}(z)
    \,.
  \end{align*}
  This holds for all $z\in\R$.
  Multiplying with $\sqrt{N}$ yields the result.
\end{proof}
Let $F_{Y(1)}(z|x):=\P[Y(1)\le z|X=x]$ denote a conditional version of the distribution function of $Y(1)$ at $x\in\mathcal{X}$.
We also need the propensity score
$
  \pi(x):=\P[T=1|X=x]
$
and the weights function
$
  w(x)
  :=
  (
  f^{'}
  )^{-1}
  \left( 
    \inner{B(x)}{\lambda^\dagger}
    +
    \lambda_0^\dagger
  \right)
$.
\begin{lemma}
  \label{aa:mean:l:r1}
Let the weights function $w$ satisfy the box constraints in 
Problem~\ref{bw:1:primal} and 
$\sqrt{N}\norm{\delta}_1\overset{\P}{\to}0$.
Then it holds
$\sup_{z\in\R}|R_1(z)|\overset{\P}{\to}0$.
  \end{lemma}
\begin{proof}
%  By Theorem~\ref{dual_solution_th}
%  it holds $w_i^\dagger=w(X_i)$ for $i\in \left\{ 1,\ldots,n \right\}$, that is, for $i\le n$ we can identify $w(X_i)$ with the optimal
%  solution to 
%  problem~\ref{bw:1:primal}. 
%  Thus the constraints of the problem apply.
%   Let's bound $R_1$.
It holds
  \begin{align}
    \label{R_1:1}
    \begin{split}
    \sup_{z\in\R}
    \left| 
    R_1(z)
    \right|
    &
    \ 
    =
    \ 
    \sup_{z\in\R}
    \left| 
  \sqrt{N}
  \sum_{k=1}^{N} 
  \left[ 
  \frac{1}{N}
  \left( 
    \sum_{i=1}^{n} 
    w(X_i)
    B_k(X_i)
    -
    \sum_{i=1}^{N} 
    B_k(X_i)
  \right)
  F_{Y(1)}(z|X_k)
  \right]
    \right|
    \\
    &
    \ 
    \le
    \ 
  \sqrt{N}
  \sum_{k=1}^{N} 
  \left| 
  \frac{1}{N}
  \left( 
    \sum_{i=1}^{n} 
    w(X_i)
    B_k(X_i)
    -
    \sum_{i=1}^{N} 
    B_k(X_i)
  \right)
  \right|
    \sup_{z\in\R}
  F_{Y(1)}(z|X_k)
  \\
    &
    \ 
    \le
    \ 
  \sqrt{N}
  \norm{\delta}_1
    \end{split}
  \end{align}
  The last inequality is due to $F_{Y(1)}\in[0,1]$ and the assumption that $(w(X_i))$ satisfies the box constraints of Problem~\ref{bw:1:primal}.
  Since we assume 
$\sqrt{N}\norm{\delta}_1\overset{\P}{\to}0$
it holds
$\sup_{z\in\R}|R_1(z)|\overset{\P}{\to}0$.
\end{proof}
\begin{remark}
  We want to comment on the box constraints of Problem~\ref{bw:1:primal}, that is,
 \begin{gather*}
      \left| 
      \frac{1}{N} 
      \left( 
      \sum_{i = 1}^{n} 
      w(X_i)
      B_k(X_i)
      -
      \sum_{i=1}^{N} 
      B_k(X_i)
      \right)
    \right|
    \ 
    \le 
    \ 
    \delta_k
    \qquad
    \text{for all}\ 
    k \in \left\{ 1, \ldots, N \right\}
    \,.
  \end{gather*}
  Note, that the first sum goes over $\left\{ 1,\ldots,n \right\}$ while the second sum goes over $\left\{ 1,\ldots,N \right\}$.
  A second, equivalent version of the constraints is
  \begin{gather*}
      \left| 
      \frac{1}{N} 
      \left( 
      \sum_{i = 1}^{N} 
      T_i
      w(X_i)
      B_k(X_i)
      -
      \sum_{i=1}^{N} 
      B_k(X_i)
      \right)
    \right|
    \ 
    \le 
    \ 
    \delta_k
    \qquad
    \text{for all}\ 
    k \in \left\{ 1, \ldots, N \right\}
    \,.
  \end{gather*}
  Now both sums go over $\left\{ 1,\ldots,N \right\}$ and the
  indicator of treatment $T_i$ takes care that in the first sum only the terms with $i\le n$ are effective. 
  Having this flexibility with the versions helps. I regard the first version as suitable for non-probabilistic computations, although $n$ is of course a random variable. On the other hand, the second version is more honest, exactly telling the dependence on the indicator of treatment. This version is useful in probabilistic computations. 

  Also we want to comment on the assumption on $\norm{\delta}$.
  Playing around with norm equivalences we discover that 
  $\sqrt{N}\norm{\delta}_1\overset{\P}{\to}0$ for $N\to \infty$ is the weakest
  (natural) assumption to
  control $R_1$.
  Indeed, other ways to continue the second row in \eqref{R_1:1} are
  \begin{gather*}
    (\,\cdots)
    \ 
  \le
    \ 
  \sqrt{N}
  \norm{\delta}_2
  \left( 
  \sum_{k=1}^{N} 
  \left( 
    \sup_{z\in\R}
  F_{Y(1)}(z|X_k)
  \right)^2
\right)^{1/2}
\ 
\le
\ 
N
  \norm{\delta}_2\,,
  \end{gather*}
  by the Cauchy-Schwarz inequality and
  $
  F_{Y(1)}\in [0,1]
  $,
or
\begin{gather*}
  (\,\cdots)
  \ 
  \le
  \ 
  \sqrt{N}
  \norm{\delta}_\infty
  \sum_{k=1}^{N} 
    \sup_{z\in\R}
  F_{Y(1)}(z|X_k)
  \ 
  \le
  \ 
  N^{3/2}
  \norm{\delta}_\infty
  \,.
\end{gather*}
Since $\delta\in \R^N$, however, it holds
\begin{gather*}
  \sqrt{N}\norm{\delta}_1
  \ 
  \le
  \ 
  N\norm{\delta}_2
  \ 
  \le
  \ 
  N^{3/2}\norm{\delta}_\infty
  \,.
\end{gather*}
With hindsight, the assumption 
$\sqrt{N}\norm{\delta}_1\overset{\P}{\to}0$ for $N\to \infty$ 
  also 
  suffices 
  to control the second (or first) occurrence of a term, that we control by assumptions on $\norm{\delta}$.
This is the \textbf{second term} of \eqref{c:1}, where we estimate
\begin{gather*}
  \inner{\delta}{\left| \Delta \right|}
  \ 
  =
  \ 
  \sum_{k=1}^{N} 
  \delta_k
  \left| \Delta_k \right|
  \ 
  \le
  \ 
  \norm{\delta}_1
  \norm{\Delta}_\infty
  \ 
  \le
  \ 
  \norm{\delta}_1
  \norm{\Delta}_2
  \ 
  \le
  \ 
  \norm{\delta}_1
  \varepsilon
  \ 
  \overset{\P}{\to}
  \ 
  0
  \quad
  \text{for}\ 
  N\to \infty
  \,.
\end{gather*}

\end{remark}
  \begin{lemma}
    \label{aa:mean:l:r2}
    Let the conditions of Theorem~\ref{aa:weights:th} hold true.
    Furthermore assume, that the width of the partitioning estimate $h_N$ and a conditional version of the distribution function of $Y(1)$ satisfy
\begin{gather*}
  \sqrt{N}
  \sup_{z\in\R}
  \omega
  \left( 
    F_{Y(1)}(z|\cdot)
    ,h_N
  \right)
  \to
  0
  \qquad
  \text{for}\ 
  N\to \infty
  \,,
\end{gather*}
where $\omega$ is the modulus of continuity.
Then it holds
$\sup_{z\in\R}|R_2(z)|\overset{\P}{\to}0$.
  \end{lemma}
\begin{proof}
\begin{align*}
&
  \sup_{z\in\R}
  \left| 
  R_2(z)
  \right|
  \\
  &
  \ 
  \le
  \ 
  \sqrt{N}
    \sum_{i=1}^{N} 
    \left[ 
  \frac{
    T_i\cdot w(X_i) -1 }{N}
  \sup_{z\in\R}
    \left| 
  F_{Y(1)}(z|X_i)
    \ 
    -
    \ 
    \sum_{k=1}^{N} 
    B_k(X_i)
    \cdot
  F_{Y(1)}(z|X_k)
    \right|
    \right]
    \\
    &
  \ 
    \le
  \ 
    \sqrt{N}
  \sup_{z\in\R}
  \omega
  \left( 
    F_{Y(1)}(z|\cdot)
    ,h_N
  \right)
  \sum_{i=1}^{N} 
  \frac{
    T_i\cdot w(X_i) +1 }{N}
    \\
    &
  \ 
    =
  \ 
    2
    \sqrt{N}
  \sup_{z\in\R}
  \omega
  \left( 
    F_{Y(1)}(z|\cdot)
    ,h_N
  \right)
  \,.
\end{align*}
The equality is due to 
\begin{gather}
  1
  \ 
  =
  \ 
\frac{1}{N}\sum_{i=1}^{n}w_i^\dagger
  \ 
  =
  \ 
\frac{1}{N}\sum_{i=1}^{n}w(X_i)
  \ 
=
  \ 
\frac{1}{N}\sum_{i=1}^{N}T_iw(X_i)
\,,
\end{gather}
that is, $w(X_i)$ satisfy the second constraint of Problem~\ref{bw:1:primal}.
The second inequality follows from 
$\sum_{k=1}^{N}B_k(X)=1$ and the convexity of the absolute value. 
Indeed,
\begin{align*}
&
  \sup_{z\in\R}
  \left| 
  F_{Y(1)}(z|X_i)
  -
  \sum_{k=1}^{N} 
  B_k(X_i)
  \cdot
  F_{Y(1)}(z|X_k)
  \right|
  \\
  &
  \ 
  \le
  \ 
  \sum_{k=1}^{N} 
  \frac{\mathbf{1}_{\left\{ X_k\in A_N(X_i) \right\}}}
  {
    \sum_{j=1}^{N} 
\mathbf{1}_{\left\{ X_j\in A_N(X_i) \right\}}
  }
  \sup_{z\in\R}
  \left| 
  F_{Y(1)}(z|X_i)
  -
  F_{Y(1)}(z|X_k)
  \right|
  \\
  &
  \ 
  \le
  \ 
  \sup_{z\in\R}
  \omega
  \left( 
    F_{Y(1)}(z|\cdot)
    ,h_N
  \right)
  \,.
\end{align*}

Since we assume
\begin{gather*}
  \sqrt{N}
  \sup_{z\in\R}
  \omega
  \left( 
    F_{Y(1)}(z|\cdot)
    ,h_N
  \right)
  \to
  0
  \qquad
  \text{for}\ 
  N\to \infty
  \,,
\end{gather*}
it follows
$\sup_{z\in\R}|R_2(z)|\overset{\P}{\to}0$.
\end{proof}
\begin{remark}
In the original paper \cite{Wang2019} the authors derive concrete learning rates for the weights and employ them in bounding this term. They obtain a multiplied learning rate, which is sufficiently fast. Their approach, however, calls for concrete learning rates of the weights. Arguably, the process of deriving such rates is the most complicated part of the paper. 
I found out, that we don't need concrete rates for the weights. 
Consistency of the weights is enough and gives us an (arbitrarily slow but sufficient) learning rate to establish the results.
We don't even need rates for the weights to control $R_2$.
They only play a role in bounding $R_3$.

We also want to comment on the assumption
\begin{gather*}
  \sqrt{N}
  \sup_{z\in\R}
  \omega
  \left( 
    F_{Y(1)}(z|\cdot)
    ,h_N
  \right)
  \to
  0
  \qquad
  \text{for}\ 
  N\to \infty
  \,,
\end{gather*}
I decided to keep this more general (and abstract) assumption, althogh
there are many (more concrete, yet stronger) assumptions on the regularity of
$
    F_{Y(1)}(z|\cdot)
$
and the convergence speed of $h_N$.
If for example 
$
    F_{Y(1)}(z|\cdot)
$
is $\alpha$-Hölder continuous with $\alpha\in(0,1]$ for all $z\in\R$, it suffices $\sqrt{N}h_N^\alpha\to0$ to control $R_2$.
\end{remark}

To control the remaining terms $R_3$ and $R_4$ we use empirical processes. We introduce the concept and the results we need in the next paragraphs. 
For an introduction to empirical processes see \cite{Vaart2000}. More advanced techniques are in \cite{vaart2013}.

Let 
$
  \left( 
    \Omega,
    \mathcal{A},
    \P
  \right)
$
be a probability space,
$
  \left( 
    \mathcal{Z},
    \Sigma
  \right)
$
a measurable space, and 
$
  \xi_1,\ldots,\xi_N
  :
  \left( 
    \Omega,
    \mathcal{A},
    \P
  \right)
  \to
  \left( 
    \mathcal{Z},
    \Sigma
  \right)
$
a sample 
of independent and identically-distributed
random variables
with probability distribution $\P_{\!\xi}$.
A family $\mathcal{F}$ of measurable functions 
$
  f:
  \left( 
    \mathcal{Z},
    \Sigma
  \right)
    \to
  \left( 
    \R,
    \mathcal{B}(\R)
  \right)
$
induces a stochastic process by
\begin{gather}
  f
  \ 
  \mapsto
  \ 
  \G_N f 
  \ 
  :=
  \ 
  \frac{1}{\sqrt{n}}
  \sum_{i=1}^{N} 
  (
    f(\xi_i)
    -
    \E_\xi[f]
  \,.
\end{gather}
We call this the  \textbf{empirical process} $\G_N$ indexed by $\mathcal{F}$.
We define the (random) norm
\begin{gather}
  \norm{\G_n}_\mathcal{F}
  :=
  \sup_
        { f \in \mathcal{F}}
        \left|
          \G_N f
        \right|
        .
\end{gather}
\begin{remark}
We stress that 
$
  \norm{\G_n}_\mathcal{F}
$
often ceases to be measurable, even in simple situations~\cite[page 3]{vaart2013}.
To deal with this, we introduce the notion of \textbf{outer expectation} $\E^*$, that is,
\begin{gather}
  \E^*[Z]
  \ 
  :=
  \ 
    \inf
  \left\{ 
    \E[U]
  \ 
  \lvert
  \ 
    U\ge Z,
    \ 
    U:
  \left( 
    \Omega,
    \mathcal{A},
    \P
  \right)
  \to 
  \left( 
    \overline{\R},
    \mathcal{B}(\overline{\R})
  \right)
  \text{measurable and}
  \ 
  \E[U]<\infty
  \right\}
  \,.
\end{gather}
In our application the technical difficulties halt at this point, because we only consider $Z$ with $\E^*[Z]<\infty$. Then there exists a smallest measurable function $Z^*$ dominating $Z$ with
$\E^*[Z]=\E[Z^*]$. Thus, we may assume $Z$ to be measurable in this regard.


\end{remark}
In our application we need concentration inequalities for 
$
  \norm{\G_n}_\mathcal{F}
$.
One easy way is to use maximal inequalities for the expectation together with Markov's inequality. There are also Bernstein-like inequalities for empirical processes. We need to introduce more concepts.

Given two functions $\underline{f}$ and $\overline{f}$, the bracket
$[\underline{f},\overline{f}]$ 
is the set of all functions $f$ with 
$\underline{f}\le f \le \overline{f}$.
We define a
$(\varepsilon, L^{r}(\P))$-bracket
to be a bracket
$[\underline{f},\overline{f}]$ with
$\norm{\overline{f}-\underline{f}}_{ L^r(\P)}< \varepsilon$.
The bracketing number 
$
N_{[\,]}(\varepsilon, \mathcal{F}, L^r(\P))
$
is 
the minimum number of 
$(\varepsilon, L^{r}(\P))$-brackets needed to cover $\mathcal{F}$.
For most classes $\mathcal{F}$ the bracketing number grows to infinity for $\varepsilon\to 0$.
To measure the speed of convergence we introduce the bracketing integral
\begin{gather}
     J
    _{[]}
    (
    \delta
    ,
    \mathcal{F}
    ,
    L_r(\P)
    )
    =
  \int_0^{\delta}
      \sqrt{
        \log 
      N_{[\,]}
\left( \varepsilon, \mathcal{F}_N, L^r(\P) \right)
    }
    \,
    d\varepsilon
    \,.
\end{gather}
An envelope function $F$ of a class $\mathcal{F}$ satisfies 
$|f(x)|\le F(x)< \infty$ for all $f\in\mathcal{F}$ and all $x$.
\begin{theorem}
  \label{th:max_ineq}
  \emph{(Maximal inequality)}
  For any class $\mathcal{F}$ of measurable functions with envelope function $F,$
  \begin{gather}
    \E^*
    _\P
    [
    \norm{
      \G
      _n
      }
      _\mathcal{F}
    ]
    \lesssim
    J
    _{[]}
    (
    \norm{
      F
    }
    _{ L^2(\P)}
    ,
    \mathcal{F}
    ,
    L_2(\P)
    )
    .
  \end{gather}
\end{theorem}
\begin{proof}
  \cite[Corollary~19.35]{Vaart2000}
\end{proof}
%
%et 
%  $(\varepsilon_N)\subset(0,1]$ be a decreasing sequence with 
%  $\varepsilon_N\to 0$ as $N\to\infty$ and define for
%  $N\in\mathbb{N}$ and $z\in\R$ the function
% \begin{align*}
%    f_{\varepsilon_N,\pm}^z
%    \ 
%    :
%    \ 
%    &
%      \left\{ 0,1 \right\}
%      \times
%      \mathcal{X}
%      \times
%      \mathcal{Y}
%    \ 
%    \to
%    \ 
%    \R
%    \\
%    &
%      (T,X,Y(T))
%      \ 
%      \mapsto
%      \ 
%      \mathbf{1}
%      _
%      {\left\{ 
%          \left|
%      w(X)-\frac{1}{\pi(X)}
%          \right|
%          \,
%          \le
%          \,
%          \varepsilon_N/2
%      \right\}}
%      \left( 
%      w(X)-\frac{1}{\pi(X)}
%      \right)
%      ^{\pm}
%      T
%      \left( 
%        \mathbf{1}
%        _{\left\{  Y(T)\,\le\,z \right\}}
%        -
%        F_{Y(1)}(z|X)
%      \right)
%      \,,
%  \end{align*}
%  and define 
%  \begin{gather*}
%    \mathcal{F}_N^\pm
%    :=
%    \left\{ 
%    f_{\varepsilon_N,\pm}^z
%    |
%    z\in\R
%    \right\}
%    \,.
%  \end{gather*}
%\begin{lemma}
%  Consider the weights function
%  \begin{align*}
%    w
%    \colon
%    \mathcal{X}
%    \times
%    \R^{N+1}
%    \ 
%    \to
%    \ 
%    \R
%    \,,
%    \quad
%    (X,(\lambda,\lambda_0))
%    \ 
%    \mapsto
%    \ 
%    (
%    f^{'}
%    )^{-1}
%    \left( 
%      \inner{B(X)}{\lambda}
%      +
%      \lambda_0
%    \right)
%    \,.
%  \end{align*}
%    If the optimal solution 
%    $
%    (\lambda^\dagger,\lambda_0^\dagger)
%    $ to Problem? exists and is measurable for all $N\in\mathbb{N}$ 
%    then $
%    w
%    \circ
%    (
%    X
%    ,
%    (\lambda^\dagger,\lambda_0^\dagger)
%    )
%    $
%    is measurable.
%\end{lemma}
%\begin{lemma}
%  \label{bounded_f_lemma}
%    If the optimal solution $(\lambda^\dagger,\lambda_0^\dagger)$ to Problem? exists and is measurable for all $N\in\mathbb{N}$ 
%  the function class
%  $
%    \mathcal{F}_N^\pm
%  $
%    is measurable for all $N\in\mathbb{N}$.
%    Furthermore, $\mathcal{F}_N^\pm$ is bounded above by $\varepsilon_N$ and
%    \begin{gather*}
%     J_{[\, ]}
%    \left( 
%      \varepsilon_N
%    ,
%    \mathcal{F}_N^\pm
%    ,
%     L^2(\P)
%    \right)
%    \ 
%    \to
%    \ 
%    0
%    \qquad
%    \text{for}
%    \ 
%    N
%    \to
%    \infty
%\,. 
%    \end{gather*}
%\end{lemma}
%\begin{proof}
%  If the optimal solution $(\lambda^\dagger,\lambda_0^\dagger)$ to Problem? exists and is measurable for all $N\in\mathbb{N}$ 
%  then the weights function $w$ is measurable. Indeed,
%  this follows from
%  \begin{gather*}
%    w(X)=
%    (
%    f^{'}
%    )
%    ^{-1}
%    \left( \inner{B(X)}{\lambda^\dagger}+\lambda^\dagger_0 \right)
%    \,,
%  \end{gather*}
%  the measurability of the basis functions $B$ and the continuity of $
%    (
%    f^{'}
%    )
%    ^{-1}
%    $. Thus $f_{\varepsilon_N,\pm}^z$ are measurable for all $N\in\mathbb{N}$ and all $z\in\R$. Clearly, 
%    $\mathcal{F}_N^\pm$ is bounded above by $\varepsilon_N$.
%    Since 
%    \begin{gather*}
%       \mathbf{1}
%      _
%      {\left\{ 
%          \left|
%      w(X)-\frac{1}{\pi(X)}
%          \right|
%          \,
%          \le
%          \,
%          \varepsilon_N/2
%      \right\}}
%      \left( 
%      w(X)-\frac{1}{\pi(X)}
%      \right)
%      ^{\pm}
%    \end{gather*}
%    is non-negative and bounded above by $\varepsilon_N$
%    it follows from Lemma~\ref
%    {lem:brack_n}
%  \begin{gather*}
%    N_{[\,]}
%    (
%    \varepsilon
%    ,
%    \mathcal{F}_N^\pm, L^2(\P))
%    \ 
%    \lesssim
%    \ 
%    \left( 
%    \frac
%    {\varepsilon_N}
%    {\varepsilon}
%    \right)^2
%    \ 
%    \lesssim
%    \ 
%    \left( 
%    \frac
%    {1}
%    {\varepsilon}
%    \right)^2
%    \qquad
%    \text{for all}
%    \ 
%    \varepsilon>0
%    \,.
%  \end{gather*}
%Thus
%    \begin{align*}
%     J_{[\, ]}
%    \left( 
%      \varepsilon_N
%    ,
%    \mathcal{F}_N^\pm
%    ,
%     L^2(\P)
%    \right)
%    &
%    \ 
%=
%\  
%\int^{\varepsilon_N}_0
%\sqrt{
%\log
%    N_{[\,]}
%    (
%    \varepsilon
%    ,
%    \mathcal{F}_N^\pm, L^2(\P))
%}
%\, 
%d\varepsilon
%\\
%&
%\ 
%\lesssim
%\ 
%\int^{\varepsilon_N}_0
%\log
%\left( 
%  \frac{1}{\varepsilon}
%\right)
%\, 
%d\varepsilon
%\ 
%=
%\ 
%\varepsilon_N
%-
%\varepsilon_N \log(\varepsilon_N)
%\ 
%\to
%\ 
%0 
%\qquad
%\text{for}\ 
%N\to\infty
%\,.
%    \end{align*}
%\end{proof}
%In the next lemma we get rid of the $\varepsilon_N$ bound
%\begin{lemma}
%
%  \label{lemma_fpm}
% Consider for $z\in\R$ the function
% \begin{align*}
%    f_{\pm}^z
%    \ 
%    :
%    \ 
%      \left\{ 0,1 \right\}
%      \times
%      \mathcal{X}
%      \times
%      \mathcal{Y}
%    &
%    \ 
%    \to
%    \ 
%    \R
%    \\
%      (T,X,Y(T))
%    &
%      \ 
%      \mapsto
%      \ 
%      \left( 
%      w(X)-\frac{1}{\pi(X)}
%      \right)
%      ^{\pm}
%      T
%      \left( 
%        \mathbf{1}
%        _{\left\{  Y(T)\,\le\,z \right\}}
%        -
%        F_{Y(1)}(z|X)
%      \right)
%      \,.
%  \end{align*}
%  It holds
%  $\sup_{z\in\R}\left| \G_N f^z_\pm \right|\overset{\P}{\to}0$.
%  Furthermore, for
% \begin{align*}
%    f_z
%    \ 
%    :
%    \ 
%      \left\{ 0,1 \right\}
%      \times
%      \mathcal{X}
%      \times
%      \mathcal{Y}
%    &
%    \ 
%    \to
%    \ 
%    \R
%    \\
%      (T,X,Y(T))
%    &
%      \ 
%      \mapsto
%      \ 
%      \left( 
%      w(X)-\frac{1}{\pi(X)}
%      \right)
%      T
%      \left( 
%        \mathbf{1}
%        _{\left\{  Y(T)\,\le\,z \right\}}
%        -
%        F_{Y(1)}(z|X)
%      \right)
%      \,.
%  \end{align*}
%  it holds
%  $\sup_{z\in\R}\left| \G_N f_z \right|\overset{\P}{\to}0$.
%\end{lemma}
%\begin{proof}
%  It holds for all $z\in\R$
%  $
%  $
%  \begin{gather*}
%  f^z_\pm
%      (T,X,Y(T))
%      \ 
%      \lesssim
%      \ 
%     \left(
%      w(X)- \frac{1}{\pi(X)}
%      \right)
%      ^\pm
%      \,.
%  \end{gather*}
%  By Theorem~\ref{aa:weights:th}
%  there exists a decreasing sequence 
%  $(\varepsilon_N)\subset(0,1]$
%  with $\varepsilon_N\to 0$
%  and
%  $
%  \P[
%\left| 
%      w(X)- 1/\pi(X)
%\right|
%\le
%\varepsilon_N/2
%  ]
%  \to 1
%  $
%  for $N\to\infty$.
%  Therefore
%  \begin{gather*}
%  \P
%  \left[
%  f^z_\pm
%  \in
%  \mathcal{F}_N^\pm
%  \
%  \forall
%  z\in\R
%  \right]
%    =
%  \P[
%\left| 
%      w(X)- 1/\pi(X)
%\right|
%\le
%\varepsilon_N/2
%  ]
%  \to 1
%  \,.
%  \end{gather*}
%Then, for all $\varepsilon>0$ it holds
%\begin{align*}
%  \P
%  \left[
%  \sup_{z\in\R}
%  \left| 
%  \G_N
%  f^z_\pm
%  \right|
%  \le \varepsilon
%  \right]
%  &
%  \ 
%  \ge
%  \ 
%  \P
%  \left[
%  f^z_\pm
%  \in
%  \mathcal{F}_N^\pm
%  \
%  \forall
%  z\in\R
%  \,,
%  \ 
%  \norm{\G_N}^*_{\mathcal{F}_N^\pm}
%  \le \varepsilon
%  \right]
%  \\
%  &
%  \ 
%  \ge
%  \ 
%  \P
%  \left[
%  f^z_\pm
%  \in
%  \mathcal{F}_N^\pm
%  \
%  \forall
%  z\in\R
%  \right]
%  \ 
%  -
%  \ 
%  \P
%  \left[
%  \norm{\G_N}^*_{\mathcal{F}_N^\pm}
%  \le \varepsilon
%  \right]
%  \\
%  &
%  \ 
%  \to
%  \ 
%  1
%  \,.
%\end{align*}
%The convergence of the last term to 0 is due to 
%Lemma~\ref{bounded_f_lemma} and Lemma~\ref{markov_max_lemma}.
%Thus
%  $\sup_{z\in\R}\left| \G_N f^z_\pm \right|\overset{\P}{\to}0$.
%  Note, that 
%  \begin{gather*}
%\G_N f_z 
%=
%\G_N f^z_+
%-
%\G_N f^z_-
%\qquad
%\text{for all}\ 
%z\in\R
%\,.
%  \end{gather*}
%  Thus by Slutzky's Theorem\cite[Theorem~13.18]{Klenke2020}
%  it holds
%  $\sup_{z\in\R}\left| \G_N f_z \right|\overset{\P}{\to}0$.
%\end{proof}
%%\begin{lemma}
%%  Define for
%%  $z\in\R$
%% \begin{align*}
%%    f_z
%%    \ 
%%    :
%%    \ 
%%      \left\{ 0,1 \right\}
%%      \times
%%      \mathcal{X}
%%      \times
%%      \mathcal{Y}
%%&
%%    \ 
%%    \to
%%    \ 
%%      \R
%%    \\
%%    \left(
%%      T,X,Y(T)
%%    \right)
%%    &
%%      \ 
%%      \mapsto
%%      \ 
%%      \frac{T}{\pi(X)}
%%      \left( 
%%        \mathbf{1}
%%        _{\left\{  Y(T)\,\le\,z \right\}}
%%        -
%%        F_{Y(1)}(z|X)
%%      \right)
%%      \ 
%%      +
%%      \ 
%%      \left( 
%%        F_{Y(1)}(z|X)
%%        -
%%        F_{Y(1)}(z)
%%      \right)
%%      \,,
%%  \end{align*}
%%and consider the function class
%%$
%%  \mathcal{G}
%%  \ 
%%  :=
%%  \ 
%%  \left\{ 
%%    f_z
%%  \ 
%%    |
%%  \ 
%%    z\in\R
%%  \right\}
%%$
%%\begin{gather}
%%    \,.
%%\end{gather}
%%  Then $\mathcal{G}$ is a class of measurable functions 
%%  and it holds
%%  \begin{gather}
%%    N_{[\,]}(\varepsilon,\mathcal{G}, L^2(\P))
%%    \le
%%    ??
%%    \qquad
%%    \text{for all}
%%    \ 
%%    \varepsilon>0
%%    \,.
%%  \end{gather}
%%\end{lemma}
%%\begin{lemma}
%%  \label{lemma_max_ineq}
%%  Consider a function class $\mathcal{F}$ with unit ball
%%  $
%%  B_{\mathcal{F}}
%%  :=
%%  \left\{ 
%%    f\in \mathcal{F}
%%    \colon
%%    \norm{f}_\infty
%%    \le
%%    1
%%  \right\}
%%  $.
%%  Let $(\varepsilon_N)$ be a sequence converging to 0
%%  and let 
%%  $
%%  \left( 
%%    \mathcal{F}_N
%%  \right)
%%    :=
%%    \left( 
%%      C\cdot
%%    \varepsilon_N\cdot B_\mathcal{F}
%%    \right)
%%  $
%%  denote the sequence of scaled unit balls in $\mathcal{F}$.
%%  Assume that 
%%  there exists
%%  $k<2$ such that 
%%  the covering number of the unit ball in $\mathcal{F}$
%%  satisfies
%%  \begin{gather}
%%        \log 
%%      N_{[\,]}
%%\left( \varepsilon, B_\mathcal{F}, L^2(\P) \right)
%%\ 
%%\lesssim
%%\ 
%%\left( \frac{1}{\varepsilon} \right)^k
%%\qquad
%%\text{for all}\ 
%%\varepsilon>0
%%\,.
%%  \end{gather}
%%  Then it holds
%%  $
%%      \norm{G_N}^*_{\mathcal{F}_N}
%%    \ 
%%    \overset{\P}
%%    {\to}
%%    \ 
%%    0
%%  $
%%  for $N\to \infty$. 
%%\end{lemma}
%%\begin{proof}
%%  By maximal inequalities it holds
%%  \begin{align*}
%%    \E^*
%%    \left[ 
%%      \norm{G_N}_{\mathcal{F}_N}
%%    \right]
%%    &
%%      \ 
%%      \lesssim
%%      \ 
%%      J_{[\,]}\left( \varepsilon_N, \mathcal{F}_N, L^2(\P) \right)
%%      \\
%%    &
%%      \ 
%%      =
%%      \ 
%%      \int_0^{\varepsilon_N}
%%      \sqrt{
%%        \log 
%%      N_{[\,]}
%%\left( \varepsilon, \mathcal{F}_N, L^2(\P) \right)
%%    }
%%    \,
%%    d\varepsilon
%%    \\
%%    &
%%    \ 
%%    =
%%    \ 
%%      \int_0^{\varepsilon_N}
%%      \sqrt{\log 
%%      N_{[\,]}
%%\left( \varepsilon/(C\cdot \varepsilon_N), B_\mathcal{F}, L^2(\P) \right)
%%    }
%%    \,
%%    d\varepsilon
%%    \\
%%    &
%%    \ 
%%    \lesssim
%%    \ 
%%      \int_0^{\varepsilon_N}
%%      \left( 
%%      \frac{\varepsilon_N}{\varepsilon}
%%    \right)^{k/2}
%%    \,
%%    d\varepsilon
%%    \\
%%    &
%%    \ 
%%    =
%%    \ 
%%  \varepsilon_N^{k/2}
%%  \frac{1}{1-k/2}
%%  \varepsilon_N^{1-k/2}
%%  \\
%%    &
%%    \ 
%%  \lesssim
%%    \ 
%%  \varepsilon_N
%%  \\
%%    &
%%    \ 
%%  \to
%%  \ 
%%  0
%%  \qquad
%%  \text{for}\ 
%%  N\to
%%  \infty
%%  \,.
%%  \end{align*}
%%  Note, that $k<2$.
%%  By the boundedness of $\E^*$ there is no measurability problem.
%%  By Markov's Inequality it holds
%%  \begin{align*}
%%    \P
%%    \left[ 
%%      \norm{G_N}^*_{\mathcal{F}_N}\ge \varepsilon
%%    \right]
%%    \le
%%    \varepsilon^{-1}
%%    \,
%%    \E^*
%%    \left[ 
%%      \norm{G_N}_{\mathcal{F}_N}
%%    \right]
%%    \ 
%%    \to
%%    \ 
%%    0
%%    \qquad
%%    \text{for}\ 
%%    N\to\infty
%%    \,.
%%  \end{align*}
%%\end{proof}
%%The next two lemmas connect $R_3$ to the theory of empirical processes.
%%\begin{lemma}
%%  \label{aa:mean:l:fz}
%%  Consider 
%%  the (random) function
%%  $
%%  f_D^z
%%  $ given by
%%  \begin{gather}
%%    f_{D}^z(T,X,Y(T))
%%    :=
%%    T
%%    \left( 
%%    w(D,X)- \frac{1}{\pi(X)}
%%    \right)
%%    \left( 
%%    \mathbf{1}_{\left\{ Y(T) \le z \right\}}
%%    -
%%  F_{Y(1)}(z|X)
%%    \right)
%%    \,.
%%  \end{gather}
%%  Assume that 
%%  there exists a function class $\mathcal{F}$ satisfying the requirements of Lemma~\ref{lemma_max_ineq} and that
%%  $
%%  f_D^z
%%  \in \mathcal{F}
%%  $
%%  for all $z\in\R$ almost surely.
%%  It then holds
%%  $
%%  \sup_{z\in\R} \left| G_Nf_D^z \right|
%%  \overset{\P}{\to}0$ for 
%%  $N\to\infty$.
%%\end{lemma}
%%\begin{proof}
%%  By the consistency of the weights there exists a learning rate $(\varepsilon_N)$ such that
%%  \begin{gather}
%%    \P
%%    \left[ 
%%      \left| 
%%      w(X,D)
%%      -
%%      \frac{1}{\pi(X)}
%%      \right|
%%      \le
%%      \varepsilon_N
%%    \right]
%%    \to 1 
%%    \qquad
%%    \text{for}\ 
%%    N\to \infty
%%    \,.
%%  \end{gather}
%%  Let
%%  $\mathcal{F}_N:=\varepsilon_NB_\mathcal{F}$ as in 
%%  Lemma~\ref{lemma_max_ineq}.
%%  It holds
%%\begin{gather}
%%      \sup_{z\in\R}
%%      \left| 
%%    f_D^z
%%      \right|
%%      \lesssim
%%      \left| 
%%      w(X,D)
%%      -
%%      \frac{1}{\pi(X)}
%%      \right|
%%      \le
%%      \varepsilon_N
%%\end{gather}
%%with probability going to 1 as $N\to\infty$.
%%Thus
%% \begin{gather}
%%    \P
%%    \left[ 
%%    f_D^z
%%    \in
%%  \mathcal{F}_N
%%  \ \forall\,z\in\R
%%    \right]
%%    =
%%    \P
%%    \left[ 
%%      \sup_{z\in\R}
%%      \left| 
%%    f_D^z
%%      \right|
%%      \lesssim
%%      \varepsilon_N
%%    \right]
%%    \to 1
%%    \qquad
%%    \text{as}
%%    \ 
%%    N\to\infty
%%    \,.
%%  \end{gather}
%%  Then it holds
%%  for all $\varepsilon>0$ 
%%  \begin{align*}
%%    \P
%%    \left[ 
%%      \sup_{z\in\R}
%%  \left| G_Nf_D^z \right|
%%  \le
%%  \varepsilon
%%    \right]
%%    &
%%    \ 
%%    \ge
%%    \ 
%%    \P
%%    \left[ 
%%      \sup_{z\in\R}
%%  \left| G_Nf_D^z \right|
%%  \le
%%  \norm{G_N}^*_{\mathcal{F}_N}
%%  \le
%%  \varepsilon
%%    \right]
%%    \\
%%    &
%%    \ 
%%    \ge
%%    \ 
%%    \P
%%    \left[ 
%%    f_D^z
%%    \in
%%  \mathcal{F}_N
%%  \ \forall\,z\in\R
%%      \ 
%%      \text{and}\ 
%%  \norm{G_N}^*_{\mathcal{F}_N}
%%  \le
%%  \varepsilon
%%    \right]
%%    \\
%%    &
%%    \ 
%%    \ge
%%    \ 
%%    \P
%%    \left[ 
%%    f_D^z
%%    \in
%%  \mathcal{F}_N
%%  \ \forall\,z\in\R
%%    \right]
%%    -
%%    \P
%%    \left[ 
%%  \norm{G_N}^*_{\mathcal{F}_N}
%%    \ 
%%  \ge
%%    \ 
%%  \varepsilon
%%    \right]
%%    \\
%%    &
%%    \ 
%%    \to
%%    \ 
%%    1
%%    \,.
%%  \end{align*}
%%  The convergence of the second term is due to Lemma~\ref{lemma_max_ineq}.
%%
%%\end{proof}
%%
%\begin{lemma}
%  \label{aa:mean:l:r3}
%  Assume conditional unconfoundedness, that is,
%  \begin{gather}
%  (Y(0),Y(1))\perp T \ |\ X
%  \,.
%  \end{gather}
%  Then for all
%  $z\in\R$
%  it holds
%  $
%  G_Nf_z=R_3(z)
%  $.
%  Furthermore, under conditions it holds
%  $\sup_{z\in\R}\left| R_3(z) \right|\overset{\P}{\to}0$.
%\end{lemma}
%\begin{proof}
%  A standard computation shows
%  \begin{gather}
%    \E
%    \left[ 
%    \frac{T}{\pi(X)}
%    \left( 
%    \mathbf{1}_{\left\{ Y(T) \le z \right\}}
%    -
%  F_{Y(1)}(z|X)
%    \right)
%    \right]
%    =0
%    \,.
%  \end{gather}
%  Furthermore
%  \begin{align*}
%    &
%    \E
%    \left[ 
%      Tw(X,D)
%    \left( 
%    \mathbf{1}_{\left\{ Y(T) \le z \right\}}
%    -
%  F_{Y(1)}(z|X)
%    \right)
%    \right]
%    \\
%    &
%    \ 
%    =
%    \ 
%    \E
%    \left[ 
%      \E
%      \left[ 
%      w(X,D)
%      \left( 
%    \mathbf{1}_{\left\{ Y(1) \le z \right\}}
%    -
%  F_{Y(1)}(z|X)
%      \right)
%  |T=1,X,D
%      \right]
%    \right]
%    \\
%    &
%    \ 
%    =
%    \ 
%    \E
%    \left[ 
%      w(X,D)
%      \E
%      \left[ 
%    \mathbf{1}_{\left\{ Y(1) \le z \right\}}
%    -
%  F_{Y(1)}(z|X)
%  |X,D
%      \right]
%    \right]
%    \\
%    &
%    \ 
%    =
%    \ 
%    \E
%    \left[ 
%      w(X,D)
%      \E
%      \left[ 
%    \mathbf{1}_{\left\{ Y(1) \le z \right\}}
%    -
%  F_{Y(1)}(z|X)
%  |X
%      \right]
%    \right]
%    \\
%    &
%    \ 
%    =
%    \ 
%    0
%  \end{align*}
%  The second equality is due to the assumption of 
%  $(Y(0),Y(1))\perp T |X$.
%  The third equality is due to $X\perp D$.
%  Thus
%  $
%    \E
%    f_D^z
%    =
%    0
%  $
%\end{proof}
%

Next, we define some auxiliary functions.
  For $z\in\R$ we define the function
  \begin{align*}
    f_z
    \ 
    :
    \ 
      \left\{ 0,1 \right\}
      \times
      \mathcal{X}
      \times
      \mathcal{Y}
    &
    \ 
    \to
    \ 
    \R
    \\
      (t,x,y)
    &
      \ 
      \mapsto
      \ 
      t
      \left( 
        \mathbf{1}
        _{\left\{  y\,\le\,z \right\}}
        -
        F_{Y(1)}(z|x)
      \right)
      \,,
  \end{align*}
  and the function classes
  \begin{gather}
    \label{F_g}
    \begin{split}
    \mathcal{F}
    &
    \ 
    :=
    \ 
    \left\{ 
      f_z
      \ 
      |
      \ 
      z\in\R\ 
    \right\}
    \\
    \mathcal{G}
    &
    \ 
    :=
    \ 
    \left\{ 
      \frac{f_z}{\pi(\cdot)}
      +
      F_{Y(1)}(z|\cdot)
      -
      F_{Y(1)}(z)
      \ 
      \colon
      \ 
      z\in\R\ 
    \right\}
    \,.
    \end{split}
  \end{gather}
  \begin{assumption}
    \label{aa:assumption:treatment_str_ign}
    It holds 
    \begin{gather*}
    (Y(0),Y(1))\, \perp\,  T \ |\  X
    \qquad
    \text{and}
    \qquad
      0
      \ 
      <
      \ 
      \pi(x)
      \ 
      <
      \ 
      1
      \quad
      \text{for all}\ 
      x\in\mathcal{X}
      \,.
    \end{gather*}
  \end{assumption}
  \begin{lemma}
    \label{aa:mean:r3:lem:fz_E}
    It holds
    $f_z(T,X,Y(T))\in L^1(\P)$
    and 
    $f_z(T,X,Y(T))\perp D_N$
    for all $z\in\R$.
    If also Assumption~\ref{aa:assumption:treatment_str_ign} holds,
    then 
    for all $z\in\R$
    \begin{gather*}
      \E
      \left[
        f_z
        \left( 
          T,
          X,
          Y(T)
        \right)
        \,
        |
        \,
        X
      \right]
      \ 
      =
      \ 
      0
      \qquad
      \text{almost surely.}
    \end{gather*}
  \end{lemma}
  \begin{proof}
    Since $f_z$ is bounded by 1,
    it holds
    $f_z(T,X,Y(T))\in L^1(\P)$.
    Since
    \begin{gather*}
    (T,X,Y(T))
    \ 
    \perp 
    \ 
    D_N
    \ 
    =
    \ 
    (T_i,X_i)_{i\in \left\{ 1,\ldots,N \right\}}
    \end{gather*}
    it holds
    $f_z(T,X,Y(T))\perp D_N$
    for all $z\in\R$.
    For the third statement, note that
    \begin{align*}
      \E
      \left[
        f_z
        \left( 
          T,
          X,
          Y(T)
        \right)
        \,
        |
        \,
        X
      \right]
      &
      \ 
      =
      \ 
      \E
      \left[
      T
      \left( 
        \mathbf{1}
        _{\left\{  Y(T)\,\le\,z \right\}}
        -
        F_{Y(1)}(z|X)
      \right)
        \,
        |
        \,
        X
      \right]
      \\
      &
      \ 
      =
      \ 
      \E
      \left[
        \mathbf{1}
        _{\left\{  Y(1)\,\le\,z \right\}}
        -
        F_{Y(1)}(z|X)
        \,
        |
        \,
        X
        ,
        T=1
      \right]
      \pi(X)
      \\
      &
      \ 
      =
      \ 
      \left( 
      \E
      \left[
        \mathbf{1}
        _{\left\{  Y(1)\,\le\,z \right\}}
        \,
        |
        \,
        X
      \right]
      \ 
        -
      \ 
        F_{Y(1)}(z|X)
      \right)
      \pi(X)
      \\
      &
      \ 
      =
      \ 
      0
      \qquad
      \text{almost surely.}
    \end{align*}
    The third equality is due to Assumption~\ref{aa:assumption:treatment_str_ign}.
  \end{proof}
The next lemma provides bracketing numbers for specific function classes needed to control $R_3$ and $R_4$.
\newpage
\begin{lemma}
  \label{aa:mean:l:br}
  The function class $\mathcal{F}$ and $\mathcal{G}$ defined in \eqref{F_g} are measurable.
  Furthermore, 
  \begin{gather*}
    N_{[\,]}
    (
    \varepsilon
    ,
    \mathcal{F}, L^2(\P))
    \ 
    \lesssim
    \ 
    \left( 
      \frac{1}{\varepsilon}
    \right)^2
    \qquad
    \text{for all}
    \ 
    \varepsilon>0
    \,.
  \end{gather*}
  If $1/\pi(X)\in L^2(\P)$, it also holds 
  \begin{gather*}
    N_{[\,]}
    (
    \varepsilon
    ,
    \mathcal{G}, L^2(\P))
    \ 
    \lesssim
    \ 
    \left( 
    \frac{
      1+
    \norm{1/\pi(X)}_{ L^2(\P)}
    }
    {\varepsilon}
    \right)^4
    \qquad
    \text{for all}
    \ 
    \varepsilon>0
    \,.
  \end{gather*}
%  Furthermore, 
%  consider the function class
%  \begin{gather*}
%    \mathcal{G}
%    :=
%    \left\{ 
%      f_{1/\pi}^z
%      +
%        F_{Y(1)}(z|\cdot)
%        -
%        F_{Y(1)}(z)
%      \ 
%      |
%      \ 
%      z\in\R
%    \right\}
%    \,.
%  \end{gather*}
%  If $1/\pi(X)\in  L^2(\P)$
%  it holds
%  \begin{gather}
%    N_{[\,]}(\varepsilon,\mathcal{G}, L^2(\P))
%    \le
%    ??
%    \qquad
%    \text{for all}
%    \ 
%    \varepsilon>0
%    \,.
%  \end{gather}
\end{lemma}
\begin{proof}
  As in \cite[Example~19.6]{Vaart2000}
  we choose for
  $\varepsilon>0$ and $m\in\mathbb{N}$
  \begin{gather*}
  -\infty=z_0\ <\ z_1\ <\ \cdots\ <\ z_{m-1}\ <\ z_m=\infty
  \,
  \end{gather*}
  such that
  \begin{gather}
    \label{size_z}
    \P
    \left[ 
      Y(1)\in \left[ z_{l-1},z_l \right]\,
    \right]
    \ 
    \le
    \ 
    \varepsilon
    \qquad
    \text{for all}\ 
    l\in \left\{ 1,\ldots,m \right\}
  \end{gather}
  and $m \le 2/\varepsilon$.
  Next, we define $m$ brackets by
\begin{align*}
  \overline{f_l}
  (t,x,y)
  &
  \ 
  :=
  \ 
      t
      \left( 
        \mathbf{1}
        _{\left\{  y\,\le\,z_{l} \right\}}
        -
        F_{Y(1)}(z_{l-1}|x)
      \right)
      \,,
      \\
  \underline{f_l}
  (t,x,y)
  &
  \ 
  :=
  \ 
      t
      \left( 
        \mathbf{1}
        _{\left\{  y\,\le\,z_{l-1} \right\}}
        -
        F_{Y(1)}(z_l|x)
      \right)
      \,,
\end{align*}
for $l\in \left\{ 1,\ldots,m \right\}$.
These brackets cover $\mathcal{F}$.
Indeed,
\begin{gather*}
  \text{for all}\ 
  z\in\R
  \ 
  \text{there exists} \ 
l\in \left\{ 1,\ldots,m \right\}
\qquad 
\text{such that}\qquad
z_{l-1}
\ 
\le
\ 
z
\ 
\le
\ 
z_l
\,.
\end{gather*}
By the monotonicity of 
$
        \mathbf{1}
      _{\left\{  y\,\le\,(\cdot) \right\}}
$
and
$
        F_{Y(1)}(\cdot|x)
$
and the non-negativity of $T$ it follows
\begin{gather*}
  \text{for all}\ 
  z\in\R
  \ 
  \text{there exists} \ 
l\in \left\{ 1,\ldots,m \right\}
\qquad 
\text{such that}\qquad
  \underline{f_l}
  \ 
  \le
  \ 
  f_z
  \ 
  \le
  \ 
  \overline{f_l}
  \,.
\end{gather*}
Thus, the $m$ brackets 
$
[
  \underline{f_l}
  ,
  \overline{f_l}
]
$
cover $\mathcal{F}$.

Let's calculate the size of the brackets.
It holds
\begin{align*}
  &
\E
\left[ 
      T
      \cdot
      \left( 
        \mathbf{1}
        _{\left\{  Y(T)\,\le\,z_{l} \right\}}
        -
        F_{Y(1)}(z_{l-1}|X)
        \ 
        -
        \ 
        \mathbf{1}
        _{\left\{  Y(T)\,\le\,z_{l-1} \right\}}
        +
        F_{Y(1)}(z_{l}|X)
      \right)
      \,
\right]
\\
  &
  \ 
=
  \ 
\E
\left[ 
      T
      \cdot
      \left( 
        \mathbf{1}
        _{\left\{
        Y(T)
        \,
        \in 
        \,
    [z_{l-1},z_l]
\right\}}
\ 
        +
\ 
        \P
        \left[ 
          Y(1)
          \in
    [z_{l-1},z_l]
        \,
    |
        \,
    X
        \right]
      \right)
      \,
\right]
\\
  &
  \ 
\le
  \ 
\E
\left[ 
  \,
  \pi(X)
  \cdot
        \P
        \left[ 
          Y(1)
          \in
    [z_{l-1},z_l]
          \,
    |
          \,
    X
        \right]
          \ 
\right]
\ 
+
\ 
\varepsilon
\\
  &
  \ 
\le
  \ 
2
\,
\varepsilon
\,.
\end{align*}
We used \eqref{size_z}, $0\le T,\pi(X)\le 1$ and Lemma~\ref{ps_weights_lemma}.
It follows
\begin{align*}
  &
  \norm{
    \left( 
  \overline{f_l}
-
  \underline{f_l}
    \right)
  (T,X,Y(T))
}_
{ L^2(\P)}
\\
&
\ 
\lesssim
\ 
\E
\left[ 
  \,
      T
      \cdot
      \left( 
        \mathbf{1}
        _{\left\{
        Y(T)
        \,
        \in 
        \,
    [z_{l-1},z_l]
\right\}}
\ 
        +
\ 
        \P
        \left[ 
          Y(1)
          \in
    [z_{l-1},z_l]
        \,
    |
        \,
    X
        \right]
      \right)
      \,
   \right]^{1/2}
\ 
\lesssim
\ 
\varepsilon^{1/2}
\,.
\end{align*}
Since $m\le 2/\varepsilon$ it holds
  \begin{align*}
    N_{[\,]}
    \left(
\varepsilon^{1/2}
    ,
    \,
    \mathcal{F}\,,\, L^2(\P)
    \right)
    &
    \ 
    \lesssim
    \ 
    \frac{1}{\varepsilon}
    \intertext{and thus}
    N_{[\,]}
    (
    \varepsilon
    ,
    \mathcal{F}, L^2(\P))
    &
    \ 
    \lesssim
    \ 
    \left( 
      \frac{1}{\varepsilon}
    \right)^2
    \,.
  \end{align*}
  Next, we look at $\mathcal{G}$. To this end, we define 
  $m$ brackets by
 \begin{align*}
    \overline{g_l}
    (t,x,y)
    \ 
    :=
    \ 
    \frac{t}{\pi(x)}
    \left( 
      \mathbf{1}_{\left\{  y\,\le\,z_{l} \right\}}
      -
      F_{Y(1)}(z_{l-1}|x)
    \right)
    \ 
    +
    \ 
    F_{Y(1)}(z_{l}|x)
-
F_{Y(1)}(z_{l-1})
\,,
\\
    \underline{g_l}
    (t,x,y)
    \ 
    :=
    \ 
    \frac{t}{\pi(x)}
    \left( 
      \mathbf{1}_{\left\{  y\,\le\,z_{l-1} \right\}}
      -
      F_{Y(1)}(z_l|x)
    \right)
    \ 
    +
    \ 
    F_{Y(1)}(z_{l-1}|x)
-
      F_{Y(1)}(z_l)
\,,
  \end{align*}
  for $l\in \left\{ 1,\ldots,m \right\}$.
  With the same arguments as before, we see that these brackets cover $\mathcal{G}$.
  Let's calculate the size.
  It holds
  \begin{align*}
    &
    \norm{
      \frac{T}{\pi(X)}
      \left( 
      \mathbf{1}_{
      \left\{ 
      Y(T)\in [z_{l-1},z_l] 
    \right\}
    }
      +
      \P
      \left[ 
      Y(1)\in [z_{l-1},z_l] 
      \,
      |
      \,
      X
      \right]
      \right)
    }_{ L^2(\P)}
    \\
    &
    \
    \lesssim
    \
    \left( 
      \E
      \left[ 
        \frac{1}{\pi(X)}
        \frac{T}{\pi(X)}
      \left( 
      \mathbf{1}_{
      \left\{ 
      Y(T)\in [z_{l-1},z_l] 
    \right\}
    }
      +
      \P
      \left[ 
      Y(1)\in [z_{l-1},z_l] 
      \,
      |
      \,
      X
      \right]
      \right)
      \right]
    \right)
    ^{1/2}
    \\
    &
    \
    \lesssim
    \
    \left( 
      \E
      \left[ 
        \frac{1}{\pi(X)}
      \P
      \left[ 
      Y(1)\in [z_{l-1},z_l] 
      \,
      |
      \,
      X
      \right]
      \right]
    \right)
    ^{1/2}
    \\
    &
    \ 
    \lesssim
    \ 
    \left( 
      \norm{1/\pi(X)}_{ L^2(\P)}
      \sqrt{\varepsilon}
    \right)
    ^{1/2}
    \ 
    =
    \ 
    \varepsilon^{1/4}
    \norm{1/\pi(X)}_{ L^2(\P)}^{1/2}
  \end{align*}
  and
  \begin{align*}
     \norm{
      \P
      \left[ 
      Y(1)\in [z_{l-1},z_l] 
      \,
      |
      \,
      X
      \right]
      \ 
     + 
      \ 
      \P
      \left[ 
      Y(1)\in [z_{l-1},z_l] \,
      \right]
    }_{ L^2(\P)}
    \ 
    \lesssim
    \ 
    \varepsilon^{1/2}
    \,.
  \end{align*}
  Thus
  \begin{align*}
  \norm{
    \left( 
  \overline{g_l}
-
  \underline{g_l}
    \right)
  (T,X,Y(T))
}_
{ L^2(\P)}
    &
\ 
\lesssim
\ 
\varepsilon^{1/4}
\left( 
  1
  +
    \norm{1/\pi(X)}_{ L^2(\P)}^{1/2}
\right)
\\
    &
\ 
\lesssim
\ 
\varepsilon^{1/4}
\left( 
  1
  +
    \norm{1/\pi(X)}_{ L^2(\P)}
\right)
\,.
  \end{align*}
As before, it follows
\begin{gather*}
    N_{[\,]}
    (
    \varepsilon
    ,
    \mathcal{G}, L^2(\P))
    \ 
    \lesssim
    \ 
    \left( 
    \frac{
      1+
    \norm{1/\pi(X)}_{ L^2(\P)}
    }
    {\varepsilon}
    \right)^4
    \,.
\end{gather*}
\end{proof}

\begin{lemma}
  \label{markov_max_lemma}
  Let $(\mathcal{H}_N)$ be a sequence of measurable function classes with envelope functions $(H_N)$.
  If
  \begin{gather}
    J_{[\, ]}
    \left( 
    \norm{F_N}_{ L^2(\P)}
    ,
    \mathcal{H}_N
    ,
     L^2(\P)
    \right)
    \ 
    \to
    \ 
    0
    \qquad
    \text{for}
    \ 
    N
    \to
    \infty
  \end{gather}
  it holds 
  $
  \norm{\G_N}^*_{\mathcal{H}_N}\overset{\P}{\to}0
  $.
\end{lemma}
\begin{proof}
  By Markov's inequality and Theorem~\ref{th:max_ineq} it holds for all $\varepsilon>0$
  \begin{align*}
    &
    \P
    [
  \norm{\G_N}^*_{\mathcal{H}_N}
  \ge
  \varepsilon
    ]
    \\
    &
    \ 
    \le
    \ 
    \varepsilon^{-1}
    \E
    [
  \norm{\G_N}^*_{\mathcal{H}_N}
    ]
    \ 
    =
    \ 
    \varepsilon^{-1}
    \E^*
    [
  \norm{\G_N}_{\mathcal{H}_N}
    ]
    \ 
    \lesssim
    \ 
    \varepsilon^{-1}
    J_{[\, ]}
    \left( 
    \norm{H_N}_{ L^2(\P)}
    ,
    \mathcal{H}_N
    ,
     L^2(\P)
    \right)
    \ 
    \to
    \ 
    0
  \end{align*}
  for $N\to\infty$. 
\end{proof}
A technical lemma for products of function classes.

Define the product of two function classes
\begin{gather*}
  \mathcal{F}\cdot \mathcal{G}
  :=
  \left\{ 
    f\cdot g
    \colon
    f\in\mathcal{F},
    g\in\mathcal{G}
  \right\}\,.
\end{gather*}
\begin{lemma}
  \label{lem_prod_br}
  Let
  $\mathcal{F}$ and $\mathcal{G}$ be two function classes 
  with envelope functions $F$ and $G$ satisfying
  $\norm{F}_\infty,\norm{G}_\infty\le 1$.
  For all $\varepsilon>0$ and all $r\in [1,\infty)$ it holds
  \begin{gather*}
    N_{[\,]}(2\varepsilon,\mathcal{F}\cdot\mathcal{G},\mathrm{L}_r(\P))
    \
    \le
    \ 
    N_{[\,]}(\varepsilon,\mathcal{F},\mathrm{L}_r(\P))
    \cdot
    N_{[\,]}(\varepsilon,\mathcal{G},\mathrm{L}_r(\P))
    \,.
  \end{gather*}
\end{lemma}
\begin{proof}
  Let $f\in\mathcal{F}$ and $g\in\mathcal{G}$.
  We can choose two 
  $(\varepsilon,L^r(\P))$
  brackets
  $[\underline{f},\overline{f}]$
  and
  $[\underline{g},\overline{g}]$
  containing $f$ and $g$ with 
  $\norm{\underline{f}}_\infty,\norm{\overline{f}}_\infty\le\norm{F}_\infty\le 1$
  and
  $\norm{\underline{g}}_\infty,\norm{\overline{g}}_\infty\le\norm{G}_\infty\le 1$.
  We the get an 
  $(2\varepsilon,L^r(\P))$
  $[\underline{h},\overline{h}]$
  bracket, containing $f\cdot g$, by
\end{proof}

\begin{lemma}
  \label{aa:r3:lemma:1}
  Let
$(\varepsilon_N)\subset(0,1]$
be 
a decreasing sequence
with $\varepsilon_N\to 0$ for $N\to\infty$ 
and let Assumption~\ref{aa:assumption:1} hold true
for a sequence of function classes $\mathcal{F}_N$.
Then
\begin{gather*}
  J_{[\,]}(
\norm{F_N}_{L^2(\P)}
,\mathcal{F}_N\cdot\mathcal{F},\mathrm{L}_2(\P))
  \to 0
  \quad
  \text{and}
  \quad
  \norm{\G_N}^*_{\mathcal{F}_N\cdot\mathcal{F}}\overset{\P}{\to}0
  \qquad
  \text{for}\ 
  N\to\infty
  \,.
\end{gather*}
\end{lemma}
\begin{proof}
  By Assumption~\ref{aa:assumption:1}
  and Lemma~\ref{aa:mean:l:br} it holds
for some $k<2$
\begin{gather*}
\norm{F_N}_{L^2(\P)}
\ 
\le
\ 
\varepsilon_N
\quad
\text{and}
\quad
  \log
  N_{[\,]}(\varepsilon,\mathcal{F}_N,\mathrm{L}_2(\P))
  \ 
  \lesssim
  \ 
  \left( 
  \frac{1}{\varepsilon}
  \right)^k
  \quad
  \text{for all}
  \ 
  N\in\mathbb{N}
  \,,
\end{gather*}
and
  \begin{gather*}
    N_{[\,]}
    (
    \varepsilon
    ,
    \mathcal{F}, L^2(\P))
    \ 
    \lesssim
    \ 
    \left( 
      \frac{1}{\varepsilon}
    \right)^2
    \qquad
    \text{for all}
    \ 
    \varepsilon>0
    \,.
  \end{gather*}
  Since $\mathcal{F}_N$ and $\mathcal{F}$ have envelope function smaller 1, we can apply Lemma~\ref{lem_prod_br} to get
  \begin{gather*}
  \log
  N_{[\,]}(\varepsilon,\mathcal{F}_N\cdot\mathcal{F},\mathrm{L}_2(\P))
  \ 
  \lesssim
  \ 
  \left( 
  \frac{1}{\varepsilon}
  \right)^k
  +
  \log
  (1/\varepsilon)
  \ 
  \lesssim
  \ 
  \left( 
  \frac{1}{\varepsilon}
  \right)^k
  \quad
  \text{for all}\ 
  \varepsilon>0
  \,.
  \end{gather*}
  Since 
  $k/2\in(0,1)$
  it holds
\begin{align*}
  J_{[\,]}(
\norm{F_N}_{L^2(\P)}
,\mathcal{F}_N\cdot\mathcal{F},\mathrm{L}_2(\P))
  &
  \ 
=
  \ 
\int_0^{
\norm{F_N}_{L^2(\P)}
}
\sqrt{
  \log
  N_{[\,]}(\varepsilon,\mathcal{F}_N\cdot\mathcal{F},\mathrm{L}_2(\P))
}
\,d\varepsilon
\\
&
\ 
\lesssim
\ 
\int_0^{
  \varepsilon_N
}
  \left( 
  \frac{1}{\varepsilon}
\right)^{k/2}
\,d\varepsilon
\\
&
\ 
=
\ 
\frac{
\varepsilon_N^{1-k/2}
}{1-k/2}
\ 
\to 0
\ 
\qquad
\text{for}
\ 
N\to\infty
\,.
\end{align*}
The second statement follows from Lemma~\ref{markov_max_lemma}
for 
$\mathcal{H}_N:=\mathcal{F}_N\cdot\mathcal{F}$ and
$H_N:=F_N$.
\end{proof}

\begin{lemma}
  \label{ps_weights_lemma}
  Let
  $
  g_1\colon
  \mathcal{X}\to\R
  $
  and
  $
  g_2\colon
  \mathcal{Y}\to\R
  $
  be a measurable functions.
  It holds
  \begin{gather*}
    \E
    \left[
    \frac{T}{\pi(X)}
    g_1(X)
    \right]
    \ 
    =
    \ 
    \E
    \left[
    g_1(X)
    \right]
    \,.
  \end{gather*}
  If Assumption~\ref{aa:assumption:treatment_str_ign} holds true, then
  \begin{gather*}
    \E
    \left[
    \frac{T}{\pi(X)}
    g_2(Y(T))
    \right]
    \ 
    =
    \ 
    \E
    \left[
    f(Y(1))
    \right]
    \,.
  \end{gather*}
\end{lemma}
\begin{lemma}
  \label{aa:mean:r3:lem:conv}
  Under conditions it holds
  $\sup_{z\in\R}\left| R_3(z) \right|\overset{\P}{\to}0$.
\end{lemma}
\begin{proof}
  Let $N\ge\underline{N}$,
  $z\in\R$,
  and 
  let
  $g^\dagger$ 
  denote the function \eqref{ghost_function} with 
  $
(\lambda^\dagger,\lambda_0^\dagger)
  $.
  If
  \begin{gather*}
    \left| 
    w^\dagger(X)- \frac{1}{\pi(X)} 
    \right|
    \ 
    \le
    \ 
    \varepsilon_N
  \end{gather*}
  it holds
  \begin{gather*}
    g^\dagger
    (X)
    \cdot
    f_z
    (T,X,Y(T))
    \ 
    =
    \ 
  \left( 
      w^\dagger(X)
      -
      \frac{1}{\pi(X)}
  \right)
      T
      \left( 
        \mathbf{1}
        _{\left\{  Y(T)\,\le\,z \right\}}
        -
        F_{Y(1)}(z|X)
      \right)
      \,.
  \end{gather*}
  By Lemma~\ref{aa:mean:r3:lem:fz_E}
  it holds 
  \begin{align*}
    &
  f_z(T,X,Y(T))
  \ 
  \in
  \ 
  L^1(\P)\,, 
  \\
  &
  f_z(T,X,Y(T))
  \ 
  \perp
  \ 
  D_N
  \,,
  \\
  \E
  [
  &
  f_z(T,X,Y(T))
    |X
  ]
  \ 
  =
  \ 
  0
  \,.
  \end{align*}
  Thus, 
  it follows from Lemma~\ref{w.Z=0}
  \begin{gather*}
    \E
    \left[
      w^\dagger(X)
      \cdot
      f_z(T,X,Y(T))
    \right]
    \ 
    =
    \ 
    0
    \,.
  \end{gather*}
  Since
  \begin{gather*}
    \E
    \left[
      \frac{T}{\pi(X)}
      f_z(T,X,Y(T))
    \right]
    \ 
    =
    \ 
    \E
    \left[
      \frac{T}{\pi(X)}
      \left( 
        \mathbf{1}
        _{\left\{  Y(T)\,\le\,z \right\}}
        -
        F_{Y(1)}(z|X)
      \right)
    \right]
    \  
    =
    \ 
    0
  \end{gather*}
  by Lemma~\ref{ps_weights_lemma},
  it follows
  \begin{align*}
    &
  \E
  \left[
    g^\dagger(X)
    f_z(T,X,Y(T))
  \right]
  \\
    &
  \ 
  =
  \ 
    \E
    \left[
      w^\dagger(X)
      \cdot
      f_z(T,X,Y(T))
    \right]
    \ 
    -
    \ 
    \E
    \left[
      \frac{T}{\pi(X)}
      f_z(T,X,Y(T))
    \right]
    \ 
    =
    \ 
  0
  \,.
  \end{align*}
  But then 
  \begin{align*}
    R_3(z)
    &
    \ 
    =
    \ 
  \frac{1}
  {
\sqrt{N}
  }
    \sum_{i=1}^{N} 
    \left[ 
    \left( 
    w(X_i) 
    -
    \frac{1}{\pi(X_i)}
    \right)
    T_i
    \left( 
    \mathbf{1}_{\left\{ Y_i \le z \right\}}
    -
  F_{Y(1)}(z|X_i)
    \right)
    \right]
    \ 
    =
    \ 
    \G_N
    \left( 
      g^\dagger
      \cdot
      f_z
    \right)
    \,.
  \end{align*}
  It follows
\begin{align*}
    \P
    \left[ 
      \sup_{z\in\R}
     | 
    R_3(z)
    |
      \ge
      \varepsilon
    \right]
    &
    \ 
    \le
    \ 
    \P
    \left[ 
      \sup_{z\in\R}
     | 
    R_3(z)
    |
      \ge
      \varepsilon
      \ 
      \text{and}
      \ 
    | 
    w^\dagger(X)- 1/\pi(X)
    |
    \ 
    \le
    \ 
    \varepsilon_N
    \right]
    \\
    &
    \quad
    +
    \ 
    \P
    \left[ 
    | 
    w^\dagger(X)- 1/\pi(X)
    |
    \ 
    >
    \ 
    \varepsilon_N
    \right]
    \\
    &
    \ 
    \le
    \ 
    \P
    \left[ 
      \norm{\G_N}^*_{\mathcal{F}_N\cdot\mathcal{F}}
      \ge
      \varepsilon
    \right]
    \ 
    +
    \ 
    \ 
    \P
    \left[ 
    | 
    w^\dagger(X)- 1/\pi(X)
    |
    \ 
    >
    \ 
    \varepsilon_N
    \right]
    \\
    &
    \ 
    \to
    \ 
    0
    \,.
\end{align*}
The convergence of the first term follows from
Lemma~\ref{aa:r3:lemma:1}.
The convergence of the second term follows from
Theorem~\ref{aa:weights:th}.

\end{proof}


Until now, all parts of the error decomposition converge to 0.
The last term $R_4$ will decide the profile of the limiting process.
To this end we need the following concept.
\begin{definition}
  We call a class 
  $\mathcal{F}$ of measurable functions 
$\P$-Donsker
if the sequence of processes 
$\left\{ G_N f \colon f\in\mathcal{F}\right\}$
converges in
$l^\infty(\mathcal{F})$
to a tight limit process.
\end{definition}

\begin{theorem}
  Every class $\mathcal{F}$ of measurable functions 
  with
  $
    J
    _{[]}
    (
    1
    ,
    \mathcal{F}
    ,
    L_2(\P)
    )
    <\infty
  $
  is

  $\P$-Donsker, that is,
  the sequence of processes 
$\left\{ G_N f \colon f\in\mathcal{F}\right\}$
  converges 
  in
$l^\infty(\mathcal{F})$
to a Gaussian process with mean 0 and covariance function given by
\begin{gather}
  \mathrm{Cov}(f,g)
  :=
  \E[fg]-\E[f]\E[g]
  \,.
\end{gather}
\end{theorem}
\begin{proof}
  \cite[Theorem~19.5]{Vaart2000}
\end{proof}



\begin{lemma}
  \label{aa:mean:l:r4}
  Let
  $1/\pi(X)\in L^2(\P)$.
  $R_4$ converges
  converges in
  $l^\infty(\R)$
  to a Gaussian process with mean 0 and covariance
\begin{align*}
  &
  \mathbf{Cov}
  (z_1,z_2)
  \\
  &
  =\ 
  \E
  \left[ 
 \frac{
 F_{Y(1)}(z_1 \land z_2\,|\,X)
}{\pi(X)}
\ 
-
\ 
 \frac{1-\pi(X)}{\pi(X)}
 F_{Y(1)}(z_1|X)
 \cdot
 F_{Y(1)}(z_2|X)
  \right]
  \ 
 -
 \ 
 F_{Y(1)}(z_1)
 \cdot
 F_{Y(1)}(z_2)
\end{align*}

\end{lemma}
\begin{proof}
  By Lemma~\ref{aa:mean:r3:lem:fz_E} it follows
  \begin{align*}
    \E
    \left[
      \frac{f_z(T,X,Y(T))}{\pi(X)}
      +
      F_{Y(1)}(z|X)
      -
      F_{Y(1)}(z)
      \right]
      \ 
      =
      \ 
      \E
      \left[
      \frac{1}{\pi(X)}
      \E
      \left[
        f_z(T,X,Y(T))
        |X
      \right]
      \right]
      \ 
      =
      \ 
      0
      \,.
  \end{align*}
  Thus
  \begin{align*}
    R_4(z)
    &
  \
  =
  \ 
  \frac{1}{
  \sqrt{N}
  }
    \sum_{i=1}^{N} 
    \frac{T_i}{\pi(X_i)}
    \left( 
    \mathbf{1}_{\left\{ Y_i \le z \right\}}
    -
  F_{Y(1)}(z|X_i)
    \right)
    \ 
    +
    \ 
    \left( 
  F_{Y(1)}(z|X_i)
    -
  F_{Y(1)}(z)
    \right)
    \\
    &
    \ 
  =
    \ 
  \frac{1}{
  \sqrt{N}
  }
    \sum_{i=1}^{N} 
      \frac{f_z(T_i,X_i,Y_i)}{\pi(X_i)}
      +
      \left( 
      F_{Y(1)}(z|X_i)
      -
      F_{Y(1)}(z)
      \right)
      \\
      &
      \ 
      =
      \ 
      \G_N 
      \left(
       \frac{f_z}{\pi(\cdot)}
      +
      F_{Y(1)}(z|\cdot)
      -
      F_{Y(1)}(z)
      \right)
      \,.
  \end{align*}
  By Lemma~\ref{aa:mean:l:br}
  it holds
  \begin{align*}
    &
  \log
  N_{[\,]}
    (
    \varepsilon
    ,
    \mathcal{G}, L^2(\P))
    \\
    &
    \ 
    \lesssim
    \ 
    \log
    \left(
      \frac
      {
      1+
    \norm{1/\pi(X)}_{ L^2(\P)}
      }
      {\varepsilon}
    \right)
    \ 
    \lesssim
    \ 
      \frac
      {
      1+
    \norm{1/\pi(X)}_{ L^2(\P)}
      }
      {\varepsilon}
    \qquad
    \text{for all}
    \ 
    \varepsilon\in (0,1)
    \,.
  \end{align*}
  Thus
  \begin{gather*}
    J_{[\,]}(1,\mathcal{G},L^2(\P))
    \ 
    \lesssim
    \ 
    \int_0^1
    \sqrt
    {
      \frac
      {
      1+
    \norm{1/\pi(X)}_{ L^2(\P)}
      }
      {\varepsilon}
    }
    \,
    d\varepsilon
    \ 
    \lesssim
    \ 
      1+
    \norm{1/\pi(X)}_{ L^2(\P)}
    \ 
    <
    \ 
    \infty
    \,.
  \end{gather*}
But then $\mathcal{G}$ is $\P$-Donsker.
By the Donsker Theorem \cite[Theorem~19.5]{Vaart2000}
the process $R_4$ converges in $l^\infty(\R)$ to a Gaussian process, called $\P$-Brownian bridge, with mean 0.
We now calculate the covariance of the limiting process.
\subsubsection*{Covariance}
\begin{align*}
  &
  \E
  \left[
  \left( 
  f^{z_1}_{1/\pi}
  +
  F_{Y(1)}(z_1|X)
  -
F_{Y(1)}(z_1)
  \right)
  \left( 
  f^{z_2}_{1/\pi}
  +
  F_{Y(1)}(z_2|X)
  -
F_{Y(1)}(z_2)
  \right)
  \right]
  \\
  &
  \ 
  =
  \ 
\E
\left[
  f^{z_1}_{1/\pi}
  \cdot
  f^{z_2}_{1/\pi}
\right]
\\
  &
  \quad
  +
  \ 
  \E
  \left[
  f^{z_1}_{1/\pi}
  \left( 
  F_{Y(1)}(z_2|X)
  -
F_{Y(1)}(z_2)
  \right)
  \right]
  \ 
  +
  \ 
  \E
  \left[
  f^{z_2}_{1/\pi}
  \left( 
  F_{Y(1)}(z_1|X)
  -
F_{Y(1)}(z_1)
  \right)
  \right]
  \\
  &
  \quad
  +
  \ 
  \E
  \left[
  \left( 
  F_{Y(1)}(z_1|X)
  -
F_{Y(1)}(z_1)
  \right)
  \left( 
  F_{Y(1)}(z_2|X)
  -
F_{Y(1)}(z_2)
  \right)
  \right]
  \\
  &
  \ 
  =:
  \ 
  C_0
  \quad 
  +
  \quad 
  C_1
  +
  C_2
  \quad 
  +
  \quad 
  C_3
  \,.
\end{align*}
It holds
\begin{align*}
  C_0 
  &
  \ 
  =
  \ 
\E
\left[
  f^{z_1}_{1/\pi}
  \cdot
  f^{z_2}_{1/\pi}
\right]
\\
&
\ 
=
\ 
\E
\left[
\frac{1}{\pi(X)}
\frac{T}{\pi(X)}
\left( 
\mathbf{1}_{\left\{ Y(T)\,\le\, z_1 \right\}}
-
F_{Y(1)}(z_1|X)
\right)
\left( 
\mathbf{1}_{\left\{ Y(T)\,\le\, z_2 \right\}}
-
F_{Y(1)}(z_2|X)
\right)
\right]
\\
&
\ 
=
\ 
\E
\left[
\frac{1}{\pi(X)}
\left( 
\mathbf{1}_{\left\{ Y(1)\,\le\, z_1 \right\}}
-
F_{Y(1)}(z_1|X)
\right)
\left( 
\mathbf{1}_{\left\{ Y(1)\,\le\, z_2 \right\}}
-
F_{Y(1)}(z_2|X)
\right)
\right]
\\
&
\ 
=
\ 
\E
\left[
\frac{1}{\pi(X)}
\left( 
F_{Y(1)}(z_1\land z_2|X)
\ 
-
\ 
F_{Y(1)}(z_1|X)
\cdot
F_{Y(1)}(z_2|X)
\right)
\right]
\,.
\end{align*}
\begin{align*}
  C_1
  &
  \ 
  =
  \ 
 \E
  \left[
  f^{z_1}_{1/\pi}
  \left( 
  F_{Y(1)}(z_2|X)
  -
F_{Y(1)}(z_2)
  \right)
  \right]
  \\
  &
  \ 
  =
  \ 
 \E
  \left[
\frac{T}{\pi(X)}
\left( 
\mathbf{1}_{\left\{ Y(T)\,\le\, z_1 \right\}}
-
F_{Y(1)}(z_1|X)
\right)
  \left( 
  F_{Y(1)}(z_2|X)
  -
F_{Y(1)}(z_2)
  \right)
  \right]
  \\
  &
  \ 
  =
  \ 
 \E
  \left[
\left( 
\mathbf{1}_{\left\{ Y(1)\,\le\, z_1 \right\}}
-
F_{Y(1)}(z_1|X)
\right)
  \left( 
  F_{Y(1)}(z_2|X)
  -
F_{Y(1)}(z_2)
  \right)
  \right]
  \\
  &
  \ 
  =
  \ 
  0
  \,.
\end{align*}
In the same way we see $C_2=0$.
\begin{align*}
  C_3
  &
  \ 
  =
  \ 
  \E
  \left[
  \left( 
  F_{Y(1)}(z_1|X)
  -
F_{Y(1)}(z_1)
  \right)
  \left( 
  F_{Y(1)}(z_2|X)
  -
F_{Y(1)}(z_2)
  \right)
  \right]
  \\
  &
  \ 
  =
  \ 
  \E
  \left[
  F_{Y(1)}(z_1|X)
  \cdot
  F_{Y(1)}(z_2|X)
  \right]
  \ 
  -
  \ 
  F_{Y(1)}(z_1)
  \cdot
  F_{Y(1)}(z_2)
  \,.
\end{align*}
Adding up the results gives us \eqref{cov:lp}.
\end{proof}
We have gathered all the results to prove Theorem~\ref{aa:mean:th}.
\begin{proof}
  \emph{(Theorem~\ref{aa:mean:th})}
  We connect the statement of the theorem to the error decomposition by Lemma~\ref{aa:mean:lemma_decomp}.
  By Lemma~\ref{aa:mean:l:r1}, Lemma~\ref{aa:mean:l:r2},
  Lemma~\ref{aa:mean:r3:lem:conv}
   it follows 
   $\sup_{z\in\R}|R_i(z)|\overset{\P}{\to}0$ for $i=1,2,3$.
   Thus, by Slutzky's theorem (cf.\cite[Theorem~13.18]{Klenke2020})
   the behaviour of the limiting process is the one of Lemma~\ref{aa:mean:l:r4}.
\end{proof}

  \section{Application to Plug In Estimators}
  A plethora of applications of the delta method to estimates of the distribution function are to be found in \cite{Vaart2000} and \cite{vaart2013}.
This includes Quantile estimation \cite[§21]{Vaart2000}\cite[§3.9.21/24]{vaart2013},
survival analysis via Nelson-Aalen and Kaplan-Meier estimator\cite[§3.9.19/31]{vaart2013},
Wilcoxon Test~\cite[§3.9.4.1]{vaart2013},
and much more.
Maybe Boostrapping from the weighted distribution is also sensible .


\chapter{Convex Analysis}
In our application we want to analyse a convex optimization problem by its dual problem.
In particular we want to obtain primal optimal solutions from dual solutions.
To accomplish the task we need technical tools from convex analysis, 
mainly conjugate calculus and some KKT related results.

Our starting point is 
the support function intersection rule~\cite[Theorem 4.23]{Mordukhovich2022}.
We give the details in the case of finite dimensions and refer for the rest of the proof to the book.
The support function intersection rule is applied to give first conjugate sum and then chain rule,
which are vital to calculating convex conjugates. The proofs are omited, since the book is thorough enough. 
%The well known Fenche-Rockafellar Duality theorem is a corollary of conjugate sum and chain rule. It gives general conditions under which dual and primal values coincide.
The material we present is very well known.
As an introduction, we recommend the recent book \cite{Mordukhovich2022} and classical reference \cite{Rockafellar1970}.
We finish the chapter with ideas from \cite{Tseng1991}. 
They provide the high-level ideas to obtain for strictly convex
functions a dual relationship between optimal solutions.
We will deliver the details that are omited in the paper.
  \section{A Convex Analysis Primer}
  \subsection*{My Contribution}
I present the relevant facts from Convex analysis.
I prove some results that I did not find in the literature, but likely are folklore.

Throughout this section let $n\in\mathbb{N}$.
\subsection*{Sets}
A subset $C\subseteq \R^n$ is called \textbf{convex set}, 
\index{convex set}
if for all $x,y\in C$ and all $\theta\in [0,1]$,
we have 
$
  \theta x + (1-\theta)y 
  \in
  C
$.
Many set operations preserve convexity. Among them
forming the 
\textbf{Cartesian product} of two convex sets, 
\textbf{intersection} of a collection of convex sets and 
taking the \textbf{inverse image under linear functions}.

The classical theory evolves around the question 
if convex sets can be separated.
\begin{definition*}
  Let 
  $C_1$ and $C_2$
  be two non-empty convex sets in $\R^n$. 
  A hyperplane $H$ is said to \textbf{separate}
  $C_1$ and $C_2$
  if $C_1$ is contained in one of the closed half-spaces associated with
  $H$ and $C_2$ lies in the opposite closed half-space. It is said to separate 
  $C_1$ and $C_2$
  \textbf{properly} if 
  $C_1$ and $C_2$
  are not both contained in $H$.
\end{definition*}

We need a refined concept of interiors, since some convex sets have empty interior. To this end, 
  we call a set
  \index{affine set}
  $A\subseteq \R^n$ 
  \textbf{affine set}, if
  $
    \alpha x + (1-\alpha)y \in A
    \quad
    \text{for all}
    \ 
    x,y \in A
    \ 
    \text{and all}
    \ 
    \alpha \in \R
  $.
  The \textbf{affine hull} 
  \index{$\mathrm{aff(\cdot)}$, affine hull}
  $\mathrm{aff}(\Omega)$
  of a set 
  $\Omega\subseteq \R^n$
  is the smallest affine set that includes $\Omega$.
  We define the \textbf{relative interior}
  \index{$\mathrm{ri}(\cdot)$, relative interior}
  $\mathrm{ri}\,\Omega$ 
  of a set 
  $\Omega\subseteq \R^n$
  to be the interior relative to the affine hull, that is,
    \begin{gather}
    \mathrm{ri}(\Omega)
    \ 
    :=
    \ 
    \left\{ 
      x \in \Omega 
      \ 
      |
      \ 
      \exists
      \,
      \varepsilon > 0\ 
      \colon
      (
      x+\varepsilon B_{\R^n}
      )
      \cap
      \mathrm{aff}(\Omega)
      \ 
      \subset
      \ 
      \Omega
      \,
    \right\}
    \,.
  \end{gather}

\begin{ftheorem}
  \label{cv:primer:sep}
  \emph{(Convex separation in finite dimension)}
  Let $C_1$ and $C_2$ be two non-empty convex sets in $\R^n$. 
  Then $C_1$ and $C_2$ can be properly separated if and only if 
  $\mathrm{ri}(C_1)\cap\mathrm{ri}(C_2)=\emptyset.$
\end{ftheorem}
\begin{proof}
  \cite[Theorem~11.3]{Rockafellar1970}
\end{proof}
We collect some useful 
properties of relative interiors
before we get on to convex functions.
\begin{proposition}
  \label{cv:primer:prop}
  Let $C$ be a non-empty convex set in $\R^n.$ The following holds:
\begin{enumerate}[label={(\roman*)}]
  \item
    $
      \mathrm{ri}(C)
      \ 
      \neq
      \ 
      \emptyset
      $
if and only if
      $
      C
      \ 
      \neq
      \ 
      \emptyset
    $
  \item
    $
      \mathrm{cl}(\mathrm{ri}\,C)
      \ 
      =
      \ 
      \mathrm{cl}\,C
      $
      and
      $
      \mathrm{ri}(\mathrm{cl}\,C)
      \ 
      =
      \ 
      \mathrm{ri}(C)
    $
  \item
    $
    \mathrm{ri}(C)
      \ 
    =
      \ 
    \left\{ 
      z \in C
      \colon
      \text{for all}\ 
      x \in C \ 
      \text{there exists}\ 
      t > 0 \ 
      \text{such that}\ 
      z + t (z-x)
      \in C
    \right\}
    $
  \item
    Suppose
    $
      \bigcap_{i\in I} C_i
      \ 
      \neq
      \ 
      \emptyset
    $
    for a finite index set $I$.
    Then
    $
      \mathrm{ri}
      \left( 
        \bigcap_{i\in I} C_i
      \right)
      \ 
      =
      \ 
      \bigcap_{i\in I}  
      \mathrm{ri}(C_i)
    $.
    \item
      Let 
      $
        L\,:\,\R^n \to\  \R^m
      $
      be a linear function. Then
      $
        \mathrm{ri}\,L(C)
        \ 
        =
        \ 
        L(\mathrm{ri}\,C)
      $.
      If it also holds
      $
        L^{\!-1}(\mathrm{ri}\,C)
        \ 
        \neq
        \ 
        \emptyset
      $,
      we have
      $
      \mathrm{ri}\,L^{\!-1}(C)
      \ 
        =
      \ 
        L^{\!-1}(\mathrm{ri}\,C)
      $.
      \item
        $
          \mathrm{ri}(C_1\!\times C_2)
          \ 
          = 
          \ 
          \mathrm{ri}\,C_1
          \! 
          \times
          \mathrm{ri}\,C_2
        $
\end{enumerate}

\end{proposition}


\begin{proof}
  For a proof of (i)-(v) we refer to~\cite[Theorem 6.2 - 6.7]{Rockafellar1970}.

To prove (vi) we use (iii).
Let
  $
  (z_1, z_2)
  \in 
  \mathrm{ri}(C_1\!\times C_2).
  $
  Then for all 
  $
  (x_1, x_2)
  \in 
  C_1\!\times C_2
  $
  there exists
  $t>0$
  such that
  \begin{gather}
    \label{cv:primer:prop:1}
      z_i + t (z_i-x_i)
      \in C_i
      \qquad
      \text{for all}\ 
      i\in \left\{ 1,2 \right\}.
  \end{gather}
  Using (iii) again, we get
  $
  \mathrm{ri}(C_1\!\times C_2)
  \, 
  \subseteq
  \,
          \mathrm{ri}\,C_1
          \! 
          \times
          \mathrm{ri}\,C_2
  $.
  Suppose 
  $
  (z_1,z_2)
    \in
    \mathrm{ri}\,C_1
    \!
    \times
    \mathrm{ri}\,C_2
  $.
  By (iii), for all
  $
    (x_1,x_2)\in C_1\times C_2
  $
  there exist
  $
    (t_1,t_2)>0
  $
  such that
  \begin{gather}
    \label{cv:primer:prop:2}
      z_i + t_i (z_i-x_i)
      \in C_i
      \qquad
      \text{for all}\ 
      i\in \left\{ 1,2 \right\}.
  \end{gather}
  If $t_1=t_2$
  we recover
  \eqref{cv:primer:prop:1}
  from
  \eqref{cv:primer:prop:2}.
  By (iii) it holds
  $
  (z_1,z_2)
    \in
    \mathrm{ri}
    (C_1
    \!
    \times
    C_2)
  $.
  If
  $t_1<t_2$
  we
  define $\theta:=\frac{t_1}{t_2}\in (0,1).$
  Consider  
  \eqref{cv:primer:prop:2} with $i=2$,
  together with $z_2 \in C_2$
  and
  the convexity of $C_2$.
  It follows
  \begin{gather}
    \label{cv:primer:prop:3}
    z_2 + t_1 (z_2 - x_2)
    \ 
    =
    \ 
    \theta
    \cdot
    (
    z_2 + t_2 (z_2 - x_2)
    )
    \ 
    +
    \ 
    (1-\theta)
    \cdot
    z_2
    \in C_2
    \,.
  \end{gather}
  Now we consider
  \eqref{cv:primer:prop:3} and
  \eqref{cv:primer:prop:2} with $i=1$.
  This gives \eqref{cv:primer:prop:1} with $t=t_1$.
  As before, it follows
  $
  (z_1,z_2)\in\mathrm{ri}(C_1\!\times C_2)
  $.
  If 
  $t_1>t_2$
  similar arguments lead to the same result.
  We have proven 
  $
  \mathrm{ri}(C_1\!\times C_2)
  \, 
  \supseteq
  \,
          \mathrm{ri}\,C_1
          \! 
          \times
          \mathrm{ri}\,C_2
  $
  and equality.
\end{proof}
\subsection*{Functions}
A function 
$
f
\colon
\R^n
\to
\overline{\R}
$
is called \textbf{convex function},
\index{convex function}
if the area above its graph, that is, its epigraph(cf.\cite[§2.4.1]{Mordukhovich2022}), is convex. We shall often use an equivalent definition.
To this end, 
a function $f$ is convex if and only if 
\begin{gather}
  \label{cv:cf}
  f(\theta x + (1-\theta)y)
  \ 
  \le
  \ 
  \theta f(x)
  +
  (1-\theta)f(y)
  \qquad
  \text{for all}\ 
  x,y\in \R^n
  \ 
  \text{and all}\ 
  \theta\in[0,1]
  \,.
\end{gather}
This definition extends to convex cominbinations
$
  \theta_1,\ldots,\theta_m\in[0,1]
$
with
$
  \sum_{i=1}^{m} 
  \theta_i
  =1
$, that is, 
a function $f$ is convex if and only if 
\begin{gather}
  f
  \left( 
    \sum_{i=1}^{m} 
    \theta_i
    x_i
  \right)
  \ 
  \le
  \ 
    \sum_{i=1}^{m} 
    \theta_i
    f(x_i)
  \qquad
  \text{for all}\ 
  x_1,\ldots,x_m\in \R^n
  \,.
\end{gather}
We call a function \textbf{strictly convex} if the inequality in
\index{strictly convex}
\eqref{cv:cf} is strict.

We define the \textbf{domain} $\mathrm{dom}\,f$
\index{$\mathrm{dom}\,f$, domain of a convex function}
of a convex function $f$ to be the set where $f$ is finite, that is,
\begin{gather}
  \mathrm{dom}\,f
  \ 
  :=
  \ 
  \left\{ 
x\in\R^n
:
f(x)<\infty
  \right\}
  \,.
\end{gather}
The domain of a convex function is convex. 
We say that $f$ is a \textbf{proper function} if  $\mathrm{dom}\,f\neq\emptyset$. 
\index{proper (convex) function}

For any $\overline{x}\in\mathrm{dom}\,f$ we call $x^*\in\R^n$ a 
\textbf{subgradient} of $f$ at $\overline{x}$ if for all 
\index{subgradient}
$x\in\R^n$ it holds
\begin{gather}
  \inner{x^*}{x-\overline{x}}
  \le
  f(x)
  -
  f(\overline{x})
  \,.
\end{gather}
We denote the collection of all subgradients at $\overline{x}$, that is, the \textbf{subdifferential} of $f$ at $\overline{x}$, as
\index{
  $
\partial f(x)
  $,
  subdifferential of $f$ at $x$
}
$
\partial f(\overline{x})
$.
If $f$ is differentiable at $\overline{x}$ it holds
$
\partial f(\overline{x})
=
\left\{ 
  \nabla
  f(\overline{x})
\right\}
$
and thus
\begin{gather}
  \label{cv:primer:mvthe}
  \inner{
  \nabla
  f(\overline{x})
}{x-\overline{x}}
  \le
  f(x)
  -
  f(\overline{x})
  \,.
\end{gather}
\begin{definition}
  \index{$\sigma_\omega$, support function}
  Given a nonempty subset 
  $\Omega \subseteq \R^n$,
  we define
  the \textbf{support function} 
  $
  $
  of $\Omega$
  to be
  \begin{gather*}
  \sigma_\Omega 
  \,
  :
  \,
  \R^n \to\  \overline{\R}
  \,,
  \qquad
  x^*
  \ 
  \mapsto
  \ 
    \sup_{x \in \Omega}
    \ 
    \inner{x^*\!}{x}
    \,.
  \end{gather*}
\end{definition}


\begin{definition}
  \index{$f\square g$, infimal convolution of $f$ and $g$}
  Given functions
  $
    f_i\,:\,
    \R^n \to\  \overline{\R}
  $
  for $ i = 1, \ldots, m $,
  we define the \textbf{infimal convolution} of these functions to be
  \begin{gather*}
    f_1 \square \cdots \square f_m
    \ 
    \colon
    \ 
    \R^n
    \to
    \ 
    \overline{\R}
    \,,
    \quad
    x
    \ 
    \mapsto
    \ 
    \inf
    \left\{ 
    \sum_{i = 1}^{m}
      f_i(x_i)
      \ 
      \colon
      \ 
      x_i \in \R^n 
      \ 
      \mathrm{and}\ 
      \sum_{i = 1}^{m} 
        x_i
      =
      x
    \right\}
    \,.
  \end{gather*}
\end{definition}
 
The next result establishes a connection between the support function of the intersection of two convex sets and the infimal convolution of the support functions of the sets taken by themselfes.
The proof translates the geometric concept of convex separation to the world of convex functions.

\begin{lemma}
  \label{cv:primer:lem}
  Let $C_1$ and $C_2$ be two non-empty convex sets in $\R^n$.
  For any
  $ x^* \in \mathrm{dom}\, \sigma_{C_1\cap C_2} $
  the sets
  \begin{align*}
    \Theta_1
    &
    \ :=\ 
    C_1 \times [\,0,\infty)
    \,,
    \\
    \Theta_2
    (x^*)
    &
    \ :=\ 
    \left\{ 
      (x,\lambda)\in \R^n
      \ 
      \colon
      \ 
      x \in C_2
      \ 
      \text{and}
      \ 
      \lambda
      \,
      \le
      \,
      \inner{x^*\!}{x} 
      \ 
      -
      \ 
      \sigma_{C_1\cap C_2}(x^*)
    \right\}
  \end{align*}
  can by properly separated.
\end{lemma}
\begin{proof}
  We fix 
  $ x^* \in \mathrm{dom}\, \sigma_{C_1\cap C_2} $
  and write
  $ 
  \alpha
  \ 
  :=
  \ 
  \sigma_{C_1\cap C_2}(x^*)
  $.
  In order to apply convex separation in finite dimension 
  (Theorem~\ref{cv:primer:sep})
  to
  the sets
  $ \Theta_1 $ and $ \Theta_2(x^*) $,
  it suffics to show
  their convexity and
  $
    \mathrm{ri}\, 
    \Theta_1
    \cap
    \mathrm{ri}\, 
    \Theta_2(x^*)
    =
    \emptyset
  $.
  \subsubsection*{Convexity of 
  $ \Theta_1 $ and $ \Theta_2(x^*) $
  }
  Clearly, 
  $ \Theta_1 $ is convex by the convexity of 
  $ C_1 $ and $ [0,\infty) $.
 To see that $\Theta_2(x^*)$ is convex consider the linear function
 \begin{gather*}
    L
    \,
    :
    \,
    \R^n\times\,  \R 
    \ 
    \to
    \ 
    \R
    \,,
    \qquad 
    (x,\lambda)
    \ 
    \mapsto
    \ 
    \inner{x^*\!}{x} - \lambda
    \,.
 \end{gather*}
 From the definitions of $L$ and $\Theta_2(x^*)$ we get 
  \begin{gather*}
 \Theta_2
 (x^*)
    \ 
    =
    \ 
    (
    C_2\!\times\R
    )
    \ 
    \cap
    \ 
    L^{\!-1}
    [\,\alpha,\infty)
    \,
    .
  \end{gather*}
  Thus,
  by
  Proposition~\ref{cv:primer:prop}~(v)
  and the convexity of $C_2$ we get the convexity of
  $
    L^{\!-1}
    [\,\alpha,\infty)
  $ and with it that of $\Theta_2(x^*)$.

  \subsubsection*{Relative interiors of
  $ \Theta_1 $ and $ \Theta_2(x^*) $
  are disjoint}
  We start by calculating the relative interiors. It holds
  \begin{alignat*}{3}
    \mathrm{ri}\,
    \Theta_1
    &
    \ 
    =
    \ 
    \mathrm{ri}
    ( C_1\times [0,\infty) )
    &&
    \ 
    =
    \ 
    \mathrm{ri}\,
    C_1
    \!
    \times
    \mathrm{ri}\,
    [0,\infty)
    \ 
    =
    \ 
    \mathrm{ri}\,
    C_1
    \!
    \times
    (0,\infty)
    \,,
    %%%%%%%%%%%%%%%%%
    \\
    \mathrm{ri}\,
    \Theta_2(x^*)
    & 
    \ 
    =
    \ 
    \mathrm{ri}
    (
    L^{\!-1}
    [\,\alpha,\infty)
    )
    &&
    \ 
    =
    \ 
    L^{\!-1}
    (
    \mathrm{ri}\,
    [\,\alpha,\infty)
    )
    \ 
    \ 
    =
    \ 
    L^{\!-1}
    (\alpha,\infty)
    \,.
  \end{alignat*}
  Suppose there exists
  $ (\lambda,x) \in
    \mathrm{ri}\, 
    \Theta_1
    \cap
    \,
    \mathrm{ri}\, 
    \Theta_2(x^*)
  $.
  Then it holds 
  $ x \in C_1\!\times C_2 $
  and 
  $ \lambda >0 $.
  We also note, that
  \begin{gather*}
  \alpha
  \ 
  =
  \ 
  \sigma_{C_1\cap\, C_2}(x^*)
    \ 
  =
    \ 
  \sup_{z \in C_1\cap\, C_2}
  \inner
  {x^*}
  {z}
  \ 
  \ge
  \ 
  \inner
  {x^*}
  {x}
  \,.
  \end{gather*}
  Then it follows
  \begin{gather*}
    \alpha
    \ 
    <
    \ 
  \inner
  {x^*}
  {x}
  - \lambda
    \ 
  \le
    \ 
  \alpha\,,
  \end{gather*}
  a contradiction.
  Thus, the relative interiors of
  $ \Theta_1 $ and $ \Theta_2(x^*) $
  are disjoint.

  Applying Theorem~\ref{cv:primer:sep} finishes the proof.
\end{proof}


\begin{theorem*}
  Let $C_1$ and $C_2$ be two non-empty convex sets in $\R^n$ with
  $\mathrm{ri}\,C_1\cap\mathrm{ri}\,C_2\neq\emptyset.$
  Then the support function of the intersection 
  $
    C_1\! \cap C_2
  $
  is represented as
  \begin{gather}
    (\sigma_{
    C_1 \cap\, C_2
    })
    (x^*)
    =
    (\sigma_{C_1}\square \,\sigma_{C_2})
    (x^*)
    \qquad
    \text{for all}\ 
    x^* \in \R^n.
  \end{gather}
  Furthermore, for any
  $
  x^*\in \mathrm{dom}
    (\sigma_{
    C_1 \cap\, C_2
    })
  $
  there exist dual elements 
  $
    x_1^*
    ,
    x_2^*
    \in \R^n
  $ 
  such that 
  $
    x^*
    =
    x_1^*
    +
    x_2^*.
  $
  and
  \begin{gather}
    (\sigma_{
    C_1 \cap\, C_2
    })
    (x^*)
    =
    \sigma_{C_1}(x_1^*)
    +
    \sigma_{C_2}(x_2^*).
  \end{gather}
\end{theorem*}
\begin{proof}
  Using Lemma~\ref{cv:primer:lem}
  the rest of the proof is as that of
  \emph{\cite[Theorem~4.23(b)]{Mordukhovich2022}}.
\end{proof}

\begin{takeaways}
  The support function intersection rule connects the geometric 
  property of convex separation to an identity of support functions
  This result is central to the analysis of convex conjugates.
\end{takeaways}
One important application of convex functions is in optimization.
There we often analyse a dual problem instead, which relies on the notion of \textbf{convex conjugate} 
\index{$f^*$, convex conjugate of f}
$
    f^*:
    \R^n \to \overline{\R}
  $
  of $f$ defined by
  \begin{gather}
    \label{def:convex_conjugate}
    f^*(x^*)
    :=
    \sup_{ x \in \R^n }
    \inner
    {x^*}{x}
    - f(x)
    \,.
  \end{gather}
  Even for arbitrary functions, the convex conjugate is convex(cf.
  \cite[Proposition~4.2]{Mordukhovich2022}
  ).
  Like in differential calculus, there exist sum and chain rule for computing the convex conjugate.
\begin{theorem}
  \index{conjugate sum rule}
  Let
  $
    f,g:
    \R^n \to (-\infty, \infty]
  $
  be proper convex functions 
  and
  \begin{align*}
  \text{ri}\left( \text{dom}(f) \right)
  \cap
  \text{ri}\left( \text{dom}(g) \right)
  \neq 
  \emptyset
  \,.
  \end{align*}
  Then we have the 
  \textbf{
  conjugate sum rule
  }
  \begin{gather}
    ( f + g )^*(x^*)
    =
    ( f^* \square g^*)(x^*)
  \end{gather}
  for all $x^* \in \R^n$.
  Moreover, the infimum in 
  $
    ( f^* \square g^*)(x^*)
  $
  is attained, i.e., for any
  $
    x^* \in \text{dom}(f+g)^*
  $
  there exists vectors $x_1^*, x_2^*$
  for which
  \begin{gather}
    (f+g)^*(x^*)
    =
    f^*(x_1^*)
    +
    g^*(x_2^*),
    \quad
    x^* = x_1^* + x_2^*.
  \end{gather}
\end{theorem}
\begin{proof}
  \cite[Theorem~4.27(c)]{Mordukhovich2022}
\end{proof}



% conjugate chain rule %
 %%%%%%%%%%%%%%%%%%%%%%
\begin{theorem}
  \index{conjugate chain rule}
  \label{cvxa_conjugate_chain_rule}
  Let 
  $
    A:
      \R^m \to \R^n
  $
  be a linear map (matrix)
  and
  $
    g:
      \R^n \to (-\infty, \infty]
  $
  a proper convex function. If
  $
    \text{Im}(A) \cap \text{ri}(\text{dom}(g))
    \neq
    \emptyset
  $
  it follows
  the 
  \textbf{conjugate chain rule}
  \begin{gather}
    ( g \circ A )^* ( x^* )
    =
    \inf_
          { y^* \in ( A^* )^{ -1 } ( x^* )}
                                          g^*( y^* )
                                          .
  \end{gather}
  Furthermore, 
    for any 
      $
        x^* \in \text{dom}( g \circ A)^*
      $
        there exists
          $
            y^* \in ( A^* )^{ -1 } ( x^* )
          $
            such that
              $
                ( g \circ A)^* ( x^* )
                =
                g^*( y^* )
              $.
\end{theorem}
\begin{proof}
  \cite[Theorem~4.28(c)]{Mordukhovich2022}
\end{proof}
%%%%%%%%%%%%%%%%%
%%%% EXAMPLE %%%%
%%%%%%%%%%%%%%%%%
\begin{example}
  \label{cv:cc:ex}
  Let 
  $
    f:\R\to \overline{\R}
  $
  be a proper convex function, that is, 
  $
    \mathrm{dom}\,f
    \neq
    \emptyset
  $
  and $f$ is convex.
  In steps we apply the conjugate chain and sum rule, together with mathematical induction,
  to prove the conjugate relationship 
  \begin{align*}
    &S_{f,n}:\R^n \to \overline{\R},
    \qquad
    (x_1,\ldots,x_n)
    \mapsto
    \sum_{i=1}^{n} 
    f(x_i)
    ,
    \\
    &S_{f,n}^*:\R^n \to \overline{\R},
    \qquad
    (x^*_1,\ldots,x^*_n)
    \mapsto
    \sum_{i=1}^{n} 
    f^*(x^*_i)
    \,.
  \end{align*}
  This relationship is very natural and the ensuing calculations serve to confirm our intuition.

  First, we work in the projections on the coordinates. 
  For the $i$-th coordinate, where $i=1,\ldots,n$, this is 
  \begin{gather}
    p_i:\R^n\to \R
    ,
    \quad
    (x_1,\ldots,x_n)
    \mapsto
    x_i\,.
  \end{gather}
  All projections 
  $p_i$
  are linear function with matrix representation
  $
    e_i^\top
  $,
  where $e_i$ is $i$-the coordinate vector.
  The adjoint of $p_i$ is therefore
  \begin{gather}
    p^*_i:\R\to \R^n
    ,
    \quad
    x
    \mapsto
    e_i\cdot x
    \,.
  \end{gather}
  For the inverse image of the adjoint of $p_i$ it holds
  \begin{gather}
    (p_i^*)^{-1}
    \left\{ 
    (x_1^*,\ldots,x_n^*)
    \right\}
    \ 
    =
    \ 
    \begin{cases}
      \left\{ x_i^* \right\},
      \quad
      &\text{if}\ 
      x_j^*=0\ \text{for all}\ j\neq i\,,
      \\
      \ \ \emptyset
      \quad
      &\text{else.}
    \end{cases}
  \end{gather}
  Throughout this example we use the asterisk character $^*$ somewhat inconsistently. 
  Note that $f^*$ is the convex conjugate 
  of the function $f$ and $p_i^*$ is the adjoint linear function of the projection on the $i$-th coordinate. Likewise, we denote dual variables, that is, the arguments of convex conjugates, as $x^*$.

  Next, we employ the conjugate chain rule to establish the conjugate relationship 
  \begin{align*}
    f_i&:\R^n\to \overline{\R}
    ,
    \quad
    (x_1,\ldots,x_n)
    \mapsto x_i \mapsto f(x_i)
    \,,
    \\
    f^*_i&:\R^n\to \overline{\R}
    ,
    \quad
    (x^*_1,\ldots,x^*_n)\mapsto 
    \begin{cases}
      f^*(x_i^*),
      \quad
      &\text{if}\ 
      x_j^*=0\ \text{for all}\ j\neq i\,,
      \\
      \infty
      \quad
      &\text{else.}
    \end{cases}
  \end{align*}
  Note, that 
  $
    f_i
    =
    (f\circ p_i)
  $
  and
  $
    f^*_i
    =
    (f\circ p_i)^*
  $.
  Since 
  $
    \mathrm{Im}\,p_i=\R
  $
  and 
  $
    \mathrm{dom}\, f
    \neq
    \emptyset
  $,
  it holds
  $
    \mathrm{Im}\, p_i
    \cap
    \mathrm{ri}(
    \mathrm{dom}\, f
    )
    \neq
    \emptyset
  $.
  Then $f$ and $p_i$ conform with the demands of the conjugate chain rule.
  It follows
  \begin{align*}
    &f_i^*
    (x^*_1,\ldots,x^*_n) 
    \ 
    =
    \ 
    (f\circ p_i)^*
    (x^*_1,\ldots,x^*_n) 
    \ =
    \ 
    \inf
    \left\{ 
    f^*(y)
    \ 
    |
    \ 
    y\in 
    (p_i^*)^{-1}
    \left\{ 
    (x_1^*,\ldots,x_n^*)
    \right\}
    \right\}
    \\
    &\quad=
    \ 
    \begin{cases}
      f^*(x_i^*),
      \quad
      &\text{if}\ 
      x_j^*=0\ \text{for all}\ j\neq i\,,
      \\
      \infty
      \quad
      &\text{else,}
    \end{cases}
  \end{align*}
  where we keep to the convention $\inf\emptyset=\infty$.
  In the same way it follows
  \begin{gather}
    \left( 
      S_{f,n}
      \circ
      p_{\left\{ 1,\ldots,n \right\}}
    \right)^*
    (x^*_1,\ldots,x^*_{n+1})
    =
    \begin{cases}
      S_{f,n}^*
    (x^*_1,\ldots,x^*_{n})
      \quad
      &\text{if}\ 
      x_{n+1}^*=0\,,
      \\
      \infty
      \quad
      &\text{else,}
    \end{cases}
  \end{gather}

  Next, note that for $n=1$ we arrive at the result. Thus, for some $n\in \mathbb{N}$ it holds
  $
  \left( 
    S_{f,n}
  \right)
  ^*
  =
    S_{f,n}^*
  $.
  In order to apply the conjugate sum rule to 
  $
    S_{f,n}
  $
  and
  $
    f_{n+1}
  $
  we note that
  \begin{align*}
    \mathrm{dom}\, f_i
    &
    \ =\ 
    \left\{ 
      (x_1,\ldots,x_{n+1})
      \in \R^{n+1}
      :
      x_i\in \mathrm{dom}\,f
    \right\}
    \ 
    \neq 
    \ 
    \emptyset
    \qquad
    \text{for all}
    \ 
    i=1,\ldots,n+1
    \,,
    \\
    \bigcap_{i=1}^{n+1}
    \mathrm{dom}\, f_i
    &
    \ =\ 
    \left\{ 
      (x_1,\ldots,x_{n+1})
      \in \R^{n+1}
      :
      x_i\in \mathrm{dom}\,f
      \ 
    \text{for all}
    \ 
    i=1,\ldots,n+1
    \right\}
    \ 
    \neq 
    \ 
    \emptyset
    \,,
  \end{align*}
  and
\begin{align*}
  &
  \mathrm{ri}\left( 
    \mathrm{dom}
    \left( 
    S_{f,n}
    \circ
    p_{\left\{ 1,\ldots,n \right\}}
    \right)
  \right)
  \ 
  \cap
  \ 
  \mathrm{ri}\left( 
    \mathrm{dom}\,f_{n+1}
  \right)
  \\
  &
  \hspace{25mm}
  =
  \ 
  \mathrm{ri}\left( 
    \mathrm{dom}
    \left( 
    S_{f,n}
    \circ
    p_{\left\{ 1,\ldots,n \right\}}
    \right)
  \ 
  \cap
  \ 
    \mathrm{dom}\,f_{n+1}
  \right)
  \ 
  =
  \ 
  \mathrm{ri}
  \left( 
    \bigcap_{i=1}^{n+1}
    \mathrm{dom}\, f_i
  \right)
  \ 
  \neq
  \ 
  \emptyset
  \,.
\end{align*}
By the conjugate sum rule it follows
\begin{align*}
  (
  S_{f,n+1}
  )^*
  =
  (
    S_{f,n}
    \circ
    p_{\left\{ 1,\ldots,n \right\}}
  +
  f_{n+1}
  )^*
  =
  (
    S_{f,n}
    \circ
    p_{\left\{ 1,\ldots,n \right\}}
  )
  ^*
  \square
  f_{n+1}^*
  \\
  =
    S_{f,n}^*
    \circ
    p_{\left\{ 1,\ldots,n \right\}}
    +
  f_{n+1}^*
  =
  S^*_{f,n+1}
  \,.
\end{align*}
\end{example}






  %\section{Conjugate Calculus}
  %The goal of this section is to establish the tools to calculate convex conjugates. 
We prove the conjugate sum and chain rule.
After some examples, we will derive the Fenchel-Rockafellar Theorem.
\begin{definition}
  \label{ def_convex_conjugate }
  \emph{(Convex conjugate)}
  Given a function
  $
    f:
    \R^n \to \overline{\R}
  $
  ,
  the 
  \textbf{convex conjugate}
  $
    f^*:
    \R^n \to \overline{\R}
  $
  of $f$ is defined as
  \begin{gather}
    f^*(x^*)
    :=
    \sup_{ x \in \R^n }
    (x^*)^T x - f(x)
  \end{gather}
\end{definition}

Note that $f$ in Definition~\ref{ def_convex_conjugate }
does not have to be convex. On the other hand, the convex conjugate is always convex:

\begin{proposition}
  Let  
  $
    f:
    \R^n \to ( - \infty, \infty ]
  $
  be a proper function. 
  Then its convex conjugate
  $
    f^*:
    \R^n \to ( - \infty, \infty ]
  $
  is convex.
\end{proposition}


\begin{lemma}
  For any proper function
  $
    f:\R^n\to\overline{\R}
  $
  we have
  \begin{gather}
    f^*(x^*) 
    =
    \sigma_{\mathrm{epi}(f)}
    (x^*,-1)
    \qquad
    \text{for}
    \ 
    x^* \in \R^n.
  \end{gather}
\end{lemma}
\begin{proof}
  Let $x^*\in\R^n$
  and
  $
    (x,\lambda)\in \mathrm{epi}(f).
  $
  Then
  $
    x \in \mathrm{dom}(f)
  $
  and
  $
    f(x)\le \lambda.
  $
  Thus
  \begin{gather}
    \inner{x^*}{x} - f(x)
    \ge
    \inner{x^*}{x} - \lambda
    \qquad
    \text{for all}\ 
    (x,\lambda)\in \mathrm{epi}(f).
  \end{gather}
  On the other hand 
  $
    (x,f(x))\in \mathrm{epi}(f)
  $
  for all
  $
    x \in \mathrm{dom}(f).
  $
  It follows
  \begin{gather}
    \inner{x^*}{x} - f(x)
    \le
    \sup_{(x,\lambda)\in\mathrm{epi}(f)}
    \inner{x^*}{x} - \lambda
    \qquad
    \text{for all}\ 
    x \in \mathrm{dom}(f).
  \end{gather}
  Taking the supremum in the last two displays yields
  \begin{align}
    f^*(x^*)
    =
    \sup_{x\in\mathrm{dom}(f)}
    \inner{x^*}{x} - f(x)
    &=
    \sup_{(x,\lambda)\in\mathrm{epi}(f)}
    \inner{x^*}{x} - \lambda
    \\
    &=
    \sup_{(x,\lambda)\in\mathrm{epi}(f)}
    \inner{(x^*,-1)}{(x,\lambda)} 
    =
    \sigma_{\mathrm{epi}(f)}
    (x^*,-1).
  \end{align}
\end{proof}
% conjugate chain rule %
 %%%%%%%%%%%%%%%%%%%%%%
\begin{theorem}
  \emph{(Conjugate Chain Rule)}
  \label{cvxa_conjugate_chain_rule}
  Let 
  $
    A:
      \R^m \to \R^n
  $
  be a linear map (matrix)
  and
  $
    g:
      \R^n \to (-\infty, \infty]
  $
  a proper convex function. If
  $
    \text{Im}(A) \cap \text{ri}(\text{dom}(g))
    \neq
    \emptyset
  $
  it follows
  \begin{gather}
    ( g \circ A )^* ( x^* )
    =
    \inf_
          { y^* \in ( A^* )^{ -1 } ( x^* )}
                                          g^*( y^* )
                                          .
  \end{gather}
  Furthermore, 
    for any 
      $
        x^* \in \text{dom}( g \circ A)^*
      $
        there exists
          $
            y^* \in ( A^* )^{ -1 } ( x^* )
          $
            such that
              $
                ( g \circ A)^* ( x^* )
                =
                g^*( y^* )
              $.
\end{theorem}

% conjugate sum rule %
 %%%%%%%%%%%%%%%%%%%%



\begin{theorem}
  Let
  $
    f,g:
    \R^n \to (-\infty, \infty]
  $
  be proper convex functions 
  and
  $
  \text{ri}\left( \text{dom}(f) \right)
  \cap
  \text{ri}\left( \text{dom}(g) \right)
  \neq 
  \emptyset
  .
  $
  Then we have the conjugate sum rule
  \begin{gather}
    ( f + g )^*(x^*)
    =
    ( f^* \square g^*)(x^*)
  \end{gather}
  for all $x^* \in \R^n$.
  Moreover, the infimum in 
  $
    ( f^* \square g^*)(x^*)
  $
  is attained, i.e., for any
  $
    x^* \in \text{dom}(f+g)^*
  $
  there exists vectors $x_1^*, x_2^*$
  for which
  \begin{gather}
    (f+g)^*(x^*)
    =
    f^*(x_1^*)
    +
    g^*(x_2^*),
    \quad
    x^* = x_1^* + x_2^*.
  \end{gather}
\end{theorem}
\begin{proof}
  Let $x^*\in\R^n$ and fix $x_1^*,x_2^*\in\R^n$ such that
  $x^*=x^*_1+x^*_2$.
  We get
  \begin{align*}
    f^*(x^*_1)+g^*(x^*_2)
    &=
    \sup_{x\in\R^n}
    \inner{x^*_1}{x}-f(x)
    +
    \sup_{x\in\R^n}
    \inner{x^*_2}{x}-g(x)
    \\
    &\ge
    \sup_{x\in\R^n}
    \inner{x^*_1}{x}-f(x)
    +
    \inner{x^*_2}{x}-g(x)
    =
    \sup_{x\in\R^n}
    \inner{x^*_1+x^*_2}{x}-(f(x)+g(x))
    \\
    &=
    \sup_{x\in\R^n}
    \inner{x^*}{x}-(f+g)(x)
    =(f+g)^*(x^*)
  \end{align*}
  Taking the infimum over $x_1^*,x_2^*\in\R^n$ in the above display gives 
  $
  (f^*\square g^*)(x^*)
  \ge
  (f+g)^*(x^*).
  $
  Let us prove now $\le$ under the condition
  $
  \text{ri}\left( \text{dom}(f) \right)
  \cap
  \text{ri}\left( \text{dom}(g) \right)
  \neq 
  \emptyset
  .
  $
  The only case we need to consider is
  $
    (f+g)^*(x^*)<\infty.
  $
  Define two convex sets by
  \begin{align}
    \Omega_1
    &:=
    \left\{ 
      (x,\alpha,\beta)\in\R^{n+2}
      \colon
      \alpha\ge f(x)
    \right\}
    =
    \mathrm{epi}(f)\times \R,
    \\
    \Omega_2
    &:=
    \left\{ 
      (x,\alpha,\beta)\in\R^{n+2}
      \colon
      \beta\ge g(x)
    \right\}.
  \end{align}
  Similar to Lemma we get the representation
  \begin{gather}
    (f+g)^*(x^*)
    =
    \sigma_{\Omega_1\cap\Omega_2}
    (x^*,-1,-1).
  \end{gather}
  Indeed, the only thing we need to verify is
  $
    \mathrm{dom}(f)\cap\mathrm{dom}(g)
    =
    \mathrm{dom}(f+g).
  $
  The inclusion $\subseteq$ is clear.
  Assume towards a contradiction that
  $
    (f+g)(x)<\infty
  $
  and
  $
    f(x)=\infty.
  $
  Since $g(x)>-\infty$ it holds
  \begin{gather}
    \infty
    =
    \infty+g(x)
    =f(x)+g(x)
    =(f+g)(x)
    <
    \infty.
  \end{gather}
  This is a contradiction. The same holds for $f$ and $g$ reversed. It follows the inclusion $\supseteq$ and equality.
  By the support function intersection rule there exist triples
  \begin{gather}
    (x^*_1,-\alpha_1,-\beta_1),
    (x^*_2,-\alpha_2,-\beta_2)
    \in \R^{n+2}
    \quad
    \text{such that}
    \quad
    (x^*,-1,-1)
    =
    (x^*_1+x^*_2,-(\alpha_1+\alpha_2),-(\beta_1+\beta_2))
  \end{gather}
  and
  \begin{gather}
    (f+g)^*(x^*)
    =
    \sigma_{\Omega_1\cap\Omega_2}
    (x^*,-1,-1)
    =
    \sigma_{\Omega_1}
    (x^*_1,-\alpha_1,-\beta_1)
    +
    \sigma_{\Omega_2}
    (x^*_2,-\alpha_2,-\beta_2).
  \end{gather}
  Next we show
  $\beta_1=\alpha_2=0.$
  Suppose towards a contradiction that 
  $\beta_1\neq 0.$ 
  We fix 
  $(\overline{x},\overline{\alpha})\in\mathrm{epi}(f).$
  Then
  \begin{gather}
    \sigma_{\Omega_1}
    (x^*_1,-\alpha_1,-\beta_1)
    =
    \sup_{(x,\alpha,\beta)\in \mathrm{epi}(f)\times \R}
    \inner{x^*}{x}-\alpha \alpha_1 -\beta \beta_1
    \ge
    \sup_{\beta\in \R}
    \inner{x^*}{\overline{x}}-\overline{\alpha} \alpha_1 -\beta \beta_1
    =\infty.
  \end{gather}
  This contradicts
  $
    (f+g)^*(x^*)<\infty.
  $
  In a similar fashion we can derive a contradiction for $\alpha_2\neq0.$
  Employing Lemma and taking into account the structures of the sets 
  $\Omega_1$ and $\Omega_2$ this implies
  \begin{align}
    (f+g)^*(x^*)
    &=
    \sigma_{\Omega_1\cap\Omega_2}
    (x^*,-1,-1)
    =
    \sigma_{\Omega_1}
    (x^*_1,-1,0)
    +
    \sigma_{\Omega_2}
    (x^*_2,0,-1)
    \\
    &=
    \sigma_{\mathrm{epi}(f)}(x^*_1,-1)
    +
    \sigma_{\mathrm{epi}(g)}(x^*_2,-1)
    =
    f^*(x^*_1)
    +
    g^*(x^*_2)
    \ge
    (f^*\square g^*)(x^*).
  \end{align}
  This finishes the proof.
\end{proof}




Given 
proper convex functions $f, g : \R^n \to \overline{\R}$ 
and
a matrix $A \in \R^{n \times n}$,
we define 
the primal minimization problem as follows:
\begin{problem}
  \emph{(Primal)}
  Given proper convex functions 
  $
  f:\R^n \to \overline{\R} 
  $,
  $
  g:\R^m \to \overline{\R} 
  $
  and a matrix 
  $
    A\in \R^{m \times n}
  $
  we define the \textbf{primal optimization problem} to be
  \label{cvxa_primal_problem}
  \begin{gather*}
    \underset{x \in \R^n}{\mathrm{minimize}}
    \qquad
    f(x) + g(Ax)
  \end{gather*}
\end{problem}
\begin{remark}
  Problem \autoref{cvxa_primal_problem}
  appears in the unconstrained form. We can impose constraints by controling for the domains of $f$ and $g$.
  To incorporate linear constraints $Ax \le 0$
  or more general constraints $x\in\Omega$, where $\Omega$ is a convex set,
  we can choose
  \begin{gather}
    g(x)=\delta_\Omega(x):=
  \end{gather}
  where $x\notin\Omega$
  leads to 
  $f(x) + g(x)=\infty$
  and the optimization problem (if feasible) will exclude $x$ from the solutions.
\end{remark}
\begin{problem}
  \emph{(Dual)}
  \label{cvxa_dual_problem}
  Consider the same setting as in Problem~\autoref{cvxa_primal_problem}.
  Using the convex conjugates of $f,g$ and the transpose of $A$
  we define the \textbf{dual problem} of Problem~\autoref{cvxa_primal_problem} to be
  \begin{gather*}
    \underset{y^* \in \R^m}{\mathrm{maximize}}
    \qquad
   - f^*(A^\top y^*) - g^*(y^*).
  \end{gather*}
\end{problem}
\begin{proposition}
  Consider the optimization problem~\autoref{cvxa_primal_problem} and its dual~\autoref{cvxa_dual_problem}, where the functions $f$ and $g$ are not assumed to be convex. Define the \textbf{optimal values} of these problems by
  \begin{gather*}
    \widehat{p}:= \inf_{x\in\R^n}f(x)+g(Ax)
    \quad
    \text{and}
    \quad
    \widehat{d}:= \sup_{y\in\R^m} - f^*(A^\top y) - g^*(y).
  \end{gather*}
  Then we have the relationship
  $\widehat{d}\le \widehat{p}$.
\end{proposition}
\begin{proof}
  It holds
  \begin{align*}
    - f^*(A^\top y^*) - g^*(y^*) 
    &=
      -\sup_{x\in \R^n}\inner{A^\top y^*}{x}-f(x)
      -\sup_{y\in \R^m}\inner{-y^*}{y}-g(y)
      \\
    &=
    \inf_{x\in \R^n}f(x)-\inner{y^*}{Ax}
      +\inf_{y\in \R^m}g(y)+\inner{y^*}{y}
      \\
    &\le
      \inf_{x\in \R^n}f(x)-\inner{y^*}{Ax}
      +\inf_{x\in \R^n}g(Ax)+\inner{y^*}{Ax}
      \\
    &\le
      \inf_{x\in \R^n}f(x)-\inner{y^*}{Ax} + g(Ax) + \inner{y^*}{Ax}
      \\
    &=  
      \inf_{x\in \R^n}f(x)+g(Ax)
    =
      \widehat{p}
  \end{align*}
  The first equality is due to the definition of convex conjugates, the second equality due to $\inner{A^\top y}{x}=\inner{y}{Ax}$ and $\inf \left\{ -B \right\}=-\sup \left\{ B \right\}$ for all $B \subseteq \overline{\R}$ and the first inequality due to $\mathrm{Im}(A)\subseteq \R^m$.
  Taking the supremum with respect to all 
  $y^*\in\R^m$
  yields the result.
\end{proof}
\begin{theorem}
  \label{cvxa_fenchel_theorem}
  Let 
  $f, g : \R^n \to \overline{\R}$ 
  be proper convex functions
  and
  $0 \in \text{ri}(\text{dom}(g) - A (\text{dom}(f)) )$
  .
  Then the optimal values of \eqref{cvxa_primal_problem} and \eqref{cvxa_dual_problem} are equal, 
  i.e.
  \begin{gather}
    \inf_{x \in \R^n} 
    \left\{ f(x) + g(Ax) \right\}
    =
    \sup_{y \in \R^n} \left\{   -f^* \left( A^T y \right) - g^*(-y) \right\}
    .
  \end{gather}
\end{theorem}
%%%%%%%%%%%%%%%%%%%%%%%%%%%%%%%%%%%%%%%%%%%%%%%
\begin{lemma}
  \label{syu_1_result}
  Let 
  $f : \R^n \to (-\infty, \infty]$ 
  be convex.
  Then 
  for all $y \in \R^n$ and $C>0$ 
    \begin{gather}
      \label{7060_0}
      \inf_{\norm{\Delta}=C} f(y+\Delta) - f(y) \ge 0 \quad
      \Longrightarrow
      \quad
    \exists y^* \in \R^n
    \colon
    y^* \,\text{is global minimum of $f$ and}\,
      \norm{y^* - y} \le C.
    \end{gather}
\end{lemma}
\begin{proof}
  Since 
  $\mathcal{C}:=\left\{ \norm{\Delta}\le C \right\}$
  is convex
  $f$ has a local minimum in 
  $
    y + \mathcal{C}
    :=
    \left\{ 
      y + \Delta \,
      \mid \,
      \norm{\Delta}\le C
    \right\}
    .
  $
  Suppose towards a contradiction that
  $
    y^* \in 
            y + \mathcal{C}
  $
  is a local minimum, but not a global minimum 
  and
  the left-hand side of 
  \eqref{7060_0} is true.
  Then it holds
  \begin{gather}
    \label{7060_3}
    f(x) < f(y^*)
    \quad
    \text{for some}\ 
    x 
    \in 
    \R^n 
    \setminus 
      y + \mathcal{C}
    .
  \end{gather}
  Furthermore since $y + \mathcal{C}$ is compact and contains $y^*$,
  the line segment $\mathcal{L}[y^*,x]$ contains a point on the boundary of 
  $y + \mathcal{C}$, i.e.
  \begin{gather}
    \label{7060_4}
    \theta x + (1 - \theta) y^* = y + \Delta_x
    \quad
    \text{for some}\ 
    \theta \in (0,1)\ 
    \text{and}\ 
    \Delta_x \ 
    \text{with}\ 
    \norm{\Delta_x}=C
    .
  \end{gather}
    It follows
    \begin{align}
      \label{7060_5}
      \begin{split}
      f(y^*)
      \le
      f(y)
      \le
      f(y + \Delta_x)
      &=
      f(
        \theta x + (1 - \theta) y^*
      )
      \\
      &\le
      \theta f(x)
      + 
      (1 - \theta)
      f(y^*)
      <
      f(y^*)
      ,
      \end{split}
    \end{align}
    which is a contradiction.
    Thus every local minimum of $f$ in $y + \mathcal{C}$ is also a global minimum.
    The first inequality is due to
    $y^*$ being a local minimum of $f$ in
    $
      y + \mathcal{C},
    $
    the second inequality is due to the left-hand side of 
    \eqref{7060_0} being true,
    the equality is due to \eqref{7060_4},
    the third inequality is due to the convexity of $f$
    and the strict inequality is due to \eqref{7060_3}.
    %It follows the right-hand side of \eqref{7060_0}.
\end{proof}


\begin{takeaways}
  Almost there 
  \lipsum[1]
\end{takeaways}

%%%%%%%%%%%%%%%%%
% INTRODUCTION %
%%%%%%%%%%%%%%%%
In the following we do not expect the reader to be familiar with convex analysis. However, some very well known results will be stated without proof. The interested reader can study \cite{Mordukhovich2022} for the bedrock analysis.


We begin by defining convex sets
%

\begin{definition}
  A subset $\Omega\subseteq \R^n$ is called CONVEX if we have $\lambda x+(1-\lambda)y\in \Omega$ for all $x,y\in \Omega$ and $\lambda\in (0,1)$. 
\end{definition}

Clearly, the line segment 
$[a,b]:=\left\{ \lambda a+(1-\lambda)b\,\mid \, \lambda\in [0,1] \right\}$ is contained in $\Omega$ for all $a,b\in \Omega$ if and only if $\Omega$ is a convex set.
%

Next we define convex functions. 
%

The concept of convex functions is closely related to convex sets.
%  
 
The line segment between two points on the graph of a convex function lies on or above and does not intersect the graph.
%

In other words: The area above the graph of a convex function $f$ is a convex set, i.e. the \textit{epigraph}
$\text{epi}(f):=\left\{ (x,\alpha)\in \R^n\times\R\,\mid\, f(x)\le \alpha\right\}$ is a convex set in $\R^{n+1}$.
%

Often an equivalent characterisation of convex functions is more useful.
%

\begin{theorem}
  The convexity of a function $f:\R^n\to \overline{\R}$ on $\R^n$ is equivalent to the following statement:

  For all $x,y\in \R^n$ and $\lambda\in(0,1)$ we have 
    \begin{align}
      f(\lambda x + (1-\lambda)y)\le \lambda f(x)+(1-\lambda)f(y).
    \end{align}
\end{theorem}


\begin{definition}
  \emph{proper convex function}
\end{definition}

\begin{definition}
  \label{ def_convex_conjugate }
  \emph{(Convex Conjugate)}
  Given a function
  $
    f:
    \R^n \to ( - \infty, \infty ]
  $
  ,
  the convex conjugate
  $
    f^*:
    \R^n \to ( - \infty, \infty ]
  $
  of $f$ is defined as
  \begin{gather}
    f^*(x^*)
    :=
    \sup_{ x \in \R^n }
    (x^*)^T x - f(x)
  \end{gather}
\end{definition}

Note that $f$ in Definition~\ref{ def_convex_conjugate }
does not have to be convex. On the other hand, the convex conjugate is always convex:

\begin{proposition}
  Let  
  $
    f:
    \R^n \to ( - \infty, \infty ]
  $
  be a proper function. 
  Then its convex conjugate
  $
    f^*:
    \R^n \to ( - \infty, \infty ]
  $
  is convex.
\end{proposition}

%%%%%%%%%%%%%%%%%%%%%%
% CONJUGATE CALCULUS %
%%%%%%%%%%%%%%%%%%%%%%
\section{Conjugate Calculus}
When studying different primal problems such as \eqref{primal_weighting_binary} we often turn to the dual instead.
Therefore we need some reliable tools.
Beign able to compute specific convex conjugates is one tool required.



%%%%%%%%%%%%%%%%%%%
% FENCHEL DUALITY %
%%%%%%%%%%%%%%%%%%%
\section{Fenchel Duality}


Given 
proper convex functions $f, g : \R^n \to \overline{\R}$ 
and
a matrix $A \in \R^{n \times n}$,
we define 
the primal minimization problem as follows:

\begin{gather}
  \label{cvxa_primal_problem}
  \text{minimize}
  \quad
  f(x) + g(Ax) 
  \quad
  \text{subject to}
  \quad
  x \in \R^n.
\end{gather}

The Fenchel dual problem is then

\begin{gather}
  \label{cvxa_dual_problem}
  \text{maximize}
  \quad
  -f^* \left( A^T y \right) - g^*(-y) 
  \quad
  \text{subject to}
  \quad
  y \in \R^n.
\end{gather}

\begin{theorem}
  \label{cvxa_fenchel_theorem}
  Let 
  $f, g : \R^n \to \overline{\R}$ 
  be proper convex functions
  and
  $0 \in \text{ri}(\text{dom}(g) - A (\text{dom}(f)) )$
  .
  Then the optimal values of \eqref{cvxa_primal_problem} and \eqref{cvxa_dual_problem} are equal, 
  i.e.
  \begin{gather}
    \inf_{x \in \R^n} 
    \left\{ f(x) + g(Ax) \right\}
    =
    \sup_{y \in \R^n} \left\{   -f^* \left( A^T y \right) - g^*(-y) \right\}
    .
  \end{gather}
\end{theorem}

  \section{Duality of Optimal Solutions}
  \subsection*{My Contribution}
I adapt ideas from \cite{Tseng1991} to take also equality constraints. For this, I had to understand the connection to my version of the primal optimization problem. 
I filled in many details that were omitted in the paper:
I derived the Karush-Kuhn-Tucker conditions for the problem
from the general result \cite[Theorem~28.3]{Rockafellar1970}.
I prove in detail, that they hold for the adapted problem.


We consider a general convex optimization problem 
with matrix equality and inequality constraints.
For this problem there exists a related problem,
which we call its dual.
With ideas from \cite{Tseng1991} we establish 
a functional relationship
between the optimal solution of the original problem 
and
optimal solutions of the dual.
The main assumption is that in the original problem we have a strictly convex objective function 
with continuously differentiable 
convex conjugate(cf. Definition~\ref{cv:cc:d:cc}). 
\begin{assumption}
  \label{cv:ts:asu}
  The objective function $f\colon \R^n\to \overline{\R}$ is strictly convex and its
convex conjugate $f^*$ is continuously differentiable.
\end{assumption}
\begin{ftheorem}
  \label{cv:ts:th}
  Consider the optimization problem
\begin{align}
  \label{cv:ts:primal}
  %%%% objective %%%%
    &\underset{w \in \R^n}
    {\mathrm{minimize}}
    &&\qquad\qquad
    f(w)
    &&&
    \\
    %%%% Ax >= b %%%%
    \nonumber
    &\mathrm{subject}\ \mathrm{to} 
    &&\qquad\qquad
    \mathbf{U}w
    \ 
    \ge
    \ 
    d
    \,.
    \\
    \nonumber
    &
    &&\qquad\qquad
    \mathbf{A}w
    \ 
    =
    \ 
    a
    \,,
\end{align}
and its dual problem
  \begin{alignat}{2}
    \label{cv:ts:dual}
  %%%% objective %%%%
    &\underset{
    \lambda_d \in \R^r
,
    \lambda_a \in \R^s
  }
    {\mathrm{maximize}}
    &&\qquad\qquad
    \inner
    {\lambda_d}
    {d}
    \ 
    +
    \ 
    \inner
    {\lambda_a}
    {a}
    \ 
    -
    \ 
    f^*
    \!
    \left( 
      \mathbf{U}^\top \! \lambda_d
      +
      \mathbf{A}^\top \! \lambda_a
    \right)
    \\
    %%%% Ax >= b %%%%
    \nonumber
    &\mathrm{subject}\ \mathrm{to} 
    &&\qquad\qquad
    \lambda_d
    \ 
    \ge
    \ 
    0
    \,.
\end{alignat}
  Let 
$
(\lambda_d^\dagger,\lambda_a^\dagger)
$
be an optimal solution to \eqref{cv:ts:dual}.
If the objective function $f$ of 
\eqref{cv:ts:primal} is strictly convex and its
convex conjugate $f^*$ is continuously differentiable,
then the unique optimal solution to 
\eqref{cv:ts:primal}
is given by
\begin{gather}
  w^\dagger
  =
  \nabla
    f^*
    \!
    \left( 
      \mathbf{U}^\top  \lambda_d^\dagger
      +
      \mathbf{A}^\top  \lambda_a^\dagger
    \right)
    \,.
\end{gather}
\end{ftheorem}

\subsubsection*{Plan of Proof}
We show that 
$w^\dagger$ and 
$
(\lambda_d^\dagger,\lambda_a^\dagger)
$
meet the 
Karush-Kuhn-Tucker conditions for \ref{cv:ts:primal},
that is,
\textbf{complementary slackness}
\begin{gather}
  \label{cv:ts:comps}
\inner
{\lambda_d^\dagger\,}{d-\mathbf{U} w^\dagger}
\ 
=
\ 
0
\,,
\end{gather}
\textbf{primal} and \textbf{dual feasibility}
\begin{align}
  \label{cv:ts:pfea}
    \mathbf{U}w^\dagger
    &
    \ 
    \ge
    \ 
    d
    \,,
    \\
    \nonumber
    \mathbf{A}w^\dagger
    &
    \ 
    =
    \ 
    a
    \,,
  \\
  \label{cv:ts:dfea}
  \lambda_d^\dagger
    &
    \ 
  \ge
  \ 
  0
    \,,
\end{align}
and 
\textbf{stationarity}
\begin{gather}
  \label{cv:ts:st}
  \mathrm{0}_n
  \ 
  \in
  \ 
  [
  \partial
  f(w^\dagger)
  \ 
  +
  \ 
    \partial
    \left( 
      w
      \mapsto
      d
      -
      \mathbf{U}w
    \right)
    (w^\dagger)
    \cdot
    \lambda_d^\dagger
    \ 
    +
    \ 
    \partial
    \left( 
      w
      \mapsto
      a
      -
      \mathbf{A}w
    \right)
    (w^\dagger)
    \cdot
    \lambda_a^\dagger
    \,
  ]
  \,.
\end{gather}
Applying the well know result\cite[Theorem~28.3]{Rockafellar1970}
finishes the proof.
Apart from elementary calculations, our main tools are the 
strict convexity of $f$, the smoothness of $f^*$ and 
\begin{proposition}
  \emph{
\cite[Theorem~23.5(a)-(b)]{Rockafellar1970}.
  }
  \label{cv:ts:prop}
   For any proper convex function $g$ and any vector $w$, 
   it holds $t\in \partial f(w)$ 
   if and only if 
   $
   x
   \mapsto
   \inner
   {x}{t}
   -
   f(x)
   $
   achieves its supremum at $w$.
\end{proposition}

\begin{proof}
  Let 
$
(\lambda_d^\dagger,\lambda_a^\dagger)
$
be an optimal solution to \eqref{cv:ts:dual}. 

\subsubsection*{Complementary Slackness}
  We fix 
  $
  \lambda_a^\dagger
  $
  and
  work with the objective function $G$ of the dual problem, that is,
  \begin{gather*}
    G
(\lambda_d)
\
:
=
\
    \inner
    {\lambda_d}
    {d}
    \ 
    +
    \ 
    \inner
    {\lambda_a^\dagger}
    {a}
    \ 
    -
    \ 
    f^*
    \left( 
      \mathbf{U}^\top \!\lambda_d
      +
      \mathbf{A}^\top \! \lambda_a^\dagger
    \right)
    \,.
  \end{gather*}
  Since $f^*$ is continuously differentiable, so is $G$.
  Thus
  \begin{gather*}
    \nabla
    G
(\lambda_d^\dagger)
\
:
=
\
d
\ 
    -
    \ 
    \mathbf{U}
    \cdot
    \nabla
    f^*
    \!
    \left( 
      \mathbf{U}^\top \lambda_d^\dagger
      +
      \mathbf{A}^\top  \lambda_a^\dagger
    \right)
    \ 
    =
    \ 
d
\ 
    -
    \ 
    \mathbf{U}
    w^\dagger
    \,.
  \end{gather*}
Let
$\lambda_{d,i}^\dagger$ be the $i$-th coordinate of $\lambda_d^\dagger$ 
and
$
\nabla
G_i
(\lambda_d^\dagger)
$
be the $i$-th coordinate of 
$
\nabla
G
(\lambda_d^\dagger)
$.
  To establish \eqref{cv:ts:comps} we will show
  for all coordinates 
\begin{alignat*}{2}
  \text{either}
  &
  &&
  \qquad
  \lambda_{d,i}^\dagger
  = 0
  \quad
  \text{and}
  \quad
  \nabla
  G
  _i(
  \lambda_{d}^\dagger
  ) \le 0
  \\
  \text{or}
  &
  &&
  \qquad
  \lambda_{d,i}^\dagger
  > 0
  \quad
  \text{and}
  \quad
  \nabla
  G
  _i(
  \lambda_{d}^\dagger
  ) = 0
  \,.
\end{alignat*}
It is well know that a concave functions $g$ satisfies
  \begin{gather}
    \label{cv:ts:concD}
    g(x)-g(y)
    \ge
    \nabla
    g(x)^\top
    (x-y)
    \qquad 
    \text{for all}\ 
    x,y\,.
  \end{gather}
  But $G$ is concave 
  by the convexity of $f^*$(cf. Proposition~\ref{cv:cc:pr:ccc}).

First, we show 
\begin{gather}
  \label{cv:ts:comps:d}
\nabla G_i(\lambda_d^\dagger)
\ 
\le
\ 
0
\qquad
\text{for all}\ 
  i\in \left\{ 1,\ldots, s \right\}
  \,.
\end{gather}
Assume towards a contradiction that 
$
\nabla G_i(\lambda_d^\dagger)>0
$
for some 
$
  i\in \left\{ 1,\ldots, s \right\}
$.
By the continuity of $\nabla G$ there exists $\varepsilon>0$ such that 
$
\nabla G_i(
\lambda_d^\dagger
+
e_i\cdot \varepsilon
)
>
0
$.
It follows from \eqref{cv:ts:concD}
\begin{gather*}
  G
  (
\lambda_d^\dagger
+
e_i\cdot \varepsilon
  )
  \ 
  -
  \ 
  G
  (
\lambda_d^\dagger
  )
  \ 
  \ge
  \ 
\nabla G_i(
\lambda_d^\dagger
+
e_i\cdot \varepsilon
)
\cdot
\varepsilon
\ 
>
\ 
0
\,,
\end{gather*}
which contradicts the optimality of 
$
\lambda_d^\dagger
$
for \eqref{cv:ts:dual}.
It follows \eqref{cv:ts:comps:d}.

Next, 
we assume that
$ \lambda_{d,i}^\dagger>0 $ 
and 
$
  \nabla G_i(\lambda_d^\dagger)< 0
$
for some
$
  i\in \left\{ 1,\ldots, s \right\}
$.
Again, by the 
continuity of $\nabla G$ there exists $\varepsilon>0$ such that
$
  \nabla G_i(\lambda_d^\dagger-e_i\cdot \varepsilon)< 0
$
and
$
\varepsilon
-
\lambda_{d,i}^\dagger
<0
$.
Thus
\begin{gather*}
  G
  (
\lambda_d^\dagger
-
e_i\cdot \varepsilon
  )
  -
  G
  (
\lambda_d^\dagger
  )
  \ge
\nabla G_i(
\lambda_d^\dagger
-
e_i\cdot \varepsilon
)
\cdot
\left( 
  -
\varepsilon
\right)
>0
\,,
\end{gather*}
which contradicts the optimality of 
$
\lambda_d^\dagger
$.
It follows \eqref{cv:ts:comps}, that is,
we proved complementary slackness.
\subsubsection*{Primal Feasibility}
Since $f^*$ is continuously differentiable it holds
\begin{gather*}
  \nabla G(\lambda_d^\dagger)
  \ 
  =
  \ 
  d
  \ 
  -
  \ 
  \mathbf{U}
  \cdot
  \nabla
    f^*
    \left( 
      \mathbf{U}^\top  \lambda_d^\dagger
      +
      \mathbf{A}^\top  \lambda_a^\dagger
    \right)
    \ 
  =
  \ 
  d
  -
  \mathbf{U}w^\dagger
  \,.
\end{gather*}
Thus, by \eqref{cv:ts:comps:d},
$w^\dagger$ satisfies the inequality constraints in \eqref{cv:ts:primal}. 
To prove this for the equality constraints,
we view $G$ from a different angel. Let for fixed
$\lambda^\dagger_d$
\begin{gather*}
  G(\lambda_a)
  \ 
  :=
  \ 
  \inner
  {\lambda_a}{a}
  \ 
  -
  \ 
  \left( 
    f^*
    \left( 
      \mathbf{U}^\top  \lambda_d^\dagger
      +
      \mathbf{A}^\top  \lambda_a
    \right)
    -
  \inner
  {\lambda_d^\dagger}{d}
  \right)
  \ 
  =:
  \ 
  \inner
  {\lambda_a}{a}
  \ 
  -
  \ 
  g(\lambda_a)
  \,
  .
\end{gather*}
The function $g$ inherits convexity and differentiability from 
$f^*$.
From the optimality of $\lambda_a^\dagger$ we know that
$G$ takes its maximum there. But then by Proposition~\ref{cv:ts:prop}
and the differentiability of $g$ it holds
\begin{gather}
  a
  \ 
  \in
  \ 
  \partial
  g(\lambda_a^\dagger)
  \ 
  =
  \ 
  \left\{ 
    \mathbf{A}
    \cdot
    \nabla
    f^*
    \left( 
      \mathbf{U}^\top  \lambda_d^\dagger
      +
      \mathbf{A}^\top  \lambda_a^\dagger
    \right)
  \right\}
  \ 
  =
  \ 
  \left\{ 
    \mathbf{A}
    w^\dagger
  \right\}
  \,.
\end{gather}
Thus $a=
    \mathbf{A}
    w^\dagger
$. But then $w^\dagger$ satisfies also the equality constraints.
We proved \eqref{cv:ts:pfea}.
\subsubsection*{Stationarity}
First we show
\begin{gather}
  \label{cv:ts:st:1}
      \mathbf{U}^\top  \lambda_d^\dagger
      \ 
      +
      \ 
      \mathbf{A}^\top  \lambda_a^\dagger
      \ 
      \in
      \ 
\partial f (w^\dagger)
\,.
\end{gather}
By Proposition~\ref{cv:ts:prop}
it suffices to show
that
\begin{gather*}
  w
  \ 
  \mapsto
  \ 
\inner
{w}
{
      \mathbf{U}^\top  \lambda_d^\dagger
      +
      \mathbf{A}^\top  \lambda_a^\dagger
}
  \ 
-
  \ 
f(w)
\end{gather*}
achieves its supremum at
$w^\dagger$.
Since $f$ is strictly convex 
there exists a unique vector $x^\dagger$
where
the above expression achieves its maximum.
Since
$f^*$ is differentiable it holds
\begin{gather*}
  w^\dagger
  \ 
  =
  \ 
  \nabla
    f^*
    \left( 
      \mathbf{U}^\top  \lambda_d^\dagger
      +
      \mathbf{A}^\top  \lambda_a^\dagger
    \right)
  \ 
    =
  \ 
    \nabla
    \left( 
      \lambda
      \mapsto
\inner
{x^\dagger}
{
  \lambda
}
  \ 
-
  \ 
f(x^\dagger)
    \right)
    \left( 
      \mathbf{U}^\top  \lambda_d^\dagger
      +
      \mathbf{A}^\top  \lambda_a^\dagger
    \right)
  \ 
    =
  \ 
x^\dagger
\,.
\end{gather*}
It follows 
\eqref{cv:ts:st:1}.
Next we show
\begin{gather}
  \label{cv:ts:st:2}
-
\mathbf{U}^\top
\in
\ 
    \partial
    \left( 
      w
      \mapsto
      d
      -
      \mathbf{U}w
    \right)
    (w^\dagger)
    \qquad
    \text{and}
\qquad
-
\mathbf{A}^\top
\in
\ 
    \partial
    \left( 
      w
      \mapsto
      d
      -
      \mathbf{A}w
    \right)
    (w^\dagger)
    \,.
\end{gather}
To this end, note that
\begin{gather*}
  \inner
  {
  -
\mathbf{U}^\top
\!
 e_i
}
  {w-w^\dagger}
  \ 
  =
  \ 
  \left( 
    d-
\mathbf{U}
w
  \right)_i
  \ 
  -
  \ 
  (
    d-
\mathbf{U} w^\dagger
  )
  _i
  \qquad
  \text{for all}\ 
  i\in\left\{ 1,\ldots,r \right\}
  \,.
\end{gather*}
Thus
$
-
\mathbf{U}^\top
\in
    \partial
    \left( 
      w
      \mapsto
      d
      -
      \mathbf{U}w
    \right)
    (w^\dagger)
$.
In the same way it follows
$
-
\mathbf{A}^\top
\in
    \partial
    \left( 
      w
      \mapsto
      d
      -
      \mathbf{A}w
    \right)
    (w^\dagger)
$.
From \eqref{cv:ts:st:1} and \eqref{cv:ts:st:2} we conclude
\begin{align*}
  \mathrm{0}_n
  &
      \ 
  =
      \ 
      \left( 
      \mathbf{U}^\top  \lambda_d^\dagger
      \ 
      +
      \ 
      \mathbf{A}^\top  \lambda_a^\dagger
      \right)
      -
      \mathbf{U}^\top  \lambda_d^\dagger
      \ 
      -
      \ 
      \mathbf{A}^\top  \lambda_a^\dagger
      \\
      &
      \ 
  \in
      \ 
  [
  \partial
  f(w^\dagger)
  \ 
  +
  \ 
    \partial
    \left( 
      w
      \mapsto
      d
      -
      \mathbf{U}w
    \right)
    (w^\dagger)
    \cdot
    \lambda_d^\dagger
  \ 
    +
  \ 
    \partial
    \left( 
      w
      \mapsto
      a
      -
      \mathbf{A}w
    \right)
    (w^\dagger)
    \cdot
    \lambda_a^\dagger
    \, 
  ]
  \,.
\end{align*}
We have proved \eqref{cv:ts:st}, that is, stationarity.
\subsubsection*{Dual Feasibility and Conclusion}
Dual feasibility \eqref{cv:ts:dfea} follows immediately from the optimality of $\lambda_d^\dagger$ for \eqref{cv:ts:dual}.
Thus, $(\lambda_d^\dagger,\lambda_a^\dagger)$ and $w^\dagger$ satisfy the Karush-Kuhn-Tucker conditions for \eqref{cv:ts:primal}.
Applying 
\cite[Theorem~28.3]{Rockafellar1970} finishes the proof.
\end{proof}
\begin{takeaways}
  For strictly convexity objective functions with continuously
  differentiable convex conjugate we get a functional relationship
  of primal and dual solutions via the Karush-Kuhn-Tucker conditions.
\end{takeaways}



%\chapter{Random Matrix Inequalities}
%  In our application we want to bound moments of vector-valued random variables.
%  For this we choose the theory of random matrix inequalities
%  which lately received a lot of attention.
%  In particular an approach via the method of exchangable pairs \cite{Mackey2014}
%  has been fruitful in simplifying the proofs of long standing results such as the matrix Khintchin inequality.
%  The paper offers a comprehensive introduction to this method.
%
%  We will cite the matrix Khintchin inequality and 
%  inequalities for moments of matrices that follow from it\cite{Chen2012}. 
%  As a novelty, we will apply intrinsic dimension results and Hermition Dilitation from \cite{Tropp2015}  
%  to matrix moments inequalities. Even though it is straightforward, to the best of our knowledge the calculations have not been carried out in any publication so far.
%  \section{A Matrix Analysis Primer}
%    The \textbf{trace} of a square matrix, denoted by $\mathrm{tr},$
  is the sum of its diagonal entries, i.e. 
  $
    \mathrm{tr}(\mathbf{B})
    =
    \sum_{j=1}^{d}b_{jj}
    \quad 
    \text{for}\ 
    \mathbf{B} \in \mathbb{M}_d.
  $
  The trace is unitarily invariant, i.e.
  $
    \mathrm{tr}(\mathbf{B})
    =
    \mathrm{tr}(\mathbf{Q}\mathbf{B}\mathbf{Q}^*)
    \quad 
    \text{for all}
    \ 
    \mathbf{B}\in \mathbb{M}_d
    \ 
    \text{for all unitary}\ 
    \mathbf{Q} \in \mathbb{M}_d.
  $
  In particular, the existence of an eigenvalue value decomposition shows 
  that the trace of a Hermitian matrix equals the sum of its  eigenvalues.
  Let
  $
  f: I\to \R
  $
  where 
  $I\subseteq\R$ 
  is an interval.
  Consider a matrix 
  $\mathbf{A}\in \mathbb{H}_d$
  whose eigenvalues are contained in $I.$
  We define the matrix 
  $
    f(\mathbf{A})\in \mathbb{H}_d
  $
  using an eigenvalue decomposition of $\mathbf{A}:$
  \begin{gather}
    f(\mathbf{A})
    =
    \mathbf{Q}
    \begin{bmatrix}
      f(\lambda_1) &&\\
                   &\ddots&\\
                   && f(\lambda_d)
    \end{bmatrix}
    \mathbf{Q}^*
    \qquad
    \text{where}
    \qquad
    \mathbf{A}
    =
    \mathbf{Q}
    \begin{bmatrix}
      \lambda_1 &&\\
                   &\ddots&\\
                   && \lambda_d
    \end{bmatrix}
    \mathbf{Q}^*
    =
    \sum_{i=1}^{d} 
    \lambda_i
    \mathbf{Q}_{\bullet i}
    \mathbf{Q}_{\bullet i}^*
    .
  \end{gather}
  The definition of $f(\mathbf{A})$ does not depend on which 
  eigenvalue decomposition we choose.
  Any matrix function that arises in this fashion is called a \textbf{standard matrix function}.

  For each $p\ge 1$
  the \textbf{Schatten $p$-norm} is defined as 
  $
    \norm{\mathbf{B}}_p
    :=
    (
    \mathrm{tr}(\left| \mathbf{B} \right|^p)
    )
    ^{1/p}
    \ 
    \text{for}
    \ 
    \mathbf{B} \in \mathbb{M}_d.
  $
  In this setting,
  $
  \left| \mathbf{B} \right|
  :=
  (
    \mathbf{B}^*
    \mathbf{B}
    )^{1/2}
    .
  $
  The \textbf{spectral norm}
  of an Hermitian matrix $\mathbf{A}$
  is defined by the relation
  $
    \norm{\mathbf{A}}
    :=
    \lambda_{\mathrm{max}}(\mathbf{A})
    \lor
    (
    -\lambda_{\mathrm{min}}(\mathbf{B})
    )
    .
  $
  For a general matrix $\mathbf{B},$
  the spectral norm is defined to be the largest singular value:
  $
    \norm{\mathbf{B}}
    :=
    \sigma_1(\mathbf{B})
    .
  $
  The Schatten $p$-norm dominates the spectral norm for all $p\ge 1.$



\begin{proposition}
  Let
  $
  f,g: I\to \R
  $
  be real-valued functions on an interval $I\subseteq\R,$ 
  and let
  $\mathbf{A}\in \mathbb{H}_d$
  be a Hermitian matrix
  whose eigenvalues are contained in $I.$

  \begin{enumerate}[label={(\roman*)}]
    \item
      If $\lambda$ is an eigenvalue of of $\mathbf{A},$
      then $f(\lambda)$ is an eigenvalue of $f(\mathbf{A}).$
    \item
      $
        f(a)
        \le
        g(a)
        \quad
        \text{for all}\ 
        a\in I
        \quad
        \text{implies}
        \quad
        f(\mathbf{A})
        \preccurlyeq
        g(\mathbf{A})
        .
      $
  \end{enumerate}
\end{proposition}


\begin{takeaways}
  This Primer is not a prim number.
  \lipsum[5]
\end{takeaways}

%  %\section{The Method of Exchangeable Pairs}
%  %\subsection{Matrix Stein pairs}
We first define an exchangable pair.
\begin{definition}
  Let 
  $Z$ and $Z^{'}$
  random variables taking values
  in a Polish space $\mathcal{Z}.$
  We say that
  $
  (
    Z
    ,
    Z^{'}
  )
  $
  is an \textbf{exchangable pair}
  if it has the same distribution as 
  $
  (
  Z^{'}
    ,
    Z
  )
  .
  $
  In particular, 
  $Z$ and $Z^{'}$
  must share the same distribution.
\end{definition}

The following approach originates in the work of Charles Stein \cite{Stein1972} 
on normal approximation for a sum of dependent random variable.
We will explain how some central ideas of this theory extends to matrices.

We can obtain a lot of information about the fluctuation of a random matrix
$\mathbf{X}$
if we can construct a good exchangable pair 
$
(
  \mathbf{X}
  ,
  \mathbf{X}^{'}
)
.
$
With this motivation in mind, let us introduce a special class of exchangable pairs.
\begin{definition}
  Let
  $
  (
    Z
    ,
    Z^{'}
  )
  $
  be an exchangable pair of random variables taking values
  in a Polish space $\mathcal{Z},$
  and let 
  $
    \mathbf{\Psi}
    : 
    \mathcal{Z}
    \to 
  \mathbb{H}_d
  $
  be a measurable function.
  Define the random Hermitian matrices
  \begin{gather}
    \mathbf{X}
    :=
    \mathbf{\Psi}
    (Z)
    \quad
    \text{and}
    \quad
    \mathbf{X}^{'}
    :=
    \mathbf{\Psi}
    (Z^{'})
    .
  \end{gather}
  We say that 
  $
  (
    \mathbf{X}
    ,
    \mathbf{X}^{'}
  )
  $
  is a \textbf{matrix Stein pair}
  if there is a constant 
  $\alpha\in (0,1]$
  for which
  \begin{gather}
    \E[
    \mathbf{X}
    -
    \mathbf{X}^{'}
    \vert
    Z
    ]
    =
    \alpha 
    \mathbf{X}
    \qquad
    \text{almost surely.}
  \end{gather}
  The constant 
  $\alpha$
  is called the \textbf{scale factor} of the pair.
  We always assume 
  $
    \E
    \left[ 
      \norm{\mathbf{X}}^2
    \right]
    <
    \infty
    .
  $
\end{definition}

A matrix Stein pair 
$
(
\mathbf{X}
,
\mathbf{X}^{'}
)
$
has several useful propreties. First,
$
(
\mathbf{X}
,
\mathbf{X}^{'}
)
$
always forms an exchangable pair. Second, it must be the case that
$\E[\mathbf{X}]=\mathbf{0}.$
Indeed,
\begin{gather*}
  \E[\mathbf{X}]
  =
  \frac{1}{\alpha}
  \E[
  \E
  [
    \mathbf{X}
    -
    \mathbf{X}^{'}
    \vert
    Z
  ]
  ]
  =
  \frac{1}{\alpha}
  \E[
    \mathbf{X}
    -
    \mathbf{X}^{'}
    \vert
  ]
  =
  \mathbf{0}
  .
\end{gather*}
\subsection{The method of exchangable pairs}
A well-chosen matrix Stein pair 
$
(
\mathbf{X}
,
\mathbf{X}^{'}
)
$
provides a surprisingly powerful tool for studying 
the random matrix $\mathbf{X}.$
The technique depends on a fundamental technical lemma.
\begin{lemma}
  Suppose that
  $
  (
    \mathbf{X}
    ,
    \mathbf{X}^{'}
  )
  $
  is a matrix Stein pair with scale factor $\alpha.$
  Let 
  $
    \mathbf{F}
    :
    \mathbb{H}_d
    \to
    \mathbb{H}_d
  $
  be a measurable function that satisfies the regularity condition
  $
  \E
  \left[
  \norm{
    (
    \mathbf{X}
    -
    \mathbf{X}^{'}
  )
  \mathbf{F}(\mathbf{X})
  }
  \right]
  <
  \infty
  .
  $
  Then
\begin{gather}
  \E
  [
    \mathbf{X}
    \cdot
    \mathbf{F}
    (\mathbf{X})
  ]
  =
  \frac{1}{2\alpha}
  \E
  [
    (
    \mathbf{X}
    -
    \mathbf{X}^{'}
    )
    (
    \mathbf{F}
    (
    \mathbf{X}
    )
    -
    \mathbf{F}
    (
    \mathbf{X}^{'}
    )
  )
  ]
  .
\end{gather}
\end{lemma}
In short, the randomness in the Stein pair furnishes an alternative expression for the expected product of $\mathbf{X}$
and a function $\mathbf{F}.$
It allows us to estimate the expectation using the smoothness properties of the function $\mathbf{F}$ and the discrepancy between $\mathbf{X}$
and $\mathbf{X}^{'}.$
\begin{proof}
  \emph{\cite[Lemma~2.4]{Mackey2014}}
  Suppose that
  $
  (
    \mathbf{X}
    ,
    \mathbf{X}^{'}
  )
  $
  constructed from an auxiliary exchangable pair 
  $
  (
  Z,
  Z^{'}
  )
  .
  $
  The defining property implies
 \begin{gather}
   \alpha \cdot 
   \E[
   \mathbf{X}
   \cdot
   \mathbf{F}
   (\mathbf{X})
   ]
   =
   \E[
    \E[
    \mathbf{X}
    -
    \mathbf{X}^{'}
    \vert
    Z
    ]
    \cdot
   \mathbf{F}
   (\mathbf{X})
   ]
   =
    \E[
    (
    \mathbf{X}
    -
    \mathbf{X}^{'}
    )
   \mathbf{F}
   (\mathbf{X})
   ]
 \end{gather} 
\end{proof}
\subsection{The conditional variance}
To each matrix Stein pair 
  $
  (
    \mathbf{X}
    ,
    \mathbf{X}^{'}
  ),
  $
  we may associate a random matrix called the conditional variance of $\mathbf{X}.$
  The purpose of this section is to argue that the spectral norm of $\mathbf{X}$
  is unlikely to be large, when the conditional variance is small.
\begin{definition}
  Suppose that
  $
  (
    \mathbf{X}
    ,
    \mathbf{X}^{'}
  ),
  $
  is a matrix Stein pair, constructed from an auxiliary exchangeable pair
  $
  (
  Z
    ,
  Z^{'}
  ).
  $
  The \textbf{conditional variance}
  is the random matrix
  \begin{gather}
    \mathbf{\Delta_X}
    :=
    \mathbf{\Delta_X}
    (Z)
    :=
    \frac{1}{2\alpha}
    \E
    [
    (
    \mathbf{X}
    -
    \mathbf{X}^{'}
    )^2
    \vert
    Z
    ]
    ,
  \end{gather}
  where $\alpha$ is the scale factor of the pair. We may take any version of the conditional expectation in this definition.
\end{definition}

The conditional variance
$
    \mathbf{\Delta_X}
$
can be regarded as a stochastic estimate for the variance 
of the random matrix $\mathbf{X}.$
To see this, assume
the second moment of $\mathbf{X}$ exists. Then it follows from Lemma 
with $\mathbf{F}(\mathbf{X})=\mathbf{X}$
\begin{gather}
  \E
  [
    \mathbf{\Delta_X}
  ]
  =
  \E
  [
  \mathbf{X}^2
  ]
  .
\end{gather}
To verify the regularity condition, note that
\begin{gather}
  \E
  [
  \lVert
    (
    \mathbf{X}
    -
    \mathbf{X}^{'}
    )
    \mathbf{X}
    \rVert
  ]
  \le
  \E[
  \norm{
    \mathbf{X}
  }
  ^2
  ]
  +
  \E
  [
  \norm{
    \mathbf{X}
  }
  \cdot
  \lVert
    \mathbf{X}^{'}
    \rVert
  ]
  \le
  2
  \E
  [
  \norm{
    \mathbf{X}
  }
  ^2
  ]
  <
  \infty
  .
\end{gather}
\begin{example}
  \emph{\cite[Example~2.4]{Mackey2014}}
\end{example}
nrt\\
is not necessary, as, for example, the normal distribution does not fulfill it. For

% \section{Matrix Khintchin Inequality and Applications}
% In this section we state the matrix Khintchin inequality and matrix moment inequalities as an applications.
% We provide the proof of auxiliary theorems which are cited without proof in \cite{Mackey2014}. They are needed to prove the matrix Khintchin inequality. 
%  \begin{theorem}
  \emph{(Matrix BDG inequality)}
  Let
  $
    p = 1
    \ 
    \text{or}\ 
    p \ge 3/2
    .
  $
  Suppose that 
  $
  (
    \mathbf{X}
   , 
   \mathbf{X}^{'}
  )
  $
  is a matrix Stein pair where
  $
   \E
   [ 
    \norm{\mathbf{X}}
    _{2p}^{2p}
   ]
   <
   \intfy
   .
  $
\end{theorem}




\begin{ftheorem}
  \emph{\cite[Corollary~7.3]{Mackey2014}}
  Suppose that
  $
    p = 1
    \ 
    \text{or}\ 
    p \ge 3/2
    .
  $
  Consider a finite sequence
  $
    (\mathbf{Y}_k)_{k\ge 1}
  $
  of independent, random, Hermitian matrices 
  and a deterministic sequence
  $
    (\mathbf{A}_k)_{k\ge 1}
  $
  for which
  \begin{gather}
    \E[\mathbf{Y}_k]
    =
    0
    \quad 
    \text{and}
    \quad
    \mathbf{Y}_k^2
    \preccurlyeq
    \mathbf{A}_k^2
    \qquad
    \text{almost surely for all}\ 
    k \ge 1.
  \end{gather}
  Then
  \begin{gather}
      \E
      \left[
        \norm{
          \sum_{k\ge 1}
            \mathbf{Y}_k
        }
        _{2p}
        ^{2p}
      \right]
      ^{1/(2p)}
      \le
      \sqrt{
        p - \frac{1}{2}
      }
      \,
      \norm{
        \left( 
          \sum_{k\ge 1}
          (
            \mathbf{A}_k^2
            + 
            \E[
              \mathbf{Y}_k^2
            ]
          )
        \right)
        ^{1/2}
        }
      _{2p}
      .
  \end{gather}
  In particular, when 
  $
    (\xi_k)_{k\ge 1}
  $
  is an independent sequence of Rademacher random variables,
  \begin{gather}
      \E
      \left[
        \norm{
          \sum_{k\ge 1}
            \xi_k
            \mathbf{A}_k
        }
        _{2p}
        ^{2p}
      \right]
      ^{1/(2p)}
      \le
      \sqrt{
        2p - 1
        }
      \,
      \norm{
        \left( 
          \sum_{k\ge 1}
            \mathbf{A}_k^2
        \right)
        ^{1/2}
        }
      _{2p}
      .
  \end{gather}
\end{ftheorem}

%  \section{Generalzed Inequalities by Hermitian Dilation}
%  For an introduction to Hermitian Dilation see \cite[§2.1.16]{Tropp2015}
\begin{definition}
  \emph{(Hermitian Dilation)}
  The Hermitian dilation
  \begin{gather*}
    \mathfrak{H} : \C^{d_1 \times d_2} \to \mathbb{H}_{d_1 \times d_2}
  \end{gather*}
  is a map from a general matrix to an Hermitian matrix defined by
  \begin{gather}
    \label{ rmineq_hermitian_dilation } 
    \mathfrak{H}(B)
    :=
    \begin{bmatrix}
      0   & B \\
      B^* & 0 \\
    \end{bmatrix}
  \end{gather}
\end{definition}

Properties:

$
\norm{\mathbf{B}}
=
\norm{\mathfrak{H}\mathbf{B}}
$
Linearity
confer \cite[§7.7.3]{Tropp2015}
for combination with intrinsic dimension argument.

%  \begin{ftheorem}
  %\label{rmineq_bernstein}
  Let $\mathbf{A}_1, \ldots, \mathbf{A}_n$ be independent, random matrices with dimension 
  $d_1 \times d_2$.
    Introduce the random matrix
      \begin{gather*}
        \mathbf{S}:=\sum_{k=1}^n \mathbf{A}_k.
      \end{gather*}
    Let $v(\mathbf{S})$ be the matrix variance statistic of the sum:
      \begin{align}
        \label{rmineq_bernstein_cond_2}
        v(\mathbf{S}):= \norm{\E[\mathbf{S}\mathbf{S}^\top ]} \lor \norm{\E[\mathbf{S}^\top \mathbf{S}]} 
             = \norm{\sum_{k=1}^n\E[\mathbf{A}_k\mathbf{A}_k^\top]} \lor \norm{\sum_{k=1}^n\E[\mathbf{A}_k^T \mathbf{A}_k]} .
      \end{align}
    Then
      \begin{align}
        \label{rmineq_bernstein_expectation_bound}
        \left(
          \E \left[ \norm{\mathbf{S}}^2 \right]
        \right)^{\frac{1}{2}}
        \le
        \sqrt{
          2ev(\mathbf{S})\log(d_1+d_2)
        } 
        + 
        4e \left( 
          \E[\max_{k \le n}\norm{\mathbf{A}_k}^2]
        \right)^\frac{1}{2}
        \log(d_1+d_2).
      \end{align}
\end{ftheorem}
  

%  
%  \newpage
%  \section{Intrinsic Dimension}
%  \begin{definition*}
  \label{rmineq_intrinsic_bernstein}
  For a positive-semidefinite matrix $\,\mathbf{S}$,
  the \textbf{intrinic dimension} is the quantity
  \begin{gather*}
    \mathrm{intdim}
    \,
    \mathbf{A}
    \ 
    :=
    \ 
\mathrm{tr}\,\mathbf{A}
\,
/
\,
\norm{\mathbf{A}}
    \,
    .
  \end{gather*}
\end{definition*}
\begin{lemma*}
  \emph{(Intrinsic dimenision)}
  Let 
  $
  \,
    \varphi\,:\, [\,0,\infty) \,\to\, \R
    \,
  $
  be a convex function with
  $
    \varphi(0)\,=\,0
  $.
  For any positive-semidefinite matrix $\,\mathbf{S}$ it holds 
  \begin{gather*}
    \mathrm{tr}\,\varphi(\mathbf{S})
    \ 
    \le
    \ 
    \mathrm{intdim}\,\mathbf{S}
    \ 
    \cdot
    \ 
    \varphi\!\norm{\mathbf{S}}
    \,
    .
  \end{gather*}
\end{lemma*}
\begin{proof}
  \emph{\cite[Lemma~7.5.1]{Tropp2015}}
  Since $\varphi$ is convex on any interval $[\,0,L\,]$ with $L\,>\,0$, and $\varphi(0)\,=\,0$, it holds
  \begin{gather*}
    \varphi(a)
    \ 
    \le
    \ 
    \left( 
      1 - a/L
    \right)
    \cdot
    \varphi(0)
    \ 
    +
    \ 
    a/L
    \cdot
    \varphi(L)
    \ 
    =
    \ 
    a/L
    \cdot
    \varphi(L)
    \qquad
    \text{for all}\ 
    a \,\in\, [\,0,L\,]
    \,
    .
  \end{gather*}
  Since $\,\mathbf{S}\,$ is positive-semidefinite, the eigenvalues of $\,\mathbf{S}\,$ 
  fall in the interval $\,[\,0,L\,]\,$, where $\,L\,=\,\norm{\mathbf{S}}\,$.
  It follows 
  \begin{align*}
    \mathrm{tr}\,\varphi(\mathbf{S})
    &
    \ 
    =
    \ 
    \sum_{i=1}^{d}
    \varphi(\lambda_i)
    \ 
    \le
    \ 
    \sum_{i=1}^{d}
    \lambda_i
    /
    \norm{\mathbf{S}}
    \ 
    \cdot
    \ 
    \varphi(\norm{\mathbf{S}})
    \\
    &
    \ 
    =
    \ 
    \mathrm{tr}(\mathbf{S})
    /
    \norm{\mathbf{S}}
    \ 
    \cdot
    \ 
    \varphi(\norm{\mathbf{S}})
    \ 
    =
    \ 
    \mathrm{intdim}\,\mathbf{S}
    \ 
    \cdot
    \ 
    \varphi\!\norm{\mathbf{S}}
    \,
    .
  \end{align*}
\end{proof}
The next example applies the preceding lemma to bound 
the $p$-Schatten-norm, when $\,p\,\ge\, 2\,$, by the spectral norm and the intrinsic dimension.
  \begin{example*}
    Let 
    $
    \,
      \mathbf{B} \in \mathbb{C}^{m\times n}
      \,
    $
    be any rectangular matrix and let $\,p\ge 2\,$.
    Then 
    $
    \,
      \varphi(x)
      \,
      :=
      \,
      \left| x \right|
      \,
    $
  defines a convex function with $\,\varphi(0)\,=\,0\,$.
  The intrinsic dimension lemma yields
  \begin{gather*}
    \norm{\mathbf{B}}^{\,p}_p
    \ 
    =
    \ 
    \mathrm{tr}
      \left|
      \mathbf{B}^* \mathbf{B}
      \right|
      ^{\,p/2}
      \ 
    \le
    \ 
    \mathrm{intdim}
    \,
      \mathbf{B}^* \mathbf{B}
      \ 
      \cdot
      \ 
      \norm{
      \mathbf{B}^* \mathbf{B}
    }^{\,p/2}
    \ 
    =
    \ 
    \mathrm{intdim}
    \,
      \mathbf{B}^* \mathbf{B}
      \ 
      \cdot
      \ 
      \norm{
        \mathbf{B}
    }^{\,p}
    \,.
  \end{gather*}
  If, additionally, $\mathbf{B}$ is self-adjoint and positive-semidefinite
  then it holds 
  \begin{gather*}
    \mathrm{tr}
    \,
      \mathbf{B}^* \mathbf{B}
      \ 
    =
    \ 
    \mathrm{tr}
    \,
    \mathbf{B}^{\,2}
    \ 
    =
    \ 
    \sum_{i=1}^{n} 
    \,
    \lambda_i^{\,2}
    \ 
    \le 
    \ 
    \left( 
      \,
    \sum_{i=1}^{n} 
    \,
    \lambda_i
    \,
    \right)
    ^{\!2}
    \ 
    =
    \ 
    \left( 
    \mathrm{tr}
    \,
    \mathbf{B}
    \right)
    ^{\,2}
    \,,
  \end{gather*}
  and consequently
  \begin{gather*}
    \norm{\mathbf{B}}^{\,p}_{p}
    \ 
    \le
    \ 
    \left( 
    \mathrm{intdim}
    \,
    \mathbf{B}
    \right)
    ^{\,2}
    \ 
      \cdot
      \ 
      \norm{
        \mathbf{B}
    }^{\,p}
    \,.
  \end{gather*}
  \end{example*}


  \todo[color=red!40, inline]{Apply to estimate \cite[above (A.4)]{Chen2012} to get intrinsic dimension version of Rosenthal-Pinelis inequality.}
\begin{takeaways}
  The notion of intrinsic dimension is useful when bounding convex
  functions of a positive-semidefinite matrix by its spectral norm.
  We saw how to derive bounds on the $p\,$-Schatten-norm when $\,p\,\ge\, 2$.
\end{takeaways}

%
%
%\chapter{Empirical Processes}
%Classical references are \cite{Vaart2000} and \cite{vaart2013}.
%For maximal inequalities see \cite[§19]{Vaart2000}
%For Functional Delta-Method see \cite[§20]{Vaart2000}
%For an introduction to empirical processes and outer expectation 
%see the beginning of \cite{vaart2013}.
%\section{A Primer on Empirical Processes}
%\todo[color=red!40,inline]{Add outer probability calculus. \cite{vaart2013} p.6}

Let 
$
(
\mathbb{D}
,
d
)
$
be a metric space, and let 
$
(\P_n)_{n\in \mathbb{N}}
  \P
$
be (Borel) probability measures
on
$
(
\mathbb{D}
,
\mathcal{D}
)
,
$
where 
$\mathcal{D}$
is the Borel $\sigma$-algebra on $\mathbb{D},$
the smallest $\sigma$-algebra
containing all open sets.
Then the sequence 
$\P_n$
\textbf{converges weakly}
to 
$\P,$
which we denote as $\P_n \rightsquigarrow \P,$
if and only if 
\begin{gather}
  \int_\mathbb{D}
  f
  \text{d}
  \P_n
  \to
  \int_\mathbb{D}
  f
  \text{d}
  \P
  \qquad
  \text{for all}
  \ 
  f
  \in
  C_b(\mathbb{D})
  .
\end{gather}
Here 
$
  C_b(\mathbb{D})
$
denotes the set of all bounded, continuous, real functions on $\mathbb{D}.$
Equivalently, if 
$X_n$ and $X$
are 
$\mathbb{D}$-valued
random variables with distribution 
$\P_n$ and $\P$
respectively, then 
$X_n \to X$
if and only if 
\begin{gather}
  \E[f(X_n)]
  \to
  \E[f(X)]
  \qquad
  \text{for all}
  \ 
  f
  \in
  C_b(\mathbb{D})
  .
\end{gather}

This definitions yield the classical theory of weak convergence.
For a modern treatment see \cite{Klenke2020}.

The classical theory requires that 
$\P_n$
is defined, for each $n\in \mathbb{N},$
on the Borel $\sigma$-algebra $\mathcal{D},$
or, equivalently, that $X_n$ is a Borel measurable map for each $n\in \mathbb{N}.$
If 
$
(
\Omega_n,
\mathcal{A}_n,
\P_n
)
$
are the underlying probability spaces on which the maps 
$X_n$
are defined, this means that
$X_n^{-1}(D)\in \mathcal{A}_n$
for every Borel set $D \in \mathcal{D}.$
This required measurability usually holds when $\mathbb{D}$
is a separable metric space such as $\R^k$
or $C([0,1])$ with the supremum metric.

However, this apparently modest requirement can and does easily fail when the metric space $\mathbb{D}$ is not separable.

\begin{example}
  \emph{\cite[Problem~1.7.3]{vaart2013}}
  Let $\mathbb{D}=D([0,1])$ be the \textbf{Skorohod space} of all right-continuous functions on $[0,1]$
  with left limits endowed with the metric induced by the supremum norm.
  Define 
  $
    X:
    [0,1]
    \to
    \mathbb{D}
    ,\ 
    \omega
    \mapsto
    \mathbf{1}_{[\omega,1]}
    .
  $
  If we equip $[0,1]$ with the Borel $\sigma$-algebra 
  $\mathcal{B}([0,1])$, then 
  $X$ is not measurable. To see this, let $B_s$ be the open ball of radius $1/2$ in $\mathbb{D}$ around the function $\mathbf{1}_{[s,1]}.$
  Now $X(\omega)\in B_s$
  if and only if $\omega=s.$ Indeed, if $\omega\neq s$ there exists an $x$ between $\omega$ and $s$ such that the difference of the indicator functions is 1 at $x$. Conversely, if the distance is greater than
  $1/2$ at a point $x\in [0,1]$, it is because $x$ lies between $\omega$ and $s$ and the indicator functions have difference 1.
  Since arbitrary (even uncountable) unions of open sets are open,
  we get for every $S\subseteq [0,1]$ the open set 
  $
  G
  :=
  \bigcup_{s\in S}
    B_s
    \in \mathcal{D}
    .
  $
  It follows
  $
  X^{-1}(G)=S 
  \ 
  \text{for all}
  \ 
  S \subseteq [0,1]
  .
  $
  Since not all subsets of $[0,1]$ are measurable, we have
  $
  X^{-1}(\mathcal{D})\nsubseteq \mathcal{B}([0,1])
.
$
But then $X$ is not measurable. The $\sigma$-algebra $\mathcal{D}$ is to large.
\end{example}



Let 
$
  \left( 
    \Omega,
    \mathcal{A},
    \P
  \right)
$
be a probability space
and
$
  \left( 
    \mathcal{X},
    \Sigma
  \right)
$
a measurable space.
Let 
$
  X_j
  :
  \left( 
    \Omega,
    \mathcal{A},
    \P
  \right)
  \to
  \left( 
    \mathcal{X},
    \Sigma
  \right)
  ,
  j=1,\ldots, n
$ be independent and identically-distributed (i.i.d.)
random variables
with probability distribution $\P_X$ 
and
$\mathcal{F}$ a family of measurable functions
$
  f:
  \left( 
    \mathcal{X},
    \Sigma
  \right)
    \to
  \left( 
    \R,
    \mathcal{B}(\R)
  \right)
$.
Consider the map
\begin{gather}
  f
  \mapsto
  G_n f
  :=
  \sqrt{n}
  \left( 
    \frac{1}{n}
    \sum_{i = 1}^{n}
      f(X_i)
    -
    \P_X f
  \right),
\end{gather}
where
$
  \P_X f 
  :=
  \int_\mathcal{X} f \text{d}\P_X.
$
We call 
$
  \left( 
    G_n f
  \right)_{f \in \mathcal{F}}
$
the empirical process indexed by $\mathcal{F}$.
Furthermore
\begin{gather}
  \norm{G_n f}_\mathcal{F}
  :=
  \sup_
        { f \in \mathcal{F}}
        \left|
          G_n f
        \right|
        .
\end{gather}



%
%\section{Maximal Inequalities}
%%%%%%%%%%%%%%%%%
% INTRODUCTION %
%%%%%%%%%%%%%%%%

Let 
$
  \left( 
    \Omega,
    \mathcal{A},
    \P
  \right)
$
be a probability space
and
$
  \left( 
    \mathcal{X},
    \Sigma
  \right)
$
a measurable space.
Let 
$
  X_j
  :
  \left( 
    \Omega,
    \mathcal{A},
    \P
  \right)
  \to
  \left( 
    \mathcal{X},
    \Sigma
  \right)
  ,
  j=1,\ldots, n
$ be independent and identically-distributed (i.i.d.)
random variables
with probability distribution $\P_X$ 
and
$\mathcal{F}$ a family of measurable functions
$
  f:
  \left( 
    \mathcal{X},
    \Sigma
  \right)
    \to
  \left( 
    \R,
    \mathcal{B}(\R)
  \right)
$.
Consider the map
\begin{gather}
  f
  \mapsto
  G_n f
  :=
  \sqrt{n}
  \left( 
    \frac{1}{n}
    \sum_{i = 1}^{n}
      f(X_i)
    -
    \P_X f
  \right),
\end{gather}
where
$
  \P_X f 
  :=
  \int_\mathcal{X} f \text{d}\P_X.
$
We call 
$
  \left( 
    G_n f
  \right)_{f \in \mathcal{F}}
$
the empirical process indexed by $\mathcal{F}$.
Furthermore
\begin{gather}
  \norm{G_n f}_\mathcal{F}
  :=
  \sup_
        { f \in \mathcal{F}}
        \left|
          G_n f
        \right|
        .
\end{gather}

\begin{lemma}
  \emph{(Bernstein Inequality for Empirical Processes)}
  For any bounded, measurable function $f$
  it holds for all $t > 0$
  \begin{gather}
    \P 
    \left(
      \left| 
        G_n f
      \right|
      >
      t
    \right)
    \le
    2
    \exp
    \left( 
      - \frac{1}{4}
      \frac{t^2}
      {
        \P_X(f^2)
        +
        t
        \norm{f}_\infty
        /
        \sqrt{n}
      }
    \right)
  \end{gather}
\end{lemma}
\begin{proof}
  By the Markov inequality it holds for all $\lambda > 0$
  \begin{gather}
    \P
    \left( 
      G_n f 
      > 
      t
    \right)
    \le
    e^{-\lambda t}
    \E
    \exp
    \left( 
      \lambda
      G_n f 
    \right)
  \end{gather}
\end{proof}


\begin{lemma}
  For any finite class $\mathcal{F}$ of bounded, measurable, square-integrable functions, with $\left| \mathcal{F} \right|$ elements, it holds
  \begin{gather}
    \E \norm{G_n f}_\mathcal{F}
    \lesssim
    \max_
          {f \in \mathcal{F}}
          \frac{\norm{f}_\infty}{\sqrt{n}}
          \log
          \left( 1 + \left| \mathcal{F} \right| \right)
          +
          \max_
              {f \in \mathcal{F}}
              \norm{f}_{\P,2}
              \sqrt{
                \log \left( 
                  1 + \left| \mathcal{F} \right|
                \right)
              }
              .
  \end{gather}
\end{lemma}

\begin{lemma}
  For any class 
  $\mathcal{F}$
  of measurable functions 
  $
    f:
    \mathcal{X}
    \to \R
  $
  such that 
  $
    \P f^2
    <
    \delta^2
  $
  for every $f,$
  we have,
  with
  $
    a(\delta)
    =
    \delta
    /
    \sqrt{
      \mathrm{Log}
    N
    _{[]}
    (
    \delta
    ,
    \mathcal{F}
    ,
    L_2(\P)
    )
    }
    ,
  $
  and $F$
  an envelope function,

  \begin{gather}
    \E^*
    _\P
    [
    \norm{
      \G
      _n
      }
      _\mathcal{F}
    ]
    \lesssim
    J
    _{[]}
    (
      \delta
    ,
    \mathcal{F}
    ,
    L_2(\P)
    )
    +
    \sqrt{n}
    \P^*
    F
    \left\{ 
      F > \sqrt{n}
      a(\delta)
    \right\}
    .
  \end{gather}
\end{lemma}


\begin{corollary}
  For any class $\mathcal{F}$ of measurable functions with envelope function $F,$
  \begin{gather}
    \E^*
    _\P
    [
    \norm{
      \G
      _n
      }
      _\mathcal{F}
    ]
    \lesssim
    J
    _{[]}
    (
    \norm{
      F
    }
    _{\P,2}
    ,
    \mathcal{F}
    ,
    L_2(\P)
    )
    .
  \end{gather}
\end{corollary}





%\section{Functional Delta Method}
%\begin{definition}
  A map 
  $
  \phi:
  \mathbb{D}_\phi
  \to 
  \mathbb{E}
  ,
  $
  defined on a subset 
  $
  \mathbb{D}_\phi
  $
  of a normed space
  $\mathbb{D}$
  that contains 
  $\theta,$
  is called 
  \textbf{Hadamard diffenertiable}
  at $\theta$
  if there exists a continuous,
  linear map
  $
  \phi_\theta^{'}
    :
    \mathbb{D}
    \to 
    \mathbb{E}
  $
  such that
  \begin{gather}
    \norm{
      \frac{
        \phi(\theta + t h_t)
        -
        \phi(\theta)
      }{
        t
      }
      -
      \phi^{'}_\theta
      (h)
    }_\mathbb{E}
    \to
    0
    \quad
    \text{as}
    \ 
    t\searrow 0
    \ 
    \text{for all}
    \ 
    h_t \to h
  \end{gather}
  $
    \text{such that $\theta + th_t$ is contained in $\mathbb{D}_\phi$ for all small $t>0.$}
  $
\end{definition}


\begin{ftheorem}
  \emph{(Delta Method)}
  Let 
  $
    \mathbb{D}
    \ \text{and}
    \ 
    \mathbb{E}
  $
  be normed linear spaces.
  Let
  $
    \phi
    :
    \mathbb{D}_\phi
    \subseteq
    \mathbb{D}
    \to
    \mathbb{E}
  $
  be Hadamard differentiable it $\theta$
  tangentially to 
  $\mathbb{D}_0.$
  Let
  $
    T_n
    :
    \Omega_n
    \to
    \mathbb{D}_\phi
  $
  be maps such that 
  $
    r_n
    (T_n - \theta)
    \rightsquigarrow
    T
  $
  for some sequence of numbers $r_n \to \infty$
  and a random element $T$
  that takes its values in $\mathbb{D}_0.$
  Then 
  $
    r_n(\phi(T_n)-\phi(\theta))
    \rightsquigarrow
    \phi^{'}
    _\theta
    (T)
    .
  $
  If 
  $
    \phi^{'}
    _\theta
  $
  is defined and continuous on the whole space $\mathbb{D},$
  then we also have 
  $
    r_n
    (
      \phi(T_n)
      -
      \phi(\theta)
    )
    =
    \phi^{'}
    _\theta
    (
    r_n
    (
    T_n
    -
    \theta
    )
    )
    +
    o_\P
    (1)
    .
  $
\end{ftheorem}
\begin{proof}
  \cite[Theorem~20.8]{Vaart1998}
\end{proof}


%\chapter{Simple yet useful Calculations} 
%\begin{proposition}
  Let 
  $f : \R^n \to \R$ 
  be continuous such that 
  a minimum $x^*$ exists and is unique.
  Then 
  for all $y \in \R^n$ and $C>0$ 
  it follows
    \begin{gather}
      \inf_{\norm{\Delta}=C} f(y+\Delta) - f(y) > 0 \qquad
      \Rightarrow \qquad 
      \norm{x^* - y} \le C.
    \end{gather}
\end{proposition}


\begin{proof}
Since 
$\mathcal{C}:=\left\{ \norm{\Delta}\le C \right\}$
is compact and
\begin{gather*}
  f(x^*) \le f(y) <  \inf_{\norm{\Delta}=C} f(y+\Delta)
\end{gather*}
the continious function $f(y+\,\cdot\,)$ has a minimum in 
$\overset{\circ}{\mathcal{C}}:=\left\{ \norm{\Delta} < C \right\}$. 
Since 
$x^*$ is the unique minimum of $f$
there exists $\Delta^* \in \overset{\circ}{\mathcal{C}}$ 
such that 
$x^* - y = \Delta^*$.
We conclude that
$\norm{x^* - y} \le C$.
\end{proof}

 %%%%%%%%%%%%%%%%%%%%%%%%%%%%%%%%%%%%%%%%%%%%%%%%%%%%%%%%%%%%%%%%


\begin{theorem}
  \emph{(Multivariate Taylor Theorem)}
  Let 
  $f \in C^2(\R^n, \R)$.
  Then 
  for all $x, \Delta \in \R^n$
  there exists $\xi \in [0,1]$
  such that 
  it holds
  \begin{gather}
    \label{syu_taylor}
    f(x + \Delta) - f(x)
    =
    \sum_{i = 1}^{n} \frac{\partial f(x)}{\partial x_i} \Delta_i    
    + \sum_{i,j = 1}^{n}  \frac{\partial^2 f(x + \xi \Delta)}{\partial x_i \partial x_j} \Delta_i\Delta_j
    + \frac{1}{2}\sum_{i = 1}^{n} \frac{\partial^2 f(x + \xi \Delta)}{\partial x_i^2}\Delta_i^2 
  \end{gather}
\end{theorem}


\begin{corollary}
  Let 
  $f\in C^2(\R)$. 
  Then
  for all $a,x,\Delta \in \R^n$ 
  there exist $\xi \in [0,1]$ 
  such that it holds
  \begin{gather}
    \label{syu_2.result}
    f(a^T (x + \Delta)) - f(a^T x) = 
    f^{'}(a^T x)\, \Delta^T a + 
    \frac{1}{2}f^{''}(a^T (x + \xi \Delta))\, \Delta^T A\ \Delta,
  \end{gather}
  where 
  $A:= a a^T \in \R^{n \times n}$ .
\end{corollary}

\begin{proof}
  By the chain rule 
  we have
  for all $a, x, \Delta \in \R^n$ and $\xi \in [0,1]$
  \begin{gather}
    \label{syu_2.0}
   \frac{\partial^2  f(a^T (x + \xi \Delta)) }{\partial x_i \partial x_j}=
    f^{''}(a^T (x + \xi \Delta))\, a_i a_j.
  \end{gather}
  Since 
  $A:= a a^T$ is symmetric
  we have
  \begin{gather}
    \label{syu_2.1}
    \Delta^T A\ \Delta 
    =
    2
    \sum_{
    \begin{smallmatrix}
      i,j = 1 \\ 
      i \neq j
    \end{smallmatrix}
    }
    ^{n}
    a_i a_j \Delta_i \Delta_j
    + 
    \sum_{i=1}^{n} 
    a_i^2 \Delta_i^2.
  \end{gather}
  Plugging \eqref{syu_2.0} and \eqref{syu_2.1} into \eqref{syu_taylor}
  yields
  \eqref{syu_2.result}.
\end{proof}



%
%makeindex main.nlo -s nomencl.ist -o main.nls

\nomenclature{$\P$}{generic probability measure}
\nomenclature{$\overset{\mathcal{D}}{\longrightarrow}$}{convergence of distributions}

%\printnomenclature
\renewcommand{\bibsection}
{
\chapter*{References}
\addcontentsline{toc}{chapter}{References}
}
\bibliographystyle{alpha}
\bibliography{literature}
\printindex
\end{document}
