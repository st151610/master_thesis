\documentclass[11pt, a4paper, BCOR=10mm, DIV=11]{scrbook}
\linespread{1.25}
\usepackage[utf8]{inputenc}
\usepackage{graphicx}
%\usepackage[a4paper, margin=2.5cm]{geometry}
\usepackage{hyperref}
\usepackage{amsmath}
\usepackage{enumitem}
\usepackage{mdframed}
\usepackage{amsthm}
\usepackage{amssymb}
\usepackage{mathtools}
%\usepackage{fdsymbol}
\usepackage{cite}
\graphicspath{ {images/} }
\usepackage{xcolor}         % Extended colors
\usepackage{color}         % Color extended names
\usepackage{nomencl}
\usepackage{lipsum}
\usepackage[tickmarkheight=0.1cm, colorinlistoftodos]{todonotes}
\makenomenclature
\renewcommand{\nomname}{Notation Index}
\mathsurround=2pt


\newsavebox{\selvestebox}
\newenvironment{takeaways}
  {
   \begin{lrbox}{\selvestebox}%
   \begin{minipage}{12.4cm}
     \textbf{Takeaways}
   }
  {\end{minipage}\end{lrbox}%
   \begin{center}
\setlength\fboxsep{.5cm}
   \colorbox[HTML]{F8E0E0}{\usebox{\selvestebox}}
   \end{center}}
% Theorem handle


\newtheorem{theorem}{Theorem}[chapter]
\newtheorem*{theorem*}{Theorem}
\newtheorem*{definition*}{Definition}
\newtheorem*{lemma*}{Lemma}
\newenvironment{ftheorem}
  {\begin{mdframed}\begin{theorem}}
  {\end{theorem}\end{mdframed}}
\newtheorem{assumption}{Assumption}[chapter]
\newtheorem*{assumptions*}{Assumptions}
\newtheorem{corollary}{Corollary}[theorem]
\newtheorem{definition}{Definition}[chapter]
\theoremstyle{definition}
\newtheorem{example/}{Example}[chapter]
\newenvironment{example}
  {\renewcommand{\qedsymbol}{$\diamondsuit$}%
   \pushQED{\qed}\begin{example/}}
  {\popQED\end{example/}}
\newtheorem*{example*/}{Example}
\newenvironment{example*}
  {\renewcommand{\qedsymbol}{$\diamondsuit$}%
   \pushQED{\qed}\begin{example*/}}
  {\popQED\end{example*/}}
\newtheorem*{reflection*/}{Reflection}
\newenvironment{reflection*}
  {\renewcommand{\qedsymbol}{$\spadesuit$}%
   \pushQED{\qed}\begin{reflection*/}}
  {\popQED\end{reflection*/}}
\theoremstyle{plain}
\newtheorem{lemma}{Lemma}[chapter]
\newtheorem{problem}{Problem}[chapter]
\newtheorem{proposition}{Proposition}[chapter]
\newtheorem{remark/}{Remark}[chapter]
\newenvironment{remark}
  {\renewcommand{\qedsymbol}{$\diamondsuit$}%
   \pushQED{\qed}\begin{remark/}}
  {\popQED\end{remark/}}
\newtheorem{subassumption}{}[assumption]

\renewcommand{\proofname}{\textbf{Proof}}
% definition of constants

\newcommand{\CP}{C_\P}
\newcommand{\Ctau}{C_\tau}
\newcommand{\LearnRate}{\varepsilon_n}

%
\DeclarePairedDelimiterX{\inner}[2]{\langle}{\rangle}{#1, #2}
\newcommand{\C}{\mathbb{C}}
\newcommand{\G}{\mathbb{G}}
\newcommand{\E}{\mathbf{E}}
\renewcommand{\P}{\mathbf{P}}
\newcommand{\R}{\mathbb{R}}
\newcommand\norm[1]{\left\lVert#1\right\rVert}

\title{
  {
    Robust Weighting and Matching Techniques for Causal Inference in Observational Studies with Continuous Treatment
  }
  \\
  {\large Universität Stuttgart}
  \\
  {\includegraphics{unistuttgart_logo_deutsch.jpg}}
}

\author{Ioan Scheffel}
\date{\today}
\setlength {\marginparwidth }{2cm}

\begin{document}

\listoftodos

\maketitle

\tableofcontents 

%\chapter{Introduction}
%% SAY THAT RESEARCHERS ARE OFTEN LEFT WITH OBSERVATIONAL STUDYS TO ANSWER THEIR QUESTIONS

% SAY SOMETHING ABOUT PS METHODS AS TO DO WITH CAUSAL INFERENCE IN OS

% SAY THAT MODEL DEPENDENCY IS OFTEN A PROBLEM WITH PS

% MENTION DIFFERENT APPROACHES TO ALLEVIATE THIS PROBLEM

% HIGHLIGHT THE METHODS BEING DISCUSSED IN THE THESIS

% OUTLINE STRUCTURE 
%% - WEIGHTING APPROACH BINARY
%%  - MODIFICATIONS IN BINARY SETTING TO ESTIMATE ATE
%%  - EXTENSION TO CONTINUOUS TREATMENT
%% - MATCHING APPROACH BINARY 
%%  - POSSIBLE EXTENSION
%% - SIMULATION STUDY
%% - REAL DATA USECASES


Researchers are often left with observational studies to answer questions about causality. When confounders are present the task of infering causality can become arbitrarily complex. Propensity score methods \cite{Rosenbaum1983}, e.g. inverse probability weighting or matching, are popular methods to adjust for confounders. Usually these methods rely heavily on estimates of the true propensity score, which are known to suffer from model dependencies and misspecification\cite{Kang2007}. This issue becomes more pressing when moving from binary to continuous treatment\cite{Hirano2005}. Therefore methods have been developed to directly target imbalances in the data\cite{Fong2018}\cite{Hainmueller2012}\cite{Zubizarreta2015}.
We take a closer look at \cite{Wang2019} and extend the analysis to settings with continuous treatment 
\cite{Vegetabile2020}\cite{Tubbicke2020}.

%\chapter{Causal Inference}
%In this chapter we want to give a introduction to causal inference.
We particularly highlight the role of propensity score analysis and explain its importance in observational studies.


\section{The Rubin Causal Model}
from wiki:
The Rubin causal model (RCM), also known as the Neyman–Rubin causal model,[1] is an approach to the statistical analysis of cause and effect based on the framework of potential outcomes, named after Donald Rubin. The name "Rubin causal model" was first coined by Paul W. Holland.[2] The potential outcomes framework was first proposed by Jerzy Neyman in his 1923 Master's thesis,[3] though he discussed it only in the context of completely randomized experiments.[4] Rubin extended it into a general framework for thinking about causation in both observational and experimental studies.[1]
\section{Propensity Score Analysis}
from wiki:
In the statistical analysis of observational data, propensity score matching (PSM) is a statistical matching technique that attempts to estimate the effect of a treatment, policy, or other intervention by accounting for the covariates that predict receiving the treatment. PSM attempts to reduce the bias due to confounding variables that could be found in an estimate of the treatment effect obtained from simply comparing outcomes among units that received the treatment versus those that did not. Paul R. Rosenbaum and Donald Rubin introduced the technique in 1983.[1]

The possibility of bias arises because a difference in the treatment outcome (such as the average treatment effect) between treated and untreated groups may be caused by a factor that predicts treatment rather than the treatment itself. In randomized experiments, the randomization enables unbiased estimation of treatment effects; for each covariate, randomization implies that treatment-groups will be balanced on average, by the law of large numbers. Unfortunately, for observational studies, the assignment of treatments to research subjects is typically not random. Matching attempts to reduce the treatment assignment bias, and mimic randomization, by creating a sample of units that received the treatment that is comparable on all observed covariates to a sample of units that did not receive the treatment.

For example, one may be interested to know the consequences of smoking. An observational study is required since it is unethical to randomly assign people to the treatment 'smoking.' The treatment effect estimated by simply comparing those who smoked to those who did not smoke would be biased by any factors that predict smoking (e.g.: gender and age). PSM attempts to control for these biases by making the groups receiving treatment and not-treatment comparable with respect to the control variables. 

from a paper:
Propensity score weighting is one of the techniques used in controlling for selection biases in non-
experimental studies. Propensity scores can be used as weights to account for selection assignment
differences between treatment and comparison groups. One of the advantages of this approach is
that all the individuals in the study can be used for the outcomes evaluation


\section{Weighting beyond the PS}
from \cite{Wang2019}:
Conventionally, the weights are estimated by modeling the propensities of receiving treatment or exhibiting missingness and then inverting the predicted propensities. However, with this approach it can be difficult to properly adjust for or balance the observed covariates. The reason is that this approach only balances covariates in expectation, by the law of large numbers, but in any particular data set it can be difficult to balance covariates, especially if the data set is small or if the covariates are sparse (Zubizarreta et al., 2011). In addition, this approach can result in very unstable estimates when a few observations have very large weights (e.g., Kang and Schafer 2007). To address these problems, a number of methods have been proposed recently. Instead of explicitly modeling the propensities of treatment or missingness, these methods directly balance the covariates. Some of these methods also minimize a measure of dispersion of the weights.

Most of these weighting methods balance covariates exactly rather than approximately. This is a subtle but important difference because approximate balance can trade bias for variance whereas exact balance cannot. Also, exact balance may not admit a solution whereas approximate balance may do so. For a fixed sample size, approximate balance may balance more functions of the covariates than exact balance.


\chapter{Balancing Weights}
 % \section{Introduction}
  %We work in the Rubin Causal Model.

We assume a sample of $n$ units which is drawn from a population distribution.

In i.i.d. fashion.

We observe $ (\mathbf{X}_i, T_i, Y_i) $,
where $\mathbf{X}$ are covariates, 
$T$ is the indicator if treatment has been received
and $Y$ is the observed outcome.

In the Rubin Causal Model we assume that for each unit the potintial outcome exist, i.e. $(Y_i^0, Y_i^1)$ where $Y^1$ stands for the potential outcome had the unit received treatment and $Y^0$ for the potential outcome had the unit received \textbf{no} treatment.

It is clear that $Y_i = Y_i^{T_i}$ i.e. we can observe only one of the potential outcomes.

Thus there is a connection to missing data problems.

This is the dillema of causal inference.
 
On the population level it is possible to estimate both.

Usually the means of the potential outcomes are compared against each other.

In randomized trials this is a valid approach to causal inferece.

In observational studies however the treatment assignment is not known and direct comparison can lead to systematically wrong results.

This phenomenon is called \textbf{confounding}.
 
To address the issue of confounding many methods have been proposed.

  \section{Double Robustness}
  read the paper \cite{Zhao2017a} for discussion of double robustness for balancing weights and \cite{Hahn1998} for semiparametric efficiency bounds.


What is the best rate we can achieve?

\begin{theorem}
  \emph{(Hartman-Winter)}
  Let 
  $
    X_1,
    X_2,
    \ldots
  $
  be i.i.d. real random variables with 
  $
    \E[X_1]=0
    \ 
    \text{and}
    \ 
    \mathbf{Var}
    [X_1]
    = 1.
  $
  Let
  $
    S_n
    :=
    X_1
    +
    \ldots
    +
    X_n,
    \ 
    n\in \mathbb{N}
    .
  $
  Then
  \begin{gather}
    \limsup_{n\to\infty}
    \frac{S_n}{
      \sqrt{
        2n
        \log
        \log
        n
      }
    }
    =
    1
    \quad
    \text{a.s.}
  \end{gather}
\end{theorem}
\begin{proof}
  \cite[Theorem~22.11]{Klenke2020}
\end{proof}


We calculate mean and variance

\begin{align}
  \E
  \left[ 
    (T/\pi(X))
    Y(T)
  \right]
  &=
  \E
  \left[ 
    Y(1)
    /\pi(X)
    \vert
    T=1
  \right]
  \P[T=1]
  \\
  &=
  \int_\mathcal{X}
  \left[ 
    Y(1)
    \vert
    T=1,
    X=x
  \right]
  \cdot
  \P[T=1]
  /
  \pi(x)
  \P_{X|T}(dx|1)
  \\
  &=
  \int_\mathcal{X}
  \left[ 
    Y(1)
    \vert
    X=x
  \right]
  \P_X(dx)
  =
  \E[Y(1)]
  .
\end{align}
The third equality is due to weak unconfoundedness and Bayes rule.
With the same arguments as above it follows
$
  \mathbf{Var}
  [
    (T/\pi(X))
    Y(T)
  ]
    =
    \E[
    (Y(1))^2
    /
    \pi(X)
    ]
    -
    \E[Y(1)]^2
    =:
    \mathbf{V}^*
    .
$
We readily calculate the learning rate 
\begin{gather}
  \sqrt{
    \mathbf{V}^*
    \frac{\log\log n}{n}
  }
  .
\end{gather}

  \section{Error Decompositions}
  The following decomposition is flexible in $\Phi.$
We get different causal estimaands $\E[\Phi(Y(1))],$
e.g.
the popolation average of $Y(1)$
for 
$
  \Phi(Y)=Y,
$
i.e.
$
\E[Y(1)]
,
$
ot the distribution function of $Y(1)$ at $t$
for 
$
  \Phi(Y)=
  \mathbf{1}_{(-\infty, t]}
  (Y)
  ,
$
i.e.
$
\P[
Y(1)\le t
]
.
$
\begin{gather}
 \sum_{i=1}^{n}  
 w_i
 T_i
 \Phi(Y_i)
  - \E [ \Phi(Y(1)) ]
  =
  \frac{1}{n}
  \sum_{ i = 1 }^{n} S_i
    + R_0
    + R_1
    + R_2
    ,
\end{gather}
where
\begin{align*}
  S_i 
  &:= 
  \frac{T_i}{\pi_i}
 \left( 
   \Phi(Y_i) - \E [ \Phi(Y_i(1)) | X_i ]
 \right)
 +
 \left( 
   \E [ \Phi(Y_i(1)) | X_i ] - \E [ \Phi(Y(1))]
 \right)
 \quad
 \text{for}\ 
 i \in \left\{ 1, \ldots, n, \right\}
 ,
 \\
  R_0
  &:=
  \sum_{ i = 1 }^{n}
  T_i
    \left(  
      w_i - \frac{1}{n \pi_i}
    \right)
 \left( 
   \Phi(Y_i) - \E [ \Phi(Y_i(1)) | X_i ]
 \right)
 ,\\
  R_1
  &:=
  \sum_{ i = 1 }^{n}
    \left(  
      T_i
      w_i - \frac{1}{n}
    \right)
 \left( 
    \E [ \Phi(Y_i(1)) | X_i ] - B(X_i)^\top \lambda
 \right)
 ,
 \\
  R_2
  &:=
  \sum_{ i = 1 }^{n}
    \left(  
      T_i
      w_i - \frac{1}{n}
    \right)
 B(X_i)^\top \lambda
 \quad 
 \text{for}
 \ 
 \lambda \in \R^K
 .
\end{align*}

We can even view
$
\frac{1}{\sqrt{n}}
\sum_{i=1}^{n}S_i 
$
as an empirical process 
$
\mathbb{G}_n f
$
indexed over 

\begin{gather}
  f_\Phi(T,X,Y)
  =
  \frac{T}{\pi(X)}
 \left( 
   \Phi(Y) - \E [ \Phi(Y) | X ]
 \right)
 +
   \E [ \Phi(Y) | X ] 
   .
\end{gather}
If $\mathcal{F}=\left\{ f_\Phi \colon \Phi \in \ \text{some set}\right\}$
is $\P$-Donsker, the empirical process converges to a tight gaussian process.
Then the functional delta Method is applicable.

  %\section{Estimating the Population Mean of Potential Outcomes}
  %%%%%%%%%%%%%%%%%%%%ASSUMPTION 1%%%%%%%%%%%%%%%%%%%%%%%%%%%%%%%%%%%%%%%
\begin{asu}
  \label{assumption_1}
  Assume, the following conditions hold:
\\
  \subasu 
    \label{assumption_1_i} 
    The minimizer 
    $
    \lambda_0 
    =
    \arg \min_{\lambda \in \Theta}
    \E
    \left[ 
      -T n 
      \rho 
      \left( 
      B(X)^T \lambda
      \right)
      +
      B(X)^T \lambda
    \right]
    $
    is unique,
    where 
    $\Theta \subseteq \R^n$ is the parameter space for $\lambda$.
\\ %%%%%%%%%%%%%%%%%%%%%%%%%%%%%%%%%%%%%%%%%%%%%%%%%%%%%%%%%%%%%%%%%
  \subasu 
    \label{assumption_1_ii} 
    The parameter space 
    $\Theta \subseteq \R^n$
    is compact compact with diameter
    $\text{diam}(\Theta) < \infty$.
\\ %%%%%%%%%%%%%%%%%%%%%%%%%%%%%%%%%%%%%%%%%%%%%%%%%%%%%%%%%%%%%%%%%%%
  \subasu 
    \label{assumption_1_iii}
    $\lambda_0 \in \text{int}(\Theta)$,
    where
    $\text{int}(\cdot)$
    stands for the interior of a set.
\\ %%%%%%%%%%%%%%%%%%%%%%%%%%%%%%%%%%%%%%%%%%%%%%%%%%%%%%%%%%%%%%%%%%%
  \subasu
    \label{assumption_1_iv}
    There exists
    $\lambda^*_1 \in \Theta$
    such that
    $
      \norm{
        m^*(\cdot)
        -
        B(\cdot)^T \lambda^*_1
      }_\infty
      \le 
      \varphi_{m^*}
    $,
    where
    $
      m^*(\cdot)
      :=
      \left( \rho^{'} \right)^{-1}
      \left( \frac{1}{n \pi(\cdot)} \right).
    $
\\ %%%%%%%%%%%%%%%%%%%%%%%%%%%%%%%%%%%%%%%%%%%%%%%%%%%%%%%%%%%%%%%%%%%%%
  \subasu
    \label{assumption_1_v}
    There exists a constant 
    $
      \varphi_{\rho^{'} \lor \pi} 
      \in 
      \left(0, \frac{1}{2} \right)
    $
    such that
    $
      n\rho(v) 
      \in 
      (
      \varphi_{\rho^{'} \lor \pi},
      1 - \varphi_{\rho^{'} \lor \pi}
      )
    $
    for $v=B(x)^T \lambda$ with $\lambda \in \text{int}(\Theta)$ 
    \textbf{or}
    $
      \pi(x)
      \in 
      (
      \varphi_{\rho^{'} \lor \pi},
      1 - \varphi_{\rho^{'} \lor \pi}
      )
    $.
\\ %%%%%%%%%%%%%%%%%%%%%%%%%%%%%%%%%%%%%%%%%%%%%%%%%%%%%%%%%%%%%%%
  \subasu
    \label{assumption_1_vi}
    There exists 
    $ \varphi_{\rho^{''}} > 0 $
    such that
    $ -\rho^{''} \ge \varphi_{\rho^{''}} > 0 $
\\ %%%%%%%%%%%%%%%%%%%%%%%%%%%%%%%%%%%%%%%%%%%%%%%%%%%%%%%%%%%%%%%%
  \subasu
    \label{assumption_1_vii}
    There exists 
    $ \varphi_{B(x) B(x)^T} > 0 $ 
    such that
    $
      B(x) B(x)^T 
      \succcurlyeq 
      \varphi_{B(x) B(x)^T} I 
    $
\\ %%%%%%%%%%%%%%%%%%%%%%%%%%%%%%%%%%%%%%%%%%%%%%%%%%%%%%%%%%%%%%%%
  \subasu
    \label{assumption_1_viii}
    There exists
    $ \varphi_{\norm{B}} > 0 $
    such that
    $
      \sup_{x \in \mathcal{X}} \norm{B(x)}_2
      \le 
      \varphi_{\norm{B}}
    $.
\end{asu}

%%%%%%%%%%%%%%%%%%%%%%%%%%%%%%%%%%%%%%%%%%%%%%%%%%%%%%%%%%%%%%%%%%

We study the following problem:

\begin{align}
  \label{primal_weighting_binary}
  \begin{split}
  &\underset{w \in \R^n}{\text{minimize}}
  \qquad
  \sum_{i = 1}^{n} T_i f(w_i)
  \\
  &\text{subject to}
  \left| 
    \sum_{i = 1}^{n} w_i T_i B_k(X_i)
    - 
    \frac{1}{n} \sum_{i = 1}^{n} B_k(X_i)
  \right|
  \le 
  \delta_k,\,
  k = 1, \ldots, K
  \end{split}
\end{align}


\begin{proposition}
  \label{ch_1_dual}
  The dual of Problem \eqref{primal_weighting_binary} is equivalent to the unconstrained optimization problem
  \begin{gather}
    \label{dual_weighting_binary}
      \underset{\lambda \in \R^K}{\text{minimize}}
      \quad
      \frac{1}{n}
      \sum_{j = 1}^{n} 
      \left[ 
        -T_j n 
        \rho 
        \left( 
          B(X_j)^T \lambda
        \right)
      +
      B(X_j)^T \lambda
      \right]
      +
      |\lambda|^T \delta
  \end{gather}
\end{proposition}


\begin{proposition}
  \label{ch_1_near_oracle}
  There exists a solution $\lambda^\dagger$ 
  to \eqref{dual_weighting_binary}
  such that
  \begin{gather}
    \P
    \left( 
      \norm{
        \lambda^\dagger
        -
        \lambda^*_1
      }_2
      \le
      \CP \Ctau \LearnRate
    \right)
    \ge 
    1 - \tau
    .
  \end{gather}
\end{proposition}


\section{Plan of proof}
We employ 
Theorem~\ref{cvxa_fenchel_theorem}
together with the box constraints in Problem~\eqref{primal_weighting_binary}
to obtain Proposition~\ref{ch_1_dual}.

To prove Proposition~\ref{ch_1_near_oracle}
we employ
Proposition~\ref{syu_1_result}
and 
Corollary~\ref{syu_taylor_corollary}
to get

\begin{align}
  \begin{split}
  & 
  G(\lambda^*_1 + \Delta) 
  -
  G(\lambda^*_1)
  \\
  &\ge
      \frac{1}{n}
      \sum_{j = 1}^{n} 
      \left[ 
        -T_j n 
        \rho^{'} 
        \left( 
          B(X_j)^T \lambda^*_1
        \right)
      +
      1
      \right]
      \Delta^T B(X_j)
      \\
  & +
      \frac{1}{2}
      \sum_{j = 1}^{n} 
        -T_j  
        \rho^{''} 
        \left( 
          B(X_j)^T (\lambda^*_1 + \xi \Delta)
        \right)
        \Delta^T
        \left( 
          B(X_j)
          B(X_j)^T
        \right)
        \Delta
        \\
  &-
      |\Delta|^T \delta
  \\
  &\ge
    - \norm{\Delta}_2
    \left( 
    \norm{
      \frac{1}{n}
      \sum_{j = 1}^{n} 
      \left[ 
        -T_j n 
        \rho^{'} 
        \left( 
          B(X_j)^T \lambda^*_1
        \right)
      +
      1
      \right]
      B(X_j)
    }_2
    +
    \norm{\delta}_2
    \right)
    \\
  &+
  n
  \norm{\Delta}^2_2
   \varphi_{\rho^{''}}
  \underline{\varphi_{aa^T}}
  \end{split}
\end{align}

Next we employ Bernstein inequality~\ref{rmineq_bernstein} to bound
\begin{align}
    \norm{
      \frac{1}{n}
      \sum_{j = 1}^{n} 
      \left[ 
        -T_j n 
        \rho^{'} 
        \left( 
          B(X_j)^T \lambda^*_1
        \right)
      +
      1
      \right]
      B(X_j)
    }_2
    \le
    \CP \Ctau \LearnRate
\end{align}
with probability $1 - \tau$.
Then for 
$\norm{\Delta}_2$ large enough it holds
\begin{gather}
  G(\lambda^*_1 + \Delta) 
  -
  G(\lambda^*_1)
  >
  0
\end{gather}
with probability $1 - \tau$.
Thus by Proposition~\ref{syu_1_result}
  \begin{gather}
    \P
    \left( 
      \norm{
        \lambda^\dagger
        -
        \lambda^*_1
      }_2
      \le
      \norm{\Delta}_2
    \right)
    \ge 
    1 - \tau
    .
  \end{gather}
It is then straightforward to prove

\begin{theorem}
  Let 
  $\lambda^\dagger$
  be the solution to Problem~\ref{dual_weighting_binary}
  and 
  $w^*(x)=\rho^{'}\left( B(x)^T \lambda^\dagger \right)$.
  Then under the conditions in Assumption~\ref{assumption_1}
  it holds

  \begin{align}
  \norm{
    w^*(\cdot)
    -
    \frac{1}{n \pi(\cdot)}
  }_{\P, 2}
  \le 
  \text{stuff}
  \end{align}
and 
  \begin{align}
    \P
    \left( 
  \norm{
    w^*(\cdot)
    -
    \frac{1}{n \pi(\cdot)}
  }_\infty
  \le 
  \text{stuff}
   \right)
   \ge
  1 - \tau
  .
  \end{align}
\end{theorem}


\begin{proof}
  Motivated by Proposition~\ref{syu_1_result}
  we set
  $\norm{\Delta}_2 = C$ 
  and consider
 \begin{gather}
   G(\lambda)
   :=
      \frac{1}{n}
      \sum_{j = 1}^{n} 
      \left[ 
        -T_j n 
        \rho 
        \left( 
          B(X_j)^T \lambda
        \right)
      +
      B(X_j)^T \lambda
      \right]
      +
      |\lambda|^T \delta.
 \end{gather} 
 Since 
 $\rho \in C^2(\R)$
 we can employ 
 Proposition~\ref{syu_1_result},
 Corollary~\ref{syu_taylor_corollary}
 and
 Proposition~\ref{syu_triangle}
 to get
 \begin{align}
  \begin{split}
  & 
  G(\lambda^*_1 + \Delta) 
  -
  G(\lambda^*_1)
  \\
  &\ge
      \frac{1}{n}
      \sum_{j = 1}^{n} 
      \left[ 
        -T_j n 
        \rho^{'} 
        \left( 
          B(X_j)^T \lambda^*_1
        \right)
      +
      1
      \right]
      \Delta^T B(X_j)
      \\
  & +
      \frac{1}{2}
      \sum_{j = 1}^{n} 
        -T_j  
        \rho^{''} 
        \left( 
          B(X_j)^T (\lambda^*_1 + \xi \Delta)
        \right)
        \Delta^T
        \left( 
          B(X_j)
          B(X_j)^T
        \right)
        \Delta
        \\
  &-
      |\Delta|^T \delta
  \\
  &\ge
    - \norm{\Delta}_2
    \left( 
    \norm{
      \frac{1}{n}
      \sum_{j = 1}^{n} 
      \left[ 
        -T_j n 
        \rho^{'} 
        \left( 
          B(X_j)^T \lambda^*_1
        \right)
      +
      1
      \right]
      B(X_j)
    }_2
    +
    \norm{\delta}_2
    \right)
    \\
  &+
  n
  \norm{\Delta}^2_2
   \varphi_{\rho^{''}}
  \underline{\varphi_{aa^T}}
  \\
  &:=
  -\norm{\Delta}_2
  (I_1 + \norm{\delta}_2)
  +
  \norm{\Delta}^2_2
  I_2
  .
  \end{split}
\end{align}
The second inequality is due to 
the Cauchy-Schwarz-Inequality 
and
Assumptions~\ref{assumption_1_vi} and \ref{assumption_1_vii}
.
\subsection*{Analysis of $I_1$}
We want to use Assumption~\ref{assumption_1_iii}.
Thus we perform the following split:
\begin{align}
  I_1 
  &\le
    \norm{
      \sum_{j = 1}^{n} 
        T_j  
      \left[ 
        \rho^{'} 
        \left( 
          B(X_j)^T \lambda^*_1
        \right)
      -
      \frac{1}{n \pi(X_j)}
      \right]
      B(X_j)
    }_2
  \\
  &+
    \norm{
      \frac{1}{n}
      \sum_{j = 1}^{n} 
      \left[ 
        \frac{T_j}{\pi(X_j)}
      -
      1
      \right]
      B(X_j)
    }_2
    \\
  &=:
  J_1 + J_2
\end{align}

\subsubsection*{Analysis of $J_1$}

By the Lipschitz-continuity of 
$\rho^{'}$,
Assumption~\ref{assumption_1_viii}
and
Assumption~\ref{assumption_1_iv},
$T \in \{0, 1\}$
and 
the triangle inequality 
we have
\begin{gather}
  J_1 
  \le
  n L_{\rho^{'}}\varphi_{\norm{B(x)}} \varphi_{m^*}
\end{gather}

\subsubsection*{Analysis of $J_2$}
We employ Bernstein Inequality for matrices
To this end we define
\begin{gather}
  A_j
  :=
      \frac{1}{n}
      \left[ 
        \frac{T_j}{\pi(X_j)}
      -
      1
      \right]
      B(X_j)
\end{gather}

\subsubsection*{$\E A_j = 0$}

It holds

\begin{gather}
  \E
  \left[  
    \frac{T_j}{\pi(X_j)}
    B(X_j)
  \right]
  =
  \E
  \left[  
    \E
    \left[  
      T_j
      \, | \,
      X_j
    \right]
    \frac{1}{\pi(X_j)}
    B(X_j)
  \right]
  =
  \E[B(X_j)]. 
\end{gather}
Thus 
$\E[A_j] = 0$.
\subsubsection*{L}
Since
\begin{gather}
  \left| 
      \frac{T_j}{\pi(X_j)}
      -
      1
      \right|
  \le
  1 + \frac{1 - \varphi_{\pi}}{\varphi_{\pi}}
  =
  \frac{1}{\varphi_{\pi}}
\end{gather}
by Assumption~\ref{assumption_1_v},
we can employ Assumption~\ref{assumption_1_viii}
to get
\begin{gather}
  \norm{A_j}_2
  \le
  \frac{\varphi_{\norm{B}}}{n \varphi_{\pi}}
  =:
  L.
\end{gather}
\subsubsection*{v(S)}
Since
\begin{gather}
  \E
  \left[ 
    A_j A_j^T
  \right]
  \le
  \left( 
    \frac{1}{n \varphi_{\pi}}
  \right)^2
  \E
  \left[ 
    B(X) B(X)^T
  \right]
\end{gather}
and
\begin{gather}
  \E
  \left[ 
    A_j^T A_j
  \right]
  \le
  \left( 
    \frac{\varphi_{\norm{B}}}{n \varphi_{\pi}}
  \right)^2
\end{gather}
we have
\begin{gather}
  v(S)
  \le
  \frac{|\lambda_{\max} | + \varphi_{\norm{B}}^2}{n \varphi_{\pi}^2},
\end{gather}
where 
$\lambda_{\max}$ is the maximal eigenvalue of
$
  \E
  \left[ 
    B(X) B(X)^T
  \right]
$.
Then by Bernsteins inequality~\ref{rmineq_bernstein}
we get
\begin{gather}
  \E[J_2]
  \le
  \sqrt{
    \frac{
    2 \log (K + 1)
    \left( 
      |\lambda_{\max} | + \varphi_{\norm{B}}^2
    \right)
    }
    {
      n \varphi_{\pi}^2
    }
  }
  +
  \frac{
    \log (K + 1)
    \varphi_{\norm{B}}
  }
  {
    3 n \varphi_{\pi}
  }
\end{gather}
and by the Markov-inequality
\begin{gather}
  \P
  \left( 
    J_2 
    \le
    \frac{1}{\tau}
    \E[J_2]
  \right)
  \ge 
  1 - \tau
\end{gather}
\end{proof}

  \section{Application of Convex Optimization}
  \subsection*{Introduction}
%\begin{assumption}
%  \begin{enumerate}[label={(\roman*)}]
%    Assume that the map 
%    $
%      f: \R \to \overline{\R}
%    $
%    has the following properties.
%    \item
%      $
%        f 
%        \ 
%        \text{is strictly convex.}
%      $
%    \item
%      $
%        f
%        \ 
%        \text{is lower-semicontinuous and continuously differentiable on}
%        \ 
%        \mathrm{int}(\mathrm{dom}(f))
%        .
%      $
%    \item
%      $
%        \text{The derivative of}\ 
%        f\ 
%        \text{on}\ 
%        \mathrm{int}(\mathrm{dom}(f))
%        \ 
%        \text{is a diffeomorphism.}
%      $
%    \item
%      $
%        \text{The Legendre transformation}
%        \ 
%        f^*
%        \ 
%        \text{of}\ 
%        f
%        \ 
%        \text{is finite}
%        .
%      $
%    \item
%      $
%        \text{The function}\
%        x\mapsto xt - f(x)
%        \ 
%        \text{takes its supremum on}
%        \ 
%        \mathrm{int}(\mathrm{dom}(f))
%        \ 
%        \text{for all}
%        \ 
%        t\in \R.
%      $
%  \end{enumerate}
%\end{assumption}
%Assumption (i),(ii) and (iv) are from \cite{Tseng1991}. See corresponding section in the thesis.
%Assumption (iii) and (v) is needed to get a meaningful analytic expression for the weights after establishing the properties pointed out in \cite{Tseng1991}.
%
%\todo[color=red!40, inline]{Elaborate on assumptions. Give as a counterexample 
%  \[
%  f(x):=
%  (1+\delta_{x\ge 0})
%  (x^2
%  +x\cdot \lambda)
%\]
%}
%
%We consider the following optimization problem.
%\todo[color=orange!40, inline]{Consider Non-negativity constraints only for $f(x)=x\log x$. Incorporating Non-negativity constraint for general $f$ is complicated or impossible. Maybe find counterexample.}
We consider a partition
$
  \mathcal{P}_n
  =
  \left\{ 
    A_{n,1}
    ,
    A_{n,2}
    ,
    \ldots
  \right\}
$
of $ \R^d $
and define
$ A_n(x) $ to be the cell of $ \mathcal{P}_n $ containing $x$.
Next we define $ m_n $ by
\begin{gather}
  m_n(Y|x)
  \ 
  :=
  \ 
  \frac
  {
    \sum_{k=1}^{n} 
    Y_k
    \cdot
    \mathbf{1}
    _
    {
      \left\{ 
      X_k \in A_n(x)
      \right\}
    }
  }
  {
    \sum_{j=1}^{n} 
    \mathbf{1}
    _
    {
      \left\{ 
      X_j \in A_n(x)
      \right\}
    }
  }
  \,.
\end{gather}
In the terminology of \cite[§4]{Gyorfi2002}
$m_n$ is called a partitioning estimate.
We want to control the sumands.
To this end we define a set of basis functions by
\begin{gather}
  B_k(x)
  \ 
  :=
  \ 
  \frac
  {
    \mathbf{1}
    _
    {
      \left\{ 
      X_k \in A_n(x)
      \right\}
    }
  }
  {
    \sum_{j=1}^{n} 
    \mathbf{1}
    _
    {
      \left\{ 
      X_j \in A_n(x)
      \right\}
    }
  }
  \qquad
  \text{for}\ 
  k \in \left\{ 1,\ldots,n \right\}
  \,.
\end{gather}
This yields
\begin{gather}
  m_n(Y|x)
  =
  \sum_{k=1}^{n} 
  Y_k
  \cdot
  B_k(x)
  \,.
\end{gather}
We consider the objective function
\begin{gather}
  f
  \ 
  \colon
  \ 
  [0,\infty)
  \ 
  \to 
  \ 
  \R
  ,
  \quad
  x
  \ 
  \mapsto
  \ 
  x \log x\,,
\end{gather}
together with 
										%%%%%%%%%%%%%
										%%%% (P) %%%%
										%%%%%%%%%%%%%
\begin{fproblem}
  \label{bw:1:primal}
\begin{align*}
  %%%% objective %%%%
    &\underset{w_1, \ldots, w_n \in \R}
    {\text{minimize}}
    &&\qquad\qquad
    \sum_{i = 1}^{n} 
    T_i
    f(w_i)
    &&&
    \\
    %%%% w_i T_i >= 0 %%%%
    &\text{subject to}
    &&\qquad\qquad
    w_i T_i
    \ge
    0
    &&&
    \qquad
    \text{for all}\ 
    i \in \left\{ 1, \ldots, n \right\}
    \,,
    \\
    %%%% 1/n sum w = 1 %%%%
    & 
    &&\qquad\qquad
    \frac{1}{n}
    \sum_{i=1}^{n} 
    T_i w_i
    =1
    \\
    %%%% box constraints %%%%
    & 
    &&\qquad
    \left| 
      \frac{1}{n} 
      \sum_{i = 1}^{n} 
      (
      w_i T_i 
      - 
      1
      )
      \cdot
      B_k(X_i)
    \right|
    \ 
    \le 
    \ 
    \delta_k
    &&&
    \qquad
    \text{for all}\ 
    k \in \left\{ 1, \ldots, n \right\}
    \,.
\end{align*}
\end{fproblem}
%%%%%%%%%%%%%
%%%% (D) %%%%
%%%%%%%%%%%%%
\subsubsection*{Dual Problem}
\begin{ftheorem}
  The dual of Problem~\ref{bw:1:primal} is the unconstrained optimization problem 
  \begin{gather*}
    \underset{\lambda \in \R^n}{\mathrm{minimize}}
    \qquad
    \frac{1}{n}
    \sum_{i = 1}^{n} 
    T_i 
    \cdot
    f^*
    (
      m_n(\lambda|X_i)
    )
    -
      m_n(\lambda|X_i)
    \ 
    +
    \ 
    \inner{\delta}{\left| \lambda \right|}
    \,,
  \end{gather*}
  where
  \begin{gather*}
  f^*
  \,
  \colon
  \, 
  \R
  \ 
  \to
  \ 
  \R
  \,
  ,
  \qquad 
  t 
  \ 
  \mapsto
  \ 
    t\,(f^{'})^{-1}(t)
  \ 
    -
  \ 
    f
    \left( 
      (f^{'})^{-1}(t)
    \right)
  \end{gather*}
  is the Legendre transformation of $f$,
  $
    B(X_i)
    =
    \left[ 
      B_1(X_i)
      ,
      \ldots
      ,
      B_K(X_i)
    \right]
    ^\top
  $
  denotes the $K$ basis functions of the covariates 
  of unit $i\in \left\{ 1, \ldots, n \right\}$
  and
  $
    \left| \lambda \right|
    =
    \left[ 
      \left| \lambda_1 \right|
      ,
      \ldots
      ,
      \left| \lambda_K \right|
    \right]
    ^\top
    ,
  $
  where $\left| \,\cdot\, \right|$
  is the absolute value of a real-valued scalar.
  Moreover, if $\lambda^\dagger$
  is an optimal solution then
  \begin{gather*}
    w_i^*
    \ 
    =
    \ 
    (f^{'})^{-1}
    \left( 
      m_n
      (
      \lambda^\dagger
      |
      X_i
      )
    \right)
    \quad
    \text{for all}\ 
    i
    \ 
    \text{with}\ 
    T_i=1
  \end{gather*}
  are uniquely part of any optimal solution to (P)
  .
\end{ftheorem}
\begin{proof}
  We prove the following Lemma at the end of the section.
  \begin{lemma}
    The dual of the optimization problem is
  \begin{gather*}
    \underset{\lambda \in \R^{2K}}{\mathrm{minimize}}
    \qquad
    \frac{1}{n}
    \sum_{i = 1}^{n} 
    T_i 
    \cdot
    f^*
    (
    \inner{Q_{\bullet i}}{\lambda}
    )
    -
    \inner{Q_{\bullet i}}{\lambda}
    \ 
    +
    \ 
    \inner{d}{\lambda}
  \end{gather*}
  subject to
  \begin{gather*}
    \lambda_k \ge 0
    \quad
    \text{for all}\ 
    k \in \left\{ 1, \ldots, K \right\}
    ,
  \end{gather*}
  where
  \begin{gather*}
    \mathbf{Q}
    :=
    \begin{bmatrix}
     % \mathbf{I}_n\\
     % \mathbf{B}(\mathbf{X})\\
      \pm \, \mathbf{B}(\mathbf{X})
    \end{bmatrix}
    ,
    \qquad
    \mathbf{B}(\mathbf{X})
    :=
    \begin{bmatrix}
      B(X_1), \ldots, B(X_n)
    \end{bmatrix}
    ,
    \qquad
    \text{and}
    \qquad
    d
    :=
    \begin{bmatrix}
      %0_n\\
      \delta \\
      \delta
    \end{bmatrix}
    .
  \end{gather*}

  \end{lemma}
  \begin{proof}
    First we disentangle the constraints.
    To this end, we get
    \begin{gather*}
      \pm\,\sum_{i = 1}^{n} w_i T_i B_k(X_i)
      \ 
      \ge
      \ 
      -n\cdot\delta_k
      \ 
      \pm 
      \ 
      \sum_{i = 1}^{n} B_k(X_i)
      \,,
      \qquad
      k=1,\ldots,K
      \,.
    \end{gather*}
    The corresponding matrix notation is
 \begin{gather*}
    \underset{w_1, \ldots, w_n \in \R}{\mathrm{minimize}}
    \qquad
    \sum_{i = 1}^{n} T_i \cdot f(w_i)
    \\
    \mathbf{Q}w 
    \ 
    \ge
    \ 
    d
    \,,
\end{gather*}
where
\begin{align*}
    T\mathbf{Q}
    &
    \ 
    :=
    \ 
    \begin{bmatrix}
      \mathrm{diag}
      [T_1,\ldots,T_n]
      \\
      \pm
      [T_1,\ldots,T_n]
      \\
      \pm\,T\mathbf{B}(\mathbf{X})
    \end{bmatrix}
    ,
    \\
    T\mathbf{B}(\mathbf{X})
    &
    \ 
    :=
    \ 
    \begin{bmatrix}
      T_1B(X_1), \ldots, T_nB(X_n)
    \end{bmatrix}
    ,
    \\
    d
    &
    \ 
    :=
    \ 
    \begin{bmatrix}
      0_n
      \\
      \pm n
      \\
      -n\cdot\delta 
      \pm\,
      \sum_{i = 1}^{n} B_k(X_i)
    \end{bmatrix}
    \,.
  \end{align*}
The convex conjugate is
\begin{gather*}
  \sum_{T_i=1} T_i f^*(\lambda_i)
  +
  \sum_{T_i=0} 
  \delta_{\left\{ 0 \right\}}(\lambda_i)
  \,,
\end{gather*}
where
\begin{gather*}
  \delta_{\left\{ 0 \right\}}
  (t)
  =
  \begin{cases}
    0,& \text{if}\, t=0,\\
    \infty,& \text{else}\,.
  \end{cases}
\end{gather*}
Note that the $i$-th column of $T\mathbf{Q}$ vanishes if 
$T_i=0$. Likewise, in the columns with $T_i=1$ we can ommit $T_i$.
In the ts chapter, the dual problem features
\begin{gather}
  f^*(A^\top p)
\end{gather}
which by example is here
\begin{gather*}
  \sum_{T_i=1} T_i f^*(T\mathbf{Q}_{\bullet i}^\top\lambda)
  +
  \sum_{T_i=0} 
  \delta_{\left\{ 0 \right\}}
(T\mathbf{Q}_{\bullet i}^\top\lambda)
  =
  \sum_{i=1}^n T_i f^*(\mathbf{Q}_{\bullet i}^\top\lambda)
  \,,
\end{gather*}
where
\begin{align*}
    \mathbf{Q}
    &
    \ 
    :=
    \ 
    \begin{bmatrix}
      \mathbf{I}_n
      \\
      \pm
      \mathrm{1}_n
      \\
      \pm\,\mathbf{B}(\mathbf{X})
    \end{bmatrix}
    \\
    \mathbf{B}(\mathbf{X})
    &
    \ 
    :=
    \ 
    \begin{bmatrix}
      B(X_1), \ldots, B(X_n)
    \end{bmatrix}
\end{align*}


The corresponding dual problem in \cite{Tseng1991} is then
\begin{gather*}
  \underset
  {\lambda_1,\ldots,\lambda_{K}\ge 0}
  {
  \mathrm{maximize}
  }
  \quad
  -
  \sum_{i=1} 
  ^n
  T_i
  \cdot
  f^*
(\mathbf{Q}_{\bullet i}^\top\lambda)
  \ 
  +
  \ 
  \inner{\lambda}{d}
  \,.
\end{gather*}
  $1/n$ the problem remains the same.

  Next we want to remove the non-negativity constraints on $\lambda$.
  To this end we write
  \begin{gather}
    \lambda
    :=
    \begin{bmatrix}
      \rho_1,
      \ldots,
      \rho_n,
      \ 
      \lambda_0^+,
      \lambda_0^-,
      \ 
      \lambda_1^+,
      \ldots,
      \lambda_n^+,
      \ 
      \lambda_1^-,
      \ldots,
      \lambda_n^-
    \end{bmatrix}
    ^\top
    \,.
  \end{gather}
  We expand the objective function $G$ of the dual problem.
  \begin{align*}
    G
    (
    \rho,
    \lambda_0^\pm,
    \lambda^\pm
    )
    =
  &-
  \sum_{i=1} 
  ^n
  T_i
  \cdot
  f^*
  \left( 
\rho_i
\ 
+
\ 
\lambda_0^+
\!
-
\lambda_0^-
\ 
+
\ 
\inner
{B(X_i)}
{\lambda^+ \!- \lambda^-}
  \right)
  \\
  &+
  \ 
  n
  \cdot
  (
\lambda_0^+
\!
-
\lambda_0^-
  )
  \ 
+
  \ 
\inner
{B(X_i)}
{\lambda^+ \!- \lambda^-}
  \ 
-
  \ 
  n
  \cdot
\inner
{\delta}
{\lambda^+ \!+ \lambda^-}
  \end{align*}
  To illustrate the procedure, we show 
  for all $i \in \left\{ 1,\ldots,n \right\}$


\begin{alignat*}{2}
  \text{either}
  &
  &&
  \qquad
      \lambda_i^+ > 0
  \\
  \text{or}
  &
  &&
  \qquad
      \lambda_i^- > 0
  \,.
\end{alignat*}
Assume towards a contradiction that 
there exists
$i \in \left\{ 1,\ldots,n \right\}$
such that
$
      \lambda_0^+ > 0
$
and
$
      \lambda_0^- > 0
$ 
and that 
$\lambda$ is optimal.
Consider
  \begin{gather}
    \tilde{\lambda}
    \ 
    :=
    \ 
    \begin{bmatrix}
      \rho
      ,
      \ 
      \lambda_0^\pm,
      \ 
      \lambda_1^+,
      \ldots,
      \ 
      \lambda_i^+
      \!
      -
      (
      \lambda_i^+
      \!
      \land
      \lambda_i^-
      ),
      \ldots
      \lambda_n^+,
      \ 
      \lambda_1^-,
      \ldots,
      \lambda_i^-
      \!
      -
      (
      \lambda_i^+
      \!
      \land
      \lambda_i^-
      ),
      \ldots,
      \lambda_n^-
    \end{bmatrix}
    ^\top
    \,.
  \end{gather}
  Since 
  $
      \lambda_i^\pm
      -
      (
      \lambda_i^+
      \!
      \land
      \lambda_i^-
      )
      \ge 
      0
  $,
  the perturbed vector $\tilde{\lambda}$ is in the domain of the 
  optimization problem.
  But 
  \begin{align}
  G(\tilde{\lambda})
  -
  G(\lambda)
  \ 
  =
  \ 
  2
  n
  \cdot
  \delta_i
  \cdot
      (
      \lambda_i^+
      \!
      \land
      \lambda_i^-
      )
  \ 
  >
  \ 
  0
  \,,
  \end{align}
  which contradicts the optimality of $\lambda$.
  Likewise we can show
\begin{alignat*}{2}
  \text{either}
  &
  &&
  \qquad
      \lambda_i^+ > 0
  \\
  \text{or}
  &
  &&
  \qquad
      \lambda_i^- > 0
  \,.
\end{alignat*}
But then 
$
\lambda^\pm_i
\ge 0
$
collapses to
$
\lambda_i\in \R
$ 
for 
$i\in \left\{ 0,\ldots,n \right\}$, that is,
$ \lambda_i=\lambda_i^+\!-\lambda_i^- $.
Note that
$ |\lambda_i|=\lambda_i^+\!+\lambda_i^- $.
Likewise we can see, that 
$
\lambda_0
=
\lambda_0^+
-
\lambda_0^-
\in \R
$
removes the constraint on $\lambda_0^\pm$.
Let us take this into account for $G$. We get
  \begin{align*}
    G
    (
    \rho,
    \lambda_0,
    \lambda
    )
    =
  &-
  \sum_{i=1} 
  ^n
  T_i
  \cdot
  f^*
  \left( 
\rho_i
\ 
+
\ 
\lambda_0
\ 
+
\ 
\inner
{B(X_i)}
{\lambda}
  \right)
  \\
  &+
  \ 
  n
  \cdot
\lambda_0
  \ 
+
  \ 
\inner
{B(X_i)}
{\lambda}
  \ 
-
  \ 
  n
  \cdot
\inner
{\delta}
{|\lambda|}
\,.
  \end{align*}
Next we show, that $\rho=0$.
Suppose there exists 
$
i\in \left\{ 1,\ldots, n \right\}
$
such that 
$
\rho_i>0
$
and
$
  T_i
  \cdot
  (f^{'})^{-1}
  \left( 
\rho_i
\ 
+
\ 
\lambda_0
\ 
+
\ 
\inner
{B(X_i)}
{\lambda}
  \right)
  <
  0
$.
It follows
\begin{gather}
  G(0,\lambda_0,\lambda)
  -
  G(\rho_i,\lambda_0,\lambda)
  \ge
  T_i
  \cdot
  (f^{'})^{-1}
  \left( 
\rho_i
\ 
+
\ 
\lambda_0
\ 
+
\ 
\inner
{B(X_i)}
{\lambda}
  \right)
(-\rho_i)
>0,
\end{gather}
which contradicts the optimality of $\lambda$.
Suppose
$
  T_i
  \cdot
  (f^{'})^{-1}
  \left( 
\rho_i
\ 
+
\ 
\lambda_0
\ 
+
\ 
\inner
{B(X_i)}
{\lambda}
  \right)
  >
  0
$.
Then the claim yields to a perturbation argument as in ts.
Thus
To eliminate the constraints for $\rho$ 
we use a similar argument as in the complementary slackness
section of the ts chapter.
Thus we have complementary slackness of 
$\rho_i$ and
$
  T_i
  \cdot
  (f^{'})^{-1}
  \left( 
\rho_i
\ 
+
\ 
\lambda_0
\ 
+
\ 
\inner
{B(X_i)}
{\lambda}
  \right)
$.
But then
every
optimal solution $\lambda$ remains optimal by taking $\rho=0$.

Dividing the optimization problem by $n$ and reversing it, we get

\begin{gather*}
  \underset
  {\lambda_0,\ldots,\lambda_{n}\in \R}
  {
    \mathrm{minimize}
  }
  \quad
  \frac{1}{n}
\sum_{i=1} 
  ^n
  \left[ 
  T_i
  \cdot
  f^*
  \left( 
\lambda_0
\ 
+
\ 
\inner
{B(X_i)}
{\lambda}
  \right)
  -
\inner
{B(X_i)}
{\lambda}
  \right]
\ 
-
\lambda_0
  \ 
+
\inner
{\delta}
{|\lambda|}
  \,.
\end{gather*}
  \end{proof}

  \todo[color=red!40, inline]{What to do about intercept $\lambda_0$?}
%%%%%%%%%%%
%%%% λ %%%%
%%%%%%%%%%%
\subsection*{Consistency of the dual variables}
%%%%%%%%%%%
%%%% ω %%%%
%%%%%%%%%%%
\subsection*{Consistency of the weights}
%%%%%%%%%%%
%%%% E[Y] %%%%
%%%%%%%%%%%
\subsection*{Consistency of the weighted mean}
\end{proof}

  \section{Application of Matrix Concentration Inequalities}
  \subsubsection*{
  Analysis of 
  $
  \E[\max_{i \le n}\norm{A_i}^2]
  $
}
We start from the premise that the fourth moment of 
the random quantities $B_k(X_i)$ and $1/\pi_i$ is uniformly bounded 
in $k$ and $i$.
\begin{assumptions*}
  \begin{enumerate}[label={(\roman*)}]
    \item
  There exists 
  a constant 
  $
  \,
    C_{\!\scriptscriptstyle B}
  \ge
  1
  \,
  $ such that
  \begin{gather*}
  \E\left[
    B_k(X_i)^{4}
    \,
  \right]
  \ 
  \le
  \ 
    C_{\!\scriptscriptstyle B}
    \quad
    \text{
  for all $\,(k,i)\ \in\  \left\{ 1, \ldots, K \right\}\times \left\{ 1, \ldots, n \right\}$.
    }
  \end{gather*}
  \item
  There exists a constant $\,C_\pi \!\ge 1\,$ such that
  \begin{gather*}
  \E \left[ 
    \left(
      \frac{1}{\pi_{i}}
    \right)^{\!4}
    \,
  \right]
  \ 
  \le
  \ 
  C_\pi
  \quad
  \text{
  for all $\,i\ \in\  \left\{ 1, \ldots, n \right\}$.
  }
  \end{gather*}
  \end{enumerate}
\end{assumptions*}
Note, that these assumptions allow for random covariate distributions with unbounded support. The coming example ought to reinforce this observation.
\begin{remark}
  If we assume a logistic regression model for the propensity score
  it holds for some $\theta \in \R^N$ ($N$ is the number of covariates)
  \begin{gather}
    \label{rmineq_rp_1}
    \frac{1-\pi(X)}{\pi(X)}
    =
    \exp(-\theta X)
    \qquad
    \text{and}
    \qquad
  \end{gather}
  \begin{gather}
  \E \left[ 
    \left(
      \frac{1}{\pi_{i}}
    \right)^{\!4}
    \,
  \right]
    =
    \E
    [
    \exp(-4\theta X)
    ]
    =
    M_X(-4\theta)
    ,
  \end{gather}
  where $M_X$ is the momement-generating function of $X$.
   While the first quantity in \eqref{rmineq_rp_1}
   may be unbounded when $X$ has unbounded support, the latter quantity in \eqref{rmineq_rp_1} is still bounded for reasonable choices of $X$.
\end{remark}

Next, we recall the entity we want to examin.
\begin{gather*}
  A_i
  \ 
  =
  \ 
  \frac{1}{n}
  \,
  \left( 
    \,
    1
    \ 
    -
    \ 
    \frac{T_i}{\pi_i}
    \,
  \right)
  \,
  B(X_i)
  \qquad
  \text{for}
  \ 
  i\in \{1, \ldots, n\}\,.
\end{gather*}
For all
$
  i\in \{1, \ldots, n\}
$
we get the bound
\begin{gather}
  \label{1:5:1}
  \left| 
    \,
    1
    \ 
    -
    \ 
    \frac{T_i}{\pi_i}
    \,
  \right|
  \ 
  \le
  \ 
  \left( 
    \,
  1
  \ 
  \lor
  \ 
  \frac{1-\pi_i}{\pi_i}
  \,
  \right)
  \ 
  \le
  \ 
  1
  \ 
  +
  \ 
  \frac{1-\pi_i}{\pi_i}
  \ 
  =
  \ 
  \frac{1}{\pi_i}
  \,.
\end{gather}
Let
$i^*\in \left\{ 1, \ldots, n \right\}$
be the index where 
$
\norm{A_i}
$
attains its maximum.
\begin{align}
  \label{1:5:2}
  \begin{split}
  \E\left[\max_{i \le n}\norm{A_i}^2\right]
  &
  \ 
  =
  \ 
  \E\left[\norm{A_{i^*}}^2\right]
  \ 
  \le
  \ 
  \E \left[ 
    \left(
      \,
      \frac{
    \norm{B(X_{i^*})}
      }{\pi_{i^*}}
      \,
    \right)
    ^{\!2}
    \,
  \right]
  \ 
  /
  \ 
  n^2
  \\
  &
    \ 
  \le
  \ 
  \E \left[ 
    \left(
      \frac{1}{\pi_{i^*}}
    \right)^{\!4}
    \,
  \right]^{1/2}
  \ 
  \cdot
  \ 
  \E\left[
    \norm{B(X_{i^*})}^4
  \right]^{1/2}
  \,
  /
  \ 
  n^2
  \\
  &
  \ 
  \le
  \ 
  \,
  K
  \,
  /
  \,
  n^2
  \ 
  \cdot
  \ 
  \sqrt{
    \,
    C_\pi
    C_{\!\scriptscriptstyle B}
  }
  \,.
\end{split}
\end{align}
The first inequality comes from the bound \eqref{1:5:1}.
The Cauchy-Schwarz inequality provides the second inequality.
In the last step we use the assumptions made at the start of the section. 
Paying the price of an extra $n$ factor, the 
maximal inequality~\eqref{1:5:2}
yields a bound of the sum, that is,
\begin{gather}
  \sum_{i=1}^{n}
  \,
  \E\left[\norm{A_i}^2\right]
  \ 
  \le
  \ 
  \frac{K}{n}
  \,
  \sqrt{\,C_\pi 
    C_{\!\scriptscriptstyle B}
  }
\end{gather}
\begin{assumption}
  .
\end{assumption}
 
\begin{assumption}
  .
\end{assumption}
\begin{remark}
With Assumption we also get a bound on the fourth moment of 
  $
  \norm{B(X_{i})}
  $. Indeed, by the convexity of 
  $x\mapsto x^2$, the monotonicity and linearity of the expectation it holds   
  \begin{align}
    \begin{split}
  \E[
  \norm{B(X_{i})}^4
  ] 
  &=
  \E
  \left[ 
    \left( 
      \sum_{k=1}^{K}
      B_k^2(X_i)
    \right)^2
  \right]
  =
  K^2
  \E
  \left[ 
    \left( 
      \sum_{k=1}^{K}
      \frac{1}{K}
      B_k^2(X_i)
    \right)^2
  \right]
  \le
  K^2
  \E
  \left[ 
      \sum_{k=1}^{K}
      \frac{1}{K}
      B_k^4(X_i)
  \right]
  \\
  &=
  K
  \sum_{k=1}^{K}
  \E
  \left[ 
      B_k^4(X_i)
  \right]
  \le
  K^2 C_B
  \end{split}
  \end{align}
\end{remark}
\subsubsection*{Analysis of $v(\mathbf{S})$}
We use the fact that 
$
  \norm{A}_2 
  \le
  \norm{A}_F
  =
  \sqrt{
    \sum_{i,j}^{}
    a_{ij}^2
  }
$
It holds
\begin{gather}
  \sum_{i=1}^{n}
  \E[A_iA_i^\top]
  =
  \frac{1}{n^2}
  \sum_{i=1}^{n}
  \E
  \left[ 
    \left( 
      \frac{1-\pi_i}{\pi_i}
    \right)^2
    B(X_i)B(X_i)^\top
  \right]
  =
  \frac{1}{n^2}
  \left( 
    \sum_{i=1}^{n}
    \E
    \left[ 
    \left( 
      \frac{1-\pi_i}{\pi_i}
    \right)^2
    B_k(X_i)B_l(X_i)
    \right]
  \right)
  _{1\le k,l \le K}
  .
\end{gather}
Thus
\begin{align}
  \begin{split}
  &\norm{
  \sum_{i=1}^{n}
  \E[A_iA_i^\top]
  }_2^2
  \\
  &\le
  \norm{
  \sum_{i=1}^{n}
  \E[A_iA_i^\top]
  }_F^2
  =
  \frac{1}{n^4}
  \sum_{k,l=1}^{K}
  \left( 
    \sum_{i=1}^{n}
    \E
    \left[ 
    \left( 
      \frac{1-\pi_i}{\pi_i}
    \right)^2
    B_k(X_i)B_l(X_i)
    \right]
  \right)^2
  \\
  &\le
  \frac{1}{n^4}
  \sum_{k,l=1}^{K}
  \left( 
    \sum_{i=1}^{n}
    \E
    \left[ 
    \left( 
      \frac{1-\pi_i}{\pi_i}
    \right)^4
    \right]
    ^\frac{1}{2}
    \E[
    B_k(X_i)^4
    ]^\frac{1}{4}
    \E[
    B_l(X_i)^4
    ]^\frac{1}{4}
  \right)^2
  \le
  \left(
  \frac{K}{n}
  \right)^2
  C_\pi C_B
  \end{split}
\end{align}
On the other hand
\begin{align}
  \begin{split}
  \norm{
  \sum_{i=1}^{n}
  \E[A^\top_iA_i]
  }_2
  &=
  \sum_{i=1}^{n}
  \E[A^\top_iA_i]
  =
  \frac{1}{n^2}
    \sum_{i=1}^{n}
    \E
    \left[ 
    \left( 
      \frac{1-\pi_i}{\pi_i}
    \right)^2
    \norm{B(X_i)}_2^2
    \right]
    \\
    &\le
  \frac{1}{n^2}
    \sum_{i=1}^{n}
    \E
    \left[ 
    \left( 
      \frac{1-\pi_i}{\pi_i}
    \right)^4
    \right]^\frac{1}{2}
    \norm{B(X_i)}_2^4
    ]^\frac{1}{2}
    \le
    \frac{K}{n}\sqrt{
  C_\pi C_B
    }
  \end{split}
\end{align}
It follows
\begin{gather}
  v(\mathbf{S})
  \le
    \frac{K}{n}\sqrt{
  C_\pi C_B
  }
\end{gather}
Thus we can apply Theorem~\ref{rmineq_rosenthal_pinelis}
to get
\begin{gather}
  \E[\norm{\mathbf{S}}_2]
  \le
  \sqrt{
    2e 
    \frac{K}{n}\sqrt{
  C_\pi C_B
  }
  \log
  (K+1)
  }
  +
  4e
  \frac{\sqrt{K}}{n}
  \sqrt[4]{
  C_\pi C_B
  }
  \log
  (K+1)
  \le
  14
  C_\pi C_B
  \sqrt{
    \frac{K \log(K+1)}{n}
  }
\end{gather}



\chapter{Convex Analysis}
In our application we want to analyse a convex optimization problem by its dual problem.
In particular we want to obtain primal optimal solutions from dual solutions.
To accomplish the task we need technical tools from convex analysis, 
mainly conjugate calculus and some KKT related results.

Our starting point is the support function intersection rule~\cite[Theorem 4.23]{Mordukhovich2022}.
We give the details in the case of finite dimensions and refer for the rest of the proof to the book.
The conjugate sum rule is applied to give first conjugate sum and then chain rule,
which are vital to calculating convex conjugates. The proofs are omited, since the book is thorough enough. The well known Fenche-Rockafellar Duality theorem is a corollary of conjugate sum and chain rule. It gives general conditions under which dual and primal values coincide.
The material we present is very well known, so we claim no originality. 
We paraphrase the approach of \cite{Mordukhovich2022} to Duality. As an introduction, we recommend this recently published book together with the classical reference \cite{Rockafellar1970}.

We finish the chapter with ideas from \cite{Tseng1991}. 
They provide the high-level ideas to obtain for strictly convex
functions a dual relationship between optimal solutions.
We will deliver the details that are omited in the paper.
  \section{A Convex Analysis Primer}
  Excursively, we present some well known definitions and facts from
convex analysis. For details, see, e.g., \cite{Mordukhovich2022}.

A subset $C\subseteq \R^n$ is called \textbf{convex set}, 
if for all $x,y\in C$ and all $\lambda\in [0,1]$,
we have 
$
  \lambda x + (1-\lambda)y 
  \in
  C
  .
$
The Cartesian product of convex sets is convex. The intersection of a collection of convex sets is also convex.




Given (not necessary convex) sets 
$
  \Omega,
  \Omega_1,
  \Omega_2
  \subseteq 
  \R^n
$
and
$\lambda\in\R,$
define the \textbf{set addition} and \textbf{multiplication}
by a real scalar as 
$
  \Omega_1 
  +
  \Omega_2 
  :=
  \left\{ 
    x_1 + x_2 
    \colon
    x_1 \in \Omega_1
    ,
    x_2 \in \Omega_2
  \right\}
$
and
$
  \lambda \Omega
  :=
  \left\{ 
    \lambda x
    \colon
    x\in\Omega
  \right\}.
$
For convex sets the addition and multiplication by a real scalar are convex.

Throughout this section, we shall denote by 
$
  B
  :=
  \left\{ 
    x=
    [x_1, \ldots, x_n]^\top
    \in\R^n
    \colon
    (
    \sum_{i=1}^{n} 
    x_i^2
    )
    ^{1/2}
    \le 1
  \right\}
$
\todo[color=green!40,inline]{Solve editorial issue with ball.}
the \textbf{Euclidian unit ball} in $\R^n.$
This is a closed convex set. For any $a\in\R^n,$ 
the \textbf{ball with radius $\varepsilon >0$ and center $a$}
is given by
$
  \left\{ 
   a+x 
    \in\R^n
    \colon
    (
    \sum_{i=1}^{n} 
    x_i^2
    )
    ^{1/2}
    \le \varepsilon
  \right\}
  =
  a
  +
  \varepsilon B
  .
$
For any set $\Omega$ in $\R^n,$ 
the set of points $x$ whose distance from $\Omega$
does not exceed $\varepsilon$ is 
$
\Omega + \varepsilon B.
$
The \textbf{closure} $\mathrm{cl}(\Omega)$
and \textbf{interior} $\mathrm{int}(\Omega)$
of $\Omega$
can therefore be expressed by 
$
  \mathrm{cl}(\Omega)
  =
  \bigcap_{\varepsilon>0}
  \Omega + \varepsilon B
$
and
$
  \mathrm{int}(\Omega)
  =
  \left\{ 
    x\in \Omega
    \colon
    \text{there exists $\varepsilon>0$ such that }
    x+\varepsilon B
    \subseteq
    \Omega
  \right\}
  .
$

  A set 
  $A\subseteq \R^n$
  is called \textbf{affine set}, if
  $
    \alpha x + (1-\alpha)y \in A
    \quad
    \text{for all}
    \ 
    x,y \in A
    \ 
    \text{and}
    \ 
    \alpha \in \R.
  $
  The \textbf{affine hull} 
  $\mathrm{aff}(\Omega)$
  of a set 
  $\Omega\subseteq \R^n$
  is the smallest affine set that includes $\Omega.$
A mapping 
$
  A: \R^n \to \R^m
$
is called \textbf{affine mapping} if there exist a linear mapping
$
  L: \R^n \to \R^m
$
and a vector $b\in \R^m$
such that
$
  A(x)
  =
  L(x)
  +
  b
  \ 
  \text{for all}\ 
  x \in \R^n
  .
$
The image and inverse image/preimage of convex sets under affine mappings are also convex.





Because the notion of interior is not precise enough for our purposes
we define the relative interior which is the interior relative to the affine hull.
This concept is motivated by the fact that a line segment embedded in $\R^2$
does have a natural interior in $\R$ which is not a true interior in $\R^2.$
The relative interior of $C$ is defined as the interior which results when
$C$ is regarded as a subset of its affine hull.
\begin{definition}
  Let 
  $\Omega\subseteq \R^n.$
  We define the \textbf{relative interior} of $\Omega$ by
  \begin{gather}
    \mathrm{ri}(\Omega)
    :=
    \left\{ 
      x \in \Omega 
      \colon
      \text{there exists}\ 
      \varepsilon > 0\ 
      \text{such that}\ 
      (
        x+\varepsilon B
      )
      \cap
      \mathrm{aff}(\Omega)
      \subset
      \Omega
    \right\}.
  \end{gather}
\end{definition}
Next we collect some useful properties of relative interiors.
\begin{proposition}
  Let $C$ be a non-empty convex set in $\R^n.$ Then we get the representation
\begin{enumerate}[label={(\roman*)}]
  \item
    $
    \mathrm{ri}(C)
    =
    \left\{ 
      z \in C
      \colon
      \text{for all}\ 
      x \in C \ 
      \text{there exists}\ 
      t > 0 \ 
      \text{such that}\ 
      z + t (z-x)
      \in C
    \right\}.
    $
  \item
    $
      \mathrm{ri}(C)
      \neq
      \emptyset
      \ 
      \text{if}\ 
      C\neq \emptyset
      .
    $
  \item
    $
      \mathrm{cl}(C)
      \ 
      \text{and}\ 
      \mathrm{ri}(C)
    $
    are convex sets.
  \item
    $
      \mathrm{cl}(\mathrm{ri}(C))
      =
      \mathrm{cl}(C)
      \ 
      \text{and}
      \  
      \mathrm{ri}(\mathrm{cl}(C))
      =
      \mathrm{ri}(C)
      .
    $
  \item
    Suppose
    $
      \bigcap_{i\in I} C_i
      \neq
      \emptyset
    $
    for a finite index set $I.$
    Then
    $
      \mathrm{ri}
      \left( 
        \bigcap_{i\in I} C_i
      \right)
      =
      \bigcap_{i\in I}  
      \mathrm{ri}(C_i)
      .
    $
    \item
      Let 
      $
        L:\R^n \to \R^m
      $
      be a linear mapping. Then
      $
        \mathrm{ri}(L(C))
        =
        L(\mathrm{ri}(C))
        .
      $
      If additionally it holds
      $
        L^{-1}(\mathrm{ri}(C))
        \neq
        \emptyset
      $
      we have
      $
      \mathrm{ri}(L^{-1}(C))
        =
        L^{-1}(\mathrm{ri}(C))
        .
      $
      \item
        $
          \mathrm{ri}(C_1\times C_2)
          = 
          \mathrm{ri}(C_1)
          \times
          \mathrm{ri}(C_2).
        $
      \item
        $
          \mathrm{ri}(C_1)
          \cap
          \mathrm{ri}(C_2)
          =
          \emptyset
        $
        if and only if
        $
          0 \notin
          \mathrm{ri}
          (C_1 - C_2)
          .
        $
\end{enumerate}

\end{proposition}

\todo[color=green!40,inline]{Order results to give pretty proof.}

\begin{proof}
\begin{enumerate}[label={(\roman*)}]
  \item
    \cite[Theorem~6.4]{Rockafellar1970}
  \item
    \cite[Theorem~6.2]{Rockafellar1970}
  \item
    \cite[Theorem~6.2]{Rockafellar1970}
  \item
    \cite[Theorem~6.3]{Rockafellar1970}
  \item
    \cite[Theorem~6.5]{Rockafellar1970}
  \item
    \cite[Theorem~6.6-6.7]{Rockafellar1970}
  \item
Let
  $
  (z_1, z_2)
  \in 
  \mathrm{ri}(C_1\times C_2).
  $
  Then for all 
  $
  (x_1, x_2)
  \in 
  C_1\times C_2
  $
  there exists
  $t>0$
  such that
  \begin{gather}
      z_i + t (z_i-x_i)
      \in C_i
      \qquad
      \text{for}\ 
      i\in \left\{ 1,2 \right\}.
  \end{gather}
  This proves $\subseteq.$
  Suppose 
  $
    z_1 
    \in
    \mathrm{ri}(C_1)
  $
  and
  $
    z_2 
    \in
    \mathrm{ri}(C_2).
  $
  Let
  $
    (x_1,x_2)\in C_1\times C_2
  $
  with
  If $t_1=t_2$ everything is clear.
  W.l.o.g.
  assume 
  $t_1<t_2$. Define $\theta:=\frac{t_1}{t_2}\in (0,1).$
  By the convexity of $C_2$ it follows
  \begin{gather}
    z_2 + t_1 (z_2 - x_2)
    =
    \theta
    (
    z_2 + t_2 (z_2 - x_2)
    )
    +
    (1-\theta)z_2
    \in C_2.
  \end{gather}
  Thus
  $
  (z_1,z_2)\in\mathrm{ri}(C_1\times C_2). 
  $
  This proves $\supseteq$ and equality.
  \item
    \cite[Theorem~2.92]{Mordukhovich2022}
\end{enumerate}
\end{proof}
 
We procede with convex separation results which are vital to the subsequent developments.

\begin{definition}
  Let 
  $C_1$ and $C_2$
  be two non-empty convex sets in $\R^n$. 
  A hyperplane $H$ is said to \textbf{separate}
  $C_1$ and $C_2$
  if $C_1$ is contained in one of the closed half-spaces associated with
  $H$ and $C_2$ lies in the opposite closed half-space. It is said to separate 
  $C_1$ and $C_2$
  \textbf{properly} if 
  $C_1$ and $C_2$
  are not \textit{both} actually contained in $H$ itselef.
\end{definition}
\begin{theorem}
  Let $C_1$ and $C_2$ be two non-empty convex sets in $\R^n$. 
  There exists a hyperplane separating
  $C_1$ and $C_2$
  properly 
  if and only if
  there exists a vector $b\in \R^n$ such that
  \begin{gather}
    \sup_{x\in C_2} \inner{x}{b}
    \le
    \inf_{x\in C_1} \inner{x}{b}
    \quad 
    \text{and}
    \quad 
    \inf_{x\in C_2} \inner{x}{b}
    <
    \sup_{x\in C_1} \inner{x}{b}
    .
  \end{gather}
\end{theorem}
\begin{proof}
  \cite[Theorem~11.1]{Rockafellar1970}
\end{proof}
\begin{ftheorem}
  \emph{(Convex separation in finite dimension)}
  Let $C_1$ and $C_2$ be two non-empty convex sets in $\R^n$. 
  Then $C_1$ and $C_2$ can be properly separated if and only if 
  $\mathrm{ri}(C_1)\cap\mathrm{ri}(C_2)=\emptyset.$
\end{ftheorem}
\begin{proof}
  \cite[Theorem~11.3]{Rockafellar1970}
\end{proof}




\begin{definition}
  Given a nonempty subset 
  $\Omega \subseteq \R^n$
  the \textbf{support function} 
  $
  \sigma_\Omega : \R^n \to \overline{\R}
  $
  of $\Omega$
  is defined by
  \begin{gather}
    \sigma_\Omega
    (x^*)
    :=
    \sup_{x \in \Omega}
    \ 
    \inner{x^*}{x}
    \qquad
    \text{for}\ 
    x^* \in \R^n
    .
  \end{gather}
\end{definition}


\begin{definition}
  Given functions
  $
    f_i:
    \R^n \to (-\infty, \infty]
  $
  for $ i = 1, \ldots, n $
  the \textbf{infimal convolution} of these functions is defined as
  \begin{gather}
    (f_1 \square \ldots \square f_m)(x)
    :=
    \inf_{
    \begin{smallmatrix}
      x_i \in \R^n \\
      \sum_{i = 1}^{m} 
        x_i
      =
      x
    \end{smallmatrix}
    }
    \sum_{i = 1}^{m}
      f_i(x_i)
  \end{gather}
\end{definition}
 
The next result establishes a connection between the support function of the intersection of two convex sets and the infimal convolution of the support functions of the sets taken by themselfes.
The proof translates the geometric concept of convex separation to the world of convex functions.

\begin{ftheorem}
  Let $C_1$ and $C_2$ be two non-empty convex sets in $\R^n$ with
  $\mathrm{ri}(C_1)\cap\mathrm{ri}(C_2)\neq\emptyset.$
  Then the support function of the intersection 
  $
    C_1 \cap C_2
  $
  is represented as
  \begin{gather}
    (\sigma_{
    C_1 \cap C_2
    })
    (x^*)
    =
    (\sigma_{C_1}\square \sigma_{C_2})
    (x^*)
    \qquad
    \text{for all}\ 
    x^* \in \R^n.
  \end{gather}
  Furthermore, for any
  $
  x^*\in \mathrm{dom}
    (\sigma_{
    C_1 \cap C_2
    })
  $
  there exist dual elements 
  $
    x_1^*
    ,
    x_2^*
    \in \R^n
  $ 
  such that 
  $
    x^*
    =
    x_1^*
    +
    x_2^*.
  $
  and
  \begin{gather}
    (\sigma_{
    C_1 \cap C_2
    })
    (x^*)
    =
    \sigma_{C_1}(x_1^*)
    +
    \sigma_{C_2}(x_2^*).
  \end{gather}
\end{ftheorem}
\begin{proof}
  \emph{\cite[Theorem~4.23]{Mordukhovich2022}}
  We want to use results on convex separation. To make the geometric property of convex separation fruitful
  to our purpose we consider two special sets. We will verify that these sets meet the requirements for convex separation, i.e., that they are convex and the intersection of their relative interiors is empty. The rest of the proof is as in the cited reference at the beginning of the proof.
  To this end, consider 
  the sets
  \begin{gather}
    \Theta_1
    \ :=\ 
    C_1 \times [\,0,\infty)
    \quad
    \text{and}
    \quad
    \Theta_2
    \ :=\ 
    \left\{ 
      (x,\lambda)\in \R^n
      \ 
      \colon
      \ 
      x \in C_2
      \ 
      \text{and}
      \ 
      \lambda
      \,
      \le
      \,
      \inner{x^*\!}{x} - \alpha
    \right\}\ .
  \end{gather}
  \todo[color=green!40,inline]{Simplify proof with properties of relative interiors.}
  Clearly, $\Theta_1$ is convex by the convexity of $C_1$. To see that $\Theta_2$ is convex consider the affine function
  $
    \varphi:
    \R^{n}\!\times \R \to \R
    ,
    \ 
    (x,\lambda)
    \mapsto
    \alpha - \inner{x^*\!}{x} - \lambda
  $ .
  From the definitions of $\varphi$ and $\Theta_2$ we get the identity
  \begin{gather*}
 \Theta_2
    \ 
    =
    \ 
    (
      C_2\!\times\R
    )
    \ 
    \cap
    \ 
    \varphi^{-1}
    (-\infty,0\,]
    \,
    .
  \end{gather*}
  Thus, by the convexity of the sets
  $C_2$
  and
  $
    \varphi^{-1}(-\infty,0]
  $
  it follows the convexity of $\Theta_2$.
  Next we show that the relative interiors of 
  $\Theta_1$ and $\Theta_2$
  do not intersect, i.e.,
  ~$
  \mathrm{ri}\,\Theta_1\cap\mathrm{ri}\,\Theta_2=\emptyset.
  $
  First note that
  \begin{gather}
    \mathrm{ri}(\Theta_1)
    =
    \mathrm{ri}(C_1)
    \times
    \mathrm{ri}([0,\infty))
    \subseteq
    \mathrm{ri}(C_1)
    \times
    (0,\infty).
  \end{gather}
  Indeed, if 
  $
    0\in
    \mathrm{ri}([0,\infty))
  $
  then there exists $t>0$ such that $-tx\ge 0$ for some $x>0.$ A contradiction.
  Furthermore
\begin{gather}
  \mathrm{ri}(\Theta_2)
  \subseteq
    \left\{ 
      (x,\lambda)\in \R^n
      \colon
      x \in \mathrm{ri}(C_2) 
      \ 
      \text{and}
      \ 
      \lambda
      <
      \inner{x^*}{x} - \alpha
    \right\}.
\end{gather}
To see this, assume there is 
$
(x,\lambda)
\in \mathrm{ri}(\Theta_2)
$
with
$
      \lambda
      =
      \inner{x^*}{x} - \alpha
      .
$
Then for some 
$
  (y,\mu)
  \in \Theta_2
$
with
$
      \mu    
      <
      \inner{x^*}{y} - \alpha
$
there exists $t>0$ such that 
$
  (x,\lambda)
  +
  t
  (
  (x,\lambda)
  -
  (y,\mu)
  )
  \in \Theta_2.
$
It follows
\begin{align}
  0
  \le
  (1+t)(\inner{x^*}{x}-\alpha -\lambda)
  +
  t
  (
    \mu 
    -\inner{x^*}{y}
    +
    \alpha
  )
  <0
,
\end{align}
a contradiction.
The first inequality is due to 
$
  (x,\lambda)
  +
  t
  (
  (x,\lambda)
  -
  (y,\mu)
  )
  \in \Theta_2
$
and the second inequality due to
$
      \mu    
      <
      \inner{x^*}{y} - \alpha
$
and
$
      \lambda
      =
      \inner{x^*}{x} - \alpha
      .
$
But then 
$
\mathrm{ri}(\Theta_1)\cap\mathrm{ri}(\Theta_2)=\emptyset
.
$
Indeed, suppose that there exists 
$
  (x,\lambda)
  \in
  \mathrm{ri}(\Theta_1)\cap\mathrm{ri}(\Theta_2)
  .
$
Then it holds
$
  \inner{x^*}{x}
  -
  \alpha
  \le
  0
$
and $\lambda>0$
since 
$
 x
 \in
  \mathrm{ri}(C_1)\cap\mathrm{ri}(C_2)
  \subseteq
  C_1\cap C_2.
$
On the other hand
\begin{gather}
  0
  <
  \lambda
  <
  \inner{x^*}{x}
  -
  \alpha
  \le
  0
  ,
\end{gather}
a contradiction.

Thus, convex separation is applicable. Confer \cite[Theorem 4.23]{Mordukhovich2022}
about the rest of the proof.
\end{proof}

\begin{takeaways}
  The support function intersection rule connects the geometric 
  property of convex separation to an identity of support functions
  This result is central to the analysis of convex conjugates.
\end{takeaways}

  \section{Conjugate Calculus and Fenchel-Rockafellar Theorem}
  When studying different primal problems such as \eqref{primal_weighting_binary} we often turn to the dual instead.
Therefore we need some reliable tools.
Begin able to compute specific convex conjugates is one tool required.


\begin{definition}
  \label{ def_convex_conjugate }
  \emph{(Convex conjugate)}
  Given a function
  $
    f:
    \R^n \to \overline{\R}
  $
  ,
  the 
  \textbf{convex conjugate}
  $
    f^*:
    \R^n \to \overline{\R}
  $
  of $f$ is defined as
  \begin{gather}
    f^*(x^*)
    :=
    \sup_{ x \in \R^n }
    (x^*)^T x - f(x)
  \end{gather}
\end{definition}

Note that $f$ in Definition~\ref{ def_convex_conjugate }
does not have to be convex. On the other hand, the convex conjugate is always convex:

\begin{proposition}
  Let  
  $
    f:
    \R^n \to ( - \infty, \infty ]
  $
  be a proper function. 
  Then its convex conjugate
  $
    f^*:
    \R^n \to ( - \infty, \infty ]
  $
  is convex.
\end{proposition}

\begin{definition}
  Given a nonempty subset 
  $\Omega \subseteq \R^n$
  the \textbf{support function} 
  $
  \sigma_\Omega : \R^n \to \overline{\R}
  $
  of $\Omega$
  is defined by
  \begin{gather}
    \sigma_\Omega
    (x^*)
    :=
    \sup_{x \in \Omega}
    \ 
    \inner{x^*}{x}
    \qquad
    \text{for}\ 
    x^* \in \R^n
    .
  \end{gather}
\end{definition}

\begin{lemma}
  For any proper function
  $
    f:\R^n\to\overline{\R}
  $
  we have
  \begin{gather}
    f^*(x^*) 
    =
    \sigma_{\mathrm{epi}(f)}
    (x^*,-1)
    \qquad
    \text{for}
    \ 
    x^* \in \R^n.
  \end{gather}
\end{lemma}
\begin{proof}
  Let $x^*\in\R^n$
  and
  $
    (x,\lambda)\in \mathrm{epi}(f).
  $
  Then
  $
    x \in \mathrm{dom}(f)
  $
  and
  $
    f(x)\le \lambda.
  $
  Thus
  \begin{gather}
    \inner{x^*}{x} - f(x)
    \ge
    \inner{x^*}{x} - \lambda
    \qquad
    \text{for all}\ 
    (x,\lambda)\in \mathrm{epi}(f).
  \end{gather}
  On the other hand 
  $
    (x,f(x))\in \mathrm{epi}(f)
  $
  for all
  $
    x \in \mathrm{dom}(f).
  $
  It follows
  \begin{gather}
    \inner{x^*}{x} - f(x)
    \le
    \sup_{(x,\lambda)\in\mathrm{epi}(f)}
    \inner{x^*}{x} - \lambda
    \qquad
    \text{for all}\ 
    x \in \mathrm{dom}(f).
  \end{gather}
  Taking the supremum in the last two displays yields
  \begin{align}
    f^*(x^*)
    =
    \sup_{x\in\mathrm{dom}(f)}
    \inner{x^*}{x} - f(x)
    &=
    \sup_{(x,\lambda)\in\mathrm{epi}(f)}
    \inner{x^*}{x} - \lambda
    \\
    &=
    \sup_{(x,\lambda)\in\mathrm{epi}(f)}
    \inner{(x^*,-1)}{(x,\lambda)} 
    =
    \sigma_{\mathrm{epi}(f)}
    (x^*,-1).
  \end{align}
\end{proof}
% conjugate chain rule %
 %%%%%%%%%%%%%%%%%%%%%%
\begin{proposition}

\end{proposition}
\begin{theorem}
  \emph{(Conjugate Chain Rule)}
  \label{cvxa_conjugate_chain_rule}
  Let 
  $
    A:
      \R^m \to \R^n
  $
  be a linear map (matrix)
  and
  $
    g:
      \R^n \to (-\infty, \infty]
  $
  a proper convex function. If
  $
    \text{Im}(A) \cap \text{ri}(\text{dom}(g))
    \neq
    \emptyset
  $
  it follows
  \begin{gather}
    ( g \circ A )^* ( x^* )
    =
    \inf_
          { y^* \in ( A^* )^{ -1 } ( x^* )}
                                          g^*( y^* )
                                          .
  \end{gather}
  Furthermore, 
    for any 
      $
        x^* \in \text{dom}( g \circ A)^*
      $
        there exists
          $
            y^* \in ( A^* )^{ -1 } ( x^* )
          $
            such that
              $
                ( g \circ A)^* ( x^* )
                =
                g^*( y^* )
              $.
\end{theorem}

% conjugate sum rule %
 %%%%%%%%%%%%%%%%%%%%

\begin{definition}
  \emph{(Infimal convolution)}
  Given functions
  $
    f_i:
    \R^n \to (-\infty, \infty]
  $
  for $ i = 1, \ldots, n $
  the \textbf{infimal convolution} of these functions as defined as
  \begin{gather}
    (f_1 \square \ldots \square f_m)(x)
    :=
    \inf_{
    \begin{smallmatrix}
      x_i \in \R^n \\
      \sum_{i = 1}^{m} 
        x_i
      =
      x
    \end{smallmatrix}
    }
    \sum_{i = 1}^{m}
      f_i(x_i)
  \end{gather}
\end{definition}


\begin{theorem}
  Let
  $
    f,g:
    \R^n \to (-\infty, \infty]
  $
  be proper convex functions 
  and
  $
  \text{ri}\left( \text{dom}(f) \right)
  \cap
  \text{ri}\left( \text{dom}(g) \right)
  \neq 
  \emptyset
  .
  $
  Then we have the conjugate sum rule
  \begin{gather}
    ( f + g )^*(x^*)
    =
    ( f^* \square g^*)(x^*)
  \end{gather}
  for all $x^* \in \R^n$.
  Moreover, the infimum in 
  $
    ( f^* \square g^*)(x^*)
  $
  is attained, i.e., for any
  $
    x^* \in \text{dom}(f+g)^*
  $
  there exists vectors $x_1^*, x_2^*$
  for which
  \begin{gather}
    (f+g)^*(x^*)
    =
    f^*(x_1^*)
    +
    g^*(x_2^*),
    \quad
    x^* = x_1^* + x_2^*.
  \end{gather}
\end{theorem}
\begin{proof}
  Let $x^*\in\R^n$ and fix $x_1^*,x_2^*\in\R^n$ such that
  $x^*=x^*_1+x^*_2$.
  We get
  \begin{align*}
    f^*(x^*_1)+g^*(x^*_2)
    &=
    \sup_{x\in\R^n}
    \inner{x^*_1}{x}-f(x)
    +
    \sup_{x\in\R^n}
    \inner{x^*_2}{x}-g(x)
    \\
    &\ge
    \sup_{x\in\R^n}
    \inner{x^*_1}{x}-f(x)
    +
    \inner{x^*_2}{x}-g(x)
    =
    \sup_{x\in\R^n}
    \inner{x^*_1+x^*_2}{x}-(f(x)+g(x))
    \\
    &=
    \sup_{x\in\R^n}
    \inner{x^*}{x}-(f+g)(x)
    =(f+g)^*(x^*)
  \end{align*}
  Taking the infimum over $x_1^*,x_2^*\in\R^n$ in the above display gives 
  $
  (f^*\square g^*)(x^*)
  \ge
  (f+g)^*(x^*).
  $
  Let us prove now $\le$ under the condition
  $
  \text{ri}\left( \text{dom}(f) \right)
  \cap
  \text{ri}\left( \text{dom}(g) \right)
  \neq 
  \emptyset
  .
  $
  The only case we need to consider is
  $
    (f+g)^*(x^*)<\infty.
  $
  Define two convex sets by
  \begin{align}
    \Omega_1
    &:=
    \left\{ 
      (x,\alpha,\beta)\in\R^{n+2}
      \colon
      \alpha\ge f(x)
    \right\}
    =
    \mathrm{epi}(f)\times \R,
    \\
    \Omega_2
    &:=
    \left\{ 
      (x,\alpha,\beta)\in\R^{n+2}
      \colon
      \beta\ge g(x)
    \right\}.
  \end{align}
  Similar to Lemma we get the representation
  \begin{gather}
    (f+g)^*(x^*)
    =
    \sigma_{\Omega_1\cap\Omega_2}
    (x^*,-1,-1).
  \end{gather}
  Indeed, the only thing we need to verify is
  $
    \mathrm{dom}(f)\cap\mathrm{dom}(g)
    =
    \mathrm{dom}(f+g).
  $
  The inclusion $\subseteq$ is clear.
  Assume towards a contradiction that
  $
    (f+g)(x)<\infty
  $
  and
  $
    f(x)=\infty.
  $
  Since $g(x)>-\infty$ it holds
  \begin{gather}
    \infty
    =
    \infty+g(x)
    =f(x)+g(x)
    =(f+g)(x)
    <
    \infty.
  \end{gather}
  This is a contradiction. The same holds for $f$ and $g$ reversed. It follows the inclusion $\supseteq$ and equality.
  By the support function intersection rule there exist triples
  \begin{gather}
    (x^*_1,-\alpha_1,-\beta_1),
    (x^*_2,-\alpha_2,-\beta_2)
    \in \R^{n+2}
    \quad
    \text{such that}
    \quad
    (x^*,-1,-1)
    =
    (x^*_1+x^*_2,-(\alpha_1+\alpha_2),-(\beta_1+\beta_2))
  \end{gather}
  and
  \begin{gather}
    (f+g)^*(x^*)
    =
    \sigma_{\Omega_1\cap\Omega_2}
    (x^*,-1,-1)
    =
    \sigma_{\Omega_1}
    (x^*_1,-\alpha_1,-\beta_1)
    +
    \sigma_{\Omega_2}
    (x^*_2,-\alpha_2,-\beta_2).
  \end{gather}
  Next we show
  $\beta_1=\alpha_2=0.$
  Suppose towards a contradiction that 
  $\beta_1\neq 0.$ 
  We fix 
  $(\overline{x},\overline{\alpha})\in\mathrm{epi}(f).$
  Then
  \begin{gather}
    \sigma_{\Omega_1}
    (x^*_1,-\alpha_1,-\beta_1)
    =
    \sup_{(x,\alpha,\beta)\in \mathrm{epi}(f)\times \R}
    \inner{x^*}{x}-\alpha \alpha_1 -\beta \beta_1
    \ge
    \sup_{\beta\in \R}
    \inner{x^*}{\overline{x}}-\overline{\alpha} \alpha_1 -\beta \beta_1
    =\infty.
  \end{gather}
  This contradicts
  $
    (f+g)^*(x^*)<\infty.
  $
  In a similar fashion we can derive a contradiction for $\alpha_2\neq0.$
  Employing Lemma and taking into account the structures of the sets 
  $\Omega_1$ and $\Omega_2$ this implies
  \begin{align}
    (f+g)^*(x^*)
    &=
    \sigma_{\Omega_1\cap\Omega_2}
    (x^*,-1,-1)
    =
    \sigma_{\Omega_1}
    (x^*_1,-1,0)
    +
    \sigma_{\Omega_2}
    (x^*_2,0,-1)
    \\
    &=
    \sigma_{\mathrm{epi}(f)}(x^*_1,-1)
    +
    \sigma_{\mathrm{epi}(g)}(x^*_2,-1)
    =
    f^*(x^*_1)
    +
    g^*(x^*_2)
    \ge
    (f^*\square g^*)(x^*).
  \end{align}
  This finishes the proof.
\end{proof}

%We begin by defining convex sets
%

\begin{definition}
  A subset $\Omega\subseteq \R^n$ is called CONVEX if we have $\lambda x+(1-\lambda)y\in \Omega$ for all $x,y\in \Omega$ and $\lambda\in (0,1)$. 
\end{definition}

Clearly, the line segment 
$[a,b]:=\left\{ \lambda a+(1-\lambda)b\,\mid \, \lambda\in [0,1] \right\}$ is contained in $\Omega$ for all $a,b\in \Omega$ if and only if $\Omega$ is a convex set.
%

Next we define convex functions. 
%

The concept of convex functions is closely related to convex sets.
%  
 
The line segment between two points on the graph of a convex function lies on or above and does not intersect the graph.
%

In other words: The area above the graph of a convex function $f$ is a convex set, i.e. the \textit{epigraph}
$\text{epi}(f):=\left\{ (x,\alpha)\in \R^n\times\R\,\mid\, f(x)\le \alpha\right\}$ is a convex set in $\R^{n+1}$.
%

Often an equivalent characterisation of convex functions is more useful.
%

\begin{theorem}
  The convexity of a function $f:\R^n\to \overline{\R}$ on $\R^n$ is equivalent to the following statement:

  For all $x,y\in \R^n$ and $\lambda\in(0,1)$ we have 
    \begin{align}
      f(\lambda x + (1-\lambda)y)\le \lambda f(x)+(1-\lambda)f(y).
    \end{align}
\end{theorem}

  \section{Tseng Bertsekas}
  We present the relevant parts of the paper \cite{Bertsekas2003}.

Consider the following optimization problem

  \begin{gather*}
    \underset{x\in \R^m}{\mathrm{minimize}}
    \qquad
    f(x)
  \end{gather*}
subject to the constraints
\begin{gather}
  \mathbf{A}x \ge b
  ,
\end{gather}
Where 
$
  f:
  \R^m
  \to 
  \overline{\R}
  ,
  \ 
  \mathbf{A} \ 
  \text{is a given}\ 
  n\times m
  \ 
  \text{matrix,}
$
 and $b$ is a vector in $\R^n.$
\begin{assumption}
  \begin{enumerate}[label={(\roman*)}]
    Assume that the map 
    $
      f: \R^m \to \overline{\R}
    $
    has the following properties.
    \item
      $
        f 
        \ 
        \text{is strictly convex.}
      $
    \item
      $
        f
        \ 
        \text{is lower-semicontinuous and continuous}
        \ 
        \mathrm{dom}(f)
        .
      $
    \item
      $
        \text{The convex conjugate}
        \ 
        f^*
        \ 
        \text{of}\ 
        f
        \ 
        \text{is finite}
        .
      $
  \end{enumerate}
\end{assumption}


\begin{proposition}
  \emph{(Danskin's Theorem \cite[page 649]{Bertsekas2003})}
  text
\end{proposition}


\begin{definition}
  \emph{\cite[§28]{Rockafellar1970}}
  By an \textbf{ordinary convex program}
  $(P)$
  we mean an optimization problem of the following form
  \begin{gather*}
    \underset{x\in C}{\mathrm{minimize}}
    \qquad
    f_0(x)
  \end{gather*}
subject to the constraints
\begin{gather}
  f_1(x)\le 0,
  \ldots,
  f_r(x)\le 0,
  \qquad
  f_{r+1}(x)= 0,
  \ldots,
  f_m(x)= 0,
\end{gather}
where $C\subseteq \R^n$
is a non-empty convex set,
$f_i$ is a finite convex function on $C$ for $i\in \left\{ 1,\ldots,r \right\}$
 and 
 $f_i$ is an affine function on $C$ for $i\in \left\{ r+1, \ldots, m \right\}.$
\end{definition}

\begin{definition}
  We define 
  $
    [\lambda_1, \ldots, \lambda_m]\in \R^m
  $
  to be a \textbf{Karush-Kuhn-Tucker (KKT) vector}
  for $(P)$, if
  \begin{enumerate}[label={(\roman*)}]
    \item
      $
        \lambda_i \ge 0
        \ 
        \text{for all}
        \ 
        i\in \left\{ 1,\ldots,r \right\}.
      $
    \item
      The infimum of the proper convex function 
      $
        f_0
        +
        \sum_{i=1}^{m}
        \lambda_1 f_i
      $
      is finite and equal to the optimal value in $(P).$
  \end{enumerate}
\end{definition}

\begin{ftheorem}
  \emph{(Karush-Kuhn-Tucker conditions)}
  Let $(P)$
  be an ordinary convex program,
  $
  \overline{\alpha}
  \in \R^m
  $,
   and 
   $
   \overline{z}
   \in \R^n.
   $
   Then 
   $
  \overline{\alpha}
   $
   is a KKT vector for $(P)$
   and 
   $
   \overline{z}
   $
   is an optimal solution to $(P)$
   if and only if 
   $
   \overline{z}
   $
   and 
   the components $\alpha_i$ of
   $
  \overline{\alpha}
   $
   satisfy 
   the following conditions.

  \begin{enumerate}[label={(\roman*)}]
    \item
      $
        \alpha_i \ge 0,
        \ 
        f_i(
   \overline{z}
        )
        \le 0,
        \ 
        \text{and}
        \ 
        \alpha_i 
        f_i(
   \overline{z}
        )
        =0
        \ 
        \text{for all}
        \ 
        i\in \left\{ 1, \ldots, r \right\}
        .
      $
      \item
        $
        f_i(
   \overline{z}
        )
        =0
        \ 
        \text{for}
        \ 
        i\in \left\{ r+1, \ldots, m \right\}
        .
        $
      \item
        $
         0
         _n
         \in 
         [
          \partial
        f_0(
   \overline{z}
        )
        +
        \sum_{\alpha_i\neq 0}
        \alpha_i 
        \partial
        f_i(
   \overline{z}
        )
         ]
.
        $
  \end{enumerate}
\end{ftheorem}
\begin{proof}
  \cite[Theorem~28.3]{Rockafellar1970}
\end{proof}



\chapter{Random Matrix Inequalities}
  In our application we want to bound moments of vector-valued random variables.
  For this we choose the theory of random matrix inequalities
  which lately received a lot of attention.
  In particular an approach via the method of exchangable pairs \cite{Mackey2014}
  has been fruitful in simplifying the proofs of long standing results such as the matrix Khintchin inequality.
  The paper offers a comprehensive introduction to this method.

  We will cite the matrix Khintchin inequality and 
  inequalities for moments of matrices that follow from it\cite{Chen2012}. 
  As a novelty, we will apply intrinsic dimension results and Hermition Dilitation from \cite{Tropp2015}  
  to matrix moments inequalities. Even though it is straightforward, to the best of our knowledge the calculations have not been carried out in any publication so far.
  \section{A Matrix Analysis Primer}
    The \textbf{trace} of a square matrix, denoted by $\mathrm{tr},$
  is the sum of its diagonal entries, i.e. 
  $
    \mathrm{tr}(\mathbf{B})
    =
    \sum_{j=1}^{d}b_{jj}
    \quad 
    \text{for}\ 
    \mathbf{B} \in \mathbb{M}_d.
  $
  The trace is unitarily invariant, i.e.
  $
    \mathrm{tr}(\mathbf{B})
    =
    \mathrm{tr}(\mathbf{Q}\mathbf{B}\mathbf{Q}^*)
    \quad 
    \text{for all}
    \ 
    \mathbf{B}\in \mathbb{M}_d
    \ 
    \text{for all unitary}\ 
    \mathbf{Q} \in \mathbb{M}_d.
  $
  In particular, the existence of an eigenvalue value decomposition shows 
  that the trace of a Hermitian matrix equals the sum of its  eigenvalues.
  Let
  $
  f: I\to \R
  $
  where 
  $I\subseteq\R$ 
  is an interval.
  Consider a matrix 
  $\mathbf{A}\in \mathbb{H}_d$
  whose eigenvalues are contained in $I.$
  We define the matrix 
  $
    f(\mathbf{A})\in \mathbb{H}_d
  $
  using an eigenvalue decomposition of $\mathbf{A}:$
  \begin{gather}
    f(\mathbf{A})
    =
    \mathbf{Q}
    \begin{bmatrix}
      f(\lambda_1) &&\\
                   &\ddots&\\
                   && f(\lambda_d)
    \end{bmatrix}
    \mathbf{Q}^*
    \qquad
    \text{where}
    \qquad
    \mathbf{A}
    =
    \mathbf{Q}
    \begin{bmatrix}
      \lambda_1 &&\\
                   &\ddots&\\
                   && \lambda_d
    \end{bmatrix}
    \mathbf{Q}^*
    .
  \end{gather}
  The definition of $f(\mathbf{A})$ does not depend on which 
  eigenvalue decomposition we choose.
  Any matrix function that arises in this fashion is called a \textbf{standard matrix function}.


\begin{proposition}
  Let
  $
  f,g: I\to \R
  $
  be real-valued functions on an interval $I\subseteq\R,$ 
  and let
  $\mathbf{A}\in \mathbb{H}_d$
  be a Hermitian matrix
  whose eigenvalues are contained in $I.$

  \begin{enumerate}[label={(\roman*)}]
    \item
      If $\lambda$ is an eigenvalue of of $\mathbf{A},$
      then $f(\lambda)$ is an eigenvalue of $f(\mathbf{A}).$
    \item
      $
        f(a)
        \le
        g(a)
        \quad
        \text{for all}\ 
        a\in I
        \quad
        \text{implies}
        \quad
        f(\mathbf{A})
        \preccurlyeq
        g(\mathbf{A})
        .
      $
  \end{enumerate}
\end{proposition}


\begin{lemma}
  \emph{(Mean value trace inequality)}
  Let 
  $I$
  be an interval of the real line. Suppose that
  $
    g:
    I \to \R
  $
  is a weakly increasing function and that 
  $
    h:
    I \to \R
  $
  is a function whose derivative $h^{'}$ is convex.
  Then for all matrices 
  $
    \mathbf{A}
    ,
    \mathbf{B}
    \in 
    \mathbb{H}_d(I)
  $
  it holds
  \begin{gather}
    \overline{\mathrm{tr}}
    [
    (
      g(\mathbf{A}) - g(\mathbf{B})
    )
    \cdot
    (
      h(\mathbf{A}) - h(\mathbf{B})
    )
    ]
    \le
    \frac{1}{2}
    \,
    \overline{\mathrm{tr}}
    [
    (
      g(\mathbf{A}) - g(\mathbf{B})
    )
    \cdot
    (
    \mathbf{A}
    -
    \mathbf{B}
    )
    \cdot
    (
    h^{'}(\mathbf{A}) + h^{'}(\mathbf{B})
    )
    ]
    .
  \end{gather}
  When $h^{'}$ is concave, the inequality is reversed. The same result holds for the standard trace.
\end{lemma}
\begin{proof}
  \emph{\cite[Lemma~3.4]{Mackey2014}}
  Fix 
  $
    a,b
    \in
    I
.
  $
  Since 
  $g$
  is
  weakly increasing,
  $
  (
    g(a) - g(b)
  )
  \cdot
  (a-b)
  \ge 0.
  $
  The 
  fundamental theorem of calculus and the convexity of 
  $h^{'}$
  yield the estimate
  \begin{align}
  (
    g(a) - g(b)
  )
  \cdot
  (
    h(a) - h(b)
  )
  &=
  (
    g(a) - g(b)
  )
  \cdot
  (a-b)
  \int_0^1
  h^{'}
  (
    \tau a + (1-\tau) b
  )
  \mathrm{d}\tau
  \\
  &\le
  (
    g(a) - g(b)
  )
  \cdot
  (a-b)
  \int_0^1
  [
  \tau
  h^{'}
  (
     a 
  )
  +
  (1-\tau)
  h^{'}
  (
     b 
  )
  ]
  \mathrm{d}\tau
  \\
  &=
  \frac{1}{2}
  \,
  [
  (
    g(a) - g(b)
  )
  \cdot
  (a-b)
  \cdot  
  (
    h^{'}
    (a)
    +
    h^{'}
    (b)
  )
  ]
  .
  \end{align}
  The inequality is reversed, if $h^{'}$ is concave.
  $
  $
  To apply the Kleins inequality we expand the terms.
  The RHS is
  \begin{align}
    \begin{split}
  &(
    g(a) - g(b)
  )
  \cdot
  (a-b)
  \cdot  
  (
    h^{'}
    (a)
    +
    h^{'}
    (b)
  )
  \\
  &\quad=
  [
g(a)\cdot a \cdot h^{'}(a)
  ]
    +
    [g(a)\cdot a] \cdot h^{'}(b)
  -
  b \cdot[h^{'}(a)\cdot g(a)]
  -
  [b \cdot h^{'}(b)]\cdot g(a)
  \\
  &\qquad
  +
    [
\ \text{the same as above with $a$ and $b$ reversed}\ 
  ]
  (a \rightleftarrows b)
  \\
  \end{split}
  \end{align}
  Taking the trace yields
  \begin{align}
    \begin{split}
&
\mathrm{tr}
  [
    g(\mathbf{A})
    \cdot
    \mathbf{A}
    \cdot
    (
    h^{'}(\mathbf{A})
    +
    h^{'}(\mathbf{B})
    )
  ]
  -
  \mathrm{tr}
  [
    \mathbf{B}
    \cdot
    (
    h^{'}(\mathbf{A})
    +
    h^{'}(\mathbf{B})
    )
    \cdot
    g(\mathbf{A})
  ]
  +
  (\mathbf{A} \rightleftarrows \mathbf{B})
  \\
  &\quad=
\mathrm{tr}
  [
    g(\mathbf{A})
    \cdot
    \mathbf{A}
    \cdot
    (
    h^{'}(\mathbf{A})
    +
    h^{'}(\mathbf{B})
    )
  ]
  -
  \mathrm{tr}
  [
    g(\mathbf{A})
    \cdot
    \mathbf{B}
    \cdot
    (
    h^{'}(\mathbf{A})
    +
    h^{'}(\mathbf{B})
    )
  ]
  +
  (\mathbf{A} \rightleftarrows \mathbf{B})
  \\
  &\quad=
  \mathrm{tr}
  [
    g(\mathbf{A})
    \cdot
    (
    \mathbf{A}
    -
    \mathbf{B}
    )
    \cdot
    (
    h^{'}(\mathbf{A})
    +
    h^{'}(\mathbf{B})
    )
  ]
  +
  (\mathbf{A} \rightleftarrows \mathbf{B})
  \\
  &\quad=
  \mathrm{tr}
  [
  (
    g(\mathbf{A})
    -
    g(\mathbf{B})
  )
    \cdot
    (
    \mathbf{A}
    -
    \mathbf{B}
    )
    \cdot
    (
    h^{'}(\mathbf{A})
    +
    h^{'}(\mathbf{B})
    )
  ].
    \end{split}
  \end{align}
  On the LHS we have only products of two factors which commute under the trace operation. Thus we may use the same expression as in the scalar case without further calculations.
  The result follows immediately from the Klein inequality.
\end{proof}

\begin{proposition}
  \emph{(Generalized Klein inequality)}
  Let 
  $
    u_1, \ldots, u_n
  $
  and
  $
    v_1, \ldots, v_n
  $
  be real-valued functions on an interval $I$
  of the real line.
  Suppose
  \begin{gather}
    \sum_{k=1}^{n}
    u_k(a)
    v_k(b)
    \ge
    0
    \qquad
    \text{for all}
    \ 
    a,b \in I
    .
  \end{gather}
  Then
  \begin{gather}
    \overline{\mathrm{tr}}
    \left( 
    \sum_{k=1}^{n}
    u_k(\mathbf{A})
    v_k(\mathbf{B})
    \right)
    \ge 0
    \qquad
    \text{for all}
    \ 
    \mathbf{A}, \mathbf{B} \in \mathbb{H}_d(I)
    .
  \end{gather}
\end{proposition}
\begin{proof}
  \emph{\cite[Proposition~3]{Petz1994}}
\end{proof}




\begin{proposition}
  \emph{(Hölder inequality for trace)}
  Let 
  $p$ and $q$
  be Hölder conjugate indices.
  Then
  \begin{gather}
    \mathrm{tr}
    (
    \mathbf{BC}
    )
    \le
    \norm{\mathbf{B}}_p
    \norm{\mathbf{C}}_q
    \qquad
    \text{for all}
    \ 
    \mathbf{B}
    ,
    \mathbf{C}
    \in 
    \mathbb{M}_d
    .
  \end{gather}
\end{proposition}
\begin{proof}
  \cite[Corollary~IV.2.6]{Bhatia1997}
\end{proof}

  %\section{The Method of Exchangeable Pairs}
  %\subsection{Matrix Stein pairs}
We first define an exchangable pair.
\begin{definition}
  Let 
  $Z$ and $Z^{'}$
  random variables taking values
  in a Polish space $\mathcal{Z}.$
  We say that
  $
  (
    Z
    ,
    Z^{'}
  )
  $
  is an \textbf{exchangable pair}
  if it has the same distribution as 
  $
  (
  Z^{'}
    ,
    Z
  )
  .
  $
  In particular, 
  $Z$ and $Z^{'}$
  must share the same distribution.
\end{definition}

The following approach originates in the work of Charles Stein \cite{Stein1972} 
on normal approximation for a sum of dependent random variable.
We will explain how some central ideas of this theory extends to matrices.

We can obtain a lot of information about the fluctuation of a random matrix
$\mathbf{X}$
if we can construct a good exchangable pair 
$
(
  \mathbf{X}
  ,
  \mathbf{X}^{'}
)
.
$
With this motivation in mind, let us introduce a special class of exchangable pairs.
\begin{definition}
  Let
  $
  (
    Z
    ,
    Z^{'}
  )
  $
  be an exchangable pair of random variables taking values
  in a Polish space $\mathcal{Z},$
  and let 
  $
    \mathbf{\Psi}
    : 
    \mathcal{Z}
    \to 
  \mathbb{H}_d
  $
  be a measurable function.
  Define the random Hermitian matrices
  \begin{gather}
    \mathbf{X}
    :=
    \mathbf{\Psi}
    (Z)
    \quad
    \text{and}
    \quad
    \mathbf{X}^{'}
    :=
    \mathbf{\Psi}
    (Z^{'})
    .
  \end{gather}
  We say that 
  $
  (
    \mathbf{X}
    ,
    \mathbf{X}^{'}
  )
  $
  is a \textbf{matrix Stein pair}
  if there is a constant 
  $\alpha\in (0,1]$
  for which
  \begin{gather}
    \E[
    \mathbf{X}
    -
    \mathbf{X}^{'}
    \vert
    Z
    ]
    =
    \alpha 
    \mathbf{X}
    \qquad
    \text{almost surely.}
  \end{gather}
  The constant 
  $\alpha$
  is called the \textbf{scale factor} of the pair.
  We always assume 
  $
    \E
    \left[ 
      \norm{\mathbf{X}}^2
    \right]
    <
    \infty
    .
  $
\end{definition}

A matrix Stein pair 
$
(
\mathbf{X}
,
\mathbf{X}^{'}
)
$
has several useful propreties. First,
$
(
\mathbf{X}
,
\mathbf{X}^{'}
)
$
always forms an exchangable pair. Second, it must be the case that
$\E[\mathbf{X}]=\mathbf{0}.$
Indeed,
\begin{gather*}
  \E[\mathbf{X}]
  =
  \frac{1}{\alpha}
  \E[
  \E
  [
    \mathbf{X}
    -
    \mathbf{X}^{'}
    \vert
    Z
  ]
  ]
  =
  \frac{1}{\alpha}
  \E[
    \mathbf{X}
    -
    \mathbf{X}^{'}
    \vert
  ]
  =
  \mathbf{0}
  .
\end{gather*}
\subsection{The method of exchangable pairs}
A well-chosen matrix Stein pair 
$
(
\mathbf{X}
,
\mathbf{X}^{'}
)
$
provides a surprisingly powerful tool for studying 
the random matrix $\mathbf{X}.$
The technique depends on a fundamental technical lemma.
\begin{lemma}
  Suppose that
  $
  (
    \mathbf{X}
    ,
    \mathbf{X}^{'}
  )
  $
  is a matrix Stein pair with scale factor $\alpha.$
  Let 
  $
    \mathbf{F}
    :
    \mathbb{H}_d
    \to
    \mathbb{H}_d
  $
  be a measurable function that satisfies the regularity condition
  $
  \E
  \left[
  \norm{
    (
    \mathbf{X}
    -
    \mathbf{X}^{'}
  )
  \mathbf{F}(\mathbf{X})
  }
  \right]
  <
  \infty
  .
  $
  Then
\begin{gather}
  \E
  [
    \mathbf{X}
    \cdot
    \mathbf{F}
    (\mathbf{X})
  ]
  =
  \frac{1}{2\alpha}
  \E
  [
    (
    \mathbf{X}
    -
    \mathbf{X}^{'}
    )
    (
    \mathbf{F}
    (
    \mathbf{X}
    )
    -
    \mathbf{F}
    (
    \mathbf{X}^{'}
    )
  )
  ]
  .
\end{gather}
\end{lemma}
In short, the randomness in the Stein pair furnishes an alternative expression for the expected product of $\mathbf{X}$
and a function $\mathbf{F}.$
It allows us to estimate the expectation using the smoothness properties of the function $\mathbf{F}$ and the discrepancy between $\mathbf{X}$
and $\mathbf{X}^{'}.$
\begin{proof}
  \emph{\cite[Lemma~2.4]{Mackey2014}}
  Suppose that
  $
  (
    \mathbf{X}
    ,
    \mathbf{X}^{'}
  )
  $
  constructed from an auxiliary exchangable pair 
  $
  (
  Z,
  Z^{'}
  )
  .
  $
  The defining property implies
 \begin{gather}
   \alpha \cdot 
   \E[
   \mathbf{X}
   \cdot
   \mathbf{F}
   (\mathbf{X})
   ]
   =
   \E[
    \E[
    \mathbf{X}
    -
    \mathbf{X}^{'}
    \vert
    Z
    ]
    \cdot
   \mathbf{F}
   (\mathbf{X})
   ]
   =
    \E[
    (
    \mathbf{X}
    -
    \mathbf{X}^{'}
    )
   \mathbf{F}
   (\mathbf{X})
   ]
 \end{gather} 
\end{proof}
\subsection{The conditional variance}
To each matrix Stein pair 
  $
  (
    \mathbf{X}
    ,
    \mathbf{X}^{'}
  ),
  $
  we may associate a random matrix called the conditional variance of $\mathbf{X}.$
  The purpose of this section is to argue that the spectral norm of $\mathbf{X}$
  is unlikely to be large, when the conditional variance is small.
\begin{definition}
  Suppose that
  $
  (
    \mathbf{X}
    ,
    \mathbf{X}^{'}
  ),
  $
  is a matrix Stein pair, constructed from an auxiliary exchangeable pair
  $
  (
  Z
    ,
  Z^{'}
  ).
  $
  The \textbf{conditional variance}
  is the random matrix
  \begin{gather}
    \mathbf{\Delta_X}
    :=
    \mathbf{\Delta_X}
    (Z)
    :=
    \frac{1}{2\alpha}
    \E
    [
    (
    \mathbf{X}
    -
    \mathbf{X}^{'}
    )^2
    \vert
    Z
    ]
    ,
  \end{gather}
  where $\alpha$ is the scale factor of the pair. We may take any version of the conditional expectation in this definition.
\end{definition}

The conditional variance
$
    \mathbf{\Delta_X}
$
can be regarded as a stochastic estimate for the variance 
of the random matrix $\mathbf{X}.$
To see this, assume
Indeed, 
\begin{gather}
  \E
  [
    \mathbf{\Delta_X}
  ]
\end{gather}

 \section{Matrix Khintchin Inequality and Applications}
 In this section we state the matrix Khintchin inequality and matrix moment inequalities as an applications.
 We provide the proof of auxiliary theorems which are cited without proof in \cite{Mackey2014}. They are needed to prove the matrix Khintchin inequality. 
  \begin{theorem}
  \emph{(Matrix BDG inequality)}
  Let
  $
    p = 1
    \ 
    \text{or}\ 
    p \ge 3/2
    .
  $
  Suppose that 
  $
  (
    \mathbf{X}
   , 
   \mathbf{X}^{'}
  )
  $
  is a matrix Stein pair where
  $
   \E
   [ 
    \norm{\mathbf{X}}
    _{2p}^{2p}
   ]
   <
   \infty
   .
  $
\end{theorem}




\begin{ftheorem}
  \emph{\cite[Corollary~7.3]{Mackey2014}}
  Suppose that
  $
    p = 1
    \ 
    \text{or}\ 
    p \ge 3/2
    .
  $
  Consider a finite sequence
  $
    (\mathbf{Y}_k)_{k\ge 1}
  $
  of independent, random, Hermitian matrices 
  and a deterministic sequence
  $
    (\mathbf{A}_k)_{k\ge 1}
  $
  for which
  \begin{gather}
    \E[\mathbf{Y}_k]
    =
    0
    \quad 
    \text{and}
    \quad
    \mathbf{Y}_k^2
    \preccurlyeq
    \mathbf{A}_k^2
    \qquad
    \text{almost surely for all}\ 
    k \ge 1.
  \end{gather}
  Then
  \begin{gather}
      \E
      \left[
        \norm{
          \sum_{k\ge 1}
            \mathbf{Y}_k
        }
        _{2p}
        ^{2p}
      \right]
      ^{1/(2p)}
      \le
      \sqrt{
        p - \frac{1}{2}
      }
      \,
      \norm{
        \left( 
          \sum_{k\ge 1}
          (
            \mathbf{A}_k^2
            + 
            \E[
              \mathbf{Y}_k^2
            ]
          )
        \right)
        ^{1/2}
        }
      _{2p}
      .
  \end{gather}
  In particular, when 
  $
    (\xi_k)_{k\ge 1}
  $
  is an independent sequence of Rademacher random variables,
  \begin{gather}
      \E
      \left[
        \norm{
          \sum_{k\ge 1}
            \xi_k
            \mathbf{A}_k
        }
        _{2p}
        ^{2p}
      \right]
      ^{1/(2p)}
      \le
      \sqrt{
        2p - 1
        }
      \,
      \norm{
        \left( 
          \sum_{k\ge 1}
            \mathbf{A}_k^2
        \right)
        ^{1/2}
        }
      _{2p}
      .
  \end{gather}
\end{ftheorem}

  \section{Generalzed Inequalities by Hermitian Dilation}
  \begin{definition}
  \emph{(Hermitian Dilation)}
  The Hermitian dilation
  \begin{gather*}
    \mathfrak{H} : \C^{d_1 \times d_2} \to \mathbb{H}_{d_1 \times d_2}
  \end{gather*}
  is a map from a general matrix to an Hermitian matrix defined by
  \begin{gather}
    \label{ rmineq_hermitian_dilation } 
    \mathfrak{H}(B)
    :=
    \begin{bmatrix}
      0   & B \\
      B^* & 0 \\
    \end{bmatrix}
  \end{gather}
\end{definition}

  \begin{ftheorem}
  \emph{(Matrix Rosenthal-Pinelis)}
  \label{rmineq_rosenthal_pinelis}
  Let $\mathbf{A}_1, \ldots, \mathbf{A}_n$ be independent, random matrices with dimension 
  $d_1 \times d_2$.
    Introduce the random matrix
      \begin{gather*}
        \mathbf{S}:=\sum_{k=1}^n \mathbf{A}_k.
      \end{gather*}
    Let $v(\mathbf{S})$ be the matrix variance statistic of the sum:
      \begin{align}
        \label{rmineq_bernstein_cond_2}
        v(\mathbf{S}):= \norm{\E[\mathbf{S}\mathbf{S}^\top ]} \lor \norm{\E[\mathbf{S}^\top \mathbf{S}]} 
             = \norm{\sum_{k=1}^n\E[\mathbf{A}_k\mathbf{A}_k^\top]} \lor \norm{\sum_{k=1}^n\E[\mathbf{A}_k^T \mathbf{A}_k]} .
      \end{align}
    Then
      \begin{align}
        \label{rmineq_rp_expectation_bound}
        \left(
          \E \left[ \norm{\mathbf{S}}^2 \right]
        \right)^{\frac{1}{2}}
        \le
        \sqrt{
          2ev(\mathbf{S})\log(d_1+d_2)
        } 
        + 
        4e \left( 
          \E[\max_{k \le n}\norm{\mathbf{A}_k}^2]
        \right)^\frac{1}{2}
        \log(d_1+d_2).
      \end{align}
\end{ftheorem}

\begin{remark}
  Since
  $
    \E[\norm{S}]
    \le
    \E[\norm{S}^2]^\frac{1}{2}
  $
  by the Cauchy-Schwarz inequality,
  Theorem~\ref{rmineq_rosenthal_pinelis}
  also holds with 
  $
    \E[\norm{S}]
  $
  on the left-hand side of \eqref{rmineq_rp_expectation_bound}.
  To obtain a tail bound we can employ the Markov inequality and 
  Theorem~\ref{rmineq_rosenthal_pinelis}:
  \begin{align}
    \label{rmineq_rp_tail_bound}
    \begin{split}
    \P[&\norm{S}\ge t]
    \\
       &\le
    \frac{
    \E[\norm{S}]
    }{t}
    \le
    \frac{1}{t}
    \left( 
        \sqrt{
          2ev(\mathbf{S})\log(d_1+d_2)
        } 
        + 
        4e \left( 
          \E[\max_{k \le n}\norm{\mathbf{A}_k}^2]
        \right)^\frac{1}{2}
        \log(d_1+d_2)
    \right)
    \quad
    \text{for}\ 
    t>0.
  \end{split}
  \end{align}
  It might be possible to improve the $\log$ term employing an intrinsic dimension argument.
\end{remark}
  

  
  \newpage
  \section{Intrinsic Dimension}
  \begin{definition}
  \label{rmineq_intrinsic_bernstein}
  For a positive-semidefinite matrix $\mathbf{S}$,
  the \textbf{intrinic dimension} is the quantity
  \begin{gather*}
    \mathrm{intdim}
    (\mathbf{A})
    :=
    \frac{\mathrm{tr}\mathbf{A}}{\norm{\mathbf{A}}}
    .
  \end{gather*}
\end{definition}
\begin{lemma}
  \emph{(Intrinsic dimenision)}
  Let 
  $
    \varphi: [0,\infty) \to \R
  $
  be a convex function with
  $
    \varphi(0)=0.
  $
  For any positive-semidefinite matrix $\mathbf{S}$ it holds that
  \begin{gather*}
    \mathrm{tr}(\varphi(\mathbf{S}))
    \le
    \mathrm{intdim}(\mathbf{S})
    \cdot
    \varphi(\norm{\mathbf{S}})
    .
  \end{gather*}
\end{lemma}
\begin{proof}
  \emph{\cite[Lemma~7.5.1]{Tropp2015}}
  Since $\varphi$ is convex on any interval $[0,L]$ with $L>0$ and $\varphi(0)=0$, it holds
  \begin{gather}
    \varphi(a)
    \le
    \left( 
      1 - \frac{a}{L}
    \right)
    \varphi(0)
    +
    \frac{a}{L}
    \varphi(L)
    =
    \frac{a}{L}
    \varphi(L)
    \qquad
    \text{for all}\ 
    a \in [0,L]
    .
  \end{gather}
  Since $\mathbf{S}$ is positive-semidefinite, the eigenvalues of $\mathbf{S}$ 
  fall in the interval $[0,L]$, where $L=\norm{\mathbf{S}}.$
  \begin{gather}
    \mathrm{tr}(\varphi(\mathbf{S}))
    =
    \sum_{i=1}^{d}
    \varphi(\lambda_i)
    \le
    \frac{
    \sum_{i=1}^{d}
    \lambda_i
    }{\norm{\mathbf{S}}}
    \varphi(\norm{\mathbf{S}})
    =
    \frac{\mathrm{tr}(\mathbf{S})}{\norm{\mathbf{S}}}
    \varphi(\norm{\mathbf{S}})
    =
    \mathrm{intdim}(\mathbf{S})
    \cdot
    \varphi(\norm{\mathbf{S}})
    .
  \end{gather}
\end{proof}
The next example applies the preceding lemma to bound the Schatten-norm in terms of the spectral norm and the intrinsic dimension.
  \begin{example*}
    Let 
    $
      \mathbf{B} \in \mathbb{C}^{m\times n}
    $
    be any rectangular matrix and let $p\ge 2$.
    Then 
    $
      \varphi(x)
      :=
      \left| x \right|
    $
  defines a convex function with $\varphi(0)=0$.
  The intrinsic dimension lemma yields
  \begin{gather}
    \norm{B}^p_p
    =
    \mathrm{tr}
      \left|
      B^* B
      \right|
      ^{p/2}
    \le
    \mathrm{intdim}
    (
      B^* B
    )
      \cdot
      \norm{
      B^* B
    }^{p/2}
    =
    \mathrm{intdim}
    (
      B^* B
    )
      \cdot
      \norm{
        B
    }^{p}
    \,.
  \end{gather}
  If, additionally, $\mathbf{B}$ is self-adjoint and positive-semidefinite
  then it holds 
  \begin{gather}
    \mathrm{tr}
    (
      B^* B
    )
    =
    \mathrm{tr}
    (
    B^2
    )
    =
    \sum_{i=1}^{n} 
    \lambda_i^2
    \le 
    \left( 
    \sum_{i=1}^{n} 
    \lambda_i
    \right)
    ^{2}
    =
    \left( 
    \mathrm{tr}
    B
    \right)
    ^2
    \,,
  \end{gather}
  und consequently
  \begin{gather}
    \norm{B}^p_p
    \le
    \left( 
    \mathrm{intdim}
    B
    \right)
    ^2
      \cdot
      \norm{
        B
    }^{p}
    \,.
  \end{gather}
  \end{example*}
\begin{takeaways}
  Is it intrinsic or extrinsic?
  \lipsum[4]
\end{takeaways}



\chapter{Empirical Processes}
Classical references are \cite{Vaart2000} and \cite{vaart2013}.
For maximal inequalities see \cite[§19]{Vaart2000}
For Functional Delta-Method see \cite[§20]{Vaart2000}
For an introduction to empirical processes and outer expectation 
see the beginning of \cite{vaart2013}.
\section{A Primer on Empirical Processes}

Let 
$
  \left( 
    \Omega,
    \mathcal{A},
    \P
  \right)
$
be a probability space,
$
  \left( 
    \mathcal{X},
    \Sigma
  \right)
$
a measurable space, and 
$
  X_1,\ldots,X_n
  :
  \left( 
    \Omega,
    \mathcal{A},
    \P
  \right)
  \to
  \left( 
    \mathcal{X},
    \Sigma
  \right)
$
a sample 
of independent and identically-distributed
random variables
with probability distribution $\P_{\!X}$.
Throughout this section we consider the~\textbf{empirical~measure}
of this sample, that is, the discrete random measure
\begin{gather}
  \P_{\!n}:\Sigma \to [0,1]
  ,
  \quad
  C\mapsto 
  \frac{1}{n}
  \#\left\{ 
1\le i \le n \colon
X_i \in C
  \right\}
  \,.
\end{gather}
A family $\mathcal{F}$ of measurable functions 
$
  f:
  \left( 
    \mathcal{X},
    \Sigma
  \right)
    \to
  \left( 
    \R,
    \mathcal{B}(\R)
  \right)
$
induces a stochastic process by
\begin{gather}
  f\mapsto \P_{\!n} f\,,
\end{gather}
where for a measure $Q$ on 
$
  \left( 
    \mathcal{X},
    \Sigma
  \right)
$
we denote $Q f:= \int_\mathcal{X}f\, Q(dx)$.
In this way we define the $\mathcal{F}\!$-indexed \textbf{empirical process} $\G_n$ by
\begin{gather}
  f
  \ 
  \mapsto
  \ 
  \G_n f 
  \ 
  :=
  \ 
  \sqrt{n}
  (\P_{\!n}-\P)f 
  \ 
  =
  \ 
  \frac{1}{\sqrt{n}}
  \sum_{i=1}^{n} 
  (
    f(X_i)
    -
    \P f
  )
  \,.
\end{gather}
The purpose of this notation is to abstract the behaviour of $\G_n$ ranging over $\mathcal{F}$.
Conforming with this integral viewpoint, we define the (random) norm
\begin{gather}
  \norm{\G_n}_\mathcal{F}
  :=
  \sup_
        { f \in \mathcal{F}}
        \left|
          G_n f
        \right|
        .
\end{gather}
We stress that 
$
  \norm{\G_n}_\mathcal{F}
$
often ceases to be measurable, even in simple situations~\cite[page 3]{vaart2013}.
To deal with this, we introduce the notion of \textbf{outer expectation} $\E^*$, that is,
\begin{gather}
  \E^*[T]
  \ 
  :=
  \ 
    \inf
  \left\{ 
    \E[U]
  \ 
  \lvert
  \ 
    U\ge T,
    \ 
    U:
  \left( 
    \Omega,
    \mathcal{A},
    \P
  \right)
  \to 
  \left( 
    \overline{\R},
    \mathcal{B}(\overline{\R})
  \right)
  \text{measurable and}
  \ 
  \E[U]<\infty
  \right\}
  \,.
\end{gather}
In our application the technical difficulties halt at this point, because we only consider $T$ with $\E^*[T]<\infty$. Then there exists a smallest measurable function $T^*$ dominating $T$ with
$\E^*[T]=\E[T^*]$. Thus, we may assume $T$ to be measurable in this regard.

In our application we need concentration inequalities for 
$
  \norm{\G_n}_\mathcal{F}
$.
One easy way is to use maximal inequalities for the expectation together with Markov's inequality. There are also Bernstein-like inequalities for empirical processes.





%\section{Maximal Inequalities}
%\input{chapters/empirical_processes/maximal_ineq.tex}

%\section{Functional Delta Method}
%\begin{definition}
  A map 
  $
  \phi:
  \mathbb{D}_\phi
  \to 
  \mathbb{E}
  ,
  $
  defined on a subset 
  $
  \mathbb{D}_\phi
  $
  of a normed space
  $\mathbb{D}$
  that contains 
  $\theta,$
  is called 
  \textbf{Hadamard diffenertiable}
  at $\theta$
  if there exists a continuous,
  linear map
  $
  \phi_\theta^{'}
    :
    \mathbb{D}
    \to 
    \mathbb{E}
  $
  such that
  \begin{gather}
    \norm{
      \frac{
        \phi(\theta + t h_t)
        -
        \phi(\theta)
      }{
        t
      }
      -
      \phi^{'}_\theta
      (h)
    }_\mathbb{E}
    \to
    0
    \quad
    \text{as}
    \ 
    t\searrow 0
    \ 
    \text{for all}
    \ 
    h_t \to h
  \end{gather}
  $
    \text{such that $\theta + th_t$ is contained in $\mathbb{D}_\phi$ for all small $t>0.$}
  $
\end{definition}


\begin{ftheorem}
  \emph{(Delta Method)}
  Let 
  $
    \mathbb{D}
    \ \text{and}
    \ 
    \mathbb{E}
  $
  be normed linear spaces.
  Let
  $
    \phi
    :
    \mathbb{D}_\phi
    \subseteq
    \mathbb{D}
    \to
    \mathbb{E}
  $
  be Hadamard differentiable it $\theta$
  tangentially to 
  $\mathbb{D}_0.$
  Let
  $
    T_n
    :
    \Omega_n
    \to
    \mathbb{D}_\phi
  $
  be maps such that 
  $
    r_n
    (T_n - \theta)
    \rightsquigarrow
    T
  $
  for some sequence of numbers $r_n \to \infty$
  and a random element $T$
  that takes its values in $\mathbb{D}_0.$
  Then 
  $
    r_n(\phi(T_n)-\phi(\theta))
    \rightsquigarrow
    \phi^{'}
    _\theta
    (T)
    .
  $
  If 
  $
    \phi^{'}
    _\theta
  $
  is defined and continuous on the whole space $\mathbb{D},$
  then we also have 
  $
    r_n
    (
      \phi(T_n)
      -
      \phi(\theta)
    )
    =
    \phi^{'}
    _\theta
    (
    r_n
    (
    T_n
    -
    \theta
    )
    )
    +
    o_\P
    (1)
    .
  $
\end{ftheorem}
\begin{proof}
  \cite[Theorem~20.8]{Vaart1998}
\end{proof}


\chapter{Simple yet useful Calculations} 
\begin{proposition}
  Let 
  $f : \R^n \to \R$ 
  be continuous such that 
  a minimum $x^*$ exists and is unique.
  Then 
  for all $y \in \R^n$ and $C>0$ 
  it follows
    \begin{gather}
      \inf_{\norm{\Delta}=C} f(y+\Delta) - f(y) > 0 \qquad
      \Rightarrow \qquad 
      \norm{x^* - y} \le C.
    \end{gather}
\end{proposition}


\begin{proof}
Since 
$\mathcal{C}:=\left\{ \norm{\Delta}\le C \right\}$
is compact and
\begin{gather*}
  f(x^*) \le f(y) <  \inf_{\norm{\Delta}=C} f(y+\Delta)
\end{gather*}
the continious function $f(y+\,\cdot\,)$ has a minimum in 
$\overset{\circ}{\mathcal{C}}:=\left\{ \norm{\Delta} < C \right\}$. 
Since 
$x^*$ is the unique minimum of $f$
there exists $\Delta^* \in \overset{\circ}{\mathcal{C}}$ 
such that 
$x^* - y = \Delta^*$.
We conclude that
$\norm{x^* - y} \le C$.
\end{proof}

 %%%%%%%%%%%%%%%%%%%%%%%%%%%%%%%%%%%%%%%%%%%%%%%%%%%%%%%%%%%%%%%%

\begin{proposition}
  Let 
  $f\in C^2(\R)$. 
  Then
  for all $a,x,\Delta \in \R^n$ 
  there exist $\xi_1, \xi_2 \in (0,1)$ such that it holds
  \begin{gather}
    f(a^T (x + \Delta)) - f(a^T x) = 
    f^{'}(a^T x)\,a^T x + 
    f^{''}(a^T (x + \xi_1\xi_2 \Delta))\, \Delta^T A\ \Delta,
  \end{gather}
  where 
  $A:= a a^T \in \R^{n \times n}$ .
\end{proposition}




%makeindex main.nlo -s nomencl.ist -o main.nls

\nomenclature{$\P$}{generic probability measure}
\nomenclature{$\overset{\mathcal{D}}{\longrightarrow}$}{convergence of distributions}

\printnomenclature

\bibliography{literature}{}
\bibliographystyle{alpha}
\end{document}
