\documentclass[11pt, a4paper, BCOR=10mm, DIV=11]{scrbook}
\linespread{1.25}
\usepackage[utf8]{inputenc}
\usepackage{graphicx}
%\usepackage[a4paper, margin=2.5cm]{geometry}
\usepackage{hyperref}
\usepackage{amsmath}
\usepackage{enumitem}
\usepackage{mdframed}
\usepackage{amsthm}
\usepackage{amssymb}
\usepackage{mathtools}
%\usepackage{fdsymbol}
\usepackage{cite}
\graphicspath{ {images/} }
\usepackage{xcolor}         % Extended colors
\usepackage{color}         % Color extended names
\usepackage{nomencl}
\usepackage{lipsum}
\usepackage[tickmarkheight=0.1cm, colorinlistoftodos]{todonotes}
\makenomenclature
\renewcommand{\nomname}{Notation Index}
\mathsurround=2pt


\newsavebox{\selvestebox}
\newenvironment{takeaways}
  {
   \begin{lrbox}{\selvestebox}%
   \begin{minipage}{12.4cm}
     \textbf{Takeaways}
   }
  {\end{minipage}\end{lrbox}%
   \begin{center}
\setlength\fboxsep{.5cm}
   \colorbox[HTML]{F8E0E0}{\usebox{\selvestebox}}
   \end{center}}
% Theorem handle


\newtheorem{theorem}{Theorem}[chapter]
\newtheorem*{theorem*}{Theorem}
\newtheorem*{definition*}{Definition}
\newtheorem*{lemma*}{Lemma}
\newenvironment{ftheorem}
  {\begin{mdframed}\begin{theorem}}
  {\end{theorem}\end{mdframed}}
\newenvironment{ftheorem*}
  {\begin{mdframed}\begin{theorem*}}
  {\end{theorem*}\end{mdframed}}
\newtheorem{assumption}{Assumption}[chapter]
\newtheorem*{assumptions*}{Assumptions}
\newtheorem{corollary}{Corollary}[theorem]
\newtheorem{definition}{Definition}[chapter]

\theoremstyle{definition}

\newtheorem*{remark/}{Remark}
\newenvironment{remark}
  {\renewcommand{\qedsymbol}{$\diamondsuit$}%
   \pushQED{\qed}\begin{remark/}}
  {\popQED\end{remark/}}
\newtheorem{example/}{Example}[chapter]
\newenvironment{example}
  {\renewcommand{\qedsymbol}{$\diamondsuit$}%
   \pushQED{\qed}\begin{example/}}
  {\popQED\end{example/}}
\newtheorem*{example*/}{Example}
\newenvironment{example*}
  {\renewcommand{\qedsymbol}{$\diamondsuit$}%
   \pushQED{\qed}\begin{example*/}}
  {\popQED\end{example*/}}
\newtheorem*{reflection*/}{Reflection}
\newenvironment{reflection*}
  {\renewcommand{\qedsymbol}{$\spadesuit$}%
   \pushQED{\qed}\begin{reflection*/}}
  {\popQED\end{reflection*/}}
\newtheorem{problem}{Problem}[chapter]
\newenvironment{fproblem}
  {\begin{mdframed}\begin{problem}}
  {\end{problem}\end{mdframed}}
  %%%% plain %%%%
\theoremstyle{plain}
\newtheorem{lemma}{Lemma}[chapter]
\newtheorem{proposition}{Proposition}[chapter]
\newtheorem*{proposition*}{Proposition}
\newtheorem{subassumption}{}[assumption]

\renewcommand{\proofname}{\textbf{Proof}}
% definition of constants

\newcommand{\CP}{C_\P}
\newcommand{\Ctau}{C_\tau}
\newcommand{\LearnRate}{\varepsilon_n}

%
\DeclarePairedDelimiterX{\inner}[2]{\langle}{\rangle}{#1, #2}
\newcommand{\C}{\mathbb{C}}
\newcommand{\G}{\mathbb{G}}
\newcommand{\E}{\mathbf{E}}
\renewcommand{\P}{\mathbf{P}}
\newcommand{\R}{\mathbb{R}}
\newcommand\norm[1]{\left\lVert#1\right\rVert}

\title{
  {
    Title?
  }
  \\
  {\large Universität Stuttgart}
  \\
  {\includegraphics{unistuttgart_logo_deutsch.jpg}}
}

\author{Ioan Scheffel}
\date{\today}
\setlength {\marginparwidth }{2cm}

\begin{document}

\listoftodos

\maketitle

\tableofcontents 

%\chapter{Introduction}
%% SAY THAT RESEARCHERS ARE OFTEN LEFT WITH OBSERVATIONAL STUDYS TO ANSWER THEIR QUESTIONS

% SAY SOMETHING ABOUT PS METHODS AS TO DO WITH CAUSAL INFERENCE IN OS

% SAY THAT MODEL DEPENDENCY IS OFTEN A PROBLEM WITH PS

% MENTION DIFFERENT APPROACHES TO ALLEVIATE THIS PROBLEM

% HIGHLIGHT THE METHODS BEING DISCUSSED IN THE THESIS

% OUTLINE STRUCTURE 
%% - WEIGHTING APPROACH BINARY
%%  - MODIFICATIONS IN BINARY SETTING TO ESTIMATE ATE
%%  - EXTENSION TO CONTINUOUS TREATMENT
%% - MATCHING APPROACH BINARY 
%%  - POSSIBLE EXTENSION
%% - SIMULATION STUDY
%% - REAL DATA USECASES


Researchers are often left with observational studies to answer questions about causality. When confounders are present the task of infering causality can become arbitrarily complex. Propensity score methods \cite{Rosenbaum1983}, e.g. inverse probability weighting or matching, are popular methods to adjust for confounders. Usually these methods rely heavily on estimates of the true propensity score, which are known to suffer from model dependencies and misspecification\cite{Kang2007}. This issue becomes more pressing when moving from binary to continuous treatment\cite{Hirano2005}. Therefore methods have been developed to directly target imbalances in the data\cite{Fong2018}\cite{Hainmueller2012}\cite{Zubizarreta2015}.
We take a closer look at \cite{Wang2019} and extend the analysis to settings with continuous treatment 
\cite{Vegetabile2020}\cite{Tubbicke2020}.

%\chapter{Causal Inference}
%In this chapter we want to give a introduction to causal inference.
We particularly highlight the role of propensity score analysis and explain its importance in observational studies.


\section{The Rubin Causal Model}
from wiki:
The Rubin causal model (RCM), also known as the Neyman–Rubin causal model,[1] is an approach to the statistical analysis of cause and effect based on the framework of potential outcomes, named after Donald Rubin. The name "Rubin causal model" was first coined by Paul W. Holland.[2] The potential outcomes framework was first proposed by Jerzy Neyman in his 1923 Master's thesis,[3] though he discussed it only in the context of completely randomized experiments.[4] Rubin extended it into a general framework for thinking about causation in both observational and experimental studies.[1]
\section{Propensity Score Analysis}
from wiki:
In the statistical analysis of observational data, propensity score matching (PSM) is a statistical matching technique that attempts to estimate the effect of a treatment, policy, or other intervention by accounting for the covariates that predict receiving the treatment. PSM attempts to reduce the bias due to confounding variables that could be found in an estimate of the treatment effect obtained from simply comparing outcomes among units that received the treatment versus those that did not. Paul R. Rosenbaum and Donald Rubin introduced the technique in 1983.[1]

The possibility of bias arises because a difference in the treatment outcome (such as the average treatment effect) between treated and untreated groups may be caused by a factor that predicts treatment rather than the treatment itself. In randomized experiments, the randomization enables unbiased estimation of treatment effects; for each covariate, randomization implies that treatment-groups will be balanced on average, by the law of large numbers. Unfortunately, for observational studies, the assignment of treatments to research subjects is typically not random. Matching attempts to reduce the treatment assignment bias, and mimic randomization, by creating a sample of units that received the treatment that is comparable on all observed covariates to a sample of units that did not receive the treatment.

For example, one may be interested to know the consequences of smoking. An observational study is required since it is unethical to randomly assign people to the treatment 'smoking.' The treatment effect estimated by simply comparing those who smoked to those who did not smoke would be biased by any factors that predict smoking (e.g.: gender and age). PSM attempts to control for these biases by making the groups receiving treatment and not-treatment comparable with respect to the control variables. 

from a paper:
Propensity score weighting is one of the techniques used in controlling for selection biases in non-
experimental studies. Propensity scores can be used as weights to account for selection assignment
differences between treatment and comparison groups. One of the advantages of this approach is
that all the individuals in the study can be used for the outcomes evaluation


\section{Weighting beyond the PS}
from \cite{Wang2019}:
Conventionally, the weights are estimated by modeling the propensities of receiving treatment or exhibiting missingness and then inverting the predicted propensities. However, with this approach it can be difficult to properly adjust for or balance the observed covariates. The reason is that this approach only balances covariates in expectation, by the law of large numbers, but in any particular data set it can be difficult to balance covariates, especially if the data set is small or if the covariates are sparse (Zubizarreta et al., 2011). In addition, this approach can result in very unstable estimates when a few observations have very large weights (e.g., Kang and Schafer 2007). To address these problems, a number of methods have been proposed recently. Instead of explicitly modeling the propensities of treatment or missingness, these methods directly balance the covariates. Some of these methods also minimize a measure of dispersion of the weights.

Most of these weighting methods balance covariates exactly rather than approximately. This is a subtle but important difference because approximate balance can trade bias for variance whereas exact balance cannot. Also, exact balance may not admit a solution whereas approximate balance may do so. For a fixed sample size, approximate balance may balance more functions of the covariates than exact balance.

\chapter{Intro for all}

%randomized trials versus observational studies

Is study design more important then statistical analysis?

I think, they are at least equal. 

But a bad analysis can be undone,
whereas a bad design can not.

You have to stick with the data.

If you are not familiar with study design the distinction between randomized and observational study is helpful.

If you read the literature and are unsure about the design of a study, ask for this terms.

You are likely to find an answer.


It is all about how we collect the data.

Say, we want to test the effect of a drug in a study population.

There usually are differences among the units of the study population.

Some are more healthy than others.

We form a treatment and control group, that is, one group takes the drug and the other doesn't.

Then we compare the groups by their health. 

Then a critcal review comes in. What do you mean by healthy.

We mean this and that.

It seems you did not consider this factor.

Maybe the drug is not effective, but the effect we see in your analysis comes from something else.

What do we answer to this?


A good method to avoid this awkward situation is to randomize.

For every unit of the population we toss a fair coin that decides if they get the drug.

Now comes the critic.

From the tables it seems there is an effect. But what about unknown influence?

We answer: Does the coin now of them?

It is not ideal, but this way you can prevent systematic damage to your analysis.


What if we can't decide who gets treatment?

Don't think treatment has to be something good, it should not carry any label of good or bad.

But what about smoking?

Would you smoke if a coin tells you to?

So this is unethical.

But it is also unethical not to investigate the effects of smoking on the health.

Let's accept, that we sometimes (often?) can not control who gets treatment.

Some smoke, some don't, and we mearly observe.

This is typical example for an observational study.

Honestly, this is an oversimplification, but I hope you get the point.

Who still is insulted by the tone will maybe like\cite{Rubin2007}.

%propensity score

In \cite{Rubin2007} you will find the propensity score.

The propensity score is the individual probability to receive treatment, that is,
\begin{gather}
  \P
  [T=1|X]
\end{gather}
if $T$ is the random variable that decides abuot treatment and $X$ is the vector that carries your individual information.

This concept goes back to \cite{Rosenbaum1983}.

It is maybe worth to stop here and think about this definition and its connection to the two study designs.

Discover it for yourself.

\begin{reflection*}
What is the propensity score in the above example.
How does the propensity score behave in rs and os?
\end{reflection*}







\chapter{True Introduction}
Starting point:
Propensitiy score analysis~\cite{Rosenbaum1983}.
Two major branches: PS-weighting and matching.
We study a weighting method.
Procedure: Estimate PS, invert, weight.
Problem: Extreme weights when PS close to 0.
Bias when estimation model is misspecified.
Solution: Balance some measure of dependence simultaneously, e.g.
Covariate balancing PS~\cite{Imai2014a}.
Other solution: Doubly robust estimators\cite{H2005}.
They incorporate treatment and outcome model.
Problem: bad results if either are (slightly) misspecified\cite{Kang2007}.
A third option is obtaining weights (seamingly) unrelated to PS.
Entropy Balancing \cite{Hainmueller2012}, ?balancing\cite{Zubizarreta2015}.
Problem: contstraint $\delta=0$ to strict. Bad convergence.
Solution: relax to $\delta\to 0$ for $N\to\infty$.
Paper with mathematical analysis\cite{Wang2019}. Surprising connection
to PS. Also doubly robust\cite{Zhao2017a}.
This attracted my attention. 

I choose different basis as in \cite{Gyorfi2002}. Does analysis work?
Consistency? Asymptotic Normality? Beyond that?

\cite{Wang2019}: Proofs are substandard. Many mistakes. Missing assumptions. Theorems have to be differently.

This thesis is no erratum of\cite{Wang2019}, but can be consulted for writing one.

We thank Wang and Zubi for discussions.

\chapter{Introduction}
  We consider a study population in which we want to test the effect of a treatment.
We introduce the \textbf{indicator of treatment} $T\in \left\{ 0,1 \right\}$.
For each treatment level there exist the \textbf{marginal potential outcomes}
$(Y(0),Y(1))$. We would like to estimate $\E[Y(1)]$. If we succeed the same technique
shall yield an estimate of $\E[Y(0)]$. We shall compare $\E[Y(1)]$ and $\E[Y(0)]$ and 
find out something about the effect of the treatment in the population.

The data we acquire is independent and identically distributed. But usually
\begin{gather}
  Y(1)|T=1 \nsim Y(1) 
  \,,
\end{gather}
that is, $T=1$ carries more information than observing the outcome under treatment.
We say that $Y(1)|T=1$ is \textbf{confounded}. To extract that plus of information from $T=1$ and put it where it belongs by collecting more data.
We gather it in $X\in \R^d$ and assume
\begin{gather}
  (Y(0),Y(1))
  \perp
  T
  \ 
  |
  \ 
  X
  \,,
\end{gather}
that is, \textbf{conditional unconfoundedness}.
Thus, we end up collecting $N\in \mathbb{N}$ independent and identically distributed copies of 
$(T,X,Y(T))$. For convenience, we assume that the first $n\in \mathbb{N}$ copies have $T=1$.

A natural estimator for $\E[Y(1)]$ is the weighted mean
\begin{gather}
  \frac{1}{n}
  \sum_{i=1}^{n} 
  w_i Y_i
  \,.
\end{gather}
The weights should satisfy (in a broader sense)
\begin{gather}
  w_i\cdot Y_i \to Y(1)
  \qquad 
  \text{for}\ 
  N
  \ 
  \to
  \ 
  \infty
  \,.
\end{gather}
One class of such weights has been recently analyzed in \cite{Wang2019}.
We take ideas and extend.

\subsection*{The algorithm}
\begin{fproblem}
  \label{bw:1:primal}
\begin{align*}
  %%%% objective %%%%
    &\underset{w_1, \ldots, w_n \in \R}
    {\text{minimize}}
    &&\qquad\qquad
    \sum_{i = 1}^{n} 
    f(w_i)
    &&&
    \\
    %%%% w_i T_i >= 0 %%%%
    &\text{subject to}
    &&\qquad\qquad
    w_i 
    \ge
    0
    &&&
    \qquad
    \text{for all}\ 
    i \in \left\{ 1, \ldots, n \right\}
    \,,
    \\
    %%%% 1/n sum w = 1 %%%%
    & 
    &&\qquad\qquad
    \frac{1}{N}
    \sum_{i=1}^{n} 
    w_i
    =1
    \\
    %%%% box constraints %%%%
    & 
    &&\qquad
    \left| 
      \frac{1}{N} 
      \left( 
      \sum_{i = 1}^{n} 
      w_i
      B_k(X_i)
      -
      \sum_{i=1}^{N} 
      B_k(X_i)
      \right)
    \right|
    \ 
    \le 
    \ 
    \delta_k
    &&&
    \qquad
    \text{for all}\ 
    k \in \left\{ 1, \ldots, N \right\}
    \,.
\end{align*}
\end{fproblem}
This is a (convex) optimization problem. We will talk about the \textbf{objective function} $f$ and
the \textbf{equality} and \textbf{inequality constraints}, especially about the 
\textbf{regression basis} $B$.

\subsection*{Objective Function}
Strictly speaking, we consider the sum
\begin{gather}
  [w_1,\ldots,w_n]^\top
  \ 
  \mapsto
  \ 
  \sum_{i=1}^{n} 
  f(w_i)
\end{gather}
as the objective function. It is natural to consider the dual formulation of the optimization problem. This involves the \textbf{convex conjugate}(cf.Definition~?) of the original objective function. We show in Example that for the sum this is
\begin{gather}
  [\lambda_1,\ldots,\lambda_n]^\top
  \ 
  \mapsto
  \ 
  \sum_{i=1}^{n} 
  f^*(\lambda_i)
\end{gather}
where $f^*$ is the Legendre transformation of $f$.

In the sequel we need $f$ to be strictly convex and its convex conjugate (or Legendre transformation) to be continuously differentiable and strictly non-decreasing.
Two popular choices of $f$ are the \textbf{negative entropy} and the \textbf{sample variance}.
\subsubsection*{Negative Entropy}
We define the negative entropy to be
\begin{gather}
  f
  \colon
  [0,\infty)
  \to
  \R
  ,\quad
  w
  \mapsto
  \begin{cases}
    0\quad\text{if}\ w=0,\\
    w\log w\quad
    \text{else}.
  \end{cases}
\end{gather}
It is strictly convex. To compute its Legendre transformation we note, that
\begin{gather}
  (f^{'})^{-1}
  =
  \lambda\mapsto
  e^{\lambda-1}
\end{gather}
Thus
  \begin{align*}
  f^*
  (\lambda)
  &
  \ 
  =
  \ 
  \lambda
    \cdot
    (f^{'})^{-1}(\lambda)
  \ 
    -
  \ 
    f
    \left( 
      (f^{'})^{-1}(\lambda)
    \right)
    \\
  &
  \ 
  =
  \ 
  \lambda
    \cdot
  e^{\lambda-1}
  \ 
    -
  \ 
  e^{\lambda-1}
  \log
  \left( 
  e^{\lambda-1}
  \right)
  \\
  &
  \ 
  =
  \ 
  e^{\lambda-1}
  \,.
  \end{align*}
  Thus $f^*$ is smooth and strictly non-decreasing.


  \subsubsection*{Sample Variance}
We define the sample variance to be
\begin{gather}
  f
  \colon
  \R
  \to
  \R
  ,\quad
  w
  \mapsto
  (w-1/n)^2
\end{gather}
It is strictly convex. To compute its Legendre transformation we note, that
\begin{gather}
  (f^{'})^{-1}
  =
  \lambda\mapsto
  \frac{\lambda}{2}
  +
  \frac{1}{n}
\end{gather}
Thus
  \begin{align*}
  f^*
  (\lambda)
  &
  \ 
  =
  \ 
  \lambda
    \cdot
    \left( 
  \frac{\lambda}{2}
  +
  \frac{1}{n}
    \right)
  \ 
    -
  \ 
    \left( 
    \left( 
  \frac{\lambda}{2}
  +
  \frac{1}{n}
    \right)
    -
    \frac{1}{n}
    \right)
    ^2
    \\
  &
  \ 
  =
  \ 
  \frac{\lambda^2}{4}
  +
  \frac{\lambda}{n}
  \,.
  \end{align*}
  Thus $f^*$ is smooth.
  To eliminate some variables in the optimization problem,
  we need $f^*$ also to be
  strictly non-decreasing. But the sample variance violates this assumption.

\subsection*{Constraints}
Let's turn our attention to the constraints.
The first constraint makes sure we do not extrapolate from the poputation.
The second constraint norms the weights. 
The third constraint controls the bias of the resulting estimator.
\subsection*{Regression Basis}
We adopt ideas from \cite{Gyorfi2002}. Another angle would be sieve estimates\cite{Newey1997a} where the number of basis functions can grow slower than $N$.
Their notion of (weak) consistency~\cite[Definitien~1.1]{Gyorfi2002} for noiseless estimands
is
\begin{gather}
  \E
  \left[ 
    \int_\mathcal{X}
    \left| 
    \sum_{k=1}^{N} 
    B_k(x)
    \cdot
    m(X_k)
    -
    m(x)
    \right|
    ^2
    \P_X(dx)
  \right]
  \to
  0
  \qquad
  \text{as}
  \ 
  n\to \infty
  \,.
\end{gather}
Universal consistency in this sense holds, if this is true for all distributions with
$\E[m(X)^2]<\infty$(cf.\cite[Definition~1.3]{Gyorfi2002}).

We adopt a slightly different notion of consistency. The next theorem dose the translation work. 
\begin{theorem}
  Assume
  $\E[m(X)^2]<\infty$
  and the basis function are 
(weak) universal consistency in the sense of 
\cite[Definitien~1.3]{Gyorfi2002}.
Then it holds for all $\varepsilon>0$
\begin{gather}
  \P
  \left[ 
    \left| 
    \sum_{k=1}^{N} 
    B_k(X)
    \cdot
    m(X_k)
    -
    m(X)
    \right|
    \ge
    \varepsilon
  \right]
  \to
  0
  \qquad
  \text{as}
  \ 
  n\to \infty
  \,.
\end{gather}
\end{theorem}
\begin{proof}
  By Markov's inequality it holds
  \begin{align*}
    &
  \P
  \left[ 
    \left| 
    \sum_{k=1}^{N} 
    B_k(X)
    \cdot
    m(X_k)
    -
    m(X)
    \right|
    \ge
    \varepsilon
  \right]
  \\
  &
  \ 
  \le
  \ 
  \frac
  {
  \E
  \left[ 
    \left| 
    \sum_{k=1}^{N} 
    B_k(X)
    \cdot
    m(X_k)
    -
    m(X)
    \right|
    ^2
  \right]
  }
  {\varepsilon^2}
  \\
  &
  \ 
  =
  \ 
  \frac
  {
  \E
  \left[ 
    \E
    \left[ 
    \left| 
    \sum_{k=1}^{N} 
    B_k(X)
    \cdot
    m(X_k)
    -
    m(X)
    \right|
    ^2
    |
    X_1,
    \ldots,
    X_N
    \right]
  \right]
  }
  {\varepsilon^2}
  \\
  &
  \ 
  =
  \ 
  \frac
  {
  \E
  \left[ 
    \int_\mathcal{X}
    \left| 
    \sum_{k=1}^{N} 
    B_k(x)
    \cdot
    m(X_k)
    -
    m(x)
    \right|
    ^2
    \P_X(dx)
  \right]
  }
  {\varepsilon^2}
  \,.
  \end{align*}
  The last equality is due to \cite[(1.2)]{Gyorfi2002}.
  By the weak universal consistency of $B$
  the last expression goes to $0$ as $N\to \infty$.
\end{proof}

Classical choices of the basis functions are \textbf{partitioning estimates} and 
\textbf{kernel estimates}(cf.\cite[§4,§5]{Gyorfi2002}).

\subsubsection*{Partitioning Estimates}
We consider a partition
$
  \mathcal{P}_N
  =
  \left\{ 
    A_{N,1}
    ,
    A_{N,2}
    ,
    \ldots
  \right\}
$
of $ \R^d $
and define
$ A_N(x) $ to be the cell of $ \mathcal{P}_N $ containing $x$.
We define $N$ basis functions $B_k$ of the covariates by
\begin{gather*}
  B_k(x)
  :=
  \frac{
  \mathbf{1}_{X_k \in A_N(x)}
  }{
  \sum_{j=1}^{N} 
  \mathbf{1}_{X_j \in A_N(x)}
  }
  \,,
  \qquad
  k=
  1,\ldots,N
  \,.
\end{gather*}
The euclidian norm of the basis functions is bounded above by $1$.
\begin{gather*}
  \norm{B(x)}^2
  =
  \sum_{k=1}^{n} 
  \left( 
  \frac{
  \mathbf{1}_{X_k \in A_n(x)}
  }{
  \sum_{j=1}^{n} 
  \mathbf{1}_{X_j \in A_n(x)}
  }
  \right)
  ^2
  \le
  \sum_{k=1}^{n} 
  \frac{
  \mathbf{1}_{X_k \in A_n(x)}
  }{
  \sum_{j=1}^{n} 
  \mathbf{1}_{X_j \in A_n(x)}
  }
  =1
  \,.
\end{gather*}
Under mild conditions, the basis functions are universally consistent.
\begin{theorem}
  \label{bw:i:bf:pe:th:c}
  If for each sphere $S$ centered at the origin 
  \begin{gather}
    \max
    _
    {
      j\colon
      A_{N,j} 
      \cap
      S
      \neq
      \emptyset
    }
    \mathrm{diam}
    \ 
      A_{N,j} 
      \ 
      \to
      \ 
      0
      \qquad
      \text{for}\ 
      N\to \infty 
  \end{gather}
  and
  \begin{gather}
    \frac
    {
    \#
    \left\{  
      j\colon
      A_{N,j} 
      \cap
      S
      \neq
      \emptyset
    \right\}
    }
    {N}
      \ 
      \to
      \ 
      0
      \qquad
      \text{for}\ 
      N\to \infty 
  \end{gather}
  then the partitioning regression function estimate 
  (definition)
  is
  universally consistent (definition).
\end{theorem}
\begin{proof}
  \cite[Theorem~4.2.]{Gyorfi2002}
\end{proof}
\begin{corollary}
Assume
  $\E[m(X)^2]<\infty$
  and the basis functions $B$ belong to a partitioning estimate.
  Furthermore assume that the conditions of 
  Theorem~\ref{bw:i:bf:pe:th:c} are met.
Then it holds for all $\varepsilon>0$
\begin{gather}
  \P
  \left[ 
    \left| 
    \sum_{k=1}^{N} 
    B_k(X)
    \cdot
    m(X_k)
    -
    m(X)
    \right|
    \ge
    \varepsilon
  \right]
  \to
  0
  \qquad
  \text{as}
  \ 
  n\to \infty
  \,.
\end{gather}
\end{corollary}
\subsubsection*{Kernel Estimates}
Let $K\colon \R^d\to [0,1]$ (bounded kernel) and $h_n>0$ (bandwith).
For examples see \cite[§5.1.]{Gyorfi2002}.
We define
\begin{gather}
  B_k(x)
  :=
  \frac
  {
    K \left( \frac{x-X_k}{h_n} \right)
  }
  {
    \sum_{i=1}^{N} 
    K \left( \frac{x-X_i}{h_n} \right)
  }
  \,.
\end{gather}
By the boundedness of the kernel it follows
$\norm{B(x)}\le 1$.
\begin{theorem}
  \label{bw:i:bf:ke:th:c}
  Assume that there are 
  balls
  $S_{0,r}$ of radius $r$ 
  and
  balls
  $S_{0,R}$ of radius $R$ 
  centered at the origin with $0<r\le R$, and a constant $b>0$ such that
  \begin{gather}
    \mathbf{1}_{\left\{ x\in S_{0,R} \right\}}
    \ge
    K(x)
    \ge
    b
    \cdot
    \mathbf{1}_{\left\{ x\in S_{0,r} \right\}}
  \end{gather}
\end{theorem}
(boxed kernel). Then for bandwiths with $h_n\to0$ and
$n\cdot h_n^d\to\infty$ as $n\to \infty$ the kernel estimate is weakly universally consistent.
\begin{corollary}
Assume
  $\E[m(X)^2]<\infty$
  and the basis functions $B$ belong to a kernel estimate.
  Furthermore assume that the conditions of 
  Theorem~\ref{bw:i:bf:ke:th:c} are met.
Then it holds for all $\varepsilon>0$
\begin{gather}
  \P
  \left[ 
    \left| 
    \sum_{k=1}^{N} 
    B_k(X)
    \cdot
    m(X_k)
    -
    m(X)
    \right|
    \ge
    \varepsilon
  \right]
  \to
  0
  \qquad
  \text{as}
  \ 
  n\to \infty
  \,.
\end{gather}
\end{corollary}

We have gathered all the tools to tackle consistency of the weighted mean.

  \newpage
%  \section{Dual Formulation}
%  We use a hint from the last display of~\cite[p.22]{Wang2019}

%  \newpage
  %\section{Application of Convex Optimization}
  %\subsection*{Introduction}
%\begin{assumption}
%  \begin{enumerate}[label={(\roman*)}]
%    Assume that the map 
%    $
%      f: \R \to \overline{\R}
%    $
%    has the following properties.
%    \item
%      $
%        f 
%        \ 
%        \text{is strictly convex.}
%      $
%    \item
%      $
%        f
%        \ 
%        \text{is lower-semicontinuous and continuously differentiable on}
%        \ 
%        \mathrm{int}(\mathrm{dom}(f))
%        .
%      $
%    \item
%      $
%        \text{The derivative of}\ 
%        f\ 
%        \text{on}\ 
%        \mathrm{int}(\mathrm{dom}(f))
%        \ 
%        \text{is a diffeomorphism.}
%      $
%    \item
%      $
%        \text{The Legendre transformation}
%        \ 
%        f^*
%        \ 
%        \text{of}\ 
%        f
%        \ 
%        \text{is finite}
%        .
%      $
%    \item
%      $
%        \text{The function}\
%        x\mapsto xt - f(x)
%        \ 
%        \text{takes its supremum on}
%        \ 
%        \mathrm{int}(\mathrm{dom}(f))
%        \ 
%        \text{for all}
%        \ 
%        t\in \R.
%      $
%  \end{enumerate}
%\end{assumption}
%Assumption (i),(ii) and (iv) are from \cite{Tseng1991}. See corresponding section in the thesis.
%Assumption (iii) and (v) is needed to get a meaningful analytic expression for the weights after establishing the properties pointed out in \cite{Tseng1991}.
%
%\todo[color=red!40, inline]{Elaborate on assumptions. Give as a counterexample 
%  \[
%  f(x):=
%  (1+\delta_{x\ge 0})
%  (x^2
%  +x\cdot \lambda)
%\]
%}
%
%We consider the following optimization problem.
%\todo[color=orange!40, inline]{Consider Non-negativity constraints only for $f(x)=x\log x$. Incorporating Non-negativity constraint for general $f$ is complicated or impossible. Maybe find counterexample.}
We consider a partition
$
  \mathcal{P}_n
  =
  \left\{ 
    A_{n,1}
    ,
    A_{n,2}
    ,
    \ldots
  \right\}
$
of $ \R^d $
and define
$ A_n(x) $ to be the cell of $ \mathcal{P}_n $ containing $x$.
Next we define $ m_n $ by
\begin{gather}
  m_n(Y|x)
  \ 
  :=
  \ 
  \frac
  {
    \sum_{k=1}^{n} 
    Y_k
    \cdot
    \mathbf{1}
    _
    {
      \left\{ 
      X_k \in A_n(x)
      \right\}
    }
  }
  {
    \sum_{j=1}^{n} 
    \mathbf{1}
    _
    {
      \left\{ 
      X_j \in A_n(x)
      \right\}
    }
  }
  \,.
\end{gather}
In the terminology of \cite[§4]{Gyorfi2002}
$m_n$ is called a partitioning estimate.
We want to control the sumands.
To this end we define a set of basis functions by
\begin{gather}
  B_k(x)
  \ 
  :=
  \ 
  \frac
  {
    \mathbf{1}
    _
    {
      \left\{ 
      X_k \in A_n(x)
      \right\}
    }
  }
  {
    \sum_{j=1}^{n} 
    \mathbf{1}
    _
    {
      \left\{ 
      X_j \in A_n(x)
      \right\}
    }
  }
  \qquad
  \text{for}\ 
  k \in \left\{ 1,\ldots,n \right\}
  \,.
\end{gather}
This yields
\begin{gather}
  m_n(Y|x)
  =
  \sum_{k=1}^{n} 
  Y_k
  \cdot
  B_k(x)
  \,.
\end{gather}
We consider the objective function
\begin{gather}
  f
  \ 
  \colon
  \ 
  [0,\infty)
  \ 
  \to 
  \ 
  \R
  ,
  \quad
  x
  \ 
  \mapsto
  \ 
  x \log x\,,
\end{gather}
together with 
										%%%%%%%%%%%%%
										%%%% (P) %%%%
										%%%%%%%%%%%%%
\begin{fproblem}
  \label{bw:1:primal}
\begin{align*}
  %%%% objective %%%%
    &\underset{w_1, \ldots, w_n \in \R}
    {\text{minimize}}
    &&\qquad\qquad
    \sum_{i = 1}^{n} 
    T_i
    f(w_i)
    &&&
    \\
    %%%% w_i T_i >= 0 %%%%
    &\text{subject to}
    &&\qquad\qquad
    w_i T_i
    \ge
    0
    &&&
    \qquad
    \text{for all}\ 
    i \in \left\{ 1, \ldots, n \right\}
    \,,
    \\
    %%%% 1/n sum w = 1 %%%%
    & 
    &&\qquad\qquad
    \frac{1}{n}
    \sum_{i=1}^{n} 
    T_i w_i
    =1
    \\
    %%%% box constraints %%%%
    & 
    &&\qquad
    \left| 
      \frac{1}{n} 
      \sum_{i = 1}^{n} 
      (
      w_i T_i 
      - 
      1
      )
      \cdot
      B_k(X_i)
    \right|
    \ 
    \le 
    \ 
    \delta_k
    &&&
    \qquad
    \text{for all}\ 
    k \in \left\{ 1, \ldots, n \right\}
    \,.
\end{align*}
\end{fproblem}
%%%%%%%%%%%%%
%%%% (D) %%%%
%%%%%%%%%%%%%
\subsubsection*{Dual Problem}
\begin{ftheorem}
  The dual of Problem~\ref{bw:1:primal} is the unconstrained optimization problem 
  \begin{gather*}
    \underset{\lambda \in \R^n}{\mathrm{minimize}}
    \qquad
    \frac{1}{n}
    \sum_{i = 1}^{n} 
    T_i 
    \cdot
    f^*
    (
      m_n(\lambda|X_i)
    )
    -
      m_n(\lambda|X_i)
    \ 
    +
    \ 
    \inner{\delta}{\left| \lambda \right|}
    \,,
  \end{gather*}
  where
  \begin{gather*}
  f^*
  \,
  \colon
  \, 
  \R
  \ 
  \to
  \ 
  \R
  \,
  ,
  \qquad 
  t 
  \ 
  \mapsto
  \ 
    t\,(f^{'})^{-1}(t)
  \ 
    -
  \ 
    f
    \left( 
      (f^{'})^{-1}(t)
    \right)
  \end{gather*}
  is the Legendre transformation of $f$,
  $
    B(X_i)
    =
    \left[ 
      B_1(X_i)
      ,
      \ldots
      ,
      B_K(X_i)
    \right]
    ^\top
  $
  denotes the $K$ basis functions of the covariates 
  of unit $i\in \left\{ 1, \ldots, n \right\}$
  and
  $
    \left| \lambda \right|
    =
    \left[ 
      \left| \lambda_1 \right|
      ,
      \ldots
      ,
      \left| \lambda_K \right|
    \right]
    ^\top
    ,
  $
  where $\left| \,\cdot\, \right|$
  is the absolute value of a real-valued scalar.
  Moreover, if $\lambda^\dagger$
  is an optimal solution then
  \begin{gather*}
    w_i^*
    \ 
    =
    \ 
    (f^{'})^{-1}
    \left( 
      m_n
      (
      \lambda^\dagger
      |
      X_i
      )
    \right)
    \quad
    \text{for all}\ 
    i
    \ 
    \text{with}\ 
    T_i=1
  \end{gather*}
  are uniquely part of any optimal solution to (P)
  .
\end{ftheorem}
\begin{proof}
  We prove the following Lemma at the end of the section.
  \begin{lemma}
    The dual of the optimization problem is
  \begin{gather*}
    \underset{\lambda \in \R^{2K}}{\mathrm{minimize}}
    \qquad
    \frac{1}{n}
    \sum_{i = 1}^{n} 
    T_i 
    \cdot
    f^*
    (
    \inner{Q_{\bullet i}}{\lambda}
    )
    -
    \inner{Q_{\bullet i}}{\lambda}
    \ 
    +
    \ 
    \inner{d}{\lambda}
  \end{gather*}
  subject to
  \begin{gather*}
    \lambda_k \ge 0
    \quad
    \text{for all}\ 
    k \in \left\{ 1, \ldots, K \right\}
    ,
  \end{gather*}
  where
  \begin{gather*}
    \mathbf{Q}
    :=
    \begin{bmatrix}
     % \mathbf{I}_n\\
     % \mathbf{B}(\mathbf{X})\\
      \pm \, \mathbf{B}(\mathbf{X})
    \end{bmatrix}
    ,
    \qquad
    \mathbf{B}(\mathbf{X})
    :=
    \begin{bmatrix}
      B(X_1), \ldots, B(X_n)
    \end{bmatrix}
    ,
    \qquad
    \text{and}
    \qquad
    d
    :=
    \begin{bmatrix}
      %0_n\\
      \delta \\
      \delta
    \end{bmatrix}
    .
  \end{gather*}

  \end{lemma}
  \begin{proof}
    First we disentangle the constraints.
    To this end, we get
    \begin{gather*}
      \pm\,\sum_{i = 1}^{n} w_i T_i B_k(X_i)
      \ 
      \ge
      \ 
      -n\cdot\delta_k
      \ 
      \pm 
      \ 
      \sum_{i = 1}^{n} B_k(X_i)
      \,,
      \qquad
      k=1,\ldots,K
      \,.
    \end{gather*}
    The corresponding matrix notation is
 \begin{gather*}
    \underset{w_1, \ldots, w_n \in \R}{\mathrm{minimize}}
    \qquad
    \sum_{i = 1}^{n} T_i \cdot f(w_i)
    \\
    \mathbf{Q}w 
    \ 
    \ge
    \ 
    d
    \,,
\end{gather*}
where
\begin{align*}
    T\mathbf{Q}
    &
    \ 
    :=
    \ 
    \begin{bmatrix}
      \mathrm{diag}
      [T_1,\ldots,T_n]
      \\
      \pm
      [T_1,\ldots,T_n]
      \\
      \pm\,T\mathbf{B}(\mathbf{X})
    \end{bmatrix}
    ,
    \\
    T\mathbf{B}(\mathbf{X})
    &
    \ 
    :=
    \ 
    \begin{bmatrix}
      T_1B(X_1), \ldots, T_nB(X_n)
    \end{bmatrix}
    ,
    \\
    d
    &
    \ 
    :=
    \ 
    \begin{bmatrix}
      0_n
      \\
      \pm n
      \\
      -n\cdot\delta 
      \pm\,
      \sum_{i = 1}^{n} B_k(X_i)
    \end{bmatrix}
    \,.
  \end{align*}
The convex conjugate is
\begin{gather*}
  \sum_{T_i=1} T_i f^*(\lambda_i)
  +
  \sum_{T_i=0} 
  \delta_{\left\{ 0 \right\}}(\lambda_i)
  \,,
\end{gather*}
where
\begin{gather*}
  \delta_{\left\{ 0 \right\}}
  (t)
  =
  \begin{cases}
    0,& \text{if}\, t=0,\\
    \infty,& \text{else}\,.
  \end{cases}
\end{gather*}
Note that the $i$-th column of $T\mathbf{Q}$ vanishes if 
$T_i=0$. Likewise, in the columns with $T_i=1$ we can ommit $T_i$.
In the ts chapter, the dual problem features
\begin{gather}
  f^*(A^\top p)
\end{gather}
which by example is here
\begin{gather*}
  \sum_{T_i=1} T_i f^*(T\mathbf{Q}_{\bullet i}^\top\lambda)
  +
  \sum_{T_i=0} 
  \delta_{\left\{ 0 \right\}}
(T\mathbf{Q}_{\bullet i}^\top\lambda)
  =
  \sum_{i=1}^n T_i f^*(\mathbf{Q}_{\bullet i}^\top\lambda)
  \,,
\end{gather*}
where
\begin{align*}
    \mathbf{Q}
    &
    \ 
    :=
    \ 
    \begin{bmatrix}
      \mathbf{I}_n
      \\
      \pm
      \mathrm{1}_n
      \\
      \pm\,\mathbf{B}(\mathbf{X})
    \end{bmatrix}
    \\
    \mathbf{B}(\mathbf{X})
    &
    \ 
    :=
    \ 
    \begin{bmatrix}
      B(X_1), \ldots, B(X_n)
    \end{bmatrix}
\end{align*}


The corresponding dual problem in \cite{Tseng1991} is then
\begin{gather*}
  \underset
  {\lambda_1,\ldots,\lambda_{K}\ge 0}
  {
  \mathrm{maximize}
  }
  \quad
  -
  \sum_{i=1} 
  ^n
  T_i
  \cdot
  f^*
(\mathbf{Q}_{\bullet i}^\top\lambda)
  \ 
  +
  \ 
  \inner{\lambda}{d}
  \,.
\end{gather*}
  $1/n$ the problem remains the same.

  Next we want to remove the non-negativity constraints on $\lambda$.
  To this end we write
  \begin{gather}
    \lambda
    :=
    \begin{bmatrix}
      \rho_1,
      \ldots,
      \rho_n,
      \ 
      \lambda_0^+,
      \lambda_0^-,
      \ 
      \lambda_1^+,
      \ldots,
      \lambda_n^+,
      \ 
      \lambda_1^-,
      \ldots,
      \lambda_n^-
    \end{bmatrix}
    ^\top
    \,.
  \end{gather}
  We expand the objective function $G$ of the dual problem.
  \begin{align*}
    G
    (
    \rho,
    \lambda_0^\pm,
    \lambda^\pm
    )
    =
  &-
  \sum_{i=1} 
  ^n
  T_i
  \cdot
  f^*
  \left( 
\rho_i
\ 
+
\ 
\lambda_0^+
\!
-
\lambda_0^-
\ 
+
\ 
\inner
{B(X_i)}
{\lambda^+ \!- \lambda^-}
  \right)
  \\
  &+
  \ 
  n
  \cdot
  (
\lambda_0^+
\!
-
\lambda_0^-
  )
  \ 
+
  \ 
\inner
{B(X_i)}
{\lambda^+ \!- \lambda^-}
  \ 
-
  \ 
  n
  \cdot
\inner
{\delta}
{\lambda^+ \!+ \lambda^-}
  \end{align*}
  To illustrate the procedure, we show 
  for all $i \in \left\{ 1,\ldots,n \right\}$


\begin{alignat*}{2}
  \text{either}
  &
  &&
  \qquad
      \lambda_i^+ > 0
  \\
  \text{or}
  &
  &&
  \qquad
      \lambda_i^- > 0
  \,.
\end{alignat*}
Assume towards a contradiction that 
there exists
$i \in \left\{ 1,\ldots,n \right\}$
such that
$
      \lambda_0^+ > 0
$
and
$
      \lambda_0^- > 0
$ 
and that 
$\lambda$ is optimal.
Consider
  \begin{gather}
    \tilde{\lambda}
    \ 
    :=
    \ 
    \begin{bmatrix}
      \rho
      ,
      \ 
      \lambda_0^\pm,
      \ 
      \lambda_1^+,
      \ldots,
      \ 
      \lambda_i^+
      \!
      -
      (
      \lambda_i^+
      \!
      \land
      \lambda_i^-
      ),
      \ldots
      \lambda_n^+,
      \ 
      \lambda_1^-,
      \ldots,
      \lambda_i^-
      \!
      -
      (
      \lambda_i^+
      \!
      \land
      \lambda_i^-
      ),
      \ldots,
      \lambda_n^-
    \end{bmatrix}
    ^\top
    \,.
  \end{gather}
  Since 
  $
      \lambda_i^\pm
      -
      (
      \lambda_i^+
      \!
      \land
      \lambda_i^-
      )
      \ge 
      0
  $,
  the perturbed vector $\tilde{\lambda}$ is in the domain of the 
  optimization problem.
  But 
  \begin{align}
  G(\tilde{\lambda})
  -
  G(\lambda)
  \ 
  =
  \ 
  2
  n
  \cdot
  \delta_i
  \cdot
      (
      \lambda_i^+
      \!
      \land
      \lambda_i^-
      )
  \ 
  >
  \ 
  0
  \,,
  \end{align}
  which contradicts the optimality of $\lambda$.
  Likewise we can show
\begin{alignat*}{2}
  \text{either}
  &
  &&
  \qquad
      \lambda_i^+ > 0
  \\
  \text{or}
  &
  &&
  \qquad
      \lambda_i^- > 0
  \,.
\end{alignat*}
But then 
$
\lambda^\pm_i
\ge 0
$
collapses to
$
\lambda_i\in \R
$ 
for 
$i\in \left\{ 0,\ldots,n \right\}$, that is,
$ \lambda_i=\lambda_i^+\!-\lambda_i^- $.
Note that
$ |\lambda_i|=\lambda_i^+\!+\lambda_i^- $.
Likewise we can see, that 
$
\lambda_0
=
\lambda_0^+
-
\lambda_0^-
\in \R
$
removes the constraint on $\lambda_0^\pm$.
Let us take this into account for $G$. We get
  \begin{align*}
    G
    (
    \rho,
    \lambda_0,
    \lambda
    )
    =
  &-
  \sum_{i=1} 
  ^n
  T_i
  \cdot
  f^*
  \left( 
\rho_i
\ 
+
\ 
\lambda_0
\ 
+
\ 
\inner
{B(X_i)}
{\lambda}
  \right)
  \\
  &+
  \ 
  n
  \cdot
\lambda_0
  \ 
+
  \ 
\inner
{B(X_i)}
{\lambda}
  \ 
-
  \ 
  n
  \cdot
\inner
{\delta}
{|\lambda|}
\,.
  \end{align*}
Next we show, that $\rho=0$.
Suppose there exists 
$
i\in \left\{ 1,\ldots, n \right\}
$
such that 
$
\rho_i>0
$
and
$
  T_i
  \cdot
  (f^{'})^{-1}
  \left( 
\rho_i
\ 
+
\ 
\lambda_0
\ 
+
\ 
\inner
{B(X_i)}
{\lambda}
  \right)
  <
  0
$.
It follows
\begin{gather}
  G(0,\lambda_0,\lambda)
  -
  G(\rho_i,\lambda_0,\lambda)
  \ge
  T_i
  \cdot
  (f^{'})^{-1}
  \left( 
\rho_i
\ 
+
\ 
\lambda_0
\ 
+
\ 
\inner
{B(X_i)}
{\lambda}
  \right)
(-\rho_i)
>0,
\end{gather}
which contradicts the optimality of $\lambda$.
Suppose
$
  T_i
  \cdot
  (f^{'})^{-1}
  \left( 
\rho_i
\ 
+
\ 
\lambda_0
\ 
+
\ 
\inner
{B(X_i)}
{\lambda}
  \right)
  >
  0
$.
Then the claim yields to a perturbation argument as in ts.
Thus
To eliminate the constraints for $\rho$ 
we use a similar argument as in the complementary slackness
section of the ts chapter.
Thus we have complementary slackness of 
$\rho_i$ and
$
  T_i
  \cdot
  (f^{'})^{-1}
  \left( 
\rho_i
\ 
+
\ 
\lambda_0
\ 
+
\ 
\inner
{B(X_i)}
{\lambda}
  \right)
$.
But then
every
optimal solution $\lambda$ remains optimal by taking $\rho=0$.

Dividing the optimization problem by $n$ and reversing it, we get

\begin{gather*}
  \underset
  {\lambda_0,\ldots,\lambda_{n}\in \R}
  {
    \mathrm{minimize}
  }
  \quad
  \frac{1}{n}
\sum_{i=1} 
  ^n
  \left[ 
  T_i
  \cdot
  f^*
  \left( 
\lambda_0
\ 
+
\ 
\inner
{B(X_i)}
{\lambda}
  \right)
  -
\inner
{B(X_i)}
{\lambda}
  \right]
\ 
-
\lambda_0
  \ 
+
\inner
{\delta}
{|\lambda|}
  \,.
\end{gather*}
  \end{proof}

  \todo[color=red!40, inline]{What to do about intercept $\lambda_0$?}
%%%%%%%%%%%
%%%% λ %%%%
%%%%%%%%%%%
\subsection*{Consistency of the dual variables}
%%%%%%%%%%%
%%%% ω %%%%
%%%%%%%%%%%
\subsection*{Consistency of the weights}
%%%%%%%%%%%
%%%% E[Y] %%%%
%%%%%%%%%%%
\subsection*{Consistency of the weighted mean}
\end{proof}


  \chapter{Consistency}
  Throughout this section assume the existence of an 
optimal solution 
$(\lambda^\dagger,\lambda_0^\dagger)$.
We use a hint from the last display of~\cite[p.22]{Wang2019}.
The high-level idea is, 
to connect the optimality of a dual solution 
to 
being in the 
neighborhood of an oracle parameter
by looking at the objective function of the dual.
We deliver the omitted technical details.
\subsection*{Neighbourhood of Oracle Parameter}
Let $\lambda^*$ denote the vector with coordinates
\begin{gather}
  \lambda^*_i
  :=
  f^{'}(1/\pi_i)
  -
  \lambda^\dagger_0
  \,,
\end{gather}
where $\pi_i=\P[T_i=1|X_i]$ is the \textbf{propensity score} of the 
$i$-th unit.

\begin{theorem}
  \label{bw:cd:th}
  For all
  $\varepsilon>0$ it holds
  \begin{gather}
    \P
    \left[ 
    \norm{
      \lambda^\dagger
      -
      \lambda^*
    }
    \ge
    \varepsilon
    \right]
    \ 
    \to
    \ 
    0
    \qquad
    \text{for}\ 
    N
    \to 
    \infty
    \,.
  \end{gather}
\end{theorem}
We want to leverage the convexity of
the objective function of the dual to get
 \begin{gather*}
   \P
   \left[ 
     \norm
     {
      \lambda ^ \dagger
      -
      \lambda^*
     }
     \le
     \varepsilon
   \right]
   =
   \P
   \left[ 
     \inf _ { 
       \norm{
         (
     \Delta
     ,
     \Delta_0
         )
 } 
= \varepsilon }
     G
     (
     \lambda^*
      +
      \Delta
      ,
      \lambda^\dagger_0
      +
     \Delta_0
     )
     -
     G
     (
     \lambda^*,
      \lambda^\dagger_0
     )
     \ge 
     0
   \right]
   \,.
 \end{gather*}
We learned about a similar idea from \cite[p.22]{Wang2019}. The next Lemma makes this rigorous. 
\begin{lemma}
  \label{bw:cd:lem}
  Let $m\in\mathbb{N}$ and
  $g \,:\, \R^m \to \overline{\R}$ 
  be convex.
  Then 
  for all $y \in \R^m$ and $\varepsilon>0$ 
    \begin{gather}
      \label{7060_0}
      \inf_{\norm{\Delta}=\varepsilon} g(y+\Delta) - g(y) \ge 0 \quad
    \end{gather}
    implies
    the existence of  
    a global minimum
    $
    y^* \in \,\R^m
    $
    of $g$
    satisfying
    $
      \norm{y^* - y} \le \varepsilon
    $.
\end{lemma}
\begin{proof}
  Since 
  $
  y
  \,
  +
  \,
  \varepsilon
  B
  $
  is convex, it contains a 
  local minimum  
  of $g$.
  Suppose towards a contradiction that
  $
    y^* 
    \ 
    \in 
    \ 
  y
  \,
  +
  \,
  \varepsilon
  B
  $
  is a local minimum, but not a global one, and
  \eqref{7060_0} is true.
  Then it holds
  \begin{gather}
    \label{7060_3}
    g(x) < g(y^*)
    \quad
    \text{for some}\ 
    x 
    \in 
    \R^m 
    \setminus 
    \left( 
  y
  \,
  +
  \,
  \varepsilon
  B
    \right)
  \,.
  \end{gather}
  Furthermore, since 
  $
  y
  \,
  +
  \,
  \varepsilon
  B
  $ is compact and contains $y^*$,
  the line segment connecting 
  $y^*$ and $x$
  intersects the boundary of 
  $y + \mathcal{C}$, that is,
  there exist
  $
    \theta \in (0,1)
  $
  and 
  $
    \Delta_x
  $
  with 
  $
    \norm{\Delta_x}=\varepsilon
  $
  such that
  \begin{gather}
    \label{7060_4}
    \theta x + (1 - \theta) y^* = y + \Delta_x
    \,.
  \end{gather}
    It follows
    \begin{align}
      \label{7060_5}
      \begin{split}
      g(y^*)
      \le
      g(y)
      \le
      g(y + \Delta_x)
      &=
      g(
        \theta x + (1 - \theta) y^*
      )
      \\
      &\le
      \theta g(x)
      + 
      (1 - \theta)
      g(y^*)
      <
      g(y^*)
      ,
      \end{split}
    \end{align}
    which is a contradiction.
    The first inequality is due to
    $y^*$ being a local minimum of $g$ in
    $
  y
  \,
  +
  \,
  \varepsilon
  B
    $,
    the second inequality is due to  
    \eqref{7060_0} being true,
    the equality is due to \eqref{7060_4},
    the third inequality is due to the convexity of $g$
    and the strict inequality is due to \eqref{7060_3}.
    Thus every local minimum of $g$ in
    $
  y
  \,
  +
  \,
  \varepsilon
  B
    $
    is also a global minimum.
    %It follows the right-hand side of \eqref{7060_0}.
\end{proof}
\begin{proof}
  The objective function $G$ of the dual satisfies
\begin{gather*}
  G(\lambda,\lambda_0)
  \ 
  :=
  \ 
    \frac{1}{N}
\sum_{i=1} 
  ^N
  \left[ 
    \,
  T_i
  \cdot
  f^*
  \!
  \left( 
\lambda_0
+
\inner
{B(X_i)}
{\lambda}
  \right)
  \ 
-
\ 
  \left( 
\lambda_0
+
\inner
{B(X_i)}
{\lambda}
  \right)
  \,
  \right]
  \ 
+
\ 
\inner
{\delta}
{|\lambda|}
\,.
\end{gather*}
Without the last term, this is a differentiable convex function.

It is well know that a differentiable convex functions $g$ satisfies
  \begin{gather}
    \label{cv:ts:concD}
    g(x)-g(y)
    \ge
    \nabla
    g(y)^\top
    (x-y)
    \qquad 
    \text{for all}\ 
    x,y\,.
  \end{gather}
  The gradient of
  \begin{gather}
    g := 
    (\lambda,\lambda_0)
    \mapsto
    \frac{1}{N}
\sum_{i=1} 
  ^N
  \left[ 
    \,
  T_i
  \cdot
  f^*
  \!
  \left( 
\lambda_0
+
\inner
{B(X_i)}
{\lambda}
  \right)
  \ 
-
\ 
  \left( 
\lambda_0
+
\inner
{B(X_i)}
{\lambda}
  \right)
  \,
  \right]
  \end{gather}
  is 

  \begin{gather}
    \nabla
    g
    =
    (\lambda,\lambda_0)
    \mapsto
\frac{1}{N}
\sum_{i=1} 
  ^N
  \left[ 
    \,
  T_i
  \cdot
  (f^{'})^{-1}
  \!
  \left( 
\lambda_0
+
\inner
{B(X_i)}
{\lambda}
  \right)
  \ 
-
\ 
1
  \right]
  [
  B(X_i)^\top
  ,
  1
  ]^\top
  \end{gather}
  Thus
\begin{align}
  \label{c:1}
  \begin{split}
     &
   G
     (
     \lambda^*
      +
      \Delta
      ,
      \lambda^\dagger_0
      +
     \Delta_0
     )
     \ 
     -
     \ 
     G
     (
     \lambda^*,
      \lambda^\dagger_0
     )
         %%%%%%%%%%%%% 1 %%%%%%%%%%%%%%
     \\
     &
     \quad
     \ge
     -
     \frac{1}{N}
     \sum_{i=1}^{N} 
     \left[ 
       B(X_i)^\top,
       1
     \right]
     \cdot
     \begin{bmatrix}
       \Delta\\
       \Delta_0
     \end{bmatrix}
     \left( 
       1
       \ 
       -
       \ 
     T_i
     \cdot
     (f^{'})^{-1}
     \left( 
       \inner
       {B(X_i)}
       {\lambda^*}
       +
      \lambda^\dagger_0
     \right)
     \right)
     \\
     &
     \qquad
     +
     \ 
     \inner
     {\delta}
     {
       |\lambda^*+\Delta|
       -
       |\lambda^*|
     }
     \,.
   \end{split}
\end{align}
We fix $
\tilde{\varepsilon}
>0
$
and establish the lower bound
$
-
\tilde{\varepsilon}
$
with probability going to $1$ as $N\to\infty$.
We control the \textbf{first term} by (what?) 
and the \textbf{second term} by $\norm{\delta}$.
\subsection*{First Term}
We note, that by $\norm{B(x)}\le 1$ and the Cauchy-Schwarz inequality
it holds
\begin{gather}
  \label{c:first:0}
      \left[ 
       B(X_i)^\top,
       1
     \right]
     \cdot
     \begin{bmatrix}
       \Delta\\
       \Delta_0
     \end{bmatrix}
     \ 
     \lesssim
     \ 
     \norm{(\Delta,\Delta_0)}
     =\varepsilon
     \,.
\end{gather}
Next, we see that
\begin{align}
  \label{c:first:1}
  \begin{split}
  &
     \frac{1}{N}
     \sum_{i=1}^{N} 
     \left( 
       1
       \ 
       -
       \ 
     T_i
     \cdot
     (f^{'})^{-1}
     \left( 
       \inner
       {B(X_i)}
       {\lambda^*}
       +
      \lambda^\dagger_0
     \right)
     \right)
     \\
     &
     \ 
     \lesssim
     \ 
     \frac{1}{N}
     \sum_{i=1}^{N} 
     \left|
     1
     -
     \frac{T_i}{\pi_i}
     \right|
      \ 
     +
      \ 
     \frac{1}{N}
     \sum_{i=1}^{N} 
     \left| 
        \inner
       {B(X_i)}
       {\lambda^*}
       +
      \lambda^\dagger_0
      \ 
        -
        \ 
        f^{'}
        \left( 
          \frac{1}{\pi_i}
        \right)
     \right|
     \\
     &
     \ 
     =:
     \ 
     S_N
     \
     +
     \ 
     M_N
     \,.
\end{split}
\end{align}
With $
\tilde{\varepsilon}
>0
$
fixed previously,
we want to
establish the upper bound
$
\tilde{\varepsilon}
/
(2\varepsilon)
$
with probability going to $1$ as $N\to\infty$.

First, we bound $S_N$.
By the properties of conditional expectation it holds
\begin{gather*}
  \E
  \left[ 
    \frac{T}{\pi(X)}
  \right]
  =
  \E
  \left[ 
    \frac{\E[T|X]}{\pi(X)}
  \right]
  =1
  \,.
\end{gather*}
By the weak law of large numbers (L1 version ? some assumption on 1/pi?)
\begin{gather}
  \P
  \left[ 
    S_N
    \ge
\tilde{\varepsilon}
/(4\varepsilon)
  \right]
  \ 
  \to 
  \ 
  0
  \qquad 
  \text{for}\ 
  N\to \infty
  \,.
\end{gather}
Next, we bound $M_N$.
Recall that $\sum_{k=1}^{N}B_k(x)=1$. Thus
\begin{gather*}
        \inner
       {B(X)}
       {\lambda^*}
      \ 
       +
      \ 
      \lambda^\dagger_0
      \ 
      =
      \ 
      \sum_{k=1}^{N} 
      B_k(X)
      \left( 
        f^{'}
        \left( 
          \frac{1}{\pi_k}
      \right)
      -
      \lambda_0^\dagger
      \right)
      \ 
      +
      \ 
      \lambda_0^\dagger
      \ 
      =
      \ 
      \sum_{k=1}^{N} 
      B_k(X)
      \cdot
        f^{'}
        \left( 
          \frac{1}{\pi_k}
      \right)
      \,.
\end{gather*}
By Markov's inequality it holds
\begin{align*}
  &
  \P \left[ 
    M_N \ge 
\tilde{\varepsilon}
/(4\varepsilon)
  \right]
  \\
  &
  \ 
  \le
  \ 
  \frac{4\varepsilon}{
\tilde{\varepsilon}
  }
  \, 
  \frac{1}{N}
  \sum_{i=1}^{N} 
  \E
  \left[ 
    \left| 
      \sum_{k=1}^{N} 
      B_k(X_i)
      \cdot
        f^{'}
        \left( 
          \frac{1}{\pi_k}
      \right)
      -
        f^{'}
        \left( 
          \frac{1}{\pi_i}
      \right)
    \right|
  \right]
  \\
  &
  \ 
  \le
  \ 
  \frac{4\varepsilon}{
\tilde{\varepsilon}
  }
  \, 
  \E
  \left[ 
    \left| 
      \sum_{k=1}^{N} 
      B_k(X)
      \cdot
        f^{'}
        \left( 
          \frac{1}{\pi_k}
      \right)
      -
        f^{'}
        \left( 
          \frac{1}{\pi(X)}
      \right)
    \right|
  \right]
  \\
  &
  \ 
  \le
  \ 
  \frac{4\varepsilon}{
\tilde{\varepsilon}
}
  \, 
  \E
  \left[ 
    \left| 
      \sum_{k=1}^{N} 
      B_k(X)
      \cdot
        f^{'}
        \left( 
          \frac{1}{\pi_k}
      \right)
      -
        f^{'}
        \left( 
          \frac{1}{\pi(X)}
      \right)
    \right|
    ^2
  \right]
  ^{1/2}
  \ 
  \to 
  \ 
  0
  \qquad 
  \text{for}\ 
  N\to \infty
  \,.
\end{align*}
The convergence is due to the universal consistency of $B$.
This establishes the desired bound of 
$
\tilde{\varepsilon}/(2\varepsilon)
$
in \eqref{c:first:1}.
Together with \eqref{c:first:0}
we conclude that the \textbf{first term} 
in
\eqref{c:1}
is bounded below by
$
-
\tilde{\varepsilon}/2
$
with probability going to $1$ as $N\to\infty$.
\subsection*{Second Term}
It holds
\begin{gather*}
  |x+y|-|x|\ge
  -|y|
  \qquad
  \text{for all}\ 
  x,y
  \,.
\end{gather*}
Since
$\delta\ge 0$
we get
\begin{align*}
  &
     \inner
     {\delta}
     {
       |\lambda^*+\Delta|
       -
       |\lambda^*|
     }
     \\
     &
     \ 
     \ge
     \ 
     -
     \inner{\delta}
     {|\Delta|}
     \ 
     \ge
     \ 
     -
     \norm{\delta}
     \norm{\Delta}
     \ 
     \ge
     \ 
     -
     \norm{\delta}
     \norm{(\Delta,\Delta_0)}
     \ 
     \ge
     \ 
     -
     \norm{\delta}
     \varepsilon
     \ 
     \ge
     \ 
     -
     \tilde{\varepsilon}/2
     \,,
\end{align*}
with probability going to $1$ as $N\to \infty$.
The convergence is due to $\norm{\delta}$ converging to $0$ in probability.
\subsection*{Conclusion}
With the analysis of the \textbf{first} and \textbf{second term} in
\eqref{c:1} we conclude
\begin{gather}
  G
     (
     \lambda^*
      +
      \Delta
      ,
      \lambda^\dagger_0
      +
     \Delta_0
     )
     \ 
     -
     \ 
     G
     (
     \lambda^*,
      \lambda^\dagger_0
     )
     \ge
     -
     \tilde{\varepsilon}
\end{gather}
with probability going to $1$ as $N\to \infty$.
Since this holds true for all $\tilde{\varepsilon}>0$ we get
\begin{gather}
  G
     (
     \lambda^*
      +
      \Delta
      ,
      \lambda^\dagger_0
      +
     \Delta_0
     )
     \ 
     -
     \ 
     G
     (
     \lambda^*,
      \lambda^\dagger_0
     )
     \ge
     0
\end{gather}
with probability going to $1$ as $N\to \infty$.
But this holds for all 
$
(\Delta,\Delta_0)
$
with 
$
\norm{
(\Delta,\Delta_0)
}
=\varepsilon
$. Thus
\begin{gather}
   \inf _ { 
       \norm{
         (
     \Delta
     ,
     \Delta_0
         )
 } 
= \varepsilon }
     G
     (
     \lambda^*
      +
      \Delta
      ,
      \lambda^\dagger_0
      +
     \Delta_0
     )
     -
     G
     (
     \lambda^*,
      \lambda^\dagger_0
     )
     \ge 
     0
\end{gather}
with probability going to $1$ as $N\to \infty$.
Thus by Lemma~\ref{bw:cd:lem}
\begin{gather}
  \P
    \left[ 
    \norm{
      \lambda^\dagger
      -
      \lambda^*
    }
    \ge
    \varepsilon
    \right]
    \ 
    \to
    \ 
    0
    \qquad
    \text{for}\ 
    N
    \to 
    \infty
    \,.
\end{gather}
Finally, note that this holds for all $\varepsilon>0$. This finishes the proof.
\end{proof}

%  \subsection*{Consistency of the Dual}
%  Let $\lambda^*$ denote the vector with coordinates
\begin{gather}
  \lambda^*_i
  :=
  f^{'}(1/\pi_i)
  -
  \lambda^\dagger_0
\end{gather}

\begin{theorem}
  \label{bw:cd:th}
  For all
  $\varepsilon>0$ it holds
  \begin{gather}
    \P
    \left[ 
    \norm{
      \lambda^\dagger
      -
      \lambda^*
    }
    \ge
    \varepsilon
    \right]
    \ 
    \to
    \ 
    0
    \qquad
    \text{for}\ 
    N
    \to 
    \infty
    \,.
  \end{gather}
\end{theorem}

The following Lemma allows us to leverage the convexity of
the objective function of the dual to get
 \begin{gather*}
   \P
   \left[ 
     \norm
     {
      \lambda ^ \dagger
      -
      \lambda^*
     }
     \le
     \varepsilon
   \right]
   =
   \P
   \left[ 
     \inf _ { 
       \norm{
         (
     \Delta
     ,
     \Delta_0
         )
 } 
= \varepsilon }
     G
     (
     \lambda^*
      +
      \Delta
      ,
      \lambda^\dagger_0
      +
     \Delta_0
     )
     -
     G
     (
     \lambda^*,
      \lambda^\dagger_0
     )
     \ge 
     0
   \right]
   \,.
 \end{gather*}
 

\begin{lemma}
  \label{bw:cd:lem}
  Let $m\in\mathbb{N}$ and
  $g \,:\, \R^m \to \overline{\R}$ 
  be convex.
  Then 
  for all $y \in \R^m$ and $\varepsilon>0$ 
    \begin{gather}
      \label{7060_0}
      \inf_{\norm{\Delta}=\varepsilon} g(y+\Delta) - g(y) \ge 0 \quad
    \end{gather}
    implies
    the existence of  
    a global minimum
    $
    y^* \in \,\R^m
    $
    of $g$
    satisfying
    $
      \norm{y^* - y} \le \varepsilon
    $.
\end{lemma}
\begin{proof}
  Since 
  $
  y
  \,
  +
  \,
  \varepsilon
  B
  $
  is convex, it contains a 
  local minimum  
  of $g$.
  Suppose towards a contradiction that
  $
    y^* 
    \ 
    \in 
    \ 
  y
  \,
  +
  \,
  \varepsilon
  B
  $
  is a local minimum, but not a global one, and
  \eqref{7060_0} is true.
  Then it holds
  \begin{gather}
    \label{7060_3}
    g(x) < g(y^*)
    \quad
    \text{for some}\ 
    x 
    \in 
    \R^m 
    \setminus 
    \left( 
  y
  \,
  +
  \,
  \varepsilon
  B
    \right)
  \,.
  \end{gather}
  Furthermore, since 
  $
  y
  \,
  +
  \,
  \varepsilon
  B
  $ is compact and contains $y^*$,
  the line segment connecting 
  $y^*$ and $x$
  intersects the boundary of 
  $y + \mathcal{C}$, that is,
  there exist
  $
    \theta \in (0,1)
  $
  and 
  $
    \Delta_x
  $
  with 
  $
    \norm{\Delta_x}=\varepsilon
  $
  such that
  \begin{gather}
    \label{7060_4}
    \theta x + (1 - \theta) y^* = y + \Delta_x
    \,.
  \end{gather}
    It follows
    \begin{align}
      \label{7060_5}
      \begin{split}
      g(y^*)
      \le
      g(y)
      \le
      g(y + \Delta_x)
      &=
      g(
        \theta x + (1 - \theta) y^*
      )
      \\
      &\le
      \theta g(x)
      + 
      (1 - \theta)
      g(y^*)
      <
      g(y^*)
      ,
      \end{split}
    \end{align}
    which is a contradiction.
    The first inequality is due to
    $y^*$ being a local minimum of $g$ in
    $
  y
  \,
  +
  \,
  \varepsilon
  B
    $,
    the second inequality is due to  
    \eqref{7060_0} being true,
    the equality is due to \eqref{7060_4},
    the third inequality is due to the convexity of $g$
    and the strict inequality is due to \eqref{7060_3}.
    Thus every local minimum of $g$ in
    $
  y
  \,
  +
  \,
  \varepsilon
  B
    $
    is also a global minimum.
    %It follows the right-hand side of \eqref{7060_0}.
\end{proof}

\begin{proof}
  \emph{
    (
  Theorem~\ref{bw:cd:th}
    )
  }
 We separate the differentiable part in $G$ to get

 \begin{align*}
   &
   G
     (
     \lambda^*
      +
      \Delta
      ,
      \lambda^\dagger_0
      +
     \Delta_0
     )
     -
     G
     (
     \lambda^*,
      \lambda^\dagger_0
     )
         %%%%%%%%%%%%% 1 %%%%%%%%%%%%%%
     \\
     &
     \quad
     \ge
     -
     \frac{1}{N}
     \sum_{i=1}^{N} 
     \left[ 
       B(X_i)^\top,
       1
     \right]
     \cdot
     \begin{bmatrix}
       \Delta\\
       \Delta_0
     \end{bmatrix}
     \left( 
       1
       \ 
       -
       \ 
     T_i
     \cdot
     (f^{'})^{-1}
     \left( 
       \inner
       {B(X_i)}
       {\lambda^*}
       +
      \lambda^\dagger_0
     \right)
     \right)
     \\
     &
     \qquad
     +
     \ 
     \inner
     {\delta}
     {
       |\lambda^*+\Delta|
       -
       |\lambda^*|
     }
     %%%% 2 %%%%
     \\
     &
     \quad
     \ge
     \ 
     -
       \norm{
         (
     \Delta
     ,
     \Delta_0
         )
 } 
     \left( 
     \norm{B(X_i)}
     \cdot
     \frac{1}{N}
     \sum_{i=1}^{N} 
     \left| 
       1
       \ 
       -
       \ 
     T_i
     \cdot
     (f^{'})^{-1}
     \left( 
       \inner
       {B(X_i)}
       {\lambda^*}
       +
      \lambda^\dagger_0
     \right)
     \right|
     \ 
     +
     \ 
     \norm{\delta}
     \right)
     %%%%% 2 %%%%
     \\
     &
     \quad
     \ge
     \ 
     -
     \varepsilon
     \left( 
     \frac{1}{N}
     \sum_{i=1}^{N} 
     \left| 
       1
       \ 
       -
       \ 
     T_i
     \cdot
     (f^{'})^{-1}
     \left( 
       \inner
       {B(X_i)}
       {\lambda^*}
       +
      \lambda^\dagger_0
     \right)
     \right|
     \ 
     +
     \ 
     \norm{\delta}
     \right)
     \ 
     =:
     -\varepsilon
     (S+\norm{\delta})
 \end{align*}
 \subsubsection*{Analysis of $S$}
 By the triangle inequality we get
 \begin{gather}
  S
  \le
  \frac{1}{N}
  \sum_{i=1}^{N} 
  |
  1
  -
  T_i
  /
  \pi_i
  |
  +
  \max_{i=1,\ldots,n}
     \left| 
     (f^{'})^{-1}
     \left( 
       \inner
       {B(X_i)}
       {\lambda^*}
       +
      \lambda^\dagger_0
     \right)
     \ 
      -
      \ 
      1/\pi_i
     \right|
     \\
     =:
     S_1
     +
     M
 \end{gather}

 \subsubsection*{Analysis of $S_1$}
 Since
 $X_i$ and $T_i$ are i.i.d. 
 and
 \begin{gather}
   \E[
  T_i
  /
  \pi_i
   ]
   =
   \E[
   \E[
  T_i
  |X_i
   ]
  /
  \pi_i
   ]
   =1
 \end{gather}
 it holds by the weak law of large numbers
 \begin{gather}
   \P[S_1\le 
   \tilde{\varepsilon}/(4\varepsilon)
   ]\to 1
   \quad
   \text{for}\ 
   n\to \infty
 \end{gather}
 \subsubsection*{Analysis of $M$}
 Since  
 \begin{gather}
   \inner{B(X_i)}{\mathrm{1_N}}=1
 \end{gather}
 and
 \begin{gather}
   \lambda^*_i
   =
  f^{'}(1/\pi_i)
  -
  \lambda^\dagger_0
 \end{gather}
 it holds
 \begin{gather}
       \inner
       {B(X_i)}
       {\lambda^*}
       +
      \lambda^\dagger_0
     =
    \sum_{k=1}^{N}  
  f^{'}(1/\pi_i)
  \cdot
       B_k(X_i)
 \end{gather}

By the universal consistency of $B$
this converges to $
f^{'}
(
1/\pi_i
)
$ in probability.
By the continuity of 
$
(
f^{'}
)^{-1}
$
it follows

 \begin{gather}
   \P[M\le 
   \tilde{\varepsilon}/(4\varepsilon)
   ]\to 1
   \quad
   \text{for}\ 
   n\to \infty
 \end{gather}
 \subsubsection*{Conclusion}
 It follows
 \begin{gather}
   \P[S\le
   \tilde{\varepsilon}/(2\varepsilon)
   ]\to 1
   \quad
   \text{for}\ 
   n\to \infty
 \end{gather}
We get
for all $\tilde{\varepsilon}>0$
\begin{gather}
   G
     (
     \lambda^*
      +
      \Delta
      ,
      \lambda^\dagger_0
      +
     \Delta_0
     )
     -
     G
     (
     \lambda^*,
      \lambda^\dagger_0
     )
     \ge
     -
     \tilde{\varepsilon}
\end{gather}
with probability going to 1 for $n\to \infty$.
Thus it also holds
\begin{gather}
  \P
  \left[ 
   G
     (
     \lambda^*
      +
      \Delta
      ,
      \lambda^\dagger_0
      +
     \Delta_0
     )
     -
     G
     (
     \lambda^*,
      \lambda^\dagger_0
     )
     \ge
     0
  \right]
  \to
  0
  \qquad
  \text{for}
  \ 
  n\to \infty
  \,.
\end{gather}
Applying Lemma~\ref{bw:cd:lem} finishes the proof.



\end{proof}


%  \subsection*{Consistency of the Primal}
%  \begin{theorem}
  \label{bw:cp:th}
  For all
  $i\in \left\{ 1,\ldots,n \right\}$
  and all
  $\varepsilon>0$ it holds
  \begin{gather}
    \P
    \left[ 
   \left| 
   1 / \pi_i
   -
   w_i
   \right|
    \ge
    \varepsilon
    \right]
    \ 
    \to
    \ 
    0
    \qquad
    \text{for}\ 
    N
    \to 
    \infty
    \,.
  \end{gather}
\end{theorem}

\begin{proof}
  It holds
 \begin{align*}
   \left| 
   1 / \pi_i
   -
   w_i
   \right|
   & 
   \ 
   =
   \ 
   \left| 
   1 / \pi_i
   -
   (
      f ^ { ' }
   )
   ^ {-1}
   \left( 
   \inner
   { B(X_i)}
   { \lambda ^ \dagger }
   +
   \lambda^\dagger_0
   \right)
   \right|
   %%%% 1 %%%%
   \\
   & 
   \ 
   \le
   \ 
   \left| 
   1 / \pi_i
   -
   (
      f ^ { ' }
   )
   ^ {-1}
   \left( 
   \inner
   { B(X_i)}
   { \lambda ^ * }
   +
   \lambda^\dagger_0
   \right)
   \right|
      %%%% 1 a %%%%
   \\
   &
   \qquad
   +
   \ 
\left| 
   (
      f ^ { ' }
   )
   ^ {-1}
   \left( 
   \inner
   { B(X_i)}
   { \lambda ^ \dagger }
   +
   \lambda^\dagger_0
   \right)
   -
   (
      f ^ { ' }
   )
   ^ {-1}
   \left( 
   \inner
   { B(X_i)}
   { \lambda ^ *}
   +
   \lambda^\dagger_0
   \right)
   \right|
      %%%% 2 %%%%
%   \\
%   &
%   \ 
%   =:
%   I_1 +
%   I_2
 \end{align*}
 By the universal consistency of $B$ and the continuity of $(f^{'})^{-1}$ the first term converges to $0$ in probability.
 By the continuity of $(f^{'})^{-1}$,
 the uniform boundedness of $\norm{B}$ and the dual consistency
 the second term converges to $0$ in probability.

\end{proof}

%  \subsection*{Consistency of the Weighted Mean}
%  \begin{theorem}
  \label{bw:ce:th}
  For all
  $\varepsilon>0$ it holds
  \begin{gather}
    \P
    \left[ 
  \left| 
    \frac{1}{N}
    \sum_{i=1}^{n} 
    w_i
    Y_i
    -
    \E[Y(1)]
  \right|
    \ge
    \varepsilon
    \right]
    \ 
    \to
    \ 
    0
    \qquad
    \text{for}\ 
    N
    \to 
    \infty
    \,,
  \end{gather}
  that is, the weighted mean is a consistent estimator.
\end{theorem}

\begin{proof}
\begin{align*}
  \left| 
    \frac{1}{N}
    \sum_{i=1}^{n} 
    w_i
    Y_i
    -
    \E[Y(1)]
  \right|
  &
  \ 
  \le
  \ 
  \left|  
  \frac{1}{N}
  \left( 
    \sum_{i=1}^{n} 
    w_i
    B(X_i)
    -
    \sum_{i=1}^{N} 
    B(X_i)
  \right)
  ^\top
     \mathbf{Y}(1) 
  \right|
  %%%% 1 %%%%
  \\
  &
  \qquad
  +
  \ 
  \left|  
  \frac{1}{N}
    \sum_{i=1}^{n} 
    w_i
    \left( 
    \E[Y(1)| X_i]
    -
    \inner
    {B(X_i))}
    { \mathbf{Y}(1) }
    \right)
  \right|
  %%%% 1 %%%%
  \\
  &
  \qquad
  +
  \ 
  \left|  
  \frac{1}{N}
    \sum_{i=1}^{N} 
    \left( 
    \E[Y(1)| X_i]
    -
    \inner
    {B(X_i))}
    { \mathbf{Y}(1) }
    \right)
  \right|
  %%%% 2 %%%%
  \\
  &
  \qquad
  +
  \ 
  \left|  
  \frac{1}{N}
    \sum_{i=1}^{n} 
    (
    w_i 
    -
    1/\pi_i
    )
    \left( 
      Y_i
    -
    \E[Y(1)| X_i]
    \right)
  \right|
  %%%% 3 %%%%
  \\
  &
  \qquad
  +
  \ 
  \left|  
  \frac{1}{N}
    \sum_{i=1}^{N} 
    T_i
    /\pi_i
    \left( 
      Y_i
    -
    \E[Y(1)| X_i]
    \right)
    +
    \left( 
    \E[Y(1)| X_i]
    -
    \E[Y(1)]
    \right)
  \right|
  \\
  %%%%%%%%%%%%%%%
  &
  \ 
  =:
  \ 
  R_1
  +
  R_2
  +
  R_3
  +
  R_4
  +
  R_5
  \end{align*}
  \subsubsection*{Analysis of $R_1$}
  By the Cauchy-Schwarz inequality it holds
  \begin{gather}
    R_1
    \ 
    =
    \ 
   \left|  
  \frac{1}{N}
  \left( 
    \sum_{i=1}^{n} 
    w_i
    B(X_i)
    -
    \sum_{i=1}^{N} 
    B(X_i)
  \right)
  ^\top
     \mathbf{Y}(1) 
  \right|
  \le
  \norm{\delta}
  \norm{
     \mathbf{Y}(1) 
  }
  \end{gather}

  \subsubsection*{Analysis of $R_2$ and $R_3$}
By the universal consistency of $B$ and 
$
  \frac{1}{N}
    \sum_{i=1}^{n} 
    w_i
    =1
$
the terms
$R_2$ and $R_3$ converge to $0$ in probability.
\subsubsection*{Analysis of $R_4$}
By the boundedness of $Y$ and the primal consistency 
$R_4$ 
converges to $0$ in probability.
\subsubsection*{Analysis of $R_5$}
Since the expectation is $0$, the term $R_5$ 
converges to $0$ in probability by the weak law of large numbers.

\end{proof}


%
%  \chapter{Asymptotic Normality and Convergence to Gaussian Bridge}
%  \begin{theorem}
  Under conditions
  the partition estimate has
  \begin{gather}
    \E\norm{m_N-m}^2
    \le
    C_\P
    N^{- \frac{2}{d+2}}
  \end{gather}
\end{theorem}
\begin{theorem}
  Under conditions
  the kernel estimate has
  \begin{gather}
    \E\norm{m_N-m}^2
    \le
    C_\P
    N^{- \frac{2}{d+2}}
  \end{gather}
\end{theorem}

%  \subsection*{Learning Rates for the Dual}
%  \begin{theorem}
  Under conditions
  \begin{gather}
    \P
    \left[ 
      \norm{\lambda^\dagger - \lambda^*}
      \le
      C_\P
      C_\tau
      \varepsilon_n
    \right]
    \ge
    1-\tau
    \,,
  \end{gather}
  where 
  $\varepsilon_n$ is the square root of the basis function Learning rate and $C_\tau$ depends on the Concentration Inequality.
  We need bernstein confidence $\sqrt{\log (1/\tau)}$ to preserve minimal Learning rate for $d=1$.
\end{theorem}

\subsection*{Plan of Proof}
The path is similar to that of the previous chapter. 
We will use strong convexity to obtain a quadratic term.
This gives us the flexibility to obtain learning rates.

\begin{theorem}
  \emph{(Bernstein's inequality)}
  Let
  $
  (\Omega,\mathcal{A},\P)
  $ 
  be a probability space, 
  $
  B>0
  $ 
  and
  $
  \sigma>0
  $
  be real numbers,
  and
  $
  n\ge 1
  $
  be an integer.
  Furthermore, 
  let
  $
  X_1,\ldots,X_n
  :
  \Omega
  \to
  \R
  $
  be independent random variables satisfying
  $
  \E[X_i]=0
  $,
  $
  \norm{X_i}_\infty
  \le
  B
  $
  and
  $
  \E[X_i^2]\le \sigma^2
  $
  for all 
  $
  i=1,\ldots,n
  $.
  Then we have
  \begin{gather*}
   \P
    \left[ 
      \,
      \left| 
      \frac{1}{n}
        \sum_{i=1}^{n} 
        X_i
      \right|
      \ 
      \le
      \ 
      \sqrt{
        \,
        \frac{2\,\sigma^2 \log (e/\tau)}{n}
        \,
      }
      \ 
      +
      \
      \frac{2 B \log(e/\tau)}{3n}
      \,
    \right]
    \ 
    \ge
    \ 
    1-\tau
    \qquad
    \text{for all}\ 
    \tau
    >
    0
    \,.
  \end{gather*}
\end{theorem}
\begin{proof}
  See \cite[Theorem~6.12]{Steinwart2008}
  for the one-sided version. 
  The two-sided version, as stated in the above theorem, is an easy consequence. We omit the details.
\end{proof}




\subsection*{Strong Convexity}
For simpler notation we define $B_0(x):=1$.
The Hessian matrix of $g$ as in ??
is
\begin{gather}
  \nabla^2 
  g(\lambda_0^\dagger,\lambda^\dagger)
  =
  \left[ 
    \frac{1}{N}
    \sum_{i=1}^{n} 
    \left( 
    (
    f^{'}
    )
    ^{-1}
    \right)
    ^{'}
    \left( 
      \lambda_0^\dagger
      +
      \inner{B(X_i)}
      {\lambda^\dagger}
    \right)
    B_k(X_i)
    \cdot
    B_l(X_i)
  \right]
  _{0\le k,l\le N}
  \,.
\end{gather}
Since
$
    \left( 
    (
    f^{'}
    )
    ^{-1}
    \right)
    ^{'}
    >0
$
by the strict convexity of $f$, 
$(\lambda_0^\dagger,\lambda^\dagger)\in \Theta$,
where $\Theta$ is compact parameter space,
and the support of $X$ is compact it holds

\begin{gather}
     \left( 
    (
    f^{'}
    )
    ^{-1}
    \right)
    ^{'}
    \left( 
      \lambda_0^\dagger
      +
      \inner{B(X_i)}
      {\lambda^\dagger}
    \right)
    >
    \frac{1}{C}
    \,.
\end{gather}
Consider
\begin{gather}
  \mathbf{B}(X):=
  \begin{bmatrix}
    1 & B_1(X_1) & \cdots & B_N(X_1)\\
    \vdots & \vdots & \ddots & \vdots \\
    1 & B_1(X_n) & \cdots & B_N(X_n)
  \end{bmatrix}
  \in \R^{n\times (N+1)}
\end{gather}

Since $B_k(x)\ge 0$ for all $k$ it holds

\begin{gather}
  \nabla^2 
  g(\lambda_0^\dagger,\lambda^\dagger)
  \succcurlyeq
  \frac{1}{C}
  N^{-1}
  \mathbf{B}(X)^\top
  \mathbf{B}(X)
  \succcurlyeq
  \frac{\lambda_{\min}(
  N^{-1}
  \mathbf{B}(X)^\top
  \mathbf{B}(X)
  )}{C}
  \cdot
  \mathbf{I}
  \succcurlyeq
  \frac{1}
  {
  C \sqrt{\log(1/\tau)}
}
  \cdot
  \mathbf{I}
\end{gather}
with probability greater than $1-\tau$.
Thus $g$ is strongly convex with parameter 
$
\left(
  C \sqrt{\log(1/\tau)}
\right)^{-1}
$
with probability greater than $1-\tau$.
We analyze
$
1-T_i/\pi_i
$.
It holds
$\E[
1-T_i/\pi_i
]=0$.
Furthermore,
\begin{gather}
  \left| 
1
-
\frac{T_i}{\pi_i}
  \right|
  \ 
\le
  \ 
1
\land
\frac{1-\pi_i}{\pi_i}
  \ 
\le
  \ 
1
+
\frac{1-\pi_i}{\pi_i}
  \ 
=
  \ 
\frac{1}{\pi_i}
  \ 
\le
  \ 
\frac{1}{C_\pi}
\qquad
\text{almost surely.}
\end{gather}
Also
\begin{gather}
  \E
  \left[ 
  \left| 
1
-
\frac{T_i}{\pi_i}
  \right|
  ^2
  \right]
\ 
=
\ 
1
-
2
\E
\left[ 
\frac{T_i}{\pi_i}
\right]
+
\E
\left[ 
\frac{T_i}{\pi_i^2}
\right]
\ 
=
\ 
\E
\left[ 
\frac{1}{\pi_i}
\right]
-1
\ 
\le
\ 
\frac{1}{C_\pi}
\,.
\end{gather}
Thus, by Bernstein's inequality, it holds
\begin{gather}
   \P
    \left[ 
      \,
      \left| 
      \frac{1}{n}
        \sum_{i=1}^{n} 
1
-
\frac{T_i}{\pi_i}
      \right|
      \ 
      \lesssim
      \ 
      \sqrt{
        \frac{
        \log (e/\tau)}
      {N}
      }
    \right]
    \ 
    \ge
    \ 
    1-\tau
    \qquad
    \text{for all}\ 
    \tau
    >
    0
    \,.
\end{gather}

Next, we analyze 
\begin{gather}
  \sum_{k=1}^{N} 
  B_k(X_i)\cdot
  f^{'}\left( \frac{1}{\pi_k} \right)
  -
  f^{'}\left( \frac{1}{\pi_i} \right)
  -
  \E[\cdots]
  \,.
\end{gather}
To this end,
\begin{align*}
  &
  \left| 
  \sum_{k=1}^{N} 
  B_k(X_i)\cdot
  f^{'}\left( \frac{1}{\pi_k} \right)
  -
  f^{'}\left( \frac{1}{\pi_i} \right)
  \right|
  \\
  &
  \ 
  \le
  \ 
  \left| 
  \sum_{k=1}^{N} 
  B_k(X_i)\cdot
  \left( 
  f^{'}\left( \frac{1}{\pi_k} \right)
  -
  f^{'}\left( \frac{1}{\pi_i} \right)
  \right)
  \right|
  \ 
  \le
  \ 
  \sum_{k=1}^{N} 
  B_k(X_i)\cdot
  \left| 
  f^{'}\left( \frac{1}{\pi_k} \right)
  -
  f^{'}\left( \frac{1}{\pi_i} \right)
  \right|
  \\
  &
  \ 
  \le
  \ 
  2
  f^{'}\left( \frac{1}{C_\pi} \right)
  \,.
\end{align*}
Also
\begin{align*}
  &
  \E
  \left[ 
  \left| 
  \sum_{k=1}^{N} 
  B_k(X_i)\cdot
  f^{'}\left( \frac{1}{\pi_k} \right)
  -
  f^{'}\left( \frac{1}{\pi_i} \right)
  -
  \E[\cdots]
  \right|^2
\right]^{1/2}
\\
  &
  \ 
\le
  \ 
  \E
  \left[ 
  \left| 
  \sum_{k=1}^{N} 
  B_k(X_i)\cdot
  f^{'}\left( \frac{1}{\pi_k} \right)
  -
  f^{'}\left( \frac{1}{\pi_i} \right)
  \right|^2
\right]^{1/2}
  \ 
+
  \ 
\left| 
\E
\left[ 
  \sum_{k=1}^{N} 
  B_k(X_i)\cdot
  f^{'}\left( \frac{1}{\pi_k} \right)
  -
  f^{'}\left( \frac{1}{\pi_i} \right)
\right]
\right|
\\
  &
  \ 
\lesssim
  \ 
N^{-1/(d+2)}
  \ 
+
  \ 
  2
  f^{'}\left( \frac{1}{C_\pi} \right)
  \,.
\end{align*}
Thus, by Bernstein's inequality, it holds
\begin{gather}
   \P
    \left[ 
      \,
      \left| 
      \frac{1}{N}
  \sum_{i,k=1}^{N} 
  B_k(X_i)\cdot
  f^{'}\left( \frac{1}{\pi_k} \right)
  -
  f^{'}\left( \frac{1}{\pi_i} \right)
  -
  \E[\cdots]
      \right|
      \ 
      \lesssim
      \ 
      \sqrt{
        \frac{
        \log (e/\tau)}
      {N}
      }
    \right]
    \ 
    \ge
    \ 
    1-\tau
    \qquad
    \text{for all}\ 
    \tau
    >
    0
    \,.
\end{gather}

Similar to the analysis of the first and second term we get

\begin{align}
      G
     (
     \lambda^*
      +
      \Delta
      ,
      \lambda^\dagger_0
      +
     \Delta_0
     )
     \ 
     -
     \ 
     G
     (
     \lambda^*,
      \lambda^\dagger_0
     )
     \ge
     \norm{\Delta}
     \left(
     \frac{\norm{\Delta}}
     {
       C \sqrt{\log(1/\tau)}
     }
     -
       \sqrt{
       \frac{\log(1/\tau)}{N}
       }
       -
       N^{-1/(d+2)}
     \right)
\end{align}
with probability greater than $1-\tau$.
Thus for
\begin{gather}
  \norm{\Delta}
  =
  C
  \log(1/\tau)
  N^{-1/(d+2)}
\end{gather}
it holds
\begin{gather}
       G
     (
     \lambda^*
      +
      \Delta
      ,
      \lambda^\dagger_0
      +
     \Delta_0
     )
     \ 
     -
     \ 
     G
     (
     \lambda^*,
      \lambda^\dagger_0
     )
     \ge
0
\end{gather}
with probability greater than $1-\tau$.
We conclude
  \begin{gather}
    \P
    \left[ 
      \norm{\lambda^\dagger - \lambda^*}
    \lesssim
  \log(1/\tau)
  N^{-1/(d+2)}
    \right]
    \ 
    \ge
    \ 
    1-\tau
    \,.
  \end{gather}

%  \subsection*{Learning Rates for the Primal}
%  \begin{theorem}
  Under conditions
  the weights satisfy
  \begin{gather}
    \E
    [
    |
    w^\dagger(X)
    -
    1/\pi(X)
    |
    ^2
    ]
    ^{1/2}
    \le
    C_\P
  \sqrt{\log(n)}
  n^{-1/(2+d)}
  \end{gather}
  where 
  $varepsilon_n$ depends on the Learning rate of the basis functions
  and the confidence of the dual.
  $C_\P$ depends on the size of the parameter space.
\end{theorem}
\begin{proof}
  \begin{align}
    \E
    [
    |
    w^\dagger(X)
    -
    1/\pi(X)
    |
    ^2
    ]
    ^{1/2}
    &
    \ 
    =
    \ 
    \E
    \left[ 
    \left| 
    (f^{'})^{-1}
    \left( 
      \inner
      {B(X)}
      {\lambda^\dagger}
      +
      \lambda_0^\dagger
    \right)
    -
    1/\pi(X)
    \right|
    ^2
    \right]
    ^{1/2}
    \\
    &
    \ 
    \le
    \ 
    \left| 
    (f^{'})^{-1}
    \right|_L
    \left( 
      I_1 + I_2
    \right)
  \end{align}
  where
  \begin{align}
    I_1
    & 
    \ 
    :=
    \ 
    \left( 
    \E\norm{\lambda^\dagger-\lambda^*}^2
    \right)
    ^{1/2}
    \\
    I_2
    & 
    \ 
    :=
    \ 
      \E
      \left[ 
        \left| 
        \sum_{k=1}^{N} 
        B_k(X)\cdot 
        f^{'}(X_k)
        -
        f^{'}(X)
        \right|
        ^2
      \right]
    ^{1/2}
  \end{align}
  It holds 
  $I_2\le n^{-1/(d+2)}$
  by the lr of the basis.
  To analyse $I_1$ we use the lr of the dual.
  \begin{align}
   I_1
   \le
   C_\tau
   n^{-1/(d+2)}
   +
   \sqrt{\tau}
   \cdot
   \mathrm{diam}\,\Theta
  \end{align}
Note that the Markov confidence $1/\sqrt{\tau}$ is insufficient. 
With Bernstein confidence, bounded diameter and $\tau=n^{-2/(d+2)}$
we get
\begin{gather}
  I_1
  \lesssim
  \log(N)
  \ 
  \cdot
  \ 
  N^{-1/(2+d)}
\end{gather}
Thus
\begin{gather}
    \E
    [
    |
    w^\dagger(X)
    -
    1/\pi(X)
    |
    ^2
    ]
    ^{1/2}
    \ 
  \lesssim
    \ 
  \log(N)
  \ 
  \cdot
  \ 
  N^{-1/(2+d)}
\end{gather}
\end{proof}

%  \subsection*{Asymptotic Normality of the Weighted Mean}
  \subsection*{Gaussian Bridge}
  \begin{theorem}
  \emph{(Slutzky's theorem)}
  Let
  $
  (E,d)
  $
  be a metric space 
  and let
  $X,X_1,X_2,\ldots$
  and
  $Y_1,Y_2,\ldots$
  be random variables with values in $E$.
  Assume
  $X_n\to X$ in distribution and 
  $d(X_n,Y_n)\to 0$ in probability. Then 
  $Y_n\to X$ in distribution.
\end{theorem}
\begin{proof}
  \cite[Theorem~13.8]{Klenke2020}
\end{proof}

\begin{ftheorem}
  Under conditions 
the stochastic process
\begin{gather}
    \sqrt{N}
    \left( 
  \frac{1}{N}
    \sum_{i=1}^{n} 
    w_i^\dagger
    \mathbf{1}
    _{\left\{ Y_i\le z \right\}}
    -
    \P[Y(1)\le z]
    \right)
    _{z\in\R}
    \,.
  \end{gather}
  converges in
  $l^\infty(\R)$
  to a Gaussian process with mean 0 and covariance ??.
\end{ftheorem}
\begin{proof}
  For fixed $z\in\R$ we use the following error decomposition.
  Recall $\pi(x):=\P[T=1|X=x]$ and
  $
  w(x)
  :=
  (
  f^{'}
  )^{-1}
  \left( 
    \inner{B(x)}{\lambda^\dagger}
    +
    \lambda_0^\dagger
  \right)
  $,
  where $(\lambda^\dagger,\lambda_0^\dagger)$ is the optimal dual solution.
  We also write
  $
  F_{Y(1)}(z|x)
  =
  \P[Y(1)\le z|X=x]
  $
  and
  $
  F_{Y(1)}(z)
  =
  \P[Y(1)\le z]
  $.
\begin{align*}
  &
  \sqrt{N}
\left( 
    \frac{1}{N}
    \sum_{i=1}^{n} 
    w(X_i)
    \mathbf{1}_{\left\{ Y_i \le z \right\}}
    \ 
    -
    \ 
    \P[Y(1)\le z]
\right)
    \\
  &
  \ 
  =
  \ 
  \sqrt{N}
  \sum_{k=1}^{N} 
  \left[ 
  \frac{1}{N}
  \left( 
    \sum_{i=1}^{n} 
    w(X_i)
    B_k(X_i)
    -
    \sum_{i=1}^{N} 
    B_k(X_i)
  \right)
  F_{Y(1)}(z|X_k)
  \right]
  %%%% 1 %%%%
  \\
  &
  \qquad
  +
  \ 
  \sqrt{N}
    \sum_{i=1}^{N} 
    \left[ 
  \frac{
    T_i\cdot w(X_i) -1 }{N}
    \left( 
  F_{Y(1)}(z|X_i)
    \ 
    -
    \ 
    \sum_{k=1}^{N} 
    B_k(X_i)
    \cdot
  F_{Y(1)}(z|X_k)
    \right)
    \right]
  %%%% 2 %%%%
  \\
  &
  \qquad
  +
  \ 
  \sqrt{N}
  \left( 
  \frac{1}{N}
    \sum_{i=1}^{N} 
    \left[ 
    T_i
    \left( 
    w(X_i) 
    -
    \frac{1}{\pi(X_i)}
    \right)
    \left( 
    \mathbf{1}_{\left\{ Y_i \le z \right\}}
    -
  F_{Y(1)}(z|X_i)
    \right)
    \right]
  \right)
  %%%% 3 %%%%
  \\
  &
  \qquad
  +
  \ 
  \sqrt{N}
  \left( 
  \frac{1}{N}
    \sum_{i=1}^{N} 
    \frac{T_i}{\pi(X_i)}
    \left( 
    \mathbf{1}_{\left\{ Y_i \le z \right\}}
    -
  F_{Y(1)}(z|X_i)
    \right)
    \ 
    +
    \ 
    \left( 
  F_{Y(1)}(z|X_i)
    -
  F_{Y(1)}(z)
    \right)
  \right)
  \\
  %%%%%%%%%%%%%%%
  &
  \ 
  =:
  \ 
  R_1(z)
  +
  R_2(z)
  +
  R_3(z)
  +
  R_4(z)
  \end{align*}
  We show that $\sup_{z\in\R}\left| R_i(z) \right|\to 0 $ in
  probability for $i=1,2,3$.
  The term $(R_4)_{z\in\R}$ is $\P$-Donsker and determines the covariance of the limiting process.
  \subsection*{Analysis of $R_1$}
  By Theorem~\ref{dual_solution_th}
  it holds $w_i^\dagger=w(X_i)$ for $i\in \left\{ 1,\ldots,n \right\}$, that is, for $i\le n$ we can identify $w(X_i)$ with the optimal
  solution to 
  problem~\ref{bw:1:primal}. 
  Thus the constraints of the problem apply.
  \begin{gather}
      \left| 
      \frac{1}{N} 
      \left( 
      \sum_{i = 1}^{n} 
      w(X_i)
      B_k(X_i)
      -
      \sum_{i=1}^{N} 
      B_k(X_i)
      \right)
    \right|
    \ 
    \le 
    \ 
    \delta_k
    \qquad
    \text{for all}\ 
    k \in \left\{ 1, \ldots, N \right\}
    \,.
  \end{gather}
  Note, that the first sum goes over $\left\{ 1,\ldots,n \right\}$ while the second sum goes over $\left\{ 1,\ldots,N \right\}$.
  A second, equivalent version of the constraints is
  \begin{gather}
      \left| 
      \frac{1}{N} 
      \left( 
      \sum_{i = 1}^{N} 
      T_i
      w(X_i)
      B_k(X_i)
      -
      \sum_{i=1}^{N} 
      B_k(X_i)
      \right)
    \right|
    \ 
    \le 
    \ 
    \delta_k
    \qquad
    \text{for all}\ 
    k \in \left\{ 1, \ldots, N \right\}
    \,.
  \end{gather}
  Now both sums go over $\left\{ 1,\ldots,N \right\}$ and the
  indicator of treatment $T_i$ takes care that in the first sum only the terms with $i\le n$ are effective. 
  Having this flexibility with the versions helps. I regard the first version as suitable for non-probabilistic computations, although $n$ is of course a random variable. On the other hand, the second version is more honest, exactly telling the dependence on the indicator of treatment. This version is useful in probabilistic computations. 

  Let's bound $R_1$.
  \begin{align}
    \label{R_1:1}
    \begin{split}
    \sup_{z\in\R}
    \left| 
    R_1(z)
    \right|
    &
    \ 
    =
    \ 
    \sup_{z\in\R}
    \left| 
  \sqrt{N}
  \sum_{k=1}^{N} 
  \left[ 
  \frac{1}{N}
  \left( 
    \sum_{i=1}^{n} 
    w(X_i)
    B_k(X_i)
    -
    \sum_{i=1}^{N} 
    B_k(X_i)
  \right)
  F_{Y(1)}(z|X_k)
  \right]
    \right|
    \\
    &
    \ 
    \le
    \ 
  \sqrt{N}
  \sum_{k=1}^{N} 
  \left| 
  \frac{1}{N}
  \left( 
    \sum_{i=1}^{n} 
    w(X_i)
    B_k(X_i)
    -
    \sum_{i=1}^{N} 
    B_k(X_i)
  \right)
  \right|
    \sup_{z\in\R}
  F_{Y(1)}(z|X_k)
  \\
    &
    \ 
    \le
    \ 
  \sqrt{N}
  \norm{\delta}_1
    \end{split}
  \end{align}
  Playing around with norm equivalences we discover that 
  $\sqrt{N}\norm{\delta}_1\to 0$ for $N\to \infty$ is the weakest
  (natural) assumption to
  control $R_1$.
  Indeed, other ways to continue the second row in \eqref{R_1:1} are
  \begin{gather*}
    (\,\cdots)
    \ 
  \le
    \ 
  \sqrt{N}
  \norm{\delta}_2
  \left( 
  \sum_{k=1}^{N} 
  \left( 
    \sup_{z\in\R}
  F_{Y(1)}(z|X_k)
  \right)^2
\right)^{1/2}
\ 
\le
\ 
N
  \norm{\delta}_2\,,
  \end{gather*}
  by the Cauchy-Schwarz inequality and
  $
  F_{Y(1)}\in [0,1]
  $,
or
\begin{gather*}
  (\,\cdots)
  \ 
  \le
  \ 
  \sqrt{N}
  \norm{\delta}_\infty
  \sum_{k=1}^{N} 
    \sup_{z\in\R}
  F_{Y(1)}(z|X_k)
  \ 
  \le
  \ 
  N^{3/2}
  \norm{\delta}_\infty
  \,.
\end{gather*}
Since $\delta\in \R^N$, however, it holds
\begin{gather*}
  \sqrt{N}\norm{\delta}_1
  \ 
  \le
  \ 
  N\norm{\delta}_2
  \ 
  \le
  \ 
  N^{3/2}\norm{\delta}_\infty
  \,.
\end{gather*}
With hindsight, the assumption 
  $\sqrt{N}\norm{\delta}_1\to 0$ for $N\to \infty$ 
  also 
  suffices 
  to control the second (or first) occurrence of a term, that we control by assumptions on $\delta$.
This is the \textbf{second term} of \eqref{c:1}, where we estimate
\begin{gather*}
  \inner{\delta}{\left| \Delta \right|}
  \ 
  =
  \ 
  \sum_{k=1}^{N} 
  \delta_k
  \left| \Delta_k \right|
  \ 
  \le
  \ 
  \norm{\delta}_1
  \norm{\Delta}_\infty
  \ 
  \le
  \ 
  \norm{\delta}_1
  \norm{\Delta}_2
  \ 
  \le
  \ 
  \norm{\delta}_1
  \varepsilon
  \ 
  \to
  \ 
  0
  \quad
  \text{for}\ 
  N\to \infty
  \,.
\end{gather*}
\subsection*{Analysis of $R_2$}
In the original paper \cite{Wang2019} the authors derive concrete learning rates for the weights and employ them in bounding this term. They obtain a multiplied learning rate, which is sufficiently fast. Their approach, however, calls for concrete learning rates of the weights. Arguably, the process of deriving such rates is the most complicated part of the paper. 
I found out, that we don't need concrete rates for the weights. 
Consistency of the weights is enough and gives us an (arbitrarily slow but sufficient) learning rate to establish the results.
We don't even need rates for the weights to control $R_2$.
They only play a role in bounding $R_3$.
Nevertheless, we use the 
second constraint of Problem~\eqref{bw:1:primal} 
\begin{gather}
  1
  \ 
  =
  \ 
\frac{1}{N}\sum_{i=1}^{n}w_i^\dagger
  \ 
  =
  \ 
\frac{1}{N}\sum_{i=1}^{n}w(X_i)
  \ 
=
  \ 
\frac{1}{N}\sum_{i=1}^{N}T_iw(X_i)
\,.
\end{gather}
To this end, we note that
\begin{align*}
&
  \sup_{z\in\R}
  \left| 
  F_{Y(1)}(z|X_i)
  -
  \sum_{k=1}^{N} 
  B_k(X_i)
  \cdot
  F_{Y(1)}(z|X_k)
  \right|
  \\
  &
  \ 
  \le
  \ 
  \sum_{k=1}^{N} 
  \frac{\mathbf{1}_{\left\{ X_k\in A_N(X_i) \right\}}}
  {
    \sum_{j=1}^{N} 
\mathbf{1}_{\left\{ X_j\in A_N(X_i) \right\}}
  }
  \sup_{z\in\R}
  \left| 
  F_{Y(1)}(z|X_i)
  -
  F_{Y(1)}(z|X_k)
  \right|
  \\
  &
  \ 
  \le
  \ 
  \sup_{z\in\R}
  \omega
  \left( 
    F_{Y(1)}(z|\cdot)
    ,h_N
  \right)
  \,,
\end{align*}
where $\omega$ is the modulus of continuity and $h_N$ is the width of
the partition $\mathcal{P}_N=\left\{A_{1,N},A_{2,N},\ldots  \right\}$.
There are many (more concrete, yet stronger) assumptions on the regularity of
$F_{Y(1)}$ and the width of the partition $h_N$ that give us
\begin{gather}
  \sqrt{N}
  \sup_{z\in\R}
  \omega
  \left( 
    F_{Y(1)}(z|\cdot)
    ,h_N
  \right)
  \to
  0
  \qquad
  \text{for}\ 
  N\to \infty
  \,.
\end{gather}
But we shall keep this more general (and abstract) assumption.
We conclude
\begin{align*}
&
  \sup_{z\in\R}
  \left| 
  R_2(z)
  \right|
  \\
  &
  \ 
  \le
  \ 
  \sqrt{N}
    \sum_{i=1}^{N} 
    \left[ 
  \frac{
    T_i\cdot w(X_i) -1 }{N}
  \sup_{z\in\R}
    \left| 
  F_{Y(1)}(z|X_i)
    \ 
    -
    \ 
    \sum_{k=1}^{N} 
    B_k(X_i)
    \cdot
  F_{Y(1)}(z|X_k)
    \right|
    \right]
    \\
    &
  \ 
    \le
  \ 
    \sqrt{N}
  \sup_{z\in\R}
  \omega
  \left( 
    F_{Y(1)}(z|\cdot)
    ,h_N
  \right)
  \sum_{i=1}^{N} 
  \frac{
    T_i\cdot w(X_i) +1 }{N}
    \\
    &
  \ 
    =
  \ 
    2
    \sqrt{N}
  \sup_{z\in\R}
  \omega
  \left( 
    F_{Y(1)}(z|\cdot)
    ,h_N
  \right)
  \ 
  \to 
  \ 
  0\,.
\end{align*}
\subsection*{Analysis of $R_3$}
We will apply theory of empirical processes to bound
\begin{gather}
  R_3(z)
  =
  \sqrt{N}
  \left( 
  \frac{1}{N}
    \sum_{i=1}^{N} 
    \left[ 
    T_i
    \left( 
    w(X_i) 
    -
    \frac{1}{\pi(X_i)}
    \right)
    \left( 
    \mathbf{1}_{\left\{ Y_i \le z \right\}}
    -
  F_{Y(1)}(z|X_i)
    \right)
    \right]
  \right)
\end{gather}
in probability. Why don't we use simple concentration inequalities such as Bernstein's  or Markov's inequality? The reason is, that
the weights
$
  w(x)
  :=
  (
  f^{'}
  )^{-1}
  \left( 
    \inner{B(x)}{\lambda^\dagger}
    +
    \lambda_0^\dagger
  \right)
  $
  depend (thorough $B$ and $(\lambda^\dagger,\lambda_0^\dagger)$)
  on the whole data set $D:=(T_i,X_i)_{i=1,\ldots,N}$. Thus, it is more honest to write
  $w(x,D)$ instead. This captures the whole dependence on probabilities. Note, that $(Y_i)_{i=1,\ldots N}$ are independent of $w$ given $D$. 
  A standard computation shows
  \begin{gather}
    \E
    \left[ 
    \frac{T}{\pi(X)}
    \left( 
    \mathbf{1}_{\left\{ Y(T) \le z \right\}}
    -
  F_{Y(1)}(z|X)
    \right)
    \right]
    =0
    \,.
  \end{gather}
  Furthermore
  \begin{align*}
    &
    \E
    \left[ 
      Tw(X,D)
    \left( 
    \mathbf{1}_{\left\{ Y(T) \le z \right\}}
    -
  F_{Y(1)}(z|X)
    \right)
    \right]
    \\
    &
    \ 
    =
    \ 
    \E
    \left[ 
      \E
      \left[ 
      w(X,D)
      \left( 
    \mathbf{1}_{\left\{ Y(1) \le z \right\}}
    -
  F_{Y(1)}(z|X)
      \right)
  |T=1,X,D
      \right]
    \right]
    \\
    &
    \ 
    =
    \ 
    \E
    \left[ 
      w(X,D)
      \E
      \left[ 
    \mathbf{1}_{\left\{ Y(1) \le z \right\}}
    -
  F_{Y(1)}(z|X)
  |X,D
      \right]
    \right]
    \\
    &
    \ 
    =
    \ 
    \E
    \left[ 
      w(X,D)
      \E
      \left[ 
    \mathbf{1}_{\left\{ Y(1) \le z \right\}}
    -
  F_{Y(1)}(z|X)
  |X
      \right]
    \right]
    \\
    &
    \ 
    =
    \ 
    0
  \end{align*}
  The second equality is due to the assumption of $(Y(0),Y(1))\perp T |X$.
  The third equality is due to $X\perp D$.
  Next we define
  the (random) function
  $f_D^z$ by
  \begin{gather}
    f_{D}^z(T,X,Y(T))
    :=
    T
    \left( 
    w(D,X)- \frac{1}{\pi(X)}
    \right)
    \left( 
    \mathbf{1}_{\left\{ Y(T) \le z \right\}}
    -
  F_{Y(1)}(z|X)
    \right)
    \,.
  \end{gather}
  We just showed
  $
  \E
  [
    f_{D}^z(T,X,Y(T))
  ]
  =
  0
  $
  for all $z\in\R$.
  Thus
  \begin{gather}
    R_3(z)
    =
    G_N
    f_D^z
    \,.
  \end{gather}
  By the consistency of the weights there exists a learning rate $(\varepsilon_N)$ such that
  \begin{gather}
    \P
    \left[ 
      \left| 
      w(X,D)
      -
      \frac{1}{\pi(X)}
      \right|
      \le
      \varepsilon_N
    \right]
    \to 1 
    \qquad
    \text{for}\ 
    N\to \infty
    \,.
  \end{gather}
  Let
  $
  \mathcal{F}_N
  :=
  \varepsilon_N B_{\mathcal{F}}$.
  It holds
  \begin{gather}
    \P
    \left[ 
    f_D^z
    \in
  \mathcal{F}_N
  \ \forall\,z\in\R
    \right]
    =
    \P
    \left[ 
      \sup_{z\in\R}
      \left| 
    f_D^z
      \right|
      \le
      \varepsilon_N
    \right]
    \to 1
  \end{gather}
  Then the lemma applies?.
\end{proof}


  %\begin{theorem}
  The estimate 
  \begin{gather}
    \sqrt{N}
    \left( 
  \frac{1}{N}
    \sum_{i=1}^{n} 
    w_i
    \mathbf{1}
    _{\left\{ Y_i\le t \right\}}
    -
    \P[Y(1)\le t]
    \right)
  \end{gather}
  is Asymptoticaly normal.
\end{theorem}
\begin{proof}
\begin{align*}
  &
    \frac{1}{N}
    \sum_{i=1}^{n} 
    w_i
    \mathbf{1}_{\left\{ Y_i \le t \right\}}
    \ 
    -
    \ 
    \P[Y(1)\le t]
    \\
  &
  \ 
  =
  \ 
  \frac{1}{N}
  \left( 
    \sum_{i=1}^{n} 
    w_i
    B(X_i)
    -
    \sum_{i=1}^{N} 
    B(X_i)
  \right)
  ^\top
     \mathbf{Y}(1) 
  %%%% 1 %%%%
  \\
  &
  \qquad
  +
  \ 
  \frac{1}{N}
    \sum_{i=1}^{N} 
    \left( T_i\cdot w_i -1 \right)
    \left( 
    \P[Y(1)\le t| X_i]
    -
    \inner
    {B(X_i)}
    { \mathbf{Y}(1) }
    \right)
  %%%% 2 %%%%
  \\
  &
  \qquad
  +
  \ 
  \frac{1}{N}
    \sum_{i=1}^{N} 
    T_i
    (
    w_i 
    -
    1/\pi_i
    )
    \left( 
    \mathbf{1}_{\left\{ Y_i \le t \right\}}
    -
    \P[Y(1)\le t| X_i]
    \right)
  %%%% 3 %%%%
  \\
  &
  \qquad
  +
  \ 
  \frac{1}{N}
    \sum_{i=1}^{N} 
    T_i
    /\pi_i
    \left( 
    \mathbf{1}_{\left\{ Y_i \le t \right\}}
    -
    \E[Y(1)\le t| X_i]
    \right)
    +
    \left( 
    \P[Y(1)\le t| X_i]
    -
    \P[Y(1)\le t]
    \right)
  \\
  %%%%%%%%%%%%%%%
  &
  \ 
  =:
  \ 
  R_1
  +
  R_2
  +
  R_3
  +
  R_4
  \end{align*}

The term 
\begin{gather}
    \left|  
  \frac{1}{N}
    \sum_{i=1}^{N} 
    \left( 
    T_i
    w_i
    -
    1
    \right)
    \left( 
    \P[Y(1)\le t | X_i]
    -
    \inner
    {B(X_i))}
    {\lambda^*}
    \right)
  \right|
\end{gather}
gives back the expectation. We control its rate 
with the factor of basis rates and primal weights.
Choose $d$ to be faster than $\sqrt{n}$.
The rest follows with standard empirical theory.
Expectations are 0.
$\norm{\delta}$ has to converge fast enough.
\begin{align*}
  &
  \sqrt{N}
  \cdot
  \E
  \left[ 
    \left( 
    T
    \cdot
    w(X)
    -
    1
    \right)
    \left( 
      \P
      [
      Y(1)\le t
      |X
      ]
      -
      \inner
      {B(X)}
      {\lambda^*}
    \right)
  \right]
  \\
  &
  \ 
  =
  \ 
  \sqrt{N}
  \cdot
  \E
  \left[ 
    \pi(X)
    \cdot
    \left( 
    w(X)
    -
    1
    /
    \pi(X)
    \right)
    \left( 
      \P
      [
      Y(1)\le t
      |X
      ]
      -
      \inner
      {B(X)}
      {\lambda^*}
    \right)
  \right]
  \\
  &
  \ 
  \le
  \ 
  \sqrt{N}
  \cdot
  \E
  \left[ 
    \left| 
    w(X)
    -
    1
    /
    \pi(X)
    \right|
    ^2
    \right]
    ^{1/2}
  \E
  \left[ 
    \left| 
      \P
      [
      Y(1)\le t
      |X
      ]
      -
      \inner
      {B(X)}
      {\lambda^*}
    \right|
    ^2
    \right]
    ^{1/2}
    \\
  &
  \ 
  \le
  \ 
  C
  \sqrt{N}
  \sqrt{\log(N)}
  N^{-1/(d+2)}
  \cdot
  N^{-1/(d+2)}
    \\
  &
  \ 
  \le
  \ 
  C
  \sqrt{\log(N)}
  N^{-1/6}
  \to 0
  \qquad
  \text{for}\ 
  N\to\infty
\end{align*}
Note, that we get no convergence for $d>1$. 
Also note that
\begin{gather*}
  \E
  \left[ 
    T
    \cdot
    w(X)
    -
    1
    |
    X,
    X_1,
    \ldots,
    X_N
    \right]
    =
  \E
  \left[ 
    T
    |
    X
    \right]
    w(X)
    -
    1
    =
    \pi(X)
    \left( 
      w(X)
      -
      1/\pi(X)
    \right)
\end{gather*}
\end{proof}
\begin{lemma}
  Under conditions it holds
  for all $\varepsilon>0$
  \begin{gather}
    \P
    \left[ 
      \norm{G_N}^*_{\mathcal{F}_N}\ge \varepsilon
    \right]
    \ 
    \to 
    \ 
    0
    \qquad
    \text{for}\ 
    N\to\infty
    \,.
  \end{gather}
\end{lemma}
\begin{proof}
  By maximal inequalities it holds
  \begin{gather}
    \E^*
    \left[ 
      \norm{G_N}_{\mathcal{F}_N}
    \right]
      \ 
      \lesssim
      \ 
      J_{[\,]}\left( \varepsilon_N, \mathcal{F}_N,\mathrm{L}_2(\P) \right)
      \\
      =
      \int_0^{\varepsilon_N}
      \sqrt{\log 
      N_{[\,]}
\left( \varepsilon, \mathcal{F}_N,\mathrm{L}_2(\P) \right)
    }
    \,
    d\varepsilon
    =
      \int_0^{\varepsilon_N}
      \sqrt{\log 
      N_{[\,]}
\left( \varepsilon/\varepsilon_N, B_\mathcal{F},\mathrm{L}_2(\P) \right)
    }
    \,
    d\varepsilon
    \\
    \le
      \int_0^{\varepsilon_N}
      \left( 
      \frac{\varepsilon_N}{\varepsilon}
    \right)^{k/2}
    \,
    d\varepsilon
    \\
    \lesssim
  \varepsilon_N^{k/2}
  \frac{1}{1-k/2}
  \varepsilon_N^{1-k/2}
  \lesssim
  \varepsilon_N
  \to
  0
  \qquad
  \text{for}\ 
  N\to
  \infty
  \,.
  \end{gather}
  Note, that $k<2$.
  By the boundedness of $\E^*$ there is no measurability problem.
  By Markov's Inequality it holds
  \begin{align}
    \P
    \left[ 
      \norm{G_N}^*_{\mathcal{F}_N}\ge \varepsilon
    \right]
    \le
    \E^*
    \left[ 
      \norm{G_N}_{\mathcal{F}_N}
    \right]
  \end{align}
\end{proof}

  %We can even view
$
\frac{1}{\sqrt{n}}
\sum_{i=1}^{n}S_i 
$
as an empirical process 
$
\mathbb{G}_n f
$
indexed over 

\begin{gather}
  f_\Phi(T,X,Y)
  =
  \frac{T}{\pi(X)}
 \left( 
   \Phi(Y) - \E [ \Phi(Y) | X ]
 \right)
 +
   \E [ \Phi(Y) | X ] 
   .
\end{gather}
If $\mathcal{F}=\left\{ f_\Phi \colon \Phi \in \ \text{some set}\right\}$
is $\P$-Donsker, the empirical process converges to a tight gaussian process.
Then the functional delta Method is applicable.



  \section{Application to Plug In Estimators}
  \input{chapters/balancing_weights/plug_in_est.tex}
  %We settle for partitioning estimates, 
although other (bounded) universal consistent basis functions
would work as well.

We leverage (weak) universal consistency
of partitioning estimates~\cite{Gyorfi2002}.
To this end, we choose the basis functions as
\begin{gather*}
  B_k(x)
  :=
  \frac{
  \mathbf{1}_{X_k \in A_n(x)}
  }{
  \sum_{j=1}^{n} 
  \mathbf{1}_{X_j \in A_n(x)}
  }
  \,,
  \qquad
  k=
  1,\ldots,n
  \,.
\end{gather*}
The euclidian norm of 
$
  B(x)
$
is bounded by one.
\begin{gather*}
  \norm{B(x)}^2
  =
  \sum_{k=1}^{n} 
  \left( 
  \frac{
  \mathbf{1}_{X_k \in A_n(x)}
  }{
  \sum_{j=1}^{n} 
  \mathbf{1}_{X_j \in A_n(x)}
  }
  \right)
  ^2
  \le
  \sum_{k=1}^{n} 
  \frac{
  \mathbf{1}_{X_k \in A_n(x)}
  }{
  \sum_{j=1}^{n} 
  \mathbf{1}_{X_j \in A_n(x)}
  }
  =1
  \,.
\end{gather*}
The oracle parameters are
$
  \left[ 
    f^{'}
    \left( 
      \frac{1}{\pi_i}
    \right)
  \right] _ { i = 1,\ldots,n }
$
and
$
  \left[ 
    Y_i(1)
  \right] _ { i = 1,\ldots,n }
$
.
It even holds
\begin{gather*}
  \lambda ^ \dagger _ i \to 
    f^{'}
    \left( 
      \frac{1}{\pi_i}
    \right)
\end{gather*}
in probability.

 \begin{align*}
   G(\lambda)
   &
   \ 
   :=
   \ 
      \frac{1}{n}
      \sum_{j = 1}^{n} 
        T_i
        \cdot
        f^* 
        \left( 
          \inner
          {B(X_i)}
          {\lambda}
        \right)
      -
          \inner
          {B(X_i)}
          {\lambda}
      +
      \inner
      { | \lambda | }
      { \delta }
      \\
   &
   \ 
   =:
   \ 
      L( \lambda )
      +
      \inner
      { | \lambda | }
      { \delta }
 \end{align*} 
 The consistency of 
 $
  \lambda ^ \dagger
 $
 relates to the difference being nonnegative.
 For all 
 $
  \varepsilon > 0
 $
 it holds
 \begin{gather*}
   \P
   \left[ 
     \norm
     {
      \lambda ^ \dagger
      -
      f ^ { ' }
      \left( 
        1 / \pi 
      \right)
     }
     \le
     \varepsilon
   \right]
   =
   \P
   \left[ 
     \inf _ { \norm{\Delta} = \varepsilon }
     G
     (
      f ^ { ' }
      \left( 
        1 / \pi 
      \right)
      +
      \Delta
     )
     -
     G
     (
      f ^ { ' }
      \left( 
        1 / \pi 
      \right)
     )
     \ge 
     0
   \right]
   \,.
 \end{gather*}
 The term on the right-hand side.
 \begin{align*}
   &
     G
     (
      f ^ { ' }
      \left( 
        1 / \pi 
      \right)
      +
      \Delta
     )
     \ 
     -
     \ 
     G
     (
      f ^ { ' }
      \left( 
        1 / \pi 
      \right)
     )
     %%%%%%%%%%%%% 1 %%%%%%%%%%%%%%
     \\
     &
     \quad
     \ge
     \ 
     \inner
     {
     \nabla 
     L
     (
      f ^ { ' }
      \left( 
        1 / \pi 
      \right)
     )
     }
     {
     \Delta
     }
     \ 
     +
     \ 
     \inner
     {
      | 
      f ^ { ' }
      \left( 
        1 / \pi 
      \right)
      +
      \Delta
     |
     \ 
      -
     \ 
     | 
      f ^ { ' }
      \left( 
        1 / \pi 
      \right)
     |
     }
     { \delta }
     %%%% 2 %%%%
     \\
     &
     \quad
     \ge
     \ 
     -
     \norm{\Delta}
     \left( 
     \norm{\delta}
     \ 
     +
     \ 
     \norm{B(X_i)}
     \cdot
     \frac{1}{n}
     \sum_{i=1}^{n} 
     \left| 
     \,
      1
     \ 
      -
     \ 
        T_i
        \cdot
     (f ^ { ' }) ^ { -1 }
          \inner
          {B(X_i)}
          {
      f ^ { ' }
      \left( 
        1 / \pi 
      \right)
          }
          \,
     \right|
     \right)
     %%%% 3 %%%%
     \\
     &
     \quad
     \ge
     \ 
     -
     \varepsilon
     \left( 
     \norm{\delta}
     \ 
     +
     \ 
     \max _ { i = 1,\ldots,n }
     \left| 
     \,
      1
      /
      \pi_i
     \ 
      -
     \ 
     (f ^ { ' }) ^ { -1 }
          \inner
          {B(X_i)}
          {
      f ^ { ' }
      \left( 
        1 / \pi 
      \right)
          }
          \,
     \right|
     +
     \frac{1}{n}
     \sum_{i=1}^{n} 
     \left| 
     \,
     1
     \ 
      -
     \ 
      T_i
      /
      \pi_i
     \right|
     \right)
     %%%% 4 %%%%
     \\
     &
     \quad
     \ge
     \ 
     -
     \varepsilon
     \left( 
     \varepsilon _ {\delta}
     \ 
     +
     \ 
     \omega (
     (f ^ { ' }) ^ { -1 }
     ,
     \varepsilon _ m
     )
     \ 
     +
     \ 
     \varepsilon _ {\mathrm{WLLN}}
     \right)
     %%%% 5 %%%%
     \\
     &
     \quad
     \ge
     \ 
     - \varepsilon _ G
     \,,
 \end{align*}
 with probability tending to 1.
 %%%% convergence G %%%%
 It follows
 for all $\varepsilon, \varepsilon_G > 0 $
 \begin{gather*}
   \P
   \left[ 
     \inf _ { \norm{\Delta} = \varepsilon }
     G
     (
      f ^ { ' }
      \left( 
        1 / \pi 
      \right)
      +
      \Delta
     )
     -
     G
     (
      f ^ { ' }
      \left( 
        1 / \pi 
      \right)
     )
     \ge 
     - \varepsilon_G
   \right]
   \ 
   \to 
   \ 
   1
   \,
 \end{gather*}
 for $ n \to \infty $.
 %%%% convergence lambda %%%%
 Thus, for all $ \varepsilon > 0 $
 it holds
 \begin{gather*}
   \P
   \left[ 
     \norm
     {
      \lambda ^ \dagger
      -
      f ^ { ' }
      \left( 
        1 / \pi 
      \right)
     }
     \le
     \varepsilon
   \right]
   =
   \P
   \left[ 
     \inf _ { \norm{\Delta} = \varepsilon }
     G
     (
      f ^ { ' }
      \left( 
        1 / \pi 
      \right)
      +
      \Delta
     )
     -
     G
     (
      f ^ { ' }
      \left( 
        1 / \pi 
      \right)
     )
     \ge 
     0
   \right]
   \ 
   \to 
   \ 
   1
   \,,
 \end{gather*}
 for $ n \to \infty $.
 Furthermore,
 \begin{align*}
   T_i
   \cdot
   \left| 
   w_i
   -
   1 / \pi_i
   \right|
   & 
   \ 
   =
   \ 
   T_i
   \cdot
   \left| 
   T_i
   (
      f ^ { ' }
   )
   ^ {-1}
   \inner
   {T_i B(X_i)}
   { \lambda ^ \dagger }
   -
     (f ^ { ' }) ^ { -1 }
     \left( 
      f ^ { ' }
      \left( 
        1 / \pi _ i
      \right)
     \right)
   \right|
   %%%% 1 %%%%
   \\
   & 
   \ 
   \le
   \ 
   \left| 
   (
      f ^ { ' }
   )
   ^ {-1}
   \inner
   {B(X_i)}
   { \lambda ^ \dagger }
   -
   (
      f ^ { ' }
   )
   ^ {-1}
          \inner
          {B(X_i)}
          {
      f ^ { ' }
      \left( 
        1 / \pi 
      \right)
          }
   \right|
   %%%% 1 a %%%%
   \\
   &
   \qquad
   +
   \ 
   \left| 
   (
      f ^ { ' }
   )
   ^ {-1}
          \inner
          {B(X_i)}
          {
      f ^ { ' }
      \left( 
        1 / \pi 
      \right)
          }
   -
     (f ^ { ' }) ^ { -1 }
     \left( 
      f ^ { ' }
      \left( 
        1 / \pi _ i
      \right)
     \right)
   \right|
   %%%% 2 %%%%
   \\
   &
   \ 
   \le
   \ 
   \omega
   \left( 
     (f ^ { ' }) ^ { -1 }
     ,
     \norm{ \lambda ^ \dagger - f ^ {'} (1/\pi) }
   \right)
   \ 
   +
   \ 
   \omega
   \left( 
     (f ^ { ' }) ^ { -1 }
     ,
     \varepsilon _ m
   \right)
   %%%% 3 %%%%
   \\
   &
   \ 
   \le
   \ 
   \omega
   \left( 
     (f ^ { ' }) ^ { -1 }
     ,
     \varepsilon _ \dagger
   \right)
   \ 
   +
   \ 
   \omega
   \left( 
     (f ^ { ' }) ^ { -1 }
     ,
     \varepsilon _ m
   \right)
   %%%% 4 %%%%
   \\
   &
   \ 
   \le
   \ 
   \varepsilon
 \end{align*}
 with probability tending to 1.

 Next we consider the estimate.
\begin{align*}
  \left| 
    \frac{1}{n}
    \sum_{i=1}^{n} 
    w_i
    T_i
    Y_i
    -
    \E[Y(1)]
  \right|
  &
  \ 
  \le
  \ 
  \left|  
  \frac{1}{n}
    \sum_{i=1}^{n} 
    (
    w_i T_i
    -
    1
    )
    \inner
    {B(X_i))}
    { \mathbf{Y}(1) }
  \right|
  %%%% 1 %%%%
  \\
  &
  \qquad
  +
  \ 
  \left|  
  \frac{1}{n}
    \sum_{i=1}^{n} 
    (
    w_i T_i
    -
    1
    )
    \left( 
    \E[Y(1)| X_i]
    -
    \inner
    {B(X_i))}
    { \mathbf{Y}(1) }
    \right)
  \right|
  %%%% 2 %%%%
  \\
  &
  \qquad
  +
  \ 
  \left|  
  \frac{1}{n}
    \sum_{i=1}^{n} 
    T_i
    \cdot
    (
    w_i 
    -
    1/\pi_i
    )
    \left( 
      Y_i
    -
    \E[Y(1)| X_i]
    \right)
  \right|
  %%%% 3 %%%%
  \\
  &
  \qquad
  +
  \ 
  \left|  
  \frac{1}{n}
    \sum_{i=1}^{n} 
    T_i
    /\pi_i
    \left( 
      Y_i
    -
    \E[Y(1)| X_i]
    \right)
    +
    \left( 
    \E[Y(1)| X_i]
    -
    \E[Y(1)]
    \right)
  \right|
  %%%% 4 %%%%
  \\
  &
  \ 
  \le
  \ 
  C_Y
  \varepsilon _ \delta
  +
  C_w
  \varepsilon _ m
  +
  C_{\E Y}
  \varepsilon _ w
  +
  \varepsilon _ {\mathrm{WLLN}}
  %%%% 4 %%%%
  \\
  &
  \ 
  \le
  \ 
  \varepsilon
\end{align*}
with probability tending to 1.
\begin{ftheorem}
  Let 
  $Y$
  be bounded
  and 
  $
    \E
    \left[ 
      f ^ { ' }
\left( 
  1 / \pi(X)
\right)
^2
    \right]
    <
    \infty
  $.
  Then 
  \begin{gather*}
    \frac{1}{n}
    \sum_{i=1}^{n} 
    w_i
    T_i
    Y_i
  \end{gather*}
  is a consistent estimator for $\E[Y(1)]$,
  that is, for all 
  $
    \varepsilon > 0
  $
  it holds
  \begin{gather*}
    \P
    \left[ 
      \left| 
    \frac{1}{n}
    \sum_{i=1}^{n} 
    w_i
    T_i
    Y_i
    -
    \E[Y(1)]
      \right|
    \ge 
    \varepsilon
    \right]
    \ 
    \to
    \ 
    0
  \end{gather*}
  for 
  $n\to \infty$.
\end{ftheorem}
\begin{remark}
  The entropy 
  \begin{gather*}
    f \colon (0, \infty)
    \to 
    \R
    \,,
    \quad
    x \mapsto x\log x
  \end{gather*}
  is a prevailing choice.
  Then the requirement on $\pi$ is
  \begin{gather*}
    \E
    [
    (
    \log \pi(X)
    )
    ^2
    ]
    <
    \infty
    \,.
  \end{gather*}
  This is a very weak requirement.
  Likewise, the requirement for the variance
  \begin{gather*}
    f \colon 
    \R
    \to 
    \R
    \,,
    \quad
    x \mapsto (x-1/n)^2
  \end{gather*}
  is
  \begin{gather*}
    \E
    [
    1/
    \pi(X)
    ^2
    ]
    <
    \infty
    \,.
  \end{gather*}
\end{remark}

\begin{remark}
  Both partitioning estimates are covered by \cite[§4]{Gyorfi2002},
  because we only estimate quantities depending on $X$,
  that is 
  $
  f^{-1}(1/\pi(X))
  $
  and 
  $
  \E[Y(1)|X]
  $.
  The corresponding coefficients in the partitioning estimate are then
  $
  f^{-1}(1/\pi_i)
  $
  and
  $
  Y_i(1)
  $,
  which is the potential outcome of unit $i$ under treatment.
  Note, that both quantities are generally unknown to us, but we can 
  nevertheless leverage there existence in the proof.
\end{remark}
\begin{takeaways}
  To ensure double robustness, we leverage primal and dual optimization
  problem. The dual problem solves propensity score estimation and the 
  primal problem the bias of the final estimate.
  Partitioning estimates as in \cite{Gyorfi2002}. 
\end{takeaways}


  %We settle for partitioning estimates, 
although other (bounded) universal consistent basis functions
would work as well.

We leverage (weak) universal consistency
of partitioning estimates~\cite{Gyorfi2002}.
To this end, we choose the basis functions as
\begin{gather*}
  B_k(x)
  :=
  \frac{
  \mathbf{1}_{X_k \in A_n(x)}
  }{
  \sum_{j=1}^{n} 
  \mathbf{1}_{X_j \in A_n(x)}
  }
  \,,
  \qquad
  k=
  1,\ldots,n
  \,.
\end{gather*}
The euclidian norm of 
$
  B(x)
$
is bounded by one.
\begin{gather*}
  \norm{B(x)}^2
  =
  \sum_{k=1}^{n} 
  \left( 
  \frac{
  \mathbf{1}_{X_k \in A_n(x)}
  }{
  \sum_{j=1}^{n} 
  \mathbf{1}_{X_j \in A_n(x)}
  }
  \right)
  ^2
  \le
  \sum_{k=1}^{n} 
  \frac{
  \mathbf{1}_{X_k \in A_n(x)}
  }{
  \sum_{j=1}^{n} 
  \mathbf{1}_{X_j \in A_n(x)}
  }
  =1
  \,.
\end{gather*}
The oracle parameters are
$
  \left[ 
    f^{'}
    \left( 
      \frac{1}{\pi_i}
    \right)
  \right] _ { i = 1,\ldots,n }
$
and
$
  \left[ 
    Y_i(1)
  \right] _ { i = 1,\ldots,n }
$
.
It even holds
\begin{gather*}
  \lambda ^ \dagger _ i \to 
    f^{'}
    \left( 
      \frac{1}{\pi_i}
    \right)
\end{gather*}
in probability.

 \begin{align*}
   G(\lambda)
   &
   \ 
   :=
   \ 
      \frac{1}{n}
      \sum_{j = 1}^{n} 
        T_i
        \cdot
        f^* 
        \left( 
          \inner
          {B(X_i)}
          {\lambda}
        \right)
      -
          \inner
          {B(X_i)}
          {\lambda}
      +
      \inner
      { | \lambda | }
      { \delta }
      \\
   &
   \ 
   =:
   \ 
      L( \lambda )
      +
      \inner
      { | \lambda | }
      { \delta }
 \end{align*} 
 The consistency of 
 $
  \lambda ^ \dagger
 $
 relates to the difference being nonnegative.
 For all 
 $
  \varepsilon > 0
 $
 it holds
 \begin{gather*}
   \P
   \left[ 
     \norm
     {
      \lambda ^ \dagger
      -
      f ^ { ' }
      \left( 
        1 / \pi 
      \right)
     }
     \le
     \varepsilon
   \right]
   =
   \P
   \left[ 
     \inf _ { \norm{\Delta} = \varepsilon }
     G
     (
      f ^ { ' }
      \left( 
        1 / \pi 
      \right)
      +
      \Delta
     )
     -
     G
     (
      f ^ { ' }
      \left( 
        1 / \pi 
      \right)
     )
     \ge 
     0
   \right]
   \,.
 \end{gather*}
 The term on the right-hand side.
 \begin{align*}
   &
     G
     (
      f ^ { ' }
      \left( 
        1 / \pi 
      \right)
      +
      \Delta
     )
     \ 
     -
     \ 
     G
     (
      f ^ { ' }
      \left( 
        1 / \pi 
      \right)
     )
     %%%%%%%%%%%%% 1 %%%%%%%%%%%%%%
     \\
     &
     \quad
     \ge
     \ 
     \inner
     {
     \nabla 
     L
     (
      f ^ { ' }
      \left( 
        1 / \pi 
      \right)
     )
     }
     {
     \Delta
     }
     \ 
     +
     \ 
     \inner
     {
      | 
      f ^ { ' }
      \left( 
        1 / \pi 
      \right)
      +
      \Delta
     |
     \ 
      -
     \ 
     | 
      f ^ { ' }
      \left( 
        1 / \pi 
      \right)
     |
     }
     { \delta }
     %%%% 2 %%%%
     \\
     &
     \quad
     \ge
     \ 
     -
     \norm{\Delta}
     \left( 
     \norm{\delta}
     \ 
     +
     \ 
     \norm{B(X_i)}
     \cdot
     \frac{1}{n}
     \sum_{i=1}^{n} 
     \left| 
     \,
      1
     \ 
      -
     \ 
        T_i
        \cdot
     (f ^ { ' }) ^ { -1 }
          \inner
          {B(X_i)}
          {
      f ^ { ' }
      \left( 
        1 / \pi 
      \right)
          }
          \,
     \right|
     \right)
     %%%% 3 %%%%
     \\
     &
     \quad
     \ge
     \ 
     -
     \varepsilon
     \left( 
     \norm{\delta}
     \ 
     +
     \ 
     \max _ { i = 1,\ldots,n }
     \left| 
     \,
      1
      /
      \pi_i
     \ 
      -
     \ 
     (f ^ { ' }) ^ { -1 }
          \inner
          {B(X_i)}
          {
      f ^ { ' }
      \left( 
        1 / \pi 
      \right)
          }
          \,
     \right|
     +
     \frac{1}{n}
     \sum_{i=1}^{n} 
     \left| 
     \,
     1
     \ 
      -
     \ 
      T_i
      /
      \pi_i
     \right|
     \right)
     %%%% 4 %%%%
     \\
     &
     \quad
     \ge
     \ 
     -
     \varepsilon
     \left( 
     \varepsilon _ {\delta}
     \ 
     +
     \ 
     \omega (
     (f ^ { ' }) ^ { -1 }
     ,
     \varepsilon _ m
     )
     \ 
     +
     \ 
     \varepsilon _ {\mathrm{WLLN}}
     \right)
     %%%% 5 %%%%
     \\
     &
     \quad
     \ge
     \ 
     - \varepsilon _ G
     \,,
 \end{align*}
 with probability tending to 1.
 %%%% convergence G %%%%
 It follows
 for all $\varepsilon, \varepsilon_G > 0 $
 \begin{gather*}
   \P
   \left[ 
     \inf _ { \norm{\Delta} = \varepsilon }
     G
     (
      f ^ { ' }
      \left( 
        1 / \pi 
      \right)
      +
      \Delta
     )
     -
     G
     (
      f ^ { ' }
      \left( 
        1 / \pi 
      \right)
     )
     \ge 
     - \varepsilon_G
   \right]
   \ 
   \to 
   \ 
   1
   \,
 \end{gather*}
 for $ n \to \infty $.
 %%%% convergence lambda %%%%
 Thus, for all $ \varepsilon > 0 $
 it holds
 \begin{gather*}
   \P
   \left[ 
     \norm
     {
      \lambda ^ \dagger
      -
      f ^ { ' }
      \left( 
        1 / \pi 
      \right)
     }
     \le
     \varepsilon
   \right]
   =
   \P
   \left[ 
     \inf _ { \norm{\Delta} = \varepsilon }
     G
     (
      f ^ { ' }
      \left( 
        1 / \pi 
      \right)
      +
      \Delta
     )
     -
     G
     (
      f ^ { ' }
      \left( 
        1 / \pi 
      \right)
     )
     \ge 
     0
   \right]
   \ 
   \to 
   \ 
   1
   \,,
 \end{gather*}
 for $ n \to \infty $.
 Furthermore,
 \begin{align*}
   T_i
   \cdot
   \left| 
   w_i
   -
   1 / \pi_i
   \right|
   & 
   \ 
   =
   \ 
   T_i
   \cdot
   \left| 
   T_i
   (
      f ^ { ' }
   )
   ^ {-1}
   \inner
   {T_i B(X_i)}
   { \lambda ^ \dagger }
   -
     (f ^ { ' }) ^ { -1 }
     \left( 
      f ^ { ' }
      \left( 
        1 / \pi _ i
      \right)
     \right)
   \right|
   %%%% 1 %%%%
   \\
   & 
   \ 
   \le
   \ 
   \left| 
   (
      f ^ { ' }
   )
   ^ {-1}
   \inner
   {B(X_i)}
   { \lambda ^ \dagger }
   -
   (
      f ^ { ' }
   )
   ^ {-1}
          \inner
          {B(X_i)}
          {
      f ^ { ' }
      \left( 
        1 / \pi 
      \right)
          }
   \right|
   %%%% 1 a %%%%
   \\
   &
   \qquad
   +
   \ 
   \left| 
   (
      f ^ { ' }
   )
   ^ {-1}
          \inner
          {B(X_i)}
          {
      f ^ { ' }
      \left( 
        1 / \pi 
      \right)
          }
   -
     (f ^ { ' }) ^ { -1 }
     \left( 
      f ^ { ' }
      \left( 
        1 / \pi _ i
      \right)
     \right)
   \right|
   %%%% 2 %%%%
   \\
   &
   \ 
   \le
   \ 
   \omega
   \left( 
     (f ^ { ' }) ^ { -1 }
     ,
     \norm{ \lambda ^ \dagger - f ^ {'} (1/\pi) }
   \right)
   \ 
   +
   \ 
   \omega
   \left( 
     (f ^ { ' }) ^ { -1 }
     ,
     \varepsilon _ m
   \right)
   %%%% 3 %%%%
   \\
   &
   \ 
   \le
   \ 
   \omega
   \left( 
     (f ^ { ' }) ^ { -1 }
     ,
     \varepsilon _ \dagger
   \right)
   \ 
   +
   \ 
   \omega
   \left( 
     (f ^ { ' }) ^ { -1 }
     ,
     \varepsilon _ m
   \right)
   %%%% 4 %%%%
   \\
   &
   \ 
   \le
   \ 
   \varepsilon
 \end{align*}
 with probability tending to 1.

 Next we consider the estimate.
\begin{align*}
  \left| 
    \frac{1}{n}
    \sum_{i=1}^{n} 
    w_i
    T_i
    Y_i
    -
    \E[Y(1)]
  \right|
  &
  \ 
  \le
  \ 
  \left|  
  \frac{1}{n}
    \sum_{i=1}^{n} 
    (
    w_i T_i
    -
    1
    )
    \inner
    {B(X_i))}
    { \mathbf{Y}(1) }
  \right|
  %%%% 1 %%%%
  \\
  &
  \qquad
  +
  \ 
  \left|  
  \frac{1}{n}
    \sum_{i=1}^{n} 
    (
    w_i T_i
    -
    1
    )
    \left( 
    \E[Y(1)| X_i]
    -
    \inner
    {B(X_i))}
    { \mathbf{Y}(1) }
    \right)
  \right|
  %%%% 2 %%%%
  \\
  &
  \qquad
  +
  \ 
  \left|  
  \frac{1}{n}
    \sum_{i=1}^{n} 
    T_i
    \cdot
    (
    w_i 
    -
    1/\pi_i
    )
    \left( 
      Y_i
    -
    \E[Y(1)| X_i]
    \right)
  \right|
  %%%% 3 %%%%
  \\
  &
  \qquad
  +
  \ 
  \left|  
  \frac{1}{n}
    \sum_{i=1}^{n} 
    T_i
    /\pi_i
    \left( 
      Y_i
    -
    \E[Y(1)| X_i]
    \right)
    +
    \left( 
    \E[Y(1)| X_i]
    -
    \E[Y(1)]
    \right)
  \right|
  %%%% 4 %%%%
  \\
  &
  \ 
  \le
  \ 
  C_Y
  \varepsilon _ \delta
  +
  C_w
  \varepsilon _ m
  +
  C_{\E Y}
  \varepsilon _ w
  +
  \varepsilon _ {\mathrm{WLLN}}
  %%%% 4 %%%%
  \\
  &
  \ 
  \le
  \ 
  \varepsilon
\end{align*}
with probability tending to 1.
\begin{ftheorem}
  Let 
  $Y$
  be bounded
  and 
  $
    \E
    \left[ 
      f ^ { ' }
\left( 
  1 / \pi(X)
\right)
^2
    \right]
    <
    \infty
  $.
  Then 
  \begin{gather*}
    \frac{1}{n}
    \sum_{i=1}^{n} 
    w_i
    T_i
    Y_i
  \end{gather*}
  is a consistent estimator for $\E[Y(1)]$,
  that is, for all 
  $
    \varepsilon > 0
  $
  it holds
  \begin{gather*}
    \P
    \left[ 
      \left| 
    \frac{1}{n}
    \sum_{i=1}^{n} 
    w_i
    T_i
    Y_i
    -
    \E[Y(1)]
      \right|
    \ge 
    \varepsilon
    \right]
    \ 
    \to
    \ 
    0
  \end{gather*}
  for 
  $n\to \infty$.
\end{ftheorem}
\begin{remark}
  The entropy 
  \begin{gather*}
    f \colon (0, \infty)
    \to 
    \R
    \,,
    \quad
    x \mapsto x\log x
  \end{gather*}
  is a prevailing choice.
  Then the requirement on $\pi$ is
  \begin{gather*}
    \E
    [
    (
    \log \pi(X)
    )
    ^2
    ]
    <
    \infty
    \,.
  \end{gather*}
  This is a very weak requirement.
  Likewise, the requirement for the variance
  \begin{gather*}
    f \colon 
    \R
    \to 
    \R
    \,,
    \quad
    x \mapsto (x-1/n)^2
  \end{gather*}
  is
  \begin{gather*}
    \E
    [
    1/
    \pi(X)
    ^2
    ]
    <
    \infty
    \,.
  \end{gather*}
\end{remark}

\begin{remark}
  Both partitioning estimates are covered by \cite[§4]{Gyorfi2002},
  because we only estimate quantities depending on $X$,
  that is 
  $
  f^{-1}(1/\pi(X))
  $
  and 
  $
  \E[Y(1)|X]
  $.
  The corresponding coefficients in the partitioning estimate are then
  $
  f^{-1}(1/\pi_i)
  $
  and
  $
  Y_i(1)
  $,
  which is the potential outcome of unit $i$ under treatment.
  Note, that both quantities are generally unknown to us, but we can 
  nevertheless leverage there existence in the proof.
\end{remark}
\begin{takeaways}
  To ensure double robustness, we leverage primal and dual optimization
  problem. The dual problem solves propensity score estimation and the 
  primal problem the bias of the final estimate.
  Partitioning estimates as in \cite{Gyorfi2002}. 
\end{takeaways}


 % \section{Introduction}
  %We work in the Rubin Causal Model.

We assume a sample of $n$ units which is drawn from a population distribution.

In i.i.d. fashion.

We observe $ (\mathbf{X}_i, T_i, Y_i) $,
where $\mathbf{X}$ are covariates, 
$T$ is the indicator if treatment has been received
and $Y$ is the observed outcome.

In the Rubin Causal Model we assume that for each unit the potintial outcome exist, i.e. $(Y_i^0, Y_i^1)$ where $Y^1$ stands for the potential outcome had the unit received treatment and $Y^0$ for the potential outcome had the unit received \textbf{no} treatment.

It is clear that $Y_i = Y_i^{T_i}$ i.e. we can observe only one of the potential outcomes.

Thus there is a connection to missing data problems.

This is the dillema of causal inference.
 
On the population level it is possible to estimate both.

Usually the means of the potential outcomes are compared against each other.

In randomized trials this is a valid approach to causal inferece.

In observational studies however the treatment assignment is not known and direct comparison can lead to systematically wrong results.

This phenomenon is called \textbf{confounding}.
 
To address the issue of confounding many methods have been proposed.

%\section{Double Robustness}
%read the paper \cite{Zhao2017a} for discussion of double robustness for balancing weights and \cite{Hahn1998} for semiparametric efficiency bounds.


What is the best rate we can achieve?

\begin{theorem}
  \emph{(Hartman-Winter)}
  Let 
  $
    X_1,
    X_2,
    \ldots
  $
  be i.i.d. real random variables with 
  $
    \E[X_1]=0
    \ 
    \text{and}
    \ 
    \mathbf{Var}
    [X_1]
    = 1.
  $
  Let
  $
    S_n
    :=
    X_1
    +
    \ldots
    +
    X_n,
    \ 
    n\in \mathbb{N}
    .
  $
  Then
  \begin{gather}
    \limsup_{n\to\infty}
    \frac{S_n}{
      \sqrt{
        2n
        \log
        \log
        n
      }
    }
    =
    1
    \quad
    \text{a.s.}
  \end{gather}
\end{theorem}
\begin{proof}
  \cite[Theorem~22.11]{Klenke2020}
\end{proof}


We calculate mean and variance

\begin{align}
  \E
  \left[ 
    (T/\pi(X))
    Y(T)
  \right]
  &=
  \E
  \left[ 
    Y(1)
    /\pi(X)
    \vert
    T=1
  \right]
  \P[T=1]
  \\
  &=
  \int_\mathcal{X}
  \left[ 
    Y(1)
    \vert
    T=1,
    X=x
  \right]
  \cdot
  \P[T=1]
  /
  \pi(x)
  \P_{X|T}(dx|1)
  \\
  &=
  \int_\mathcal{X}
  \left[ 
    Y(1)
    \vert
    X=x
  \right]
  \P_X(dx)
  =
  \E[Y(1)]
  .
\end{align}
The third equality is due to weak unconfoundedness and Bayes rule.
With the same arguments as above it follows
$
  \mathbf{Var}
  [
    (T/\pi(X))
    Y(T)
  ]
    =
    \E[
    (Y(1))^2
    /
    \pi(X)
    ]
    -
    \E[Y(1)]^2
    =:
    \mathbf{V}^*
    .
$
We readily calculate the learning rate 
\begin{gather}
  \sqrt{
    \mathbf{V}^*
    \frac{\log\log n}{n}
  }
  .
\end{gather}

%  \section{Error Decompositions}
%  The following decomposition is flexible in $\Phi.$
We get different causal estimaands $\E[\Phi(Y(1))],$
e.g.
the popolation average of $Y(1)$
for 
$
  \Phi(Y)=Y,
$
i.e.
$
\E[Y(1)]
,
$
ot the distribution function of $Y(1)$ at $t$
for 
$
  \Phi(Y)=
  \mathbf{1}_{(-\infty, t]}
  (Y)
  ,
$
i.e.
$
\P[
Y(1)\le t
]
.
$
\begin{gather}
 \sum_{i=1}^{n}  
 w_i
 T_i
 \Phi(Y_i)
  - \E [ \Phi(Y(1)) ]
  =
  \frac{1}{n}
  \sum_{ i = 1 }^{n} S_i
    + R_0
    + R_1
    + R_2
    ,
\end{gather}
where
\begin{align*}
  S_i 
  &:= 
  \frac{T_i}{\pi_i}
 \left( 
   \Phi(Y_i) - \E [ \Phi(Y_i(1)) | X_i ]
 \right)
 +
 \left( 
   \E [ \Phi(Y_i(1)) | X_i ] - \E [ \Phi(Y(1))]
 \right)
 \quad
 \text{for}\ 
 i \in \left\{ 1, \ldots, n, \right\}
 ,
 \\
  R_0
  &:=
  \sum_{ i = 1 }^{n}
  T_i
    \left(  
      w_i - \frac{1}{n \pi_i}
    \right)
 \left( 
   \Phi(Y_i) - \E [ \Phi(Y_i(1)) | X_i ]
 \right)
 ,\\
  R_1
  &:=
  \sum_{ i = 1 }^{n}
    \left(  
      T_i
      w_i - \frac{1}{n}
    \right)
 \left( 
    \E [ \Phi(Y_i(1)) | X_i ] - B(X_i)^\top \lambda
 \right)
 ,
 \\
  R_2
  &:=
  \sum_{ i = 1 }^{n}
    \left(  
      T_i
      w_i - \frac{1}{n}
    \right)
 B(X_i)^\top \lambda
 \quad 
 \text{for}
 \ 
 \lambda \in \R^K
 .
\end{align*}

We can even view
$
\frac{1}{\sqrt{n}}
\sum_{i=1}^{n}S_i 
$
as an empirical process 
$
\mathbb{G}_n f
$
indexed over 

\begin{gather}
  f_\Phi(T,X,Y)
  =
  \frac{T}{\pi(X)}
 \left( 
   \Phi(Y) - \E [ \Phi(Y) | X ]
 \right)
 +
   \E [ \Phi(Y) | X ] 
   .
\end{gather}
If $\mathcal{F}=\left\{ f_\Phi \colon \Phi \in \ \text{some set}\right\}$
is $\P$-Donsker, the empirical process converges to a tight gaussian process.
Then the functional delta Method is applicable.

%  %\section{Estimating the Population Mean of Potential Outcomes}
%  %%%%%%%%%%%%%%%%%%%%ASSUMPTION 1%%%%%%%%%%%%%%%%%%%%%%%%%%%%%%%%%%%%%%%
\begin{asu}
  \label{assumption_1}
  Assume, the following conditions hold:
\\
  \subasu 
    \label{assumption_1_i} 
    The minimizer 
    $
    \lambda_0 
    =
    \arg \min_{\lambda \in \Theta}
    \E
    \left[ 
      -T n 
      \rho 
      \left( 
      B(X)^T \lambda
      \right)
      +
      B(X)^T \lambda
    \right]
    $
    is unique,
    where 
    $\Theta \subseteq \R^n$ is the parameter space for $\lambda$.
\\ %%%%%%%%%%%%%%%%%%%%%%%%%%%%%%%%%%%%%%%%%%%%%%%%%%%%%%%%%%%%%%%%%
  \subasu 
    \label{assumption_1_ii} 
    The parameter space 
    $\Theta \subseteq \R^n$
    is compact compact with diameter
    $\text{diam}(\Theta) < \infty$.
\\ %%%%%%%%%%%%%%%%%%%%%%%%%%%%%%%%%%%%%%%%%%%%%%%%%%%%%%%%%%%%%%%%%%%
  \subasu 
    \label{assumption_1_iii}
    $\lambda_0 \in \text{int}(\Theta)$,
    where
    $\text{int}(\cdot)$
    stands for the interior of a set.
\\ %%%%%%%%%%%%%%%%%%%%%%%%%%%%%%%%%%%%%%%%%%%%%%%%%%%%%%%%%%%%%%%%%%%
  \subasu
    \label{assumption_1_iv}
    There exists
    $\lambda^*_1 \in \Theta$
    such that
    $
      \norm{
        m^*(\cdot)
        -
        B(\cdot)^T \lambda^*_1
      }_\infty
      \le 
      \varphi_{m^*}
    $,
    where
    $
      m^*(\cdot)
      :=
      \left( \rho^{'} \right)^{-1}
      \left( \frac{1}{n \pi(\cdot)} \right).
    $
\\ %%%%%%%%%%%%%%%%%%%%%%%%%%%%%%%%%%%%%%%%%%%%%%%%%%%%%%%%%%%%%%%%%%%%%
  \subasu
    \label{assumption_1_v}
    There exists a constant 
    $
      \varphi_{\rho^{'} \lor \pi} 
      \in 
      \left(0, \frac{1}{2} \right)
    $
    such that
    $
      n\rho(v) 
      \in 
      (
      \varphi_{\rho^{'} \lor \pi},
      1 - \varphi_{\rho^{'} \lor \pi}
      )
    $
    for $v=B(x)^T \lambda$ with $\lambda \in \text{int}(\Theta)$ 
    \textbf{or}
    $
      \pi(x)
      \in 
      (
      \varphi_{\rho^{'} \lor \pi},
      1 - \varphi_{\rho^{'} \lor \pi}
      )
    $.
\\ %%%%%%%%%%%%%%%%%%%%%%%%%%%%%%%%%%%%%%%%%%%%%%%%%%%%%%%%%%%%%%%
  \subasu
    \label{assumption_1_vi}
    There exists 
    $ \varphi_{\rho^{''}} > 0 $
    such that
    $ -\rho^{''} \ge \varphi_{\rho^{''}} > 0 $
\\ %%%%%%%%%%%%%%%%%%%%%%%%%%%%%%%%%%%%%%%%%%%%%%%%%%%%%%%%%%%%%%%%
  \subasu
    \label{assumption_1_vii}
    There exists 
    $ \varphi_{B(x) B(x)^T} > 0 $ 
    such that
    $
      B(x) B(x)^T 
      \succcurlyeq 
      \varphi_{B(x) B(x)^T} I 
    $
\\ %%%%%%%%%%%%%%%%%%%%%%%%%%%%%%%%%%%%%%%%%%%%%%%%%%%%%%%%%%%%%%%%
  \subasu
    \label{assumption_1_viii}
    There exists
    $ \varphi_{\norm{B}} > 0 $
    such that
    $
      \sup_{x \in \mathcal{X}} \norm{B(x)}_2
      \le 
      \varphi_{\norm{B}}
    $.
\end{asu}

%%%%%%%%%%%%%%%%%%%%%%%%%%%%%%%%%%%%%%%%%%%%%%%%%%%%%%%%%%%%%%%%%%

We study the following problem:

\begin{align}
  \label{primal_weighting_binary}
  \begin{split}
  &\underset{w \in \R^n}{\text{minimize}}
  \qquad
  \sum_{i = 1}^{n} T_i f(w_i)
  \\
  &\text{subject to}
  \left| 
    \sum_{i = 1}^{n} w_i T_i B_k(X_i)
    - 
    \frac{1}{n} \sum_{i = 1}^{n} B_k(X_i)
  \right|
  \le 
  \delta_k,\,
  k = 1, \ldots, K
  \end{split}
\end{align}


\begin{proposition}
  \label{ch_1_dual}
  The dual of Problem \eqref{primal_weighting_binary} is equivalent to the unconstrained optimization problem
  \begin{gather}
    \label{dual_weighting_binary}
      \underset{\lambda \in \R^K}{\text{minimize}}
      \quad
      \frac{1}{n}
      \sum_{j = 1}^{n} 
      \left[ 
        -T_j n 
        \rho 
        \left( 
          B(X_j)^T \lambda
        \right)
      +
      B(X_j)^T \lambda
      \right]
      +
      |\lambda|^T \delta
  \end{gather}
\end{proposition}


\begin{proposition}
  \label{ch_1_near_oracle}
  There exists a solution $\lambda^\dagger$ 
  to \eqref{dual_weighting_binary}
  such that
  \begin{gather}
    \P
    \left( 
      \norm{
        \lambda^\dagger
        -
        \lambda^*_1
      }_2
      \le
      \CP \Ctau \LearnRate
    \right)
    \ge 
    1 - \tau
    .
  \end{gather}
\end{proposition}


\section{Plan of proof}
We employ 
Theorem~\ref{cvxa_fenchel_theorem}
together with the box constraints in Problem~\eqref{primal_weighting_binary}
to obtain Proposition~\ref{ch_1_dual}.

To prove Proposition~\ref{ch_1_near_oracle}
we employ
Proposition~\ref{syu_1_result}
and 
Corollary~\ref{syu_taylor_corollary}
to get

\begin{align}
  \begin{split}
  & 
  G(\lambda^*_1 + \Delta) 
  -
  G(\lambda^*_1)
  \\
  &\ge
      \frac{1}{n}
      \sum_{j = 1}^{n} 
      \left[ 
        -T_j n 
        \rho^{'} 
        \left( 
          B(X_j)^T \lambda^*_1
        \right)
      +
      1
      \right]
      \Delta^T B(X_j)
      \\
  & +
      \frac{1}{2}
      \sum_{j = 1}^{n} 
        -T_j  
        \rho^{''} 
        \left( 
          B(X_j)^T (\lambda^*_1 + \xi \Delta)
        \right)
        \Delta^T
        \left( 
          B(X_j)
          B(X_j)^T
        \right)
        \Delta
        \\
  &-
      |\Delta|^T \delta
  \\
  &\ge
    - \norm{\Delta}_2
    \left( 
    \norm{
      \frac{1}{n}
      \sum_{j = 1}^{n} 
      \left[ 
        -T_j n 
        \rho^{'} 
        \left( 
          B(X_j)^T \lambda^*_1
        \right)
      +
      1
      \right]
      B(X_j)
    }_2
    +
    \norm{\delta}_2
    \right)
    \\
  &+
  n
  \norm{\Delta}^2_2
   \varphi_{\rho^{''}}
  \underline{\varphi_{aa^T}}
  \end{split}
\end{align}

Next we employ Bernstein inequality~\ref{rmineq_bernstein} to bound
\begin{align}
    \norm{
      \frac{1}{n}
      \sum_{j = 1}^{n} 
      \left[ 
        -T_j n 
        \rho^{'} 
        \left( 
          B(X_j)^T \lambda^*_1
        \right)
      +
      1
      \right]
      B(X_j)
    }_2
    \le
    \CP \Ctau \LearnRate
\end{align}
with probability $1 - \tau$.
Then for 
$\norm{\Delta}_2$ large enough it holds
\begin{gather}
  G(\lambda^*_1 + \Delta) 
  -
  G(\lambda^*_1)
  >
  0
\end{gather}
with probability $1 - \tau$.
Thus by Proposition~\ref{syu_1_result}
  \begin{gather}
    \P
    \left( 
      \norm{
        \lambda^\dagger
        -
        \lambda^*_1
      }_2
      \le
      \norm{\Delta}_2
    \right)
    \ge 
    1 - \tau
    .
  \end{gather}
It is then straightforward to prove

\begin{theorem}
  Let 
  $\lambda^\dagger$
  be the solution to Problem~\ref{dual_weighting_binary}
  and 
  $w^*(x)=\rho^{'}\left( B(x)^T \lambda^\dagger \right)$.
  Then under the conditions in Assumption~\ref{assumption_1}
  it holds

  \begin{align}
  \norm{
    w^*(\cdot)
    -
    \frac{1}{n \pi(\cdot)}
  }_{\P, 2}
  \le 
  \text{stuff}
  \end{align}
and 
  \begin{align}
    \P
    \left( 
  \norm{
    w^*(\cdot)
    -
    \frac{1}{n \pi(\cdot)}
  }_\infty
  \le 
  \text{stuff}
   \right)
   \ge
  1 - \tau
  .
  \end{align}
\end{theorem}


\begin{proof}
  Motivated by Proposition~\ref{syu_1_result}
  we set
  $\norm{\Delta}_2 = C$ 
  and consider
 \begin{gather}
   G(\lambda)
   :=
      \frac{1}{n}
      \sum_{j = 1}^{n} 
      \left[ 
        -T_j n 
        \rho 
        \left( 
          B(X_j)^T \lambda
        \right)
      +
      B(X_j)^T \lambda
      \right]
      +
      |\lambda|^T \delta.
 \end{gather} 
 Since 
 $\rho \in C^2(\R)$
 we can employ 
 Proposition~\ref{syu_1_result},
 Corollary~\ref{syu_taylor_corollary}
 and
 Proposition~\ref{syu_triangle}
 to get
 \begin{align}
  \begin{split}
  & 
  G(\lambda^*_1 + \Delta) 
  -
  G(\lambda^*_1)
  \\
  &\ge
      \frac{1}{n}
      \sum_{j = 1}^{n} 
      \left[ 
        -T_j n 
        \rho^{'} 
        \left( 
          B(X_j)^T \lambda^*_1
        \right)
      +
      1
      \right]
      \Delta^T B(X_j)
      \\
  & +
      \frac{1}{2}
      \sum_{j = 1}^{n} 
        -T_j  
        \rho^{''} 
        \left( 
          B(X_j)^T (\lambda^*_1 + \xi \Delta)
        \right)
        \Delta^T
        \left( 
          B(X_j)
          B(X_j)^T
        \right)
        \Delta
        \\
  &-
      |\Delta|^T \delta
  \\
  &\ge
    - \norm{\Delta}_2
    \left( 
    \norm{
      \frac{1}{n}
      \sum_{j = 1}^{n} 
      \left[ 
        -T_j n 
        \rho^{'} 
        \left( 
          B(X_j)^T \lambda^*_1
        \right)
      +
      1
      \right]
      B(X_j)
    }_2
    +
    \norm{\delta}_2
    \right)
    \\
  &+
  n
  \norm{\Delta}^2_2
   \varphi_{\rho^{''}}
  \underline{\varphi_{aa^T}}
  \\
  &:=
  -\norm{\Delta}_2
  (I_1 + \norm{\delta}_2)
  +
  \norm{\Delta}^2_2
  I_2
  .
  \end{split}
\end{align}
The second inequality is due to 
the Cauchy-Schwarz-Inequality 
and
Assumptions~\ref{assumption_1_vi} and \ref{assumption_1_vii}
.
\subsection*{Analysis of $I_1$}
We want to use Assumption~\ref{assumption_1_iii}.
Thus we perform the following split:
\begin{align}
  I_1 
  &\le
    \norm{
      \sum_{j = 1}^{n} 
        T_j  
      \left[ 
        \rho^{'} 
        \left( 
          B(X_j)^T \lambda^*_1
        \right)
      -
      \frac{1}{n \pi(X_j)}
      \right]
      B(X_j)
    }_2
  \\
  &+
    \norm{
      \frac{1}{n}
      \sum_{j = 1}^{n} 
      \left[ 
        \frac{T_j}{\pi(X_j)}
      -
      1
      \right]
      B(X_j)
    }_2
    \\
  &=:
  J_1 + J_2
\end{align}

\subsubsection*{Analysis of $J_1$}

By the Lipschitz-continuity of 
$\rho^{'}$,
Assumption~\ref{assumption_1_viii}
and
Assumption~\ref{assumption_1_iv},
$T \in \{0, 1\}$
and 
the triangle inequality 
we have
\begin{gather}
  J_1 
  \le
  n L_{\rho^{'}}\varphi_{\norm{B(x)}} \varphi_{m^*}
\end{gather}

\subsubsection*{Analysis of $J_2$}
We employ Bernstein Inequality for matrices
To this end we define
\begin{gather}
  A_j
  :=
      \frac{1}{n}
      \left[ 
        \frac{T_j}{\pi(X_j)}
      -
      1
      \right]
      B(X_j)
\end{gather}

\subsubsection*{$\E A_j = 0$}

It holds

\begin{gather}
  \E
  \left[  
    \frac{T_j}{\pi(X_j)}
    B(X_j)
  \right]
  =
  \E
  \left[  
    \E
    \left[  
      T_j
      \, | \,
      X_j
    \right]
    \frac{1}{\pi(X_j)}
    B(X_j)
  \right]
  =
  \E[B(X_j)]. 
\end{gather}
Thus 
$\E[A_j] = 0$.
\subsubsection*{L}
Since
\begin{gather}
  \left| 
      \frac{T_j}{\pi(X_j)}
      -
      1
      \right|
  \le
  1 + \frac{1 - \varphi_{\pi}}{\varphi_{\pi}}
  =
  \frac{1}{\varphi_{\pi}}
\end{gather}
by Assumption~\ref{assumption_1_v},
we can employ Assumption~\ref{assumption_1_viii}
to get
\begin{gather}
  \norm{A_j}_2
  \le
  \frac{\varphi_{\norm{B}}}{n \varphi_{\pi}}
  =:
  L.
\end{gather}
\subsubsection*{v(S)}
Since
\begin{gather}
  \E
  \left[ 
    A_j A_j^T
  \right]
  \le
  \left( 
    \frac{1}{n \varphi_{\pi}}
  \right)^2
  \E
  \left[ 
    B(X) B(X)^T
  \right]
\end{gather}
and
\begin{gather}
  \E
  \left[ 
    A_j^T A_j
  \right]
  \le
  \left( 
    \frac{\varphi_{\norm{B}}}{n \varphi_{\pi}}
  \right)^2
\end{gather}
we have
\begin{gather}
  v(S)
  \le
  \frac{|\lambda_{\max} | + \varphi_{\norm{B}}^2}{n \varphi_{\pi}^2},
\end{gather}
where 
$\lambda_{\max}$ is the maximal eigenvalue of
$
  \E
  \left[ 
    B(X) B(X)^T
  \right]
$.
Then by Bernsteins inequality~\ref{rmineq_bernstein}
we get
\begin{gather}
  \E[J_2]
  \le
  \sqrt{
    \frac{
    2 \log (K + 1)
    \left( 
      |\lambda_{\max} | + \varphi_{\norm{B}}^2
    \right)
    }
    {
      n \varphi_{\pi}^2
    }
  }
  +
  \frac{
    \log (K + 1)
    \varphi_{\norm{B}}
  }
  {
    3 n \varphi_{\pi}
  }
\end{gather}
and by the Markov-inequality
\begin{gather}
  \P
  \left( 
    J_2 
    \le
    \frac{1}{\tau}
    \E[J_2]
  \right)
  \ge 
  1 - \tau
\end{gather}
\end{proof}

%  \section{Application of Matrix Concentration Inequalities}
%  \subsubsection*{
  Analysis of 
  $
  \E[\max_{i \le n}\norm{A_i}^2]
  $
}
We start from the premise that the fourth moment of 
the random quantities $B_k(X_i)$ and $1/\pi_i$ is uniformly bounded 
in $k$ and $i$.
\begin{assumptions*}
  \begin{enumerate}[label={(\roman*)}]
    \item
  There exists 
  a constant 
  $
  \,
    C_{\!\scriptscriptstyle B}
  \ge
  1
  \,
  $ such that
  \begin{gather*}
  \E\left[
    B_k(X_i)^{4}
    \,
  \right]
  \ 
  \le
  \ 
    C_{\!\scriptscriptstyle B}
    \quad
    \text{
  for all $\,(k,i)\ \in\  \left\{ 1, \ldots, K \right\}\times \left\{ 1, \ldots, n \right\}$.
    }
  \end{gather*}
  \item
  There exists a constant $\,C_\pi \!\ge 1\,$ such that
  \begin{gather*}
  \E \left[ 
    \left(
      \frac{1}{\pi_{i}}
    \right)^{\!4}
    \,
  \right]
  \ 
  \le
  \ 
  C_\pi
  \quad
  \text{
  for all $\,i\ \in\  \left\{ 1, \ldots, n \right\}$.
  }
  \end{gather*}
  \end{enumerate}
\end{assumptions*}
Note, that these assumptions allow for random covariate distributions with unbounded support. The coming example ought to reinforce this observation.
\begin{remark}
  If we assume a logistic regression model for the propensity score
  it holds for some $\theta \in \R^N$ ($N$ is the number of covariates)
  \begin{gather}
    \label{rmineq_rp_1}
    \frac{1-\pi(X)}{\pi(X)}
    =
    \exp(-\theta X)
    \qquad
    \text{and}
    \qquad
  \end{gather}
  \begin{gather}
  \E \left[ 
    \left(
      \frac{1}{\pi_{i}}
    \right)^{\!4}
    \,
  \right]
    =
    \E
    [
    \exp(-4\theta X)
    ]
    =
    M_X(-4\theta)
    ,
  \end{gather}
  where $M_X$ is the momement-generating function of $X$.
   While the first quantity in \eqref{rmineq_rp_1}
   may be unbounded when $X$ has unbounded support, the latter quantity in \eqref{rmineq_rp_1} is still bounded for reasonable choices of $X$.
\end{remark}

Next, we recall the entity we want to examin.
\begin{gather*}
  A_i
  \ 
  =
  \ 
  \frac{1}{n}
  \,
  \left( 
    \,
    1
    \ 
    -
    \ 
    \frac{T_i}{\pi_i}
    \,
  \right)
  \,
  B(X_i)
  \qquad
  \text{for}
  \ 
  i\in \{1, \ldots, n\}\,.
\end{gather*}
For all
$
  i\in \{1, \ldots, n\}
$
we get the bound
\begin{gather}
  \label{1:5:1}
  \left| 
    \,
    1
    \ 
    -
    \ 
    \frac{T_i}{\pi_i}
    \,
  \right|
  \ 
  \le
  \ 
  \left( 
    \,
  1
  \ 
  \lor
  \ 
  \frac{1-\pi_i}{\pi_i}
  \,
  \right)
  \ 
  \le
  \ 
  1
  \ 
  +
  \ 
  \frac{1-\pi_i}{\pi_i}
  \ 
  =
  \ 
  \frac{1}{\pi_i}
  \,.
\end{gather}
Let
$i^*\in \left\{ 1, \ldots, n \right\}$
be the index where 
$
\norm{A_i}
$
attains its maximum.
\begin{align}
  \label{1:5:2}
  \begin{split}
  \E\left[\max_{i \le n}\norm{A_i}^2\right]
  &
  \ 
  =
  \ 
  \E\left[\norm{A_{i^*}}^2\right]
  \ 
  \le
  \ 
  \E \left[ 
    \left(
      \,
      \frac{
    \norm{B(X_{i^*})}
      }{\pi_{i^*}}
      \,
    \right)
    ^{\!2}
    \,
  \right]
  \ 
  /
  \ 
  n^2
  \\
  &
    \ 
  \le
  \ 
  \E \left[ 
    \left(
      \frac{1}{\pi_{i^*}}
    \right)^{\!4}
    \,
  \right]^{1/2}
  \ 
  \cdot
  \ 
  \E\left[
    \norm{B(X_{i^*})}^4
  \right]^{1/2}
  \,
  /
  \ 
  n^2
  \\
  &
  \ 
  \le
  \ 
  \,
  K
  \,
  /
  \,
  n^2
  \ 
  \cdot
  \ 
  \sqrt{
    \,
    C_\pi
    C_{\!\scriptscriptstyle B}
  }
  \,.
\end{split}
\end{align}
The first inequality comes from the bound \eqref{1:5:1}.
The Cauchy-Schwarz inequality provides the second inequality.
In the last step we use the assumptions made at the start of the section. 
Paying the price of an extra $n$ factor, the 
maximal inequality~\eqref{1:5:2}
yields a bound of the sum, that is,
\begin{gather}
  \sum_{i=1}^{n}
  \,
  \E\left[\norm{A_i}^2\right]
  \ 
  \le
  \ 
  \frac{K}{n}
  \,
  \sqrt{\,C_\pi 
    C_{\!\scriptscriptstyle B}
  }
\end{gather}
\begin{assumption}
  .
\end{assumption}
 
\begin{assumption}
  .
\end{assumption}
\begin{remark}
With Assumption we also get a bound on the fourth moment of 
  $
  \norm{B(X_{i})}
  $. Indeed, by the convexity of 
  $x\mapsto x^2$, the monotonicity and linearity of the expectation it holds   
  \begin{align}
    \begin{split}
  \E[
  \norm{B(X_{i})}^4
  ] 
  &=
  \E
  \left[ 
    \left( 
      \sum_{k=1}^{K}
      B_k^2(X_i)
    \right)^2
  \right]
  =
  K^2
  \E
  \left[ 
    \left( 
      \sum_{k=1}^{K}
      \frac{1}{K}
      B_k^2(X_i)
    \right)^2
  \right]
  \le
  K^2
  \E
  \left[ 
      \sum_{k=1}^{K}
      \frac{1}{K}
      B_k^4(X_i)
  \right]
  \\
  &=
  K
  \sum_{k=1}^{K}
  \E
  \left[ 
      B_k^4(X_i)
  \right]
  \le
  K^2 C_B
  \end{split}
  \end{align}
\end{remark}
\subsubsection*{Analysis of $v(\mathbf{S})$}
We use the fact that 
$
  \norm{A}_2 
  \le
  \norm{A}_F
  =
  \sqrt{
    \sum_{i,j}^{}
    a_{ij}^2
  }
$
It holds
\begin{gather}
  \sum_{i=1}^{n}
  \E[A_iA_i^\top]
  =
  \frac{1}{n^2}
  \sum_{i=1}^{n}
  \E
  \left[ 
    \left( 
      \frac{1-\pi_i}{\pi_i}
    \right)^2
    B(X_i)B(X_i)^\top
  \right]
  =
  \frac{1}{n^2}
  \left( 
    \sum_{i=1}^{n}
    \E
    \left[ 
    \left( 
      \frac{1-\pi_i}{\pi_i}
    \right)^2
    B_k(X_i)B_l(X_i)
    \right]
  \right)
  _{1\le k,l \le K}
  .
\end{gather}
Thus
\begin{align}
  \begin{split}
  &\norm{
  \sum_{i=1}^{n}
  \E[A_iA_i^\top]
  }_2^2
  \\
  &\le
  \norm{
  \sum_{i=1}^{n}
  \E[A_iA_i^\top]
  }_F^2
  =
  \frac{1}{n^4}
  \sum_{k,l=1}^{K}
  \left( 
    \sum_{i=1}^{n}
    \E
    \left[ 
    \left( 
      \frac{1-\pi_i}{\pi_i}
    \right)^2
    B_k(X_i)B_l(X_i)
    \right]
  \right)^2
  \\
  &\le
  \frac{1}{n^4}
  \sum_{k,l=1}^{K}
  \left( 
    \sum_{i=1}^{n}
    \E
    \left[ 
    \left( 
      \frac{1-\pi_i}{\pi_i}
    \right)^4
    \right]
    ^\frac{1}{2}
    \E[
    B_k(X_i)^4
    ]^\frac{1}{4}
    \E[
    B_l(X_i)^4
    ]^\frac{1}{4}
  \right)^2
  \le
  \left(
  \frac{K}{n}
  \right)^2
  C_\pi C_B
  \end{split}
\end{align}
On the other hand
\begin{align}
  \begin{split}
  \norm{
  \sum_{i=1}^{n}
  \E[A^\top_iA_i]
  }_2
  &=
  \sum_{i=1}^{n}
  \E[A^\top_iA_i]
  =
  \frac{1}{n^2}
    \sum_{i=1}^{n}
    \E
    \left[ 
    \left( 
      \frac{1-\pi_i}{\pi_i}
    \right)^2
    \norm{B(X_i)}_2^2
    \right]
    \\
    &\le
  \frac{1}{n^2}
    \sum_{i=1}^{n}
    \E
    \left[ 
    \left( 
      \frac{1-\pi_i}{\pi_i}
    \right)^4
    \right]^\frac{1}{2}
    \norm{B(X_i)}_2^4
    ]^\frac{1}{2}
    \le
    \frac{K}{n}\sqrt{
  C_\pi C_B
    }
  \end{split}
\end{align}
It follows
\begin{gather}
  v(\mathbf{S})
  \le
    \frac{K}{n}\sqrt{
  C_\pi C_B
  }
\end{gather}
Thus we can apply Theorem~\ref{rmineq_rosenthal_pinelis}
to get
\begin{gather}
  \E[\norm{\mathbf{S}}_2]
  \le
  \sqrt{
    2e 
    \frac{K}{n}\sqrt{
  C_\pi C_B
  }
  \log
  (K+1)
  }
  +
  4e
  \frac{\sqrt{K}}{n}
  \sqrt[4]{
  C_\pi C_B
  }
  \log
  (K+1)
  \le
  14
  C_\pi C_B
  \sqrt{
    \frac{K \log(K+1)}{n}
  }
\end{gather}

%
%  \section{Continuous Treatment}
%  We introduce the measure of proximity
\begin{gather}
  d_n(t,s)
  :=
  \frac{
  \mathbf{1} _ 
  { s \in N_n(t) }
}{
  \mathbf{\lambda}[N_n(t)]
}
\end{gather}
where $ N_n(t) $ is a neighborhood of $t$ with 
$  
  \mathbf{\lambda}[N_n(t)]
  \to
  0
  $
  for $ n \to \infty $.
  If we can apply the dominated convergence theorem we get
  \begin{align}
    \E
    \left[ 
      \frac{
        d_n(t,T_i)
      }{
        h_{T|X}(t,X_i)
      }
    \right]
    =
    \E
    \left[ 
      \frac{
        \P
        [ T_i \in N_n(t) | X_i]
      }
      {
        \mathbf{\lambda}[N_n(t)]
      }
      \cdot
      \frac{1}
      {
        h_{T|X}(t,X_i)
      }
    \right]
    \to
    \E
    \left[ 
      \frac
      {
        h_{T|X}(t,X_i)
      }
      {
        h_{T|X}(t,X_i)
      }
    \right]
    =1
    \,.
  \end{align}
  Furthermore, if $ Z_i \sim Y(t)|T_i $
  we have
  \begin{gather}
    \P
    \left[ 
      d_n(t,T_i)
      \left| 
      Y(T_i) - Z_i
      \right|
      \ge
      \varepsilon
    \right]
    =
    \P
    [
      T_i \in N_n(t)
    ]
    \cdot
    \P
    \left[ 
      \left| 
      Y(T_i) - Z_i
      \right|
      \ge
      \varepsilon
      \cdot
      \lambda(N_n(t))
    \right]
    \,.
  \end{gather}
  If $T$ is continuously distributed, 
  $
    \P
    [
      T_i \in N_n(t)
    ]
    \to 0
  $
  which implies 
  $
      d_n(t,T_i)
      \left( 
      Y(T_i) - Z_i
      \right)
      \to
      0
  $
  in probability. We shall employ concentration inequalities to 
  also derive learning rates.
  The optimization problem is for fixed
  $ t\in \mathcal{T} $
  \begin{gather*}
    \underset{w_1, \ldots, w_n \in \R}{\mathrm{minimize}}
    \qquad
    \sum_{i = 1}^{n}
    d_n(t,T_i)
    f(w_i)
  \end{gather*}
subject to the constraints
\begin{gather*}
%    w_i T_i \ge 0,
%   \qquad 
%    i = 1, \ldots, n,
%   \\
%  \sum_{ i = 1 }^{n}
%    w_i T_i
%  =
%  1
%  \\
    \left| 
      \frac{1}{n} 
      \sum_{i = 1}^{n} 
      (
      w_i  
      \cdot
    d_n(t,T_i)
      - 
      1
      )
      \cdot
      B_k(X_i)
    \right|
    \ 
    \le 
    \ 
    \delta_k,
    \qquad
    k = 1, \ldots, K
\end{gather*}
\pagebreak
We shall derive
\begin{gather}
  d_n(t,T_i)
  \cdot
  \left| 
  w_i
  -
  \frac{1}{h_{T|X}(t,X_i)}
  \right|
  \to 0
\end{gather}
in probability for all 
.
From this we control the error
\begin{align*}
  &\left| 
  \frac{1}{n}
  \sum_{i=1}^{n} 
  d_n(t,T_i)
  w_i
  Y_i
  -
  \E[Y(t)]
  \right|
  \\
  &
  \ 
  \le
  \ 
  \left|  
  \frac{1}{n}
    \sum_{i=1}^{n} 
    (
    w_i 
  d_n(t,T_i)
    -
    1
    )
    \inner
    {B(X_i))}
    { \mathbf{Y}(t) }
  \right|
  %%%% 1 %%%%
  \\
  &
  \qquad
  +
  \ 
  \left|  
  \frac{1}{n}
    \sum_{i=1}^{n} 
    (
    w_i 
  d_n(t,T_i)
    -
    1
    )
    \left( 
    \E[Y(t)| X_i]
    -
    \inner
    {B(X_i))}
    { \mathbf{Y}(t) }
    \right)
  \right|
  %%%% 2 %%%%
  \\
  &
  \qquad
  +
  \ 
  \left|  
  \frac{1}{n}
    \sum_{i=1}^{n} 
  d_n(t,T_i)
    \cdot
    (
    w_i 
    -
    1/
h_{T|X}(t,X_i)
    )
    \left( 
      Z_i
    -
    \E[Y(t)| X_i]
    \right)
  \right|
  %%%% 3 %%%%
  \\
  &
  \qquad
  +
  \ 
  \left|  
  \frac{1}{n}
    \sum_{i=1}^{n} 
h_{T}(t)
    /
h_{T|X}(t,X_i)
    \left( 
      Z_i
    -
    \E[Y(t)| X_i]
    \right)
  \right|
  %%%% 3a %%%%
  \\
  &
  \qquad
  +
  \ 
  \left|  
  \frac{1}{n}
    \sum_{i=1}^{n} 
    \left( 
h_{T}(t)
-
  d_n(t,T_i)
    \right)
    /
h_{T|X}(t,X_i)
    \left( 
      Z_i
    -
    \E[Y(t)| X_i]
    \right)
  \right|
  %%%% 3 b %%%%
  \\
  &
  \qquad
  +
  \ 
  \left|  
  \frac{1}{n}
    \sum_{i=1}^{n} 
    \left( 
    \E[Y(t)| X_i]
    -
    \E[Y(t)]
    \right)
  \right|
  %%%% 3 c %%%%
  \\
  &
  \qquad
  +
  \ 
  \left|  
  \frac{1}{n}
    \sum_{i=1}^{n} 
    w_i 
d_n(t,T_i)
    \left( 
      Y_i
    -
    Z_i
    \right)
  \right|
\end{align*}
%%%%%%%%%%%   w_i %%%%%%%%%%
\begin{align*}
&
d_n(t,T_i)
   \cdot
   \left| 
   w_i
   -
   1 / 
h_{T|X}(t,X_i)
   \right|
   \\
   & 
   \ 
   =
   \ 
d_n(t,T_i)
   \cdot
   \left| 
   (
      f ^ { ' }
   )
   ^ {-1}
   \inner
   {B(X_i)}
   { \lambda ^ \dagger }
   -
     (f ^ { ' }) ^ { -1 }
     \left( 
      f ^ { ' }
      \left( 
   1 / 
h_{T|X}(t,X_i)
      \right)
     \right)
   \right|
   %%%% 1 %%%%
   \\
   & 
   \ 
   \le
   \ 
d_n(t,T_i)
   \left| 
   (
      f ^ { ' }
   )
   ^ {-1}
   \inner
   {B(X_i)}
   { \lambda ^ \dagger }
   -
   (
      f ^ { ' }
   )
   ^ {-1}
          \inner
          {B(X_i)}
          {
      f ^ { ' }
      \left( 
   1 / 
h_{T|X}(t,X)
      \right)
          }
   \right|
   %%%% 1 a %%%%
   \\
   &
   \qquad
   +
   \ 
d_n(t,T_i)
   \left| 
   (
      f ^ { ' }
   )
   ^ {-1}
          \inner
          {B(X_i)}
          {
      f ^ { ' }
      \left( 
   1 / 
h_{T|X}(t,X)
      \right)
          }
   -
     (f ^ { ' }) ^ { -1 }
     \left( 
      f ^ { ' }
      \left( 
   1 / 
h_{T|X}(t,X_i)
      \right)
     \right)
   \right|
   %%%% 2 %%%%
   \\
   &
   \ 
   \le
   \ 
   \omega
   \left( 
     (f ^ { ' }) ^ { -1 }
     ,
     \norm{ \lambda ^ \dagger 
     - 
      f ^ { ' }
      \left( 
   1 / 
h_{T|X}(t,X)
      \right)
 }
   \right)
   \ 
   +
   \ 
   \omega
   \left( 
     (f ^ { ' }) ^ { -1 }
     ,
     \varepsilon _ m
   \right)
   %%%% 3 %%%%
   \\
   &
   \ 
   \le
   \ 
   \omega
   \left( 
     (f ^ { ' }) ^ { -1 }
     ,
     \varepsilon _ \dagger
   \right)
   \ 
   +
   \ 
   \omega
   \left( 
     (f ^ { ' }) ^ { -1 }
     ,
     \varepsilon _ m
   \right)
   %%%% 4 %%%%
   \\
   &
   \ 
   \le
   \ 
   \varepsilon
 \end{align*}
 with probability tending to 1.
%%%%%%%%%%%%%%%%%%%%%%%
\begin{align*}
   &
     G
     (
      f ^ { ' }
      \left( 
   1 / 
h_{T|X}(t,X)
      \right)
      +
      \Delta
     )
     \ 
     -
     \ 
     G
     (
      f ^ { ' }
      \left( 
        1 / \pi 
      f ^ { ' }
      \left( 
   1 / 
h_{T|X}(t,X)
      \right)
      \right)
     )
          %%%% 2 %%%%
     \\
     &
     \quad
     \ge
     \ 
     -
     \norm{\Delta}
     \left( 
     \norm{\delta}
     \ 
     +
     \ 
     \norm{B(X_i)}
     \cdot
     \frac{1}{n}
     \sum_{i=1}^{n} 
     \left| 
     \,
      1
     \ 
      -
     \ 
     d_n(t,T_i)
        \cdot
     (f ^ { ' }) ^ { -1 }
          \inner
          {B(X_i)}
          {
      f ^ { ' }
      \left( 
   1 / 
h_{T|X}(t,X)
      \right)
          }
          \,
     \right|
     \right)
 \end{align*}


\chapter{Convex Analysis}
In our application we want to analyse a convex optimization problem by its dual problem.
In particular we want to obtain primal optimal solutions from dual solutions.
To accomplish the task we need technical tools from convex analysis, 
mainly conjugate calculus and some KKT related results.

Our starting point is 
the support function intersection rule~\cite[Theorem 4.23]{Mordukhovich2022}.
We give the details in the case of finite dimensions and refer for the rest of the proof to the book.
The support function intersection rule is applied to give first conjugate sum and then chain rule,
which are vital to calculating convex conjugates. The proofs are omited, since the book is thorough enough. 
%The well known Fenche-Rockafellar Duality theorem is a corollary of conjugate sum and chain rule. It gives general conditions under which dual and primal values coincide.
The material we present is very well known.
As an introduction, we recommend the recent book \cite{Mordukhovich2022} and classical reference \cite{Rockafellar1970}.
We finish the chapter with ideas from \cite{Tseng1991}. 
They provide the high-level ideas to obtain for strictly convex
functions a dual relationship between optimal solutions.
We will deliver the details that are omited in the paper.
  \section{A Convex Analysis Primer}
  \subsection*{My Contribution}
I present the relevant facts from Convex analysis.
I prove some results that I did not find in the literature, but likely are folklore.

Throughout this section let $n\in\mathbb{N}$.
\subsection*{Sets}
A subset $C\subseteq \R^n$ is called \textbf{convex set}, 
if for all $x,y\in C$ and all $\theta\in [0,1]$,
we have 
$
  \theta x + (1-\theta)y 
  \in
  C
$.
Many set operations preserve convexity. Among them
forming the 
\textbf{Cartesian product} of two convex sets, 
\textbf{intersection} of a collection of convex sets and 
taking the \textbf{inverse image under linear functions}.

The classical theory evolves around the question 
if convex sets can be separated.
\begin{definition*}
  Let 
  $C_1$ and $C_2$
  be two non-empty convex sets in $\R^n$. 
  A hyperplane $H$ is said to \textbf{separate}
  $C_1$ and $C_2$
  if $C_1$ is contained in one of the closed half-spaces associated with
  $H$ and $C_2$ lies in the opposite closed half-space. It is said to separate 
  $C_1$ and $C_2$
  \textbf{properly} if 
  $C_1$ and $C_2$
  are not both contained in $H$.
\end{definition*}

We need a refined concept of interiors, since some convex sets have empty interior. To this end, 
  we call a set
  \index{affine set}
  $A\subseteq \R^n$ 
  \textbf{affine set}, if
  $
    \alpha x + (1-\alpha)y \in A
    \quad
    \text{for all}
    \ 
    x,y \in A
    \ 
    \text{and all}
    \ 
    \alpha \in \R
  $.
  The \textbf{affine hull} 
  \index{$\mathrm{aff(\cdot)}$, affine hull}
  $\mathrm{aff}(\Omega)$
  of a set 
  $\Omega\subseteq \R^n$
  is the smallest affine set that includes $\Omega$.
  We define the \textbf{relative interior}
  \index{$\mathrm{ri}(\cdot)$, relative interior}
  $\mathrm{ri}\,\Omega$ 
  of a set 
  $\Omega\subseteq \R^n$
  to be the interior relative to the affine hull, that is,
    \begin{gather}
    \mathrm{ri}(\Omega)
    \ 
    :=
    \ 
    \left\{ 
      x \in \Omega 
      \ 
      |
      \ 
      \exists
      \,
      \varepsilon > 0\ 
      \colon
      (
      x+\varepsilon B_{\R^n}
      )
      \cap
      \mathrm{aff}(\Omega)
      \ 
      \subset
      \ 
      \Omega
      \,
    \right\}
    \,.
  \end{gather}

\begin{ftheorem}
  \label{cv:primer:sep}
  \emph{(Convex separation in finite dimension)}
  Let $C_1$ and $C_2$ be two non-empty convex sets in $\R^n$. 
  Then $C_1$ and $C_2$ can be properly separated if and only if 
  $\mathrm{ri}(C_1)\cap\mathrm{ri}(C_2)=\emptyset.$
\end{ftheorem}
\begin{proof}
  \cite[Theorem~11.3]{Rockafellar1970}
\end{proof}
We collect some useful 
properties of relative interiors
before we get on to convex functions.
\begin{proposition}
  \label{cv:primer:prop}
  Let $C$ be a non-empty convex set in $\R^n.$ The following holds:
\begin{enumerate}[label={(\roman*)}]
  \item
    $
      \mathrm{ri}(C)
      \ 
      \neq
      \ 
      \emptyset
      $
if and only if
      $
      C
      \ 
      \neq
      \ 
      \emptyset
    $
  \item
    $
      \mathrm{cl}(\mathrm{ri}\,C)
      \ 
      =
      \ 
      \mathrm{cl}\,C
      $
      and
      $
      \mathrm{ri}(\mathrm{cl}\,C)
      \ 
      =
      \ 
      \mathrm{ri}(C)
    $
  \item
    $
    \mathrm{ri}(C)
      \ 
    =
      \ 
    \left\{ 
      z \in C
      \colon
      \text{for all}\ 
      x \in C \ 
      \text{there exists}\ 
      t > 0 \ 
      \text{such that}\ 
      z + t (z-x)
      \in C
    \right\}
    $
  \item
    Suppose
    $
      \bigcap_{i\in I} C_i
      \ 
      \neq
      \ 
      \emptyset
    $
    for a finite index set $I$.
    Then
    $
      \mathrm{ri}
      \left( 
        \bigcap_{i\in I} C_i
      \right)
      \ 
      =
      \ 
      \bigcap_{i\in I}  
      \mathrm{ri}(C_i)
    $.
    \item
      Let 
      $
        L\,:\,\R^n \to\  \R^m
      $
      be a linear function. Then
      $
        \mathrm{ri}\,L(C)
        \ 
        =
        \ 
        L(\mathrm{ri}\,C)
      $.
      If it also holds
      $
        L^{\!-1}(\mathrm{ri}\,C)
        \ 
        \neq
        \ 
        \emptyset
      $,
      we have
      $
      \mathrm{ri}\,L^{\!-1}(C)
      \ 
        =
      \ 
        L^{\!-1}(\mathrm{ri}\,C)
      $.
      \item
        $
          \mathrm{ri}(C_1\!\times C_2)
          \ 
          = 
          \ 
          \mathrm{ri}\,C_1
          \! 
          \times
          \mathrm{ri}\,C_2
        $
\end{enumerate}

\end{proposition}


\begin{proof}
  For a proof of (i)-(v) we refer to~\cite[Theorem 6.2 - 6.7]{Rockafellar1970}.

To prove (vi) we use (iii).
Let
  $
  (z_1, z_2)
  \in 
  \mathrm{ri}(C_1\!\times C_2).
  $
  Then for all 
  $
  (x_1, x_2)
  \in 
  C_1\!\times C_2
  $
  there exists
  $t>0$
  such that
  \begin{gather}
    \label{cv:primer:prop:1}
      z_i + t (z_i-x_i)
      \in C_i
      \qquad
      \text{for all}\ 
      i\in \left\{ 1,2 \right\}.
  \end{gather}
  Using (iii) again, we get
  $
  \mathrm{ri}(C_1\!\times C_2)
  \, 
  \subseteq
  \,
          \mathrm{ri}\,C_1
          \! 
          \times
          \mathrm{ri}\,C_2
  $.
  Suppose 
  $
  (z_1,z_2)
    \in
    \mathrm{ri}\,C_1
    \!
    \times
    \mathrm{ri}\,C_2
  $.
  By (iii), for all
  $
    (x_1,x_2)\in C_1\times C_2
  $
  there exist
  $
    (t_1,t_2)>0
  $
  such that
  \begin{gather}
    \label{cv:primer:prop:2}
      z_i + t_i (z_i-x_i)
      \in C_i
      \qquad
      \text{for all}\ 
      i\in \left\{ 1,2 \right\}.
  \end{gather}
  If $t_1=t_2$
  we recover
  \eqref{cv:primer:prop:1}
  from
  \eqref{cv:primer:prop:2}.
  By (iii) it holds
  $
  (z_1,z_2)
    \in
    \mathrm{ri}
    (C_1
    \!
    \times
    C_2)
  $.
  If
  $t_1<t_2$
  we
  define $\theta:=\frac{t_1}{t_2}\in (0,1).$
  Consider  
  \eqref{cv:primer:prop:2} with $i=2$,
  together with $z_2 \in C_2$
  and
  the convexity of $C_2$.
  It follows
  \begin{gather}
    \label{cv:primer:prop:3}
    z_2 + t_1 (z_2 - x_2)
    \ 
    =
    \ 
    \theta
    \cdot
    (
    z_2 + t_2 (z_2 - x_2)
    )
    \ 
    +
    \ 
    (1-\theta)
    \cdot
    z_2
    \in C_2
    \,.
  \end{gather}
  Now we consider
  \eqref{cv:primer:prop:3} and
  \eqref{cv:primer:prop:2} with $i=1$.
  This gives \eqref{cv:primer:prop:1} with $t=t_1$.
  As before, it follows
  $
  (z_1,z_2)\in\mathrm{ri}(C_1\!\times C_2)
  $.
  If 
  $t_1>t_2$
  similar arguments lead to the same result.
  We have proven 
  $
  \mathrm{ri}(C_1\!\times C_2)
  \, 
  \supseteq
  \,
          \mathrm{ri}\,C_1
          \! 
          \times
          \mathrm{ri}\,C_2
  $
  and equality.
\end{proof}
\subsection*{Functions}
A function 
$
f
\colon
\R^n
\to
\overline{\R}
$
is called \textbf{convex function}, if the area above its graph, that is, its epigraph(cf.\cite[§2.4.1]{Mordukhovich2022}), is convex. We shall often use an equivalent definition.
To this end, 
a function $f$ is convex if and only if 
\begin{gather}
  \label{cv:cf}
  f(\theta x + (1-\theta)y)
  \ 
  \le
  \ 
  \theta f(x)
  +
  (1-\theta)f(y)
  \qquad
  \text{for all}\ 
  x,y\in \R^n
  \ 
  \text{and all}\ 
  \theta\in[0,1]
  \,.
\end{gather}
This definition extends to convex cominbinations
$
  \theta_1,\ldots,\theta_m\in[0,1]
$
with
$
  \sum_{i=1}^{m} 
  \theta_i
  =1
$, that is, 
a function $f$ is convex if and only if 
\begin{gather}
  f
  \left( 
    \sum_{i=1}^{m} 
    \theta_i
    x_i
  \right)
  \ 
  \le
  \ 
    \sum_{i=1}^{m} 
    \theta_i
    f(x_i)
  \qquad
  \text{for all}\ 
  x_1,\ldots,x_m\in \R^n
  \,.
\end{gather}
We call a function \textbf{strictly convex} if the inequality in
\eqref{cv:cf} is strict.

We define the \textbf{domain} $\mathrm{dom}\,f$
of a convex function $f$ to be the set where $f$ is finite, that is,
\begin{gather}
  \mathrm{dom}\,f
  \ 
  :=
  \ 
  \left\{ 
x\in\R^n
:
f(x)<\infty
  \right\}
  \,.
\end{gather}
The domain of a convex function is convex. 
We say that $f$ is a \textbf{proper function} if  $\mathrm{dom}\,f\neq\emptyset$. 

For any $\overline{x}\in\mathrm{dom}\,f$ we call $x^*\in\R^n$ a 
\textbf{subgradient} of $f$ at $\overline{x}$ if for all 
$x\in\R^n$ it holds
\begin{gather}
  \inner{x^*}{x-\overline{x}}
  \le
  f(x)
  -
  f(\overline{x})
  \,.
\end{gather}
We denote the collection of all subgradients at $\overline{x}$, that is, the \textbf{subdifferential} of $f$ at $\overline{x}$, as
$
\partial f(\overline{x})
$.
If $f$ is differentiable at $\overline{x}$ it holds
$
\partial f(\overline{x})
=
\left\{ 
  \nabla
  f(\overline{x})
\right\}
$
and thus
\begin{gather}
  \label{cv:primer:mvthe}
  \inner{
  \nabla
  f(\overline{x})
}{x-\overline{x}}
  \le
  f(x)
  -
  f(\overline{x})
  \,.
\end{gather}
%We call a differentiable function $f$ \textbf{strongly convex} with parameter
%$m>0$ if for all
%$x,y\in\mathrm{dom}\,f$ it holds
%\begin{gather}
%  f(y)-f(x)
%  \ 
%  \ge
%  \ 
%  \inner
%  {
%  \nabla
%  f(x)
%  }
%  {y-x}
%  \ 
%  +
%  \ 
%  \frac{m}{2}
%  \norm{y-x}^2
%  \,.
%\end{gather}
%If $f$ is twice continuously differentiable, then it is strongly 
%convex with parameter $m>0$ if and only if the matrix
%\begin{gather}
%\nabla^2f(x)
%-m\cdot\mathbf{I}
%\qquad
%\text{
%is positive semi-definite for all
%}\ 
%x\in\mathrm{dom}\,f
%\,,
%\end{gather}
% where 
%$
%\nabla^2f
%$
%is the Hessian Matrix.
%

\begin{definition*}
  Given a nonempty subset 
  $\Omega \subseteq \R^n$,
  we define
  the \textbf{support function} 
  $
  $
  of $\Omega$
  to be
  \begin{gather*}
  \sigma_\Omega 
  \,
  :
  \,
  \R^n \to\  \overline{\R}
  \,,
  \qquad
  x^*
  \ 
  \mapsto
  \ 
    \sup_{x \in \Omega}
    \ 
    \inner{x^*\!}{x}
    \,.
  \end{gather*}
\end{definition*}


\begin{definition}
  Given functions
  $
    f_i\,:\,
    \R^n \to\  \overline{\R}
  $
  for $ i = 1, \ldots, m $,
  we define the \textbf{infimal convolution} of these functions to be
  \begin{gather*}
    f_1 \square \cdots \square f_m
    \ 
    \colon
    \ 
    \R^n
    \to
    \ 
    \overline{\R}
    \,,
    \quad
    x
    \ 
    \mapsto
    \ 
    \inf
    \left\{ 
    \sum_{i = 1}^{m}
      f_i(x_i)
      \ 
      \colon
      \ 
      x_i \in \R^n 
      \ 
      \mathrm{and}\ 
      \sum_{i = 1}^{m} 
        x_i
      =
      x
    \right\}
    \,.
  \end{gather*}
\end{definition}
 
The next result establishes a connection between the support function of the intersection of two convex sets and the infimal convolution of the support functions of the sets taken by themselfes.
The proof translates the geometric concept of convex separation to the world of convex functions.

\begin{lemma}
  \label{cv:primer:lem}
  Let $C_1$ and $C_2$ be two non-empty convex sets in $\R^n$.
  For any
  $ x^* \in \mathrm{dom}\, \sigma_{C_1\cap C_2} $
  the sets
  \begin{align*}
    \Theta_1
    &
    \ :=\ 
    C_1 \times [\,0,\infty)
    \,,
    \\
    \Theta_2
    (x^*)
    &
    \ :=\ 
    \left\{ 
      (x,\lambda)\in \R^n
      \ 
      \colon
      \ 
      x \in C_2
      \ 
      \text{and}
      \ 
      \lambda
      \,
      \le
      \,
      \inner{x^*\!}{x} 
      \ 
      -
      \ 
      \sigma_{C_1\cap C_2}(x^*)
    \right\}
  \end{align*}
  can by properly separated.
\end{lemma}
\begin{proof}
  We fix 
  $ x^* \in \mathrm{dom}\, \sigma_{C_1\cap C_2} $
  and write
  $ 
  \alpha
  \ 
  :=
  \ 
  \sigma_{C_1\cap C_2}(x^*)
  $.
  In order to apply convex separation in finite dimension 
  (Theorem~\ref{cv:primer:sep})
  to
  the sets
  $ \Theta_1 $ and $ \Theta_2(x^*) $,
  it suffics to show
  their convexity and
  $
    \mathrm{ri}\, 
    \Theta_1
    \cap
    \mathrm{ri}\, 
    \Theta_2(x^*)
    =
    \emptyset
  $.
  \subsubsection*{Convexity of 
  $ \Theta_1 $ and $ \Theta_2(x^*) $
  }
  Clearly, 
  $ \Theta_1 $ is convex by the convexity of 
  $ C_1 $ and $ [0,\infty) $.
 To see that $\Theta_2(x^*)$ is convex consider the linear function
 \begin{gather*}
    L
    \,
    :
    \,
    \R^n\times\,  \R 
    \ 
    \to
    \ 
    \R
    \,,
    \qquad 
    (x,\lambda)
    \ 
    \mapsto
    \ 
    \inner{x^*\!}{x} - \lambda
    \,.
 \end{gather*}
 From the definitions of $L$ and $\Theta_2(x^*)$ we get 
  \begin{gather*}
 \Theta_2
 (x^*)
    \ 
    =
    \ 
    (
    C_2\!\times\R
    )
    \ 
    \cap
    \ 
    L^{\!-1}
    [\,\alpha,\infty)
    \,
    .
  \end{gather*}
  Thus,
  by
  Proposition~\ref{cv:primer:prop}~(v)
  and the convexity of $C_2$ we get the convexity of
  $
    L^{\!-1}
    [\,\alpha,\infty)
  $ and with it that of $\Theta_2(x^*)$.

  \subsubsection*{Relative interiors of
  $ \Theta_1 $ and $ \Theta_2(x^*) $
  are disjoint}
  We start by calculating the relative interiors. It holds
  \begin{alignat*}{3}
    \mathrm{ri}\,
    \Theta_1
    &
    \ 
    =
    \ 
    \mathrm{ri}
    ( C_1\times [0,\infty) )
    &&
    \ 
    =
    \ 
    \mathrm{ri}\,
    C_1
    \!
    \times
    \mathrm{ri}\,
    [0,\infty)
    \ 
    =
    \ 
    \mathrm{ri}\,
    C_1
    \!
    \times
    (0,\infty)
    \,,
    %%%%%%%%%%%%%%%%%
    \\
    \mathrm{ri}\,
    \Theta_2(x^*)
    & 
    \ 
    =
    \ 
    \mathrm{ri}
    (
    L^{\!-1}
    [\,\alpha,\infty)
    )
    &&
    \ 
    =
    \ 
    L^{\!-1}
    (
    \mathrm{ri}\,
    [\,\alpha,\infty)
    )
    \ 
    \ 
    =
    \ 
    L^{\!-1}
    (\alpha,\infty)
    \,.
  \end{alignat*}
  Suppose there exists
  $ (\lambda,x) \in
    \mathrm{ri}\, 
    \Theta_1
    \cap
    \,
    \mathrm{ri}\, 
    \Theta_2(x^*)
  $.
  Then it holds 
  $ x \in C_1\!\times C_2 $
  and 
  $ \lambda >0 $.
  We also note, that
  \begin{gather*}
  \alpha
  \ 
  =
  \ 
  \sigma_{C_1\cap\, C_2}(x^*)
    \ 
  =
    \ 
  \sup_{z \in C_1\cap\, C_2}
  \inner
  {x^*}
  {z}
  \ 
  \ge
  \ 
  \inner
  {x^*}
  {x}
  \,.
  \end{gather*}
  Then it follows
  \begin{gather*}
    \alpha
    \ 
    <
    \ 
  \inner
  {x^*}
  {x}
  - \lambda
    \ 
  \le
    \ 
  \alpha\,,
  \end{gather*}
  a contradiction.
  Thus, the relative interiors of
  $ \Theta_1 $ and $ \Theta_2(x^*) $
  are disjoint.

  Applying Theorem~\ref{cv:primer:sep} finishes the proof.
\end{proof}


\begin{theorem*}
  Let $C_1$ and $C_2$ be two non-empty convex sets in $\R^n$ with
  $\mathrm{ri}\,C_1\cap\mathrm{ri}\,C_2\neq\emptyset.$
  Then the support function of the intersection 
  $
    C_1\! \cap C_2
  $
  is represented as
  \begin{gather}
    (\sigma_{
    C_1 \cap\, C_2
    })
    (x^*)
    =
    (\sigma_{C_1}\square \,\sigma_{C_2})
    (x^*)
    \qquad
    \text{for all}\ 
    x^* \in \R^n.
  \end{gather}
  Furthermore, for any
  $
  x^*\in \mathrm{dom}
    (\sigma_{
    C_1 \cap\, C_2
    })
  $
  there exist dual elements 
  $
    x_1^*
    ,
    x_2^*
    \in \R^n
  $ 
  such that 
  $
    x^*
    =
    x_1^*
    +
    x_2^*.
  $
  and
  \begin{gather}
    (\sigma_{
    C_1 \cap\, C_2
    })
    (x^*)
    =
    \sigma_{C_1}(x_1^*)
    +
    \sigma_{C_2}(x_2^*).
  \end{gather}
\end{theorem*}
\begin{proof}
  Using Lemma~\ref{cv:primer:lem}
  the rest of the proof is as that of
  \emph{\cite[Theorem~4.23(b)]{Mordukhovich2022}}.
\end{proof}

\begin{takeaways}
  The support function intersection rule connects the geometric 
  property of convex separation to an identity of support functions
  This result is central to the analysis of convex conjugates.
\end{takeaways}
One important application of convex functions is in optimization.
There we often analyse a dual problem instead, which relies on the notion of \textbf{convex conjugate} 
$
    f^*:
    \R^n \to \overline{\R}
  $
  of $f$ defined by
  \begin{gather}
    \label{def:convex_conjugate}
    f^*(x^*)
    :=
    \sup_{ x \in \R^n }
    \inner
    {x^*}{x}
    - f(x)
    \,.
  \end{gather}
  Even for arbitrary functions, the convex conjugate is convex(cf.
  \cite[Proposition~4.2]{Mordukhovich2022}
  ).
  Like in differential calculus, there exist sum and chain rule for computing the convex conjugate.
\begin{theorem}
  Let
  $
    f,g:
    \R^n \to (-\infty, \infty]
  $
  be proper convex functions 
  and
  $
  \text{ri}\left( \text{dom}(f) \right)
  \cap
  \text{ri}\left( \text{dom}(g) \right)
  \neq 
  \emptyset
  .
  $
  Then we have the 
  \textbf{
  conjugate sum rule
  }
  \begin{gather}
    ( f + g )^*(x^*)
    =
    ( f^* \square g^*)(x^*)
  \end{gather}
  for all $x^* \in \R^n$.
  Moreover, the infimum in 
  $
    ( f^* \square g^*)(x^*)
  $
  is attained, i.e., for any
  $
    x^* \in \text{dom}(f+g)^*
  $
  there exists vectors $x_1^*, x_2^*$
  for which
  \begin{gather}
    (f+g)^*(x^*)
    =
    f^*(x_1^*)
    +
    g^*(x_2^*),
    \quad
    x^* = x_1^* + x_2^*.
  \end{gather}
\end{theorem}
\begin{proof}
  \cite[Theorem~4.27(c)]{Mordukhovich2022}
\end{proof}



% conjugate chain rule %
 %%%%%%%%%%%%%%%%%%%%%%
\begin{theorem}
  \label{cvxa_conjugate_chain_rule}
  Let 
  $
    A:
      \R^m \to \R^n
  $
  be a linear map (matrix)
  and
  $
    g:
      \R^n \to (-\infty, \infty]
  $
  a proper convex function. If
  $
    \text{Im}(A) \cap \text{ri}(\text{dom}(g))
    \neq
    \emptyset
  $
  it follows
  the 
  \textbf{conjugate chain rule}
  \begin{gather}
    ( g \circ A )^* ( x^* )
    =
    \inf_
          { y^* \in ( A^* )^{ -1 } ( x^* )}
                                          g^*( y^* )
                                          .
  \end{gather}
  Furthermore, 
    for any 
      $
        x^* \in \text{dom}( g \circ A)^*
      $
        there exists
          $
            y^* \in ( A^* )^{ -1 } ( x^* )
          $
            such that
              $
                ( g \circ A)^* ( x^* )
                =
                g^*( y^* )
              $.
\end{theorem}
\begin{proof}
  \cite[Theorem~4.28(c)]{Mordukhovich2022}
\end{proof}
%%%%%%%%%%%%%%%%%
%%%% EXAMPLE %%%%
%%%%%%%%%%%%%%%%%
\begin{example}
  \label{cv:cc:ex}
  Let 
  $
    f:\R\to \overline{\R}
  $
  be a proper convex function, that is, 
  $
    \mathrm{dom}\,f
    \neq
    \emptyset
  $
  and $f$ is convex.
  In steps we apply the conjugate chain and sum rule, together with mathematical induction,
  to prove the conjugate relationship 
  \begin{align*}
    &S_{f,n}:\R^n \to \overline{\R},
    \qquad
    (x_1,\ldots,x_n)
    \mapsto
    \sum_{i=1}^{n} 
    f(x_i)
    ,
    \\
    &S_{f,n}^*:\R^n \to \overline{\R},
    \qquad
    (x^*_1,\ldots,x^*_n)
    \mapsto
    \sum_{i=1}^{n} 
    f^*(x^*_i)
    \,.
  \end{align*}
  This relationship is very natural and the ensuing calculations serve to confirm our intuition.

  First, we work in the projections on the coordinates. 
  For the $i$-th coordinate, where $i=1,\ldots,n$, this is 
  \begin{gather}
    p_i:\R^n\to \R
    ,
    \quad
    (x_1,\ldots,x_n)
    \mapsto
    x_i\,.
  \end{gather}
  All projections 
  $p_i$
  are linear function with matrix representation
  $
    e_i^\top
  $,
  where $e_i$ is $i$-the coordinate vector.
  The adjoint of $p_i$ is therefore
  \begin{gather}
    p^*_i:\R\to \R^n
    ,
    \quad
    x
    \mapsto
    e_i\cdot x
    \,.
  \end{gather}
  For the inverse image of the adjoint of $p_i$ it holds
  \begin{gather}
    (p_i^*)^{-1}
    \left\{ 
    (x_1^*,\ldots,x_n^*)
    \right\}
    \ 
    =
    \ 
    \begin{cases}
      \left\{ x_i^* \right\},
      \quad
      &\text{if}\ 
      x_j^*=0\ \text{for all}\ j\neq i\,,
      \\
      \ \ \emptyset
      \quad
      &\text{else.}
    \end{cases}
  \end{gather}
  Throughout this example we use the asterisk character $^*$ somewhat inconsistently. 
  Note that $f^*$ is the convex conjugate 
  of the function $f$ and $p_i^*$ is the adjoint linear function of the projection on the $i$-th coordinate. Likewise, we denote dual variables, that is, the arguments of convex conjugates, as $x^*$.

  Next, we employ the conjugate chain rule to establish the conjugate relationship 
  \begin{align*}
    f_i&:\R^n\to \overline{\R}
    ,
    \quad
    (x_1,\ldots,x_n)
    \mapsto x_i \mapsto f(x_i)
    \,,
    \\
    f^*_i&:\R^n\to \overline{\R}
    ,
    \quad
    (x^*_1,\ldots,x^*_n)\mapsto 
    \begin{cases}
      f^*(x_i^*),
      \quad
      &\text{if}\ 
      x_j^*=0\ \text{for all}\ j\neq i\,,
      \\
      \infty
      \quad
      &\text{else.}
    \end{cases}
  \end{align*}
  Note, that 
  $
    f_i
    =
    (f\circ p_i)
  $
  and
  $
    f^*_i
    =
    (f\circ p_i)^*
  $.
  Since 
  $
    \mathrm{Im}\,p_i=\R
  $
  and 
  $
    \mathrm{dom}\, f
    \neq
    \emptyset
  $,
  it holds
  $
    \mathrm{Im}\, p_i
    \cap
    \mathrm{ri}(
    \mathrm{dom}\, f
    )
    \neq
    \emptyset
  $.
  Then $f$ and $p_i$ conform with the demands of the conjugate chain rule.
  It follows
  \begin{align*}
    &f_i^*
    (x^*_1,\ldots,x^*_n) 
    \ 
    =
    \ 
    (f\circ p_i)^*
    (x^*_1,\ldots,x^*_n) 
    \ =
    \ 
    \inf
    \left\{ 
    f^*(y)
    \ 
    |
    \ 
    y\in 
    (p_i^*)^{-1}
    \left\{ 
    (x_1^*,\ldots,x_n^*)
    \right\}
    \right\}
    \\
    &\quad=
    \ 
    \begin{cases}
      f^*(x_i^*),
      \quad
      &\text{if}\ 
      x_j^*=0\ \text{for all}\ j\neq i\,,
      \\
      \infty
      \quad
      &\text{else,}
    \end{cases}
  \end{align*}
  where we keep to the convention $\inf\emptyset=\infty$.
  In the same way it follows
  \begin{gather}
    \left( 
      S_{f,n}
      \circ
      p_{\left\{ 1,\ldots,n \right\}}
    \right)^*
    (x^*_1,\ldots,x^*_{n+1})
    =
    \begin{cases}
      S_{f,n}^*
    (x^*_1,\ldots,x^*_{n})
      \quad
      &\text{if}\ 
      x_{n+1}^*=0\,,
      \\
      \infty
      \quad
      &\text{else,}
    \end{cases}
  \end{gather}

  Next, note that for $n=1$ we arrive at the result. Thus, for some $n\in \mathbb{N}$ it holds
  $
  \left( 
    S_{f,n}
  \right)
  ^*
  =
    S_{f,n}^*
  $.
  In order to apply the conjugate sum rule to 
  $
    S_{f,n}
  $
  and
  $
    f_{n+1}
  $
  we note that
  \begin{align*}
    \mathrm{dom}\, f_i
    &
    \ =\ 
    \left\{ 
      (x_1,\ldots,x_{n+1})
      \in \R^{n+1}
      :
      x_i\in \mathrm{dom}\,f
    \right\}
    \ 
    \neq 
    \ 
    \emptyset
    \qquad
    \text{for all}
    \ 
    i=1,\ldots,n+1
    \,,
    \\
    \bigcap_{i=1}^{n+1}
    \mathrm{dom}\, f_i
    &
    \ =\ 
    \left\{ 
      (x_1,\ldots,x_{n+1})
      \in \R^{n+1}
      :
      x_i\in \mathrm{dom}\,f
      \ 
    \text{for all}
    \ 
    i=1,\ldots,n+1
    \right\}
    \ 
    \neq 
    \ 
    \emptyset
    \,,
  \end{align*}
  and
\begin{align*}
  &
  \mathrm{ri}\left( 
    \mathrm{dom}
    \left( 
    S_{f,n}
    \circ
    p_{\left\{ 1,\ldots,n \right\}}
    \right)
  \right)
  \ 
  \cap
  \ 
  \mathrm{ri}\left( 
    \mathrm{dom}\,f_{n+1}
  \right)
  \\
  &
  \hspace{25mm}
  =
  \ 
  \mathrm{ri}\left( 
    \mathrm{dom}
    \left( 
    S_{f,n}
    \circ
    p_{\left\{ 1,\ldots,n \right\}}
    \right)
  \ 
  \cap
  \ 
    \mathrm{dom}\,f_{n+1}
  \right)
  \ 
  =
  \ 
  \mathrm{ri}
  \left( 
    \bigcap_{i=1}^{n+1}
    \mathrm{dom}\, f_i
  \right)
  \ 
  \neq
  \ 
  \emptyset
  \,.
\end{align*}
By the conjugate sum rule it follows
\begin{align*}
  (
  S_{f,n+1}
  )^*
  =
  (
    S_{f,n}
    \circ
    p_{\left\{ 1,\ldots,n \right\}}
  +
  f_{n+1}
  )^*
  =
  (
    S_{f,n}
    \circ
    p_{\left\{ 1,\ldots,n \right\}}
  )
  ^*
  \square
  f_{n+1}^*
  \\
  =
    S_{f,n}^*
    \circ
    p_{\left\{ 1,\ldots,n \right\}}
    +
  f_{n+1}^*
  =
  S^*_{f,n+1}
  \,.
\end{align*}
\end{example}






  \section{Conjugate Calculus}
  When studying different primal problems such as \eqref{primal_weighting_binary} we often turn to the dual instead.
Therefore we need some reliable tools.
Begin able to compute specific convex conjugates is one tool required.


\begin{definition}
  \label{ def_convex_conjugate }
  \emph{(Convex conjugate)}
  Given a function
  $
    f:
    \R^n \to \overline{\R}
  $
  ,
  the 
  \textbf{convex conjugate}
  $
    f^*:
    \R^n \to \overline{\R}
  $
  of $f$ is defined as
  \begin{gather}
    f^*(x^*)
    :=
    \sup_{ x \in \R^n }
    (x^*)^T x - f(x)
  \end{gather}
\end{definition}

Note that $f$ in Definition~\ref{ def_convex_conjugate }
does not have to be convex. On the other hand, the convex conjugate is always convex:

\begin{proposition}
  Let  
  $
    f:
    \R^n \to ( - \infty, \infty ]
  $
  be a proper function. 
  Then its convex conjugate
  $
    f^*:
    \R^n \to ( - \infty, \infty ]
  $
  is convex.
\end{proposition}

\begin{definition}
  Given a nonempty subset 
  $\Omega \subseteq \R^n$
  the \textbf{support function} 
  $
  \sigma_\Omega : \R^n \to \overline{\R}
  $
  of $\Omega$
  is defined by
  \begin{gather}
    \sigma_\Omega
    (x^*)
    :=
    \sup_{x \in \Omega}
    \ 
    \inner{x^*}{x}
    \qquad
    \text{for}\ 
    x^* \in \R^n
    .
  \end{gather}
\end{definition}

\begin{lemma}
  For any proper function
  $
    f:\R^n\to\overline{\R}
  $
  we have
  \begin{gather}
    f^*(x^*) 
    =
    \sigma_{\mathrm{epi}(f)}
    (x^*,-1)
    \qquad
    \text{for}
    \ 
    x^* \in \R^n.
  \end{gather}
\end{lemma}
\begin{proof}
  Let $x^*\in\R^n$
  and
  $
    (x,\lambda)\in \mathrm{epi}(f).
  $
  Then
  $
    x \in \mathrm{dom}(f)
  $
  and
  $
    f(x)\le \lambda.
  $
  Thus
  \begin{gather}
    \inner{x^*}{x} - f(x)
    \ge
    \inner{x^*}{x} - \lambda
    \qquad
    \text{for all}\ 
    (x,\lambda)\in \mathrm{epi}(f).
  \end{gather}
  On the other hand 
  $
    (x,f(x))\in \mathrm{epi}(f)
  $
  for all
  $
    x \in \mathrm{dom}(f).
  $
  It follows
  \begin{gather}
    \inner{x^*}{x} - f(x)
    \le
    \sup_{(x,\lambda)\in\mathrm{epi}(f)}
    \inner{x^*}{x} - \lambda
    \qquad
    \text{for all}\ 
    x \in \mathrm{dom}(f).
  \end{gather}
  Taking the supremum in the last two displays yields
  \begin{align}
    f^*(x^*)
    =
    \sup_{x\in\mathrm{dom}(f)}
    \inner{x^*}{x} - f(x)
    &=
    \sup_{(x,\lambda)\in\mathrm{epi}(f)}
    \inner{x^*}{x} - \lambda
    \\
    &=
    \sup_{(x,\lambda)\in\mathrm{epi}(f)}
    \inner{(x^*,-1)}{(x,\lambda)} 
    =
    \sigma_{\mathrm{epi}(f)}
    (x^*,-1).
  \end{align}
\end{proof}
% conjugate chain rule %
 %%%%%%%%%%%%%%%%%%%%%%
\begin{proposition}

\end{proposition}
\begin{theorem}
  \emph{(Conjugate Chain Rule)}
  \label{cvxa_conjugate_chain_rule}
  Let 
  $
    A:
      \R^m \to \R^n
  $
  be a linear map (matrix)
  and
  $
    g:
      \R^n \to (-\infty, \infty]
  $
  a proper convex function. If
  $
    \text{Im}(A) \cap \text{ri}(\text{dom}(g))
    \neq
    \emptyset
  $
  it follows
  \begin{gather}
    ( g \circ A )^* ( x^* )
    =
    \inf_
          { y^* \in ( A^* )^{ -1 } ( x^* )}
                                          g^*( y^* )
                                          .
  \end{gather}
  Furthermore, 
    for any 
      $
        x^* \in \text{dom}( g \circ A)^*
      $
        there exists
          $
            y^* \in ( A^* )^{ -1 } ( x^* )
          $
            such that
              $
                ( g \circ A)^* ( x^* )
                =
                g^*( y^* )
              $.
\end{theorem}

% conjugate sum rule %
 %%%%%%%%%%%%%%%%%%%%

\begin{definition}
  \emph{(Infimal convolution)}
  Given functions
  $
    f_i:
    \R^n \to (-\infty, \infty]
  $
  for $ i = 1, \ldots, n $
  the \textbf{infimal convolution} of these functions as defined as
  \begin{gather}
    (f_1 \square \ldots \square f_m)(x)
    :=
    \inf_{
    \begin{smallmatrix}
      x_i \in \R^n \\
      \sum_{i = 1}^{m} 
        x_i
      =
      x
    \end{smallmatrix}
    }
    \sum_{i = 1}^{m}
      f_i(x_i)
  \end{gather}
\end{definition}


\begin{theorem}
  Let
  $
    f,g:
    \R^n \to (-\infty, \infty]
  $
  be proper convex functions 
  and
  $
  \text{ri}\left( \text{dom}(f) \right)
  \cap
  \text{ri}\left( \text{dom}(g) \right)
  \neq 
  \emptyset
  .
  $
  Then we have the conjugate sum rule
  \begin{gather}
    ( f + g )^*(x^*)
    =
    ( f^* \square g^*)(x^*)
  \end{gather}
  for all $x^* \in \R^n$.
  Moreover, the infimum in 
  $
    ( f^* \square g^*)(x^*)
  $
  is attained, i.e., for any
  $
    x^* \in \text{dom}(f+g)^*
  $
  there exists vectors $x_1^*, x_2^*$
  for which
  \begin{gather}
    (f+g)^*(x^*)
    =
    f^*(x_1^*)
    +
    g^*(x_2^*),
    \quad
    x^* = x_1^* + x_2^*.
  \end{gather}
\end{theorem}
\begin{proof}
  Let $x^*\in\R^n$ and fix $x_1^*,x_2^*\in\R^n$ such that
  $x^*=x^*_1+x^*_2$.
  We get
  \begin{align*}
    f^*(x^*_1)+g^*(x^*_2)
    &=
    \sup_{x\in\R^n}
    \inner{x^*_1}{x}-f(x)
    +
    \sup_{x\in\R^n}
    \inner{x^*_2}{x}-g(x)
    \\
    &\ge
    \sup_{x\in\R^n}
    \inner{x^*_1}{x}-f(x)
    +
    \inner{x^*_2}{x}-g(x)
    =
    \sup_{x\in\R^n}
    \inner{x^*_1+x^*_2}{x}-(f(x)+g(x))
    \\
    &=
    \sup_{x\in\R^n}
    \inner{x^*}{x}-(f+g)(x)
    =(f+g)^*(x^*)
  \end{align*}
  Taking the infimum over $x_1^*,x_2^*\in\R^n$ in the above display gives 
  $
  (f^*\square g^*)(x^*)
  \ge
  (f+g)^*(x^*).
  $
  Let us prove now $\le$ under the condition
  $
  \text{ri}\left( \text{dom}(f) \right)
  \cap
  \text{ri}\left( \text{dom}(g) \right)
  \neq 
  \emptyset
  .
  $
  The only case we need to consider is
  $
    (f+g)^*(x^*)<\infty.
  $
  Define two convex sets by
  \begin{align}
    \Omega_1
    &:=
    \left\{ 
      (x,\alpha,\beta)\in\R^{n+2}
      \colon
      \alpha\ge f(x)
    \right\}
    =
    \mathrm{epi}(f)\times \R,
    \\
    \Omega_2
    &:=
    \left\{ 
      (x,\alpha,\beta)\in\R^{n+2}
      \colon
      \beta\ge g(x)
    \right\}.
  \end{align}
  Similar to Lemma we get the representation
  \begin{gather}
    (f+g)^*(x^*)
    =
    \sigma_{\Omega_1\cap\Omega_2}
    (x^*,-1,-1).
  \end{gather}
  Indeed, the only thing we need to verify is
  $
    \mathrm{dom}(f)\cap\mathrm{dom}(g)
    =
    \mathrm{dom}(f+g).
  $
  The inclusion $\subseteq$ is clear.
  Assume towards a contradiction that
  $
    (f+g)(x)<\infty
  $
  and
  $
    f(x)=\infty.
  $
  Since $g(x)>-\infty$ it holds
  \begin{gather}
    \infty
    =
    \infty+g(x)
    =f(x)+g(x)
    =(f+g)(x)
    <
    \infty.
  \end{gather}
  This is a contradiction. The same holds for $f$ and $g$ reversed. It follows the inclusion $\supseteq$ and equality.
  By the support function intersection rule there exist triples
  \begin{gather}
    (x^*_1,-\alpha_1,-\beta_1),
    (x^*_2,-\alpha_2,-\beta_2)
    \in \R^{n+2}
    \quad
    \text{such that}
    \quad
    (x^*,-1,-1)
    =
    (x^*_1+x^*_2,-(\alpha_1+\alpha_2),-(\beta_1+\beta_2))
  \end{gather}
  and
  \begin{gather}
    (f+g)^*(x^*)
    =
    \sigma_{\Omega_1\cap\Omega_2}
    (x^*,-1,-1)
    =
    \sigma_{\Omega_1}
    (x^*_1,-\alpha_1,-\beta_1)
    +
    \sigma_{\Omega_2}
    (x^*_2,-\alpha_2,-\beta_2).
  \end{gather}
  Next we show
  $\beta_1=\alpha_2=0.$
  Suppose towards a contradiction that 
  $\beta_1\neq 0.$ 
  We fix 
  $(\overline{x},\overline{\alpha})\in\mathrm{epi}(f).$
  Then
  \begin{gather}
    \sigma_{\Omega_1}
    (x^*_1,-\alpha_1,-\beta_1)
    =
    \sup_{(x,\alpha,\beta)\in \mathrm{epi}(f)\times \R}
    \inner{x^*}{x}-\alpha \alpha_1 -\beta \beta_1
    \ge
    \sup_{\beta\in \R}
    \inner{x^*}{\overline{x}}-\overline{\alpha} \alpha_1 -\beta \beta_1
    =\infty.
  \end{gather}
  This contradicts
  $
    (f+g)^*(x^*)<\infty.
  $
  In a similar fashion we can derive a contradiction for $\alpha_2\neq0.$
  Employing Lemma and taking into account the structures of the sets 
  $\Omega_1$ and $\Omega_2$ this implies
  \begin{align}
    (f+g)^*(x^*)
    &=
    \sigma_{\Omega_1\cap\Omega_2}
    (x^*,-1,-1)
    =
    \sigma_{\Omega_1}
    (x^*_1,-1,0)
    +
    \sigma_{\Omega_2}
    (x^*_2,0,-1)
    \\
    &=
    \sigma_{\mathrm{epi}(f)}(x^*_1,-1)
    +
    \sigma_{\mathrm{epi}(g)}(x^*_2,-1)
    =
    f^*(x^*_1)
    +
    g^*(x^*_2)
    \ge
    (f^*\square g^*)(x^*).
  \end{align}
  This finishes the proof.
\end{proof}

%We begin by defining convex sets
%

\begin{definition}
  A subset $\Omega\subseteq \R^n$ is called CONVEX if we have $\lambda x+(1-\lambda)y\in \Omega$ for all $x,y\in \Omega$ and $\lambda\in (0,1)$. 
\end{definition}

Clearly, the line segment 
$[a,b]:=\left\{ \lambda a+(1-\lambda)b\,\mid \, \lambda\in [0,1] \right\}$ is contained in $\Omega$ for all $a,b\in \Omega$ if and only if $\Omega$ is a convex set.
%

Next we define convex functions. 
%

The concept of convex functions is closely related to convex sets.
%  
 
The line segment between two points on the graph of a convex function lies on or above and does not intersect the graph.
%

In other words: The area above the graph of a convex function $f$ is a convex set, i.e. the \textit{epigraph}
$\text{epi}(f):=\left\{ (x,\alpha)\in \R^n\times\R\,\mid\, f(x)\le \alpha\right\}$ is a convex set in $\R^{n+1}$.
%

Often an equivalent characterisation of convex functions is more useful.
%

\begin{theorem}
  The convexity of a function $f:\R^n\to \overline{\R}$ on $\R^n$ is equivalent to the following statement:

  For all $x,y\in \R^n$ and $\lambda\in(0,1)$ we have 
    \begin{align}
      f(\lambda x + (1-\lambda)y)\le \lambda f(x)+(1-\lambda)f(y).
    \end{align}
\end{theorem}

  \newpage
  \section{Duality of Optimal Solutions}
  We consider a general convex optimization problem 
with matrix equality and inequality constraints.
For this problem there exists a related problem,
which we call its dual.
With ideas from \cite{Tseng1991} we establish 
a functional relationship
between the optimal solution of the original problem 
and
optimal solutions of the dual.
The main assumption is that in the original problem we have a strictly convex objective function 
with continuously differentiable 
convex conjugate(cf. Definition~\ref{cv:cc:d:cc}). 
\begin{ftheorem}
  \label{cv:ts:th}
  Consider the optimization problem
\begin{align}
  \label{cv:ts:primal}
  %%%% objective %%%%
    &\underset{w \in \R^n}
    {\mathrm{minimize}}
    &&\qquad\qquad
    f(w)
    &&&
    \\
    %%%% Ax >= b %%%%
    \nonumber
    &\mathrm{subject}\ \mathrm{to} 
    &&\qquad\qquad
    \mathbf{U}w
    \ 
    \ge
    \ 
    d
    \,.
    \\
    \nonumber
    &
    &&\qquad\qquad
    \mathbf{A}w
    \ 
    =
    \ 
    a
    \,,
\end{align}
and its dual problem
  \begin{alignat}{2}
    \label{cv:ts:dual}
  %%%% objective %%%%
    &\underset{
    \lambda_d \in \R^r
,
    \lambda_a \in \R^s
  }
    {\mathrm{maximize}}
    &&\qquad\qquad
    \inner
    {\lambda_d}
    {d}
    \ 
    +
    \ 
    \inner
    {\lambda_a}
    {a}
    \ 
    -
    \ 
    f^*
    \!
    \left( 
      \mathbf{U}^\top \! \lambda_d
      +
      \mathbf{A}^\top \! \lambda_a
    \right)
    \\
    %%%% Ax >= b %%%%
    \nonumber
    &\mathrm{subject}\ \mathrm{to} 
    &&\qquad\qquad
    \lambda_d
    \ 
    \ge
    \ 
    0
    \,.
\end{alignat}
  Let 
$
(\lambda_d^\dagger,\lambda_a^\dagger)
$
be an optimal solution to \eqref{cv:ts:dual}.
If the objective function $f$ of 
\eqref{cv:ts:primal} is strictly convex and its
convex conjugate $f^*$ is continuously differentiable,
then the unique optimal solution to 
\eqref{cv:ts:primal}
is given by
\begin{gather}
  w^\dagger
  =
  \nabla
    f^*
    \!
    \left( 
      \mathbf{U}^\top  \lambda_d^\dagger
      +
      \mathbf{A}^\top  \lambda_a^\dagger
    \right)
    \,.
\end{gather}
\end{ftheorem}

\subsubsection*{Plan of Proof}
We show that 
$w^\dagger$ and 
$
(\lambda_d^\dagger,\lambda_a^\dagger)
$
meet the 
Karush-Kuhn-Tucker conditions for \ref{cv:ts:primal},
that is,
\textbf{complementary slackness}
\begin{gather}
\inner
{\lambda_d^\dagger\,}{d-\mathbf{U} w^\dagger}
\ 
=
\ 
0
\,,
\end{gather}
\textbf{primal} and \textbf{dual feasibility}
\begin{align}
  \label{primal_feas}
    \mathbf{U}w^\dagger
    &
    \ 
    \ge
    \ 
    d
    \,,
    \\
    \nonumber
    \mathbf{A}w^\dagger
    &
    \ 
    =
    \ 
    a
    \,,
  \\
  \label{dual_feas}
  \lambda_d^\dagger
    &
    \ 
  \ge
  \ 
  0
    \,,
\end{align}
and 
\textbf{stationarity}
\begin{gather}
  \mathrm{0}_n
  \ 
  \in
  \ 
  [
  \partial
  f(w^\dagger)
  \ 
  +
  \ 
    \partial
    \left( 
      w
      \mapsto
      d
      -
      \mathbf{U}w
    \right)
    (w^\dagger)
    \cdot
    \lambda_d^\dagger
    \ 
    +
    \ 
    \partial
    \left( 
      w
      \mapsto
      a
      -
      \mathbf{A}w
    \right)
    (w^\dagger)
    \cdot
    \lambda_a^\dagger
    \,
  ]
  \,.
\end{gather}
Applying the well know result\cite[Theorem~28.3]{Rockafellar1970}
finishes the proof.
Apart from elementary calculations, our main tools are the 
strict convexity of $f$, the smoothness of $f^*$ and 
\begin{proposition}
  \emph{
\cite[Theorem~23.5(a)-(b)]{Rockafellar1970}.
  }
  \label{cv:ts:prop}
   For any proper convex function $g$ and any vector $w$, 
   it holds $t\in \partial f(w)$ 
   if and only if 
   $
   x
   \mapsto
   \inner
   {x}{t}
   -
   f(x)
   $
   achieves its supremum at $w$.
\end{proposition}

\begin{proof}
\end{proof}



%\chapter{Random Matrix Inequalities}
%  In our application we want to bound moments of vector-valued random variables.
%  For this we choose the theory of random matrix inequalities
%  which lately received a lot of attention.
%  In particular an approach via the method of exchangable pairs \cite{Mackey2014}
%  has been fruitful in simplifying the proofs of long standing results such as the matrix Khintchin inequality.
%  The paper offers a comprehensive introduction to this method.
%
%  We will cite the matrix Khintchin inequality and 
%  inequalities for moments of matrices that follow from it\cite{Chen2012}. 
%  As a novelty, we will apply intrinsic dimension results and Hermition Dilitation from \cite{Tropp2015}  
%  to matrix moments inequalities. Even though it is straightforward, to the best of our knowledge the calculations have not been carried out in any publication so far.
%  \section{A Matrix Analysis Primer}
%    The \textbf{trace} of a square matrix, denoted by $\mathrm{tr},$
  is the sum of its diagonal entries, i.e. 
  $
    \mathrm{tr}(\mathbf{B})
    =
    \sum_{j=1}^{d}b_{jj}
    \quad 
    \text{for}\ 
    \mathbf{B} \in \mathbb{M}_d.
  $
  The trace is unitarily invariant, i.e.
  $
    \mathrm{tr}(\mathbf{B})
    =
    \mathrm{tr}(\mathbf{Q}\mathbf{B}\mathbf{Q}^*)
    \quad 
    \text{for all}
    \ 
    \mathbf{B}\in \mathbb{M}_d
    \ 
    \text{for all unitary}\ 
    \mathbf{Q} \in \mathbb{M}_d.
  $
  In particular, the existence of an eigenvalue value decomposition shows 
  that the trace of a Hermitian matrix equals the sum of its  eigenvalues.
  Let
  $
  f: I\to \R
  $
  where 
  $I\subseteq\R$ 
  is an interval.
  Consider a matrix 
  $\mathbf{A}\in \mathbb{H}_d$
  whose eigenvalues are contained in $I.$
  We define the matrix 
  $
    f(\mathbf{A})\in \mathbb{H}_d
  $
  using an eigenvalue decomposition of $\mathbf{A}:$
  \begin{gather}
    f(\mathbf{A})
    =
    \mathbf{Q}
    \begin{bmatrix}
      f(\lambda_1) &&\\
                   &\ddots&\\
                   && f(\lambda_d)
    \end{bmatrix}
    \mathbf{Q}^*
    \qquad
    \text{where}
    \qquad
    \mathbf{A}
    =
    \mathbf{Q}
    \begin{bmatrix}
      \lambda_1 &&\\
                   &\ddots&\\
                   && \lambda_d
    \end{bmatrix}
    \mathbf{Q}^*
    .
  \end{gather}
  The definition of $f(\mathbf{A})$ does not depend on which 
  eigenvalue decomposition we choose.
  Any matrix function that arises in this fashion is called a \textbf{standard matrix function}.


\begin{proposition}
  Let
  $
  f,g: I\to \R
  $
  be real-valued functions on an interval $I\subseteq\R,$ 
  and let
  $\mathbf{A}\in \mathbb{H}_d$
  be a Hermitian matrix
  whose eigenvalues are contained in $I.$

  \begin{enumerate}[label={(\roman*)}]
    \item
      If $\lambda$ is an eigenvalue of of $\mathbf{A},$
      then $f(\lambda)$ is an eigenvalue of $f(\mathbf{A}).$
    \item
      $
        f(a)
        \le
        g(a)
        \quad
        \text{for all}\ 
        a\in I
        \quad
        \text{implies}
        \quad
        f(\mathbf{A})
        \preccurlyeq
        g(\mathbf{A})
        .
      $
  \end{enumerate}
\end{proposition}


\begin{lemma}
  \emph{(Mean value trace inequality)}
  Let 
  $I$
  be an interval of the real line. Suppose that
  $
    g:
    I \to \R
  $
  is a weakly increasing function and that 
  $
    h:
    I \to \R
  $
  is a function whose derivative $h^{'}$ is convex.
  Then for all matrices 
  $
    \mathbf{A}
    ,
    \mathbf{B}
    \in 
    \mathbb{H}_d(I)
  $
  it holds
  \begin{gather}
    \overline{\mathrm{tr}}
    [
    (
      g(\mathbf{A}) - g(\mathbf{B})
    )
    \cdot
    (
      h(\mathbf{A}) - h(\mathbf{B})
    )
    ]
    \le
    \frac{1}{2}
    \,
    \overline{\mathrm{tr}}
    [
    (
      g(\mathbf{A}) - g(\mathbf{B})
    )
    \cdot
    (
    \mathbf{A}
    -
    \mathbf{B}
    )
    \cdot
    (
    h^{'}(\mathbf{A}) + h^{'}(\mathbf{B})
    )
    ]
    .
  \end{gather}
  When $h^{'}$ is concave, the inequality is reversed. The same result holds for the standard trace.
\end{lemma}
\begin{proof}
  \emph{\cite[Lemma~3.4]{Mackey2014}}
  Fix 
  $
    a,b
    \in
    I
.
  $
  Since 
  $g$
  is
  weakly increasing,
  $
  (
    g(a) - g(b)
  )
  \cdot
  (a-b)
  \ge 0.
  $
  The 
  fundamental theorem of calculus and the convexity of 
  $h^{'}$
  yield the estimate
  \begin{align}
  (
    g(a) - g(b)
  )
  \cdot
  (
    h(a) - h(b)
  )
  &=
  (
    g(a) - g(b)
  )
  \cdot
  (a-b)
  \int_0^1
  h^{'}
  (
    \tau a + (1-\tau) b
  )
  \mathrm{d}\tau
  \\
  &\le
  (
    g(a) - g(b)
  )
  \cdot
  (a-b)
  \int_0^1
  [
  \tau
  h^{'}
  (
     a 
  )
  +
  (1-\tau)
  h^{'}
  (
     b 
  )
  ]
  \mathrm{d}\tau
  \\
  &=
  \frac{1}{2}
  \,
  [
  (
    g(a) - g(b)
  )
  \cdot
  (a-b)
  \cdot  
  (
    h^{'}
    (a)
    +
    h^{'}
    (b)
  )
  ]
  .
  \end{align}
  The inequality is reversed, if $h^{'}$ is concave.
  $
  $
  To apply the Kleins inequality we expand the terms.
  The RHS is
  \begin{align}
    \begin{split}
  &(
    g(a) - g(b)
  )
  \cdot
  (a-b)
  \cdot  
  (
    h^{'}
    (a)
    +
    h^{'}
    (b)
  )
  \\
  &\quad=
  [
g(a)\cdot a \cdot h^{'}(a)
  ]
    +
    [g(a)\cdot a] \cdot h^{'}(b)
  -
  b \cdot[h^{'}(a)\cdot g(a)]
  -
  [b \cdot h^{'}(b)]\cdot g(a)
  \\
  &\qquad
  +
    [
\ \text{the same as above with $a$ and $b$ reversed}\ 
  ]
  (a \rightleftarrows b)
  \\
  \end{split}
  \end{align}
  Taking the trace yields
  \begin{align}
    \begin{split}
&
\mathrm{tr}
  [
    g(\mathbf{A})
    \cdot
    \mathbf{A}
    \cdot
    (
    h^{'}(\mathbf{A})
    +
    h^{'}(\mathbf{B})
    )
  ]
  -
  \mathrm{tr}
  [
    \mathbf{B}
    \cdot
    (
    h^{'}(\mathbf{A})
    +
    h^{'}(\mathbf{B})
    )
    \cdot
    g(\mathbf{A})
  ]
  +
  (\mathbf{A} \rightleftarrows \mathbf{B})
  \\
  &\quad=
\mathrm{tr}
  [
    g(\mathbf{A})
    \cdot
    \mathbf{A}
    \cdot
    (
    h^{'}(\mathbf{A})
    +
    h^{'}(\mathbf{B})
    )
  ]
  -
  \mathrm{tr}
  [
    g(\mathbf{A})
    \cdot
    \mathbf{B}
    \cdot
    (
    h^{'}(\mathbf{A})
    +
    h^{'}(\mathbf{B})
    )
  ]
  +
  (\mathbf{A} \rightleftarrows \mathbf{B})
  \\
  &\quad=
  \mathrm{tr}
  [
    g(\mathbf{A})
    \cdot
    (
    \mathbf{A}
    -
    \mathbf{B}
    )
    \cdot
    (
    h^{'}(\mathbf{A})
    +
    h^{'}(\mathbf{B})
    )
  ]
  +
  (\mathbf{A} \rightleftarrows \mathbf{B})
  \\
  &\quad=
  \mathrm{tr}
  [
  (
    g(\mathbf{A})
    -
    g(\mathbf{B})
  )
    \cdot
    (
    \mathbf{A}
    -
    \mathbf{B}
    )
    \cdot
    (
    h^{'}(\mathbf{A})
    +
    h^{'}(\mathbf{B})
    )
  ].
    \end{split}
  \end{align}
  On the LHS we have only products of two factors which commute under the trace operation. Thus we may use the same expression as in the scalar case without further calculations.
  The result follows immediately from the Klein inequality.
\end{proof}

\begin{proposition}
  \emph{(Generalized Klein inequality)}
  Let 
  $
    u_1, \ldots, u_n
  $
  and
  $
    v_1, \ldots, v_n
  $
  be real-valued functions on an interval $I$
  of the real line.
  Suppose
  \begin{gather}
    \sum_{k=1}^{n}
    u_k(a)
    v_k(b)
    \ge
    0
    \qquad
    \text{for all}
    \ 
    a,b \in I
    .
  \end{gather}
  Then
  \begin{gather}
    \overline{\mathrm{tr}}
    \left( 
    \sum_{k=1}^{n}
    u_k(\mathbf{A})
    v_k(\mathbf{B})
    \right)
    \ge 0
    \qquad
    \text{for all}
    \ 
    \mathbf{A}, \mathbf{B} \in \mathbb{H}_d(I)
    .
  \end{gather}
\end{proposition}
\begin{proof}
  \emph{\cite[Proposition~3]{Petz1994}}
\end{proof}




\begin{proposition}
  \emph{(Hölder inequality for trace)}
  Let 
  $p$ and $q$
  be Hölder conjugate indices.
  Then
  \begin{gather}
    \mathrm{tr}
    (
    \mathbf{BC}
    )
    \le
    \norm{\mathbf{B}}_p
    \norm{\mathbf{C}}_q
    \qquad
    \text{for all}
    \ 
    \mathbf{B}
    ,
    \mathbf{C}
    \in 
    \mathbb{M}_d
    .
  \end{gather}
\end{proposition}
\begin{proof}
  \cite[Corollary~IV.2.6]{Bhatia1997}
\end{proof}

%  %\section{The Method of Exchangeable Pairs}
%  %\subsection{Matrix Stein pairs}
We first define an exchangable pair.
\begin{definition}
  Let 
  $Z$ and $Z^{'}$
  random variables taking values
  in a Polish space $\mathcal{Z}.$
  We say that
  $
  (
    Z
    ,
    Z^{'}
  )
  $
  is an \textbf{exchangable pair}
  if it has the same distribution as 
  $
  (
  Z^{'}
    ,
    Z
  )
  .
  $
  In particular, 
  $Z$ and $Z^{'}$
  must share the same distribution.
\end{definition}

The following approach originates in the work of Charles Stein \cite{Stein1972} 
on normal approximation for a sum of dependent random variable.
We will explain how some central ideas of this theory extends to matrices.

We can obtain a lot of information about the fluctuation of a random matrix
$\mathbf{X}$
if we can construct a good exchangable pair 
$
(
  \mathbf{X}
  ,
  \mathbf{X}^{'}
)
.
$
With this motivation in mind, let us introduce a special class of exchangable pairs.
\begin{definition}
  Let
  $
  (
    Z
    ,
    Z^{'}
  )
  $
  be an exchangable pair of random variables taking values
  in a Polish space $\mathcal{Z},$
  and let 
  $
    \mathbf{\Psi}
    : 
    \mathcal{Z}
    \to 
  \mathbb{H}_d
  $
  be a measurable function.
  Define the random Hermitian matrices
  \begin{gather}
    \mathbf{X}
    :=
    \mathbf{\Psi}
    (Z)
    \quad
    \text{and}
    \quad
    \mathbf{X}^{'}
    :=
    \mathbf{\Psi}
    (Z^{'})
    .
  \end{gather}
  We say that 
  $
  (
    \mathbf{X}
    ,
    \mathbf{X}^{'}
  )
  $
  is a \textbf{matrix Stein pair}
  if there is a constant 
  $\alpha\in (0,1]$
  for which
  \begin{gather}
    \E[
    \mathbf{X}
    -
    \mathbf{X}^{'}
    \vert
    Z
    ]
    =
    \alpha 
    \mathbf{X}
    \qquad
    \text{almost surely.}
  \end{gather}
  The constant 
  $\alpha$
  is called the \textbf{scale factor} of the pair.
  We always assume 
  $
    \E
    \left[ 
      \norm{\mathbf{X}}^2
    \right]
    <
    \infty
    .
  $
\end{definition}

A matrix Stein pair 
$
(
\mathbf{X}
,
\mathbf{X}^{'}
)
$
has several useful propreties. First,
$
(
\mathbf{X}
,
\mathbf{X}^{'}
)
$
always forms an exchangable pair. Second, it must be the case that
$\E[\mathbf{X}]=\mathbf{0}.$
Indeed,
\begin{gather*}
  \E[\mathbf{X}]
  =
  \frac{1}{\alpha}
  \E[
  \E
  [
    \mathbf{X}
    -
    \mathbf{X}^{'}
    \vert
    Z
  ]
  ]
  =
  \frac{1}{\alpha}
  \E[
    \mathbf{X}
    -
    \mathbf{X}^{'}
    \vert
  ]
  =
  \mathbf{0}
  .
\end{gather*}
\subsection{The method of exchangable pairs}
A well-chosen matrix Stein pair 
$
(
\mathbf{X}
,
\mathbf{X}^{'}
)
$
provides a surprisingly powerful tool for studying 
the random matrix $\mathbf{X}.$
The technique depends on a fundamental technical lemma.
\begin{lemma}
  Suppose that
  $
  (
    \mathbf{X}
    ,
    \mathbf{X}^{'}
  )
  $
  is a matrix Stein pair with scale factor $\alpha.$
  Let 
  $
    \mathbf{F}
    :
    \mathbb{H}_d
    \to
    \mathbb{H}_d
  $
  be a measurable function that satisfies the regularity condition
  $
  \E
  \left[
  \norm{
    (
    \mathbf{X}
    -
    \mathbf{X}^{'}
  )
  \mathbf{F}(\mathbf{X})
  }
  \right]
  <
  \infty
  .
  $
  Then
\begin{gather}
  \E
  [
    \mathbf{X}
    \cdot
    \mathbf{F}
    (\mathbf{X})
  ]
  =
  \frac{1}{2\alpha}
  \E
  [
    (
    \mathbf{X}
    -
    \mathbf{X}^{'}
    )
    (
    \mathbf{F}
    (
    \mathbf{X}
    )
    -
    \mathbf{F}
    (
    \mathbf{X}^{'}
    )
  )
  ]
  .
\end{gather}
\end{lemma}
In short, the randomness in the Stein pair furnishes an alternative expression for the expected product of $\mathbf{X}$
and a function $\mathbf{F}.$
It allows us to estimate the expectation using the smoothness properties of the function $\mathbf{F}$ and the discrepancy between $\mathbf{X}$
and $\mathbf{X}^{'}.$
\begin{proof}
  \emph{\cite[Lemma~2.4]{Mackey2014}}
  Suppose that
  $
  (
    \mathbf{X}
    ,
    \mathbf{X}^{'}
  )
  $
  constructed from an auxiliary exchangable pair 
  $
  (
  Z,
  Z^{'}
  )
  .
  $
  The defining property implies
 \begin{gather}
   \alpha \cdot 
   \E[
   \mathbf{X}
   \cdot
   \mathbf{F}
   (\mathbf{X})
   ]
   =
   \E[
    \E[
    \mathbf{X}
    -
    \mathbf{X}^{'}
    \vert
    Z
    ]
    \cdot
   \mathbf{F}
   (\mathbf{X})
   ]
   =
    \E[
    (
    \mathbf{X}
    -
    \mathbf{X}^{'}
    )
   \mathbf{F}
   (\mathbf{X})
   ]
 \end{gather} 
\end{proof}
\subsection{The conditional variance}
To each matrix Stein pair 
  $
  (
    \mathbf{X}
    ,
    \mathbf{X}^{'}
  ),
  $
  we may associate a random matrix called the conditional variance of $\mathbf{X}.$
  The purpose of this section is to argue that the spectral norm of $\mathbf{X}$
  is unlikely to be large, when the conditional variance is small.
\begin{definition}
  Suppose that
  $
  (
    \mathbf{X}
    ,
    \mathbf{X}^{'}
  ),
  $
  is a matrix Stein pair, constructed from an auxiliary exchangeable pair
  $
  (
  Z
    ,
  Z^{'}
  ).
  $
  The \textbf{conditional variance}
  is the random matrix
  \begin{gather}
    \mathbf{\Delta_X}
    :=
    \mathbf{\Delta_X}
    (Z)
    :=
    \frac{1}{2\alpha}
    \E
    [
    (
    \mathbf{X}
    -
    \mathbf{X}^{'}
    )^2
    \vert
    Z
    ]
    ,
  \end{gather}
  where $\alpha$ is the scale factor of the pair. We may take any version of the conditional expectation in this definition.
\end{definition}

The conditional variance
$
    \mathbf{\Delta_X}
$
can be regarded as a stochastic estimate for the variance 
of the random matrix $\mathbf{X}.$
To see this, assume
Indeed, 
\begin{gather}
  \E
  [
    \mathbf{\Delta_X}
  ]
\end{gather}

% \section{Matrix Khintchin Inequality and Applications}
% In this section we state the matrix Khintchin inequality and matrix moment inequalities as an applications.
% We provide the proof of auxiliary theorems which are cited without proof in \cite{Mackey2014}. They are needed to prove the matrix Khintchin inequality. 
%  \begin{theorem}
  \emph{(Matrix BDG inequality)}
  Let
  $
    p = 1
    \ 
    \text{or}\ 
    p \ge 3/2
    .
  $
  Suppose that 
  $
  (
    \mathbf{X}
   , 
   \mathbf{X}^{'}
  )
  $
  is a matrix Stein pair where
  $
   \E
   [ 
    \norm{\mathbf{X}}
    _{2p}^{2p}
   ]
   <
   \infty
   .
  $
\end{theorem}




\begin{ftheorem}
  \emph{\cite[Corollary~7.3]{Mackey2014}}
  Suppose that
  $
    p = 1
    \ 
    \text{or}\ 
    p \ge 3/2
    .
  $
  Consider a finite sequence
  $
    (\mathbf{Y}_k)_{k\ge 1}
  $
  of independent, random, Hermitian matrices 
  and a deterministic sequence
  $
    (\mathbf{A}_k)_{k\ge 1}
  $
  for which
  \begin{gather}
    \E[\mathbf{Y}_k]
    =
    0
    \quad 
    \text{and}
    \quad
    \mathbf{Y}_k^2
    \preccurlyeq
    \mathbf{A}_k^2
    \qquad
    \text{almost surely for all}\ 
    k \ge 1.
  \end{gather}
  Then
  \begin{gather}
      \E
      \left[
        \norm{
          \sum_{k\ge 1}
            \mathbf{Y}_k
        }
        _{2p}
        ^{2p}
      \right]
      ^{1/(2p)}
      \le
      \sqrt{
        p - \frac{1}{2}
      }
      \,
      \norm{
        \left( 
          \sum_{k\ge 1}
          (
            \mathbf{A}_k^2
            + 
            \E[
              \mathbf{Y}_k^2
            ]
          )
        \right)
        ^{1/2}
        }
      _{2p}
      .
  \end{gather}
  In particular, when 
  $
    (\xi_k)_{k\ge 1}
  $
  is an independent sequence of Rademacher random variables,
  \begin{gather}
      \E
      \left[
        \norm{
          \sum_{k\ge 1}
            \xi_k
            \mathbf{A}_k
        }
        _{2p}
        ^{2p}
      \right]
      ^{1/(2p)}
      \le
      \sqrt{
        2p - 1
        }
      \,
      \norm{
        \left( 
          \sum_{k\ge 1}
            \mathbf{A}_k^2
        \right)
        ^{1/2}
        }
      _{2p}
      .
  \end{gather}
\end{ftheorem}

%  \section{Generalzed Inequalities by Hermitian Dilation}
%  \begin{definition}
  \emph{(Hermitian Dilation)}
  The Hermitian dilation
  \begin{gather*}
    \mathfrak{H} : \C^{d_1 \times d_2} \to \mathbb{H}_{d_1 \times d_2}
  \end{gather*}
  is a map from a general matrix to an Hermitian matrix defined by
  \begin{gather}
    \label{ rmineq_hermitian_dilation } 
    \mathfrak{H}(B)
    :=
    \begin{bmatrix}
      0   & B \\
      B^* & 0 \\
    \end{bmatrix}
  \end{gather}
\end{definition}

%  \begin{ftheorem}
  \emph{(Matrix Rosenthal-Pinelis)}
  \label{rmineq_rosenthal_pinelis}
  Let $\mathbf{A}_1, \ldots, \mathbf{A}_n$ be independent, random matrices with dimension 
  $d_1 \times d_2$.
    Introduce the random matrix
      \begin{gather*}
        \mathbf{S}:=\sum_{k=1}^n \mathbf{A}_k.
      \end{gather*}
    Let $v(\mathbf{S})$ be the matrix variance statistic of the sum:
      \begin{align}
        \label{rmineq_bernstein_cond_2}
        v(\mathbf{S}):= \norm{\E[\mathbf{S}\mathbf{S}^\top ]} \lor \norm{\E[\mathbf{S}^\top \mathbf{S}]} 
             = \norm{\sum_{k=1}^n\E[\mathbf{A}_k\mathbf{A}_k^\top]} \lor \norm{\sum_{k=1}^n\E[\mathbf{A}_k^T \mathbf{A}_k]} .
      \end{align}
    Then
      \begin{align}
        \label{rmineq_rp_expectation_bound}
        \left(
          \E \left[ \norm{\mathbf{S}}^2 \right]
        \right)^{\frac{1}{2}}
        \le
        \sqrt{
          2ev(\mathbf{S})\log(d_1+d_2)
        } 
        + 
        4e \left( 
          \E[\max_{k \le n}\norm{\mathbf{A}_k}^2]
        \right)^\frac{1}{2}
        \log(d_1+d_2).
      \end{align}
\end{ftheorem}

\begin{remark}
  Since
  $
    \E[\norm{S}]
    \le
    \E[\norm{S}^2]^\frac{1}{2}
  $
  by the Cauchy-Schwarz inequality,
  Theorem~\ref{rmineq_rosenthal_pinelis}
  also holds with 
  $
    \E[\norm{S}]
  $
  on the left-hand side of \eqref{rmineq_rp_expectation_bound}.
  To obtain a tail bound we can employ the Markov inequality and 
  Theorem~\ref{rmineq_rosenthal_pinelis}:
  \begin{align}
    \label{rmineq_rp_tail_bound}
    \begin{split}
    \P[&\norm{S}\ge t]
    \\
       &\le
    \frac{
    \E[\norm{S}]
    }{t}
    \le
    \frac{1}{t}
    \left( 
        \sqrt{
          2ev(\mathbf{S})\log(d_1+d_2)
        } 
        + 
        4e \left( 
          \E[\max_{k \le n}\norm{\mathbf{A}_k}^2]
        \right)^\frac{1}{2}
        \log(d_1+d_2)
    \right)
    \quad
    \text{for}\ 
    t>0.
  \end{split}
  \end{align}
  It might be possible to improve the $\log$ term employing an intrinsic dimension argument.
\end{remark}
  

%  
%  \newpage
%  \section{Intrinsic Dimension}
%  \begin{definition}
  \label{rmineq_intrinsic_bernstein}
  For a positive-semidefinite matrix $\mathbf{S}$,
  the \textbf{intrinic dimension} is the quantity
  \begin{gather*}
    \mathrm{intdim}
    (\mathbf{A})
    :=
    \frac{\mathrm{tr}\mathbf{A}}{\norm{\mathbf{A}}}
    .
  \end{gather*}
\end{definition}
\begin{lemma}
  \emph{(Intrinsic dimenision)}
  Let 
  $
    \varphi: [0,\infty) \to \R
  $
  be a convex function with
  $
    \varphi(0)=0.
  $
  For any positive-semidefinite matrix $\mathbf{S}$ it holds that
  \begin{gather*}
    \mathrm{tr}(\varphi(\mathbf{S}))
    \le
    \mathrm{intdim}(\mathbf{S})
    \cdot
    \varphi(\norm{\mathbf{S}})
    .
  \end{gather*}
\end{lemma}
\begin{proof}
  \emph{\cite[Lemma~7.5.1]{Tropp2015}}
  Since $\varphi$ is convex on any interval $[0,L]$ with $L>0$ and $\varphi(0)=0$, it holds
  \begin{gather}
    \varphi(a)
    \le
    \left( 
      1 - \frac{a}{L}
    \right)
    \varphi(0)
    +
    \frac{a}{L}
    \varphi(L)
    =
    \frac{a}{L}
    \varphi(L)
    \qquad
    \text{for all}\ 
    a \in [0,L]
    .
  \end{gather}
  Since $\mathbf{S}$ is positive-semidefinite, the eigenvalues of $\mathbf{S}$ 
  fall in the interval $[0,L]$, where $L=\norm{\mathbf{S}}.$
  \begin{gather}
    \mathrm{tr}(\varphi(\mathbf{S}))
    =
    \sum_{i=1}^{d}
    \varphi(\lambda_i)
    \le
    \frac{
    \sum_{i=1}^{d}
    \lambda_i
    }{\norm{\mathbf{S}}}
    \varphi(\norm{\mathbf{S}})
    =
    \frac{\mathrm{tr}(\mathbf{S})}{\norm{\mathbf{S}}}
    \varphi(\norm{\mathbf{S}})
    =
    \mathrm{intdim}(\mathbf{S})
    \cdot
    \varphi(\norm{\mathbf{S}})
    .
  \end{gather}
\end{proof}
The next example applies the preceding lemma to bound the Schatten-norm in terms of the spectral norm and the intrinsic dimension.
  \begin{example*}
    Let 
    $
      \mathbf{B} \in \mathbb{C}^{m\times n}
    $
    be any rectangular matrix and let $p\ge 2$.
    Then 
    $
      \varphi(x)
      :=
      \left| x \right|
    $
  defines a convex function with $\varphi(0)=0$.
  The intrinsic dimension lemma yields
  \begin{gather}
    \norm{B}^p_p
    =
    \mathrm{tr}
      \left|
      B^* B
      \right|
      ^{p/2}
    \le
    \mathrm{intdim}
    (
      B^* B
    )
      \cdot
      \norm{
      B^* B
    }^{p/2}
    =
    \mathrm{intdim}
    (
      B^* B
    )
      \cdot
      \norm{
        B
    }^{p}
    \,.
  \end{gather}
  If, additionally, $\mathbf{B}$ is self-adjoint and positive-semidefinite
  then it holds 
  \begin{gather}
    \mathrm{tr}
    (
      B^* B
    )
    =
    \mathrm{tr}
    (
    B^2
    )
    =
    \sum_{i=1}^{n} 
    \lambda_i^2
    \le 
    \left( 
    \sum_{i=1}^{n} 
    \lambda_i
    \right)
    ^{2}
    =
    \left( 
    \mathrm{tr}
    B
    \right)
    ^2
    \,,
  \end{gather}
  und consequently
  \begin{gather}
    \norm{B}^p_p
    \le
    \left( 
    \mathrm{intdim}
    B
    \right)
    ^2
      \cdot
      \norm{
        B
    }^{p}
    \,.
  \end{gather}
  \end{example*}
\begin{takeaways}
  Is it intrinsic or extrinsic?
  \lipsum[4]
\end{takeaways}

%
%
%\chapter{Empirical Processes}
%Classical references are \cite{Vaart2000} and \cite{vaart2013}.
%For maximal inequalities see \cite[§19]{Vaart2000}
%For Functional Delta-Method see \cite[§20]{Vaart2000}
%For an introduction to empirical processes and outer expectation 
%see the beginning of \cite{vaart2013}.
%\section{A Primer on Empirical Processes}
%
Let 
$
  \left( 
    \Omega,
    \mathcal{A},
    \P
  \right)
$
be a probability space,
$
  \left( 
    \mathcal{X},
    \Sigma
  \right)
$
a measurable space, and 
$
  X_1,\ldots,X_n
  :
  \left( 
    \Omega,
    \mathcal{A},
    \P
  \right)
  \to
  \left( 
    \mathcal{X},
    \Sigma
  \right)
$
a sample 
of independent and identically-distributed
random variables
with probability distribution $\P_{\!X}$.
Throughout this section we consider the~\textbf{empirical~measure}
of this sample, that is, the discrete random measure
\begin{gather}
  \P_{\!n}:\Sigma \to [0,1]
  ,
  \quad
  C\mapsto 
  \frac{1}{n}
  \#\left\{ 
1\le i \le n \colon
X_i \in C
  \right\}
  \,.
\end{gather}
A family $\mathcal{F}$ of measurable functions 
$
  f:
  \left( 
    \mathcal{X},
    \Sigma
  \right)
    \to
  \left( 
    \R,
    \mathcal{B}(\R)
  \right)
$
induces a stochastic process by
\begin{gather}
  f\mapsto \P_{\!n} f\,,
\end{gather}
where for a measure $Q$ on 
$
  \left( 
    \mathcal{X},
    \Sigma
  \right)
$
we denote $Q f:= \int_\mathcal{X}f\, Q(dx)$.
In this way we define the $\mathcal{F}\!$-indexed \textbf{empirical process} $\G_n$ by
\begin{gather}
  f
  \ 
  \mapsto
  \ 
  \G_n f 
  \ 
  :=
  \ 
  \sqrt{n}
  (\P_{\!n}-\P)f 
  \ 
  =
  \ 
  \frac{1}{\sqrt{n}}
  \sum_{i=1}^{n} 
  (
    f(X_i)
    -
    \P f
  )
  \,.
\end{gather}
The purpose of this notation is to abstract the behaviour of $\G_n$ ranging over $\mathcal{F}$.
Conforming with this integral viewpoint, we define the (random) norm
\begin{gather}
  \norm{\G_n}_\mathcal{F}
  :=
  \sup_
        { f \in \mathcal{F}}
        \left|
          G_n f
        \right|
        .
\end{gather}
We stress that 
$
  \norm{\G_n}_\mathcal{F}
$
often ceases to be measurable, even in simple situations~\cite[page 3]{vaart2013}.
To deal with this, we introduce the notion of \textbf{outer expectation} $\E^*$, that is,
\begin{gather}
  \E^*[T]
  \ 
  :=
  \ 
    \inf
  \left\{ 
    \E[U]
  \ 
  \lvert
  \ 
    U\ge T,
    \ 
    U:
  \left( 
    \Omega,
    \mathcal{A},
    \P
  \right)
  \to 
  \left( 
    \overline{\R},
    \mathcal{B}(\overline{\R})
  \right)
  \text{measurable and}
  \ 
  \E[U]<\infty
  \right\}
  \,.
\end{gather}
In our application the technical difficulties halt at this point, because we only consider $T$ with $\E^*[T]<\infty$. Then there exists a smallest measurable function $T^*$ dominating $T$ with
$\E^*[T]=\E[T^*]$. Thus, we may assume $T$ to be measurable in this regard.

In our application we need concentration inequalities for 
$
  \norm{\G_n}_\mathcal{F}
$.
One easy way is to use maximal inequalities for the expectation together with Markov's inequality. There are also Bernstein-like inequalities for empirical processes.




%
%\section{Maximal Inequalities}
%\input{chapters/empirical_processes/maximal_ineq.tex}

%\section{Functional Delta Method}
%\begin{definition}
  A map 
  $
  \phi:
  \mathbb{D}_\phi
  \to 
  \mathbb{E}
  ,
  $
  defined on a subset 
  $
  \mathbb{D}_\phi
  $
  of a normed space
  $\mathbb{D}$
  that contains 
  $\theta,$
  is called 
  \textbf{Hadamard diffenertiable}
  at $\theta$
  if there exists a continuous,
  linear map
  $
  \phi_\theta^{'}
    :
    \mathbb{D}
    \to 
    \mathbb{E}
  $
  such that
  \begin{gather}
    \norm{
      \frac{
        \phi(\theta + t h_t)
        -
        \phi(\theta)
      }{
        t
      }
      -
      \phi^{'}_\theta
      (h)
    }_\mathbb{E}
    \to
    0
    \quad
    \text{as}
    \ 
    t\searrow 0
    \ 
    \text{for all}
    \ 
    h_t \to h
  \end{gather}
  $
    \text{such that $\theta + th_t$ is contained in $\mathbb{D}_\phi$ for all small $t>0.$}
  $
\end{definition}


\begin{ftheorem}
  \emph{(Delta Method)}
  Let 
  $
    \mathbb{D}
    \ \text{and}
    \ 
    \mathbb{E}
  $
  be normed linear spaces.
  Let
  $
    \phi
    :
    \mathbb{D}_\phi
    \subseteq
    \mathbb{D}
    \to
    \mathbb{E}
  $
  be Hadamard differentiable it $\theta$
  tangentially to 
  $\mathbb{D}_0.$
  Let
  $
    T_n
    :
    \Omega_n
    \to
    \mathbb{D}_\phi
  $
  be maps such that 
  $
    r_n
    (T_n - \theta)
    \rightsquigarrow
    T
  $
  for some sequence of numbers $r_n \to \infty$
  and a random element $T$
  that takes its values in $\mathbb{D}_0.$
  Then 
  $
    r_n(\phi(T_n)-\phi(\theta))
    \rightsquigarrow
    \phi^{'}
    _\theta
    (T)
    .
  $
  If 
  $
    \phi^{'}
    _\theta
  $
  is defined and continuous on the whole space $\mathbb{D},$
  then we also have 
  $
    r_n
    (
      \phi(T_n)
      -
      \phi(\theta)
    )
    =
    \phi^{'}
    _\theta
    (
    r_n
    (
    T_n
    -
    \theta
    )
    )
    +
    o_\P
    (1)
    .
  $
\end{ftheorem}
\begin{proof}
  \cite[Theorem~20.8]{Vaart1998}
\end{proof}


%\chapter{Simple yet useful Calculations} 
%\begin{proposition}
  Let 
  $f : \R^n \to \R$ 
  be continuous such that 
  a minimum $x^*$ exists and is unique.
  Then 
  for all $y \in \R^n$ and $C>0$ 
  it follows
    \begin{gather}
      \inf_{\norm{\Delta}=C} f(y+\Delta) - f(y) > 0 \qquad
      \Rightarrow \qquad 
      \norm{x^* - y} \le C.
    \end{gather}
\end{proposition}


\begin{proof}
Since 
$\mathcal{C}:=\left\{ \norm{\Delta}\le C \right\}$
is compact and
\begin{gather*}
  f(x^*) \le f(y) <  \inf_{\norm{\Delta}=C} f(y+\Delta)
\end{gather*}
the continious function $f(y+\,\cdot\,)$ has a minimum in 
$\overset{\circ}{\mathcal{C}}:=\left\{ \norm{\Delta} < C \right\}$. 
Since 
$x^*$ is the unique minimum of $f$
there exists $\Delta^* \in \overset{\circ}{\mathcal{C}}$ 
such that 
$x^* - y = \Delta^*$.
We conclude that
$\norm{x^* - y} \le C$.
\end{proof}

 %%%%%%%%%%%%%%%%%%%%%%%%%%%%%%%%%%%%%%%%%%%%%%%%%%%%%%%%%%%%%%%%

\begin{proposition}
  Let 
  $f\in C^2(\R)$. 
  Then
  for all $a,x,\Delta \in \R^n$ 
  there exist $\xi_1, \xi_2 \in (0,1)$ such that it holds
  \begin{gather}
    f(a^T (x + \Delta)) - f(a^T x) = 
    f^{'}(a^T x)\,a^T x + 
    f^{''}(a^T (x + \xi_1\xi_2 \Delta))\, \Delta^T A\ \Delta,
  \end{gather}
  where 
  $A:= a a^T \in \R^{n \times n}$ .
\end{proposition}



%
%makeindex main.nlo -s nomencl.ist -o main.nls

\nomenclature{$\P$}{generic probability measure}
\nomenclature{$\overset{\mathcal{D}}{\longrightarrow}$}{convergence of distributions}

%\printnomenclature

\bibliography{literature}{}
\bibliographystyle{alpha}
\end{document}
