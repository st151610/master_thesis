\documentclass[11pt, a4paper, BCOR=7mm, DIV=11]{scrbook}
\linespread{1.25}
\usepackage[utf8]{inputenc}
\usepackage{graphicx}
%\usepackage[a4paper, margin=2.5cm]{geometry}
\usepackage{hyperref}
\usepackage{amsmath}
\usepackage{enumitem}
%\usepackage{mdframed}
\usepackage{amsthm}
\usepackage{amssymb}
\usepackage{imakeidx}
\makeindex[columns=3, title=Index, intoc]
\usepackage{mathtools}
%\usepackage{fdsymbol}
\usepackage{cite}
\graphicspath{ {images/} }
\usepackage{xcolor}         % Extended colors
\usepackage{color}         % Color extended names
\usepackage{nomencl}
\usepackage[nottoc]{tocbibind}
\usepackage{lipsum}
\usepackage[tickmarkheight=0.1cm, colorinlistoftodos]{todonotes}
\makenomenclature
\renewcommand{\nomname}{Notation Index}
\mathsurround=2pt
\usepackage[framemethod=TikZ]{mdframed}
\mdfsetup{%
   middlelinecolor=black,
   middlelinewidth=.5pt,
   backgroundcolor=gray!2,
   roundcorner=5pt}
\newsavebox{\selvestebox}
\newenvironment{takeaways}
  {
   \begin{lrbox}{\selvestebox}%
   \begin{minipage}{12.4cm}
     \textbf{Takeaways}
   }
  {\end{minipage}\end{lrbox}%
   \begin{center}
\setlength\fboxsep{.5cm}
   \colorbox[HTML]{F8E0E0}{\usebox{\selvestebox}}
   \end{center}}
% Theorem handle

\newtheorem{gtheorem}{Theorem}[chapter]
\newenvironment{theorem}
  {\begin{mdframed}\begin{gtheorem}}
  {\end{gtheorem}\end{mdframed}}
\newtheorem*{theorem*}{Theorem}
\newtheorem*{definition*}{Definition}
\newtheorem*{lemma*}{Lemma}
\newenvironment{ftheorem}
  {\begin{mdframed}\begin{gtheorem}}
  {\end{gtheorem}\end{mdframed}}
\newenvironment{ftheorem*}
  {\begin{mdframed}\begin{theorem*}}
  {\end{theorem*}\end{mdframed}}
\newtheorem{fassumption}{Assumption}
\newenvironment{assumption}
{\begin{mdframed}[
   backgroundcolor=red!2,
    ]
    \begin{fassumption}}
  {\end{fassumption}\end{mdframed}}

\newtheorem*{assumptions*}{Assumptions}
\newtheorem{corollary}{Corollary}[gtheorem]

\theoremstyle{definition}
\newtheorem{fdefinition}{Definition}[chapter]
\newenvironment{definition}
{\begin{mdframed}[
   backgroundcolor=gray!2,
    ]
    \begin{fdefinition}}
  {\end{fdefinition}\end{mdframed}}
\newtheorem*{remark/}{Remark}
\newenvironment{remark}
  {\renewcommand{\qedsymbol}{$\diamondsuit$}%
   \pushQED{\qed}\begin{remark/}}
  {\popQED\end{remark/}}
\newtheorem{example/}{Example}[chapter]
\newenvironment{example}
  {\renewcommand{\qedsymbol}{$\diamondsuit$}%
   \pushQED{\qed}\begin{example/}}
  {\popQED\end{example/}}
\newtheorem*{example*/}{Example}
\newenvironment{example*}
  {\renewcommand{\qedsymbol}{$\diamondsuit$}%
   \pushQED{\qed}\begin{example*/}}
  {\popQED\end{example*/}}
\newtheorem*{reflection*/}{Reflection}
\newenvironment{reflection*}
  {\renewcommand{\qedsymbol}{$\spadesuit$}%
   \pushQED{\qed}\begin{reflection*/}}
  {\popQED\end{reflection*/}}
\newtheorem{problem}{Problem}
\newenvironment{fproblem}
  {\begin{mdframed}\begin{problem}}
  {\end{problem}\end{mdframed}}
  %%%% plain %%%%
\theoremstyle{plain}
\newtheorem{flemma}{Lemma}[chapter]
\newenvironment{lemma}
{\begin{mdframed}\begin{flemma}}
  {\end{flemma}\end{mdframed}}
\newtheorem{proposition}{Proposition}[chapter]
\newtheorem*{proposition*}{Proposition}
%\newtheorem{subassumption}{}[assumption]

\renewcommand{\proofname}{\textbf{Proof}}
% definition of constants

\newcommand{\CP}{C_\P}
\newcommand{\Ctau}{C_\tau}
\newcommand{\LearnRate}{\varepsilon_n}

%
\DeclarePairedDelimiterX{\inner}[2]{\langle}{\rangle}{#1, #2}
\newcommand{\C}{\mathbb{C}}
\newcommand{\G}{\mathbb{G}}
\newcommand{\E}{\mathbf{E}}
\renewcommand{\P}{\mathbf{P}}
\newcommand{\R}{\mathbb{R}}
\newcommand\norm[1]{\left\lVert#1\right\rVert}


\setlength {\marginparwidth }{2cm}

\begin{document}

\begin{titlepage}
 \begin{center}
       \vspace*{1cm}
       \textbf{
    A Novel Weighted Mean Approach to Estimate the Distribution Function of Potential Outcomes in Observational Studies 
  }

       \vspace{0.5cm}
       Asymptotic Analysis
            
       \vspace{1.5cm}

       \textbf{Ioan Scheffel}

       \vfill
            
       A thesis presented for the degree of\\
       Master of Science Mathematics
            
       \vspace{0.8cm}
     
     %  \includegraphics[width=0.4\textwidth]{university}
      
       supervised by PD Dr. Jürgen Dippon
       \\
            Institute for Stochastics and Applications
            \\
       Faculty 8: Mathematics and Physics
       \\
University of Stuttgart
\\
submitted at April 18, 2023
   \end{center}
\end{titlepage}
\cleardoublepage
%\maketitle
\frontmatter

\pagenumbering{roman}
\newpage
\begin{center}
\textbf{
Eigenständigkeitserklärung
}
\end{center}
Ich erkläre mit meiner Unterschrift,
dass ich diese Arbeit selbstständig verfasst habe und keine
anderen als die angegebenen Quellen benutzt habe.
Alle Stellen dieser Arbeit, die dem Wortlaut,
dem Sinn oder der Argumentation nach anderen Werken entnommen sind 
(einschließlich des World
Wide Web und anderer elektronischer Text- und Datensammlungen), habe ich unter Angabe der
Quellen vollständig kenntlich gemacht.
\newpage
  \begin{center}
  \textbf{Abstract} \\
  english
  \end{center}
  In this thesis I extend the balancing weights framework of \cite{Wang2019} to estimate the distribution function of potential outcomes in observational studies.
  %
  I also suggest to balance basis functions of non-parametric partitioning estimates. 
  %
  This greatly simplifies the proofs and allows for rigorous mathematical treatment of the method.
  %
  The asymptotic analysis shows convergence of the error to a Gaussian process.
  %
  My findings allow to apply the functional delta method to a large class of plug-in estimators.
  %
  This makes classical statistical methods such as quantile estimation, hypothesis testing or survival analysis accessible to causal inference in observational studies.
  %
  While the theoretical results are promising, this novel approach waits for testing in practice.
\newpage
  \begin{center}
  \textbf{Abstract}
  \\
  german
  \end{center}
  In dieser Arbeit erweitere ich das Balancing Weights Framework von \cite{Wang2019}, um die Verteilungsfunktion von Potential Outcomes in Beobachtungsstudien zu schätzen.
  %
  Ich schlage auch vor, Basisfunktionen nicht-parametrischen Partitioning Estimates auszugleichen.
  %
  Diese Wahl vereinfacht die Beweise wesentlich und erlaubt gründliches mathematisches Vorgehen.
  %
  Die asymptotische Analyse zeigt Konvergenz des Fehlers gegen einen Gaußschen Prozess.
  %
  Die Ergebnisse dieser Arbeit erlauben es die Funktionale Delta Methode auf eine große Klasse von Substitutionsschätzern anzuwenden.
  %
  So erschließt dieser neuartige Zugang der Kausalen Inferenz in Beobachtungsstudien klassische Bereiche der Statistik, wie zum Beispiel Quantil-Schätzung, Hypothesentests, oder Überlebenszeitanalysen.
  %
  Die theoretischen Ergebnisse sind vielversprechend - der Praxis-Test steht allerdings noch aus.
\tableofcontents 
\mainmatter
\pagenumbering{arabic}
\chapter{Introduction}
  
%randomized trials versus observational studies

Is study design more important then statistical analysis?

I think, they are at least equal. 

But a bad analysis can be undone,
whereas a bad design can not.

You have to stick with the data.

If you are not familiar with study design the distinction between randomized and observational study is helpful.

If you read the literature and are unsure about the design of a study, ask for this terms.

You are likely to find an answer.


It is all about how we collect the data.

Say, we want to test the effect of a drug in a study population.

There usually are differences among the units of the study population.

Some are more healthy than others.

We form a treatment and control group, that is, one group takes the drug and the other doesn't.

Then we compare the groups by their health. 

Then a critcal review comes in. What do you mean by healthy.

We mean this and that.

It seems you did not consider this factor.

Maybe the drug is not effective, but the effect we see in your analysis comes from something else.

What do we answer to this?


A good method to avoid this awkward situation is to randomize.

For every unit of the population we toss a fair coin that decides if they get the drug.

Now comes the critic.

From the tables it seems there is an effect. But what about unknown influence?

We answer: Does the coin now of them?

It is not ideal, but this way you can prevent systematic damage to your analysis.


What if we can't decide who gets treatment?

Don't think treatment has to be something good, it should not carry any label of good or bad.

But what about smoking?

Would you smoke if a coin tells you to?

So this is unethical.

But it is also unethical not to investigate the effects of smoking on the health.

Let's accept, that we sometimes (often?) can not control who gets treatment.

Some smoke, some don't, and we mearly observe.

This is typical example for an observational study.

Honestly, this is an oversimplification, but I hope you get the point.

Who still is insulted by the tone will maybe like\cite{Rubin2007}.

%propensity score

In \cite{Rubin2007} you will find the propensity score.

The propensity score is the individual probability to receive treatment, that is,
\begin{gather}
  \P
  [T=1|X]
\end{gather}
if $T$ is the random variable that decides abuot treatment and $X$ is the vector that carries your individual information.

This concept goes back to \cite{Rosenbaum1983}.

It is maybe worth to stop here and think about this definition and its connection to the two study designs.

Discover it for yourself.

\begin{reflection*}
What is the propensity score in the above example.
How does the propensity score behave in rs and os?
\end{reflection*}







\chapter{The Optimization Problem behind the Weights}
  \section{Introduction}
Let  $N\in\mathbb{N}$\index{$N$, sample size} be the sample size, and let $(T_1,X_1),\ldots,(T_N,X_N)$ be independent and identically-distributed copies of $T$ and $X$. 
We gather them in the (random) data set\index{$D_N$, (random) data set without observed outcome} 
\begin{gather*}
D_N
\ 
:=
\ 
\left\{\, (T_i,X_i)\ \colon\  i\in \left\{ 1,\ldots,N \right\}\, \right\}
\qquad
\text{for}\ N\in\mathbb{N}
\,.
\end{gather*}
Furthermore, let \index{$n$, (random) number of treated units}
\begin{gather*}
  n
  \ 
  :=
  \ 
  \# 
  \left\{ 
    i\in \left\{ 1,\ldots,N \right\}
    \ 
    \colon
    \ 
    T_i=1
  \right\}
\end{gather*}
be the number of treated units. This is a random variable. 
To streamline the analysis we assume in this chapter the order $T_i=1$ for all $i\le n$.
We shall drop this assumption at the end of this chapter.


Let
$\# A\in\mathbb{N}_0$ be the number of elements of a finite set $A$, let
$\overline{\R}:=\R\cup \left\{ \infty \right\}$ 
be the extended real numbers, 
$\Sigma_\mathcal{X}$ be a $\sigma$-algebra on the covariate space $\mathcal{X}$,
and let $\mathcal{B}(\R)$ denote the Borel-$\sigma$-algebra on the real numbers.

For a convex function $\varphi\colon \R\to\overline{\R}$,
a finite set of measurable (basis-)functions on the covariate space
\begin{gather*}
  \mathfrak{B}
  :
  =
  \left\{ 
B_k\colon (\mathcal{X},\Sigma_\mathcal{X})\to(\R,\mathcal{B}(\R))
\ 
|
\ 
k\in \left\{ 1,\ldots,\# \mathfrak{B} \right\}
  \right\}
\end{gather*}
and a constraints vector $\delta=[\delta_1,\ldots,\delta_{\# \mathfrak{B}}]^\top>0$
we consider the following (random) convex optimization problem.
\newpage
\begin{fproblem}
  \label{bw:1:primal}
\begin{align*}
  %%%% objective %%%%
    &\underset{w_1, \ldots, w_n \in \R}
    {\text{minimize}}
    &&\qquad\qquad
    \sum_{i = 1}^{n} 
    \varphi(w_i)
    &&&
    \\
    %%%% w_i T_i >= 0 %%%%
    &\text{subject to}
    &&\qquad\qquad
    w_i 
    \ge
    0
    &&&
    \qquad
    \text{for all}\ 
    i \in \left\{ 1, \ldots, n \right\}
    \,,
    \\
    %%%% 1/n sum w = 1 %%%%
    & 
    &&\qquad\qquad
    \frac{1}{N}
    \sum_{i=1}^{n} 
    w_i
    =1
    \\
    %%%% box constraints %%%%
    & 
    &&\qquad
    \left| 
      \frac{1}{N} 
      \left( 
      \sum_{i = 1}^{n} 
      w_i
      B_k(X_i)
      -
      \sum_{i=1}^{N} 
      B_k(X_i)
      \right)
    \right|
    \ 
    \le 
    \ 
    \delta_k
    &&&
    \qquad
    \text{for all}\ 
    k \in \left\{ 1, \ldots, \# \mathfrak{B} \right\}
    \,.
\end{align*}
\end{fproblem}
What is random in Problem~\ref{bw:1:primal}?
First, the dimension of the search space $(w\in\R^n)$ depends on the random variable $n$. 
Thus, we only compute weights for the treated units (the ones with $T_i=1$).
Next consider the \textbf{objective function}
\begin{gather*}
    \sum_{i = 1}^{n} 
    \varphi(w_i)
    \,.
\end{gather*}
The number of summands is random (again $n$). We sometimes use the equivalent notation
\begin{gather*}
    \sum_{i = 1}^{N} 
    T_i
    \cdot
    \varphi(w_i)
    \,,
\end{gather*}
where we set the weights of the untreated (the ones with $T_i=0$) to some arbitrary value in the domain of $\varphi$.
Let's consider the constraints. There is no randomness in the first two constraints.
\begin{gather*}
    w_i 
    \ge
    0
    \qquad
    \text{for all}\ 
    i \in \left\{ 1, \ldots, n \right\}
    \quad
    \text{and}
    \quad
    \frac{1}{N}
    \sum_{i=1}^{n} 
    w_i
    =1
    \,.
\end{gather*}
They only make sure, that the weights (divided by $N$) form a convex combination.
If, for example, the outcome space $\mathcal{Y}$ is convex we make sure that a weighted-mean-estimate of $\E[Y(1)]$ satisfies
\begin{gather*}
  \widehat{Y}(1) 
  \ 
  :=
  \ 
  \frac{1}{N}
  \sum_{i=1}^{n} 
  w_i\cdot Y_i
  \ 
  \in
  \ 
  \mathcal{Y}
\end{gather*}
or that a weighted-mean-estimate of the distribution function of $Y(1)$ satisfies
\begin{gather*}
  \widehat{F}_{Y(1)} 
  \ 
  :=
  \ 
  \frac{1}{N}
  \sum_{i=1}^{n} 
  w_i\cdot \mathbf{1}\left\{ Y_i\le z \right\}
  \ 
  \in
  \ 
  [0,1]
  \,.
\end{gather*}
There remain the box constraints
\begin{gather*}
    \left| 
      \frac{1}{N} 
      \left( 
      \sum_{i = 1}^{n} 
      w_i
      B_k(X_i)
      -
      \sum_{i=1}^{N} 
      B_k(X_i)
      \right)
    \right|
    \ 
    \le 
    \ 
    \delta_k
    \qquad
    \text{for all}\ 
    k \in \left\{ 1, \ldots, \# \mathfrak{B} \right\}
    \,.
\end{gather*}
Here the number of summands in
\begin{gather*}
      \sum_{i = 1}^{n} 
      w_i
      B_k(X_i)
\end{gather*}
is again random, and we sometimes use the equivalent notation
\begin{gather*}
      \sum_{i = 1}^{N} 
      T_i
      \cdot
      w_i
      \cdot
      B_k(X_i)
      \,.
\end{gather*}
The (basis-)functions in $\mathfrak{B}$ can be random functions, for example, if they depend on the data $D_N$.
Also the constraint vector $\delta$ can depend on the data (see \cite[Algorithm~1 on page 11]{Wang2019}).


Next, we formulate and discuss assumptions on (the core of) the objective function $\varphi$.
\subsection{Objective Function}
\begin{assumption}
  \label{asu:objective_f}
  The objective function $\varphi\colon \R\to\overline{\R}$ of Problem~\ref{bw:1:primal} 
  satisfies the following conditions
  \begin{enumerate}[label=(\roman*)]
    \item
      $\varphi(x)=\infty$ for all $x<0$
    \item
      $\varphi$ is strictly convex,
      continuous on $[0,\infty)$,
      and twice continuously differentiable on $(0,\infty)$ with derivatives $\varphi^{'}$ and $\varphi^{''}$
    \item
      $\varphi^{'}(0,\infty)=\R$
    \item
      the inverse of the derivative $(\varphi^{'})^{-1}$ is continuously differentiable on $\R$.
  \end{enumerate}
\end{assumption}
The next lemma provides a link to the assumptions on the objective function in Theorem~\ref{cv:ts:th}.
\newpage
\begin{lemma}
  \label{1065}
  Let Assumption\ref{asu:objective_f} hold true. Then the convex conjugate of $\varphi$ (see \eqref{def:convex_conjugate}) is
  \begin{gather*}
    \varphi^*
    \colon
    \R
    \ 
    \to
    \ 
    \R
    \,,
    \quad
    x^*
    \ 
    \mapsto
    \ 
    x^*
    \!
    \cdot
    (
    \varphi^{'}
    )^{-1}
    (x^*)
    \ 
    -
    \ 
    \varphi
    \left( 
      (
    \varphi^{'}
    )^{-1}
    (x^*)
    \right)
    \,.
  \end{gather*}
  Furthermore, $\varphi^*$ is strictly convex and continuously differentiable on $\R$.
\end{lemma}
\begin{proof}
We define
\begin{gather*}
 \phi
 \ 
 \colon
 [0,\infty)
 \times
 \R
 \ 
 \to
 \ 
 \R
 \,,
 \quad
 (x,x^*)
 \ 
 \mapsto
 \ 
 x\cdot x^*
 \ 
 -
 \ 
 \varphi(x)
 \,.
\end{gather*}
Let $x^*\in\R$.
Since
(by assumption)
      $\varphi$ is continuously differentiable on $(0,\infty)$ with derivative $\varphi^{'}$,
      so is $\phi(\cdot,x^*)$ with derivative
      satisfying 
      \begin{gather*}
        \frac{\partial}{\partial x}
        \phi(x,x^*)
        \ 
        =
 \ 
        x^*
        -
        \varphi^{'}(x)
        \qquad
        \text{for all}\ 
        x\in(0,\infty)
        \,.
      \end{gather*}
  It follows that 
  \begin{gather*}
    z
    \ 
    :=
    \ 
      (
    \varphi^{'}
    )^{-1}
    (x^*)
  \end{gather*}
  is an extreme value of $\phi(\cdot,x^*)$.
  Note, that $x^*\in\R$ is in the domain of 
  $
      (
    \varphi^{'}
    )^{-1}
  $
  by the assumption $\varphi^{'}(0,\infty)=\R$.
  Since $\varphi$ is strictly convex, $\phi(\cdot,x^*)$ is strictly concave. 
  Thus,
  $z>0$ is the unique maximum in $(0,\infty)$.
  By the continuity of $\phi(\cdot,x^*)$ on $[0,\infty)$ it follows, that $z$ is the unique maximum on $[0,\infty)$.
  Thus
  \begin{align*}
    \varphi^*(x^*)
    &
    \ 
    =
    \ 
    \sup_{x\in\R}
    x\cdot x^* - \varphi(x)
    \ 
    =
    \ 
    \sup_{x\in [0,\infty)}
    x\cdot x^* - \varphi(x)
    \ 
    =
    \ 
    \sup_{x\in [0,\infty)}
    \phi(x,x^*)
    \\
    &
    \ 
    =
    \ 
    \phi(z,x^*)
    \\
    &
    \ 
    =
    \ 
    x^*
    \!
    \cdot
    (
    \varphi^{'}
    )^{-1}
    (x^*)
    \ 
    -
    \ 
    \varphi
    \left( 
      (
    \varphi^{'}
    )^{-1}
    (x^*)
    \right)
    \qquad 
    \text{for all}\ 
    x^*\in\R
    \,.
  \end{align*}
  Now we proof the second statement.
  Since
  $
      (
    \varphi^{'}
    )^{-1}
  $
  is (by assumption) continuously differentiable, it holds
  \begin{align}
    \label{0098}
    \begin{split}
    \frac{\partial}{\partial x^*}
     \varphi^*(x^*)
    &
    \ 
    =
    \ 
    (
    \varphi^{'}
    )^{-1}
    (x^*)
    \ 
    +
    \ 
    x^*
    \!
    \cdot
    \frac{\partial}{\partial x^*}
    (
    \varphi^{'}
    )^{-1}
    (x^*)
    \ 
    -
    \ 
    \varphi^{'}
    \left( 
      (
    \varphi^{'}
    )^{-1}
    (x^*)
    \right)
    \cdot
    \frac{\partial}{\partial x^*}
    (
    \varphi^{'}
    )^{-1}
    (x^*)
    \\
    &
    \ 
    =
    \ 
    (
    \varphi^{'}
    )^{-1}
    (x^*)
    \ 
    +
    \ 
    x^*
    \!
    \cdot
    \frac{\partial}{\partial x^*}
    (
    \varphi^{'}
    )^{-1}
    (x^*)
    \ 
    -
    \ 
    x^*
    \cdot
    \frac{\partial}{\partial x^*}
    (
    \varphi^{'}
    )^{-1}
    (x^*)
    \\
    &
    \ 
    =
    \ 
    (
    \varphi^{'}
    )^{-1}
    (x^*)
    \qquad
    \text{for all}\ 
    x^*\in\R
    \,.
    \end{split}
  \end{align}
  Since $\varphi$ is strictly convex and continuously differentiable, 
  $\varphi^{'}$ is continuous and strictly non-decreasing.
  Thus 
  $
    (
    \varphi^{'}
    )^{-1}
  $
  is continuous and strictly non-decreasing.
  It follows from \eqref{0098} that $\varphi^*$ is strictly convex and continuously differentiable.
\end{proof}
The next lemma completes the link.
\begin{lemma}
  \label{9991}
  Let Assumption~\ref{asu:objective_f} hold true. Then 
\begin{gather*}
  \Phi
  \ 
  :
  \ 
  \R^n
  \to
  \ 
  \overline{\R}
  \,
  ,
  \qquad
  [w_1,\ldots,w_n]^\top
  \ 
  \mapsto
  \ 
  \sum_{i=1}^n \varphi(w_i)
  \,,
\end{gather*}
satisfies Assumption~\ref{cv:ts:asu}.
\end{lemma}
\begin{proof}
  By Example~\ref{cv:cc:ex}
  the convex conjugate of $\Phi$ is 
\begin{gather*}
  \Phi^*
  \ 
  :
  \ 
  \R^n
  \to
  \ 
  \overline{\R}
  \,
  ,
  \qquad
  [\lambda_1,\ldots,\lambda_n]^\top
  \ 
  \mapsto
  \ 
  \sum_{i=1}^n \varphi^*(\lambda_i)
  \,,
\end{gather*}
where $\varphi^*$ is the convex conjugate of $\varphi$.
By Assumption~\ref{asu:objective_f} $\varphi$ is strictly convex. Thus,
$\Phi$ is strictly convex. By Lemma~\ref{1065}, $\varphi^*$ continuously differentiable on $\R$. Thus,
$\Phi$ is continuously differentiable on $\R^n$.
It follows the statement of Assumption~\ref{cv:ts:asu} for $\Phi$.
\end{proof}


Next we discuss some concrete choices of $\varphi$

\begin{example}
  For two discrete distributions
  \begin{gather*}
    p:=[p_1,\ldots,p_N]
    \qquad
    \text{and}
    \qquad
    q:=[q_1,\ldots,p_N]
  \end{gather*}
  we consider the following distance measure
  \begin{gather*}
   D(p|q)
   \ 
   :=
   \ 
   \sum_{i=1}^{N} 
   p_i
   \cdot
   \log
   \left( 
     \frac{p_i}{q_i}
   \right)
   \,.
  \end{gather*}
  This is known as the Kullback-Leibler-Entropy.
  In \cite[§3.1]{Hainmueller2012} the author connects this concept to a convex optimization problem.
  The idea is, to optimize the Kullback-Leibler-Entropy of the distribution induced by the weights and some base weights.
  For example, if we choose
  \begin{gather*}
    w:=
    \frac{1}{N}[w_1\cdot T_1,\ldots,w_N\cdot T_N]
    \qquad
    \text{and}
    \qquad
    q:=\frac{1}{N}[1,\ldots,1]
  \end{gather*}
  we get
  \begin{align}
    \label{8926}
   D(w|q)
   \ 
   =
   \ 
   \frac{1}{N}
   \sum_{i=1}^{N} 
w_i\cdot T_i
   \cdot
   \log
   \left( 
     \frac{w_i\cdot T_i}{N}\cdot N
   \right)
   \ 
   =
   \ 
   \frac{1}{N}
   \sum_{i=1}^{n} 
w_i
   \cdot
   \log
   \left( 
     w_i
   \right)
   \,,
  \end{align}
  where we set $"0\cdot \log(0)"=0$. Thus, the optimization problem
  \begin{gather*}
    \underset{w_1, \ldots, w_n \in \R}
    {\text{minimize}}
    \qquad\qquad
    \sum_{i = 1}^{n} 
w_i
   \cdot
   \log
   \left( 
     w_i
   \right)
  \end{gather*}
  produces the same optimal solutions as minimizing the Kullback-Leibler-Entropy \eqref{8926} with respect to $w$.
Thus, we consider 
\begin{gather*}
  \varphi
  \ 
  \colon
  \R
  \ 
  \to
  \ 
  \overline{\R}
  \,,
  \qquad
  x
  \ 
  \mapsto
  \ 
  \begin{cases}
    x\cdot \log x\ &\text{if}\ x>0\,, \\
    0 \ &\text{if}\, x=0\,,\\
    \infty\ &\text{if}\ x<0\,.
  \end{cases}
\end{gather*}
We show, that this choices satisfies Assumption~\ref{asu:objective_f}.
The first point is met by definition.
\end{example}



Next, we derive the dual formulation of Problem~\ref{bw:1:primal}.
\subsection{Dual Problem}
\begin{lemma}
  \label{matrix_notation}
  A matrix formulation of Problem~\ref{bw:1:primal} is 
\begin{align}
  \label{cv:ts:primal}
  %%%% objective %%%%
    &\underset{w \in \R^n}
    {\mathrm{minimize}}
    &&\qquad\qquad
    \Phi(w)
    &&&
    \\
    %%%% Ax >= b %%%%
    \nonumber
    &\mathrm{subject}\ \mathrm{to} 
    &&\qquad\qquad
    \mathbf{U}w
    \ 
    \ge
    \ 
    d
    \,,
    \\
    \nonumber
    &
    &&\qquad\qquad
    \mathbf{A}w
    \ 
    =
    \ 
    a
    \,,
\end{align}
with objective function
\begin{gather*}
  \Phi
  \ 
  :
  \ 
  \R^n
  \to
  \ 
  \overline{\R}
  \,
  ,
  \qquad
  [w_1,\ldots,w_n]^\top
  \ 
  \mapsto
  \ 
  \sum_{i=1}^n \varphi(w_i)
  \,,
\end{gather*}
inequality matrix and vector
\begin{alignat*}{2}
    \mathbf{U}
    &
    \ 
    :=
    \ 
    \begin{bmatrix}
      \mathbf{I}_n
      \\
      \pm\,\mathbf{B}(\mathbf{X})
    \end{bmatrix}
    \in
    \R^{(n+  2 N)\times n}
        \qquad
    &&
d
    \ 
    :=
    \ 
    \begin{bmatrix}
      0_n
      \\
      -N\cdot\delta 
      \ 
      \pm\ 
      \sum_{i = 1}^{N} B(X_i)
    \end{bmatrix}
    \in
    \R^{n+  2 N}
    \,,
    \intertext{and equality matrix and vector}
    \mathbf{A}
    &
    \ 
    :=
    \ 
      \mathrm{1}_n
      ^\top
      \in\R^{1\times n}
      \qquad
    &&
    a
  \ 
    :=
    \ 
    N
    \in\mathbb{N}
    \,.
\end{alignat*}
\end{lemma}

\begin{proof}
  Recall that the box constraints of Problem~\ref{bw:1:primal} are
  \begin{gather*}
        \left| 
      \frac{1}{N} 
      \left( 
      \sum_{i = 1}^{n} 
      w_i
      B_k(X_i)
      -
      \sum_{i=1}^{N} 
      B_k(X_i)
      \right)
    \right|
    \ 
    \le 
    \ 
    \delta_k
    \qquad
    \text{for all}\ 
    k\in \left\{ 1,\ldots, N \right\}
    \,.
  \end{gather*}
  Put differently, it holds both
  \begin{align*}
    -
      \sum_{i = 1}^{n} 
      w_i
      B_k(X_i)
    \ge 
    -
    N
    \delta_k
      -
      \sum_{i=1}^{N} 
      B_k(X_i)
      \quad 
    \text{and}
      \quad
      \sum_{i = 1}^{n} 
      w_i
      B_k(X_i)
    \ge 
    -
    N
    \delta_k
      +
      \sum_{i=1}^{N} 
      B_k(X_i)
  \end{align*}
  for all 
  $
    k\in \left\{ 1,\ldots, N \right\}
  $. In matrix notation this is 
  \begin{gather*}
    \pm\mathbf{B}(\mathbf{X})w
    \ 
    \ge
    \ 
    [d_{n+1},\ldots, d_{n+  2 N}]^\top
    \,.
  \end{gather*}
  Proving the rest of the statements is straightforward. We omit the details.
\end{proof}
\begin{remark}
  The inequality constraints of
  Lemma~\ref{matrix_notation} differ from its counterpart
  \cite[Proof of Lemma~1]{Wang2019}.
  We don't transform the variable $w$, but shift to $d$ what prevents us from keeping $w$.
  Note, that the choice of
  \cite[Proof of Lemma~1]{Wang2019} leads to a mistake on page 21.
  The mistake is most obvious in the second display, where the first implication follows from dividing by 0.
  I discussed this with the authors and proposed a version of Lemma\ref{matrix_notation} to solve the problem. I think it's best not to transform variables, because the mistake comes from (wrongly) calculating the convex conjugate of the (more complicated) transformed version of the objective function. The subsequent analysis even simplifies with my version.

  I was surprised to find the (exact) same mistake in the earlier paper 
  \cite[page 35 second display]{Chan2016}. 
  There is no reference in
  \cite[Proof of Lemma~1]{Wang2019} 
  to
  \cite{Chan2016}. Yet the formulation and the mistake are the same.
  Did the authors of \cite{Wang2019} (inadvertently?) plagiarize
  the mathematical analysis of 
  \cite{Chan2016}
  ?
\end{remark}
%
In the next lemma we apply Theorem~\ref{cv:ts:th} to Problem~\ref{bw:1:primal}.
%
\begin{lemma}
  Consider the optimization problem
\begin{align}
  \label{9993}
  \begin{split}
  \underset
  {\begin{smallmatrix}
\rho\,,\, \lambda^+,\,\lambda^-\ge 0 \\
\lambda_0\in\R
  \end{smallmatrix}}
  {
    \mathrm{maximize}
  }
  \quad
  &
  -
\sum_{i=1} 
  ^n
    \,
  \varphi^*
  \!
  \left( 
    \rho_i
    +
\lambda_0
+
\inner
{B(X_i)}
{
\lambda^+
-
\lambda^-
}
  \right)
  \\
  &
+
\ 
\sum_{i=1}^{N} 
  \left( 
\lambda_0
+
\inner
{B(X_i)}
{
\lambda^+
-
\lambda^-
}
  \right)
  \,
  \ 
-
\ 
\inner
{\delta}
{
\lambda^+
+
\lambda^-
}
  \,.
  \end{split}
\end{align}
If there exists the optimal solution 
$
(\rho^\dagger,\lambda_0^\dagger,\lambda^{+,\dagger},\lambda^{-,\dagger})
$
then the unique optimal solutions to Problem~\ref{bw:1:primal} are 
\begin{gather*}
  w^\dagger_i
  \ 
  :=
  \ 
  (
  \varphi^{'}
  )^{-1}
  \left(
    \rho^\dagger_i
  \ 
    +
  \ 
\lambda_0^\dagger
  \ 
+
  \ 
\inner
{B(X_i)}
{
  \lambda^{+,\dagger}
-
\lambda^{-,\dagger}
}
  \right)
  \qquad
  \text{for all}\ 
  i\in
  \left\{ 1,\ldots,n \right\}
  \,.
\end{gather*}
\end{lemma}
\begin{proof}
  First, note that by the strict convexity of $\varphi^*$ (see Lemma~\ref{1165}), a solution to Problem~\eqref{9993}
  is unique (if it exists).
  By Lemma~\ref{matrix_notation},
  Problem~\ref{bw:1:primal} has the form required in Theorem~\ref{cv:ts:th}.
  By Lemma~\ref{9991}, the objective function $\Phi$ of Problem~\ref{bw:1:primal}
  satisfies Assumption~\ref{cv:ts:asu}.
  Thus we can apply
  Theorem~\ref{cv:ts:th} to Problem~\ref{bw:1:primal}.
  Calculations yield the result.
\end{proof}
With the next theorem we merge $\lambda^{+}$,$\lambda^{-}\ge 0$ to 
$\lambda=\lambda^{+}-\lambda^{-}\in\R$.
\begin{ftheorem}
  \label{dual_solution_th}
  Consider the optimization problem
\begin{align}
  \label{dual}
  \begin{split}
  \underset
  {\begin{smallmatrix}
      \rho&\in&&\R^N 
      \\
      \lambda_0 & \in&&\R
      \\
      \lambda&\in&&\R^{N}
  \end{smallmatrix}}
  {
    \mathrm{minimize}
  }
  \quad
  \frac{1}{N}
\sum_{i=1} 
  ^N
  &
  \Big[
  T_i
  \cdot
  \varphi^*
  \!
  \left( 
    \rho_i
    +
\lambda_0
+
\inner
{B(X_i)}
{
\lambda
}
  \right)
  \ 
  -
  \ 
\lambda_0
-
\inner
{B(X_i)}
{
\lambda
}
\Big]
  \ 
+
\ 
\inner
{\delta}
{
  |\lambda|
}
  \,,
  \\
  \mathrm{subject}\ \mathrm{to}
  \quad
  \qquad
  &
  \rho_i \ge 0 
  \quad 
  \mathrm{for}\ \mathrm{all}\ i\le n
  \qquad 
  \mathrm{and}
  \qquad
  \rho_i=0
  \quad 
  \mathrm{for}\ \mathrm{all}\ i>n
  \,.
\end{split}
\end{align}
If there exists the optimal solution 
$
(\rho^\dagger,\lambda_0^\dagger,\lambda^\dagger)
$
then the unique optimal solutions to Problem~\ref{bw:1:primal} are 
\begin{gather*}
  w^\dagger_i
  \ 
  :=
  \ 
  (
  \varphi^{'}
  )^{-1}
  \left(
    \rho^\dagger_i
  \ 
    +
  \ 
\lambda_0^\dagger
  \ 
+
  \ 
\inner
{B(X_i)}
{
\lambda^{\dagger}
}
  \right)
  \qquad
  \text{for all}\ 
  i\in
  \left\{ 1,\ldots,n \right\}
  \,.
\end{gather*}
\end{ftheorem}

\begin{proof}
  Assume that
$
  (\rho^\dagger,\lambda_0^\dagger,\lambda^{+,\dagger},\lambda^{-,\dagger})
$
is an optimal solution to Problem~\ref{9993}.
We write
\begin{align*}
  G
  (\rho,\lambda_0,\lambda^+,\lambda^-)
  &
  \ 
  :=
  \ 
 -
\sum_{i=1} 
  ^n
    \,
  \varphi^*
  \!
  \left( 
    \rho_i
    +
\lambda_0
+
\inner
{B(X_i)}
{
\lambda^+
-
\lambda^-
}
  \right)
  \\
  &
  \qquad
+
\ 
\sum_{i=1}^{N} 
  \left( 
\lambda_0
+
\inner
{B(X_i)}
{
\lambda^+
-
\lambda^-
}
  \right)
  \,
  \ 
-
\ 
\inner
{\delta}
{
\lambda^+
+
\lambda^-
}
  \,.
\end{align*}
 To eliminate the remaining constraints, 
  we paraphrase \cite[pages~19-20]{Wang2019}.
  We show 
  for all $i \in \left\{ 1,\ldots,N \right\}$
\begin{alignat}{2}
  \notag
  \text{either}
  &
  &&
  \qquad
  \lambda_i^{+,\dagger} > 0
  \\
  \label{9992}
  \text{or}
  &
  &&
  \qquad
  \lambda_i^{-,\dagger} > 0
  \,.
\end{alignat}
Assume towards a contradiction that 
\begin{gather}
  \label{1232}
  \text{
there exists
  } 
  \ 
i \in \left\{ 1,\ldots,N \right\}
\ 
\text{such that}
\qquad
  \lambda_i^{+,\dagger} > 0
  \qquad 
  \text{and}
  \qquad
  \lambda_i^{-,\dagger} > 0
  \,.
\end{gather}
Consider
  \begin{align*}
    \tilde{\lambda}^{+,\dagger}
    &
    \ 
    :=
    \ 
    \begin{bmatrix}
      \ 
      \lambda_1^{+,\dagger}
      \ldots,
      \ 
      \lambda_i^{+,\dagger}
      \!
      \ 
      -
      \ 
      (
      \lambda_i^{+,\dagger}
      \!
      \land
      \lambda_i^{-,\dagger}
      )\,,
      \ 
      \ldots,
      \lambda_{N}^{+,\dagger}
    \end{bmatrix}
    ^\top
    \intertext{and}
    \tilde{\lambda}^{-,\dagger}
    &
    \ 
    :=
    \ 
    \begin{bmatrix}
      \ 
      \lambda_1^{-,\dagger}
      \ldots,
      \ 
      \lambda_i^{-,\dagger}
      \!
      \ 
      -
      \ 
      (
      \lambda_i^{+,\dagger}
      \!
      \land
      \lambda_i^{-,\dagger}
      )\,,
      \ 
      \ldots,
      \lambda_{N}^{-,\dagger}
    \end{bmatrix}
    ^\top
    \,.
  \end{align*}
  Since
  \begin{gather*}
      \lambda_i^{\pm,\dagger}
      \!
      \ 
      -
      \ 
      (
      \lambda_i^{+,\dagger}
      \!
      \land
      \lambda_i^{-,\dagger}
      )
      \ 
      \ge 
      \ 
      0
      \,,
  \end{gather*}
  the perturbed vectors $\tilde{\lambda}^{\pm,\dagger}$ are  in the domain of the 
  optimization problem.
  By Assumption~\eqref{1232} and $\delta>0$ it follows
  \begin{align*}
  G
  \left( 
  \rho^\dagger,\lambda_0^\dagger,\tilde{\lambda}^{+,\dagger},\tilde{\lambda}^{-,\dagger}
  \right)
  \ 
  -
  \ 
  G
  \left( 
  \rho^\dagger,\lambda_0^\dagger,\lambda^{+,\dagger},\lambda^{-,\dagger}
  \right)
  \ 
  =
  \ 
  2
  \cdot
  \delta_i
  \cdot
      (
      \lambda_i^{+,\dagger}
      \!
      \land
      \lambda_i^{-,\dagger}
      )
  \ 
  >
  \ 
  0
  \,,
  \end{align*}
  which contradicts the optimality of
$
  (\rho^\dagger,\lambda^{+,\dagger},\lambda^{-,\dagger},\lambda_0^\dagger)
$
(it is supposed to be a maximum in the domain of the optimization problem)
.
It follows \eqref{9992}.
But then 
$
\lambda^{\pm,\dagger}_i
\ge 0
$
collapses to
$
\lambda_i^\dagger\in \R
$ 
for all
$i\in \left\{ 0,\ldots,N \right\}$, that is, we set
\begin{gather*}
 \lambda_i^\dagger
 \ 
 =
 \ 
 \lambda_i^{+,\dagger}
 \ 
 -
 \ 
 \lambda_i^{-,\dagger}
 \qquad
 \text{and}
 \qquad
|\lambda_i^\dagger|
\ 
=
\ 
\lambda_i^{+,\dagger}
\ 
+
\ 
\lambda_i^{-,\dagger}
\,.
\end{gather*}
Thus, we can extend the domain of Problem~\ref{9993} to $\lambda\in\R^{N}$ and update the objective function in the following way
(without changing the optimal solution).
\begin{align*}
  G
  (\rho,\lambda_0,\lambda)
  &
  \ 
  :=
  \ 
 -
\sum_{i=1} 
  ^n
    \,
  \varphi^*
  \!
  \left( 
    \rho_i
    +
\lambda_0
+
\inner
{B(X_i)}
{
\lambda
}
  \right)
  \\
  &
  \qquad
+
\ 
\sum_{i=1}^{N} 
  \left( 
\lambda_0
+
\inner
{B(X_i)}
{
\lambda
}
  \right)
  \,
  \ 
-
\ 
\inner
{\delta}
{
  |\lambda|
}
  \,.
\end{align*}
Multiplying $G$ with $-1/N$ doesn't change the solution either
(if we search instead for a minimum).
To finish the proof, we choose the notation with $T_i$ instead of $n$. This extends the domain of $\rho$ to $\R^N_{\ge 0}$, but the 
new $\rho_i$ are not effective because of $T_i=0$ for all $i>n$. 
Thus we may set them to 0.
\end{proof}
\begin{remark}
  This is the final form of the dual of Problem~\ref{bw:1:primal}.
  Since the constraints in the dual problem are elementary,
  a result such as Lemma~\ref{lem:link_conv_p}
  keeps the initiative going.
  The dual variables $(\rho,\lambda_0,\lambda)$ are connected to the constraints of Problem~\ref{bw:1:primal}, that is,
  $\rho\in\R^N_{\ge 0}$
  to
  $T_i\cdot w_i\ge 0$ for all $i\in \left\{ 1,\ldots,N \right\}$,
  $\lambda_0\in\R$ 
  to 
  $
  \frac{1}{N}
  \sum_{i=1}^{N} 
  T_i\cdot w_i
  -
  1
  =
  0
  $,
  and
  $\lambda\in\R^N$ to the $N$ box constraints.
\end{remark}
\begin{takeaways}
  We derive a dual formulation of Problem~\ref{bw:1:primal} that is easier to analyse.
  Theorem~\ref{dual_solution_th} provides a functional relationship of optimal dual solutions and 
  optimal weights.
\end{takeaways}


%Bring optimal solutions together. with argmax meas. and Proposition for aSsumption and assumption.

%It is useful too define a weights function
%New chapter?
In the formulation of Theorem~\ref{dual} we encounter "If (...) there exists the optimal solution $(\rho^\dagger,\lambda_0^\dagger,\lambda)$ ... " .
To be able to study asymptotic properties of the solutions we have to become independent of this assumption.
For this we need some tools from functional analysis.
\subsection{Argmax Measurability Theorem}
We follow \cite{Aliprantis2007}
A \textbf{correspondence}$ \psi$ from a set $X$ to a set $Y$ assigns to each $x\in X$ a subset $\psi(x)\subset Y$.
To clarify that we map $x$ to a set, we use the double arrow, that is,
$
  \psi
  \colon
  X
  \twoheadrightarrow
  Y
$.
  Let 
  $(S,\Sigma_S)$ be a measurable space and $X$ a topological space.
  We say, that a correspondence 
  $
  \psi
  \colon
  S
  \twoheadrightarrow
  X
  $
  is 
  \textbf{
  weakly measurable
  },
  if
  \begin{gather*}
    \left\{ 
      s\in S
      \ 
      |
      \ 
      \psi(s)
      \cap
      O
      \neq
      \emptyset
    \right\}
    \in
    \Sigma_S
    \qquad
    \text{for all open subsets}
    \ 
    O\subset X
    \,.
  \end{gather*}

A selector from a correspondence $\psi\colon X\twoheadrightarrow Y$ is a function $s\colon X\to Y$ that satisfies 
$
s(x)\in\psi(x)
$
for all $x\in X$.

\begin{definition}
  Let 
  $(S,\Sigma_S)$ be a measurable space, and let $X$ and $Y$  be topological space.
  A function 
  $f\colon S\times X \to Y$
  is a \textbf{Caratheodory function} if
  \begin{align*}
    f(\cdot,x)
    &
    \colon
    S\to Y
    \qquad
    \text{is}\ 
    (\Sigma_{S},\mathcal{B}(Y))-measurable
    \ 
    \text{for all}
    \ 
    x\in X
    \,,
    \intertext{and}
    f(s,\cdot)
    &
    \colon
    X\to Y
    \qquad
    \text{is continuous for all}\ 
    s\in S
    \,.
  \end{align*}
\end{definition}
\begin{theorem}
  Let $X$ be a separable metrizable space and
  $
  (S,\Sigma_S)
  $
  a measurable space.
  Let $\psi\colon S \twoheadrightarrow X$ be a weakly measurable correspondence with non-empty compact values, and suppose
  $f\colon S\times X \to \R$
  is a Caratheodory function. Define the value function 
  $m\colon S\to \R$ by
  \begin{gather*}
    m(s):=\max_{\psi(s)}f(s,x)
    \,,
  \end{gather*}
  and the correspondence 
  $mu\colon S\twoheadrightarrow X$ of maximizers by
  \begin{gather*}
    \mu(s):= \left\{ 
      x\in \psi(s)
      |
      f(s,x)=m(s)
    \right\}
    \,.
  \end{gather*}
  Then the value function $m$ is measurable, 
  the argmax correspondence $\mu$ has non-empty and compact values,
  is measurable and admits a measurable selector.
\end{theorem}
\begin{proof}
  \cite[Theorem~18.19]{Aliprantis2007}
\end{proof}

Our goal is to find (plausible) assumptions under which we can measurably select optimal solutions of Problem~\ref{dual}.

\begin{assumption}
  There exists $\underline{N}\in\mathbb{N}$ such that 
  for all $N\ge \underline{N}$ 
  there exists a compact and deterministic parameter space
  $
  \Theta_N
  \subset
  \R^N_{\ge 0}
  \times
  \R
  \times
  \R^{N}
  $
  such that for all data sets $D_N$
  a sequence of optimal solution
  $(\rho^\dagger,\lambda_0^\dagger,\lambda^\dagger)$
  satisfying the Lemma
exist and it holds
\begin{gather*}
  (
  \rho^\dagger,
  0_{N-n},
  \lambda_0^\dagger,\lambda^\dagger)
  \in
  \Theta_N
  \,.
\end{gather*}
\end{assumption}

With this assumption we can construct the (constant) correspondence
\begin{align*}
  \psi
  \colon
  (\Omega,\mathcal{A},\P)
  \ 
  \twoheadrightarrow
  \ 
  \R^N_{\ge 0}
  \times
  \R
  \times
  \R^{N}
  \,,
  \qquad
  \omega
  \ 
  \to
  \ 
  \Theta_N
  \,.
\end{align*}
This is weakly measurable, because $\Theta_N$ is deterministic and thus
$\Theta_N\cap O$ is either true or false, that is,
  \begin{gather*}
    \left\{ 
      \omega\in \Omega
      \ 
      |
      \ 
      \psi(\omega)
      \cap
      O
      \neq
      \emptyset
    \right\}
    \ 
    \in
    \ 
    \left\{ \Omega,\emptyset \right\}
    \subset
    \Sigma_S
    \qquad
    \text{for all open subsets}
    \ 
    O
    \subset
  \R^N_{\ge 0}
  \times
  \R
  \times
  \R^{N}
    \,.
  \end{gather*}
  Furthermore the objective function is a Caratheodory function.
  Thus, by Theorem there exists the argmax correspondence  of Problem~\ref{dual} and a measurable selector that selects
  $(\rho^\dagger,0_{N-n},\lambda_0^\dagger,\lambda^\dagger)$ 
  of Assumption.
  With this, we can define the optimal weights processes
  \begin{gather*}
    w^\dagger(x)
    \ 
    :=
    \ 
    w
    \left( 
    x,\rho^\dagger,0_{N-n},\lambda_0^\dagger,\lambda^\dagger
    \right)
    \qquad
    \text{indexed over}\ 
    x\in\mathcal{X}\,.
  \end{gather*}
  Due to the definition of the (general) weights processes and the measurability of the argmax selector, the random variables $w^\dagger(x)$ are measurable for all $x\in\mathcal{X}$.
  To end the measurability discussion, note that $w^\dagger(X)$ is a random variable.

%\subsection{Propensity Score Function}

%\begin{definition}
  We define the \textbf{propensity score function} by
  \begin{gather*}
    \pi
    \ 
    \colon
    \R^d
    \ 
    \to
    \ 
    [0,1]
    \,,
    \qquad
    x
    \ 
    \mapsto
    \ 
    \P
    [
    T=1
    |
    X=x
    ]
    \,,
  \end{gather*}
  that is, 
  as the conditional probability of treatment given individual characteristics $x\in\R^d$.
\end{definition}
\begin{remark}
  We define $\pi$ on the whole $\R^d$, although $\mathcal{X}\subset \R^d$ may be a much smaller (possibly finite or countable) subset.
  The reason is that we want to assume continuity.
\end{remark}
\begin{assumption}
  \label{asu:ps}
  The propensity score function $\pi$ satisfies
  \begin{itemize}
    \item
      $\pi(x)\in (0,1)$ for all $x\in\R^d$
    \item
      $\pi$ is continuously differentiable on $\R^d$
  \end{itemize}
\end{assumption}

Next we give a standard example for a propensity score model

\begin{example}
  Logistic regression
\end{example}


%Next we formulate and discuss assumptions on the basis functions.
%\subsection{Basis Functions}
%\begin{assumption}
  \label{asu:basis}
  The set of (measurable) basis functions 
\begin{gather*}
  \mathfrak{B}
  :
  =
  \left\{ 
B_k\colon (\mathcal{X},\Sigma_\mathcal{X})\to(\R,\mathcal{B}(\R))
\ 
|
\ 
k\in \left\{ 1,\ldots,\# \mathfrak{B} \right\}
  \right\}
\end{gather*}
  satisfies
  \begin{itemize}
    \item
      $
      \sum_{k=1}^{\# B} 
      B_k(X_i)
      \ 
      =
      \ 
      1
      $
      for all 
      $i\in \left\{ 1,\ldots ,N \right\}$
    \item
      $
      \sum_{k=1}^{\# B} 
      B_k^2(x)
      \ 
      \lesssim
      \ 
      1
      $
      for all $x\in\mathcal{X}$.
    \item
      $
   \sum_{k=1}^{\# \mathfrak{B}} 
      B_k(X_i)
      \cdot
      \norm{X_k - X_i}_2
      \to
      0
      \ 
      $
      for all 
      $
      (i,k)
      \in
      \left\{ 1,\ldots, N \right\}
      \times
      \left\{ 1,\ldots,\# B \right\}
      $
      and
      $N\to\infty$
  \end{itemize}
\end{assumption}


The next lemma connects objective function $\varphi$, the set of basis functions $\mathfrak{B}$ and the propensity score.

\begin{lemma}
  Let Assumptions~\ref{asu:objective_f}, \ref{asu:ps} and \ref{asu:basis} hold true.
  Then for all
  $i\in \left\{ 1,\ldots,N \right\}$ it holds
  \begin{gather*}
   \left| 
   \sum_{k=1}^{\# \mathfrak{B}} 
   B_k(X_i)
   \cdot
   \varphi^{'}
   \left( 
     \frac{1}{\pi(X_k)}
   \right)
   \ 
   -
   \ 
   \varphi^{'}
   \left( 
     \frac{1}{\pi(X_i)}
   \right)
   \right| 
   \ 
   \to
   \ 
   0
   \qquad
   \text{for}
   \ 
   N\to\infty
   \,.
  \end{gather*}
\end{lemma}

\begin{proof}
  Consider the function 
  \begin{gather*}
  g
  \ 
  :=
  \ 
  \varphi^{'}
  \ 
  \circ 
  \ 
  (x\mapsto 1/x) 
  \ 
  \circ
  \ 
  \pi
  \,.
  \end{gather*}
  Then $g$ is continuously differentiable.
  This follows from the assumption that
  $\varphi$ is twice continuously differentiable on $(0,\infty)$, $\pi\in (0,1)$ and that $\pi$ is continuously differentiable on 
  $\R^d$.
  Thus, by the chain rule we get
  \begin{gather*}
   \nabla
   g
   \ 
   =
   \ 
   x
   \ 
   \mapsto
   \ 
   \frac{\nabla \pi(x)}{\pi^2(x)}
   \cdot
   \varphi^{''}\left( \frac{1}{\pi(x)} \right)
   \,.
  \end{gather*}
  Note, that $\nabla g$ is well defined, continuous on $\R^d$
  and does not depend on $N$.
  Since we assume
      $
      \sum_{k=1}^{\# B} 
      B_k(X_i)
      \ 
      =
      \ 
      1
      $
      for all 
      $i\in \left\{ 1,\ldots ,N \right\}$
      it holds
      \begin{align*}
   &
   \left| 
   \sum_{k=1}^{\# \mathfrak{B}} 
   B_k(X_i)
   \cdot
   \varphi^{'}
   \left( 
     \frac{1}{\pi(X_k)}
   \right)
   \ 
   -
   \ 
   \varphi^{'}
   \left( 
     \frac{1}{\pi(X_i)}
   \right)
   \right| 
   \\
   &
   \ 
   =
   \ 
   \left| 
   \sum_{k=1}^{\# \mathfrak{B}} 
   B_k(X_i)
   \left( 
   \varphi^{'}
   \left( 
     \frac{1}{\pi(X_k)}
   \right)
   \ 
   -
   \ 
   \varphi^{'}
   \left( 
     \frac{1}{\pi(X_i)}
   \right)
   \right)
   \right| 
\\
   &
   \ 
   \le
   \ 
   \sum_{k=1}^{\# \mathfrak{B}} 
   B_k(X_i)
   \left| 
   \left( 
   \varphi^{'}
   \left( 
     \frac{1}{\pi(X_k)}
   \right)
   \ 
   -
   \ 
   \varphi^{'}
   \left( 
     \frac{1}{\pi(X_i)}
   \right)
   \right)
   \right| 
\\
   &
   \ 
   \le
   \ 
   \sum_{k=1}^{\# \mathfrak{B}} 
   B_k(X_i)
   \sup_{\theta\in[0,1]}
   \norm{
     \nabla g
     (
     \theta X_k
     +
     (1-\theta)
     X_i
     )
   }_2
   \norm{X_k-X_i}_2
\\
   &
   \ 
   \lesssim
   \ 
   \sum_{k=1}^{\# \mathfrak{B}} 
   B_k(X_i)
   \norm{X_k-X_i}_2
   \ 
   \to
   \ 
   0
   \qquad
   \text{for}
   \ 
   N\to\infty
   \,.
      \end{align*}
\end{proof}

Next, we give some examples. 
We begin with a histogram basis \cite[§4]{Gyorfi2002}
\begin{example}

\end{example}

Next, we consider kernel bases \cite[§5]{Gyorfi2002}
\begin{example}

\end{example}

\subsection{Pseudo Solutions}
There are two observations that help achieving our goal.
First, if we solve Problem~\ref{dual} on a compact search space, there always exists a measurable solution (if the objective function is Caratheodory).
Second (with hindsight), we want optimal solutions to estimate an oracle parameter $\lambda^*$ well. 
To this end, we will define a (random) compact parameter space that always contains the oracle parameter $\lambda^*$ and restrict Problem~\ref{dual} to it.
Then Theorem~\ref{th:argmax} gives us (under weak measurability conditions) a measurable solution of the restricted Problem~\ref{dual}.
In some cases, this is only a local solution. It becomes global if it is in the interior of the compact parameter space.
Thus we call it the restricted, or the pseudo solution.

Consider 
the random vector (the oracle parameter)
\begin{align*}
  \lambda^*
  \ 
  :=
  \ 
  \left(
  0_{N},0,
  \left[
  \varphi^{'}
  \left(
  \frac
  {1}
  {\pi(X_i)}
  \right)
\right]_{i\in \left\{
  1,\ldots,N
\right\}}
  \right)
\end{align*}
with values in 
$\R^N_{\ge 0}\times \R\times \R^N$.
Clearly,  
\begin{align*}
 \lambda^*
 \qquad
 \text{is}
 \qquad
 \left(
( 
\sigma
\left( 
  (T_i,X_i)_{i\in 
\left\{ 1,\ldots,N \right\}
  } 
\right)
,
\mathcal{B}
\left( 
\R^N_{\ge 0}\times \R\times \R^N
\right)
\right)
-\text{measurable}
\,.
\end{align*}
Next we define for $N\in\mathbb{N}$ the correspondence  
\begin{align*}
  \Theta_N\colon
  &
  \left( 
  \Omega,
\sigma
\left( 
  (T_i,X_i)_{i\in 
\left\{ 1,\ldots,N \right\}
  } 
\right)
  \right)
  \ 
\to
  \ 
\R^N_{\ge 0}\times \R\times \R^N
\\
&
\omega
  \ 
\mapsto
  \ 
\left\{ 
  \left( 
  \rho,\lambda_0,\lambda
  \right)
  \ 
  \in
  \ 
\R^N_{\ge 0}\times \R\times \R^N
  \ 
\colon
  \ 
\norm{
  \left( 
  \rho,\lambda_0,\lambda
  \right)
-\lambda^*}_2\le 1
\right\}
\,.
\end{align*}

\begin{lemma}
  \label{Theta_maes}
  For all $N\in\mathbb{N}$ it holds that 
  $\Theta_N$ is a weakly-measurable correspondence  with non-empty compact values.
\end{lemma}
\begin{proof}
  Let $N\in\mathbb{N}$. 
  Since 
  \begin{align*}
  \Theta_N(\omega)
  \ 
  \text{
  is the closed unit ball
  in
  }
  \ 
\R^N_{\ge 0}\times \R\times \R^N
\quad 
\text{around}
\quad 
  \lambda^*(\omega) 
  \,,
  \end{align*}
  it is non-empty and compact for all $\omega\in\Omega$.
  To prove that $\Theta_N$ is weakly-measurable,
  let 
  $
  O\subset
\R^N_{\ge 0}\times \R\times \R^N
  $
  be an open set and note, that
  \begin{align*}
    \left\{ 
      \Theta_N\cap O \neq \emptyset
    \right\}
    &
    \ 
    =
    \ 
    \left\{ 
      \exists
  \left( 
  \rho,\lambda_0,\lambda
  \right)
  \ 
  \in
  \ 
  O
\ 
\colon
\ 
\norm{
  \left( 
  \rho,\lambda_0,\lambda
  \right)
-\lambda^*}_2\le 1
    \right\}
    \\
    &
    \ 
    =
    \ 
    \left\{ 
      \exists
  \left( 
  \rho,\lambda_0,\lambda
  \right)
  \ 
  \in
  \ 
  O
  \cap
  \left( 
  \mathbb{Q}^N_{\ge 0}\times \mathbb{Q}\times \mathbb{Q}^N
  \right)
\ 
\colon
\ 
\norm{
  \left( 
  \rho,\lambda_0,\lambda
  \right)
-\lambda^*}_2\le 1
    \right\}
    \\
&
    \ 
    =
    \ 
    \bigcup_{
  \left( 
  \rho,\lambda_0,\lambda
  \right)
  \ 
  \in
  \ 
  O
  \cap
  \left( 
  \mathbb{Q}^N_{\ge 0}\times \mathbb{Q}\times \mathbb{Q}^N
  \right)
    }
    \left\{ 
\norm{
  \left( 
  \rho,\lambda_0,\lambda
  \right)
-\lambda^*}_2\le 1
    \right\}
    \,.
  \end{align*}
  Since
  \begin{align*}
    \left\{ 
\norm{
  \left( 
  \rho,\lambda_0,\lambda
  \right)
-\lambda^*}_2\le 1
    \right\}
    \in
\sigma
\left( 
  (T_i,X_i)_{i\in 
\left\{ 1,\ldots,N \right\}
  } 
\right)
\quad
\text{for all}\quad
  \left( 
  \rho,\lambda_0,\lambda
  \right)
  \in
\R^N_{\ge 0}\times \R\times \R^N
\,,
  \end{align*}
  and the union is countable, it follows
  $
    \left\{ 
      \Theta_N\cap O \neq \emptyset
    \right\}
    \in
\sigma
\left( 
  (T_i,X_i)_{i\in 
\left\{ 1,\ldots,N \right\}
  } 
\right)
  $.
  Thus $\Theta_N$ is weakly-measurable.
\end{proof}
We are ready to construct the measurable (pseudo) solution.
\begin{lemma}
  \label{lem:pseud_sol}
  For all $N\in\mathbb{N}$ there exists a (pseudo) solution
  \begin{align*}
    s_N
    \ 
    \colon
   \  
    \Omega
    \ 
    \to
    \ 
  \R^N_{\ge 0}
  \times
  \R
  \times
  \R^{N}
  \end{align*}
  to
  Problem~\ref{dual} restricted to $\Theta_N$ 
  that is
  \begin{align*}
  \left(
    \sigma \left( \left( T_i,X_i \right)_{i\in \left\{ 1,\ldots,N \right\}} \right),\mathcal{B}
  \left(
  \R^N_{\ge 0}
  \times
  \R
  \times
  \R^{N}
  \right)
  \right)
  -\text{measurable}
  \,.
  \end{align*}
  Furthermore,
if $s_N\in \mathrm{int}\, \Theta_N$, then it is the global solution
  $
  (\rho^\dagger,\lambda_0^\dagger,\lambda^\dagger)
  $.
\end{lemma}
\begin{proof}
  By Lemma~\ref{Theta_maes}
  the correspondence $\Theta_N$ satisfies the conditions of
  Theorem~\ref{th:argmax}.
Next, we consider the (random) objective function of (the maximize version of) Problem~\ref{dual}, that is,
  \begin{align*}
    &
  G\colon
  \left(
  \Omega,
\sigma
\left( 
  (T_i,X_i)_{i\in 
\left\{ 1,\ldots,N \right\}
  } 
  \right)
  \right)
  \times
  \left(
  \R^N_{\ge 0}
  \times
  \R
  \times
  \R^{N}
  \right)
  \to
  \R\cup \left\{
    -\infty
  \right\}
  \intertext{with}
    &
  G(\omega,(\rho,\lambda_0,\lambda))
  \ 
  =
  \ 
  -\infty
  \qquad 
  \text{if}\ 
  \rho_i\neq 0\  \text{for some}\ i>n
  \,,
  \\
  \intertext{and}\qquad
  &
  G(\omega,(\rho,\lambda_0,\lambda))
  \\
  &
  \ 
  =
  \ 
  \frac{1}{N}
\sum_{i=1} 
  ^N
  \Big[
    -
  T_i(\omega)
  \cdot
  \varphi^*
  \!
  \left( 
    \rho_i
    +
\lambda_0
+
\inner
{B(X_i)(\omega)}
{
\lambda
}
  \right)
  \ 
  +
  \ 
\lambda_0
+
\inner
{B(X_i)(\omega)}
{
\lambda
}
\Big]
  \ 
-
\ 
\inner
{\delta(\omega)}
{
  |\lambda|
}
\,.
  \end{align*}
  Clearly, the (random) objective function $G$  is a Caratheodory function
  . This follows from the (assumed) continuity of $\varphi^*$ and the measurability 
  of all random variables included.
  Since $G$ is also strictly concave, it has a unique argmax $s_N$  in $\Theta_N$. 
  By Theorem~\ref{th:argmax} this is $
  \left(
    \sigma
    \left( \left( T_i,X_i \right)_{i\in \left\{ 1,\ldots,N \right\}}\right)
    ,\mathcal{B}
  \left(
  \R^N_{\ge 0}
  \times
  \R
  \times
  \R^{N}
  \right)
  \right)
$-measurable.
Furthermore, by the strict concavity of $G$,
if $s_N\in \mathrm{int}\,\Theta_N$, then it's the global optimal solution. 
\end{proof}
We will prove later, that with probability going to 1 the latter will be the case.
Furthermore, we will prove, that the (measurable) solution converges to the random variable in probability.


Before we define the weights function we specify the basis $\mathfrak{B}$.
\subsection{basis}
Let $
\left(
\mathcal{P}_N
\right)
$
denote a sequence of countable, $\mathcal{B}$-measurable partitions 
\begin{align*}
\mathcal{P}_N= \left\{
  A_{N,1},
  A_{N,2},
  \ldots
\right\}
\subset \mathcal{B}(\R^d)
\end{align*}
of $\R^d$, that is, 
\begin{align*}
  A_{N,i}\cap A_{N,j}=\emptyset
  \qquad
  \text{if}\ i\neq j
  \qquad
  \text{and}
  \qquad 
  \bigcap_{i\in\mathbb{N}}A_{N,i}
  \ 
  =
  \ 
  \R^d
  \,.
\end{align*}
We define
$ A_N(x) $ to be the cell of $ \mathcal{P}_N $ containing $x$, that is,
\begin{align*}
  A_N
  \colon
  \R^d 
  \ 
  \twoheadrightarrow 
  \ 
  \R^d  
  \,,\qquad
  x
  \ 
  \mapsto
  \ 
  A_N(x)
  \,,
\end{align*}
where $A_N(x)$ is the only cell containing $x$. 

Next, we define the (empirical) basis functions vector
\begin{align}
  \label{def:basis}
  B\colon
  \R^d\times \R^{d\cdot N}
  \ 
  \to
  \ 
  \R
  \,,
  \qquad
  (x,(x_1,\ldots,x_N))
  \ 
  \mapsto
  \ 
  \frac
  {
    \left[
    \mathbf{1}
    _{
      A_N(x)
    }
    (x_k)
    \right]
    _{k\in \left\{
        1,\ldots,N
    \right\}}
  }
  {
    \sum_{j=1}^N
    \mathbf{1}
    _{
      A_N(x)
    }
    (x_j)
    }
  \,,
\end{align}
where we keep to the convention $"0/0=0"$.
We shall extend $B$ to depend on the random vectors
$X,X_1,\ldots,X_N$.
The next lemma studies the measurability of the extensions.
\begin{lemma}
  \quad
  \begin{enumerate}[label=(\roman*)]
\item
  $B(\cdot,(X_1,\ldots,X_N))(\omega)$ is 
  $\left(
    \mathcal{B}(\R^d),\mathcal{B}(\R^N)
  \right)$-measurable
  and
  constant on each cell 
  $A_N\in\mathcal{P}_N$
  for all $\omega\in\Omega$. 
\item
  $B(X,(X_1,\ldots,X_N))$ is $\left(
    \mathcal{A},\mathcal{B}(\R^N)
  \right)$-measurable. 
  \end{enumerate}
\end{lemma}

\begin{proof}
Consider
for $k\in \left\{
  1,\ldots,N
\right\}$
and $\omega\in\Omega$
the indicator function
\begin{align}
  \label{33342}
  \mathbf{1}
  _
  {A_N(X_k(\omega))}
  \colon \R^d\to \left\{
    0,1
  \right\}
  \,.
\end{align}
Since 
$
  {A_N(X_k(\omega))}
  \in\mathcal{B}(\R^d)
$
this is a 
  $\left(
    \mathcal{B}(\R^d),\mathcal{B}(\R)
  \right)$-measurable
  function.
  From the definition of $B$ \eqref{def:basis} it follows the first part of (i).
  Since the indicator function in \eqref{33342} is 1 if $
  x\in
  {A_N(X_k(\omega))}
  $
  and 0 else, it is also constant on each cell
  $A_N\in\mathcal{P}_N$.
  It follows (i).
  To prove (ii), note that
\begin{align*}
  \mathbf{1}
  _
  {A_N(X_k(\omega))}(X(\omega))
  \ 
  =
  \ 
  \mathbf{1}
  \bigcup_{i\in\mathbb{N}}
  \left\{
    X,X_k \in A_{N,i}
  \right\}
  (\omega)
  \qquad
  \text{for all}\ 
  \omega\in\Omega
  \,,
\end{align*}
and
$
  \bigcup_{i\in\mathbb{N}}
  \left\{
    X,X_k \in A_{N,i}
  \right\}
  \in\mathcal{A}
$ by the 
$
\left( 
  \mathcal{A},
  \mathcal{B}(\R^d)
\right)
$-
measurability of $X$ and $X_k$.
\end{proof}

\subsection{Weights Function}
Theorem~\ref{dual_solution_th} tells us that if an optimal solution
$
(\rho^\dagger,\lambda_0^\dagger,\lambda^\dagger)
$
to Problem~\ref{dual} exists,
then the unique optimal solutions to Problem~\ref{bw:1:primal} are 
\begin{gather*}
  w^\dagger_i
  \ 
  :=
  \ 
  (
  \varphi^{'}
  )^{-1}
  \left(
    \rho^\dagger_i
  \ 
    +
  \ 
\lambda_0^\dagger
  \ 
+
  \ 
\inner
{B(X_i)}
{
\lambda^{\dagger}
}
  \right)
  \qquad
  \text{for all}\ 
  i\in
  \left\{ 1,\ldots,n \right\}
  \,.
\end{gather*}

This point of view is sufficient from a practical point of view
(when we are interested in actually computing the optimal weights).

For the subsequent analysis we need to view the weights as random quantities. To this end, we shall introduce the weights process. 

\begin{definition}
  \label{def:weights_function}
  We define the weights process to be a stochastic process indexed 
  over
  $
  \mathcal{X}
  \,
  \times
  \,
  \R^N_{\ge 0}
  \,
  \times
  \,
  \R
  \times
  \,
  \R^{N}
  $
  with values in $\R^N$ such that
\begin{align*}
  w
  \left( 
  x
  ,
  \rho
  ,
  \lambda_0
  ,
  \lambda
  \right)
  \ 
  :=
  \ 
  \left[ 
    (
    \varphi^{'}
    )^{-1}
    \left( 
     \rho_i 
     +
      \lambda_0
      +
      \inner{B(x)}{\lambda}
    \right)
  \right]_{i\in \left\{ 1, \ldots,N \right\}}
    \,.
\end{align*}
\end{definition}
By the measurability of the basis functions (or the basis processes),
the weights processes is also measurable 
(the quantities 
$
  w
  \left( 
  x
  ,
  \rho
  ,
  \lambda_0
  ,
  \lambda
  \right)
$
are random variables).
The next question is if there are (plausible) assumptions such that the weights processes at a random parameter
$
(\rho^\dagger,\lambda_0^\dagger,\lambda^\dagger)
$
is measurable.
To answer this in a good way we need the argmax measurability theorem \cite[Theorem~18.19]{Aliprantis2007}.
\subsection{Argmax Measurability Theorem}
We follow \cite{Aliprantis2007}
A \textbf{correspondence}$ \psi$ from a set $X$ to a set $Y$ assigns to each $x\in X$ a subset $\psi(x)\subset Y$.
To clarify that we map $x$ to a set, we use the double arrow, that is,
$
  \psi
  \colon
  X
  \twoheadrightarrow
  Y
$.
  Let 
  $(S,\Sigma_S)$ be a measurable space and $X$ a topological space.
  We say, that a correspondence 
  $
  \psi
  \colon
  S
  \twoheadrightarrow
  X
  $
  is 
  \textbf{
  weakly measurable
  },
  if
  \begin{gather*}
    \left\{ 
      s\in S
      \ 
      |
      \ 
      \psi(s)
      \cap
      O
      \neq
      \emptyset
    \right\}
    \in
    \Sigma_S
    \qquad
    \text{for all open subsets}
    \ 
    O\subset X
    \,.
  \end{gather*}

A selector from a correspondence $\psi\colon X\twoheadrightarrow Y$ is a function $s\colon X\to Y$ that satisfies 
$
s(x)\in\psi(x)
$
for all $x\in X$.

\begin{definition}
  Let 
  $(S,\Sigma_S)$ be a measurable space, and let $X$ and $Y$  be topological space.
  A function 
  $f\colon S\times X \to Y$
  is a \textbf{Caratheodory function} if
  \begin{align*}
    f(\cdot,x)
    &
    \colon
    S\to Y
    \qquad
    \text{is}\ 
    (\Sigma_{S},\mathcal{B}(Y))-measurable
    \ 
    \text{for all}
    \ 
    x\in X
    \,,
    \intertext{and}
    f(s,\cdot)
    &
    \colon
    X\to Y
    \qquad
    \text{is continuous for all}\ 
    s\in S
    \,.
  \end{align*}
\end{definition}
\begin{theorem}
  Let $X$ be a separable metrizable space and
  $
  (S,\Sigma_S)
  $
  a measurable space.
  Let $\psi\colon S \twoheadrightarrow X$ be a weakly measurable correspondence with non-empty compact values, and suppose
  $f\colon S\times X \to \R$
  is a Caratheodory function. Define the value function 
  $m\colon S\to \R$ by
  \begin{gather*}
    m(s):=\max_{\psi(s)}f(s,x)
    \,,
  \end{gather*}
  and the correspondence 
  $mu\colon S\twoheadrightarrow X$ of maximizers by
  \begin{gather*}
    \mu(s):= \left\{ 
      x\in \psi(s)
      |
      f(s,x)=m(s)
    \right\}
    \,.
  \end{gather*}
  Then the value function $m$ is measurable, 
  the argmax correspondence $\mu$ has non-empty and compact values,
  is measurable and admits a measurable selector.
\end{theorem}
\begin{proof}
  \cite[Theorem~18.19]{Aliprantis2007}
\end{proof}

Our goal is to find (plausible) assumptions under which we can measurably select optimal solutions of Problem~\ref{dual}.

\begin{assumption}
  There exists $\underline{N}\in\mathbb{N}$ such that 
  for all $N\ge \underline{N}$ 
  there exists a compact and deterministic parameter space
  $
  \Theta_N
  \subset
  \R^N_{\ge 0}
  \times
  \R
  \times
  \R^{N}
  $
  such that for all data sets $D_N$
  a sequence of optimal solution
  $(\rho^\dagger,\lambda_0^\dagger,\lambda^\dagger)$
  satisfying the Lemma
exist and it holds
\begin{gather*}
  (
  \rho^\dagger,
  0_{N-n},
  \lambda_0^\dagger,\lambda^\dagger)
  \in
  \Theta_N
  \,.
\end{gather*}
\end{assumption}

With this assumption we can construct the (constant) correspondence
\begin{align*}
  \psi
  \colon
  (\Omega,\mathcal{A},\P)
  \ 
  \twoheadrightarrow
  \ 
  \R^N_{\ge 0}
  \times
  \R
  \times
  \R^{N}
  \,,
  \qquad
  \omega
  \ 
  \to
  \ 
  \Theta_N
  \,.
\end{align*}
This is weakly measurable, because $\Theta_N$ is deterministic and thus
$\Theta_N\cap O$ is either true or false, that is,
  \begin{gather*}
    \left\{ 
      \omega\in \Omega
      \ 
      |
      \ 
      \psi(\omega)
      \cap
      O
      \neq
      \emptyset
    \right\}
    \ 
    \in
    \ 
    \left\{ \Omega,\emptyset \right\}
    \subset
    \Sigma_S
    \qquad
    \text{for all open subsets}
    \ 
    O
    \subset
  \R^N_{\ge 0}
  \times
  \R
  \times
  \R^{N}
    \,.
  \end{gather*}
  Furthermore the objective function is a Caratheodory function.
  Thus, by Theorem there exists the argmax correspondence  of Problem~\ref{dual} and a measurable selector that selects
  $(\rho^\dagger,0_{N-n},\lambda_0^\dagger,\lambda^\dagger)$ 
  of Assumption.
  With this, we can define the optimal weights processes
  \begin{gather*}
    w^\dagger(x)
    \ 
    :=
    \ 
    w
    \left( 
    x,\rho^\dagger,0_{N-n},\lambda_0^\dagger,\lambda^\dagger
    \right)
    \qquad
    \text{indexed over}\ 
    x\in\mathcal{X}\,.
  \end{gather*}
  Due to the definition of the (general) weights processes and the measurability of the argmax selector, the random variables $w^\dagger(x)$ are measurable for all $x\in\mathcal{X}$.
  To end the measurability discussion, note that $w^\dagger(X)$ is a random variable.


\begin{lemma}
  \label{weights_l_inf}
  Let Assumption~\ref{asu:existence_sol}, 
  Assumption~\ref{asu:basis},
  and Assumption~\ref{asu:objective_f} hold true.
  Then it holds 
  \begin{gather*}
    w_i^\dagger(X)
  \ 
  \in
  \ 
  L^\infty(\P)
  \qquad
  \text{for all}
  \ 
  i\in \left\{ 1,\ldots,N \right\}
  \ 
  \text{
    and
  for all 
  }\ 
  N\ge \underline{N}
  \,.
  \end{gather*}
\end{lemma}
\begin{proof}
  Let $N\ge \underline{N}$.
  By Assumption~\ref{asu:basis} it holds 
  $
    \text{for all}\ 
  $ 
  \begin{gather}
    \label{3967}
    \left| 
     \rho_i 
     +
      \lambda_0
      +
      \inner{B(x)}{\lambda}
    \right|
    \ 
    \lesssim
    \ 
    \norm{(\rho,\lambda_0,\lambda)}_2
    \,.
  \end{gather}
  $
    \text{for all}
    \ 
    x\in\mathcal{X}
    \
    \text{and for all}\ 
    (\rho,\lambda,\lambda_0)
    \in
  \R^N_{\ge 0}
  \times
  \R
  \times
  \R^{N}
  $.
  By Assumption~\ref{asu:existence_sol}, 
  there exists a (compact) parameter space
  $\Theta_N$ around the origin, with $\mathrm{diam}\,  \Theta_N<\infty$, 
  such that for all data sets $D_N$ it holds  $(\rho^\dagger,\lambda^\dagger,\lambda_0^\dagger)\in\Theta_N$.
  By Assumption~\ref{asu:objective_f}, $(\varphi^{'})^{-1}$ is non-decreasing and continuous. Thus
    by \eqref{3967} it holds
  \begin{align*}
    \left| 
    w_i^\dagger(x)
    \right|
    \ 
    \lesssim
    \ 
    \left| 
    (\varphi^{'})^{-1}
    \left( 
      \,
      -
    \norm{(\rho^\dagger,\lambda_0^\dagger,\lambda^\dagger)}_2
    \right)
    \right|
    \ 
    +
    \ 
    \left| 
    (\varphi^{'})^{-1}
    \left( 
      \,
    \norm{(\rho^\dagger,\lambda_0^\dagger,\lambda^\dagger)}_2
    \right)
    \right|
    \ 
    \ 
    <
    \ 
    \infty
  \end{align*}
  for all $x\in\mathcal{X}$
  and all $i\in \left\{ 1,\ldots,n \right\}$
  .
\end{proof}
\begin{lemma}
  \label{w.Z=0}
  Let Assumption~\ref{aa:assumption:2} and Assumption~\ref{aa:assumption:3} hold true.
  Furthermore, 
  let
  $N\ge\underline{N}$, and
  let
  $Z\in L^1(\P)$
  be a random variable that is independent of $D_N$ 
  with
  $
\E
\left[
  Z
  \,
  |
  \, 
  X
\right]
= 0
  $
  almost surely.
  It holds
  \begin{gather*}
  \E
  \left[
  w_i(X,\rho^\dagger,\lambda^\dagger,\lambda_0^\dagger)
  \cdot Z
  \right]
  =0
  \qquad
  \text{for all}
  \ 
  i\in \left\{ 1,\ldots,n \right\}
  \,.
  \end{gather*}
\end{lemma}
\begin{proof}
  Let
  $N\ge\underline{N}$.
  We write
  \begin{gather*}
  w_i(X,\rho^\dagger,\lambda^\dagger,\lambda_0^\dagger)
  \ 
  =
  \ 
  w^\dagger(X)
  \end{gather*}
  and ignore the index $i$.
  By Lemma~\ref{weights_l_inf} and 
  $Z\in L^1(\P)$
  it holds
  \begin{gather}
    \label{9876}
    \norm{
  w^\dagger(X)\cdot Z
    }_{L^1(\P)}
    \ 
  \le
    \ 
  \norm{w^\dagger(X)}_{L^\infty(\P)}
  \norm{Z}_{L^1(\P)}
  \ 
  <
  \ 
  \infty
  \,.
  \end{gather}
  By 
  \eqref{9876},
  $Z\perp D_N$
  and
  $
\E
\left[
  Z
  \,
  |
  \, 
  X
\right]
= 0
  $
  almost surely
  it holds 
  \begin{align*}
    \E
  \left[
  w^\dagger(X)
  \cdot
  Z
  \,
  |
  \,
  D_N,X
  \right]
  &
  \ 
  =
  \ 
  w^\dagger(X)
  \cdot
  \E
  \left[
  Z
  \,
  |
  \,
  D_N,X
  \right]
  \\
  &
  \ 
  =
  \ 
  w^\dagger(X)
  \cdot
  \E
  \left[
  Z
  \,
  |
  \,
  X
  \right]
  \
  =
  \ 
  0
  \qquad
  \text{almost surely.}
  \end{align*}
  Thus
  \begin{gather*}
    \E
    \left[
  w^\dagger(X)
  \cdot
  Z
  \,
    \right]
    \ 
    =
    \ 
    \E
    \left[
 \E
  \left[
  w^\dagger(X)
  \cdot
  Z
  \,
  |
  \,
  D_N,X
  \right]
    \right]
    \ 
    =
    \ 
    0
    \,.
     \end{gather*}
\end{proof}




\chapter{Constructing the Weights Process}
  In the formulation of Theorem~\ref{dual} we encounter "If (...) there exists the optimal solution $(\rho^\dagger,\lambda_0^\dagger,\lambda)$ ... " .
To be able to study asymptotic properties of the weights, we 
shall assume that Problem~\ref{dual} is feasible,
construct a measurable dual solution, and plug it in $(\varphi^{'})^{-1}$.
Before we formulate concrete assumptions, we provide tools from functional analysis
to obtain measurability. Afterwards, we shall tailor the assumptions to the capability of this tools.
\section{Argmax Measurability Theorem}
  We follow \cite{Aliprantis2007}.
A \textbf{correspondence} $\psi$ from a set $S_1$ to a set $S_2$ assigns to each $s_1\in S_1$ a subset $\psi(s_1)\subset S_2$.
To clarify that we map $s_1$ to a set, we use the double arrow, that is,
$
  \psi
  \colon
  S_1
  \twoheadrightarrow
  S_2
$.
  Let 
  $(\mathcal{Z},\Sigma_{\mathcal{Z}})$ be a measurable space and $\mathcal{S}$  a topological space.
  We say, that a correspondence 
  $
  \psi
  \colon
  \mathcal{Z}
  \twoheadrightarrow
  \mathcal{S}
  $
  is 
  \textbf{
  weakly measurable
  },
  if
  \begin{gather*}
    \left\{ 
      z\in \mathcal{Z}
      \ 
      |
      \ 
      \psi(z)
      \cap
      O
      \neq
      \emptyset
    \right\}
    \in
    \Sigma_{\mathcal{Z}}
    \qquad
    \text{for all open subsets}
    \ 
    O\subset \mathcal{S}
    \,.
  \end{gather*}
  A \textbf{selector} from a correspondence $\psi\colon \mathcal{Z}\twoheadrightarrow \mathcal{S}$ is a function $s\colon \mathcal{Z}\to \mathcal{S}$ that satisfies 
  \begin{align*}
s(z)\in\psi(z)
\qquad
\text{for all}\ 
z\in\mathcal{Z}
\,.
  \end{align*}
  

\begin{definition}
  Let 
  $(\mathcal{Z},\Sigma_{\mathcal{Z}})$ be a measurable space, and let $\mathcal{S}_1$ and $\mathcal{S}_2$  be topological space.
  A function 
  $f\colon \mathcal{Z}\times \mathcal{S}_1 \to \mathcal{S}_2$
  is a \textbf{Caratheodory function} if
  \begin{align*}
    f(\cdot,s_1)
    &
    \colon
    \mathcal{Z}\to \mathcal{S}_2
    \qquad
    \text{is}\ 
    (\Sigma_{\mathcal{Z}},\mathcal{B}(\mathcal{S}_2))-measurable
    \ 
    \text{for all}
    \ 
    s_1\in \mathcal{S}_1
    \,,
    \intertext{and}
    f(z,\cdot)
    &
    \colon
    \mathcal{Z}\to \mathcal{S}_2
    \qquad
    \text{is continuous for all}\ 
    z\in \mathcal{Z}
    \,.
  \end{align*}
\end{definition}
\begin{theorem}
  \label{th:argmax}
  Let $\mathcal{S}$ be a separable metrizable space and
  $
  (\mathcal{Z},\Sigma_{\mathcal{Z}})
  $
  a measurable space.
  Let $\psi\colon \mathcal{Z} \twoheadrightarrow \mathcal{S}$ be a weakly measurable correspondence with non-empty compact values, and suppose
  $f\colon \mathcal{Z}\times \mathcal{S} \to \R$
  is a Caratheodory function. Define the value function 
  $m\colon \mathcal{Z}\to \R$ by
  \begin{gather*}
    m(z):=\max_{s\in\psi(z)}f(z,s)
    \,,
  \end{gather*}
  and the correspondence 
  $\mu\colon \mathcal{Z}\twoheadrightarrow \mathcal{S}$ of maximizers by
  \begin{gather*}
    \mu(z):= \left\{ 
      s\in \psi(z)
      |
      f(z,s)=m(z)
    \right\}
    \,.
  \end{gather*}
  Then the value function $m$ is measurable, 
  the argmax correspondence $\mu$ has non-empty and compact values,
  is measurable and admits a measurable selector.
\end{theorem}
\begin{proof}
  \cite[Theorem~18.19]{Aliprantis2007}
\end{proof}
\begin{takeaways}
  Solving an optimization problem that has a Caratheodory objective function on a weakly-measurable, non-empty and compact search space, allows for measurable optimal solutions.
\end{takeaways}

\section{Measurable Dual Solution}
  Next, we formulate the feasibility assumption. The assumption is (asymptotically) justified by Theorem~\ref{th:cons_dual}.
Note that we assume compactness to be able to apply Theorem~\ref{th:argmax}.
\begin{assumption}
  \label{asu:feas_dual_sol}
  For all $N\in\mathbb{N}$ there exists a non-empty, compact, and deterministic 
  parameter space 
  $
  \Theta_N
  \subset
  \R^{N}_{\ge 0}
  \times
  \R
  \times
  \R^N
  $
  such that the optimal solution 
  $
  \left( \rho^\dagger,\lambda_0^\dagger,\lambda^\dagger \right)
  $
  of Problem~\ref{dual}
  are contained in $\Theta_N$.
\end{assumption}
Based on this assumption it is easy to derive measurability for the dual solutions 
  $
  \left( \rho^\dagger,\lambda_0^\dagger,\lambda^\dagger \right)
  $.
  To this end, we take a closer look at the objective function.
  \begin{definition}
    \label{def:rand_obj_f}
We define the (random) objective function of
Problem~\ref{dual} by
  \begin{align*}
    &
  G
  \ 
  \colon
  \ 
  \left(
  \Omega,
\sigma
(D_N)
  \right)
  \times
  \left(
  \R^N_{\ge 0}
  \times
  \R
  \times
  \R^{N}
  \right)
  \ 
  \to
  \ 
  \overline{\R}
  \intertext{with}
    &
  G(\omega,(\rho,\lambda_0,\lambda))
  \ 
  =
  \ 
  \infty
  \qquad 
  \text{if}\quad 
  \rho_i
  \neq
  \left[ 
  \varphi^{-1}
  (0)
  -
  \left( 
  \lambda_0 + \inner{B(X_i)}{\lambda}
  \right)
  \right]^+
  \
  \text{for some}\ i>n
  \,,
  \\
  \intertext{and else}\qquad
  &
  G(\omega,(\rho,\lambda_0,\lambda))
  \\
  &
  \ 
  =
  \ 
  \frac{1}{N}
\sum_{i=1} 
  ^N
  \Big[
  T_i(\omega)
  \cdot
  \varphi^*
  \!
  \left( 
    \rho_i
    +
\lambda_0
+
\inner
{B(X_i)(\omega)}
{
\lambda
}
  \right)
  \ 
  -
  \ 
\lambda_0
-
\inner
{B(X_i)(\omega)}
{
\lambda
}
\Big]
\\
&
  \qquad 
+
\ 
\inner
{\delta(\omega)}
{
  |\lambda|
}
\,.
  \end{align*}
  \end{definition}
  \begin{lemma}
    \label{lem:caratheo_G}
    The function $G$ of Definition~\ref{def:rand_obj_f}
    is Caratheodory.
  \end{lemma}
  \begin{proof}
    This follows from Lemma~\ref{1165}
    (continuity of $\varphi^*$) and the measurability 
  of all random variables included.
  \end{proof}
  In the proof of the next lemma we gather the arguments and apply Theorem~\ref{th:argmax}.
\begin{lemma}
  \label{lem:meas_dual_sol}
  Let Assumption~\ref{asu:feas_dual_sol} hold true.
  Then,
  for all $N\in\mathbb{N}$ the dual solution
  \begin{align*}
  \left( \rho^\dagger,\lambda_0^\dagger,\lambda^\dagger \right)
    \ 
    \colon
   \  
    \Omega
    \ 
    \to
    \ 
  \R^N_{\ge 0}
  \times
  \R
  \times
  \R^{N}
  \end{align*}
  to
  Problem~\ref{dual} 
  is
  \begin{align*}
  \left(
    \sigma \left( D_N \right),\mathcal{B}
  \left(
  \R^N_{\ge 0}
  \times
  \R
  \times
  \R^{N}
  \right)
  \right)
  -\text{measurable}
  \,.
  \end{align*}
\end{lemma}
\begin{proof}
  Since $\Theta_N$ is deterministic (by Assumption~\ref{asu:feas_dual_sol})
  we can define the (constant) correspondence
  $\omega \mapsto \Theta_N$.
  Clearly, this is weakly-measurable, non-empty and compact.
  Next, we consider the (random) objective function of (the maximize version of) Problem~\ref{dual}, that is, $-G$ (see Definition~\ref{def:rand_obj_f}).
  By Lemme~\ref{lem:caratheo_G}, $-G$  is a Caratheodory function.
  Since $-G$ is also strictly concave, it has a unique argmax in $\Theta_N$.
  By Assumption~\ref{asu:feas_dual_sol} this is 
  $
  \left( \rho^\dagger,\lambda_0^\dagger,\lambda^\dagger \right)
  $.
  By Theorem~\ref{th:argmax} this is
  \begin{align*}
  \left(
    \sigma
    (D_N)
    ,\mathcal{B}
  \left(
  \R^N_{\ge 0}
  \times
  \R
  \times
  \R^{N}
  \right)
  \right)
  -\text{measurable}
  \,.
  \end{align*}
\end{proof}

\begin{takeaways}
  With suitable assumptions on the feasibility of Problem~\ref{dual}, we can construct measurable dual solutions.
  An important tool to obtain measurability is the argmax measurability theorem (Theorem~\ref{th:argmax}).
\end{takeaways}

Before we can define the weights process (based on the dual solution), we specify the basis functions.
\section{Basis Functions}
  Let $
\left(
\mathcal{P}_N
\right)
$
denote a sequence of countable, $\mathcal{B}$-measurable partitions 
\index{$\mathcal{P}_N$, partition of $\R^d$}
\begin{align*}
\mathcal{P}_N= \left\{
  A_{N,1},
  A_{N,2},
  \ldots
\right\}
\subset \mathcal{B}(\R^d)
\end{align*}
of $\R^d$, that is, 
\begin{align*}
  A_{N,i}\cap A_{N,j}=\emptyset
  \qquad
  \text{if}\ i\neq j
  \qquad
  \text{and}
  \qquad 
  \bigcap_{i\in\mathbb{N}}A_{N,i}
  \ 
  =
  \ 
  \R^d
  \,.
\end{align*}
\index{$A_N$, cell of the partition $\mathcal{P}_N$}
We define
$ A_N(x) $ to be the cell of $ \mathcal{P}_N $ containing $x$, that is,
\begin{align*}
  A_N
  \colon
  \R^d 
  \ 
  \twoheadrightarrow 
  \ 
  \R^d  
  \,,\qquad
  x
  \ 
  \mapsto
  \ 
  A_N(x)
  \,,
\end{align*}
where $A_N(x)$ is the only cell containing $x$. 

\begin{lemma}
  \label{lem:basis_equiv_r}
  The relation
  \begin{align*}
    x\sim y
    \qquad
    :\Leftrightarrow
    \qquad
    x\in A_N(y)
  \end{align*}
  is an equivalence relation.
\end{lemma}
\begin{proof}
  The proof is simple. We omit it.
\end{proof}
Before we define the basis vector, we assume 
uniform partition width such that
\begin{align*}
  \lambda(A_N)
  \ 
  =:
  \ 
  h_N^d
  \ 
  \to
  \ 
  0 
  \qquad
  \text{for}\ N\to\infty
  \,.
\end{align*}
Next, we define the (empirical) basis functions vector
\begin{align}
  \label{def:basis}
  B\colon
  \R^d\times \R^{d\cdot N}
  \ 
  \to
  \ 
  \R
  \,,
  \qquad
  (x,(x_1,\ldots,x_N))
  \ 
  \mapsto
  \ 
  \frac
  {
    \left[
    \mathbf{1}
    _{
      A_N(x)
    }
    (x_k)
    \right]
    _{k\in \left\{
        1,\ldots,N
    \right\}}
  }
  {
    \sum_{j=1}^N
    \mathbf{1}
    _{
      A_N(x)
    }
    (x_j)
    }
  \,,
\end{align}
where we keep to the convention $"0/0=0"$.
We shall extend $B$ to depend on the random vectors
$X,X_1,\ldots,X_N$.
The next lemma studies the measurability of the extensions.
\begin{lemma}
  \label{lem:basis_meas}
  \begin{enumerate}[label=(\roman*)]
\item
  $B(\cdot,(X_1,\ldots,X_N))(\omega)$ is 
  $\left(
    \mathcal{B}(\R^d),\mathcal{B}(\R^N)
  \right)$-measurable
  and
  constant on each cell 
  $A_N\in\mathcal{P}_N$
  for all $\omega\in\Omega$. 
\item
  $B(X,(X_1,\ldots,X_N))$ is $\left(
    \sigma(X,D_N),\mathcal{B}(\R^N)
  \right)$-measurable. 
  \end{enumerate}
\end{lemma}
%
\begin{proof}
Consider,
for $k\in \left\{
  1,\ldots,N
\right\}$
and $\omega\in\Omega$,
the indicator function
\index{$\mathbf{1_A}$, indicator function of the set $A$}
\begin{align}
  \label{33342}
  \mathbf{1}
  _
  {A_N(X_k(\omega))}
  \ 
  \colon
  \ 
  \R^d
  \ 
  \to
  \ 
  \left\{
    0,1
  \right\}
  \,.
\end{align}
Since 
$
  {A_N(X_k(\omega))}
  \in\mathcal{B}(\R^d)
$,
this is a 
  $\left(
    \mathcal{B}(\R^d),\mathcal{B}(\R)
  \right)$-measurable
  function.
  From the definition of $B$ \eqref{def:basis} it follows the first part of \textit{(i)}.
  Since the indicator function in \eqref{33342} is 1 if $
  x\in
  {A_N(X_k(\omega))}
  $
  and 0 else, it is also constant on each cell
  $A_N\in\mathcal{P}_N$.
  It follows \textit{(i)}.
  To prove \textit{(ii)}, note that
\begin{align*}
  \mathbf{1}
  _
  {A_N(X_k(\omega))}(X(\omega))
  \ 
  =
  \ 
  \mathbf{1}
  \bigcup_{i\in\mathbb{N}}
  \left\{
    X,X_k \in A_{N,i}
  \right\}
  (\omega)
  \qquad
  \text{for all}\ 
  \omega\in\Omega
  \,,
\end{align*}
and
$
  \bigcup_{i\in\mathbb{N}}
  \left\{
    X,X_k \in A_{N,i}
  \right\}
  \in\sigma(X,D_N)
  $.
\end{proof}
%
Now we gather some useful properties of the (empirical) basis vector.
\begin{lemma}
  \label{lem:basis_sum}
  Let $(x,x_1,\ldots,x_N)\in\R^{d(N+1)}$.
  \begin{enumerate}[label=(\roman*)]
    \item
      $
      \sum_{k=1}^{N} 
      B_k(x,x_1,\ldots,x_N)
      \in
      \left\{ 0,1 \right\}
      $. 
      In particular,
      $
        x_1,\ldots,x_N\notin A_N(x)
      $
      is equivalent to
      $
      \sum_{k=1}^{N} 
      B_k(x,x_1,\ldots,x_N)
      =0
      $
    \item
      $
      \sum_{k=1}^{N} 
      B_k(x_i,x_1,\ldots,x_N)
      \ 
      =
      \ 
      1
      \qquad
      \text{for all}\ 
      i\in \left\{ 1,\ldots,N \right\}
      $.
      \item
        $
        \norm
        {
      B(x,x_1,\ldots,x_N)
        }_2
        \ 
        \le 1
        \ 
        $
      \item
        $
        B_k(x_i,x_1,\ldots,x_N)
        \ 
        =
        \ 
        B_i(x_k,x_1,\ldots,x_N)
        \qquad
        \text{for all}\ 
        i,k\in \left\{ 1,\ldots,N \right\}
        $
  \end{enumerate}
\end{lemma}
\begin{proof}
  Let $(x,x_1,\ldots,x_N)\in\R^{d(N+1)}$.
  We prove \textit{(i)}.
  Then \textit{(ii)} is a direct consequence of \textit{(i)}.
  If 
      $
        x_1,\ldots,x_N\notin A_N(x)
      $,
  then
  \begin{align*}
    B_k(
        x,x_1,\ldots,x_N
    )
    \ 
    =
    \ 
\frac
  {
    \mathbf{1}
    _{
      A_N(x)
    }
    (x_k)
  }
  {
    \sum_{j=1}^N
    \mathbf{1}
    _{
      A_N(x)
    }
    (x_j)
    }
    \ 
    =
    \ 
    0
    \qquad
    \text{for all}\ 
    k\in \left\{ 1,\ldots,N \right\}
    \,.
  \end{align*}
  On the other hand, if the sum is 0 it holds
  \begin{align*}
    \mathbf{1}
    _{
      A_N(x)
    }
    (x_k)
    \ 
  =
  \ 
  0
    \qquad
    \text{for all}\ 
    k\in \left\{ 1,\ldots,N \right\}
    \,.
  \end{align*}
  It follows the desired equivalence.
  If 
  \begin{align*}
    \mathbf{1}
    _{
      A_N(x)
    }
    (x_k)
    \ 
  =
  \ 
  1
    \qquad
    \text{for some}\ 
    k\in \left\{ 1,\ldots,N \right\}
    \,,
  \end{align*}
  then
  $
    \sum_{j=1}^N
    \mathbf{1}
    _{
      A_N(x)
    }
    (x_j)
    \ge 1
  $
  and thus "$0/0$" doesn't occure. 
  It follows
  \begin{align*}
      \sum_{k=1}^{N} 
      B_k(x,x_1,\ldots,x_N)
      \ 
      =
      \ 
\frac
  {
      \sum_{k=1}^{N} 
    \mathbf{1}
    _{
      A_N(x)
    }
    (x_k)
  }
  {
    \sum_{j=1}^N
    \mathbf{1}
    _{
      A_N(x)
    }
    (x_j)
    }
    \ 
    =
    \ 
    1
    \,.
  \end{align*}
  To prove \textit{(iii)}, note that by \textit{(i)}
  \begin{align*}
        \norm
        {
      B(x,x_1,\ldots,x_N)
        }_2^2
        \ 
        =
        \ 
      \sum_{k=1}^{N} 
      B_k(x,x_1,\ldots,x_N)^2
        \ 
      \le
        \ 
      \sum_{k=1}^{N} 
      B_k(x,x_1,\ldots,x_N)
        \ 
      \le
        \ 
      1
      \,.
  \end{align*}
  To prove \textit{(iv)}, note that by Lemma~\ref{lem:basis_equiv_r}
  and by symmetry and transitivity of the equivalence relation
  $x\in A_N(y)$
  it holds
  \begin{align*}
      B_k(x_i,x_1,\ldots,x_N)
      &
    \ 
      =
    \ 
 \frac
  {
    \mathbf{1}
    \left\{ 
      x_k
      \in
      A_N(x_i)
    \right\}
  }
  {
    \sum_{j=1}^N
    \mathbf{1}
    \left\{ 
      x_j
      ,
      x_k
      \in
      A_N(x_i)
    \right\}
    }
    \ 
    =
    \ 
 \frac
  {
    \mathbf{1}
    \left\{ 
      x_i
      \in
      A_N(x_k)
    \right\}
  }
  {
    \sum_{j=1}^N
    \mathbf{1}
    \left\{ 
      x_j
      \in
      A_N(x_k)
    \right\}
    }
    \\
    &
    \ 
    =
    \ 
      B_i(x_k,x_1,\ldots,x_N)
      \,.
  \end{align*}
\end{proof}
%
Now we show that the basis vector plays well with uniformly continuous functions. The result seems simple, yet the consequence are great. It allows us later on to specify an oracle parameter instead of assuming its existence (see \cite[Assumption~1.6]{Wang2019}). This greatly clarifies the proofs.
\begin{lemma}
  \label{lem:basis_approx_f}
  Let $(x,x_1,\ldots,x_N)\in\R^{d(N+1)}$.
  For all uniformly continuous functions $f\colon \R^d\to \R$ it holds
 \begin{align*}
   \begin{split}
   &
   \left|
  \sum_{k=1}^{N}
    B_k(x_i,x_1,\ldots,x_N)\cdot 
    f(x_k)
    -
    f(x_i)
   \right|
   \ 
   \le
   \ 
   \omega
   \left(
    f,h_N^d
   \right)
   \qquad
   \text{for all}\ 
   i\in \left\{ 1,\ldots,N \right\}
   \,,
   \end{split}
 \end{align*}
 where $\omega(f,\cdot)$ is the uniform modulus of continuity of $f$. 
\end{lemma}
\begin{proof}
  It follows from Lemma~\ref{lem:basis_sum}.\textit{(ii)}
  \begin{align*}
   \begin{split}
   &
   \left|
  \sum_{k=1}^{N}
    B_k(x_i,x_1,\ldots,x_N)\cdot 
    f(x_k)
    -
    f(x_i)
   \right|
   \\
   &
   \ 
   \le
   \ 
   \left|
  \sum_{k=1}^{N}
    B_k(x_i,x_1,\ldots,x_N)
    \left(
    f(x_k)
    -
    f(x_i)
    \right)
   \right|
   \\
   &
   \ 
   \le
   \ 
  \sum_{k=1}^{N}
    B_k(x_i,x_1,\ldots,x_N)
    \cdot
    \mathbf{1}\left\{
      x_k\in A_N(x_i)
    \right\}
    \left|
    f(x_k)
    -
    f(x_i)
    \right|
   \\
   &
   \ 
   \le
   \ 
   \omega
   \left(
    f,h_N^d
   \right)
   \,.
   \end{split}
 \end{align*}
\end{proof}
Next, we apply  Lemma~\ref{lem:basis_approx_f}.
On a high-level, the next lemma says that the basis functions estimate both treatment \textit{(i)} and outcome model \textit{(ii)} well.
This feature is connected to double robustness, discussed in \cite{Zhao2017a}.

In the following,
let $F_{Y(1)}(\cdot|x)$ denote the distribution function of $Y(1)$ conditional on $X=x\in\mathcal{X}$ (see \eqref{3228}).
\newpage
\begin{lemma}
  \label{lem:basis_2}
  Let $(x,x_1,\ldots,x_N)\in\mathcal{X}^{N+1}$.
  It holds
  for $N\to\infty$
  \begin{enumerate}[label=(\roman*)]
      \item
        If Assumption~\ref{asu:treatment_asign_str_ing} and Assumption~\ref{asu:x_finite} hold true
      \begin{align*}
        \frac
        {1}
        {N}
        \sum_{i,k=1}^{N}
            \left|
        B_k(x_i,x_1,\ldots,x_N)
        \cdot
        \varphi^{'}
            \left(
              \frac
              {1}
              {\pi(x_k)}
            \right)
            \ 
            -
            \ 
            \varphi^{'}
            \left(
              \frac
              {1}
              {\pi(x_i)}
            \right)
            \right|
            \ 
            \to
            \ 
            0
            \,,
          \end{align*}
\item
  If if holds
  $
  \sqrt{N}
  \sup_{z\in\R}
  \omega
  \left( 
    F_{Y(1)}(z|\cdot)
    ,h_N^d
  \right)
  \to
  0
  \qquad
  \text{for}\ 
  N\to \infty
  $, then
      \begin{align*}
        \sqrt{N}
        \sup_{z\in\R}
        \max_{i\in \left\{ 1,\ldots,N \right\}}
        \sum_{k=1}^{N}
            \left|
        B_k(x_i,x_1,\ldots,x_N)
        \cdot
        F_{Y(1)}(z|x_k)
            \ 
            -
            \ 
        F_{Y(1)}(z|x_i)
            \right|
            \ 
            \to
            \ 
            0
            \,.
      \end{align*}
\end{enumerate}
\end{lemma}
\begin{proof}
  By Lemma~\ref{lem:basis_approx_f} (good approximation of uniformly continuous functions) and
  Lemma~\ref{lem:ips_unif_cont} 
  (uniform continuity of $\varphi^{'}\circ (x\mapsto 1/x)\circ \pi$),
  it holds
      \begin{align*}
        \frac
        {1}
        {N}
        \sum_{i,k=1}^{N}
            \left|
        B_k(x,x_1,\ldots,x_N)
        \cdot
            \varphi^{'}
            \left(
              \frac
              {1}
              {\pi(x_k)}
            \right)
            \ 
            -
            \ 
            \varphi^{'}
            \left(
              \frac
              {1}
              {\pi(x_i)}
            \right)
            \right|
            \ 
            \le
            \ 
   \omega
   \left(
     \varphi^{'},h_N^d
   \right)
            \ 
            \to
            \ 
            0
          \end{align*}
          for $N\to\infty$.
          Likewise
\begin{align*}
  &
        \sqrt{N}
        \sup_{z\in\R}
        \max_{i\in \left\{ 1,\ldots,N \right\}}
        \sum_{k=1}^{N}
            \left|
        B_k(x_i,x_1,\ldots,x_N)
        \cdot
        F_{Y(1)}(z|x_k)
            \ 
            -
            \ 
        F_{Y(1)}(z|x_i)
          \right|
             \\
            &
            \ 
            \le
            \ 
            \sqrt{N}
            \sup_{z\in\R}
            \omega
            \left(
        F_{Y(1)}(z|\cdot)
        ,
        h_N^d
            \right)
            \ 
            \to
            \ 
            0
            \qquad
            \text{for}
            \ 
            N\to\infty
            \,.
\end{align*}
        \end{proof}
        \begin{remark}
I want to comment on the assumption
\begin{gather*}
  \sqrt{N}
  \sup_{z\in\R}
  \omega
  \left( 
    F_{Y(1)}(z|\cdot)
    ,h_N^d
  \right)
  \to
  0
  \qquad
  \text{for}\ 
  N\to \infty
  \,.
\end{gather*}
I decided to keep this more general (and abstract) assumption, although
there are many (more concrete, yet stronger) sufficient assumptions on the regularity of
$
    F_{Y(1)}(z|\cdot)
$
and the convergence speed of $h_N$.
If for example 
$
    F_{Y(1)}(z|\cdot)
$
is $\alpha$-Hölder continuous with $\alpha\in(0,1]$ for all $z\in\R$, it suffices $\sqrt{N}h_N^{\alpha\cdot d}\to0$.

        \end{remark}
\begin{takeaways}
  Basis functions of non-parametric
  partitioning estimates are new to the framework of balancing weights.
  They play well with uniformly continuous functions and promise to simplify the analysis. 
  This choice of basis functions waits to be tested in practice.
\end{takeaways}

\section{Weights Process}
  Based on Theorem~\ref{dual_solution_th}
and Assumption~\ref{asu:feas_dual_sol},
we want to use the dual 
solution 
$
\left( \rho^\dagger,\lambda_0^\dagger,\lambda^\dagger \right)
$
to construct weights.
To this end, we define the (empirical) weights function
\begin{align*}
 w\ \colon\
 &
 \left( 
  \R^d\times \R^{d\cdot N}
 \right)
  \times
  \left( 
\R^N_{\ge 0}\times \R\times \R^N
  \right)
  \to
  \R^N
  \\
 &
  \left( 
  (x,x_1,\ldots,x_N),(\rho,\lambda_0,\lambda)
  \right)
  \ 
  \mapsto
  \ 
  \left[ 
  (\varphi^{'})^{-1}
  \left( 
    \rho_i
    +
    \lambda_0
    +
    \inner
    {B(x,x_1,\ldots,x_N)}
    {\lambda}
  \right)
\right]_{i\in \left\{ 1,\ldots,N \right\}}
\,.
\end{align*}
\begin{definition}
  Let 
  $
\left( \rho^\dagger,\lambda_0^\dagger,\lambda^\dagger \right)
  $
  be the dual solution of Lemma~\ref{lem:meas_dual_sol}.
  We define the weights process 
  $\left\{ w^\dagger(x) | x\in\R^d\right\}$
  by
  \begin{align*}
    w^\dagger(x) 
    \ 
    :=
    \ 
    w
    \left( 
    \left( 
    x,X_1,\ldots,X_N,
    \right)
    ,
\left( \rho^\dagger,\lambda_0^\dagger,\lambda^\dagger \right)
    \right)
    \qquad
    \text{for all}\ 
    x\in\R^d
    \,.
  \end{align*}
\end{definition}
\begin{lemma}
  \label{lem:weights:meas}
  \quad
  \begin{enumerate}[label=(\roman*)]
\item
  $w^\dagger(\cdot)(\omega)$ is 
  $\left(
    \mathcal{B}(\R^d),\mathcal{B}(\R^N)
  \right)$-measurable
  and
  constant on each cell 
  $A_N\in\mathcal{P}_N$
  for all $\omega\in\Omega$. 
\item
  $w^\dagger(X)$ is $\left(
    \sigma(X,D_N),\mathcal{B}(\R^N)
  \right)$-measurable. 
  \end{enumerate}
\end{lemma}
\begin{proof}
  This is a direct consequence of 
  Lemme~\ref{lem:basis_meas} (measurability of the basis functions), Lemma~\ref{lem:meas_dual_sol} (measurability of the dual solution),
  and 
  Lemma~\ref{lem:obj_f}.\textit{(iii)} (continuity of $(\varphi^{'})^{-1}$).
\end{proof}
  Let $\lesssim$ denote the lesser-or-equal-up-to-a-uniform-constant order, that is, we choose a uniform constant $C>1$ that is independent of $N$ and always large enough, such that $a\lesssim b$ is equivalent to $a\le C\cdot b$.
  \index{$\lesssim$, 
lesser-or-equal-up-to-a-uniform-constant order
  }
\begin{lemma}
  \label{weights_l_inf}
  It holds
  $w_i^\dagger(X)\in L^\infty(\P)$
  for all $i\in \left\{ 1,\ldots,N \right\}$.
\end{lemma}
\begin{proof}
  By Lemma~\ref{lem:basis_sum}.\textit{(iii)} ($B$ has uniformly bounded norm),
  it holds
  \begin{align*}
  \left| 
    \rho_i^\dagger
    +
    \lambda_0^\dagger
    +
    \inner
    {B(x,x_1,\ldots,x_N)}
    {\lambda^\dagger}
  \right|
  \ 
  \lesssim
  \ 
  \norm{
\left( \rho^\dagger,\lambda_0^\dagger,\lambda^\dagger \right)
  }_2
  \qquad
  \text{for all}\ 
  i \in \left\{ 1,\ldots,N \right\}
  \,.
  \end{align*}
  Since
  $
\left( \rho^\dagger,\lambda_0^\dagger,\lambda^\dagger \right)
  $ is contained in the deterministic and compact parameter space $\Theta_N$,
  it holds
  \begin{align*}
  \norm{
\left( \rho^\dagger,\lambda_0^\dagger,\lambda^\dagger \right)
  }_2
  \in 
  L^{\infty}(\P)
  \,.
  \end{align*}
  By Lemma~\ref{lem:obj_f}.\textit{(iii)} 
  (uniform continuity of 
  $
  (\varphi^{'})^{-1}
  $), it follows 
  $w_i^\dagger(X)\in L^\infty(\P)$
  for all $i\in \left\{ 1,\ldots,N \right\}$.
\end{proof}
Next, we want to simplify the weights process in the spirit of Lemma~\ref{lem:simple_weights}.
In other words, we want to become independent of the index $i$ in $w_i^\dagger$. This will be helpful in the subsequent analysis.
To this end, we define the (empirical) simplified weights function
\begin{align*}
 w_0\ \colon\
 &
 \left( 
  \R^d\times \R^{d\cdot N}
 \right)
  \times
  \left( 
    \R\times \R^N
  \right)
  \to
  [0,\infty)
  \\
 &
  \left( 
  (x,x_1,\ldots,x_N),(\lambda_0,\lambda)
  \right)
  \ 
  \mapsto
  \ 
  \left[ 
  (\varphi^{'})^{-1}
  \left( 
    \lambda_0
    +
    \inner
    {B(x,x_1,\ldots,x_N)}
    {\lambda}
  \right)
\right]^+
\,.
\end{align*}
\begin{definition}
  Let 
  $
\left( \rho^\dagger,\lambda_0^\dagger,\lambda^\dagger \right)
  $
  be the dual solution of Lemma~\ref{lem:meas_dual_sol}.
  We define the simplified weights process 
  $\left\{ w_0^\dagger(x) \,|\, x\in\R^d\right\}$
  by
  \begin{align*}
    w_0^\dagger(x) 
    \ 
    :=
    \ 
    w_0
    \left( 
    \left( 
    x,X_1,\ldots,X_N,
    \right)
    ,
\left( \lambda_0^\dagger,\lambda^\dagger \right)
    \right)
    \qquad
    \text{for all}\ 
    x\in\R^d
    \,.
  \end{align*}
\end{definition}
The next two lemmas extend results from $w^\dagger_i$ to $w^\dagger_0$.
\begin{lemma}
  \label{lem:meas_w_0}
  \quad
  \begin{enumerate}[label=(\roman*)]
\item
  $w_0^\dagger(\cdot)(\omega)$ is 
  $\left(
    \mathcal{B}(\R^d),\mathcal{B}(\R^N)
  \right)$-measurable
  and
  constant on each cell 
  $A_N\in\mathcal{P}_N$
  for all $\omega\in\Omega$. 
\item
  $w_0^\dagger(X)$ is $\left(
    \sigma(X,D_N),\mathcal{B}(\R^N)
  \right)$-measurable. 
  \end{enumerate}
\end{lemma}
\begin{proof}
The proof is as that of Lemma~\ref{lem:weights:meas}.
\end{proof}
\begin{lemma}
  \label{weights_0_l_inf}
  It holds $w_0^\dagger(X)\in L^\infty(\P)$.
\end{lemma}
\begin{proof}
  By Lemma~\ref{weights_l_inf},
  the monotonicity of 
  $
  (\varphi^{'})^{-1}
  $
  and $\rho_i\ge 0$ for $i\le n$,
  it holds
  \begin{align*}
    w_0^\dagger(X) 
    &
    \ 
    \le
    \ 
  \left[ 
  (\varphi^{'})^{-1}
  \left( 
    \lambda_0^\dagger
    +
    \inner
    {B(X)}
    {\lambda^\dagger}
  \right)
\right]^+
\\
&
\ 
\le
\ 
  \left[ 
  (\varphi^{'})^{-1}
  \left( 
    \rho_i^\dagger
    +
    \lambda_0^\dagger
    +
    \inner
    {B(X)}
    {\lambda^\dagger}
  \right)
\right]^+
\ 
\le
\ 
\left| 
    w_i^\dagger(X) 
\right|
\ 
\in
\ 
L^\infty(\P)
  \end{align*}
\end{proof}
Then next lemma shows that $w^\dagger_0$ plays well with random variables that vanish in expectation conditional on $X$.
\begin{lemma}
  \label{w.Z=0}
 Let 
 $Z\in L^1(\P)$
  be a random variable that is independent of $D_N=(T_i,X_i)_{i\in \left\{
    1,\ldots,N
  \right\}}$ 
  with
  $
\E
\left[
  Z
  \,
  |
  \, 
  X
\right]
= 0
  $
  almost surely.
  It holds
  \begin{gather*}
  \E
  \left[
    w_0^\dagger(X)
  \cdot Z
  \right]
  \ 
  =
  \ 
  0
  \,.
  \end{gather*}
\end{lemma}
\begin{proof}
  By Lemma~\ref{weights_0_l_inf} it holds
  \begin{gather}
    \label{9876}
    \norm{
  w_0^\dagger(X)\cdot Z
    }_{L^1(\P)}
    \ 
  \le
    \ 
  \norm{w_0^\dagger(X)}_{L^\infty(\P)}
  \norm{Z}_{L^1(\P)}
  \ 
  <
  \ 
  \infty
  \,.
  \end{gather}
  By 
  \eqref{9876},
  $Z\perp D_N$
  and
  $
\E
\left[
  Z
  \,
  |
  \, 
  X
\right]
= 0
  $
  almost surely
  it holds 
  \begin{align*}
    \E
  \left[
  w_0^\dagger(X)
  \cdot
  Z
  \,
  |
  \,
  D_N,X
  \right]
  &
  \ 
  =
  \ 
  w_0^\dagger(X)
  \cdot
  \E
  \left[
  Z
  \,
  |
  \,
  D_N,X
  \right]
  \\
  &
  \ 
  =
  \ 
  w_0^\dagger(X)
  \cdot
  \E
  \left[
  Z
  \,
  |
  \,
  X
  \right]
  \
  =
  \ 
  0
  \qquad
  \text{almost surely.}
  \end{align*}
  Note, that $w_0^\dagger(X)$ is 
  $
  \left(
  \sigma(D_N,X),\mathcal{B}(\R)
  \right)
  $-measurable by Lemma~\ref{lem:meas_w_0}.\textit{(ii)}.  
  Thus
  \begin{gather*}
    \E
    \left[
  w_0^\dagger(X)
  \cdot
  Z
  \,
    \right]
    \ 
    =
    \ 
    \E
    \left[
 \E
  \left[
  w_0^\dagger(X)
  \cdot
  Z
  \,
  |
  \,
  D_N,X
  \right]
    \right]
    \ 
    =
    \ 
    0
    \,.
     \end{gather*}
\end{proof}

We finish the section with the emphasis that $w^\dagger_0$ is (still) connected to Problem~\ref{bw:1:primal}.
\begin{theorem}
  \label{th:weights_constr}
  The simplified weights process satisfies the constraints
  of Problem~\ref{bw:1:primal}, that is,
  \begin{enumerate}[label=(\roman*)]
    \item
      $
      T_i\cdot w_0^\dagger(X_i)
      \ 
      \ge
      \ 
      0
      \qquad
      \text{for all}\ 
      i\in  \left\{ 1,\ldots,N \right\}
      $
    \item
      $
      \frac{1}{N}
      \sum_{i=1}^{N} 
      T_i\cdot w_0^\dagger(X_i)
      \ 
      =
      \ 
      1
      $
    \item
      For all $k\in \left\{ 1,\ldots,N \right\}$
      it holds
      \begin{align*}
      \left| 
      \frac{1}{N}
      \left( 
        \sum_{i=1}^{N} 
      T_i\cdot w_0^\dagger(X_i)
      \cdot
        B_k(X_i,X_1,\ldots,X_N)
        \
        -
        \
        \sum_{i=1}^{N} 
        B_k(X_i,X_1,\ldots,X_N)
      \right)
      \right|
      \ 
      \le
      \ 
      \delta_k
      \end{align*}
  \end{enumerate}
\end{theorem}
\begin{proof}
  This follows from Theorem~\ref{dual_solution_th}
  (dual relationship of optimal solutions),
  Lemma~\ref{lem:simple_weights} (simplification of the solutions),
  and the construction of the simplified weights process.
\end{proof}
To avoid notational overload, from now on we write
\begin{align*}
  B(x)
  \ 
  :=
  \ 
  B(x,X_1,\ldots,X_N)
  \qquad
  \text{for all}\ 
  x\in\R^d
  \,.
\end{align*}
\begin{takeaways}
  The functional relationship of dual solutions and optimal weights (Theorem~\ref{dual_solution_th}) gives us an idea how to construct weights.
  The ingredients come from the objective function of Problem~\ref{bw:1:primal},
  the basis functions that we balance, and the measurable dual solutions.
  We study and simplify the constructed weights to facilitate the subsequent analysis.
\end{takeaways}


  %In the formulation of Theorem~\ref{dual} we encounter "If (...) there exists the optimal solution $(\rho^\dagger,\lambda_0^\dagger,\lambda)$ ... " .
%To be able to study asymptotic properties of the solutions we have to become independent of this assumption.
%For this we need some tools from functional analysis.
%\section{Argmax Measurability Theorem}
%We follow \cite{Aliprantis2007}
A \textbf{correspondence}$ \psi$ from a set $X$ to a set $Y$ assigns to each $x\in X$ a subset $\psi(x)\subset Y$.
To clarify that we map $x$ to a set, we use the double arrow, that is,
$
  \psi
  \colon
  X
  \twoheadrightarrow
  Y
$.
  Let 
  $(S,\Sigma_S)$ be a measurable space and $X$ a topological space.
  We say, that a correspondence 
  $
  \psi
  \colon
  S
  \twoheadrightarrow
  X
  $
  is 
  \textbf{
  weakly measurable
  },
  if
  \begin{gather*}
    \left\{ 
      s\in S
      \ 
      |
      \ 
      \psi(s)
      \cap
      O
      \neq
      \emptyset
    \right\}
    \in
    \Sigma_S
    \qquad
    \text{for all open subsets}
    \ 
    O\subset X
    \,.
  \end{gather*}

A selector from a correspondence $\psi\colon X\twoheadrightarrow Y$ is a function $s\colon X\to Y$ that satisfies 
$
s(x)\in\psi(x)
$
for all $x\in X$.

\begin{definition}
  Let 
  $(S,\Sigma_S)$ be a measurable space, and let $X$ and $Y$  be topological space.
  A function 
  $f\colon S\times X \to Y$
  is a \textbf{Caratheodory function} if
  \begin{align*}
    f(\cdot,x)
    &
    \colon
    S\to Y
    \qquad
    \text{is}\ 
    (\Sigma_{S},\mathcal{B}(Y))-measurable
    \ 
    \text{for all}
    \ 
    x\in X
    \,,
    \intertext{and}
    f(s,\cdot)
    &
    \colon
    X\to Y
    \qquad
    \text{is continuous for all}\ 
    s\in S
    \,.
  \end{align*}
\end{definition}
\begin{theorem}
  Let $X$ be a separable metrizable space and
  $
  (S,\Sigma_S)
  $
  a measurable space.
  Let $\psi\colon S \twoheadrightarrow X$ be a weakly measurable correspondence with non-empty compact values, and suppose
  $f\colon S\times X \to \R$
  is a Caratheodory function. Define the value function 
  $m\colon S\to \R$ by
  \begin{gather*}
    m(s):=\max_{\psi(s)}f(s,x)
    \,,
  \end{gather*}
  and the correspondence 
  $mu\colon S\twoheadrightarrow X$ of maximizers by
  \begin{gather*}
    \mu(s):= \left\{ 
      x\in \psi(s)
      |
      f(s,x)=m(s)
    \right\}
    \,.
  \end{gather*}
  Then the value function $m$ is measurable, 
  the argmax correspondence $\mu$ has non-empty and compact values,
  is measurable and admits a measurable selector.
\end{theorem}
\begin{proof}
  \cite[Theorem~18.19]{Aliprantis2007}
\end{proof}

Our goal is to find (plausible) assumptions under which we can measurably select optimal solutions of Problem~\ref{dual}.

\begin{assumption}
  There exists $\underline{N}\in\mathbb{N}$ such that 
  for all $N\ge \underline{N}$ 
  there exists a compact and deterministic parameter space
  $
  \Theta_N
  \subset
  \R^N_{\ge 0}
  \times
  \R
  \times
  \R^{N}
  $
  such that for all data sets $D_N$
  a sequence of optimal solution
  $(\rho^\dagger,\lambda_0^\dagger,\lambda^\dagger)$
  satisfying the Lemma
exist and it holds
\begin{gather*}
  (
  \rho^\dagger,
  0_{N-n},
  \lambda_0^\dagger,\lambda^\dagger)
  \in
  \Theta_N
  \,.
\end{gather*}
\end{assumption}

With this assumption we can construct the (constant) correspondence
\begin{align*}
  \psi
  \colon
  (\Omega,\mathcal{A},\P)
  \ 
  \twoheadrightarrow
  \ 
  \R^N_{\ge 0}
  \times
  \R
  \times
  \R^{N}
  \,,
  \qquad
  \omega
  \ 
  \to
  \ 
  \Theta_N
  \,.
\end{align*}
This is weakly measurable, because $\Theta_N$ is deterministic and thus
$\Theta_N\cap O$ is either true or false, that is,
  \begin{gather*}
    \left\{ 
      \omega\in \Omega
      \ 
      |
      \ 
      \psi(\omega)
      \cap
      O
      \neq
      \emptyset
    \right\}
    \ 
    \in
    \ 
    \left\{ \Omega,\emptyset \right\}
    \subset
    \Sigma_S
    \qquad
    \text{for all open subsets}
    \ 
    O
    \subset
  \R^N_{\ge 0}
  \times
  \R
  \times
  \R^{N}
    \,.
  \end{gather*}
  Furthermore the objective function is a Caratheodory function.
  Thus, by Theorem there exists the argmax correspondence  of Problem~\ref{dual} and a measurable selector that selects
  $(\rho^\dagger,0_{N-n},\lambda_0^\dagger,\lambda^\dagger)$ 
  of Assumption.
  With this, we can define the optimal weights processes
  \begin{gather*}
    w^\dagger(x)
    \ 
    :=
    \ 
    w
    \left( 
    x,\rho^\dagger,0_{N-n},\lambda_0^\dagger,\lambda^\dagger
    \right)
    \qquad
    \text{indexed over}\ 
    x\in\mathcal{X}\,.
  \end{gather*}
  Due to the definition of the (general) weights processes and the measurability of the argmax selector, the random variables $w^\dagger(x)$ are measurable for all $x\in\mathcal{X}$.
  To end the measurability discussion, note that $w^\dagger(X)$ is a random variable.

%\section{Pseudo Dual Solution}
%In the formulation of Theorem~\ref{dual} we encounter "If (...) there exists the optimal solution $(\rho^\dagger,\lambda_0^\dagger,\lambda)$ ... " .
To be able to study asymptotic properties of the solutions we have to become independent of this assumption.
There are two observations that help with that.
First, if we solve Problem~\ref{dual} on a compact search space, there always exists a measurable solution (if the objective function is Caratheodory).
Second (with hindsight), we want optimal solutions to estimate an oracle parameter $\lambda^*$ well. 
To this end, we will define a compact parameter space that always contains the oracle parameter $\lambda^*$ and restrict Problem~\ref{dual} to it.
Then Theorem~\ref{th:argmax} gives us a measurable solution of the restricted Problem~\ref{dual}.
In some cases, this is only a local solution. It becomes global if it is in the interior of the compact parameter space.
Thus we call it the restricted, or the pseudo solution.

Consider 
the random vector (the oracle parameter)
\begin{align*}
  \lambda^*
  \ 
  :=
  \ 
  \left(
  0_{N},0,
  \left[
  \varphi^{'}
  \left(
  \frac
  {1}
  {\pi(X_i)}
  \right)
\right]_{i\in \left\{
  1,\ldots,N
\right\}}
  \right)
\end{align*}
with values in 
$\R^N_{\ge 0}\times \R\times \R^N$
.
By Assumption~\ref{asu:objective_f} (continuity of $\varphi^{'}$)
and Assumption\ref{asu:ps} ($\pi$ is continuous and bounded away from 0)
there exists (for all $N\in\mathbb{N}$) a compact set 
$
\Theta_N\subset
\R^N_{\ge 0}\times \R\times \R^N
$
that contains $\lambda^*$. 
We employ Theorem~\ref{th:argmax} to show, that there exists a measurable solution to Problem~\ref{dual} restricted to $\Theta_N^1$,
where 
\begin{align*}
  \Theta_N^1
  :=
  \left\{
    x\in 
\R^N_{\ge 0}\times \R\times \R^N
|
\norm{x-y}_2\le 1
\ 
\text{for some}\ 
y\in\Theta_N
  \right\}
  \,.
\end{align*}
\begin{lemma}
  \label{lem:pseud_sol}
  For all $N\in\mathbb{N}$ there exists a (pseudo) solution
  \begin{align*}
    s_N\colon
    \Omega
    \to
  \R^N_{\ge 0}
  \times
  \R
  \times
  \R^{N}
  \end{align*}
  to
  Problem~\ref{dual} restricted to $\Theta_N^1$ 
  that is
$
  \left(
  \mathcal{A},\mathcal{B}
  \left(
  \R^N_{\ge 0}
  \times
  \R
  \times
  \R^{N}
  \right)
  \right)
$-measurable.
If $s_N\in \mathrm{int}\, \Theta_N^1$, then it is the global solution
  $
  (\rho^\dagger,\lambda_0^\dagger,\lambda^\dagger)
  $.
\end{lemma}
\begin{proof}
Consider the (constant) correspondence
\begin{align*}
  \psi
  \colon
  (\Omega,\mathcal{A},\P)
  \ 
  \twoheadrightarrow
  \ 
  \R^N_{\ge 0}
  \times
  \R
  \times
  \R^{N}
  \,,
  \qquad
  \omega
  \ 
  \to
  \ 
  \Theta_N^1
  \,.
\end{align*}
This is weakly measurable, because $\Theta_N^1$ is deterministic.
Indeed,
$\Theta_N^1\cap O\neq\emptyset$ is either true or false, that is,
  \begin{gather*}
    \left\{ 
      \omega\in \Omega
      \ 
      |
      \ 
      \psi(\omega)
      \cap
      O
      \neq
      \emptyset
    \right\}
    \ 
    \in
    \ 
    \left\{ \Omega,\emptyset \right\}
    \subset
    \mathcal{A}
    \qquad
    \text{for all open subsets}
    \ 
    O
    \subset
  \R^N_{\ge 0}
  \times
  \R
  \times
  \R^{N}
    \,.
  \end{gather*}
  Next, consider the (random) objective function of (the maximize version of) Problem~\ref{dual}, that is,
  \begin{align*}
    &
  G\colon
  \Omega
  \times
  \left(
  \R^N_{\ge 0}
  \times
  \R
  \times
  \R^{N}
  \right)
  \to
  \R\cup \left\{
    -\infty
  \right\}
  \intertext{with}
    &
  G(\omega,(\rho,\lambda_0,\lambda))
  \ 
  =
  \ 
  -\infty
  \qquad 
  \text{if}\ 
  \rho_i\neq 0\  \text{for some}\ i>n
  \,,
  \\
  \intertext{and}\qquad
  &
  G(\omega,(\rho,\lambda_0,\lambda))
  \\
  &
  \ 
  =
  \ 
  \frac{1}{N}
\sum_{i=1} 
  ^N
  \Big[
    -
  T_i(\omega)
  \cdot
  \varphi^*
  \!
  \left( 
    \rho_i
    +
\lambda_0
+
\inner
{B(X_i)(\omega)}
{
\lambda
}
  \right)
  \ 
  +
  \ 
\lambda_0
+
\inner
{B(X_i)(\omega)}
{
\lambda
}
\Big]
  \ 
-
\ 
\inner
{\delta(\omega)}
{
  |\lambda|
}
\,.
  \end{align*}
  Clearly, the (random) objective function $G$  is a Caratheodory function. This follows from the (assumed) continuity of $\varphi^*$ and the measurability 
  of all random variables included.
  Since $G$ is also strictly concave, it has a unique argmax $s_N$  in $\Theta_N^1$. 
  By Theorem~\ref{th:argmax} this is $
  \left(
  \mathcal{A},\mathcal{B}
  \left(
  \R^N_{\ge 0}
  \times
  \R
  \times
  \R^{N}
  \right)
  \right)
$-measurable.
Furthermore, by the strict concavity of $G$, if $s_N\in \mathrm{int}\,\Theta_N^1$, then it's the global optimal also global. 
\end{proof}

We will prove later, that with probability going to 1 the latter will be the case.
Furthermore, we will prove, that the (measurable) solution converges to the random variable in probability.

%Before we can define the pseudo weights (based on the pseudo dual solution), we specify the basis functions.
%\section{Basis Functions}
%Let $
\left(
\mathcal{P}_N
\right)
$
denote a sequence of countable, $\mathcal{B}$-measurable partitions 
\begin{align*}
\mathcal{P}_N= \left\{
  A_{N,1},
  A_{N,2},
  \ldots
\right\}
\subset \mathcal{B}(\R^d)
\end{align*}
of $\R^d$, that is, 
\begin{align*}
  A_{N,i}\cap A_{N,j}=\emptyset
  \qquad
  \text{if}\ i\neq j
  \qquad
  \text{and}
  \qquad 
  \bigcap_{i\in\mathbb{N}}A_{N,i}
  \ 
  =
  \ 
  \R^d
  \,.
\end{align*}
We define
$ A_N(x) $ to be the cell of $ \mathcal{P}_N $ containing $x$, that is,
\begin{align*}
  A_N
  \colon
  \R^d 
  \ 
  \twoheadrightarrow 
  \ 
  \R^d  
  \,,\qquad
  x
  \ 
  \mapsto
  \ 
  A_N(x)
  \,,
\end{align*}
where $A_N(x)$ is the only cell containing $x$. 

Next, we define the (empirical) basis functions vector
\begin{align}
  \label{def:basis}
  B\colon
  \R^d\times \R^{d\cdot N}
  \ 
  \to
  \ 
  \R
  \,,
  \qquad
  (x,(x_1,\ldots,x_N))
  \ 
  \mapsto
  \ 
  \frac
  {
    \left[
    \mathbf{1}
    _{
      A_N(x)
    }
    (x_k)
    \right]
    _{k\in \left\{
        1,\ldots,N
    \right\}}
  }
  {
    \sum_{j=1}^N
    \mathbf{1}
    _{
      A_N(x)
    }
    (x_j)
    }
  \,,
\end{align}
where we keep to the convention $"0/0=0"$.
We shall extend $B$ to depend on the random vectors
$X,X_1,\ldots,X_N$.
The next lemma studies the measurability of the extensions.
\begin{lemma}
  \label{lem:basis_meas}
  \quad
  \begin{enumerate}[label=(\roman*)]
\item
  $B(\cdot,(X_1,\ldots,X_N))(\omega)$ is 
  $\left(
    \mathcal{B}(\R^d),\mathcal{B}(\R^N)
  \right)$-measurable
  and
  constant on each cell 
  $A_N\in\mathcal{P}_N$
  for all $\omega\in\Omega$. 
\item
  $B(X,(X_1,\ldots,X_N))$ is $\left(
    \mathcal{A},\mathcal{B}(\R^N)
  \right)$-measurable. 
  \end{enumerate}
\end{lemma}

\begin{proof}
Consider
for $k\in \left\{
  1,\ldots,N
\right\}$
and $\omega\in\Omega$
the indicator function
\begin{align}
  \label{33342}
  \mathbf{1}
  _
  {A_N(X_k(\omega))}
  \colon \R^d\to \left\{
    0,1
  \right\}
  \,.
\end{align}
Since 
$
  {A_N(X_k(\omega))}
  \in\mathcal{B}(\R^d)
$
this is a 
  $\left(
    \mathcal{B}(\R^d),\mathcal{B}(\R)
  \right)$-measurable
  function.
  From the definition of $B$ \eqref{def:basis} it follows the first part of (i).
  Since the indicator function in \eqref{33342} is 1 if $
  x\in
  {A_N(X_k(\omega))}
  $
  and 0 else, it is also constant on each cell
  $A_N\in\mathcal{P}_N$.
  It follows (i).
  To prove (ii), note that
\begin{align*}
  \mathbf{1}
  _
  {A_N(X_k(\omega))}(X(\omega))
  \ 
  =
  \ 
  \mathbf{1}
  \bigcup_{i\in\mathbb{N}}
  \left\{
    X,X_k \in A_{N,i}
  \right\}
  (\omega)
  \qquad
  \text{for all}\ 
  \omega\in\Omega
  \,,
\end{align*}
and
$
  \bigcup_{i\in\mathbb{N}}
  \left\{
    X,X_k \in A_{N,i}
  \right\}
  \in\mathcal{A}
$ by the 
$
\left( 
  \mathcal{A},
  \mathcal{B}(\R^d)
\right)
$-
measurability of $X$ and $X_k$.
\end{proof}

\begin{lemma}
  Let $(x,x_1,\ldots,x_N)\in\R^{d(N+1)}$.
  \begin{enumerate}[label=(\roman*)]
    \item
      $
      \sum_{k=1}^{N} 
      B_k(x,x_1,\ldots,x_N)
      \in
      \left\{ 0,1 \right\}
      $. 
      In particular,
      $
        x_1,\ldots,x_N\notin A_N(x)
      $
      is equivalent to
      $
      \sum_{k=1}^{N} 
      B_k(x,x_1,\ldots,x_N)
      =0
      $
  \end{enumerate}
\end{lemma}
\begin{proof}
  Let $(x,x_1,\ldots,x_N)\in\R^{d(N+1)}$.
  We prove \textit{(i)}.
  If 
      $
        x_1,\ldots,x_N\notin A_N(x)
      $,
  then
  \begin{align*}
    B_k(
        x,x_1,\ldots,x_N
    )
    \ 
    =
    \ 
\frac
  {
    \mathbf{1}
    _{
      A_N(x)
    }
    (x_k)
  }
  {
    \sum_{j=1}^N
    \mathbf{1}
    _{
      A_N(x)
    }
    (x_j)
    }
    \ 
    =
    \ 
    0
    \qquad
    \text{for all}\ 
    k\in \left\{ 1,\ldots,N \right\}
    \,.
  \end{align*}
  On the other hand, if the sum is 0 it holds
  \begin{align*}
    \mathbf{1}
    _{
      A_N(x)
    }
    (x_k)
    \ 
  =
  \ 
  0
    \qquad
    \text{for all}\ 
    k\in \left\{ 1,\ldots,N \right\}
    \,.
  \end{align*}
  It follows the desired equivalence.
  If 
  \begin{align*}
    \mathbf{1}
    _{
      A_N(x)
    }
    (x_k)
    \ 
  =
  \ 
  1
    \qquad
    \text{for some}\ 
    k\in \left\{ 1,\ldots,N \right\}
    \,,
  \end{align*}
  then
  $
    \sum_{j=1}^N
    \mathbf{1}
    _{
      A_N(x)
    }
    (x_j)
    \ge 1
  $
  and thus "$0/0$" doesn't occure. 
  It follows
  \begin{align*}
      \sum_{k=1}^{N} 
      B_k(x,x_1,\ldots,x_N)
      \ 
      =
      \ 
\frac
  {
      \sum_{k=1}^{N} 
    \mathbf{1}
    _{
      A_N(x)
    }
    (x_k)
  }
  {
    \sum_{j=1}^N
    \mathbf{1}
    _{
      A_N(x)
    }
    (x_j)
    }
    \ 
    =
    \ 
    1
    \,.
  \end{align*}
\end{proof}
\begin{lemma}
  For all continuous functions $f\colon \R^d\to \R$ it holds
 \begin{align}
   \begin{split}
   &
   \left|
  \sum_{k=1}^{N}
    B_k(X,X_1,\ldots,X_N)\cdot 
    f(X_k)
    -
    f(X)
   \right|
   \ 
   \le
   \ 
   \omega
   \left(
    f,h_N^d
   \right)
   \,,
   \end{split}
 \end{align}
 where $\omega(f,\cdot)$ is the modulus of continuity of $f$. 

\end{lemma}

\begin{proof}
  \begin{align}
   \begin{split}
   &
   \left|
  \sum_{k=1}^{N}
    B_k(X_i,X_1,\ldots,X_N)\cdot 
    f(X_k)
    -
    f(X_i)
   \right|
   \\
   &
   \ 
   \le
   \ 
  \sum_{k=1}^{N}
    B_k(X_i,X_1,\ldots,X_N)
    \mathbf{1}\left\{
      X_k\in A_N(X_i)
    \right\}
    \left|
    f(X_k)
    -
    f(X_i)
    \right|
   \\
   &
   \ 
   \le
   \ 
   \omega
   \left(
    f,h_N^d
   \right)
   \,,
   \end{split}
 \end{align}

\end{proof}

\begin{lemma}
  \begin{enumerate}[label=(\roman*)]
    \item
      $\norm{B(X,X_1,\ldots,X_N)}_2\lesssim 1$ 
      \item
      $
      \sum_{i=1}^{N}
        B_k(X,X_1,\ldots,X_N)
        =1
      $
      for all $i\in \left\{
        1,\ldots,N
      \right\}$
      \item
      \begin{align*}
        \frac
        {1}
        {N}
        \sum_{i,k=1}^{N}
            \left|
        B_k(X_i,X_1,\ldots,X_N)
        \cdot
            \varphi
            \left(
              \frac
              {1}
              {\pi(X_k)}
            \right)
            \ 
            -
            \ 
            \varphi
            \left(
              \frac
              {1}
              {\pi(X_i)}
            \right)
            \right|
            \ 
            \to
            \ 
            0
            \qquad
            \text{for}
            \ 
            N\to\infty
          \end{align*}
\item
      \begin{align*}
        \frac
        {1}
        {\sqrt{N}}
        \sum_{i,k=1}^{N}
        \sup_{z\in\R}
            \left|
        B_k(X,X_1,\ldots,X_N)
        \cdot
        F_{Y(1)}(z|X_k)
            \ 
            -
            \ 
        F_{Y(1)}(z|X_i)
            \right|
            \ 
            \to
            \ 
            0
            \qquad
            \text{for}
            \ 
            N\to\infty
      \end{align*}
\end{enumerate}
\end{lemma}
\begin{proof}
If at least one 
$
B_k(X,X_1,\ldots,X_N)
>
0
$, the basis functions sum to 1. If $B_k(X,X_1,\ldots,X_N)=0$ for all basis functions, the sum is 0.
Thus
\begin{align}
  \label{8882}
  &
  \sum_{k=1}^{N}
B_k(X,X_1,\ldots,X_N)
  \ 
  \in
  \ 
  \left\{ 0,1 \right\}
  \notag
  \intertext{
Since $B_i(X_i,X_1,\ldots,X_N)>0$, it holds
  }
  &
  \sum_{k=1}^{N}
  B_k(X_i)
  \ 
  =
 \  
  1
  \qquad
  \text{for all}\ 
  i\in
  \left\{ 1,\ldots,N \right\}
  \,.
\end{align}
We prove \textit{(i)}.
Since 
\begin{gather*}
B_k(X,X_1,\ldots,X_N)
\ 
\in
\ 
[0,1]
\qquad
\text{for all}\ 
k\in \left\{ 1,\ldots,N \right\}
\,,
\end{gather*}
it holds
\begin{gather*}
  \label{basis_l2_bdd}
  \norm{
B(X,X_1,\ldots,X_N)
}_2^2
  \ 
  =
  \ 
  \sum_{k=1}^{N} 
B(X,X_1,\ldots,X_N)
  ^2
  \ 
  \le
  \ 
  \sum_{k=1}^{N} 
B(X,X_1,\ldots,X_N)
  \ 
  \in
  \ 
  \left\{ 0,1 \right\}
\,.
\end{gather*}

 Note, that for all continuous functions $f\colon \R^d\to \R$ it holds
 \begin{align}
   \begin{split}
   &
   \left|
  \sum_{k=1}^{N}
    B_k(X_i,X_1,\ldots,X_N)\cdot 
    f(X_k)
    -
    f(X_i)
   \right|
   \\
   &
   \ 
   \le
   \ 
  \sum_{k=1}^{N}
    B_k(X_i,X_1,\ldots,X_N)
    \mathbf{1}\left\{
      X_k\in A_N(X_i)
    \right\}
    \left|
    f(X_k)
    -
    f(X_i)
    \right|
   \\
   &
   \ 
   \le
   \ 
   \omega
   \left(
    f,h_N^d
   \right)
   \,,
   \end{split}
 \end{align}
 where $\omega(f,\cdot)$ is the modulus of continuity of $f$. 
 Thus
      \begin{align*}
        \frac
        {1}
        {N}
        \sum_{i,k=1}^{N}
            \left|
        B_k(X,X_1,\ldots,X_N)
        \cdot
            \varphi^{'}
            \left(
              \frac
              {1}
              {\pi(X_k)}
            \right)
            \ 
            -
            \ 
            \varphi^{'}
            \left(
              \frac
              {1}
              {\pi(X_i)}
            \right)
            \right|
            \ 
            \le
            \ 
   \omega
   \left(
     \varphi^{'},h_N^d
   \right)
            \ 
            \to
            \ 
            0
          \end{align*}
          for $N\to\infty$.
          In the same way it follows
      \begin{align*}
            &
        \frac
        {1}
        {\sqrt{N}}
        \sum_{i,k=1}^{N}
        \sup_{z\in\R}
            \left|
        B_k(X,X_1,\ldots,X_N)
        \cdot
        F_{Y(1)}(z|X_k)
            \ 
            -
            \ 
        F_{Y(1)}(z|X_i)
            \right|
            \\
            &
            \ 
            \le
            \ 
            \sqrt{N}
            \sup_{z\in\R}
            \omega
            \left(
        F_{Y(1)}(z|\cdot)
        ,
        h_N^d
            \right)
            \ 
            \to
            \ 
            0
            \qquad
            \text{for}
            \ 
            N\to\infty
            \,.
      \end{align*}
\end{proof}

%\section{Pseudo Weights Process}
%Based on Theorem~\ref{dual_solution_th}
we want to use the dual (pseudo) solution $s_N$ to construct weights.
To this end, we define the (empirical) weights function
\begin{align*}
 w\ \colon\
 &
 \left( 
  \R^d\times \R^{d\cdot N}
 \right)
  \times
  \left( 
\R^N_{\ge 0}\times \R\times \R^N
  \right)
  \to
  \R^N
  \\
 &
  \left( 
  (x,x_1,\ldots,x_N),(\rho,\lambda_0,\lambda)
  \right)
  \ 
  \mapsto
  \ 
  \left[ 
  (\varphi^{'})^{-1}
  \left( 
    \rho_i
    +
    \lambda_0
    +
    \inner
    {B(x,x_1,\ldots,x_N)}
    {\lambda}
  \right)
\right]_{i\in \left\{ 1,\ldots,N \right\}}
\,.
\end{align*}
\begin{definition}
  Let $s_N$ be the (pseudo) solution of Lemma~\ref{lem:pseud_sol}.
  We define the (pseudo) weights process 
  $\left\{ w^\dagger(x) | x\in\R^d\right\}$
  by
  \begin{align*}
    w^\dagger(x) 
    \ 
    :=
    \ 
    w(x,X_1,\ldots,X_N,s_N)
    \qquad
    \text{for all}\ 
    x\in\R^d
    \,.
  \end{align*}
\end{definition}


\chapter{Consistency of the Weights Process}
  The goal of this section is to establish consistency of the
weights process for the inverse propensity score.
To this end,
we first show that asymptotically there exists an optimal solution 
$
(\rho^\dagger,\lambda^\dagger,\lambda_0^\dagger)
$
to Problem~\ref{dual} that converges to the oracle parameter
\begin{align*}
  (0_N,0,\lambda^*)
  \qquad
  \text{where}
  \qquad
  \lambda^*
  \ 
  :=
  \ 
\left[ 
\varphi^{'} \left( \frac{1}{\pi(X_k)} \right)
\right]_{k\in \left\{ 1,\ldots,N \right\}}
\end{align*}
in probability (see Theorem).
This result justifies Assumption~\ref{asu:feas_dual_sol}.
Furthermore, we will identify the dual solution from Lemma~\ref{lem:meas_dual_sol} 
with the consistent dual solution to derive consistency of the weights process for the inverse propensity score.

\subsection{Consistency of the Dual Solution}
  We get a grip by the following lemma.
The high-level idea is that the existence of the optimal dual solution and its proximity to the oracle parameter can be analysed by the objective function.
\begin{lemma}
  Let $m$,$n\in\mathbb{N}$ and let 
  $
  g \,:\, \R^m\times \R^n_{\ge 0} \to \overline{\R}
  $ 
  be a continuous and proper convex function.
  Consider 
  \begin{gather*}
    \tilde{S}(\varepsilon)
    :=
    \left\{ 
      (
      \Delta,
      \Delta_\rho
      )
      \in
      \R^m \times \R^n_{\ge 0}
      \ 
      \colon
      \ 
      \norm{
      (
      \Delta,
      \Delta_\rho
      )
      }_2
      =
      \varepsilon
    \right\}
    \qquad
    \text{for}
    \ 
    \varepsilon>0
    \,.
  \end{gather*}
Then 
  for all $y \in \R^m$ and $\varepsilon>0$ 
    \begin{gather}
      \label{696}
      \inf 
      \left\{ 
        g(y+
        \Delta,\Delta_\rho)
        -
        g(y,0)
      \ 
        \colon
      \ 
      (
      \Delta,
      \Delta_\rho
      )
      \in
    \tilde{S}(\varepsilon)
      \right\}
      \ 
      \ge
      \ 
      0
    \end{gather}
    implies
    the existence of  
    a global minimum
    \begin{gather*}
    (
    y^*
    ,
    y^*_\rho
    )
    \in \,\R^m\times\R^n_{\ge 0}
    \quad
    \text{of $g$ such that}\quad
      \norm{
    (
    y^*
    ,
    y^*_\rho
    )
      - (y,0)}_2 \le \varepsilon
      \,.
    \end{gather*}
\end{lemma}
\begin{proof}
  We start by defining the convex set
  \begin{gather*}
    \tilde{B}(\varepsilon)
    \ 
    :=
    \ 
    \left\{ 
      (
      \Delta,
      \Delta_\rho
      )
      \in
      \R^m \times \R^n_{\ge 0}
      \ 
      \colon
      \ 
      \norm{
      (
      \Delta,
      \Delta_\rho
      )
      }_2
      \le
      \varepsilon
    \right\}
    \qquad
    \text{for}
    \ 
    \varepsilon>0
    \,.
  \end{gather*}
  Then the translation 
  $
  (y,0)
  +
    \tilde{B}(\varepsilon)
  $
  is also convex.
  Assume towards a contradiction that it holds \eqref{696}
  and that there exists 
  \begin{gather}
    \label{698}
  (
x^*,x^*_\rho
  )
  \ 
\in
  \ 
\R^m\times\R^m_{\ge 0}\setminus 
\left(
  (y,0)
  +
    \tilde{B}(\varepsilon)
\right)
\quad
\text{such that}\quad
g
  (
x^*,x^*_\rho
  )
  \ 
  <
  \ 
  g(y,0)
  \,.
  \end{gather}
  Since 
  $
  (y,0)
  +
    \tilde{B}(\varepsilon)
  $
  is bounded, the line segment between 
  $
  (
x^*,x^*_\rho
  )
  $
  and
  $
  (y,0)
  $
  crosses its boundary. The boundary consists of two disjoint sets
  \begin{gather*}
    S_0(\varepsilon)
    :=
    \left\{ 
      (y+\Delta,0)
      \ 
      \colon
      \ 
      \Delta\in\R^m\ 
      \text{and}\ 
      \norm{\Delta}_2<\varepsilon
    \right\}
    \qquad
    \text{and}
    \qquad
    \tilde{S}(\varepsilon)
    \,.
  \end{gather*}
  Clearly, if the line segment does not cross $\tilde{S}(\varepsilon)$ it leaves $\R^m\times\R^n_{\ge 0}$.
  But this is not possible.
  Thus, there exists $(\Delta,\Delta_\rho)\in \tilde{S}(\varepsilon)$ and $\theta\in(0,1)$ such that 
  \begin{gather}
    \label{697}
    \theta 
    \cdot
  (
x^*,x^*_\rho
  )
  \ 
  +
  \ 
  (
  1
  -
\theta
  )
  \cdot
  (y,0)
  \ 
  =
  \ 
  (y+\Delta,\Delta_\rho)
  \,.
  \end{gather}
  It follows
 \begin{align*}
      \begin{split}
      g(y,0)
      \ 
      \le
      \ 
      g
  (y+\Delta,\Delta_\rho)
&
      \ 
      =
      \ 
      g
      \left( 
    \theta 
    \cdot
  (
x^*,x^*_\rho
  )
  \ 
  +
  \ 
  (
  1
  -
\theta
  )
  \cdot
  (y,0)
      \right)
      \\
&
      \ 
      \le
      \ 
    \theta 
    \cdot
      g
  (
x^*,x^*_\rho
  )
  \ 
  +
  \ 
  (
  1
  -
\theta
  )
  \cdot
  g
  (y,0)
  \ 
      <
  \ 
      g(y,0)
  \,    ,
      \end{split}
    \end{align*}
    which is a contradiction.
    The first inequality is due to \eqref{696}, the equality is due to \eqref{697}, the second inequality is due to the convexity of $g$,
    and the strict inequality is due to assumption \eqref{698}.
    Thus, all values outside 
    $(y,0)+\tilde{B}(\varepsilon)$
    are greater or equal $(y,0)$.
    Since 
    $(y,0)+\tilde{B}(\varepsilon)$
    is also compact, the continuous function $g$ has a local minimum
    \begin{gather*}
      (y^*,y^*_\rho)\in
    (y,0)+\tilde{B}(\varepsilon)
    \,.
    \end{gather*}
    But then it holds
    \begin{gather*}
      g
      (y^*,y^*_\rho)
      \ 
      \le
      \ 
      g(y,0)
      \ 
      \le
      \ 
      g
      (x,x_\rho)
      \qquad
      \text{for all}
      \qquad 
      (x,x_\rho)
      \in
\R^m\times\R^m_{\ge 0}\setminus 
\left(
  (y,0)
  +
    \tilde{B}(\varepsilon)
\right)
    \end{gather*}
    and
    \begin{gather*}
      g
      (y^*,y^*_\rho)
      \ 
      \le
      \ 
      g
      (z,z_\rho)
      \qquad
      \text{for all}
      \qquad 
      (z,z_\rho)
      \in
  (y,0)
  +
    \tilde{B}(\varepsilon)
    \,.
    \end{gather*}
    Thus,
    $
      (y^*,y^*_\rho)
    $
    is also a global minimum in
    $
\R^m\times\R^m_{\ge 0}
    $.
Since
    $
      (y^*,y^*_\rho)
      \in
  (y,0)
  +
    \tilde{B}(\varepsilon)
    $
    there exists
    $
    (\Delta,\Delta_\rho)\in
    \tilde{B}(\varepsilon)
    $
    such that 
    \begin{gather*}
      (y^*,y^*_\rho)
      =
      (y+\Delta,\Delta_\rho)
      \qquad
      \text{for some}\ 
      (\Delta,\Delta_\rho)
      \in
      \tilde{B}(\varepsilon)
      \,.
    \end{gather*}
  Thus
  \begin{gather*}
    \norm{
      (y^*,y^*_\rho)
      -
      (y,0)
    }_2
    =
    \norm{
      (\Delta,\Delta_\rho)
    }_2
    \le \varepsilon
    \,.
  \end{gather*}
  This finish the proof.
\end{proof}
\begin{remark}
  I learned of the high-level idea from \cite[page 22]{Wang2019}.
  I adapted it to the needs of the subsequent analysis and provided the details by myself.
  Note, that the hint in \cite[page 22]{Wang2019}
  uses strict inequality in the statement.
  I found out that this can be relaxed.
  It is crucial to my further approach that this holds (only) with inequality, because I use measurability properties to obtain convergence.
\end{remark}


On the basis of the (random) objective function $G$ of Problem~\ref{dual} (see Definition~\ref{def:rand_obj_f}) we define, for $\varepsilon>0$, an auxiliary function
   \begin{align*}
     \underline{
     \Delta G^*
     _\varepsilon
     }
     \colon
     &
     \left( \Omega,\sigma(D_N),\P \right)
     \ 
     \to
     \ 
     \overline{\R}
     \\
     &
     \omega
     \ 
     \mapsto
     \ 
   \inf
   \left\{ 
 G
   \left( 
     \omega,
     \left( 
\Delta_\rho,\Delta_0,\lambda^*(\omega)+\Delta
     \right)
   \right)
   -
   G
   \left(
     \omega,
     \left( 
0_N,0,\lambda^*(\omega)
     \right)
   \right)
   \ 
   \colon
   \ 
   \norm{
\Delta_\rho,\Delta_0,\Delta
   }_2
   =\varepsilon
   \right\}
   \end{align*}
\begin{lemma}
\label{lem:meas_inf_G}
  For all $\varepsilon>0$ the function
  $
     \underline{
     \Delta G^*
     _\varepsilon
     }
  $
  is
  $
  \left( 
  \sigma(D_N),\mathcal{B}(\overline{\R})
  \right)
  $-measurable.
\end{lemma}
\begin{proof}
  Let $\varepsilon>0$.
  By Lemma~\ref{lem:caratheo_G}, the function
\begin{align*}
     \Delta G
     _\varepsilon
     \colon
     &
     \Omega
     \times
     \left( 
     \R^N
     \times
    \left( 
    \R^N_{\ge 0}\times\R\times\R^N
    \right)
     \right)
     \ 
     \to
     \ 
     \overline{\R}
     \\
     &
     \left( 
     \omega
     ,
     \left( 
     \lambda,
     \left( 
\Delta_\rho\Delta_0\Delta
     \right)
     \right)
     \right)
     \ 
     \mapsto
     \ 
 G
   \left( 
     \omega,
     \left( 
\Delta_\rho,\Delta_0,\lambda+\Delta
     \right)
   \right)
   -
   G
   \left(
     \omega,
     \left( 
0_N,0,\lambda
     \right)
   \right)
   \end{align*}
is Caratheodory.
Since
  $
  \left\{
   \norm{
\Delta_\rho\\\Delta_0\\\Delta
   }_2
   =\varepsilon
  \right\}
  $
  is compact in 
  $
    \R^N_{\ge 0}\times\R\times\R^N
    $,
  the function
\begin{align*}
  \underline{
     \Delta G
     _\varepsilon
  }
     \colon
     &
     \Omega
     \times
     \R^N
     \ 
     \to
     \ 
     \overline{\R}
     \\
     &
     \left( 
     \omega
     ,
     \lambda
     \right)
     \ 
     \mapsto
     \ 
   \inf
    \left\{ 
 G
   \left( 
     \omega,
     \left( 
\Delta_\rho,\Delta_0,\lambda+\Delta
     \right)
   \right)
   -
   G
   \left(
     \omega,
     \left( 
0_N,0,\lambda
     \right)
   \right)
   \ 
   \colon
   \ 
   \norm{
\Delta_\rho\Delta_0\Delta
   }_2
   =\varepsilon
    \right\}
   \end{align*}
is Caratheodory.
Since 
$
\lambda^*
$
is 
$
\left( 
\sigma
(D_N)
,
\mathcal{B}(\R^N)
\right)
$-
measurable it follows the statement.
\end{proof}
 \begin{lemma}
   \label{bw:cd:lem2}
   It holds
   for all $\varepsilon>0$
\begin{gather}
   \P
   \left[ 
     \underline{
     \Delta G^*
     _\varepsilon
     }
     \ge 
     0
   \right]
   \ 
   \to
   \ 
   1
   \qquad
   \text{for}
   \ 
   N\to\infty
   \,.
\end{gather}
 \end{lemma}
 \begin{proof}
   Let $\varepsilon>0$
   and 
   $\norm{
   \Delta_\rho,\Delta_0,\Delta
   }_2=\varepsilon$.
   We show
\begin{gather}
   \P
   \left[ 
     \underline{
     \Delta G^*
     _\varepsilon
     }
     \ge 
     -\tilde{\varepsilon}
   \right]
   \ 
   \to
   \ 
   1
   \qquad
   \ 
   \text{for}
   \ 
   N\to\infty
   \ 
   \text{for all}\ 
   \tilde{\varepsilon}>0
   \,.
\end{gather}
Then the result follows from the measurability of 
$
     \underline{
     \Delta G^*
     _\varepsilon
     }
$
(see Lemma~\ref{lem:meas_inf_G}).
To this end, note, that
\begin{gather*}
  G(\rho,\lambda_0,\lambda)
  \ 
  =
  \ 
  g(\rho,\lambda_0,\lambda)
  \ 
  +
  \ 
  \inner{\delta}{|\lambda|}
  \qquad
  \text{for all}\ 
  (\rho,\lambda_0,\lambda)
  \in
  \R^N_{\ge 0}
  \times
  \R
  \times
  \R^{N}
  \,,
\end{gather*}
with
\begin{gather*}
  g
  \ 
  :=
  \ 
  (\rho,\lambda_0,\lambda)
  \ 
  \mapsto
  \ 
     \frac{1}{N}
     \left( 
\sum_{i=1} 
  ^N
  T_i
  \cdot
  \varphi^*
  \!
  \left( 
    \rho_i
    +
\lambda_0
+
\inner
{B(X_i)}
{
\lambda
}
  \right)
  \ 
  -
\ 
\lambda_0
-
\inner
{B(X_i)}
{
\lambda
}
     \right)
  \,.
\end{gather*}
Since we assume $\varphi^*$ to be continuously differentiable (it is always convex),
$g$ is a continuously differentiable convex function with gradient
\begin{align*}
  &
  (\rho,\lambda_0,\lambda)
  \\
  &
  \ 
  \mapsto
  \ 
     \frac{1}{N}
     \left( 
\sum_{i=1} 
  ^N
  T_i
  \cdot
  (
  \varphi^{'}
  )
  ^{-1}
  \!
  \left( 
    \rho_i
    +
\lambda_0
+
\inner
{B(X_i)}
{
\lambda
}
  \right)
  \left[ 
    e_i^\top,1,B(X_i)^\top
  \right]^\top
  \ 
  -
  \ 
  \left[ 
    0_N^\top,1,B(X_i)^\top
  \right]^\top
     \right)
  \,.
\end{align*}
Thus, by \eqref{cv:primer:mvthe},
it holds
\begin{align*}
\label{99909}
\begin{split}
  &
  G
   \left( 
\Delta_\rho,\Delta_0,\lambda^*+\Delta
   \right)
   \ 
   -
   \ 
   G
   \left(
0_N,0,\lambda^*
   \right)
   \\
   &
   \ 
   \ge
   \ 
   \frac{1}{N}
  \left( 
\sum_{i=1} 
  ^N
  T_i
  \cdot
  (
  \varphi^{'}
  )
  ^{-1}
  \!
  \left( 
\inner
{B(X_i)}
{
\lambda^*
}
  \right)
  \left[ 
    e_i^\top,1,B(X_i)^\top
  \right]
  \ 
  -
  \ 
  \left[ 
    0_N^\top,1,B(X_i)^\top
  \right]
  \right)
  \begin{bmatrix}
    \Delta_\rho\\\Delta_0\\\Delta
  \end{bmatrix}
    \\
  &
  \qquad
  +
  \ 
  \inner{\delta}
  {|
\lambda^*
+
\Delta
  |
  -
  |
\lambda^*
  |
}
\\
   &
   \ 
   \ge
   \ 
   \frac{1}{N}
\sum_{i=1} 
  ^N
  \left( 
    T_i\cdot
  (
  \varphi^{'}
  )
  ^{-1}
  \!
  \left( 
\inner
{B(X_i)}
{
\lambda^*
}
  \right)
  \ 
  -
  \ 
  1
  \right)
  \left[ 
    e_i^\top,1,B(X_i)^\top
  \right]
  \cdot
  \begin{bmatrix}
    \Delta_\rho\\\Delta_0\\\Delta
  \end{bmatrix}
  +
  \inner{e_i}{\Delta_\rho}
  \\
  &
  \qquad
  +
  \ 
  \inner{\delta}
  {|
\lambda^*
+
\Delta
  |
  -
  |
\lambda^*
  |
}
\\
   &
   \ 
   \ge
   \ 
   -
   \frac{1}{N}
\sum_{i=1} 
  ^N
  \left|
   \left( 
     T_i\cdot
  (
  \varphi^{'}
  )
  ^{-1}
  \!
  \left( 
\inner
{B(X_i)}
{
\lambda^*
}
  \right)
  \ 
  -
  \ 
  1
  \right)
  \left[ 
    e_i^\top,1,B(X_i)^\top
  \right]
  \cdot
  \begin{bmatrix}
    \Delta_\rho\\\Delta_0\\\Delta
  \end{bmatrix}
  \right|
  \\
  &
  \qquad
  -
  \ 
  \inner{\delta}
  {|
\Delta
  |
}
\\
&
\
=:
\ 
-
I_1
\\
&
\qquad
-
\
 I_2
\end{split}
\end{align*}
\subsection*{$I_1$}
By the Cauchy-Schwarz inequality, Lemma~\ref{lem:basis_sum}.\textit{(iii)} it holds
\begin{align*}
  \left|
   \left[ 
    e_i^\top,1,B(X_i)^\top
  \right]
  \cdot
  \begin{bmatrix}
    \Delta_\rho\\\Delta_0\\\Delta
    \end{bmatrix}
  \right|
  \ 
  \le
  \ 
  \norm{
    \Delta_\rho,\Delta_0,\Delta
  }_2
  \ 
  \le
  \ 
  \varepsilon
  \,.
\end{align*}
Furthermore,
\begin{align*}
  &
    \frac{1}{N}
\sum_{i=1} 
  ^N
  \left|
   \left( 
     T_i\cdot
  (
  \varphi^{'}
  )
  ^{-1}
  \!
  \left( 
\inner
{B(X_i)}
{
\lambda^*
}
  \right)
  \ 
  -
  \ 
  1
  \right)
  \right|
  \\
  &
  \ 
  \le
  \ 
    \frac{1}{N}
\sum_{i=1} 
^{N}
\left|
  1-
  \frac
  {T_i}
  {\pi(X_i)}
\right|
\\
&
\qquad
+
    \frac{1}{N}
    \sum_{i=1}^{N} 
\omega
\left(
  (
  \varphi^{'}
  )
  ^{-1}
  ,
  \left|
    \sum_{k=1}^{N}
  B_k(X_i)
  \cdot
  \varphi^{'}
  \left(
    \frac
    {1}
    {\pi(X_k)}
  \right)
  -
  \varphi^{'}
  \left(
    \frac
    {1}
    {\pi(X_i)}
  \right)
  \right|
  \right)
  \\
  &
  \ 
  =:
  \ 
  J_1
  \\
  &
  \qquad
  +
  \ 
  J_2
\end{align*}
\subsubsection*{$J_1$}
By the properties of conditional expectation it holds
\begin{gather*}
  \E
  \left[ 
    \frac{T}{\pi(X)}
  \right]
  =
  \E
  \left[ 
    \frac{\E[T|X]}{\pi(X)}
  \right]
  =1
  \,.
\end{gather*}
Also
\begin{gather}
  \E
  \left[ 
    \left| 
    1
    -
    \frac{T}{\pi(X)}
    \right|
  \right]
  \le
  1
  +
  \E
  \left[ 
    \frac{T}{\pi(X)}
  \right]
  =
  2
  \,.
\end{gather}
Thus Etemadi's ($\mathcal{L}_1$ version) strong law of large numbers (cf.\cite[Theorem~5.17]{Klenke2020}) applies
to $J_1$, that is,
$J_1\overset{\P}{\to}0$.
\subsubsection*{$J_2$}
By Lemma~\ref{lem:basis_2}.\textit{(i)}
and the uniform continuity of 
$
  (
  \varphi^{'}
  )
  ^{-1}
$
it holds
\begin{align*}
\omega
\left(
  (
  \varphi^{'}
  )
  ^{-1}
  ,
  \left|
    \sum_{k=1}^{N}
  B_k(X_i)
  \cdot
  \varphi^{'}
  \left(
    \frac
    {1}
    {\pi(X_k)}
  \right)
  -
  \varphi^{'}
  \left(
    \frac
    {1}
    {\pi(X_i)}
  \right)
  \right|
  \right)
  &
  \ 
  \le
  \ 
\omega
\left(
  (
  \varphi^{'}
  )
  ^{-1}
  ,
\omega
\left(
  \varphi^{'}
  ,
  h_N^d
  \right)
  \right)
  \\
  &
  \ 
  \to
  \ 
  0
  \,.
\end{align*}
Thus $J_2\to 0$.
\subsubsection*{Conclusion $I_1$}
It follows from the analysis of $J_1$ and $J_2$
\begin{align*}
  \P
  \left[
  I_1 \le \tilde{\varepsilon}
  \right]
  \to 1
  \qquad
  \text{for all}\ 
  \tilde{\varepsilon}>0
  \,.
\end{align*}
\subsection*{$I_2$}
Since
$\delta>0$
we get
\begin{align*}
     \inner{\delta}
     {|\Delta|}
     \ 
     \le
     \ 
     \norm{\delta}_1
     \norm{\Delta}_\infty
     \ 
     \le
     \ 
     \norm{\delta}_1
     \varepsilon
     \,,
\end{align*}
Since $\norm{\delta}_1$ converges to $0$ in probability we get
\begin{align*}
  \P
  \left[
  I_2 \le \tilde{\varepsilon}
  \right]
  \ 
  \to
  \ 
  1
  \qquad
  \text{for all}\ 
  \tilde{\varepsilon}>0
  \,.
\end{align*}
\subsection*{Conclusion}
By \eqref{99909} we get
\begin{align*}
  \P
  \left[ 
   G
   \left( 
\Delta_\rho,\Delta_0,\lambda^*+\Delta
   \right)
   \ 
   -
   \ 
   G
   \left(
0_N,0,\lambda^*
   \right)
   \ge - 
  \tilde{\varepsilon}
  \right]
  \ 
  \to
  \ 
  1
  \qquad
  \text{for all}\ 
  \tilde{\varepsilon}>0
  \,.
\end{align*}
Thus
\begin{align*}
  \P
  \left[ 
  \underline{\Delta G^*_\varepsilon}
   \ge - 
  \tilde{\varepsilon}
  \right]
  \ 
  \to
  \ 
  1
  \qquad
  \text{for all}\ 
  \tilde{\varepsilon}>0
  \,.
\end{align*}
From the measurability of 
$
  \underline{\Delta G^*_\varepsilon}
$ 
(see Lemma~\ref{lem:meas_inf_G})
it follows
\begin{align*}
  \P
  \left[ 
  \underline{\Delta G^*_\varepsilon}
   \ge 0
  \right]
  \ 
  \to
  \ 
  1
  \,.
\end{align*}
 \end{proof}






\chapter{Convergence of the Weighted Mean}
  \section{Tools}
  For the subsequent analysis we need the theory of empirical processes.
For an introduction to empirical processes see \cite[§19]{Vaart2000}. For a thorough treatment see \cite[§2]{vaart2013}. 
\subsection{Empirical Processes - Definition}
  Let 
$
  \left( 
    \Omega,
    \mathcal{A},
    \P
  \right)
$
be a probability space,
$
  \left( 
    \mathcal{Z},
    \Sigma
  \right)
$
a measurable space, and 
\begin{gather*}
  \xi_1,\ldots,\xi_N
  :
  \left( 
    \Omega,
    \mathcal{A},
    \P
  \right)
  \to
  \left( 
    \mathcal{Z},
    \Sigma
  \right)
  \quad
  \text{independent and identically-distributed
  }
\end{gather*}
random variables
with probability distribution $\P_{\!\xi}$.
\index{$\P_{\xi}$, probability distribution of the random variable $\xi$}
Let $\mathcal{F}$ be a class of measurable functions 
\index{$\mathcal{F}$, class of measurable functions}
$
  f:
  \left( 
    \mathcal{Z},
    \Sigma
  \right)
    \to
  \left( 
    \R,
    \mathcal{B}(\R)
  \right)
$, where
$
    \mathcal{B}(\R)
$
is the Borel-$\sigma$-algebra on $\R$.
\index{$\mathcal{B}(\cdot)$, Borel-$\sigma$-algebra}
Then $\mathcal{F}$
induces a stochastic process by
\begin{gather}
  f
  \ 
  \mapsto
  \ 
  \G_N f 
  \ 
  :=
  \ 
  \frac{1}{\sqrt{n}}
  \sum_{i=1}^{N} 
  \left(
    f(\xi_i)
    -
    \E_\xi[f]
  \right)
  \,,
\end{gather}
where
$
    \E_\xi[f]
    :=
    \int_\mathcal{Z}
    f
    \,
    d\P_\xi
$.
We call
$\G_N$ the  \textbf{empirical process} indexed by $\mathcal{F}$.
\index{$\G_N$, empirical process}
The purpose of this construction is to study the behaviour of a centered, scaled arithmetic mean uniformly over $\mathcal{F}$.
To this end, we define the (random) norm
\begin{gather}
  \norm{\G_n}_\mathcal{F}
  :=
  \sup_
        { f \in \mathcal{F}}
        \left|
          \G_N f
        \right|
        .
\end{gather}
\index{
$
  \norm{\G_n}^*_\mathcal{F}
$, (measurable version of the) uniform norm of an empirical process}
We stress that 
$
  \norm{\G_n}_\mathcal{F}
$
often ceases to be measurable, even in simple situations~\cite[page 3]{vaart2013}.
To deal with this, we introduce the notion of \textbf{outer expectation} $\E^*$ (see \cite[page~6]{vaart2013}):
\index{$
  \E^*[Z]
$, outer expectation}
\begin{gather*}
  \E^*[Z]
  \ 
  :=
  \ 
    \inf
  \left\{ 
    \E[U]
  \ 
  \lvert
  \ 
    U\ge Z,
    \ 
    U:
  \left( 
    \Omega,
    \mathcal{A},
    \P
  \right)
  \to 
  \left( 
    \overline{\R},
    \mathcal{B}(\overline{\R})
  \right)
  \text{measurable and}
  \ 
  \E[U]<\infty
  \right\}
\end{gather*}
In our application the technical difficulties halt at this point, because we only consider $Z$ with $\E^*[Z]<\infty$. Then there exists a smallest measurable function $Z^*$ dominating $Z$ with
$\E^*[Z]=\E[Z^*]$ (see \cite[Lemma~1.2.1]{vaart2013}).

An \textbf{envelope function} $F$ of a class $\mathcal{F}$ satisfies 
\index{envelope function}
\begin{align*}
|f(z)|
\ 
\le
\ 
F(z)< \infty 
\qquad
\text{for all}\ 
f\in\mathcal{F}
\ \text{and all}\ 
z\in\mathcal{Z}
\,.
\end{align*}


\subsection{Bracketing Numbers and Integral}
  To control empirical processes - apart from strong theorems - we need the notion of bracketing number and integral (see \cite[page 270]{Vaart2000}). 
Given two functions $\underline{f}\le \overline{f}$,
\begin{gather*}
  \text{
the bracket
  }\quad
[\underline{f},\overline{f}]
\quad 
\text{
is the set of all functions $f$ with 
}\quad 
\underline{f}\ \le\ f \ \le\  \overline{f}
\,.
\end{gather*}
For $\varepsilon>0$
we define a
\begin{gather*}
  \text{
$(\varepsilon, L^{r}(\P))$ -bracket
to be a bracket
  }
  \quad
[\underline{f},\overline{f}]
\quad
\text{with}
\quad
\norm{\overline{f}-\underline{f}}_{ L^r(\P)}
\ <\  \varepsilon
\,.
\end{gather*}
The \textbf{
bracketing number
} 
$
N_{[\,]}(\varepsilon, \mathcal{F}, L^r(\P))
$
is 
the minimum number of 
$(\varepsilon, L^{r}(\P))$-brackets needed to cover $\mathcal{F}$.

For most classes $\mathcal{F}$ the bracketing number grows to infinity for $\varepsilon\to 0$.
To measure the speed of growth we introduce 
for $\delta>0$
the
\textbf{bracketing integral}
\begin{gather*}
     J
    _{[\,]}
    (
    \delta
    ,
    \mathcal{F}
    ,
    L_r(\P)
    )
    \ 
    =
    \ 
  \int_0^{\delta}
      \sqrt{
        \log 
      N_{[\,]}
\left( \varepsilon, \mathcal{F}_N, L^r(\P) \right)
    }
    \,
    d\varepsilon
    \,.
\end{gather*}

Next we give a technical lemma to 
bound the bracketing numbers of products of two function classes, that is,
\begin{gather*}
  \mathcal{F}\cdot \mathcal{G}
  \ 
  :=
  \ 
  \left\{ 
    f\cdot g
    \ 
    \colon
  \ 
    f\in\mathcal{F},
    g\in\mathcal{G}
  \right\}\,.
\end{gather*}
\begin{lemma}
  \label{lem_prod_br}
  Let
  $\mathcal{F}$ and $\mathcal{G}$ be two function classes 
  with envelope functions $F$ and $G$ satisfying
  $\norm{F}_\infty,\norm{G}_\infty\le 1$.
  For all $\varepsilon>0$ and all $r\in [1,\infty)$ it holds
  \begin{gather*}
    N_{[\,]}(2\varepsilon,\mathcal{F}\cdot\mathcal{G},\mathrm{L}_r(\P))
    \
    \le
    \ 
    N_{[\,]}(\varepsilon,\mathcal{F},\mathrm{L}_r(\P))
    \cdot
    N_{[\,]}(\varepsilon,\mathcal{G},\mathrm{L}_r(\P))
    \,.
  \end{gather*}
\end{lemma}
\begin{proof}
  The proof is simple. We omit the details.
%  Let $f\in\mathcal{F}$ and $g\in\mathcal{G}$.
%  We can choose two 
%  $(\varepsilon,L^r(\P))$
%  brackets
%  $[\underline{f},\overline{f}]$
%  and
%  $[\underline{g},\overline{g}]$
%  containing $f$ and $g$ with 
%  $\norm{\underline{f}}_\infty,\norm{\overline{f}}_\infty\le\norm{F}_\infty\le 1$
%  and
%  $\norm{\underline{g}}_\infty,\norm{\overline{g}}_\infty\le\norm{G}_\infty\le 1$.
%  We the get an 
%  $(2\varepsilon,L^r(\P))$
%  $[\underline{h},\overline{h}]$
%  bracket, containing $f\cdot g$, by
\end{proof}


\subsection{Maximal Inequality}
  In our application we need concentration inequalities for 
$
  \norm{\G_n}^*_\mathcal{F}
$.
One easy way to obtain this is, to use a maximal inequality (see Theorem~\ref{th:max_ineq}) to control the expectation,
together with Markov's inequality. There are also Bernstein-like inequalities for empirical processes (see \cite[§2.14.2]{vaart2013}). 
\begin{theorem}
  \label{th:max_ineq}
  \emph{(Maximal inequality)}
  For any class $\mathcal{F}$ of measurable functions with envelope function $F,$
  \begin{gather*}
    \E^*
    \norm{
      \G
      _n
      }
      _\mathcal{F}
    \ 
    \lesssim
    \ 
    J
    _{[\,]}
    (
    \norm{
      F
    }
    _{ L^2(\P)}
    ,
    \mathcal{F}
    ,
    L^2(\P)
    )
    \, 
    .
  \end{gather*}
\end{theorem}
\begin{proof}
  \cite[Corollary~19.35]{Vaart2000}
\end{proof}

\begin{lemma}
  \label{markov_max_lemma}
  Let $(\mathcal{H}_N)$ be a sequence of measurable function classes with envelope functions $(H_N)$.
  If
  \begin{gather*}
    J_{[\, ]}
    \left( 
    \norm{H_N}_{ L^2(\P)}
    ,
    \mathcal{H}_N
    ,
     L^2(\P)
    \right)
    \ 
    \to
    \ 
    0
    \qquad
    \text{for}
    \ 
    N
    \to
    \infty
    \,,
  \end{gather*}
  it holds 
  $
  \norm{\G_N}^*_{\mathcal{H}_N}\overset{\P}{\to}0
  $.
\end{lemma}
\begin{proof}
  By Markov's inequality and Theorem~\ref{th:max_ineq} it holds for all $\varepsilon>0$
  \begin{align*}
    \P
    [
  \norm{\G_N}^*_{\mathcal{H}_N}
  \ge
  \varepsilon
    ]
    &
    \ 
    \le
    \ 
    \varepsilon^{-1}
    \E
    [
  \norm{\G_N}^*_{\mathcal{H}_N}
    ]
    \ 
    =
    \ 
    \varepsilon^{-1}
    \E^*
    [
  \norm{\G_N}_{\mathcal{H}_N}
    ]
    \\
    &
    \ 
    \lesssim
    \ 
    \varepsilon^{-1}
    J_{[\, ]}
    \left( 
    \norm{H_N}_{ L^2(\P)}
    ,
    \mathcal{H}_N
    ,
     L^2(\P)
    \right)
    \\
    &
    \ 
    \to
    \ 
    0
    \qquad
    \text{for}
    \ 
    N\to\infty
    \,.
  \end{align*}
\end{proof}

\begin{lemma}
  \label{aa:r3:lemma:1}
  Let
$(\varepsilon_N)\subset(0,1]$
be 
a decreasing sequence
with $\varepsilon_N\to 0$ for $N\to\infty$ and
$(\mathcal{F}_N)$
a sequence of (measurable) function classes
with envelope functions
$(F_N)$,
satisfying 
for some $k<2$
\begin{gather*}
\norm{F_N}_{L^2(\P)}
\ 
\le
\ 
\varepsilon_N
\quad
\text{and}
\quad
  \log
  N_{[\,]}(\varepsilon,\mathcal{F}_N,\mathrm{L}_2(\P_{X}))
  \ 
  \lesssim
  \ 
  \left( 
  \frac{1}{\varepsilon}
  \right)^k
  \quad
  \text{for all}
  \ 
  N\in\mathbb{N}
  \,.
\end{gather*}
Then
\begin{gather*}
  J_{[\,]}(
\norm{F_N}_{L^2(\P)}
,\mathcal{F}_N\cdot\mathcal{F},\mathrm{L}_2(\P))
  \to 0
  \quad
  \text{and}
  \quad
  \norm{\G_N}^*_{\mathcal{F}_N\cdot\mathcal{F}}\overset{\P}{\to}0
  \qquad
  \text{for}\ 
  N\to\infty
  \,,
\end{gather*}
where $\mathcal{F}$ is defined in \eqref{F_g}.
\end{lemma}
\begin{proof}
  By assumption
  and Lemma~\ref{aa:mean:l:br} it holds
for some $k<2$
\begin{gather*}
\norm{F_N}_{L^2(\P)}
\ 
\le
\ 
\varepsilon_N
\quad
\text{and}
\quad
  \log
  N_{[\,]}(\varepsilon,\mathcal{F}_N,\mathrm{L}_2(\P))
  \ 
  \lesssim
  \ 
  \left( 
  \frac{1}{\varepsilon}
  \right)^k
  \quad
  \text{for all}
  \ 
  N\in\mathbb{N}
  \,,
\end{gather*}
and
  \begin{gather*}
    N_{[\,]}
    (
    \varepsilon
    ,
    \mathcal{F}, L^2(\P))
    \ 
    \lesssim
    \ 
    \left( 
      \frac{1}{\varepsilon}
    \right)^2
    \qquad
    \text{for all}
    \ 
    \varepsilon>0
    \,.
  \end{gather*}
  Since $\mathcal{F}_N$ and $\mathcal{F}$ have envelope functions smaller 1, we can apply Lemma~\ref{lem_prod_br} to get
  \begin{gather*}
  \log
  N_{[\,]}(\varepsilon,\mathcal{F}_N\cdot\mathcal{F},\mathrm{L}_2(\P))
  \ 
  \lesssim
  \ 
  \left( 
  \frac{1}{\varepsilon}
  \right)^k
  +
  \log
  (1/\varepsilon)
  \ 
  \lesssim
  \ 
  \left( 
  \frac{1}{\varepsilon}
  \right)^k
  \quad
  \text{for all}\ 
  \varepsilon>0
  \,.
  \end{gather*}
  Since 
  $k/2\in(0,1)$
  it holds
\begin{align*}
  J_{[\,]}(
\norm{F_N}_{L^2(\P)}
,\mathcal{F}_N\cdot\mathcal{F},\mathrm{L}_2(\P))
  &
  \ 
=
  \ 
\int_0^{
\norm{F_N}_{L^2(\P)}
}
\sqrt{
  \log
  N_{[\,]}(\varepsilon,\mathcal{F}_N\cdot\mathcal{F},\mathrm{L}_2(\P))
}
\,d\varepsilon
\\
&
\ 
\lesssim
\ 
\int_0^{
  \varepsilon_N
}
  \left( 
  \frac{1}{\varepsilon}
\right)^{k/2}
\,d\varepsilon
\\
&
\ 
=
\ 
\frac{
\varepsilon_N^{1-k/2}
}{1-k/2}
\ 
\to 0
\ 
\qquad
\text{for}
\ 
N\to\infty
\,.
\end{align*}
The second statement follows from Lemma~\ref{markov_max_lemma}
for 
$\mathcal{H}_N:=\mathcal{F}_N\cdot\mathcal{F}$ and
$H_N:=F_N$.
\end{proof}





\subsection{Donsker's Theorem}
  There is a powerful theorem --- a central limit theorem for $\G_N$ uniform in $\mathcal{F}$ --- that we now introduce.
\begin{definition}
  We call a class 
  $\mathcal{F}$ of measurable functions 
$\P$-Donsker
if the sequence of processes 
$\left\{ \G_N f \colon f\in\mathcal{F}\right\}$
converges in
$l^\infty(\mathcal{F})$
to a tight limit process.
\end{definition}

\begin{theorem}
  \label{th:donsker}
  Every class $\mathcal{F}$ of measurable functions 
  with
  \begin{gather*}
    J
    _{[\,]}
    (
    1
    ,
    \mathcal{F}
    ,
    L_2(\P)
    )
    <\infty
  \end{gather*}
  is
  $\P$-Donsker.
  Furthermore,
  the sequence of processes 
$\left\{ \G_N f \colon f\in\mathcal{F}\right\}$
  converges 
  in
$l^\infty(\mathcal{F})$
to a Gaussian process with mean 0 and covariance function given by
\begin{gather*}
  \mathbf{Cov}(f,g)
  \ 
  :=
  \ 
  \E[fg]
  \ 
  -
  \ 
  \E[f]\E[g]
  \,.
\end{gather*}
\end{theorem}
\begin{proof}
  \cite[Theorem~19.5]{Vaart2000}
\end{proof}
\begin{lemma}
  \label{lem:G_P_donsker}
  Let Assumption~\eqref{asu:treatment_asign_str_ing} and Assumption~\ref{asu:x_finite} hold true.
Then the function class $\mathcal{G}$ defined in \eqref{F_g} is $\P$-Donsker. 
\end{lemma}
\begin{proof}
  By Theorem~\ref{th:donsker} it suffices to show that the bracketing integral is finite.
Note that by Assumption~\ref{asu:x_finite} ($\mathcal{X}$ is finite) and Assumption~\eqref{asu:treatment_asign_str_ing} ($\pi(X)>0$) it holds $1/\pi(X)\in L^2(\P)$.
  Thus,
  by Lemma~\ref{aa:mean:l:br}
  it holds
  \begin{align*}
    &
  \log
  N_{[\,]}
    (
    \varepsilon
    ,
    \mathcal{G}, L^2(\P))
    \\
    &
    \ 
    \lesssim
    \ 
    \log
    \left(
      \frac
      {
      1+
    \norm{1/\pi(X)}_{ L^2(\P)}
      }
      {\varepsilon}
    \right)
    \ 
    \lesssim
    \ 
      \frac
      {
      1+
    \norm{1/\pi(X)}_{ L^2(\P)}
      }
      {\varepsilon}
    \qquad
    \text{for all}
    \ 
    \varepsilon\in (0,1)
    \,.
  \end{align*}
  Thus
  \begin{gather*}
    J_{[\,]}(1,\mathcal{G},L^2(\P))
    \ 
    \lesssim
    \ 
    \int_0^1
    \sqrt
    {
      \frac
      {
      1+
    \norm{1/\pi(X)}_{ L^2(\P)}
      }
      {\varepsilon}
    }
    \,
    d\varepsilon
    \ 
    \lesssim
    \ 
      1+
    \norm{1/\pi(X)}_{ L^2(\P)}
    \ 
    <
    \ 
    \infty
    \,.
  \end{gather*}
But then $\mathcal{G}$ is $\P$-Donsker.

\end{proof}

\subsection{Propensity Score Weights}
  The next lemma shows what effect the 
\textbf{propensity score weights}
$T/\pi(X)$ have on other functions.
\begin{lemma}
  \label{lem:ps_weights}
  \label{ps_weights_lemma}
  Let
  $
  g_1\colon
  \mathcal{X}\to\R
  $
  and
  $
  g_2\colon
  \mathcal{Y}\to\R
  $
  be a measurable functions.
  \begin{enumerate}[label=(\roman*)]
    \item
  It holds
  \begin{gather*}
    \E
    \left[
    \frac{T}{\pi(X)}
    g_1(X)
    \right]
    \ 
    =
    \ 
    \E
    \left[
    g_1(X)
    \right]
    \,.
  \end{gather*}
  \item
 If Assumption~\ref{aa:assumption:treatment_str_ign} holds true, then
  \begin{gather*}
    \E
    \left[
    \frac{T}{\pi(X)}
    g_2(Y(T))
    \right]
    \ 
    =
    \ 
    \E
    \left[
    f(Y(1))
    \right]
    \,.
  \end{gather*}
  \end{enumerate}
 \end{lemma}




\section{Main Result}
  Before we state the main result we collect all assumptions. Note that in the proofs we refer to the assumptions by their initial location, for example, Assumption~\ref{main_asu}.\textit{(iv)} is Assumption~\ref{asu:feas_dual_sol}.
\begin{assumption}
  \label{main_asu}
  \begin{enumerate}[label=(\roman*)]
 \item
$\sqrt{N}\norm{\delta}_1\overset{\P}{\to}0$ for $N\to\infty$.
\item
  \begin{align*}
    \sqrt{N}
    \sup_{z\in\R}
    \omega
    \left( 
      F_{Y(1)}(z|\cdot)
      ,h_N^d
    \right)
    \ 
    \to 
    \ 
    0
    \qquad
    \text{for}\ 
    N\to\infty
    \,.
  \end{align*}
  \item
\begin{gather*}
  (Y(0),Y(1))\ \perp \ T \,|\,X
  \quad
  \text{and}
  \quad
  0<\pi(X)<1
  \,,
\end{gather*}
\item
  For all $N\in\mathbb{N}$ there exists a non-empty, compact, and deterministic 
  parameter space 
  $
  \Theta_N
  \subset
  \R^{N}_{\ge 0}
  \times
  \R
  \times
  \R^N
  $
  such that the optimal solution 
  $
  \left( \rho^\dagger,\lambda_0^\dagger,\lambda^\dagger \right)
  $
  of Problem~\ref{dual}
  are contained in $\Theta_N$.
     \item 
       $\# J_N\le \# \mathcal{X}<\infty$ for all $N\in\mathbb{N}$, where 
       $
   J_N
   :=
   \left\{ j\in\mathbb{N}\colon
     \P[X\in A_{n,j}]>0
   \right\}
       $
\item
  For all $N\in\mathbb{N}$ there exist $(M_{N,j})_{j\in J_N}$ such that $\infty>M_{N,j}\ge 0$ for all $j\in J_n$, and 
  $\frac{1}{\pi(\cdot)}\in C^\alpha_{M_{N,j}}(\mathrm{cl}\,A_{N,j})$ for all $(j,N)\in J_N\times \mathbb{N}$, with $\alpha>d/2$.
 \end{enumerate} 
\end{assumption}
\begin{ftheorem}
  \label{th:main}
  Let Assumption~\ref{main_asu} hold true.
  Then the stochastic process
\begin{gather}
    \sqrt{N}
    \left( 
  \frac{1}{N}
    \sum_{i=1}^{N} 
    T_i
    \cdot
    w_0^\dagger(X_i)
    \cdot
    \mathbf{1}
    \left\{ Y_i\,\le\, z \right\}
    \ 
    -
    \ 
    F_{Y(1)}(z)
    \right)
    _{z\in\R}
    \,
  \end{gather}
  converges in
  $l^\infty(\R)$
  to a Gaussian process with mean 0 and covariance function
  satisfying for all $z_1,z_2\in\R$
\begin{align}
  \label{cov:lp}
 \begin{split}
  &
  \mathbf{Cov}
  (z_1,z_2)
  \\
  &
  =\ 
  \E
  \left[ 
 \frac{
 F_{Y(1)}(z_1 \land z_2\,|\,X)
}{\pi(X)}
\ 
-
\ 
 \frac{1-\pi(X)}{\pi(X)}
 F_{Y(1)}(z_1|X)
 \cdot
 F_{Y(1)}(z_2|X)
  \right]
  \\
  &
  \qquad 
 -
 \ 
 F_{Y(1)}(z_1)
 \cdot
 F_{Y(1)}(z_2)
 \,.
 \end{split}
\end{align}
\end{ftheorem}



\section{Error Decomposition}
  \begin{lemma}
  \label{aa:mean:lemma_decomp}
  It holds
  \begin{gather}
    \sqrt{N}
    \left( 
  \frac{1}{N}
    \sum_{i=1}^{N} 
    T_i
    \cdot
    w_0^\dagger(X_i)
    \cdot
    \mathbf{1}
    \left\{ Y_i\,\le\, z \right\}
    \ 
    -
    \ 
    F_{Y(1)}(z)
    \right)
    _{z\in\R}
    \ 
    =
    \ 
    R_1
    \ 
    +
    \ 
    R_2
    \ 
    +
    \ 
    R_3
    \ 
    +
    \ 
    R_4
  \end{gather}
  with
\begin{align*}
  R_1
  &
  \ 
  :=
  \ 
  \sqrt{N}
  \sum_{k=1}^{N} 
  \left[ 
  \frac{1}{N}
  \left( 
    \sum_{i=1}^{N} 
    T_i
    \cdot
    w_0^\dagger(X_i)
    \cdot
    B_k(X_i)
    \ 
    -
    \ 
    \sum_{i=1}^{N} 
    B_k(X_i)
  \right)
  \cdot
  F_{Y(1)}(z|X_k)
  \right]
  _{z\in\R}
  \,,
  %%%% 1 %%%%
  \\
  R_2
  &
  \
  :=
  \ 
  \sqrt{N}
    \sum_{i=1}^{N} 
    \frac{1}{N}
    \left[ 
      \left( 
    T_i\cdot w_0^\dagger(X_i) 
    \ 
    -
    \ 
    1 
      \right)
    \left( 
  F_{Y(1)}(z|X_i)
    \ 
    -
    \ 
    \sum_{k=1}^{N} 
    B_k(X_i)
    \cdot
  F_{Y(1)}(z|X_k)
    \right)
    \right]
  _{z\in\R}
  \,,
  %%%  %%%%%%%%%%m
  \\
  R_3
  &
  \
  :=
  \ 
  \sqrt{N}
  \left( 
  \frac{1}{N}
    \sum_{i=1}^{N} 
    \left[ 
    T_i
    \cdot
    \left( 
    w^\dagger_0(X_i) 
    \ 
    -
    \ 
    \frac{1}{\pi(X_i)}
    \right)
    \cdot
    \left( 
    \mathbf{1}{\left\{ Y_i \le z \right\}}
    \ 
    -
    \ 
  F_{Y(1)}(z|X_i)
    \right)
    \,
    \right]
  \right)
  _{z\in\R}
  \,,
  %%%% 3 %%%%
  \\
  R_4
  &
  \
  :=
  \ 
  \sqrt{N}
  \left( 
  \frac{1}{N}
    \sum_{i=1}^{N} 
    \frac{T_i}{\pi(X_i)}
    \left( 
    \mathbf{1}{\left\{ Y_i \le z \right\}}
    -
  F_{Y(1)}(z|X_i)
    \right)
    \ 
    +
    \ 
    \left( 
  F_{Y(1)}(z|X_i)
    -
  F_{Y(1)}(z)
    \right)
  \right)
  _{z\in\R}
  \,.
  \end{align*}
\end{lemma}
\begin{proof}
  We fix $z\in\R$.
  It holds
  \begin{align*}
    &
    \frac{1}{N}
    \sum_{i=1}^{N} 
    w_0^\dagger(X_i)
    \cdot
    T_i
    \cdot
    \mathbf{1}{\left\{ Y_i\, \le\, z \right\}}
    \\
    &
    \ 
    =
    \ 
    \frac{1}{N}
    \sum_{i=1}^{N} 
    \left( 
    w_0^\dagger(X_i)
    \ 
    -
    \ 
    \frac{1}{\pi(X_i)}
    \right)
    T_i
    \cdot
    \mathbf{1}{\left\{ Y_i\, \le\, z \right\}}
    \\
    &
    \quad 
    +
    \ 
    \frac{1}{N}
    \sum_{i=1}^{N} 
    \frac{T_i}{\pi(X_i)}
    \mathbf{1}{\left\{ Y_i\, \le\, z \right\}}
    \\
    &
    \ 
    =
    \ 
    \frac{1}{N}
    \sum_{i=1}^{N} 
    \left( 
    w_0^\dagger(X_i)
    -
    \frac{1}{\pi(X_i)}
    \right)
    T_i
    \left( 
    \mathbf{1}{\left\{ Y_i\, \le\, z \right\}}
    -
    F_{Y(1)}(z|X_i)
    \right)
    \\
    &
    \quad 
    +
    \ 
    \frac{1}{N}
    \sum_{i=1}^{N} 
    \frac{T_i}{\pi(X_i)}
    \left( 
    \mathbf{1}{\left\{ Y_i\, \le\, z \right\}}
    -
    F_{Y(1)}(z|X_i)
    \right)
    \\
    &
    \qquad 
    +
    \ 
    \frac{1}{N}
    \sum_{i=1}^{N} 
    w_0^\dagger(X_i)\cdot T_i\cdot
    F_{Y(1)}(z|X_i)
    \\
    &
    \ 
    =
    \ 
    R_3(z)
    /\sqrt{N}
    \\
    &
    \quad 
    +
    \ 
    \frac{1}{N}
    \sum_{i=1}^{N} 
    \frac{T_i}{\pi(X_i)}
    \left( 
    \mathbf{1}{\left\{ Y_i\, \le\, z \right\}}
    -
    F_{Y(1)}(z|X_i)
    \right)
    +
    \left( 
    F_{Y(1)}(z|X_i)
    -
    F_{Y(1)}(z)
    \right)
    \\
    &
    \qquad
    +
    \ 
    \frac{1}{N}
    \sum_{i=1}^{N} 
    \left( 
    w_0^\dagger(X_i)\cdot T_i
    \ 
    -
    \ 
    1
    \right)
    F_{Y(1)}(z|X_i)
    \\
    &
    \quad\qquad
    +
    \ 
    F_{Y(1)}(z)
    \\
    &
    \ 
    =
    \ 
    R_3(z)
    /\sqrt{N}
    \\
    &
    \quad
    +
    \ 
    R_4(z)/\sqrt{N}
    \\
    &
    \qquad
    +
    \ 
    \frac{1}{N}
    \sum_{i=1}^{N} 
    \left( 
    w_0^\dagger(X_i)\cdot T_i
    \ 
    -
    \ 
    1
    \right)
    \left( 
    F_{Y(1)}(z|X_i)
    -
    \sum_{k=1}^{N} 
    B_k(X_i)
    \cdot
  F_{Y(1)}(z|X_k)
    \right)
    \\
    &
    \quad\qquad
    +
    \ 
    \frac{1}{N}
    \sum_{i=1}^{N} 
    \left( 
    w_0^\dagger(X_i)\cdot T_i
    \ 
    -
    \ 
    1
    \right)
    \sum_{k=1}^{N} 
    B_k(X_i)
    \cdot
  F_{Y(1)}(z|X_k)
    \\
    &
    \qquad\qquad
    +
    \ 
    F_{Y(1)}(z)
\\
    &
    \ 
    =
    \ 
    (
    R_3(z)
    \ 
    +
    R_4(z)
    )
    /\sqrt{N}
    \\
    &
    \qquad
    +
    \ 
    R_2(z)/\sqrt{N}
    \\
    &
    \quad\qquad
    +
    \ 
    \sum_{k=1}^{N} 
    \frac{1}{N}
    \sum_{i=1}^{N} 
    \left( 
    w_0^\dagger(X_i)\cdot T_i
    B_k(X_i)
    \ 
    -
    \ 
    B_k(X_i)
    \right)
    \cdot
  F_{Y(1)}(z|X_k)
    \\
    &
    \qquad\qquad
    +
    \ 
    F_{Y(1)}(z)
    \\
    &
    \ 
    =
    \ 
    \left( 
R_3(z)
    \ 
    +
    \ 
    R_4(z)
    \ 
    +
    \ 
    R_2(z)
    \ 
    +
    \ 
    R_1(z)
    \right)
    /\sqrt{N}
    \ 
    +
    \ 
    F_{Y(1)}(z)
    \,.
  \end{align*}
\end{proof}


\section{Analysis of the Error Terms}
  \subsection{Analysis of $R_1$}
    \begin{lemma}
  \label{aa:mean:l:r1}
  Let
$\sqrt{N}\norm{\delta}_1\overset{\P}{\to}0$.
Then it holds
$\sup_{z\in\R}|R_1(z)|\overset{\P}{\to}0$.
  \end{lemma}
\begin{proof}
If $s_N\in \mathrm{int}\,\Theta_N$, then 
$
w_1^\dagger(X_1),\cdots,w_N^\dagger(X_N)
$
satisfy the box constraints of Problem~\ref{bw:1:primal} (in the form with the $T_i$ instead of $n$).
  \begin{align}
    \label{R_1:1}
    \begin{split}
    \sup_{z\in\R}
    \left| 
    R_1(z)
    \right|
    &
    \ 
    =
    \ 
  \sqrt{N}
  \sup_{z\in\R}
  \sum_{k=1}^{N} 
  \left[ 
  \frac{1}{N}
  \left( 
    \sum_{i=1}^{N} 
    T_i
    \cdot
    w^\dagger(X_i)
    \cdot
    B_k(X_i)
    \ 
    -
    \ 
    \sum_{i=1}^{N} 
    B_k(X_i)
  \right)
  \cdot
  F_{Y(1)}(z|X_k)
  \right]
    \\
    &
    \ 
    \le
    \ 
  \sqrt{N}
  \sum_{k=1}^{N} 
  \left| 
  \frac{1}{N}
  \left( 
    \sum_{i=1}^{N} 
    T_i
    \cdot
    w^\dagger(X_i)
    \cdot
    B_k(X_i)
    \ 
    -
    \ 
    \sum_{i=1}^{N} 
    B_k(X_i)
  \right)
  \right|
  \cdot
    \sup_{z\in\R}
  F_{Y(1)}(z|X_k)
  \\
    &
    \ 
    \le
    \ 
  \sqrt{N}
  \norm{\delta}_1
    \end{split}
  \end{align}
  The last inequality is due to $F_{Y(1)}\in[0,1]$ and the assumption that $(w_i^\dagger(X_i))$ satisfies the box constraints of Problem~\ref{bw:1:primal}.
  It remains to analyse the case $s_N\notin \mathrm{int}\,\Theta_N$.
  Then $(w_i^\dagger(X_i))$ are not the optimal weights of Problem~\ref{bw:1:primal}
  and we can't employ the box constraints to bound $R_1$.
  But we can use 
  $
  \P \left[ s_N\notin \Theta_N \right]
  \to 0
  $.
  To this end, note 
  \begin{align*}
    \P
    \left[ 
      \frac{
  \P \left[ s_N\notin \Theta_N \right]
      }{\sqrt{N}}
      \sum_{i=1}^{N} 
      \left( 
      \frac{T_i}{\pi(X_i)}
      -
      1
      \right)
      \ge \varepsilon
    \right]
    \le
  \P \left[ s_N\notin \Theta_N \right]
  \frac{\mathbf{Var}[T/\pi(X)]}{\varepsilon^2}
  \to 
  0
  \end{align*}
  for all $\varepsilon>0$.


  Since we assume 
$\sqrt{N}\norm{\delta}_1\overset{\P}{\to}0$
it holds
$\sup_{z\in\R}|R_1(z)|\overset{\P}{\to}0$.
\end{proof}
\begin{remark}
  We want to comment on the box constraints of Problem~\ref{bw:1:primal}, that is,
 \begin{gather*}
      \left| 
      \frac{1}{N} 
      \left( 
      \sum_{i = 1}^{n} 
      w^\dagger(X_i)
      B_k(X_i)
      -
      \sum_{i=1}^{N} 
      B_k(X_i)
      \right)
    \right|
    \ 
    \le 
    \ 
    \delta_k
    \qquad
    \text{for all}\ 
    k \in \left\{ 1, \ldots, N \right\}
    \,.
  \end{gather*}
  Note, that the first sum goes over $\left\{ 1,\ldots,n \right\}$ while the second sum goes over $\left\{ 1,\ldots,N \right\}$.
  A second, equivalent version of the constraints is
  \begin{gather*}
      \left| 
      \frac{1}{N} 
      \left( 
      \sum_{i = 1}^{N} 
      T_i
      w^\dagger(X_i)
      B_k(X_i)
      -
      \sum_{i=1}^{N} 
      B_k(X_i)
      \right)
    \right|
    \ 
    \le 
    \ 
    \delta_k
    \qquad
    \text{for all}\ 
    k \in \left\{ 1, \ldots, N \right\}
    \,.
  \end{gather*}
  Now both sums go over $\left\{ 1,\ldots,N \right\}$ and the
  indicator of treatment $T_i$ takes care that in the first sum only the terms with $i\le n$ are effective. 
  Having this flexibility with the versions helps. I regard the first version as suitable for non-probabilistic computations, although $n$ is of course a random variable. On the other hand, the second version is more honest, exactly telling the dependence on the indicator of treatment. This version is useful in probabilistic computations. 

  Also we want to comment on the assumption on $\norm{\delta}$.
  Playing around with norm equivalences we discover that 
  $\sqrt{N}\norm{\delta}_1\overset{\P}{\to}0$ for $N\to \infty$ is the weakest
  (natural) assumption to
  control $R_1$.
  Indeed, other ways to continue the second row in \eqref{R_1:1} are
  \begin{gather*}
    (\,\cdots)
    \ 
  \le
    \ 
  \sqrt{N}
  \norm{\delta}_2
  \left( 
  \sum_{k=1}^{N} 
  \left( 
    \sup_{z\in\R}
  F_{Y(1)}(z|X_k)
  \right)^2
\right)^{1/2}
\ 
\le
\ 
N
  \norm{\delta}_2\,,
  \end{gather*}
  by the Cauchy-Schwarz inequality and
  $
  F_{Y(1)}\in [0,1]
  $,
or
\begin{gather*}
  (\,\cdots)
  \ 
  \le
  \ 
  \sqrt{N}
  \norm{\delta}_\infty
  \sum_{k=1}^{N} 
    \sup_{z\in\R}
  F_{Y(1)}(z|X_k)
  \ 
  \le
  \ 
  N^{3/2}
  \norm{\delta}_\infty
  \,.
\end{gather*}
Since $\delta\in \R^N$, however, it holds
\begin{gather*}
  \sqrt{N}\norm{\delta}_1
  \ 
  \le
  \ 
  N\norm{\delta}_2
  \ 
  \le
  \ 
  N^{3/2}\norm{\delta}_\infty
  \,.
\end{gather*}
With hindsight, the assumption 
$\sqrt{N}\norm{\delta}_1\overset{\P}{\to}0$ for $N\to \infty$ 
  also 
  suffices 
  to control the second (or first) occurrence of a term, that we control by assumptions on $\norm{\delta}$.
This is the \textbf{second term} of \eqref{c:1}, where we estimate
\begin{gather*}
  \inner{\delta}{\left| \Delta \right|}
  \ 
  =
  \ 
  \sum_{k=1}^{N} 
  \delta_k
  \left| \Delta_k \right|
  \ 
  \le
  \ 
  \norm{\delta}_1
  \norm{\Delta}_\infty
  \ 
  \le
  \ 
  \norm{\delta}_1
  \norm{\Delta}_2
  \ 
  \le
  \ 
  \norm{\delta}_1
  \varepsilon
  \ 
  \overset{\P}{\to}
  \ 
  0
  \quad
  \text{for}\ 
  N\to \infty
  \,.
\end{gather*}

\end{remark}


  \subsection{Analysis of $R_2$}
    The convergence of this term is closely related to good approximation properties of $B$ (see Lemme~\ref{lem:basis_2}.\textit{(ii)}). 
\begin{lemma}
\label{aa:mean:l:r2}
  Assume
  \begin{align*}
    \sqrt{N}
    \sup_{z\in\R}
    \omega
    \left( 
      F_{Y(1)}(z|\cdot)
      ,h_N^d
    \right)
    \ 
    \to 
    \ 
    0
    \qquad
    \text{for}\ 
    N\to\infty
    \,.
  \end{align*}
  Then $\sup_{z\in\R}|R_2(z)|\overset{\P}{\to} 0$.
\end{lemma}
\begin{proof}
  \begin{align*}
    \sup_{z\in\R}
    \left| R_2(z) \right|
    &
    \
    =
    \
  \sqrt{N}
  \sup_{z\in\R}
  \left|
    \sum_{i=1}^{N} 
    \frac{1}{N}
    \left[ 
      \left( 
    T_i\cdot w_0^\dagger(X_i) 
    \ 
    -
    \ 
    1 
      \right)
    \left( 
  F_{Y(1)}(z|X_i)
    \ 
    -
    \ 
    \sum_{k=1}^{N} 
    B_k(X_i)
    \cdot
  F_{Y(1)}(z|X_k)
    \right)
    \right]
  \right|
\\
    &
    \  
    \le
    \  
        \sqrt{N}
        \sup_{z\in\R}
        \max_{i\in \left\{ 1,\ldots,N \right\}}
        \sum_{k=1}^{N}
            \left|
        B_k(X_i,X_1,\ldots,X_N)
        \cdot
        F_{Y(1)}(z|X_k)
            \ 
            -
            \ 
        F_{Y(1)}(z|X_i)
            \right|
            \\
            &
            \qquad
            \cdot
            \ 
    \frac{1}{N}
    \sum_{i=1}^{N} 
      \left| 
    T_i\cdot w^\dagger_0(X_i) 
    \ 
    -
    \ 
    1 
      \right|
  \end{align*}
  Note that by Theorem~\ref{th:weights_constr}.\textit{(i)-(ii)}
  it holds
  \begin{align*}
    \frac{1}{N}
    \sum_{i=1}^{N} 
      \left| 
    T_i\cdot w^\dagger_0(X_i) 
    \ 
    -
    \ 
    1 
      \right|
      \ 
    \le
      \ 
    1
    \ 
    +
    \ 
    \frac{1}{N}
    \sum_{i=1}^{N} 
    T_i\cdot w^\dagger_0(X_i) 
    \ 
    =
    \ 
    2
    \,.
  \end{align*}
  The statement follows from Lemma~\ref{lem:basis_2}.\textit{(ii)}
\end{proof}
\begin{remark}
In the original paper \cite{Wang2019} the authors derive concrete learning rates for the weights and employ them in bounding this term. They obtain a multiplied learning rate that is sufficiently fast. Their approach, however, calls for concrete learning rates of the weights. Arguably, the process of deriving such rates is the most complicated part of the paper. 
I found out that with the basis functions of partitioning estimates (or similar basis functions) we don't need concrete rates for the weights. 
Consistency of the weights is enough and gives us an (arbitrarily slow but sufficient) learning rate to establish the results.
We don't even need rates for the weights to control $R_2$.
They only play a role in bounding $R_3$. 
\end{remark}



  \subsection{Analysis of $R_3$}
    \begin{lemma}
  It holds
  \begin{align*}
    \E
    \left[ 
      T
    \left( 
    \mathbf{1}{\left\{ Y(T) \le z \right\}}
    \ 
    -
    \ 
  F_{Y(1)}(z|X)
    \right)
    \right]
    \ 
    =
    \ 
    0
    \qquad
    \text{for all}\ 
    z\in\R
    \,.
  \end{align*}
\end{lemma}
\begin{proof}
  Let $z\in\R$.
  By Lemma~\ref{lem:ps_weights}.\textit{(i)} 
  and the properties of conditional expectation
  it holds
  \begin{align*}
    \E
    \left[ 
      T
      \cdot
    \mathbf{1}{\left\{ Y(T) \le z \right\}}
    \right]
    \ 
    =
    \ 
    \E
    \left[ 
      \pi(X)
      \cdot
    \mathbf{1}{\left\{ Y(1) \le z \right\}}
    \right]
    \ 
    =
    \ 
    \E
    \left[ 
      \pi(X)
      \cdot
      F_{Y(1)}(z|X)
    \right]
    \,.
  \end{align*}
  Furthermore,
  by Lemma~\ref{lem:ps_weights}.\textit{(ii)} 
  it holds
  \begin{align*}
    \E
    \left[ 
      T
      \cdot
  F_{Y(1)}(z|X)
    \right]
    \ 
    =
    \ 
    \E
    \left[ 
      \pi(X)
      \cdot
  F_{Y(1)}(z|X)
    \right]
    \,.
  \end{align*}
  It follows the result.
\end{proof}

  \subsection{Analysis of $R_4$}
    \begin{lemma}
  \label{aa:mean:l:r4}
  Let
  Assumption~\eqref{asu:treatment_asign_str_ing} and Assumption~\ref{asu:x_finite} hold true.
  Then
  $R_4$ 
  converges in
  $l^\infty(\R)$
  to a Gaussian process with mean 0 and covariance
\begin{align*}
  &
  \mathbf{Cov}
  (z_1,z_2)
  \\
  &
  =\ 
  \E
  \left[ 
 \frac{
 F_{Y(1)}(z_1 \land z_2\,|\,X)
}{\pi(X)}
\ 
-
\ 
 \frac{1-\pi(X)}{\pi(X)}
 F_{Y(1)}(z_1|X)
 \cdot
 F_{Y(1)}(z_2|X)
  \right]
  \ 
 -
 \ 
 F_{Y(1)}(z_1)
 \cdot
 F_{Y(1)}(z_2)
\end{align*}

\end{lemma}
\begin{proof}
  By Lemma~\ref{lem:f_z} it holds
  \begin{align*}
    \E
    \left[
      \frac{f_z(T,X,Y(T))}{\pi(X)}
      +
      F_{Y(1)}(z|X)
      -
      F_{Y(1)}(z)
      \right]
      \ 
      =
      \ 
      \E
      \left[
      \frac{1}{\pi(X)}
      \E
      \left[
        f_z(T,X,Y(T))
        |X
      \right]
      \right]
      \ 
      =
      \ 
      0
      \,.
  \end{align*}
  Thus
  \begin{align*}
    R_4(z)
    &
  \
  =
  \ 
  \frac{1}{
  \sqrt{N}
  }
    \sum_{i=1}^{N} 
    \frac{T_i}{\pi(X_i)}
    \left( 
    \mathbf{1}{\left\{ Y_i \le z \right\}}
    -
  F_{Y(1)}(z|X_i)
    \right)
    \ 
    +
    \ 
    \left( 
  F_{Y(1)}(z|X_i)
    -
  F_{Y(1)}(z)
    \right)
    \\
    &
    \ 
  =
    \ 
  \frac{1}{
  \sqrt{N}
  }
    \sum_{i=1}^{N} 
      \frac{f_z(T_i,X_i,Y_i)}{\pi(X_i)}
      +
      \left( 
      F_{Y(1)}(z|X_i)
      -
      F_{Y(1)}(z)
      \right)
      \\
      &
      \ 
      =
      \ 
      \G_N 
      \left(
       \frac{f_z}{\pi(\cdot)}
      +
      F_{Y(1)}(z|\cdot)
      -
      F_{Y(1)}(z)
      \right)
      \,.
  \end{align*}
  By Assumption~\eqref{asu:treatment_asign_str_ing}, Assumption~\ref{asu:x_finite},
and
Lemma~\ref{lem:G_P_donsker}
the function class 
\begin{align*}
  \mathcal{G}
    &
    \ 
    :=
    \ 
    \left\{ 
      \frac{f_z}{\pi(\cdot)}
      +
      F_{Y(1)}(z|\cdot)
      -
      F_{Y(1)}(z)
      \ 
      \colon
      \ 
      z\in\R\ 
    \right\}
\end{align*}
is $\P$-Donsker
Thus, by Theorem~\ref{th:donsker},
the process $R_4$ converges in $l^\infty(\R)$ to a Gaussian process with mean 0.
It remains to calculate the covariance of the limiting process.
We write
\begin{align*}
  &
  \E
  \left[
  \left( 
  f^{z_1}_{1/\pi}
  +
  F_{Y(1)}(z_1|X)
  -
F_{Y(1)}(z_1)
  \right)
  \left( 
  f^{z_2}_{1/\pi}
  +
  F_{Y(1)}(z_2|X)
  -
F_{Y(1)}(z_2)
  \right)
  \right]
  \\
  &
  \ 
  =
  \ 
\E
\left[
  f^{z_1}_{1/\pi}
  \cdot
  f^{z_2}_{1/\pi}
\right]
\\
  &
  \quad
  +
  \ 
  \E
  \left[
  f^{z_1}_{1/\pi}
  \left( 
  F_{Y(1)}(z_2|X)
  -
F_{Y(1)}(z_2)
  \right)
  \right]
  \ 
  +
  \ 
  \E
  \left[
  f^{z_2}_{1/\pi}
  \left( 
  F_{Y(1)}(z_1|X)
  -
F_{Y(1)}(z_1)
  \right)
  \right]
  \\
  &
  \quad
  +
  \ 
  \E
  \left[
  \left( 
  F_{Y(1)}(z_1|X)
  -
F_{Y(1)}(z_1)
  \right)
  \left( 
  F_{Y(1)}(z_2|X)
  -
F_{Y(1)}(z_2)
  \right)
  \right]
  \\
  &
  \ 
  =:
  \ 
  C_0
  \quad 
  +
  \quad 
  C_1
  +
  C_2
  \quad 
  +
  \quad 
  C_3
  \,.
\end{align*}
It holds
by Assumption~\eqref{asu:treatment_asign_str_ing} and Lemma~\ref{lem:ps_weights}
\begin{align*}
  C_0 
  &
  \ 
  =
  \ 
\E
\left[
  f^{z_1}_{1/\pi}
  \cdot
  f^{z_2}_{1/\pi}
\right]
\\
&
\ 
=
\ 
\E
\left[
\frac{1}{\pi(X)}
\frac{T}{\pi(X)}
\left( 
\mathbf{1}{\left\{ Y(T)\,\le\, z_1 \right\}}
-
F_{Y(1)}(z_1|X)
\right)
\left( 
\mathbf{1}{\left\{ Y(T)\,\le\, z_2 \right\}}
-
F_{Y(1)}(z_2|X)
\right)
\right]
\\
&
\ 
=
\ 
\E
\left[
\frac{1}{\pi(X)}
\left( 
\mathbf{1}{\left\{ Y(1)\,\le\, z_1 \right\}}
-
F_{Y(1)}(z_1|X)
\right)
\left( 
\mathbf{1}{\left\{ Y(1)\,\le\, z_2 \right\}}
-
F_{Y(1)}(z_2|X)
\right)
\right]
\\
&
\ 
=
\ 
\E
\left[
\frac{1}{\pi(X)}
\left( 
F_{Y(1)}(z_1\land z_2|X)
\ 
-
\ 
F_{Y(1)}(z_1|X)
\cdot
F_{Y(1)}(z_2|X)
\right)
\right]
\,,
\end{align*}
and
\begin{align*}
  C_1
  &
  \ 
  =
  \ 
 \E
  \left[
  f^{z_1}_{1/\pi}
  \left( 
  F_{Y(1)}(z_2|X)
  -
F_{Y(1)}(z_2)
  \right)
  \right]
  \\
  &
  \ 
  =
  \ 
 \E
  \left[
\frac{T}{\pi(X)}
\left( 
\mathbf{1}{\left\{ Y(T)\,\le\, z_1 \right\}}
-
F_{Y(1)}(z_1|X)
\right)
  \left( 
  F_{Y(1)}(z_2|X)
  -
F_{Y(1)}(z_2)
  \right)
  \right]
  \\
  &
  \ 
  =
  \ 
 \E
  \left[
\left( 
\mathbf{1}{\left\{ Y(1)\,\le\, z_1 \right\}}
-
F_{Y(1)}(z_1|X)
\right)
  \left( 
  F_{Y(1)}(z_2|X)
  -
F_{Y(1)}(z_2)
  \right)
  \right]
  \\
  &
  \ 
  =
  \ 
  0
  \,.
\end{align*}
In the same way we see $C_2=0$. Finally,
\begin{align*}
  C_3
  &
  \ 
  =
  \ 
  \E
  \left[
  \left( 
  F_{Y(1)}(z_1|X)
  -
F_{Y(1)}(z_1)
  \right)
  \left( 
  F_{Y(1)}(z_2|X)
  -
F_{Y(1)}(z_2)
  \right)
  \right]
  \\
  &
  \ 
  =
  \ 
  \E
  \left[
  F_{Y(1)}(z_1|X)
  \cdot
  F_{Y(1)}(z_2|X)
  \right]
  \ 
  -
  \ 
  F_{Y(1)}(z_1)
  \cdot
  F_{Y(1)}(z_2)
  \,.
\end{align*}
Adding up the results gives us \eqref{cov:lp}.
\end{proof}



\chapter{Discussion and Outlook}
  \section{Discussion}
We start the discussion with an application example. We shall see that many more such examples exist.
\subsection{Application to Nelson Aalen Estimator}
  We follow \cite[Example~3.9.19]{vaart2013}
Let 
$Z_1,\ldots,Z_N$
and
$C_1,\ldots,C_N$ 
be independent and identically distributed failure and censoring  times.
Failure and censoring times are assumed independent, that is, 
\begin{align*}
  Z_i \perp C_i
  \qquad
  \text{for all}\ 
  i\in \left\{ 1,\ldots,N \right\}
  \,.
\end{align*}
We only observe the outcome
\begin{align*}
  Y_i
  \ 
  :=
  \ 
  \left( 
  Z_i
  \land
  C_i
  ,
  \Delta_i
  \right)
  \ 
  \qquad
  \text{for all}\ 
  i\in \left\{ 1,\ldots,N \right\}
  \,,
\end{align*}
where 
$
\Delta_i:=
\mathbf{1}
\left\{ 
  Z_i\le C_i
\right\}
$
indicates whether a failure time is censored. 
We consider the weighted Nelson-Aalen estimator for the treated.
\begin{align*}
  \Lambda^1_N(t)
  \ 
  :=
  \ 
  \sum_{i=1}^{N} \frac{
    T_i\cdot w^\dagger_0(X_i)\cdot\mathbf{1}\left\{ Y_i\le t \right\}
    \cdot \Delta_i
  }{
    \sum_{j=1}^{N} 
    T_j\cdot w^\dagger_0(X_j)\cdot\mathbf{1}\left\{ Y_j\ge Y_i \right\}
  } 
  \,.
\end{align*}
Likewise, we can compute weights for the untreated (just switch the treatment status) and get the weighted Nelson-Aalen estimator of the untreated.
This procedure allows to compare treatment and control group while adjusting for imbalances.
This may be an appealing alternative to semi-parametric adjusted survival analysis methods, such as conditional cox regression.
The theoretical properties of the Nelson-Aalen estimator as a plug-in estimator are studied in \cite[Example~3.9.19]{vaart2013}.

\subsection{Summary of Assumptions}
  Next, we gather all assumptions.

\section{Outlook}
\subsection{Matching}
  There is a similar paper \cite{Wang2023}.


\subsection{Application of the Functional Delta Method}
  A plethora of applications of the delta method to estimates of the distribution function are to be found in \cite{Vaart2000} and \cite{vaart2013}.
This includes Quantile estimation \cite[§21]{Vaart2000}\cite[§3.9.21/24]{vaart2013},
survival analysis via Nelson-Aalen and Kaplan-Meier estimator\cite[§3.9.19/31]{vaart2013},
Wilcoxon Test~\cite[§3.9.4.1]{vaart2013},
and much more.
Maybe Boostrapping from the weighted distribution is also sensible .

\subsection{Bootstrapping} 
  \subsubsection{Motivation}
A very natural idea is to bootstrap from the weighted distribution
$
(w_i\cdot X_i)
$.
I discussed this with Jose Zubizaretta, one of the authors of \cite{Wang2019, Wang2023}.
Jose told me that testing in practice showed promising results.
To the best of his and my knowledge the theoretical properties of this particular weighted bootstrap wait to be studied.
\subsubsection{Conjecture}
Results similar to \cite[Theorem~23.5]{Vaart2000} holds for the weighted bootstrap.
\subsubsection{Ideas/Brainstorming}
A good starting point to become familiar with the asymptotic theory of bootstrap is \cite[§3.6]{vaart2013} and \cite[§23]{Vaart2000}.
For more details, a good starting point could be \cite{Barbe95}.
The project seems to be challenging - maybe at PHD level.
\subsubsection{Organisation}
Understand the mathematical theory of bootstrap.
Talk to Jose Zubizaretta about practical results and possible collaboration.
Develop ideas based on the existing literature.
\subsubsection{Next Step}
Get acquainted with the method of bootstrap by reading the (non-mathematical) introduction \cite{Efron1994}.

\subsection{Non-binary Treatment}
  \subsubsection{Motivation}
In practice, there often exists multiple treatments.
For example, $T\in \left\{
  0,1,2
\right\}$, $T\in I\subset \mathbb{N}$ or even $T\in\R$. 
There exists a general notion of propensity score \cite{Hirano2005}.
There is a need for methods covering this scenarios.
\subsubsection{Conjecture}
\subsubsection{Ideas/Brainstorming}
\subsubsection{Organisation}
\subsubsection{Next Step}

\subsection{Different Basis Functions}
  \subsubsection{Motivation}
\subsubsection{Conjecture}
\subsubsection{Ideas/Brainstorming}
\subsubsection{Organisation}
\subsubsection{Next Step}


\chapter{Convex Analysis}
In our application we want to analyse a convex optimization problem by its dual problem.
In particular we want to obtain primal optimal solutions from dual solutions.
To accomplish the task we need technical tools from convex analysis, 
mainly conjugate calculus and some Karush-Kuhn-Tucker related results.

Our starting point is 
the support function intersection rule~\cite[Theorem 4.23]{Mordukhovich2022}.
We give the details in the case of finite dimensions and refer for the rest of the proof to the book.
The support function intersection rule is applied to give first conjugate sum and then chain rule,
which are vital to calculating convex conjugates. The proofs are omited, since the book is thorough enough. 
%The well known Fenche-Rockafellar Duality theorem is a corollary of conjugate sum and chain rule. It gives general conditions under which dual and primal values coincide.
The material we present is very well known.
As an introduction, we recommend the recent book \cite{Mordukhovich2022} and the classical reference \cite{Rockafellar1970}.
We finish the chapter with ideas from \cite{Tseng1991}. 
They provide the high-level ideas to obtain for strictly convex
functions a dual relationship between optimal solutions.
We will deliver the details that are omited in the paper.
\section{A Convex Analysis Primer}
  \subsection*{My Contribution}
I present the relevant facts from Convex analysis.
I prove some results that I did not find in the literature, but likely are folklore.

Throughout this section let $n\in\mathbb{N}$.
\subsection*{Sets}
A subset $C\subseteq \R^n$ is called \textbf{convex set}, 
if for all $x,y\in C$ and all $\theta\in [0,1]$,
we have 
$
  \theta x + (1-\theta)y 
  \in
  C
$.
Many set operations preserve convexity. Among them
forming the 
\textbf{Cartesian product} of two convex sets, 
\textbf{intersection} of a collection of convex sets and 
taking the \textbf{inverse image under linear functions}.

The classical theory evolves around the question 
if convex sets can be separated.
\begin{definition*}
  Let 
  $C_1$ and $C_2$
  be two non-empty convex sets in $\R^n$. 
  A hyperplane $H$ is said to \textbf{separate}
  $C_1$ and $C_2$
  if $C_1$ is contained in one of the closed half-spaces associated with
  $H$ and $C_2$ lies in the opposite closed half-space. It is said to separate 
  $C_1$ and $C_2$
  \textbf{properly} if 
  $C_1$ and $C_2$
  are not both contained in $H$.
\end{definition*}

We need a refined concept of interiors, since some convex sets have empty interior. To this end, 
  we call a set
  \index{affine set}
  $A\subseteq \R^n$ 
  \textbf{affine set}, if
  $
    \alpha x + (1-\alpha)y \in A
    \quad
    \text{for all}
    \ 
    x,y \in A
    \ 
    \text{and all}
    \ 
    \alpha \in \R
  $.
  The \textbf{affine hull} 
  \index{$\mathrm{aff(\cdot)}$, affine hull}
  $\mathrm{aff}(\Omega)$
  of a set 
  $\Omega\subseteq \R^n$
  is the smallest affine set that includes $\Omega$.
  We define the \textbf{relative interior}
  \index{$\mathrm{ri}(\cdot)$, relative interior}
  $\mathrm{ri}\,\Omega$ 
  of a set 
  $\Omega\subseteq \R^n$
  to be the interior relative to the affine hull, that is,
    \begin{gather}
    \mathrm{ri}(\Omega)
    \ 
    :=
    \ 
    \left\{ 
      x \in \Omega 
      \ 
      |
      \ 
      \exists
      \,
      \varepsilon > 0\ 
      \colon
      (
      x+\varepsilon B_{\R^n}
      )
      \cap
      \mathrm{aff}(\Omega)
      \ 
      \subset
      \ 
      \Omega
      \,
    \right\}
    \,.
  \end{gather}

\begin{ftheorem}
  \label{cv:primer:sep}
  \emph{(Convex separation in finite dimension)}
  Let $C_1$ and $C_2$ be two non-empty convex sets in $\R^n$. 
  Then $C_1$ and $C_2$ can be properly separated if and only if 
  $\mathrm{ri}(C_1)\cap\mathrm{ri}(C_2)=\emptyset.$
\end{ftheorem}
\begin{proof}
  \cite[Theorem~11.3]{Rockafellar1970}
\end{proof}
We collect some useful 
properties of relative interiors
before we get on to convex functions.
\begin{proposition}
  \label{cv:primer:prop}
  Let $C$ be a non-empty convex set in $\R^n.$ The following holds:
\begin{enumerate}[label={(\roman*)}]
  \item
    $
      \mathrm{ri}(C)
      \ 
      \neq
      \ 
      \emptyset
      $
if and only if
      $
      C
      \ 
      \neq
      \ 
      \emptyset
    $
  \item
    $
      \mathrm{cl}(\mathrm{ri}\,C)
      \ 
      =
      \ 
      \mathrm{cl}\,C
      $
      and
      $
      \mathrm{ri}(\mathrm{cl}\,C)
      \ 
      =
      \ 
      \mathrm{ri}(C)
    $
  \item
    $
    \mathrm{ri}(C)
      \ 
    =
      \ 
    \left\{ 
      z \in C
      \colon
      \text{for all}\ 
      x \in C \ 
      \text{there exists}\ 
      t > 0 \ 
      \text{such that}\ 
      z + t (z-x)
      \in C
    \right\}
    $
  \item
    Suppose
    $
      \bigcap_{i\in I} C_i
      \ 
      \neq
      \ 
      \emptyset
    $
    for a finite index set $I$.
    Then
    $
      \mathrm{ri}
      \left( 
        \bigcap_{i\in I} C_i
      \right)
      \ 
      =
      \ 
      \bigcap_{i\in I}  
      \mathrm{ri}(C_i)
    $.
    \item
      Let 
      $
        L\,:\,\R^n \to\  \R^m
      $
      be a linear function. Then
      $
        \mathrm{ri}\,L(C)
        \ 
        =
        \ 
        L(\mathrm{ri}\,C)
      $.
      If it also holds
      $
        L^{\!-1}(\mathrm{ri}\,C)
        \ 
        \neq
        \ 
        \emptyset
      $,
      we have
      $
      \mathrm{ri}\,L^{\!-1}(C)
      \ 
        =
      \ 
        L^{\!-1}(\mathrm{ri}\,C)
      $.
      \item
        $
          \mathrm{ri}(C_1\!\times C_2)
          \ 
          = 
          \ 
          \mathrm{ri}\,C_1
          \! 
          \times
          \mathrm{ri}\,C_2
        $
\end{enumerate}

\end{proposition}


\begin{proof}
  For a proof of (i)-(v) we refer to~\cite[Theorem 6.2 - 6.7]{Rockafellar1970}.

To prove (vi) we use (iii).
Let
  $
  (z_1, z_2)
  \in 
  \mathrm{ri}(C_1\!\times C_2).
  $
  Then for all 
  $
  (x_1, x_2)
  \in 
  C_1\!\times C_2
  $
  there exists
  $t>0$
  such that
  \begin{gather}
    \label{cv:primer:prop:1}
      z_i + t (z_i-x_i)
      \in C_i
      \qquad
      \text{for all}\ 
      i\in \left\{ 1,2 \right\}.
  \end{gather}
  Using (iii) again, we get
  $
  \mathrm{ri}(C_1\!\times C_2)
  \, 
  \subseteq
  \,
          \mathrm{ri}\,C_1
          \! 
          \times
          \mathrm{ri}\,C_2
  $.
  Suppose 
  $
  (z_1,z_2)
    \in
    \mathrm{ri}\,C_1
    \!
    \times
    \mathrm{ri}\,C_2
  $.
  By (iii), for all
  $
    (x_1,x_2)\in C_1\times C_2
  $
  there exist
  $
    (t_1,t_2)>0
  $
  such that
  \begin{gather}
    \label{cv:primer:prop:2}
      z_i + t_i (z_i-x_i)
      \in C_i
      \qquad
      \text{for all}\ 
      i\in \left\{ 1,2 \right\}.
  \end{gather}
  If $t_1=t_2$
  we recover
  \eqref{cv:primer:prop:1}
  from
  \eqref{cv:primer:prop:2}.
  By (iii) it holds
  $
  (z_1,z_2)
    \in
    \mathrm{ri}
    (C_1
    \!
    \times
    C_2)
  $.
  If
  $t_1<t_2$
  we
  define $\theta:=\frac{t_1}{t_2}\in (0,1).$
  Consider  
  \eqref{cv:primer:prop:2} with $i=2$,
  together with $z_2 \in C_2$
  and
  the convexity of $C_2$.
  It follows
  \begin{gather}
    \label{cv:primer:prop:3}
    z_2 + t_1 (z_2 - x_2)
    \ 
    =
    \ 
    \theta
    \cdot
    (
    z_2 + t_2 (z_2 - x_2)
    )
    \ 
    +
    \ 
    (1-\theta)
    \cdot
    z_2
    \in C_2
    \,.
  \end{gather}
  Now we consider
  \eqref{cv:primer:prop:3} and
  \eqref{cv:primer:prop:2} with $i=1$.
  This gives \eqref{cv:primer:prop:1} with $t=t_1$.
  As before, it follows
  $
  (z_1,z_2)\in\mathrm{ri}(C_1\!\times C_2)
  $.
  If 
  $t_1>t_2$
  similar arguments lead to the same result.
  We have proven 
  $
  \mathrm{ri}(C_1\!\times C_2)
  \, 
  \supseteq
  \,
          \mathrm{ri}\,C_1
          \! 
          \times
          \mathrm{ri}\,C_2
  $
  and equality.
\end{proof}
\subsection*{Functions}
A function 
$
f
\colon
\R^n
\to
\overline{\R}
$
is called \textbf{convex function}, if the area above its graph, that is, its epigraph(cf.\cite[§2.4.1]{Mordukhovich2022}), is convex. We shall often use an equivalent definition.
To this end, 
a function $f$ is convex if and only if 
\begin{gather}
  \label{cv:cf}
  f(\theta x + (1-\theta)y)
  \ 
  \le
  \ 
  \theta f(x)
  +
  (1-\theta)f(y)
  \qquad
  \text{for all}\ 
  x,y\in \R^n
  \ 
  \text{and all}\ 
  \theta\in[0,1]
  \,.
\end{gather}
This definition extends to convex cominbinations
$
  \theta_1,\ldots,\theta_m\in[0,1]
$
with
$
  \sum_{i=1}^{m} 
  \theta_i
  =1
$, that is, 
a function $f$ is convex if and only if 
\begin{gather}
  f
  \left( 
    \sum_{i=1}^{m} 
    \theta_i
    x_i
  \right)
  \ 
  \le
  \ 
    \sum_{i=1}^{m} 
    \theta_i
    f(x_i)
  \qquad
  \text{for all}\ 
  x_1,\ldots,x_m\in \R^n
  \,.
\end{gather}
We call a function \textbf{strictly convex} if the inequality in
\eqref{cv:cf} is strict.

We define the \textbf{domain} $\mathrm{dom}\,f$
of a convex function $f$ to be the set where $f$ is finite, that is,
\begin{gather}
  \mathrm{dom}\,f
  \ 
  :=
  \ 
  \left\{ 
x\in\R^n
:
f(x)<\infty
  \right\}
  \,.
\end{gather}
The domain of a convex function is convex. 
We say that $f$ is a \textbf{proper function} if  $\mathrm{dom}\,f\neq\emptyset$. 

For any $\overline{x}\in\mathrm{dom}\,f$ we call $x^*\in\R^n$ a 
\textbf{subgradient} of $f$ at $\overline{x}$ if for all 
$x\in\R^n$ it holds
\begin{gather}
  \inner{x^*}{x-\overline{x}}
  \le
  f(x)
  -
  f(\overline{x})
  \,.
\end{gather}
We denote the collection of all subgradients at $\overline{x}$, that is, the \textbf{subdifferential} of $f$ at $\overline{x}$, as
$
\partial f(\overline{x})
$.
If $f$ is differentiable at $\overline{x}$ it holds
$
\partial f(\overline{x})
=
\left\{ 
  \nabla
  f(\overline{x})
\right\}
$
and thus
\begin{gather}
  \label{cv:primer:mvthe}
  \inner{
  \nabla
  f(\overline{x})
}{x-\overline{x}}
  \le
  f(x)
  -
  f(\overline{x})
  \,.
\end{gather}
%We call a differentiable function $f$ \textbf{strongly convex} with parameter
%$m>0$ if for all
%$x,y\in\mathrm{dom}\,f$ it holds
%\begin{gather}
%  f(y)-f(x)
%  \ 
%  \ge
%  \ 
%  \inner
%  {
%  \nabla
%  f(x)
%  }
%  {y-x}
%  \ 
%  +
%  \ 
%  \frac{m}{2}
%  \norm{y-x}^2
%  \,.
%\end{gather}
%If $f$ is twice continuously differentiable, then it is strongly 
%convex with parameter $m>0$ if and only if the matrix
%\begin{gather}
%\nabla^2f(x)
%-m\cdot\mathbf{I}
%\qquad
%\text{
%is positive semi-definite for all
%}\ 
%x\in\mathrm{dom}\,f
%\,,
%\end{gather}
% where 
%$
%\nabla^2f
%$
%is the Hessian Matrix.
%

\begin{definition*}
  Given a nonempty subset 
  $\Omega \subseteq \R^n$,
  we define
  the \textbf{support function} 
  $
  $
  of $\Omega$
  to be
  \begin{gather*}
  \sigma_\Omega 
  \,
  :
  \,
  \R^n \to\  \overline{\R}
  \,,
  \qquad
  x^*
  \ 
  \mapsto
  \ 
    \sup_{x \in \Omega}
    \ 
    \inner{x^*\!}{x}
    \,.
  \end{gather*}
\end{definition*}


\begin{definition}
  Given functions
  $
    f_i\,:\,
    \R^n \to\  \overline{\R}
  $
  for $ i = 1, \ldots, m $,
  we define the \textbf{infimal convolution} of these functions to be
  \begin{gather*}
    f_1 \square \cdots \square f_m
    \ 
    \colon
    \ 
    \R^n
    \to
    \ 
    \overline{\R}
    \,,
    \quad
    x
    \ 
    \mapsto
    \ 
    \inf
    \left\{ 
    \sum_{i = 1}^{m}
      f_i(x_i)
      \ 
      \colon
      \ 
      x_i \in \R^n 
      \ 
      \mathrm{and}\ 
      \sum_{i = 1}^{m} 
        x_i
      =
      x
    \right\}
    \,.
  \end{gather*}
\end{definition}
 
The next result establishes a connection between the support function of the intersection of two convex sets and the infimal convolution of the support functions of the sets taken by themselfes.
The proof translates the geometric concept of convex separation to the world of convex functions.

\begin{lemma}
  \label{cv:primer:lem}
  Let $C_1$ and $C_2$ be two non-empty convex sets in $\R^n$.
  For any
  $ x^* \in \mathrm{dom}\, \sigma_{C_1\cap C_2} $
  the sets
  \begin{align*}
    \Theta_1
    &
    \ :=\ 
    C_1 \times [\,0,\infty)
    \,,
    \\
    \Theta_2
    (x^*)
    &
    \ :=\ 
    \left\{ 
      (x,\lambda)\in \R^n
      \ 
      \colon
      \ 
      x \in C_2
      \ 
      \text{and}
      \ 
      \lambda
      \,
      \le
      \,
      \inner{x^*\!}{x} 
      \ 
      -
      \ 
      \sigma_{C_1\cap C_2}(x^*)
    \right\}
  \end{align*}
  can by properly separated.
\end{lemma}
\begin{proof}
  We fix 
  $ x^* \in \mathrm{dom}\, \sigma_{C_1\cap C_2} $
  and write
  $ 
  \alpha
  \ 
  :=
  \ 
  \sigma_{C_1\cap C_2}(x^*)
  $.
  In order to apply convex separation in finite dimension 
  (Theorem~\ref{cv:primer:sep})
  to
  the sets
  $ \Theta_1 $ and $ \Theta_2(x^*) $,
  it suffics to show
  their convexity and
  $
    \mathrm{ri}\, 
    \Theta_1
    \cap
    \mathrm{ri}\, 
    \Theta_2(x^*)
    =
    \emptyset
  $.
  \subsubsection*{Convexity of 
  $ \Theta_1 $ and $ \Theta_2(x^*) $
  }
  Clearly, 
  $ \Theta_1 $ is convex by the convexity of 
  $ C_1 $ and $ [0,\infty) $.
 To see that $\Theta_2(x^*)$ is convex consider the linear function
 \begin{gather*}
    L
    \,
    :
    \,
    \R^n\times\,  \R 
    \ 
    \to
    \ 
    \R
    \,,
    \qquad 
    (x,\lambda)
    \ 
    \mapsto
    \ 
    \inner{x^*\!}{x} - \lambda
    \,.
 \end{gather*}
 From the definitions of $L$ and $\Theta_2(x^*)$ we get 
  \begin{gather*}
 \Theta_2
 (x^*)
    \ 
    =
    \ 
    (
    C_2\!\times\R
    )
    \ 
    \cap
    \ 
    L^{\!-1}
    [\,\alpha,\infty)
    \,
    .
  \end{gather*}
  Thus,
  by
  Proposition~\ref{cv:primer:prop}~(v)
  and the convexity of $C_2$ we get the convexity of
  $
    L^{\!-1}
    [\,\alpha,\infty)
  $ and with it that of $\Theta_2(x^*)$.

  \subsubsection*{Relative interiors of
  $ \Theta_1 $ and $ \Theta_2(x^*) $
  are disjoint}
  We start by calculating the relative interiors. It holds
  \begin{alignat*}{3}
    \mathrm{ri}\,
    \Theta_1
    &
    \ 
    =
    \ 
    \mathrm{ri}
    ( C_1\times [0,\infty) )
    &&
    \ 
    =
    \ 
    \mathrm{ri}\,
    C_1
    \!
    \times
    \mathrm{ri}\,
    [0,\infty)
    \ 
    =
    \ 
    \mathrm{ri}\,
    C_1
    \!
    \times
    (0,\infty)
    \,,
    %%%%%%%%%%%%%%%%%
    \\
    \mathrm{ri}\,
    \Theta_2(x^*)
    & 
    \ 
    =
    \ 
    \mathrm{ri}
    (
    L^{\!-1}
    [\,\alpha,\infty)
    )
    &&
    \ 
    =
    \ 
    L^{\!-1}
    (
    \mathrm{ri}\,
    [\,\alpha,\infty)
    )
    \ 
    \ 
    =
    \ 
    L^{\!-1}
    (\alpha,\infty)
    \,.
  \end{alignat*}
  Suppose there exists
  $ (\lambda,x) \in
    \mathrm{ri}\, 
    \Theta_1
    \cap
    \,
    \mathrm{ri}\, 
    \Theta_2(x^*)
  $.
  Then it holds 
  $ x \in C_1\!\times C_2 $
  and 
  $ \lambda >0 $.
  We also note, that
  \begin{gather*}
  \alpha
  \ 
  =
  \ 
  \sigma_{C_1\cap\, C_2}(x^*)
    \ 
  =
    \ 
  \sup_{z \in C_1\cap\, C_2}
  \inner
  {x^*}
  {z}
  \ 
  \ge
  \ 
  \inner
  {x^*}
  {x}
  \,.
  \end{gather*}
  Then it follows
  \begin{gather*}
    \alpha
    \ 
    <
    \ 
  \inner
  {x^*}
  {x}
  - \lambda
    \ 
  \le
    \ 
  \alpha\,,
  \end{gather*}
  a contradiction.
  Thus, the relative interiors of
  $ \Theta_1 $ and $ \Theta_2(x^*) $
  are disjoint.

  Applying Theorem~\ref{cv:primer:sep} finishes the proof.
\end{proof}


\begin{theorem*}
  Let $C_1$ and $C_2$ be two non-empty convex sets in $\R^n$ with
  $\mathrm{ri}\,C_1\cap\mathrm{ri}\,C_2\neq\emptyset.$
  Then the support function of the intersection 
  $
    C_1\! \cap C_2
  $
  is represented as
  \begin{gather}
    (\sigma_{
    C_1 \cap\, C_2
    })
    (x^*)
    =
    (\sigma_{C_1}\square \,\sigma_{C_2})
    (x^*)
    \qquad
    \text{for all}\ 
    x^* \in \R^n.
  \end{gather}
  Furthermore, for any
  $
  x^*\in \mathrm{dom}
    (\sigma_{
    C_1 \cap\, C_2
    })
  $
  there exist dual elements 
  $
    x_1^*
    ,
    x_2^*
    \in \R^n
  $ 
  such that 
  $
    x^*
    =
    x_1^*
    +
    x_2^*.
  $
  and
  \begin{gather}
    (\sigma_{
    C_1 \cap\, C_2
    })
    (x^*)
    =
    \sigma_{C_1}(x_1^*)
    +
    \sigma_{C_2}(x_2^*).
  \end{gather}
\end{theorem*}
\begin{proof}
  Using Lemma~\ref{cv:primer:lem}
  the rest of the proof is as that of
  \emph{\cite[Theorem~4.23(b)]{Mordukhovich2022}}.
\end{proof}

\begin{takeaways}
  The support function intersection rule connects the geometric 
  property of convex separation to an identity of support functions
  This result is central to the analysis of convex conjugates.
\end{takeaways}
One important application of convex functions is in optimization.
There we often analyse a dual problem instead, which relies on the notion of \textbf{convex conjugate} 
$
    f^*:
    \R^n \to \overline{\R}
  $
  of $f$ defined by
  \begin{gather}
    \label{def:convex_conjugate}
    f^*(x^*)
    :=
    \sup_{ x \in \R^n }
    \inner
    {x^*}{x}
    - f(x)
    \,.
  \end{gather}
  Even for arbitrary functions, the convex conjugate is convex(cf.
  \cite[Proposition~4.2]{Mordukhovich2022}
  ).
  Like in differential calculus, there exist sum and chain rule for computing the convex conjugate.
\begin{theorem}
  Let
  $
    f,g:
    \R^n \to (-\infty, \infty]
  $
  be proper convex functions 
  and
  $
  \text{ri}\left( \text{dom}(f) \right)
  \cap
  \text{ri}\left( \text{dom}(g) \right)
  \neq 
  \emptyset
  .
  $
  Then we have the 
  \textbf{
  conjugate sum rule
  }
  \begin{gather}
    ( f + g )^*(x^*)
    =
    ( f^* \square g^*)(x^*)
  \end{gather}
  for all $x^* \in \R^n$.
  Moreover, the infimum in 
  $
    ( f^* \square g^*)(x^*)
  $
  is attained, i.e., for any
  $
    x^* \in \text{dom}(f+g)^*
  $
  there exists vectors $x_1^*, x_2^*$
  for which
  \begin{gather}
    (f+g)^*(x^*)
    =
    f^*(x_1^*)
    +
    g^*(x_2^*),
    \quad
    x^* = x_1^* + x_2^*.
  \end{gather}
\end{theorem}
\begin{proof}
  \cite[Theorem~4.27(c)]{Mordukhovich2022}
\end{proof}



% conjugate chain rule %
 %%%%%%%%%%%%%%%%%%%%%%
\begin{theorem}
  \label{cvxa_conjugate_chain_rule}
  Let 
  $
    A:
      \R^m \to \R^n
  $
  be a linear map (matrix)
  and
  $
    g:
      \R^n \to (-\infty, \infty]
  $
  a proper convex function. If
  $
    \text{Im}(A) \cap \text{ri}(\text{dom}(g))
    \neq
    \emptyset
  $
  it follows
  the 
  \textbf{conjugate chain rule}
  \begin{gather}
    ( g \circ A )^* ( x^* )
    =
    \inf_
          { y^* \in ( A^* )^{ -1 } ( x^* )}
                                          g^*( y^* )
                                          .
  \end{gather}
  Furthermore, 
    for any 
      $
        x^* \in \text{dom}( g \circ A)^*
      $
        there exists
          $
            y^* \in ( A^* )^{ -1 } ( x^* )
          $
            such that
              $
                ( g \circ A)^* ( x^* )
                =
                g^*( y^* )
              $.
\end{theorem}
\begin{proof}
  \cite[Theorem~4.28(c)]{Mordukhovich2022}
\end{proof}
%%%%%%%%%%%%%%%%%
%%%% EXAMPLE %%%%
%%%%%%%%%%%%%%%%%
\begin{example}
  \label{cv:cc:ex}
  Let 
  $
    f:\R\to \overline{\R}
  $
  be a proper convex function, that is, 
  $
    \mathrm{dom}\,f
    \neq
    \emptyset
  $
  and $f$ is convex.
  In steps we apply the conjugate chain and sum rule, together with mathematical induction,
  to prove the conjugate relationship 
  \begin{align*}
    &S_{f,n}:\R^n \to \overline{\R},
    \qquad
    (x_1,\ldots,x_n)
    \mapsto
    \sum_{i=1}^{n} 
    f(x_i)
    ,
    \\
    &S_{f,n}^*:\R^n \to \overline{\R},
    \qquad
    (x^*_1,\ldots,x^*_n)
    \mapsto
    \sum_{i=1}^{n} 
    f^*(x^*_i)
    \,.
  \end{align*}
  This relationship is very natural and the ensuing calculations serve to confirm our intuition.

  First, we work in the projections on the coordinates. 
  For the $i$-th coordinate, where $i=1,\ldots,n$, this is 
  \begin{gather}
    p_i:\R^n\to \R
    ,
    \quad
    (x_1,\ldots,x_n)
    \mapsto
    x_i\,.
  \end{gather}
  All projections 
  $p_i$
  are linear function with matrix representation
  $
    e_i^\top
  $,
  where $e_i$ is $i$-the coordinate vector.
  The adjoint of $p_i$ is therefore
  \begin{gather}
    p^*_i:\R\to \R^n
    ,
    \quad
    x
    \mapsto
    e_i\cdot x
    \,.
  \end{gather}
  For the inverse image of the adjoint of $p_i$ it holds
  \begin{gather}
    (p_i^*)^{-1}
    \left\{ 
    (x_1^*,\ldots,x_n^*)
    \right\}
    \ 
    =
    \ 
    \begin{cases}
      \left\{ x_i^* \right\},
      \quad
      &\text{if}\ 
      x_j^*=0\ \text{for all}\ j\neq i\,,
      \\
      \ \ \emptyset
      \quad
      &\text{else.}
    \end{cases}
  \end{gather}
  Throughout this example we use the asterisk character $^*$ somewhat inconsistently. 
  Note that $f^*$ is the convex conjugate 
  of the function $f$ and $p_i^*$ is the adjoint linear function of the projection on the $i$-th coordinate. Likewise, we denote dual variables, that is, the arguments of convex conjugates, as $x^*$.

  Next, we employ the conjugate chain rule to establish the conjugate relationship 
  \begin{align*}
    f_i&:\R^n\to \overline{\R}
    ,
    \quad
    (x_1,\ldots,x_n)
    \mapsto x_i \mapsto f(x_i)
    \,,
    \\
    f^*_i&:\R^n\to \overline{\R}
    ,
    \quad
    (x^*_1,\ldots,x^*_n)\mapsto 
    \begin{cases}
      f^*(x_i^*),
      \quad
      &\text{if}\ 
      x_j^*=0\ \text{for all}\ j\neq i\,,
      \\
      \infty
      \quad
      &\text{else.}
    \end{cases}
  \end{align*}
  Note, that 
  $
    f_i
    =
    (f\circ p_i)
  $
  and
  $
    f^*_i
    =
    (f\circ p_i)^*
  $.
  Since 
  $
    \mathrm{Im}\,p_i=\R
  $
  and 
  $
    \mathrm{dom}\, f
    \neq
    \emptyset
  $,
  it holds
  $
    \mathrm{Im}\, p_i
    \cap
    \mathrm{ri}(
    \mathrm{dom}\, f
    )
    \neq
    \emptyset
  $.
  Then $f$ and $p_i$ conform with the demands of the conjugate chain rule.
  It follows
  \begin{align*}
    &f_i^*
    (x^*_1,\ldots,x^*_n) 
    \ 
    =
    \ 
    (f\circ p_i)^*
    (x^*_1,\ldots,x^*_n) 
    \ =
    \ 
    \inf
    \left\{ 
    f^*(y)
    \ 
    |
    \ 
    y\in 
    (p_i^*)^{-1}
    \left\{ 
    (x_1^*,\ldots,x_n^*)
    \right\}
    \right\}
    \\
    &\quad=
    \ 
    \begin{cases}
      f^*(x_i^*),
      \quad
      &\text{if}\ 
      x_j^*=0\ \text{for all}\ j\neq i\,,
      \\
      \infty
      \quad
      &\text{else,}
    \end{cases}
  \end{align*}
  where we keep to the convention $\inf\emptyset=\infty$.
  In the same way it follows
  \begin{gather}
    \left( 
      S_{f,n}
      \circ
      p_{\left\{ 1,\ldots,n \right\}}
    \right)^*
    (x^*_1,\ldots,x^*_{n+1})
    =
    \begin{cases}
      S_{f,n}^*
    (x^*_1,\ldots,x^*_{n})
      \quad
      &\text{if}\ 
      x_{n+1}^*=0\,,
      \\
      \infty
      \quad
      &\text{else,}
    \end{cases}
  \end{gather}

  Next, note that for $n=1$ we arrive at the result. Thus, for some $n\in \mathbb{N}$ it holds
  $
  \left( 
    S_{f,n}
  \right)
  ^*
  =
    S_{f,n}^*
  $.
  In order to apply the conjugate sum rule to 
  $
    S_{f,n}
  $
  and
  $
    f_{n+1}
  $
  we note that
  \begin{align*}
    \mathrm{dom}\, f_i
    &
    \ =\ 
    \left\{ 
      (x_1,\ldots,x_{n+1})
      \in \R^{n+1}
      :
      x_i\in \mathrm{dom}\,f
    \right\}
    \ 
    \neq 
    \ 
    \emptyset
    \qquad
    \text{for all}
    \ 
    i=1,\ldots,n+1
    \,,
    \\
    \bigcap_{i=1}^{n+1}
    \mathrm{dom}\, f_i
    &
    \ =\ 
    \left\{ 
      (x_1,\ldots,x_{n+1})
      \in \R^{n+1}
      :
      x_i\in \mathrm{dom}\,f
      \ 
    \text{for all}
    \ 
    i=1,\ldots,n+1
    \right\}
    \ 
    \neq 
    \ 
    \emptyset
    \,,
  \end{align*}
  and
\begin{align*}
  &
  \mathrm{ri}\left( 
    \mathrm{dom}
    \left( 
    S_{f,n}
    \circ
    p_{\left\{ 1,\ldots,n \right\}}
    \right)
  \right)
  \ 
  \cap
  \ 
  \mathrm{ri}\left( 
    \mathrm{dom}\,f_{n+1}
  \right)
  \\
  &
  \hspace{25mm}
  =
  \ 
  \mathrm{ri}\left( 
    \mathrm{dom}
    \left( 
    S_{f,n}
    \circ
    p_{\left\{ 1,\ldots,n \right\}}
    \right)
  \ 
  \cap
  \ 
    \mathrm{dom}\,f_{n+1}
  \right)
  \ 
  =
  \ 
  \mathrm{ri}
  \left( 
    \bigcap_{i=1}^{n+1}
    \mathrm{dom}\, f_i
  \right)
  \ 
  \neq
  \ 
  \emptyset
  \,.
\end{align*}
By the conjugate sum rule it follows
\begin{align*}
  (
  S_{f,n+1}
  )^*
  =
  (
    S_{f,n}
    \circ
    p_{\left\{ 1,\ldots,n \right\}}
  +
  f_{n+1}
  )^*
  =
  (
    S_{f,n}
    \circ
    p_{\left\{ 1,\ldots,n \right\}}
  )
  ^*
  \square
  f_{n+1}^*
  \\
  =
    S_{f,n}^*
    \circ
    p_{\left\{ 1,\ldots,n \right\}}
    +
  f_{n+1}^*
  =
  S^*_{f,n+1}
  \,.
\end{align*}
\end{example}






\section{Duality of Optimal Solutions}
  We consider a general convex optimization problem 
with matrix equality and inequality constraints.
For this problem there exists a related problem,
which we call its dual.
With ideas from \cite{Tseng1991} we establish 
a functional relationship
between the optimal solution of the original problem 
and
optimal solutions of the dual.
The main assumption is that in the original problem we have a strictly convex objective function 
with continuously differentiable 
convex conjugate(cf. Definition~\ref{cv:cc:d:cc}). 
\begin{ftheorem}
  \label{cv:ts:th}
  Consider the optimization problem
\begin{align}
  \label{cv:ts:primal}
  %%%% objective %%%%
    &\underset{w \in \R^n}
    {\mathrm{minimize}}
    &&\qquad\qquad
    f(w)
    &&&
    \\
    %%%% Ax >= b %%%%
    \nonumber
    &\mathrm{subject}\ \mathrm{to} 
    &&\qquad\qquad
    \mathbf{U}w
    \ 
    \ge
    \ 
    d
    \,.
    \\
    \nonumber
    &
    &&\qquad\qquad
    \mathbf{A}w
    \ 
    =
    \ 
    a
    \,,
\end{align}
and its dual problem
  \begin{alignat}{2}
    \label{cv:ts:dual}
  %%%% objective %%%%
    &\underset{
    \lambda_d \in \R^r
,
    \lambda_a \in \R^s
  }
    {\mathrm{maximize}}
    &&\qquad\qquad
    \inner
    {\lambda_d}
    {d}
    \ 
    +
    \ 
    \inner
    {\lambda_a}
    {a}
    \ 
    -
    \ 
    f^*
    \!
    \left( 
      \mathbf{U}^\top \! \lambda_d
      +
      \mathbf{A}^\top \! \lambda_a
    \right)
    \\
    %%%% Ax >= b %%%%
    \nonumber
    &\mathrm{subject}\ \mathrm{to} 
    &&\qquad\qquad
    \lambda_d
    \ 
    \ge
    \ 
    0
    \,.
\end{alignat}
  Let 
$
(\lambda_d^\dagger,\lambda_a^\dagger)
$
be an optimal solution to \eqref{cv:ts:dual}.
If the objective function $f$ of 
\eqref{cv:ts:primal} is strictly convex and its
convex conjugate $f^*$ is continuously differentiable,
then the unique optimal solution to 
\eqref{cv:ts:primal}
is given by
\begin{gather}
  w^\dagger
  =
  \nabla
    f^*
    \!
    \left( 
      \mathbf{U}^\top  \lambda_d^\dagger
      +
      \mathbf{A}^\top  \lambda_a^\dagger
    \right)
    \,.
\end{gather}
\end{ftheorem}

\subsubsection*{Plan of Proof}
We show that 
$w^\dagger$ and 
$
(\lambda_d^\dagger,\lambda_a^\dagger)
$
meet the 
Karush-Kuhn-Tucker conditions for \ref{cv:ts:primal},
that is,
\textbf{complementary slackness}
\begin{gather}
\inner
{\lambda_d^\dagger\,}{d-\mathbf{U} w^\dagger}
\ 
=
\ 
0
\,,
\end{gather}
\textbf{primal} and \textbf{dual feasibility}
\begin{align}
  \label{primal_feas}
    \mathbf{U}w^\dagger
    &
    \ 
    \ge
    \ 
    d
    \,,
    \\
    \nonumber
    \mathbf{A}w^\dagger
    &
    \ 
    =
    \ 
    a
    \,,
  \\
  \label{dual_feas}
  \lambda_d^\dagger
    &
    \ 
  \ge
  \ 
  0
    \,,
\end{align}
and 
\textbf{stationarity}
\begin{gather}
  \mathrm{0}_n
  \ 
  \in
  \ 
  [
  \partial
  f(w^\dagger)
  \ 
  +
  \ 
    \partial
    \left( 
      w
      \mapsto
      d
      -
      \mathbf{U}w
    \right)
    (w^\dagger)
    \cdot
    \lambda_d^\dagger
    \ 
    +
    \ 
    \partial
    \left( 
      w
      \mapsto
      a
      -
      \mathbf{A}w
    \right)
    (w^\dagger)
    \cdot
    \lambda_a^\dagger
    \,
  ]
  \,.
\end{gather}
Applying the well know result\cite[Theorem~28.3]{Rockafellar1970}
finishes the proof.
Apart from elementary calculations, our main tools are the 
strict convexity of $f$, the smoothness of $f^*$ and 
\begin{proposition}
  \emph{
\cite[Theorem~23.5(a)-(b)]{Rockafellar1970}.
  }
  \label{cv:ts:prop}
   For any proper convex function $g$ and any vector $w$, 
   it holds $t\in \partial f(w)$ 
   if and only if 
   $
   x
   \mapsto
   \inner
   {x}{t}
   -
   f(x)
   $
   achieves its supremum at $w$.
\end{proposition}

\begin{proof}
\end{proof}


\renewcommand{\bibsection}
{
\chapter*{References}
\addcontentsline{toc}{chapter}{References}
}
\bibliographystyle{alpha}
\bibliography{literature}
\printindex
\end{document}
