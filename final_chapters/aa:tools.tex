\subsection{Tools and Assumptions}

Let 
$
  \left( 
    \Omega,
    \mathcal{A},
    \P
  \right)
$
be a probability space,
$
  \left( 
    \mathcal{Z},
    \Sigma
  \right)
$
a measurable space, and 
\begin{gather*}
  \xi_1,\ldots,\xi_N
  :
  \left( 
    \Omega,
    \mathcal{A},
    \P
  \right)
  \to
  \left( 
    \mathcal{Z},
    \Sigma
  \right)
  \quad
  \text{independent and identically-distributed
  }
\end{gather*}
random variables
with probability distribution $\P_{\!\xi}$.
Let $\mathcal{F}$ be a class of measurable functions 
$
  f:
  \left( 
    \mathcal{Z},
    \Sigma
  \right)
    \to
  \left( 
    \R,
    \mathcal{B}(\R)
  \right)
$, where
$
    \mathcal{B}(\R)
$
is the Borel-$\sigma$-algebra on $\R$.
Then $\mathcal{F}$
induces a stochastic process by
\begin{gather}
  f
  \ 
  \mapsto
  \ 
  \G_N f 
  \ 
  :=
  \ 
  \frac{1}{\sqrt{n}}
  \sum_{i=1}^{N} 
  \left(
    f(\xi_i)
    -
    \E_\xi[f]
  \right)
  \,,
\end{gather}
where
$
    \E_\xi[f]
    :=
    \int_\mathcal{Z}
    f
    \,
    d\P_\xi
$.
We call
$\G_N$ the  \textbf{empirical process} indexed by $\mathcal{F}$.
Often, the purpose of this construction is, to study the behaviour of a centered, scaled arithmetic mean uniformly over $\mathcal{F}$.
To this end, we define the (random) norm
\begin{gather}
  \norm{\G_n}_\mathcal{F}
  :=
  \sup_
        { f \in \mathcal{F}}
        \left|
          \G_N f
        \right|
        .
\end{gather}
We stress that 
$
  \norm{\G_n}_\mathcal{F}
$
often ceases to be measurable, even in simple situations~\cite[page 3]{vaart2013}.
To deal with this, we introduce the notion of \textbf{outer expectation} $\E^*$ (see \cite[page~6]{vaart2013})
\begin{gather*}
  \E^*[Z]
  \ 
  :=
  \ 
    \inf
  \left\{ 
    \E[U]
  \ 
  \lvert
  \ 
    U\ge Z,
    \ 
    U:
  \left( 
    \Omega,
    \mathcal{A},
    \P
  \right)
  \to 
  \left( 
    \overline{\R},
    \mathcal{B}(\overline{\R})
  \right)
  \text{measurable and}
  \ 
  \E[U]<\infty
  \right\}
  \,.
\end{gather*}
In our application the technical difficulties halt at this point, because we only consider $Z$ with $\E^*[Z]<\infty$. Then there exists a smallest measurable function $Z^*$ dominating $Z$ with
$\E^*[Z]=\E[Z^*]$ (see \cite[Lemma~1.2.1]{vaart2013}).

To control empirical processes - apart from strong theorems - we need the notion of bracketing number and integral (see \cite[page 270]{Vaart2000}). 
Given two functions $\underline{f}\le \overline{f}$,
\begin{gather*}
  \text{
the bracket
  }\quad
[\underline{f},\overline{f}]
\quad 
\text{
is the set of all functions $f$ with 
}\quad 
\underline{f}\ \le\ f \ \le\  \overline{f}
\,.
\end{gather*}
For $\varepsilon>0$
we define a
\begin{gather*}
  \text{
$(\varepsilon, L^{r}(\P))$ -bracket
to be a bracket
  }
  \quad
[\underline{f},\overline{f}]
\quad
\text{with}
\quad
\norm{\overline{f}-\underline{f}}_{ L^r(\P)}
\ <\  \varepsilon
\,.
\end{gather*}
The \textbf{
bracketing number
} 
$
N_{[\,]}(\varepsilon, \mathcal{F}, L^r(\P))
$
is 
the minimum number of 
$(\varepsilon, L^{r}(\P))$-brackets needed to cover $\mathcal{F}$.

For most classes $\mathcal{F}$ the bracketing number grows to infinity for $\varepsilon\to 0$.
To measure the speed of growth we introduce 
for $\delta>0$
the
\textbf{bracketing integral}
\begin{gather*}
     J
    _{[\,]}
    (
    \delta
    ,
    \mathcal{F}
    ,
    L_r(\P)
    )
    \ 
    =
    \ 
  \int_0^{\delta}
      \sqrt{
        \log 
      N_{[\,]}
\left( \varepsilon, \mathcal{F}_N, L^r(\P) \right)
    }
    \,
    d\varepsilon
    \,.
\end{gather*}
Before we give some results, we introduce the notion of envelope function.
An envelope function $F$ of a class $\mathcal{F}$ satisfies 
$|f(z)|\le F(z)< \infty$ for all $f\in\mathcal{F}$ and all $z\in\mathcal{Z}$.
\newpage
There is a powerful theorem - a central limit theorem for $\G_N$ uniform in $\mathcal{F}$ - that we now introduce.
\begin{definition}
  We call a class 
  $\mathcal{F}$ of measurable functions 
$\P$-Donsker
if the sequence of processes 
$\left\{ \G_N f \colon f\in\mathcal{F}\right\}$
converges in
$l^\infty(\mathcal{F})$
to a tight limit process.
\end{definition}

\begin{theorem}
  Every class $\mathcal{F}$ of measurable functions 
  with
  \begin{gather*}
    J
    _{[\,]}
    (
    1
    ,
    \mathcal{F}
    ,
    L_2(\P)
    )
    <\infty
  \end{gather*}
  is
  $\P$-Donsker.
  Furthermore,
  the sequence of processes 
$\left\{ G_N f \colon f\in\mathcal{F}\right\}$
  converges 
  in
$l^\infty(\mathcal{F})$
to a Gaussian process with mean 0 and covariance function given by
\begin{gather*}
  \mathbf{Cov}(f,g)
  \ 
  :=
  \ 
  \E[fg]
  \ 
  -
  \ 
  \E[f]\E[g]
  \,.
\end{gather*}
\end{theorem}
\begin{proof}
  \cite[Theorem~19.5]{Vaart2000}
\end{proof}



In our application we need concentration inequalities for 
$
  \norm{\G_n}^*_\mathcal{F}
$.
One easy way to obtain this is, to use a maximal inequality (see Theorem~\ref{th:max_ineq}) to control the expectation,
together with Markov's inequality. There are also Bernstein-like inequalities for empirical processes (see \cite[§2.14.2]{vaart2013}). 
\begin{theorem}
  \label{th:max_ineq}
  \emph{(Maximal inequality)}
  For any class $\mathcal{F}$ of measurable functions with envelope function $F,$
  \begin{gather*}
    \E^*
    \norm{
      \G
      _n
      }
      _\mathcal{F}
    \ 
    \lesssim
    \ 
    J
    _{[\,]}
    (
    \norm{
      F
    }
    _{ L^2(\P)}
    ,
    \mathcal{F}
    ,
    L^2(\P)
    )
    \, 
    .
  \end{gather*}
\end{theorem}
\begin{proof}
  \cite[Corollary~19.35]{Vaart2000}
\end{proof}

\begin{lemma}
  \label{markov_max_lemma}
  Let $(\mathcal{H}_N)$ be a sequence of measurable function classes with envelope functions $(H_N)$.
  If
  \begin{gather*}
    J_{[\, ]}
    \left( 
    \norm{H_N}_{ L^2(\P)}
    ,
    \mathcal{H}_N
    ,
     L^2(\P)
    \right)
    \ 
    \to
    \ 
    0
    \qquad
    \text{for}
    \ 
    N
    \to
    \infty
    \,,
  \end{gather*}
  it holds 
  $
  \norm{\G_N}^*_{\mathcal{H}_N}\overset{\P}{\to}0
  $.
\end{lemma}
\begin{proof}
  By Markov's inequality and Theorem~\ref{th:max_ineq} it holds for all $\varepsilon>0$
  \begin{align*}
    \P
    [
  \norm{\G_N}^*_{\mathcal{H}_N}
  \ge
  \varepsilon
    ]
    &
    \ 
    \le
    \ 
    \varepsilon^{-1}
    \E
    [
  \norm{\G_N}^*_{\mathcal{H}_N}
    ]
    \ 
    =
    \ 
    \varepsilon^{-1}
    \E^*
    [
  \norm{\G_N}_{\mathcal{H}_N}
    ]
    \\
    &
    \ 
    \lesssim
    \ 
    \varepsilon^{-1}
    J_{[\, ]}
    \left( 
    \norm{H_N}_{ L^2(\P)}
    ,
    \mathcal{H}_N
    ,
     L^2(\P)
    \right)
    \\
    &
    \ 
    \to
    \ 
    0
    \qquad
    \text{for}
    \ 
    N\to\infty
    \,.
  \end{align*}
\end{proof}
Next we give a technical lemma to 
bound the bracketing numbers of products of two function classes, that is,
\begin{gather*}
  \mathcal{F}\cdot \mathcal{G}
  \ 
  :=
  \ 
  \left\{ 
    f\cdot g
    \ 
    \colon
  \ 
    f\in\mathcal{F},
    g\in\mathcal{G}
  \right\}\,.
\end{gather*}
\begin{lemma}
  \label{lem_prod_br}
  Let
  $\mathcal{F}$ and $\mathcal{G}$ be two function classes 
  with envelope functions $F$ and $G$ satisfying
  $\norm{F}_\infty,\norm{G}_\infty\le 1$.
  For all $\varepsilon>0$ and all $r\in [1,\infty)$ it holds
  \begin{gather*}
    N_{[\,]}(2\varepsilon,\mathcal{F}\cdot\mathcal{G},\mathrm{L}_r(\P))
    \
    \le
    \ 
    N_{[\,]}(\varepsilon,\mathcal{F},\mathrm{L}_r(\P))
    \cdot
    N_{[\,]}(\varepsilon,\mathcal{G},\mathrm{L}_r(\P))
    \,.
  \end{gather*}
\end{lemma}
\begin{proof}
  Let $f\in\mathcal{F}$ and $g\in\mathcal{G}$.
  We can choose two 
  $(\varepsilon,L^r(\P))$
  brackets
  $[\underline{f},\overline{f}]$
  and
  $[\underline{g},\overline{g}]$
  containing $f$ and $g$ with 
  $\norm{\underline{f}}_\infty,\norm{\overline{f}}_\infty\le\norm{F}_\infty\le 1$
  and
  $\norm{\underline{g}}_\infty,\norm{\overline{g}}_\infty\le\norm{G}_\infty\le 1$.
  We the get an 
  $(2\varepsilon,L^r(\P))$
  $[\underline{h},\overline{h}]$
  bracket, containing $f\cdot g$, by
\end{proof}
This is as much theory of empirical processes as we need for the moment. 
The next lemma shows what effect the weights
$T/\pi(X)$ have on other functions.
\begin{lemma}
  \label{ps_weights_lemma}
  Let
  $
  g_1\colon
  \mathcal{X}\to\R
  $
  and
  $
  g_2\colon
  \mathcal{Y}\to\R
  $
  be a measurable functions.
  It holds
  \begin{gather*}
    \E
    \left[
    \frac{T}{\pi(X)}
    g_1(X)
    \right]
    \ 
    =
    \ 
    \E
    \left[
    g_1(X)
    \right]
    \,.
  \end{gather*}
  If Assumption~\ref{aa:assumption:treatment_str_ign} holds true, then
  \begin{gather*}
    \E
    \left[
    \frac{T}{\pi(X)}
    g_2(Y(T))
    \right]
    \ 
    =
    \ 
    \E
    \left[
    f(Y(1))
    \right]
    \,.
  \end{gather*}
\end{lemma}


\begin{lemma}
  \label{aa:r3:lemma:1}
  Let
$(\varepsilon_N)\subset(0,1]$
be 
a decreasing sequence
with $\varepsilon_N\to 0$ for $N\to\infty$ 
and let Assumption~\ref{aa:assumption:1} hold true
for a sequence of function classes $\mathcal{F}_N$.
Then
\begin{gather*}
  J_{[\,]}(
\norm{F_N}_{L^2(\P)}
,\mathcal{F}_N\cdot\mathcal{F},\mathrm{L}_2(\P))
  \to 0
  \quad
  \text{and}
  \quad
  \norm{\G_N}^*_{\mathcal{F}_N\cdot\mathcal{F}}\overset{\P}{\to}0
  \qquad
  \text{for}\ 
  N\to\infty
  \,.
\end{gather*}
\end{lemma}
\begin{proof}
  By Assumption~\ref{aa:assumption:1}
  and Lemma~\ref{aa:mean:l:br} it holds
for some $k<2$
\begin{gather*}
\norm{F_N}_{L^2(\P)}
\ 
\le
\ 
\varepsilon_N
\quad
\text{and}
\quad
  \log
  N_{[\,]}(\varepsilon,\mathcal{F}_N,\mathrm{L}_2(\P))
  \ 
  \lesssim
  \ 
  \left( 
  \frac{1}{\varepsilon}
  \right)^k
  \quad
  \text{for all}
  \ 
  N\in\mathbb{N}
  \,,
\end{gather*}
and
  \begin{gather*}
    N_{[\,]}
    (
    \varepsilon
    ,
    \mathcal{F}, L^2(\P))
    \ 
    \lesssim
    \ 
    \left( 
      \frac{1}{\varepsilon}
    \right)^2
    \qquad
    \text{for all}
    \ 
    \varepsilon>0
    \,.
  \end{gather*}
  Since $\mathcal{F}_N$ and $\mathcal{F}$ have envelope function smaller 1, we can apply Lemma~\ref{lem_prod_br} to get
  \begin{gather*}
  \log
  N_{[\,]}(\varepsilon,\mathcal{F}_N\cdot\mathcal{F},\mathrm{L}_2(\P))
  \ 
  \lesssim
  \ 
  \left( 
  \frac{1}{\varepsilon}
  \right)^k
  +
  \log
  (1/\varepsilon)
  \ 
  \lesssim
  \ 
  \left( 
  \frac{1}{\varepsilon}
  \right)^k
  \quad
  \text{for all}\ 
  \varepsilon>0
  \,.
  \end{gather*}
  Since 
  $k/2\in(0,1)$
  it holds
\begin{align*}
  J_{[\,]}(
\norm{F_N}_{L^2(\P)}
,\mathcal{F}_N\cdot\mathcal{F},\mathrm{L}_2(\P))
  &
  \ 
=
  \ 
\int_0^{
\norm{F_N}_{L^2(\P)}
}
\sqrt{
  \log
  N_{[\,]}(\varepsilon,\mathcal{F}_N\cdot\mathcal{F},\mathrm{L}_2(\P))
}
\,d\varepsilon
\\
&
\ 
\lesssim
\ 
\int_0^{
  \varepsilon_N
}
  \left( 
  \frac{1}{\varepsilon}
\right)^{k/2}
\,d\varepsilon
\\
&
\ 
=
\ 
\frac{
\varepsilon_N^{1-k/2}
}{1-k/2}
\ 
\to 0
\ 
\qquad
\text{for}
\ 
N\to\infty
\,.
\end{align*}
The second statement follows from Lemma~\ref{markov_max_lemma}
for 
$\mathcal{H}_N:=\mathcal{F}_N\cdot\mathcal{F}$ and
$H_N:=F_N$.
\end{proof}

\begin{assumption}
  \label{aa:assumption:1}
  For any decreasing sequence
  $(\varepsilon_N)$ with $\varepsilon_N\to 0$ for $N\to\infty$,
there exists a sequence of (measurable) function classes
$(\mathcal{F}_N)$
with envelope functions
$(F_N)$,
satisfying 
for some $k<2$
\begin{gather*}
\norm{F_N}_{L^2(\P)}
\ 
\le
\ 
\varepsilon_N
\quad
\text{and}
\quad
  \log
  N_{[\,]}(\varepsilon,\mathcal{F}_N,\mathrm{L}_2(\P))
  \ 
  \lesssim
  \ 
  \left( 
  \frac{1}{\varepsilon}
  \right)^k
  \quad
  \text{for all}
  \ 
  N\in\mathbb{N}
  \,,
\end{gather*}
for all $N\in\mathbb{N}$ the function
\begin{gather}
  \mathcal{X}
  \ 
  \to
  \ 
  \R
  \,,
  \qquad
  x
  \ 
  \mapsto
  \ 
  \mathbf{1}{
    \left\{ 
      \sup_{y\in A_N(x)}
      \left| 
      w^\dagger(y)
      -
      \frac{1}{\pi(y)}
      \right|
      \,
      \le
      \,
      \varepsilon_N
    \right\}
  }
  \left( 
    w^\dagger(x)
      -
      \frac{1}{\pi(x)}
  \right)
\end{gather}
is contained in $\mathcal{F}_N$.
\end{assumption}
The next Lemma and the ensuing examples show, what assumptions on the regularity of the inverse propensity score and the distribution of $X$ imply Assumption~\ref{aa:assumption:1}.
To this end, we need notation from \cite[§2.7.1]{vaart2013}.
To this end, let for any vector $k\in\mathbb{N}_0^d \ (d\in\R)$
\begin{gather*}
  D^k
  \ :=\ 
  \frac
  {\partial^{\norm{k}_1}}
  {
    \partial^{k_1}x_1
    \cdots
    \partial^{k_d}x_d
  }
  \,,
\end{gather*}
and let $\lfloor a \rfloor$ be the greatest integer smaller than $a>0$.
For $\alpha>0$, a bounded set 
$\mathcal{X}\subset\R^d\ (d\in\mathbb{N})$
and
$M>0$, we define $C^\alpha_M(\mathcal{X})$ to be the space of all continuous functions $f\colon \mathcal{X}\to\R$ with
\begin{gather*}
  \max_{\norm{k}_1\le \alpha}\sup_{x\in\mathcal{X}}
  \left| D^k f(x) \right|
  \ 
  +
  \ 
  \max_{\norm{k}_1=\lfloor \alpha \rfloor}\sup_{x,y}
  \frac
  {
  \left|
  D^k f(x) 
  -
  D^k f(y) 
  \right|
  }
  {
    \norm{x-y}_2^{\alpha-\lfloor \alpha \rfloor}
  }
  \
  \le
  \ 
  M
  \,.
\end{gather*}
where the suprema in the second term are taken over all $x,y$ in the interior of $\mathcal{X}$ with $x\neq y$.
Furthermore, let
\begin{gather*}
  \mathcal{X}^1
  :=
  \left\{ 
    y\in\R^d
    \colon
    \norm{x-y}_2 <1
    \ 
    \text{for some}\ x\in\mathcal{X}
  \right\}
  \,.
\end{gather*}
\begin{lemma}
  \label{vdv_coro}
  Let $\mathcal{P}=\left\{ A_1,A_2,\ldots \right\}$ be a partition of $\R^d$ into bounded, convex sets with non-empty interior, and let $\mathcal{F}$ be a class of functions $f\colon\R^d\to\R$ such that the restrictions $\mathcal{F}_{|A_j}$ belong to $C^\alpha_{M_j}(A_j)$
  for all $j\in\mathbb{N}$.
  Then there exists a constant $K$, depending only on $\alpha$, $V$, $r$ and $d$
  such that
  \begin{gather}
    \label{667}
    \log
    N_{[\,]}
    (
    \varepsilon
    ,
    \mathcal{F}
    ,
    L^r(\mathbf{Q})
    )
    \le
    K
    \left( \frac{1}{\varepsilon} \right)^V
    \left( 
      \sum_{j=1}^{\infty}
      \lambda(A_j^1)^{r/(V+r)}
      M_j^{Vr/(V+r)}
      \mathbf{Q}(A_j)^{V/(V+r)}
    \right)
    ^{(V+r)/r}
  \end{gather}
  for every $\varepsilon>0$, $V\ge d/\alpha$, and probability measure $\mathbf{Q}$.
\end{lemma}
\begin{proof}
  \emph{\cite[Corollary~2.7.4]{vaart2013}}
\end{proof}
\pagebreak
\begin{lemma}
  Let $(\mathcal{P}_N)$ denote a sequence of qubic partitions
  $\mathcal{P}_N=\left\{ A_{N,1},A_{N,2},\ldots \right\}$ 
  of $\R^d$ 
  with decreasing width $(h_N)\subset(0,1]$ such that $h_N\to 0$ for $N\to\infty$.
  Furthermore, assume that there exists
  $\alpha>d/2$, where $\mathcal{X}\subseteq \R^d$, such
  that for 
  $V:=d/\alpha$
 and for all 
$
(j,N)\in\mathbb{N}^2
$
there exists 
$M_{N,j}\ge 1$ such that 
\begin{gather}
  \label{897}
  \frac{1}{\pi(\cdot)}
  \in C^\alpha_{M_{N,j}}(A_{N,j})
  \quad
  \text{and}
  \quad
  \sum_{j=1}^{\infty} 
  M_{N,j}^{2V/(V+2)}
  \P
  [
  X\in A_{N,j}
  ]^{V/(V+2)}
  \ 
  \lesssim
  \ 
  1
  \,.
\end{gather}
It then holds the statement of Assumption~\ref{aa:assumption:1}.
\end{lemma}
\begin{proof}
  We want to employ Lemma~\ref{vdv_coro}. 
  To do this, the crucial observation is, that
  \begin{gather*}
    w^\dagger(\omega)
    \quad
    \text{is constant on each cell}\ 
    A_N\in\mathcal{P}_N
    \ \text{for all}\ \omega\in\Omega
    \,.
  \end{gather*}
  Thus, the regularity of the function \eqref{ghost_function}
  on each cell $A_N\in\mathcal{P}_N$ is decided by 
  $1/\pi(\cdot)$.
  Indeed, \eqref{ghost_function} is either 0 if the threshold of $\varepsilon_N$ is exceeded somewhere in the cell, or has the form constant-minus-smooth-function.
In any case, it is continuous and bounded by $\varepsilon_N$.
All its derivatives are 0 (if the threshold is exceeded) or are governed by $1/\pi(\cdot)$.
Thus, it follows from \eqref{897}
\begin{gather}
  \eqref{ghost_function}
  \in C^\alpha_{M_{N,j}}(A_{N,j})
  \quad
  \text{and}
  \quad
  \sum_{j=1}^{\infty} 
  M_{N,j}^{2V/(V+2)}
  \P
  [
  X\in A_{N,j}
  ]^{V/(V+2)}
  \ 
  \lesssim
  \ 
  1
  \,.
\end{gather}
To bound the right-hand-side in \eqref{667} we note
that 
$
\lambda(A_{N,j})=h^d_N
$ 
and thus
$
\lambda(A_{N,j}^1)\lesssim 1
$
for all $(j,N)\in\mathbb{N}^2$.
Thus
\begin{gather*}
  \sum_{j=1}^{\infty} 
  \lambda(A_{N,j}^1)^{2/(V+2)}
  M_{N,j}^{2V/(V+2)}
  \P
  [
  X\in A_{N,j}
  ]^{V/(V+2)}
  \ 
  \lesssim
  \ 
  1
  \,.
\end{gather*}
We get $\eqref{ghost_function}\in\mathcal{F}_N$, where 
$
\mathcal{F}_N
$
restricted to $A_{N,j}$ is 
$
  C^\alpha_{M_{N,j}}(A_{N,j})
$
and satisfies the requirements of Lemma~\ref{vdv_coro}.
Since $V=d/\alpha \in (0,2)$ by $\alpha>d/2$, 
applying Lemma~\ref{vdv_coro} finishes the proof.
\end{proof}
\begin{remark}
  Note, that we only get $L^2(\P_X)$ bracketing numbers in this way. If we assume, that all functions in $\mathcal{F}$ are independent of $(T,Y)$
  we readily obtain $L^2(\P)$ bracketing numbers. Note, that $w^\dagger(X)$ and $1/\pi(X)$ are independent of $(T,Y)$.  
\end{remark}
In the next examples we show the concrete applications of this Lemma.
\begin{example}
Let $d\in\R$ and assume $\pi(\cdot)$ follows a logistic regression model. Then there exist $(\beta_0,\beta)\in\R^{d+1}$ such that
\begin{gather*}
  \pi(x)
  \ 
  =
  \ 
  \frac{1}
  {
    1
    +
    \exp
    \left( 
      -
        \beta_0
        -
        \inner{\beta}{x}
    \right)
  }
  \quad
  \text{and}
  \quad
  \frac
  {1}
  {
  \pi(x)
  }
  \ 
  =
  \ 
    1
    +
    \exp
    \left( 
      -
        \beta_0
        -
        \inner{\beta}{x}
    \right)
    \quad
    \text{for all}\ 
    x\in\R^d
    \,.
\end{gather*}
Clearly, by the smoothness of the exponential function, for all 
$\alpha>0$ there exist $(M_{N,j})\ge 1$ such that
$
  1/\pi(\cdot)\in C^\alpha_{M_{N,j}}(A_{N,j})
$.
Assume $\#\mathcal{X}<\infty$, that is, $X$ can take only finitely many values with positive probability.
We write
\begin{gather*}
  J_N
  :=
  \left\{ 
    j\in\mathbb{N}
    \colon
    \P[X\in A_{N,j}]>0
  \right\}
  \,.
\end{gather*}
It holds
$\#J_N\le \# \mathcal{X}<\infty$.
Thus, the following maximum is attained
\begin{gather*}
  \max_{j\in J_N} M_{N,j}
  =:M^*_N
  \,.
\end{gather*}
But the partitions increasingly better fit the support of $X$. Thus
$M^*_N$ is decreasing in $N$, that is,  $\infty>M^*_1\ge M^*_N$.
It follows
\begin{align*}
  \sum_{j=1}^{\infty} 
  M_{N,j}
  ^{2V/(V+2)}
  \P
  [
  X\in A_{N,j}
  ]^{V/(V+2)}
  \ 
  \le
  \ 
  \left( 
  M^*_1
  \right)
  ^{2V/(V+2)}
  \cdot
  \# J_N
  \ 
  \lesssim
  \ 
  1
  \,.
\end{align*}
\end{example}
The next question is if we can extend this to countable sets
$\mathcal{X}$. To focus our attention, we assume (the best case)
$M_{N,j}=1$.
\begin{example}
  We give three examples. Let $\propto$ denote the equal-up-to-a-constant order.
  Without loss of generality we assume $\mathcal{X}=\mathbb{N}$.
  First consider
  for some $\varepsilon\in(0,1)$
  \begin{gather*}
    \P[X=k]
    \propto
    k^{-(1+\varepsilon)}
    \,.
  \end{gather*}
  This (barely) is a distribution. But scaling with $V/(V+2)\in(0,1)$ we cause trouble. Indeed, by $V<2$ it holds
  \begin{gather*}
    (
    1+\varepsilon
    )
    \cdot
    \frac{V}{V+2}
    <1
    \,,
  \end{gather*}
  and thus
  \begin{align*}
  \sum_{j=1}^\infty
  \P
  [
  X\in A_{N,j}
  ]^{V/(V+2)}
  &
  \ 
  =
  \ 
  \lim_{N\to\infty}
  \sum_{j\in J_N}
  \P
  [
  X\in A_{N,j}
  ]^{V/(V+2)}
  \\
  &
  \ 
  =
  \ 
  \sum_{k=1}^{\infty} 
  \P
  [
  X=k
  ]^{V/(V+2)}
  \\
  &
  \ 
  \propto
  \ 
  \sum_{k=1}^{\infty} 
  \left( 
  \frac{1}{k}
  \right)
  ^{(1+\varepsilon)\cdot V/(V+2)}
  =\infty
  \,.
  \end{align*}
  On the other hand, the same arguments applied for
  \begin{gather*}
    \P[X=k]
    \propto
    k^{-((V+2)/V+\varepsilon)}
    \,,
  \end{gather*}
  with $\varepsilon>0$ yield
\begin{gather*}
  \sum_{j=1}^\infty
  \P
  [
  X\in A_{N,j}
  ]^{V/(V+2)}
  \lesssim 1
  \,.
\end{gather*}
Finally, let $X$ be Poisson distributed with parameter $\lambda\in(0,1)$.
Then
\begin{align*}
  \sum_{j=1}^\infty
  \P
  [
  X\in A_{N,j}
  ]^{V/(V+2)}
  &
  \ 
  =
  \ 
  \sum_{k=0}^{\infty} 
  \P
  [
  X=k
  ]^{V/(V+2)}
  \\
  &
  \ 
  =
  \ 
  e^{-\lambda\cdot V/(V+2)}
  \sum_{k=0}^{\infty} 
  \left( 
    \frac{\lambda^k}{k\text{!}}
  \right)
  ^{V/(V+2)}
  \\
  &
  \ 
  \le
  \ 
  \sum_{k=0}^{\infty} 
\left( 
    \lambda
  ^{V/(V+2)}
\right)^k
  =
  \frac{1}
  {
    1-
    \lambda
  ^{V/(V+2)}
  }
  \lesssim 1
  \,.
  \end{align*}

\end{example}
In the next example we show, that 
\begin{gather*}
  \sum_{j=1}^\infty
  \P
  [
  X\in A_{N,j}
  ]^{V/(V+2)}
  \to
  \infty
\end{gather*}
for all continuous distributions of $X$.
\begin{example}
  Let $f_X$ be the probability density of $X$. 
  Then there exists a compact set $K\subset\mathcal{X}\subset \R^d$, such that
  $
  \inf_{x\in K}f_X(x)
  >0
  $. Since $\mathcal{P}_N$ are cubic partitions, it holds for 
  \begin{gather*}
    I_N
    :=
    \left\{ 
      i\in\mathbb{N}\colon
      A_{N,i}\subset K
    \right\}
    \qquad
    \text{that}
    \qquad
    \bigcup_{i\in I_N}A_{N,i}\nearrow K
    \,.
  \end{gather*}
Thus
\begin{align*}
  \sum_{i=1}^\infty
  \P
  [
  X\in A_{N,i}
  ]^{V/(V+2)}
  &
  \ 
  \ge
  \ 
  \sum_{i\in I_N}
  \P
  [
  X\in A_{N,i}
  ]^{V/(V+2)}
  \\
  &
  \ 
  \ge
  \ 
  \inf_{x\in K}f_X(x)
  ^{V/(V+2)}
  \cdot 
  h_N^{d\cdot(V/(V+2)-1)}
  \sum_{i\in I_N}
  \lambda
  \left( 
  A_{N,i}
  \right)
  \\
  &
  \ 
  \to
  \ 
  \infty
  \,.
\end{align*}
This follows from
$
  \sum_{i\in I_N}
  \lambda
  \left( 
  A_{N,i}
  \right)
  \to 
  \lambda(K)>0
$,
  $
  \inf_{x\in K}f_X(x)
  >0
  $,
  $V/(V+2)-1<0$ and $h_N\to 0$.

\end{example}

