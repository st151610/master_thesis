The following theorem is an easy consequence of Theorem~\ref{bw:cd:th}.
\begin{ftheorem}
  \label{aa:weights:th}
  Assume that for all $N\in\mathbb{N}$ there exists the (unique) solution $(\lambda^\dagger,\lambda_0^\dagger)$
  to Problem?.
  Consider the weights function defined by
  \begin{gather}
    w(x)
    :=
    (
    f^{'}
    )^{-1}
    \left( 
      \inner{B(x)}{\lambda^\dagger}
      +
      \lambda_0^\dagger
    \right)
    \qquad
    \text{for all}\ 
    x\in\mathcal{X}
    \,.
  \end{gather}
  Under the conditions of Proposition~\ref{bw:cd:th} 
  it holds
  $w(X)\ 
  \overset{\P}{\to}
  \ 
  1/\pi(X)
  $
  .
  Furthmore, there exists a decreasing sequence $(\varepsilon_N)\subset(0,1]$ such that $\varepsilon_N\to 0$ and 
  \begin{gather}
    \P
    \left[ 
  \left| 
  w(X)
  -
  \frac{1}{\pi(X)}
  \right|
    \le
    \varepsilon_N
    \right]
  \  
  \to
  \  
  0
  \qquad
  \text{for}
  \ 
  N\to\infty
  \,.
  \end{gather}
\end{ftheorem}
\begin{proof}
  For all $\varepsilon>0$ it holds
\begin{align}
  \begin{split}
  \left| 
  w(X)
  -
  \frac{1}{\pi(X)}
  \right|
  &
  \ 
  =
  \ 
  \left| 
  (f^{'})^{-1}
  \left( 
    \inner{B(X)}
    {\lambda^\dagger}
    +
    \lambda_0^\dagger
  \right)
  -
  \frac{1}{\pi(X)}
  \right|
  \\
  &
  \ 
  \lesssim
  \ 
  \left| 
    \inner{(B(X),1)}
    {
      (\lambda^\dagger,\lambda_0^\dagger)
      -
      (\lambda^*,0)
    }
  \right|
  +
  \left| 
    \inner{B(X)}
    {
      \lambda^*
      }
    -
    f^{'}
    \left( 
  \frac{1}{\pi(X)}
    \right)
  \right|
  \\
  &
  \ 
  \lesssim
  \ 
  \norm{
      (\lambda^\dagger,\lambda_0^\dagger)
      -
      (\lambda^*,0)
}_2
  +
  \left| 
  \sum_{i=1}^{N} 
  B_k(X)
  \cdot
    f^{'}
    \left( 
  \frac{1}{\pi(X_k)}
    \right)
    -
    f^{'}
    \left( 
  \frac{1}{\pi(X)}
    \right)
  \right|
  \\
  &
  \ 
  \le
  \ 
  \frac{\varepsilon}{2}
+
  \frac{\varepsilon}{2}
  \le
  \varepsilon
  \,,
\end{split}
\end{align}
with probability going to $1$ as $N\to\infty$.
The convergence of the first term follows from Proposition~\ref{bw:cd:th} and that of the second term from \eqref{part_trick}.
The second statement of the theorem follows from the Selection lemma
\cite[A.1.4.]{Steinwart2008}.
\end{proof}
