\subsubsection{Motivation}
The introduction of partitioning estimates~\cite[§4]{Gyorfi2002} - as done in this thesis - was successful.
Thus the implementation of other local averaging regression techniques, such as kernel estimates~\cite[§5]{Gyorfi2002}
is promising. 
\subsubsection{Conjecture}
Similar results as of this thesis hold for basis functions of (boxed) kernel estimates~\cite[§5]{Gyorfi2002}.
They have good practical performance.
\subsubsection{Ideas/Brainstorming}
For boxed kernels it is likely easy to prove a lemma similar to Lemma~\ref{lem:basis_2}.
For kernels with unbounded support, such as gaussian kernels, this might be more difficult. 
Generally, the basis functions should approximate treatment and outcome model well (see \cite[Assumptions~1.6 \& 2.3]{Wang2019}).
Partitioning estimates work well in this thesis, because we can define concrete oracle parameters.
If concrete oracle parameters are not readily available, there might be theoretic results to rely on.
\subsubsection{Organisation}
Get familiar with the notion of (boxed) kernel \cite{Gyorfi2002}.
Find a result similar to Lemma~\ref{lem:basis_2}.
Try other basis functions, maybe with unbounded support.
Rely on theoretic results to derive results such as Lemma~\ref{lem:basis_2} for a more abstract oracle parameter.
\subsubsection{Next Step}
Look up the definition of boxed kernel \cite[Theorem~5.1, Figure~5.7]{Gyorfi2002}.
