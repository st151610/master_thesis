We follow \cite[Example~3.9.19]{vaart2013}
Let 
$Z_1,\ldots,Z_N$
and
$C_1,\ldots,C_N$ 
be independent and identically distributed failure and censoring  times.
Failure and censoring times are assumed independent, that is, 
\begin{align*}
  Z_i \perp C_i
  \qquad
  \text{for all}\ 
  i\in \left\{ 1,\ldots,N \right\}
  \,.
\end{align*}
We only observe the outcome
\begin{align*}
  Y_i
  \ 
  :=
  \ 
  \left( 
  Z_i
  \land
  C_i
  ,
  \Delta_i
  \right)
  \ 
  \qquad
  \text{for all}\ 
  i\in \left\{ 1,\ldots,N \right\}
  \,,
\end{align*}
where 
$
\Delta_i:=
\mathbf{1}
\left\{ 
  Z_i\le C_i
\right\}
$
indicates whether a failure time is censored. 
We consider the weighted Nelson-Aalen estimator for the treated.
\begin{align*}
  \Lambda^1_N(t)
  \ 
  :=
  \ 
  \sum_{i=1}^{N} \frac{
    T_i\cdot w^\dagger_0(X_i)\cdot\mathbf{1}\left\{ Y_i\le t \right\}
    \cdot \Delta_i
  }{
    \sum_{j=1}^{N} 
    T_j\cdot w^\dagger_0(X_j)\cdot\mathbf{1}\left\{ Y_j\ge Y_i \right\}
  } 
  \,.
\end{align*}
Likewise, we can compute weights for the untreated (just switch the treatment status) and get the weighted Nelson-Aalen estimator of the untreated.
This procedure allows to compare treatment and control group while adjusting for imbalances.
The theoretical properties of the Nelson-Aalen estimator as a plug-in estimator are studied in \cite[Example~3.9.19]{vaart2013}.
