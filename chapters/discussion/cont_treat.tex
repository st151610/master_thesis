\subsubsection{Motivation}
In practice, there often exists multiple treatments.
For example, $T\in \left\{
  0,1,2
\right\}$, $T\in I\subset \mathbb{N}$ or even $T\in\R$. 
There exists a general notion of propensity score \cite{Hirano2005}.
There is a need for methods covering this scenarios.
\subsubsection{Conjecture}
The framework of \cite{Wang2019} can be extended to for non-binary treatment.
\subsubsection{Ideas/Brainstorming}
There are already ideas~\cite{Tubbicke2020,Vegetabile2020}. They try to estimate one set of weights to cover all possible treatments.
Jose Zubizarreta, one of the authors of \cite{Wang2019}, told me, that he works on a similar (practical) project.
I think it's better to compute weights for one treatment at a time.
My idea is:
For fixed 
  $ t\in \mathcal{T} $
  this could be
  \begin{gather*}
    \underset{w_1, \ldots, w_n \in \R}{\mathrm{minimize}}
    \qquad
    \sum_{i = 1}^{n}
    d_n(t,T_i)
    \varphi(w_i)
  \end{gather*}
subject to the constraints
\begin{gather*}
%    w_i T_i \ge 0,
%   \qquad 
%    i = 1, \ldots, n,
%   \\
%  \sum_{ i = 1 }^{n}
%    w_i T_i
%  =
%  1
%  \\
    \left| 
      \frac{1}{N} 
      \left(
      \sum_{i = 1}^{n} 
      w_i  
      \cdot
    d_n(t,T_i)
      B_k(X_i)
      - 
      \sum_{i=1}^{N}
      B_k(X_i)
    \right|
      \right)
    \ 
    \le 
    \ 
    \delta_k,
    \qquad
    k = 1, \ldots, N
    \,,
\end{gather*}
where
\begin{gather}
  d_n(t,s)
  :=
  \frac{
  \mathbf{1} _ 
  { s \in N_n(t) }
}{
  \mathbf{\lambda}[N_n(t)]
}
\end{gather}
where $ N_n(t) $ is a neighborhood of $t$ with 
$  
  \mathbf{\lambda}[N_n(t)]
  \to
  0
  $
  for $ n \to \infty $.
  In a consistency proof, such as that of Lemma~\ref{lem:conv_dG},
  we have to control a term like
\begin{align*}
     \frac{1}{N}
     \sum_{i=1}^{N} 
     \left| 
     \,
      1
     \ 
      -
     \ 
     \frac
     {
     d_n(t,T_i)
     }
     {
        h_{T|X}(t,X_i)
     }
     \right|
     \,,
 \end{align*}
where 
$
        h_{T|X}
        $ is the generalized propensity score of \cite{Hirano2005}. 
        We need a result such as
  \begin{align*}
    \E
    \left[ 
      \frac{
        d_n(t,T_i)
      }{
        h_{T|X}(t,X_i)
      }
    \right]
    =
    \E
    \left[ 
      \frac{
        \P
        [ T_i \in N_n(t) | X_i]
      }
      {
        \mathbf{\lambda}[N_n(t)]
      }
      \cdot
      \frac{1}
      {
        h_{T|X}(t,X_i)
      }
    \right]
    \to
    \E
    \left[ 
      \frac
      {
        h_{T|X}(t,X_i)
      }
      {
        h_{T|X}(t,X_i)
      }
    \right]
    =1
    \,.
  \end{align*}
  A possible error decomposition could be derived from
\begin{align*}
  &\left| 
  \frac{1}{n}
  \sum_{i=1}^{n} 
  d_n(t,T_i)
  w_i
  Y_i
  -
  \E[Y(t)]
  \right|
  \\
  &
  \ 
  \le
  \ 
  \left|  
  \frac{1}{n}
    \sum_{i=1}^{n} 
    (
    w_i 
  d_n(t,T_i)
    -
    1
    )
    \inner
    {B(X_i))}
    { \mathbf{Y}(t) }
  \right|
  %%%% 1 %%%%
  \\
  &
  \qquad
  +
  \ 
  \left|  
  \frac{1}{n}
    \sum_{i=1}^{n} 
    (
    w_i 
  d_n(t,T_i)
    -
    1
    )
    \left( 
    \E[Y(t)| X_i]
    -
    \inner
    {B(X_i))}
    { \mathbf{Y}(t) }
    \right)
  \right|
  %%%% 2 %%%%
  \\
  &
  \qquad
  +
  \ 
  \left|  
  \frac{1}{n}
    \sum_{i=1}^{n} 
  d_n(t,T_i)
    \cdot
    (
    w_i 
    -
    1/
h_{T|X}(t,X_i)
    )
    \left( 
      Z_i
    -
    \E[Y(t)| X_i]
    \right)
  \right|
  %%%% 3 %%%%
  \\
  &
  \qquad
  +
  \ 
  \left|  
  \frac{1}{n}
    \sum_{i=1}^{n} 
h_{T}(t)
    /
h_{T|X}(t,X_i)
    \left( 
      Z_i
    -
    \E[Y(t)| X_i]
    \right)
  \right|
  %%%% 3a %%%%
  \\
  &
  \qquad
  +
  \ 
  \left|  
  \frac{1}{n}
    \sum_{i=1}^{n} 
    \left( 
h_{T}(t)
-
  d_n(t,T_i)
    \right)
    /
h_{T|X}(t,X_i)
    \left( 
      Z_i
    -
    \E[Y(t)| X_i]
    \right)
  \right|
  %%%% 3 b %%%%
  \\
  &
  \qquad
  +
  \ 
  \left|  
  \frac{1}{n}
    \sum_{i=1}^{n} 
    \left( 
    \E[Y(t)| X_i]
    -
    \E[Y(t)]
    \right)
  \right|
  %%%% 3 c %%%%
  \\
  &
  \qquad
  +
  \ 
  \left|  
  \frac{1}{n}
    \sum_{i=1}^{n} 
    w_i 
d_n(t,T_i)
    \left( 
      Y_i
    -
    Z_i
    \right)
  \right|
  \,,
\end{align*}
where $ Z_i \sim Y(t)|T_i $.

\subsubsection{Organisation}
Get familiar with generalized propensity score~\cite{Hirano2005}.
Try to adapt ideas to make the proofs work.
Talk to Jose Zubizarreta, one of the authors of \cite{Wang2019}. 
\subsubsection{Next Step}
Read~\cite{Hirano2005}.
