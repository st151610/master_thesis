How does action change an outcome?
How should I guide my actions towards a better outcome?
The first question is about causality, the second about ethics.

How do causality and ethics reflect on statistics?
If you have not spent much time thinking about study design, this is a good way to start: 
As an analyst, ask yourself “Who acted? Who assigned treatments?"
As researcher -- plan your study accurately. You can ask yourself “How do we act? How do we assign treatment? Can we act?"

Let's say, you gather a sample from a study population, assign treatment (but forget how you did it). Some units get the drug, others don't. Then the statistical analysis shows a strong correlation of treatment and outcome. You hurry to your supervisor. “How was treatment assigned", asks she. “I forgot", says you.
“How do you know your analysis is correct then?"
You show her the data and together find out, that all units that received treatment were significantly taller than the rest of the sample.
After all, is the drug or the height responsible for the change  in outcome?
You realise, that the data is worthless for answering this question.
But you are lucky: It is just grass and fertiliser you were studying.

You get a second chance. A new medication needs testing before it enters the market. 
A company shall recruit participants, but the board requires you to write an outline for the study.
You carefully  explain steps to minimize risks for participants. You include plans to meet other requirements of human research.
Then you have to decide how to assign treatment.
No hand waving this time. You talk to your supervisor.
“Last time, too many tall blades received fertiliser. The distribution of treatment was not really random..."
You decide to determine treatment status by the flip a fair coin.
You call the procedure 'randomization'.

%Randomization does not work in important situations. 
Would you smoke if a coin tells you to? If you say yes - you are likely smoke anyway. The point is that forcing someone to smoke is unethical. But so is not studying the risks of smoking.

A professor is curious if the smoking habits of his students affect their grades. 
He observes the smoking area through his field glasses.
His assistant gets to know his plans. He warns him. “Many students attend parties the night before exams. Maybe they are also more likely to smoke." “I shall see this for myself..." says the professor. He puts away the field glasses. After a while, he visits the local club.
He talks to a few of his students. Some smoke, some don't. The chats are enjoyable. He thinks: “Some of best students celebrate \textbf{before} the exam."

I hope, by now it's clear that sometimes it's all about how treatment was assigned. 
The probability of treatment given individual characteristics 
was introduced as the \textbf{propensity score}\cite{Rosenbaum1983}.
In the second example, where you flip a fair coin to decide treatment status, the propensity score is $1/2$.
The coin ignores everything.
What is the propensity score in the other examples?
I admit, I don't know.
It may vary.
But we can see tendencies. 

The propensity score is a simple concept that works well with potential outcomes.
They are potential, because you only get the chance to observe on of them. If you treat, it's the potential outcome under treatment. 
On a high-level: If you act, you can't observe at the same time the effect of no action.
Thus one of the potential outcomes always remains potential.
Of course there are tricks. You can  wait for the effect of an action to vanish and then observe the outcome again.
This works well when the effect of an action is short term.

If treatment assignment is random we actually observe the potential outcome under treatment.
This is because treatment assignment carries no more information.
The coin ignores it.
But we saw, that assignment often contains more information.
Then it is not clear, if the effect on the outcome comes from the new information or the treatment.
Then we don't even observe the potential outcome under treatment, but a confounded version.
A simple idea to obtain information about the true potential outcome is to weight with the inverse probability of treatment, that is, 1 divided by the propensity score. Let's introduce some notation to be more precise.

Let $T\in \left\{ 0,1 \right\}$ be the \textbf{indicator of treatment}. This is a random variable.
Let $X\in\mathcal{X}$ be a vector with individual characteristics. We call this the \textbf{covariate vector}. It is also a random variable.
Last, let $(Y(0),Y(1))$ be the potential outcomes, that is, $Y(0)$ is the potential outcome without treatment and $Y(1)$ the potential outcome with treatment. They are random variables.
We define the propensity score with individual characteristics $x$ to be
$\pi(x):=\P[T=1|X=x]$. We observe  
\begin{gather}
  \text{either}
  \qquad
  Y(0)|T=0
  \qquad
  \text{or}\qquad
  Y(1)|T=1
  \,.
\end{gather}
We saw, that 
$Y(t)|T=t$ does not have the same distribution as $Y(t)$ for $t\in \left\{ 0,1 \right\}$.
We can show, that if 
\begin{gather}
  (Y(0),Y(1))\perp T |X
\end{gather}
holds we get 
\begin{gather}
  \label{ipw}
  \E
  \left[ 
    \frac{T}{\pi(X)}Y(T)
  \right]
  =
  \E
  \left[ Y(1) \right]
  \,.
\end{gather}
That is, by weighting the observed outcome under treatment with the inverse propensity score we recover (in expectation) the potential outcome with treatment. 

If the propensity score is unknown, one method is to use estimates of it.
We hope to recover \eqref{ipw} from the estimate.
But we have to be careful.
After all, we want to extract informations on the potential outcomes and the propensity score is just a tool.

This is, why people started thinking about alternative ways to generate weights.

One way is to solve a constrained optimization problem.



%You sit in a plain room. There are two chairs. You sit in one of them.
%A person enters the room. “Subject 3225057? The coin decided. You have to smoke!"
%On your way out, anguished cries hit you from another room.
%“Please! Don't make me smoke!"
%You feel mild anxiety and wake up. It was just a dream. But it recurs. You talk to a friend. “I dream about a randomized study about the effects of smoking. A cold, indifferent coin decides who smokes. The non-smokers suffer a lot." When you realise, that this is all non-sense, you feel silly. But your friend
%But you happen to be a professor at the university.
%One day, in your office, you lean out of the window.“Give me the field glasses please. I want to observe my students smoking in the backyard."“What for?" asks your assistant. “Let's see, if this affects their grades" you murmur and look through the eyepieces. The issue comes up again at lunch. “Did you consider that some of the students that smoke also go to a party the night before an exam."“Well, maybe I should. How about I see for myself?" The night before the next exam, you openly enter the place of the party. You find some of the students you deem smartest at the bar. You're having a conversation. But it's to loud in there and in the morning you have an headache.
%
%
%%randomized trials versus observational studies
%
%Is study design more important then statistical analysis?
%
%I think, they are at least equal. 
%
%But a bad analysis can be undone,
%whereas a bad design can not.
%
%You have to stick with the data.
%
%If you are not familiar with study design the distinction between randomized and observational study is helpful.
%
%If you read the literature and are unsure about the design of a study, ask for this terms.
%
%You are likely to find an answer.
%
%
%It is all about how we collect the data.
%
%Say, we want to test the effect of a drug in a study population.
%
%There usually are differences among the units of the study population.
%
%Some are more healthy than others.
%
%We form a treatment and control group, that is, one group takes the drug and the other doesn't.
%
%Then we compare the groups by their health. 
%
%Then a critcal review comes in. What do you mean by healthy.
%
%We mean this and that.
%
%It seems you did not consider this factor.
%
%Maybe the drug is not effective, but the effect we see in your analysis comes from something else.
%
%What do we answer to this?
%
%
%A good method to avoid this awkward situation is to randomize.
%
%For every unit of the population we toss a fair coin that decides if they get the drug.
%
%Now comes the critic.
%
%From the tables it seems there is an effect. But what about unknown influence?
%
%We answer: Does the coin now of them?
%
%It is not ideal, but this way you can prevent systematic damage to your analysis.
%
%
%What if we can't decide who gets treatment?
%
%Don't think treatment has to be something good, it should not carry any label of good or bad.
%
%But what about smoking?
%
%Would you smoke if a coin tells you to?
%
%So this is unethical.
%
%But it is also unethical not to investigate the effects of smoking on the health.
%
%Let's accept, that we sometimes (often?) can not control who gets treatment.
%
%Some smoke, some don't, and we mearly observe.
%
%This is typical example for an observational study.
%
%Honestly, this is an oversimplification, but I hope you get the point.
%
%Who still is insulted by the tone will maybe like\cite{Rubin2007}.
%
%%propensity score
%
%In \cite{Rubin2007} you will find the propensity score.
%
%The propensity score is the individual probability to receive treatment, that is,
%\begin{gather}
%  \P
%  [T=1|X]
%\end{gather}
%if $T$ is the random variable that decides abuot treatment and $X$ is the vector that carries your individual information.
%
%This concept goes back to \cite{Rosenbaum1983}.
%
%It is maybe worth to stop here and think about this definition and its connection to the two study designs.
%
%Discover it for yourself.
%
%\begin{reflection*}
%What is the propensity score in the above example.
%How does the propensity score behave in rs and os?
%\end{reflection*}
%
%
%
%
%
%
