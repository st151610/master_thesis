
%randomized trials versus observational studies

Is study design more important then statistical analysis?

I think, they are at least equal. 

But a bad analysis can be undone,
whereas a bad design can not.

You have to stick with the data.

If you are not familiar with study design the distinction between randomized and observational study is helpful.

If you read the literature and are unsure about the design of a study, ask for this terms.

You are likely to find an answer.


It is all about how we collect the data.

Say, we want to test the effect of a drug in a study population.

There usually are differences among the units of the study population.

Some are more healthy than others.

We form a treatment and control group, that is, one group takes the drug and the other doesn't.

Then we compare the groups by their health. 

Then a critcal review comes in. What do you mean by healthy.

We mean this and that.

It seems you did not consider this factor.

Maybe the drug is not effective, but the effect we see in your analysis comes from something else.

What do we answer to this?


A good method to avoid this awkward situation is to randomize.

For every unit of the population we toss a fair coin that decides if they get the drug.

Now comes the critic.

From the tables it seems there is an effect. But what about unknown influence?

We answer: Does the coin now of them?

It is not ideal, but this way you can prevent systematic damage to your analysis.


What if we can't decide who gets treatment?

Don't think treatment has to be something good, it should not carry any label of good or bad.

But what about smoking?

Would you smoke if a coin tells you to?

So this is unethical.

But it is also unethical not to investigate the effects of smoking on the health.

Let's accept, that we sometimes (often?) can not control who gets treatment.

Some smoke, some don't, and we mearly observe.

This is typical example for an observational study.

Honestly, this is an oversimplification, but I hope you get the point.

Who still is insulted by the tone will maybe like\cite{Rubin2007}.

%propensity score

In \cite{Rubin2007} you will find the propensity score.

The propensity score is the individual probability to receive treatment, that is,
\begin{gather}
  \P
  [T=1|X]
\end{gather}
if $T$ is the random variable that decides abuot treatment and $X$ is the vector that carries your individual information.

This concept goes back to \cite{Rosenbaum1983}.

It is maybe worth to stop here and think about this definition and its connection to the two study designs.

Discover it for yourself.

\begin{reflection*}
What is the propensity score in the above example.
How does the propensity score behave in rs and os?
\end{reflection*}






