\subsection{Matrix Stein pairs}
We first define an exchangable pair.
\begin{definition}
  Let 
  $Z$ and $Z^{'}$
  random variables taking values
  in a Polish space $\mathcal{Z}.$
  We say that
  $
  (
    Z
    ,
    Z^{'}
  )
  $
  is an \textbf{exchangable pair}
  if it has the same distribution as 
  $
  (
  Z^{'}
    ,
    Z
  )
  .
  $
  In particular, 
  $Z$ and $Z^{'}$
  must share the same distribution.
\end{definition}

The following approach originates in the work of Charles Stein \cite{Stein1972} 
on normal approximation for a sum of dependent random variable.
We will explain how some central ideas of this theory extends to matrices.

We can obtain a lot of information about the fluctuation of a random matrix
$\mathbf{X}$
if we can construct a good exchangable pair 
$
(
  \mathbf{X}
  ,
  \mathbf{X}^{'}
)
.
$
With this motivation in mind, let us introduce a special class of exchangable pairs.
\begin{definition}
  Let
  $
  (
    Z
    ,
    Z^{'}
  )
  $
  be an exchangable pair of random variables taking values
  in a Polish space $\mathcal{Z},$
  and let 
  $
    \mathbf{\Psi}
    : 
    \mathcal{Z}
    \to 
  \mathbb{H}_d
  $
  be a measurable function.
  Define the random Hermitian matrices
  \begin{gather}
    \mathbf{X}
    :=
    \mathbf{\Psi}
    (Z)
    \quad
    \text{and}
    \quad
    \mathbf{X}^{'}
    :=
    \mathbf{\Psi}
    (Z^{'})
    .
  \end{gather}
  We say that 
  $
  (
    \mathbf{X}
    ,
    \mathbf{X}^{'}
  )
  $
  is a \textbf{matrix Stein pair}
  if there is a constant 
  $\alpha\in (0,1]$
  for which
  \begin{gather}
    \E[
    \mathbf{X}
    -
    \mathbf{X}^{'}
    \vert
    Z
    ]
    =
    \alpha 
    \mathbf{X}
    \qquad
    \text{almost surely.}
  \end{gather}
  The constant 
  $\alpha$
  is called the \textbf{scale factor} of the pair.
  We always assume 
  $
    \E
    \left[ 
      \norm{\mathbf{X}}^2
    \right]
    <
    \infty
    .
  $
\end{definition}

A matrix Stein pair 
$
(
\mathbf{X}
,
\mathbf{X}^{'}
)
$
has several useful propreties. First,
$
(
\mathbf{X}
,
\mathbf{X}^{'}
)
$
always forms an exchangable pair. Second, it must be the case that
$\E[\mathbf{X}]=\mathbf{0}.$
Indeed,
\begin{gather*}
  \E[\mathbf{X}]
  =
  \frac{1}{\alpha}
  \E[
  \E
  [
    \mathbf{X}
    -
    \mathbf{X}^{'}
    \vert
    Z
  ]
  ]
  =
  \frac{1}{\alpha}
  \E[
    \mathbf{X}
    -
    \mathbf{X}^{'}
    \vert
  ]
  =
  \mathbf{0}
  .
\end{gather*}
\subsection{The method of exchangable pairs}
A well-chosen matrix Stein pair 
$
(
\mathbf{X}
,
\mathbf{X}^{'}
)
$
provides a surprisingly powerful tool for studying 
the random matrix $\mathbf{X}.$
The technique depends on a fundamental technical lemma.
\begin{lemma}
  Suppose that
  $
  (
    \mathbf{X}
    ,
    \mathbf{X}^{'}
  )
  $
  is a matrix Stein pair with scale factor $\alpha.$
  Let 
  $
    \mathbf{F}
    :
    \mathbb{H}_d
    \to
    \mathbb{H}_d
  $
  be a measurable function that satisfies the regularity condition
  $
  \E
  \left[
  \norm{
    (
    \mathbf{X}
    -
    \mathbf{X}^{'}
  )
  \mathbf{F}(\mathbf{X})
  }
  \right]
  <
  \infty
  .
  $
  Then
\begin{gather}
  \E
  [
    \mathbf{X}
    \cdot
    \mathbf{F}
    (\mathbf{X})
  ]
  =
  \frac{1}{2\alpha}
  \E
  [
    (
    \mathbf{X}
    -
    \mathbf{X}^{'}
    )
    (
    \mathbf{F}
    (
    \mathbf{X}
    )
    -
    \mathbf{F}
    (
    \mathbf{X}^{'}
    )
  )
  ]
  .
\end{gather}
\end{lemma}
In short, the randomness in the Stein pair furnishes an alternative expression for the expected product of $\mathbf{X}$
and a function $\mathbf{F}.$
It allows us to estimate the expectation using the smoothness properties of the function $\mathbf{F}$ and the discrepancy between $\mathbf{X}$
and $\mathbf{X}^{'}.$
\begin{proof}
  \emph{\cite[Lemma~2.4]{Mackey2014}}
  Suppose that
  $
  (
    \mathbf{X}
    ,
    \mathbf{X}^{'}
  )
  $
  constructed from an auxiliary exchangable pair 
  $
  (
  Z,
  Z^{'}
  )
  .
  $
  The defining property implies
 \begin{gather}
   \alpha \cdot 
   \E[
   \mathbf{X}
   \cdot
   \mathbf{F}
   (\mathbf{X})
   ]
   =
   \E[
    \E[
    \mathbf{X}
    -
    \mathbf{X}^{'}
    \vert
    Z
    ]
    \cdot
   \mathbf{F}
   (\mathbf{X})
   ]
   =
    \E[
    (
    \mathbf{X}
    -
    \mathbf{X}^{'}
    )
   \mathbf{F}
   (\mathbf{X})
   ]
 \end{gather} 
\end{proof}
\subsection{The conditional variance}
To each matrix Stein pair 
  $
  (
    \mathbf{X}
    ,
    \mathbf{X}^{'}
  ),
  $
  we may associate a random matrix called the conditional variance of $\mathbf{X}.$
  The purpose of this section is to argue that the spectral norm of $\mathbf{X}$
  is unlikely to be large, when the conditional variance is small.
\begin{definition}
  Suppose that
  $
  (
    \mathbf{X}
    ,
    \mathbf{X}^{'}
  ),
  $
  is a matrix Stein pair, constructed from an auxiliary exchangeable pair
  $
  (
  Z
    ,
  Z^{'}
  ).
  $
  The \textbf{conditional variance}
  is the random matrix
  \begin{gather}
    \mathbf{\Delta_X}
    :=
    \mathbf{\Delta_X}
    (Z)
    :=
    \frac{1}{2\alpha}
    \E
    [
    (
    \mathbf{X}
    -
    \mathbf{X}^{'}
    )^2
    \vert
    Z
    ]
    ,
  \end{gather}
  where $\alpha$ is the scale factor of the pair. We may take any version of the conditional expectation in this definition.
\end{definition}

The conditional variance
$
    \mathbf{\Delta_X}
$
can be regarded as a stochastic estimate for the variance 
of the random matrix $\mathbf{X}.$
To see this, assume
the second moment of $\mathbf{X}$ exists. Then it follows from Lemma 
with $\mathbf{F}(\mathbf{X})=\mathbf{X}$
\begin{gather}
  \E
  [
    \mathbf{\Delta_X}
  ]
  =
  \E
  [
  \mathbf{X}^2
  ]
  .
\end{gather}
To verify the regularity condition, note that
\begin{gather}
  \E
  [
  \lVert
    (
    \mathbf{X}
    -
    \mathbf{X}^{'}
    )
    \mathbf{X}
    \rVert
  ]
  \le
  \E[
  \norm{
    \mathbf{X}
  }
  ^2
  ]
  +
  \E
  [
  \norm{
    \mathbf{X}
  }
  \cdot
  \lVert
    \mathbf{X}^{'}
    \rVert
  ]
  \le
  2
  \E
  [
  \norm{
    \mathbf{X}
  }
  ^2
  ]
  <
  \infty
  .
\end{gather}
\begin{example}
  \emph{\cite[Example~2.4]{Mackey2014}}
\end{example}
nrt\\
is not necessary, as, for example, the normal distribution does not fulfill it. For
