\begin{definition}
  \label{rmineq_intrinsic_bernstein}
  For a positive-semidefinite matrix $\mathbf{S}$,
  the \textbf{intrinic dimension} is the quantity
  \begin{gather*}
    \mathrm{intdim}
    (\mathbf{A})
    :=
    \frac{\mathrm{tr}\mathbf{A}}{\norm{\mathbf{A}}}
    .
  \end{gather*}
\end{definition}
\begin{lemma}
  \emph{(Intrinsic dimenision)}
  Let 
  $
    \varphi: [0,\infty) \to \R
  $
  be a convex function with
  $
    \varphi(0)=0.
  $
  For any positive-semidefinite matrix $\mathbf{S}$ it holds that
  \begin{gather*}
    \mathrm{tr}(\varphi(\mathbf{S}))
    \le
    \mathrm{intdim}(\mathbf{S})
    \cdot
    \varphi(\norm{\mathbf{S}})
    .
  \end{gather*}
\end{lemma}
\begin{proof}
  \emph{\cite[Lemma~7.5.1]{Tropp2015}}
  Since $\varphi$ is convex on any interval $[0,L]$ with $L>0$ and $\varphi(0)=0$, it holds
  \begin{gather}
    \varphi(a)
    \le
    \left( 
      1 - \frac{a}{L}
    \right)
    \varphi(0)
    +
    \frac{a}{L}
    \varphi(L)
    =
    \frac{a}{L}
    \varphi(L)
    \qquad
    \text{for all}\ 
    a \in [0,L]
    .
  \end{gather}
  Since $\mathbf{S}$ is positive-semidefinite, the eigenvalues of $\mathbf{S}$ 
  fall in the interval $[0,L]$, where $L=\norm{\mathbf{S}}.$
  \begin{gather}
    \mathrm{tr}(\varphi(\mathbf{S}))
    =
    \sum_{i=1}^{d}
    \varphi(\lambda_i)
    \le
    \frac{
    \sum_{i=1}^{d}
    \lambda_i
    }{\norm{\mathbf{S}}}
    \varphi(\norm{\mathbf{S}})
    =
    \frac{\mathrm{tr}(\mathbf{S})}{\norm{\mathbf{S}}}
    \varphi(\norm{\mathbf{S}})
    =
    \mathrm{intdim}(\mathbf{S})
    \cdot
    \varphi(\norm{\mathbf{S}})
    .
  \end{gather}
\end{proof}
The next example applies the preceding lemma to bound the Schatten-norm in terms of the spectral norm and the intrinsic dimension.
  \begin{example*}
    Let 
    $
      \mathbf{B} \in \mathbb{C}^{m\times n}
    $
    be any rectangular matrix and let $p\ge 2$.
    Then 
    $
      \varphi(x)
      :=
      \left| x \right|
    $
  defines a convex function with $\varphi(0)=0$.
  The intrinsic dimension lemma yields
  \begin{gather}
    \norm{B}^p_p
    =
    \mathrm{tr}
      \left|
      B^* B
      \right|
      ^{p/2}
    \le
    \mathrm{intdim}
    (
      B^* B
    )
      \cdot
      \norm{
      B^* B
    }^{p/2}
    =
    \mathrm{intdim}
    (
      B^* B
    )
      \cdot
      \norm{
        B
    }^{p}
    \,.
  \end{gather}
  If, additionally, $\mathbf{B}$ is self-adjoint and positive-semidefinite
  then it holds 
  \begin{gather}
    \mathrm{tr}
    (
      B^* B
    )
    =
    \mathrm{tr}
    (
    B^2
    )
    =
    \sum_{i=1}^{n} 
    \lambda_i^2
    \le 
    \left( 
    \sum_{i=1}^{n} 
    \lambda_i
    \right)
    ^{2}
    =
    \left( 
    \mathrm{tr}
    B
    \right)
    ^2
    \,,
  \end{gather}
  und consequently
  \begin{gather}
    \norm{B}^p_p
    \le
    \left( 
    \mathrm{intdim}
    B
    \right)
    ^2
      \cdot
      \norm{
        B
    }^{p}
    \,.
  \end{gather}
  \end{example*}
\begin{takeaways}
  Is it intrinsic or extrinsic?
  \lipsum[4]
\end{takeaways}
