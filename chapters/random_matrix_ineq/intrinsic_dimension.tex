\begin{definition*}
  \label{rmineq_intrinsic_bernstein}
  For a positive-semidefinite matrix $\,\mathbf{S}$,
  the \textbf{intrinic dimension} is the quantity
  \begin{gather*}
    \mathrm{intdim}
    \,
    \mathbf{A}
    \ 
    :=
    \ 
\mathrm{tr}\,\mathbf{A}
\,
/
\,
\norm{\mathbf{A}}
    \,
    .
  \end{gather*}
\end{definition*}
\begin{lemma*}
  \emph{(Intrinsic dimenision)}
  Let 
  $
  \,
    \varphi\,:\, [\,0,\infty) \,\to\, \R
    \,
  $
  be a convex function with
  $
    \varphi(0)\,=\,0
  $.
  For any positive-semidefinite matrix $\,\mathbf{S}$ it holds 
  \begin{gather*}
    \mathrm{tr}\,\varphi(\mathbf{S})
    \ 
    \le
    \ 
    \mathrm{intdim}\,\mathbf{S}
    \ 
    \cdot
    \ 
    \varphi\!\norm{\mathbf{S}}
    \,
    .
  \end{gather*}
\end{lemma*}
\begin{proof}
  \emph{\cite[Lemma~7.5.1]{Tropp2015}}
  Since $\varphi$ is convex on any interval $[\,0,L\,]$ with $L\,>\,0$, and $\varphi(0)\,=\,0$, it holds
  \begin{gather*}
    \varphi(a)
    \ 
    \le
    \ 
    \left( 
      1 - a/L
    \right)
    \cdot
    \varphi(0)
    \ 
    +
    \ 
    a/L
    \cdot
    \varphi(L)
    \ 
    =
    \ 
    a/L
    \cdot
    \varphi(L)
    \qquad
    \text{for all}\ 
    a \,\in\, [\,0,L\,]
    \,
    .
  \end{gather*}
  Since $\,\mathbf{S}\,$ is positive-semidefinite, the eigenvalues of $\,\mathbf{S}\,$ 
  fall in the interval $\,[\,0,L\,]\,$, where $\,L\,=\,\norm{\mathbf{S}}\,$.
  It follows 
  \begin{align*}
    \mathrm{tr}\,\varphi(\mathbf{S})
    &
    \ 
    =
    \ 
    \sum_{i=1}^{d}
    \varphi(\lambda_i)
    \ 
    \le
    \ 
    \sum_{i=1}^{d}
    \lambda_i
    /
    \norm{\mathbf{S}}
    \ 
    \cdot
    \ 
    \varphi(\norm{\mathbf{S}})
    \\
    &
    \ 
    =
    \ 
    \mathrm{tr}(\mathbf{S})
    /
    \norm{\mathbf{S}}
    \ 
    \cdot
    \ 
    \varphi(\norm{\mathbf{S}})
    \ 
    =
    \ 
    \mathrm{intdim}\,\mathbf{S}
    \ 
    \cdot
    \ 
    \varphi\!\norm{\mathbf{S}}
    \,
    .
  \end{align*}
\end{proof}
The next example applies the preceding lemma to bound 
the $p$-Schatten-norm, when $\,p\,\ge\, 2\,$, by the spectral norm and the intrinsic dimension.
  \begin{example*}
    Let 
    $
    \,
      \mathbf{B} \in \mathbb{C}^{m\times n}
      \,
    $
    be any rectangular matrix and let $\,p\ge 2\,$.
    Then 
    $
    \,
      \varphi(x)
      \,
      :=
      \,
      \left| x \right|
      \,
    $
  defines a convex function with $\,\varphi(0)\,=\,0\,$.
  The intrinsic dimension lemma yields
  \begin{gather*}
    \norm{\mathbf{B}}^{\,p}_p
    \ 
    =
    \ 
    \mathrm{tr}
      \left|
      \mathbf{B}^* \mathbf{B}
      \right|
      ^{\,p/2}
      \ 
    \le
    \ 
    \mathrm{intdim}
    \,
      \mathbf{B}^* \mathbf{B}
      \ 
      \cdot
      \ 
      \norm{
      \mathbf{B}^* \mathbf{B}
    }^{\,p/2}
    \ 
    =
    \ 
    \mathrm{intdim}
    \,
      \mathbf{B}^* \mathbf{B}
      \ 
      \cdot
      \ 
      \norm{
        \mathbf{B}
    }^{\,p}
    \,.
  \end{gather*}
  If, additionally, $\mathbf{B}$ is self-adjoint and positive-semidefinite
  then it holds 
  \begin{gather*}
    \mathrm{tr}
    \,
      \mathbf{B}^* \mathbf{B}
      \ 
    =
    \ 
    \mathrm{tr}
    \,
    \mathbf{B}^{\,2}
    \ 
    =
    \ 
    \sum_{i=1}^{n} 
    \,
    \lambda_i^{\,2}
    \ 
    \le 
    \ 
    \left( 
      \,
    \sum_{i=1}^{n} 
    \,
    \lambda_i
    \,
    \right)
    ^{\!2}
    \ 
    =
    \ 
    \left( 
    \mathrm{tr}
    \,
    \mathbf{B}
    \right)
    ^{\,2}
    \,,
  \end{gather*}
  and consequently
  \begin{gather*}
    \norm{\mathbf{B}}^{\,p}_{p}
    \ 
    \le
    \ 
    \left( 
    \mathrm{intdim}
    \,
    \mathbf{B}
    \right)
    ^{\,2}
    \ 
      \cdot
      \ 
      \norm{
        \mathbf{B}
    }^{\,p}
    \,.
  \end{gather*}
  \end{example*}


  \todo[color=red!40, inline]{Apply to estimate \cite[above (A.4)]{Chen2012} to get intrinsic dimension version of Rosenthal-Pinelis inequality.}
\begin{takeaways}
  The notion of intrinsic dimension is useful when bounding convex
  functions of a positive-semidefinite matrix by its spectral norm.
  We saw how to derive bounds on the $p\,$-Schatten-norm when $\,p\,\ge\, 2$.
\end{takeaways}
