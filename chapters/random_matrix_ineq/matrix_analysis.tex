  The \textbf{trace} of a square matrix, denoted by $\mathrm{tr},$
  is the sum of its diagonal entries, i.e. 
  $
    \mathrm{tr}(\mathbf{B})
    =
    \sum_{j=1}^{d}b_{jj}
    \quad 
    \text{for}\ 
    \mathbf{B} \in \mathbb{M}_d.
  $
  The trace is unitarily invariant, i.e.
  $
    \mathrm{tr}(\mathbf{B})
    =
    \mathrm{tr}(\mathbf{Q}\mathbf{B}\mathbf{Q}^*)
    \quad 
    \text{for all}
    \ 
    \mathbf{B}\in \mathbb{M}_d
    \ 
    \text{for all unitary}\ 
    \mathbf{Q} \in \mathbb{M}_d.
  $
  In particular, the existence of an eigenvalue value decomposition shows 
  that the trace of a Hermitian matrix equals the sum of its  eigenvalues.
  Let
  $
  f: I\to \R
  $
  where 
  $I\subseteq\R$ 
  is an interval.
  Consider a matrix 
  $\mathbf{A}\in \mathbb{H}_d$
  whose eigenvalues are contained in $I.$
  We define the matrix 
  $
    f(\mathbf{A})\in \mathbb{H}_d
  $
  using an eigenvalue decomposition of $\mathbf{A}:$
  \begin{gather}
    f(\mathbf{A})
    =
    \mathbf{Q}
    \begin{bmatrix}
      f(\lambda_1) &&\\
                   &\ddots&\\
                   && f(\lambda_d)
    \end{bmatrix}
    \mathbf{Q}^*
    \qquad
    \text{where}
    \qquad
    \mathbf{A}
    =
    \mathbf{Q}
    \begin{bmatrix}
      \lambda_1 &&\\
                   &\ddots&\\
                   && \lambda_d
    \end{bmatrix}
    \mathbf{Q}^*
    .
  \end{gather}
  The definition of $f(\mathbf{A})$ does not depend on which 
  eigenvalue decomposition we choose.
  Any matrix function that arises in this fashion is called a \textbf{standard matrix function}.


\begin{proposition}
  Let
  $
  f,g: I\to \R
  $
  be real-valued functions on an interval $I\subseteq\R,$ 
  and let
  $\mathbf{A}\in \mathbb{H}_d$
  be a Hermitian matrix
  whose eigenvalues are contained in $I.$

  \begin{enumerate}[label={(\roman*)}]
    \item
      If $\lambda$ is an eigenvalue of of $\mathbf{A},$
      then $f(\lambda)$ is an eigenvalue of $f(\mathbf{A}).$
    \item
      $
        f(a)
        \le
        g(a)
        \quad
        \text{for all}\ 
        a\in I
        \quad
        \text{implies}
        \quad
        f(\mathbf{A})
        \preccurlyeq
        g(\mathbf{A})
        .
      $
  \end{enumerate}
\end{proposition}


\begin{lemma}
  \emph{(Mean value trace inequality)}
  Let 
  $I$
  be an interval of the real line. Suppose that
  $
    g:
    I \to \R
  $
  is a weakly increasing function and that 
  $
    h:
    I \to \R
  $
  is a function whose derivative $h^{'}$ is convex.
  Then for all matrices 
  $
    \mathbf{A}
    ,
    \mathbf{B}
    \in 
    \mathbb{H}_d(I)
  $
  it holds
  \begin{gather}
    \overline{\mathrm{tr}}
    [
    (
      g(\mathbf{A}) - g(\mathbf{B})
    )
    \cdot
    (
      h(\mathbf{A}) - h(\mathbf{B})
    )
    ]
    \le
    \frac{1}{2}
    \,
    \overline{\mathrm{tr}}
    [
    (
      g(\mathbf{A}) - g(\mathbf{B})
    )
    \cdot
    (
    \mathbf{A}
    -
    \mathbf{B}
    )
    \cdot
    (
    h^{'}(\mathbf{A}) + h^{'}(\mathbf{B})
    )
    ]
    .
  \end{gather}
  When $h^{'}$ is concave, the inequality is reversed. The same result holds for the standard trace.
\end{lemma}
\begin{proof}
  \emph{\cite[Lemma~3.4]{Mackey2014}}
  Fix 
  $
    a,b
    \in
    I
.
  $
  Since 
  $g$
  is
  weakly increasing,
  $
  (
    g(a) - g(b)
  )
  \cdot
  (a-b)
  \ge 0.
  $
  The 
  fundamental theorem of calculus and the convexity of 
  $h^{'}$
  yield the estimate
  \begin{align}
  (
    g(a) - g(b)
  )
  \cdot
  (
    h(a) - h(b)
  )
  &=
  (
    g(a) - g(b)
  )
  \cdot
  (a-b)
  \int_0^1
  h^{'}
  (
    \tau a + (1-\tau) b
  )
  \mathrm{d}\tau
  \\
  &\le
  (
    g(a) - g(b)
  )
  \cdot
  (a-b)
  \int_0^1
  [
  \tau
  h^{'}
  (
     a 
  )
  +
  (1-\tau)
  h^{'}
  (
     b 
  )
  ]
  \mathrm{d}\tau
  \\
  &=
  \frac{1}{2}
  \,
  [
  (
    g(a) - g(b)
  )
  \cdot
  (a-b)
  \cdot  
  (
    h^{'}
    (a)
    +
    h^{'}
    (b)
  )
  ]
  .
  \end{align}
  The inequality is reversed, if $h^{'}$ is concave.
\end{proof}

\begin{proposition}
  \emph{(Generalized Klein inequality)}
  Let 
  $
    u_1, \ldots, u_n
  $
  and
  $
    v_1, \ldots, v_n
  $
  be real-valued functions on an interval $I$
  of the real line.
  Suppose
  \begin{gather}
    \sum_{k=1}^{n}
    u_k(a)
    v_k(b)
    \ge
    0
    \qquad
    \text{for all}
    \ 
    a,b \in I
    .
  \end{gather}
  Then
  \begin{gather}
    \overline{\mathrm{tr}}
    \left( 
    \sum_{k=1}^{n}
    u_k(\mathbf{A})
    v_k(\mathbf{B})
    \right)
    \ge 0
    \qquad
    \text{for all}
    \ 
    \mathbf{A}, \mathbf{B} \in \mathbb{H}_d(I)
    .
  \end{gather}
\end{proposition}
\begin{proof}
  \emph{\cite[Proposition~3]{Petz1994}}
\end{proof}




\begin{proposition}
  \emph{(Hölder inequality for trace)}
  Let 
  $p$ and $q$
  be Hölder conjugate indices.
  Then
  \begin{gather}
    \mathrm{tr}
    (
    \mathbf{BC}
    )
    \le
    \norm{\mathbf{B}}_p
    \norm{\mathbf{C}}_q
    \qquad
    \text{for all}
    \ 
    \mathbf{B}
    ,
    \mathbf{C}
    \in 
    \mathbb{M}_d
    .
  \end{gather}
\end{proposition}
\begin{proof}
  \cite[Corollary~IV.2.6]{Bhatia1997}
\end{proof}
