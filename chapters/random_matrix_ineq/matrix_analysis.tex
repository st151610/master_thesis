  The \textbf{trace} of a square matrix, denoted by $\mathrm{tr},$
  is the sum of its diagonal entries, i.e. 
  $
    \mathrm{tr}(\mathbf{B})
    =
    \sum_{j=1}^{d}b_{jj}
    \quad 
    \text{for}\ 
    \mathbf{B} \in \mathbb{M}_d.
  $
  The trace is unitarily invariant, i.e.
  $
    \mathrm{tr}(\mathbf{B})
    =
    \mathrm{tr}(\mathbf{Q}\mathbf{B}\mathbf{Q}^*)
    \quad 
    \text{for all}
    \ 
    \mathbf{B}\in \mathbb{M}_d
    \ 
    \text{for all unitary}\ 
    \mathbf{Q} \in \mathbb{M}_d.
  $
  In particular, the existence of an eigenvalue value decomposition shows 
  that the trace of a Hermitian matrix equals the sum of its  eigenvalues.
  Let
  $
  f: I\to \R
  $
  where 
  $I\subseteq\R$ 
  is an interval.
  Consider a matrix 
  $\mathbf{A}\in \mathbb{H}_d$
  whose eigenvalues are contained in $I.$
  We define the matrix 
  $
    f(\mathbf{A})\in \mathbb{H}_d
  $
  using an eigenvalue decomposition of $\mathbf{A}:$
  \begin{gather}
    f(\mathbf{A})
    =
    \mathbf{Q}
    \begin{bmatrix}
      f(\lambda_1) &&\\
                   &\ddots&\\
                   && f(\lambda_d)
    \end{bmatrix}
    \mathbf{Q}^*
    \qquad
    \text{where}
    \qquad
    \mathbf{A}
    =
    \mathbf{Q}
    \begin{bmatrix}
      \lambda_1 &&\\
                   &\ddots&\\
                   && \lambda_d
    \end{bmatrix}
    \mathbf{Q}^*
    .
  \end{gather}
  The definition of $f(\mathbf{A})$ does not depend on which 
  eigenvalue decomposition we choose.
  Any matrix function that arises in this fashion is called a \textbf{standard matrix function}.


\begin{proposition}
  Let
  $
  f,g: I\to \R
  $
  be real-valued functions on an interval $I\subseteq\R,$ 
  and let
  $\mathbf{A}\in \mathbb{H}_d$
  be a Hermitian matrix
  whose eigenvalues are contained in $I.$

  \begin{enumerate}[label={(\roman*)}]
    \item
      If $\lambda$ is an eigenvalue of of $\mathbf{A},$
      then $f(\lambda)$ is an eigenvalue of $f(\mathbf{A}).$
    \item
      $
        f(a)
        \le
        g(a)
        \quad
        \text{for all}\ 
        a\in I
        \quad
        \text{implies}
        \quad
        f(\mathbf{A})
        \preccurlyeq
        g(\mathbf{A})
        .
      $
  \end{enumerate}
\end{proposition}
