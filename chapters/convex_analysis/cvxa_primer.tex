Excursively, we present some well known definitions and facts from
convex analysis. For details, see, e.g., \cite{Mordukhovich2022}.

A subset $C\subseteq \R^n$ is called \textbf{convex set}, 
if for all $x,y\in C$ and all $\lambda\in [0,1]$,
we have 
$
  \lambda x + (1-\lambda)y 
  \in
  C
  .
$
The Cartesian product of convex sets is convex. The intersection of a collection of convex sets is also convex.




Given (not necessary convex) sets 
$
  \Omega,
  \Omega_1,
  \Omega_2
  \subseteq 
  \R^n
$
and
$\lambda\in\R,$
define the \textbf{set addition} and \textbf{multiplication}
by a real scalar as 
$
  \Omega_1 
  +
  \Omega_2 
  :=
  \left\{ 
    x_1 + x_2 
    \colon
    x_1 \in \Omega_1
    ,
    x_2 \in \Omega_2
  \right\}
$
and
$
  \lambda \Omega
  :=
  \left\{ 
    \lambda x
    \colon
    x\in\Omega
  \right\}.
$
For convex sets the addition and multiplication by a real scalar are convex.

Throughout this section, we shall denote by 
$
  B
  :=
  \left\{ 
    x=
    [x_1, \ldots, x_n]^\top
    \in\R^n
    \colon
    (
    \sum_{i=1}^{n} 
    x_i^2
    )
    ^{1/2}
    \le 1
  \right\}
$
\todo[color=green!40,inline]{Solve editorial issue with ball.}
the \textbf{Euclidian unit ball} in $\R^n.$
This is a closed convex set. For any $a\in\R^n,$ 
the \textbf{ball with radius $\varepsilon >0$ and center $a$}
is given by
$
  \left\{ 
   a+x 
    \in\R^n
    \colon
    (
    \sum_{i=1}^{n} 
    x_i^2
    )
    ^{1/2}
    \le \varepsilon
  \right\}
  =
  a
  +
  \varepsilon B
  .
$
For any set $\Omega$ in $\R^n,$ 
the set of points $x$ whose distance from $\Omega$
does not exceed $\varepsilon$ is 
$
\Omega + \varepsilon B.
$
The \textbf{closure} $\mathrm{cl}(\Omega)$
and \textbf{interior} $\mathrm{int}(\Omega)$
of $\Omega$
can therefore be expressed by 
$
  \mathrm{cl}(\Omega)
  =
  \bigcap_{\varepsilon>0}
  \Omega + \varepsilon B
$
and
$
  \mathrm{int}(\Omega)
  =
  \left\{ 
    x\in \Omega
    \colon
    \text{there exists $\varepsilon>0$ such that }
    x+\varepsilon B
    \subseteq
    \Omega
  \right\}
  .
$

  A set 
  $A\subseteq \R^n$
  is called \textbf{affine set}, if
  $
    \alpha x + (1-\alpha)y \in A
    \quad
    \text{for all}
    \ 
    x,y \in A
    \ 
    \text{and}
    \ 
    \alpha \in \R.
  $
  The \textbf{affine hull} 
  $\mathrm{aff}(\Omega)$
  of a set 
  $\Omega\subseteq \R^n$
  is the smallest affine set that includes $\Omega.$
A mapping 
$
  A: \R^n \to \R^m
$
is called \textbf{affine mapping} if there exist a linear mapping
$
  L: \R^n \to \R^m
$
and a vector $b\in \R^m$
such that
$
  A(x)
  =
  L(x)
  +
  b
  \ 
  \text{for all}\ 
  x \in \R^n
  .
$
The image and inverse image/preimage of convex sets under affine mappings are also convex.





Because the notion of interior is not precise enough for our purposes
we define the relative interior which is the interior relative to the affine hull.
This concept is motivated by the fact that a line segment embedded in $\R^2$
does have a natural interior in $\R$ which is not a true interior in $\R^2.$
The relative interior of $C$ is defined as the interior which results when
$C$ is regarded as a subset of its affine hull.
\begin{definition}
  Let 
  $\Omega\subseteq \R^n.$
  We define the \textbf{relative interior} of $\Omega$ by
  \begin{gather}
    \mathrm{ri}(\Omega)
    :=
    \left\{ 
      x \in \Omega 
      \colon
      \text{there exists}\ 
      \varepsilon > 0\ 
      \text{such that}\ 
      (
        x+\varepsilon B
      )
      \cap
      \mathrm{aff}(\Omega)
      \subset
      \Omega
    \right\}.
  \end{gather}
\end{definition}
Next we collect some useful properties of relative interiors.
\begin{proposition}
  Let $C$ be a non-empty convex set in $\R^n.$ Then we get the representation
\begin{enumerate}[label={(\roman*)}]
  \item
    $
    \mathrm{ri}(C)
    =
    \left\{ 
      z \in C
      \colon
      \text{for all}\ 
      x \in C \ 
      \text{there exists}\ 
      t > 0 \ 
      \text{such that}\ 
      z + t (z-x)
      \in C
    \right\}.
    $
  \item
    $
      \mathrm{ri}(C)
      \neq
      \emptyset
      \ 
      \text{if}\ 
      C\neq \emptyset
      .
    $
  \item
    $
      \mathrm{cl}(C)
      \ 
      \text{and}\ 
      \mathrm{ri}(C)
    $
    are convex sets.
  \item
    $
      \mathrm{cl}(\mathrm{ri}(C))
      =
      \mathrm{cl}(C)
      \ 
      \text{and}
      \  
      \mathrm{ri}(\mathrm{cl}(C))
      =
      \mathrm{ri}(C)
      .
    $
  \item
    Suppose
    $
      \bigcap_{i\in I} C_i
      \neq
      \emptyset
    $
    for a finite index set $I.$
    Then
    $
      \mathrm{ri}
      \left( 
        \bigcap_{i\in I} C_i
      \right)
      =
      \bigcap_{i\in I}  
      \mathrm{ri}(C_i)
      .
    $
    \item
      Let 
      $
        L:\R^n \to \R^m
      $
      be a linear mapping. Then
      $
        \mathrm{ri}(L(C))
        =
        L(\mathrm{ri}(C))
        .
      $
      If additionally it holds
      $
        L^{-1}(\mathrm{ri}(C))
        \neq
        \emptyset
      $
      we have
      $
      \mathrm{ri}(L^{-1}(C))
        =
        L^{-1}(\mathrm{ri}(C))
        .
      $
      \item
        $
          \mathrm{ri}(C_1\times C_2)
          = 
          \mathrm{ri}(C_1)
          \times
          \mathrm{ri}(C_2).
        $
      \item
        $
          \mathrm{ri}(C_1)
          \cap
          \mathrm{ri}(C_2)
          =
          \emptyset
        $
        if and only if
        $
          0 \notin
          \mathrm{ri}
          (C_1 - C_2)
          .
        $
\end{enumerate}

\end{proposition}

\todo[color=green!40,inline]{Order results to give pretty proof.}

\begin{proof}
\begin{enumerate}[label={(\roman*)}]
  \item
    \cite[Theorem~6.4]{Rockafellar1970}
  \item
    \cite[Theorem~6.2]{Rockafellar1970}
  \item
    \cite[Theorem~6.2]{Rockafellar1970}
  \item
    \cite[Theorem~6.3]{Rockafellar1970}
  \item
    \cite[Theorem~6.5]{Rockafellar1970}
  \item
    \cite[Theorem~6.6-6.7]{Rockafellar1970}
  \item
Let
  $
  (z_1, z_2)
  \in 
  \mathrm{ri}(C_1\times C_2).
  $
  Then for all 
  $
  (x_1, x_2)
  \in 
  C_1\times C_2
  $
  there exists
  $t>0$
  such that
  \begin{gather}
      z_i + t (z_i-x_i)
      \in C_i
      \qquad
      \text{for}\ 
      i\in \left\{ 1,2 \right\}.
  \end{gather}
  This proves $\subseteq.$
  Suppose 
  $
    z_1 
    \in
    \mathrm{ri}(C_1)
  $
  and
  $
    z_2 
    \in
    \mathrm{ri}(C_2).
  $
  Let
  $
    (x_1,x_2)\in C_1\times C_2
  $
  with
  If $t_1=t_2$ everything is clear.
  W.l.o.g.
  assume 
  $t_1<t_2$. Define $\theta:=\frac{t_1}{t_2}\in (0,1).$
  By the convexity of $C_2$ it follows
  \begin{gather}
    z_2 + t_1 (z_2 - x_2)
    =
    \theta
    (
    z_2 + t_2 (z_2 - x_2)
    )
    +
    (1-\theta)z_2
    \in C_2.
  \end{gather}
  Thus
  $
  (z_1,z_2)\in\mathrm{ri}(C_1\times C_2). 
  $
  This proves $\supseteq$ and equality.
  \item
    \cite[Theorem~2.92]{Mordukhovich2022}
\end{enumerate}
\end{proof}
 
We procede with convex separation results which are vital to the subsequent developments.

\begin{definition}
  Let 
  $C_1$ and $C_2$
  be two non-empty convex sets in $\R^n$. 
  A hyperplane $H$ is said to \textbf{separate}
  $C_1$ and $C_2$
  if $C_1$ is contained in one of the closed half-spaces associated with
  $H$ and $C_2$ lies in the opposite closed half-space. It is said to separate 
  $C_1$ and $C_2$
  \textbf{properly} if 
  $C_1$ and $C_2$
  are not \textit{both} actually contained in $H$ itselef.
\end{definition}
\begin{theorem}
  Let $C_1$ and $C_2$ be two non-empty convex sets in $\R^n$. 
  There exists a hyperplane separating
  $C_1$ and $C_2$
  properly 
  if and only if
  there exists a vector $b\in \R^n$ such that
  \begin{gather}
    \sup_{x\in C_2} \inner{x}{b}
    \le
    \inf_{x\in C_1} \inner{x}{b}
    \quad 
    \text{and}
    \quad 
    \inf_{x\in C_2} \inner{x}{b}
    <
    \sup_{x\in C_1} \inner{x}{b}
    .
  \end{gather}
\end{theorem}
\begin{proof}
  \cite[Theorem~11.1]{Rockafellar1970}
\end{proof}
\begin{ftheorem}
  \emph{(Convex separation in finite dimension)}
  Let $C_1$ and $C_2$ be two non-empty convex sets in $\R^n$. 
  Then $C_1$ and $C_2$ can be properly separated if and only if 
  $\mathrm{ri}(C_1)\cap\mathrm{ri}(C_2)=\emptyset.$
\end{ftheorem}
\begin{proof}
  \cite[Theorem~11.3]{Rockafellar1970}
\end{proof}




\begin{definition}
  Given a nonempty subset 
  $\Omega \subseteq \R^n$
  the \textbf{support function} 
  $
  \sigma_\Omega : \R^n \to \overline{\R}
  $
  of $\Omega$
  is defined by
  \begin{gather}
    \sigma_\Omega
    (x^*)
    :=
    \sup_{x \in \Omega}
    \ 
    \inner{x^*}{x}
    \qquad
    \text{for}\ 
    x^* \in \R^n
    .
  \end{gather}
\end{definition}


\begin{definition}
  Given functions
  $
    f_i:
    \R^n \to (-\infty, \infty]
  $
  for $ i = 1, \ldots, n $
  the \textbf{infimal convolution} of these functions is defined as
  \begin{gather}
    (f_1 \square \ldots \square f_m)(x)
    :=
    \inf_{
    \begin{smallmatrix}
      x_i \in \R^n \\
      \sum_{i = 1}^{m} 
        x_i
      =
      x
    \end{smallmatrix}
    }
    \sum_{i = 1}^{m}
      f_i(x_i)
  \end{gather}
\end{definition}
 
The next result establishes a connection between the support function of the intersection of two convex sets and the infimal convolution of the support functions of the sets taken by themselfes.
The proof translates the geometric concept of convex separation to the world of convex functions.

\begin{ftheorem}
  Let $C_1$ and $C_2$ be two non-empty convex sets in $\R^n$ with
  $\mathrm{ri}(C_1)\cap\mathrm{ri}(C_2)\neq\emptyset.$
  Then the support function of the intersection 
  $
    C_1 \cap C_2
  $
  is represented as
  \begin{gather}
    (\sigma_{
    C_1 \cap C_2
    })
    (x^*)
    =
    (\sigma_{C_1}\square \sigma_{C_2})
    (x^*)
    \qquad
    \text{for all}\ 
    x^* \in \R^n.
  \end{gather}
  Furthermore, for any
  $
  x^*\in \mathrm{dom}
    (\sigma_{
    C_1 \cap C_2
    })
  $
  there exist dual elements 
  $
    x_1^*
    ,
    x_2^*
    \in \R^n
  $ 
  such that 
  $
    x^*
    =
    x_1^*
    +
    x_2^*.
  $
  and
  \begin{gather}
    (\sigma_{
    C_1 \cap C_2
    })
    (x^*)
    =
    \sigma_{C_1}(x_1^*)
    +
    \sigma_{C_2}(x_2^*).
  \end{gather}
\end{ftheorem}
\begin{proof}
  \emph{\cite[Theorem~4.23]{Mordukhovich2022}}
  We want to use results on convex separation. To make the geometric property of convex separation fruitful
  to our purpose we consider two special sets. We will verify that these sets meet the requirements for convex separation, i.e., that they are convex and the intersection of their relative interiors is empty. The rest of the proof is as in the cited reference at the beginning of the proof.
  To this end, consider 
  the sets
  \begin{gather}
    \Theta_1
    \ :=\ 
    C_1 \times [\,0,\infty)
    \quad
    \text{and}
    \quad
    \Theta_2
    \ :=\ 
    \left\{ 
      (x,\lambda)\in \R^n
      \ 
      \colon
      \ 
      x \in C_2
      \ 
      \text{and}
      \ 
      \lambda
      \,
      \le
      \,
      \inner{x^*\!}{x} - \alpha
    \right\}\ .
  \end{gather}
  \todo[color=green!40,inline]{Simplify proof with properties of relative interiors.}
  Clearly, $\Theta_1$ is convex by the convexity of $C_1$. To see that $\Theta_2$ is convex consider the affine function
  $
    \varphi:
    \R^{n}\!\times \R \to \R
    ,
    \ 
    (x,\lambda)
    \mapsto
    \alpha - \inner{x^*\!}{x} - \lambda
  $ .
  From the definitions of $\varphi$ and $\Theta_2$ we get the identity
  \begin{gather*}
 \Theta_2
    \ 
    =
    \ 
    (
      C_2\!\times\R
    )
    \ 
    \cap
    \ 
    \varphi^{-1}
    (-\infty,0\,]
    \,
    .
  \end{gather*}
  Thus, by the convexity of the sets
  $C_2$
  and
  $
    \varphi^{-1}(-\infty,0]
  $
  it follows the convexity of $\Theta_2$.
  Next we show that the relative interiors of 
  $\Theta_1$ and $\Theta_2$
  do not intersect, i.e.,
  ~$
  \mathrm{ri}\,\Theta_1\cap\mathrm{ri}\,\Theta_2=\emptyset.
  $
  First note that
  \begin{gather}
    \mathrm{ri}(\Theta_1)
    =
    \mathrm{ri}(C_1)
    \times
    \mathrm{ri}([0,\infty))
    \subseteq
    \mathrm{ri}(C_1)
    \times
    (0,\infty).
  \end{gather}
  Indeed, if 
  $
    0\in
    \mathrm{ri}([0,\infty))
  $
  then there exists $t>0$ such that $-tx\ge 0$ for some $x>0.$ A contradiction.
  Furthermore
\begin{gather}
  \mathrm{ri}(\Theta_2)
  \subseteq
    \left\{ 
      (x,\lambda)\in \R^n
      \colon
      x \in \mathrm{ri}(C_2) 
      \ 
      \text{and}
      \ 
      \lambda
      <
      \inner{x^*}{x} - \alpha
    \right\}.
\end{gather}
To see this, assume there is 
$
(x,\lambda)
\in \mathrm{ri}(\Theta_2)
$
with
$
      \lambda
      =
      \inner{x^*}{x} - \alpha
      .
$
Then for some 
$
  (y,\mu)
  \in \Theta_2
$
with
$
      \mu    
      <
      \inner{x^*}{y} - \alpha
$
there exists $t>0$ such that 
$
  (x,\lambda)
  +
  t
  (
  (x,\lambda)
  -
  (y,\mu)
  )
  \in \Theta_2.
$
It follows
\begin{align}
  0
  \le
  (1+t)(\inner{x^*}{x}-\alpha -\lambda)
  +
  t
  (
    \mu 
    -\inner{x^*}{y}
    +
    \alpha
  )
  <0
,
\end{align}
a contradiction.
The first inequality is due to 
$
  (x,\lambda)
  +
  t
  (
  (x,\lambda)
  -
  (y,\mu)
  )
  \in \Theta_2
$
and the second inequality due to
$
      \mu    
      <
      \inner{x^*}{y} - \alpha
$
and
$
      \lambda
      =
      \inner{x^*}{x} - \alpha
      .
$
But then 
$
\mathrm{ri}(\Theta_1)\cap\mathrm{ri}(\Theta_2)=\emptyset
.
$
Indeed, suppose that there exists 
$
  (x,\lambda)
  \in
  \mathrm{ri}(\Theta_1)\cap\mathrm{ri}(\Theta_2)
  .
$
Then it holds
$
  \inner{x^*}{x}
  -
  \alpha
  \le
  0
$
and $\lambda>0$
since 
$
 x
 \in
  \mathrm{ri}(C_1)\cap\mathrm{ri}(C_2)
  \subseteq
  C_1\cap C_2.
$
On the other hand
\begin{gather}
  0
  <
  \lambda
  <
  \inner{x^*}{x}
  -
  \alpha
  \le
  0
  ,
\end{gather}
a contradiction.

Thus, convex separation is applicable. Confer \cite[Theorem 4.23]{Mordukhovich2022}
about the rest of the proof.
\end{proof}

\begin{takeaways}
  The support function intersection rule connects the geometric 
  property of convex separation to an identity of support functions
  This result is central to the analysis of convex conjugates.
\end{takeaways}
