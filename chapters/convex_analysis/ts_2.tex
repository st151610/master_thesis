
\begin{fproblem}
  \label{cv:ts:primal}
\begin{align*}
  %%%% objective %%%%
    &\underset{w \in \R^n}
    {\text{minimize}}
    &&\qquad\qquad
    f(w)
    &&&
    \\
    %%%% Ax >= b %%%%
    &\text{subject to}
    &&\qquad\qquad
    \mathbf{U}w
    \ 
    \ge
    \ 
    d
    \\
    &
    &&\qquad\qquad
    \mathbf{A}w
    \ 
    =
    \ 
    a
    \,,
\end{align*}
where 
$\mathbf{U}\in \R^{r\times n}$
and
$\mathbf{A}\in \R^{s\times n}$
\end{fproblem}
%%%%%%%%%%%%%
%%%% (D) %%%%
%%%%%%%%%%%%%
We consider the dual optimization problem of Problem~\ref{cv:ts:primal}.
\begin{fproblem}
  \label{cv:ts:dual}
  \begin{alignat*}{2}
  %%%% objective %%%%
    &\underset{
    \lambda_d \in \R^r
,
    \lambda_a \in \R^s
  }
    {\text{maximize}}
    &&\qquad\qquad
    \inner
    {\lambda_d}
    {d}
    \ 
    +
    \ 
    \inner
    {\lambda_a}
    {a}
    \ 
    -
    \ 
    f^*
    \left( 
      \mathbf{U}^\top  \lambda_d
      +
      \mathbf{A}^\top  \lambda_a
    \right)
    \\
    %%%% Ax >= b %%%%
    &\text{subject to}
    &&\qquad\qquad
    \lambda_d
    \ 
    \ge
    \ 
    0
    \,.
\end{alignat*}
\end{fproblem}
Note that we have non-negativity constraints only for 
the dual variable associated with the inequality constraints.
\subsubsection*{Plan of Proof}
We use the Karush-Kuhn-Tucker conditions for 
Problem~\ref{cv:ts:primal}
.
We assume that an optimal solution 
$(\lambda_d^*,\lambda_a^*)$
to 
Problem~\ref{cv:ts:dual}
exists.
Next we write
$
  w^*
  :=
  \nabla
    f^*
    \left( 
      \mathbf{U}^\top  \lambda_d^*
      +
      \mathbf{A}^\top  \lambda_a^*
    \right)
$.
We show complementary slackness for $\lambda_d^*$ using the continuity
of $f^*$.
Then we show $0\in$.
This follows from
$
      \mathbf{U}^\top  \lambda_d^*
      +
      \mathbf{A}^\top  \lambda_a^*
      \in
\partial f (w^*)
$
and 
$
$
\begin{gather}
  -
  \left( 
      \mathbf{U}^\top  
      +
      \mathbf{A}^\top  
  \right)
  \in
  \left[ 
    \partial
    \left( 
      w
      \mapsto
      d
      -
      \mathbf{U}w
    \right)
    (w^*)
    +
    \partial
    \left( 
      w
      \mapsto
      a
      -
      \mathbf{A}w
    \right)
    (w^*)
  \right]
  \,.
\end{gather}


\begin{ftheorem}
  Let 
$
(\lambda_d^*,\lambda_a^*)
$
be an optimal solution to
Problem~\ref{cv:ts:dual}.
Then the unique optimal solution $w^*$ to 
Problem~\ref{cv:ts:primal}
is given by
\begin{gather}
  w^*
  =
  \nabla
    f^*
    \left( 
      \mathbf{U}^\top  \lambda_d^*
      +
      \mathbf{A}^\top  \lambda_a^*
    \right)
    \,.
\end{gather}
\end{ftheorem}
\begin{proof}
  We fix 
  $
  \lambda_a^*
  $
  and
  work with the objective function $G$ of the dual problem, that is,
  \begin{gather}
    G
(\lambda_d)
:
=
    \inner
    {\lambda_d}
    {d}
    \ 
    +
    \ 
    \inner
    {\lambda_a}
    {a}
    \ 
    -
    \ 
    f^*
    \left( 
      \mathbf{U}^\top  \lambda_d
      +
      \mathbf{A}^\top  \lambda_a
    \right)
    \,.
  \end{gather}
  Since $f^*$ is continuously differentiable, so is $G$. 
  Since $f^*$ is convex, $G$ is concave.
  For concave differentiable functions it holds
  \begin{gather}
    G(x)-G(y)
    \ge
    \nabla
    G(x)^\top
    (x-y)
    \,.
  \end{gather}
Let
$\lambda_{d,i}^*$ be the $i$-th coordinate of $\lambda_d^*$ and for
fixed
$\lambda_a^*$ let
$
\nabla
G_i
(\lambda_d^*)
$
be the $i$-th coordinate of 
$
\nabla
G
(\lambda_d^*)
$.
We show that for all 
$
  i\in \left\{ 1,\ldots, s \right\}
$
it holds
\begin{alignat*}{2}
  \text{either}
  &
  &&
  \qquad
  \lambda_{d,i}^*
  = 0
  \quad
  \text{and}
  \quad
  \nabla
  G
  _i(
  \lambda_{d}^*
  ) \le 0
  \\
  \text{or}
  &
  &&
  \qquad
  \lambda_{d,i}^*
  > 0
  \quad
  \text{and}
  \quad
  \nabla
  G
  _i(
  \lambda_{d}^*
  ) = 0
  \,.
\end{alignat*}
Assume towards a contradiction that 
$
\nabla G_i(\lambda_d^*)>0
$
for some 
$
  i\in \left\{ 1,\ldots, s \right\}
$.
By the continuity of $\nabla G$ there exists $\varepsilon>0$ such that 
$
\nabla G_i(
\lambda_d^*
+
e_i\cdot \varepsilon
)
>
0
$.
Thus
\begin{gather}
  G
  (
\lambda_d^*
+
e_i\cdot \varepsilon
  )
  -
  G
  (
\lambda_d^*
  )
  \ge
\nabla G_i(
\lambda_d^*
+
e_i\cdot \varepsilon
)
\cdot
\varepsilon
>0
\,,
\end{gather}
which contradicts the optimality of 
$
\lambda_d^*
$.
Now assume that
$ \lambda_{d,i}^*>0 $ and 
$
  \nabla G_i(\lambda_d^*)< 0
$
for some
$
  i\in \left\{ 1,\ldots, s \right\}
$.
Again, by the 
continuity of $\nabla G$ there exists $\varepsilon>0$ such that
$
  \nabla G_i(\lambda_d^*-e_i\cdot \varepsilon)< 0
$
and
$
\varepsilon
-
\lambda_{d,i}^*
<0
$.
Thus
\begin{gather}
  G
  (
\lambda_d^*
-
e_i\cdot \varepsilon
  )
  -
  G
  (
\lambda_d^*
  )
  \ge
\nabla G_i(
\lambda_d^*
-
e_i\cdot \varepsilon
)
\cdot
\left( 
\varepsilon
-
\lambda_{d,i}^*
\right)
>0
\,,
\end{gather}
which contradicts the optimality of 
$
\lambda_d^*
$.
We have shown the complementary slackness.
\subsubsection*{remanining part}
We want to show

\begin{gather}
      \mathbf{U}^\top  \lambda_d^*
      +
      \mathbf{A}^\top  \lambda_a^*
      \in
\partial f (w^*)
\,,
\end{gather}
where
\begin{gather}
  w^*
  :=
  \nabla
    f^*
    \left( 
      \mathbf{U}^\top  \lambda_d^*
      +
      \mathbf{A}^\top  \lambda_a^*
    \right)
\end{gather}
By \cite[Theorem~23.5(a)-(b)]{Rockafellar1970}
it suffices to show
that
$
\inner
{(\cdot)}
{
      \mathbf{U}^\top  \lambda_d^*
      +
      \mathbf{A}^\top  \lambda_a^*
}
-
f(\cdot)
$
achieves its supremum at
$w^*$.
Since $f$ is strictly convex the above expression achieves its maximum at the unique
point $w^*$.
But then it holds
\begin{gather}
  \nabla
    f^*
    \left( 
      \mathbf{U}^\top  \lambda_d^*
      +
      \mathbf{A}^\top  \lambda_a^*
    \right)
    =
    \nabla
    \left( 
\inner
{w^*}
{
  \cdot
}
-
f(w^*)
    \right)
    \left( 
      \mathbf{U}^\top  \lambda_d^*
      +
      \mathbf{A}^\top  \lambda_a^*
    \right)
    =
w^*
\,.
\end{gather}

Next we show
\begin{gather}
  -
  \left( 
      \mathbf{U}^\top  
      +
      \mathbf{A}^\top  
  \right)
  \in
  \left[ 
    \partial
    \left( 
      w
      \mapsto
      d
      -
      \mathbf{U}w
    \right)
    (w^*)
    +
    \partial
    \left( 
      w
      \mapsto
      a
      -
      \mathbf{A}w
    \right)
    (w^*)
  \right]
  \,.
\end{gather}

To this end, note that
\begin{gather}
  \inner
  {
  -
\mathbf{U}^\top
\cdot e_i
}
  {w-w^*}
  =
  \left( 
    d-
\mathbf{U}^\top
\cdot w
  \right)_i
  -
  \left( 
    d-
\mathbf{U}^\top\cdot w^*
  \right)_i
  \,.
\end{gather}
Thus
$
-
\mathbf{U}^\top
\in
    \partial
    \left( 
      w
      \mapsto
      d
      -
      \mathbf{U}w
    \right)
    (w^*)
$.
In the same way it follows
$
-
\mathbf{A}^\top
\in
    \partial
    \left( 
      w
      \mapsto
      d
      -
      \mathbf{A}w
    \right)
    (w^*)
$.
We conclude that
\begin{gather}
  \mathrm{0}
  \in
  [
  \partial
  f(w^*)
  +
    \partial
    \left( 
      w
      \mapsto
      d
      -
      \mathbf{U}w
    \right)
    (w^*)
    \cdot
    \lambda_d^*
    +
    \partial
    \left( 
      w
      \mapsto
      a
      -
      \mathbf{A}w
    \right)
    (w^*)
    \cdot
    \lambda_a^*
  ]
  \,.
\end{gather}
By the Karush-Kuhn-Tucker conditions we finish the proof.
\end{proof}
