Throughout this section let $n\in\mathbb{N}$.
\subsection*{Sets}
A subset $C\subseteq \R^n$ is called \textbf{convex set}, 
if for all $x,y\in C$ and all $\theta\in [0,1]$,
we have 
$
  \theta x + (1-\theta)y 
  \in
  C
$.
Many set operations preserve convexity. Among them
forming the 
\textbf{Cartesian product} of two convex sets, 
\textbf{intersection} of a collection of convex sets and 
taking the \textbf{inverse image under linear functions}.

The classical theory evolves around the question 
if convex sets can be separated.
\begin{definition*}
  Let 
  $C_1$ and $C_2$
  be two non-empty convex sets in $\R^n$. 
  A hyperplane $H$ is said to \textbf{separate}
  $C_1$ and $C_2$
  if $C_1$ is contained in one of the closed half-spaces associated with
  $H$ and $C_2$ lies in the opposite closed half-space. It is said to separate 
  $C_1$ and $C_2$
  \textbf{properly} if 
  $C_1$ and $C_2$
  are not both contained in $H$.
\end{definition*}

We need a refined concept of interiors, since some convex sets have empty interior. To this end, 
  we call a set
  $A\subseteq \R^n$ 
  \textbf{affine set}, if
  $
    \alpha x + (1-\alpha)y \in A
    \quad
    \text{for all}
    \ 
    x,y \in A
    \ 
    \text{and}
    \ 
    \alpha \in \R
  $.
  The \textbf{affine hull} 
  $\mathrm{aff}(\Omega)$
  of a set 
  $\Omega\subseteq \R^n$
  is the smallest affine set that includes $\Omega$.
  We define the \textbf{relative interior} $\mathrm{ri}\,\Omega$ 
  of a set 
  $\Omega\subseteq \R^n$
  to be the interior relative to the affine hull, that is,
    \begin{gather}
    \mathrm{ri}(\Omega)
    \ 
    :=
    \ 
    \left\{ 
      x \in \Omega 
      \ 
      |
      \ 
      \exists
      \,
      \varepsilon > 0\ 
      \colon
      (
      x+\varepsilon B_{\R^n}
      )
      \cap
      \mathrm{aff}(\Omega)
      \ 
      \subset
      \ 
      \Omega
      \,
    \right\}
    \,.
  \end{gather}

\begin{ftheorem}
  \label{cv:primer:sep}
  \emph{(Convex separation in finite dimension)}
  Let $C_1$ and $C_2$ be two non-empty convex sets in $\R^n$. 
  Then $C_1$ and $C_2$ can be properly separated if and only if 
  $\mathrm{ri}(C_1)\cap\mathrm{ri}(C_2)=\emptyset.$
\end{ftheorem}
\begin{proof}
  \cite[Theorem~11.3]{Rockafellar1970}
\end{proof}
We collect some useful 
properties of relative interiors
before we get on to convex functions.
\begin{proposition}
  \label{cv:primer:prop}
  Let $C$ be a non-empty convex set in $\R^n.$ The following holds:
\begin{enumerate}[label={(\roman*)}]
  \item
    $
      \mathrm{ri}(C)
      \ 
      \neq
      \ 
      \emptyset
      $
if and only if
      $
      C
      \ 
      \neq
      \ 
      \emptyset
    $
  \item
    $
      \mathrm{cl}(\mathrm{ri}\,C)
      \ 
      =
      \ 
      \mathrm{cl}\,C
      $
      and
      $
      \mathrm{ri}(\mathrm{cl}\,C)
      \ 
      =
      \ 
      \mathrm{ri}(C)
    $
  \item
    $
    \mathrm{ri}(C)
      \ 
    =
      \ 
    \left\{ 
      z \in C
      \colon
      \text{for all}\ 
      x \in C \ 
      \text{there exists}\ 
      t > 0 \ 
      \text{such that}\ 
      z + t (z-x)
      \in C
    \right\}
    $
  \item
    Suppose
    $
      \bigcap_{i\in I} C_i
      \ 
      \neq
      \ 
      \emptyset
    $
    for a finite index set $I$.
    Then
    $
      \mathrm{ri}
      \left( 
        \bigcap_{i\in I} C_i
      \right)
      \ 
      =
      \ 
      \bigcap_{i\in I}  
      \mathrm{ri}(C_i)
    $.
    \item
      Let 
      $
        L\,:\,\R^n \to\  \R^m
      $
      be a linear function. Then
      $
        \mathrm{ri}\,L(C)
        \ 
        =
        \ 
        L(\mathrm{ri}\,C)
      $.
      If it also holds
      $
        L^{\!-1}(\mathrm{ri}\,C)
        \ 
        \neq
        \ 
        \emptyset
      $,
      we have
      $
      \mathrm{ri}\,L^{\!-1}(C)
      \ 
        =
      \ 
        L^{\!-1}(\mathrm{ri}\,C)
      $.
      \item
        $
          \mathrm{ri}(C_1\!\times C_2)
          \ 
          = 
          \ 
          \mathrm{ri}\,C_1
          \! 
          \times
          \mathrm{ri}\,C_2
        $
\end{enumerate}

\end{proposition}


\begin{proof}
  For a proof of (i)-(v) we refer to~\cite[Theorem 6.2 - 6.7]{Rockafellar1970}.

To prove (vi) we use (iii).
Let
  $
  (z_1, z_2)
  \in 
  \mathrm{ri}(C_1\!\times C_2).
  $
  Then for all 
  $
  (x_1, x_2)
  \in 
  C_1\!\times C_2
  $
  there exists
  $t>0$
  such that
  \begin{gather}
    \label{cv:primer:prop:1}
      z_i + t (z_i-x_i)
      \in C_i
      \qquad
      \text{for all}\ 
      i\in \left\{ 1,2 \right\}.
  \end{gather}
  Using (iii) again, we get
  $
  \mathrm{ri}(C_1\!\times C_2)
  \, 
  \subseteq
  \,
          \mathrm{ri}\,C_1
          \! 
          \times
          \mathrm{ri}\,C_2
  $.
  Suppose 
  $
  (z_1,z_2)
    \in
    \mathrm{ri}\,C_1
    \!
    \times
    \mathrm{ri}\,C_2
  $.
  By (iii), for all
  $
    (x_1,x_2)\in C_1\times C_2
  $
  there exist
  $
    (t_1,t_2)>0
  $
  such that
  \begin{gather}
    \label{cv:primer:prop:2}
      z_i + t_i (z_i-x_i)
      \in C_i
      \qquad
      \text{for all}\ 
      i\in \left\{ 1,2 \right\}.
  \end{gather}
  If $t_1=t_2$
  we recover
  \eqref{cv:primer:prop:1}
  from
  \eqref{cv:primer:prop:2}.
  By (iii) it holds
  $
  (z_1,z_2)
    \in
    \mathrm{ri}
    (C_1
    \!
    \times
    C_2)
  $.
  If
  $t_1<t_2$
  we
  define $\theta:=\frac{t_1}{t_2}\in (0,1).$
  Consider  
  \eqref{cv:primer:prop:2} with $i=2$,
  together with $z_2 \in C_2$
  and
  the convexity of $C_2$.
  It follows
  \begin{gather}
    \label{cv:primer:prop:3}
    z_2 + t_1 (z_2 - x_2)
    \ 
    =
    \ 
    \theta
    \cdot
    (
    z_2 + t_2 (z_2 - x_2)
    )
    \ 
    +
    \ 
    (1-\theta)
    \cdot
    z_2
    \in C_2
    \,.
  \end{gather}
  Now we consider
  \eqref{cv:primer:prop:3} and
  \eqref{cv:primer:prop:2} with $i=1$.
  This gives \eqref{cv:primer:prop:1} with $t=t_1$.
  As before, it follows
  $
  (z_1,z_2)\in\mathrm{ri}(C_1\!\times C_2)
  $.
  If 
  $t_1>t_2$
  similar arguments lead to the same result.
  We have proven 
  $
  \mathrm{ri}(C_1\!\times C_2)
  \, 
  \supseteq
  \,
          \mathrm{ri}\,C_1
          \! 
          \times
          \mathrm{ri}\,C_2
  $
  and equality.
\end{proof}
\subsection*{Functions}
A function 
$
f
\colon
\R^n
\to
\overline{\R}
$
is called \textbf{convex function}, if the area above its graph, that is, its epigraph(cf.\cite[§2.4.1]{Mordukhovich2022}), is convex. We shall often use an equivalent definition.
To this end, 
a function $f$ is convex if and only if 
\begin{gather}
  \label{cv:cf}
  f(\theta x + (1-\theta)y)
  \ 
  \le
  \ 
  \theta f(x)
  +
  (1-\theta)f(y)
  \qquad
  \text{for all}\ 
  x,y\in \R^n
  \ 
  \text{and all}\ 
  \theta\in[0,1]
  \,.
\end{gather}
This definition extends to convex cominbinations
$
  \theta_1,\ldots,\theta_m\in[0,1]
$
with
$
  \sum_{i=1}^{m} 
  \theta_i
  =1
$, that is, 
a function $f$ is convex if and only if 
\begin{gather}
  f
  \left( 
    \sum_{i=1}^{m} 
    \theta_i
    x_i
  \right)
  \ 
  \le
  \ 
    \sum_{i=1}^{m} 
    \theta_i
    f(x_i)
  \qquad
  \text{for all}\ 
  x_1,\ldots,x_m\in \R^n
  \,.
\end{gather}
We call a function \textbf{strictly convex} if the inequality in
\eqref{cv:cf} is strict.

We define the \textbf{domain} $\mathrm{dom}\,f$
of a convex function $f$ to be the set where $f$ is finite, that is,
\begin{gather}
  \mathrm{dom}\,f
  \ 
  :=
  \ 
  \left\{ 
x\in\R^n
:
f(x)<\infty
  \right\}
  \,.
\end{gather}
The domain of a convex function is convex. 
We say that $f$ is a \textbf{proper function} if  $\mathrm{dom}\,f\neq\emptyset$. 

For any $\overline{x}\in\mathrm{dom}\,f$ we call $x^*\in\R^n$ a 
\textbf{subgradient} of $f$ at $\overline{x}$ if for all 
$x\in\R^n$ it holds
\begin{gather}
  \inner{x^*}{x-\overline{x}}
  \le
  f(x)
  -
  f(\overline{x})
  \,.
\end{gather}
We denote the collection of all subgradients at $\overline{x}$, that is, the \textbf{subdifferential} of $f$ at $\overline{x}$, as
$
\partial f(\overline{x})
$.
If $f$ is differentiable at $\overline{x}$ it holds
$
\partial f(\overline{x})
=
\left\{ 
  \nabla
  f(\overline{x})
\right\}
$
and thus
\begin{gather}
  \inner{
  \nabla
  f(\overline{x})
}{x-\overline{x}}
  \le
  f(x)
  -
  f(\overline{x})
  \,.
\end{gather}
We call a differentiable function $f$ \textbf{strongly convex} with parameter
$m>0$ if for all
$x,y\in\mathrm{dom}\,f$ it holds
\begin{gather}
  f(y)-f(x)
  \ 
  \ge
  \ 
  \inner
  {
  \nabla
  f(x)
  }
  {y-x}
  \ 
  +
  \ 
  \frac{m}{2}
  \norm{y-x}^2
  \,.
\end{gather}
If $f$ is twice continuously differentiable, then it is strongly 
convex with parameter $m>0$ if and only if the matrix
\begin{gather}
\nabla^2f(x)
-m\cdot\mathbf{I}
\qquad
\text{
is positive semi-definite for all
}\ 
x\in\mathrm{dom}\,f
\,,
\end{gather}
 where 
$
\nabla^2f
$
is the Hessian Matrix.

One important application of convex functions is in optimization.
There we often analyse a dual problem instead, which relies on the notion of \textbf{convex conjugate} 
$
    f^*:
    \R^n \to \overline{\R}
  $
  of $f$ defined by
  \begin{gather}
    f^*(x^*)
    :=
    \sup_{ x \in \R^n }
    \inner
    {x^*}{x}
    - f(x)
    \,.
  \end{gather}
  Even for arbitrary functions, the convex conjugate is convex(cf.).
  Like in differential calculus, there exist sum and chain rule for computing the convex conjugate.
