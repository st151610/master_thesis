\begin{definition}
  \emph{(Affine set and hull)}
  A set 
  $A\subseteq \R^n$
  is called \textbf{affine}, if
  \begin{gather}
    \alpha x + (1-\alpha)y \in A
    \quad
    \text{for all}
    \ 
    x,y \in A
    \ 
    \text{and}
    \ 
    \alpha \in \R.
  \end{gather}
  The \textbf{affine hull} 
  $\mathrm{aff}(C)$
  of a set 
  $C\subseteq \R^n$
  is the smallest affine set that includes $C.$
\end{definition}

\begin{definition}
  \emph{(Relative interior)}
  Let 
  $\Omega\subseteq \R^n.$
  We define the \textbf{relative interior} of $\Omega$ by
  \begin{gather}
    \mathrm{ri}(\Omega)
    :=
    \left\{ 
      x \in \Omega 
      \colon
      \text{there exists}\ 
      \gamma > 0\ 
      \text{such that}\ 
      \mathrm{B}_\gamma(x)
      \cap
      \mathrm{aff}(\Omega)
      \in
      \Omega
    \right\}.
  \end{gather}


\end{definition}
