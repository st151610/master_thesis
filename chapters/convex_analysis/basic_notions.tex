Excursively, we present some well known definitions and facts from
convex analysis. For details, see, e.g., \cite{Mordukhovich2022}.

A subset $C\subseteq \R^n$ is called \textbf{convex set}, 
if for all $x,y\in C$ and all $\lambda\in [0,1]$,
we have 
$
  \lambda x + (1-\lambda)y 
  \in
  C
  .
$
The Cartesian product of convex sets is convex. The intersection of a collection of convex sets is also convex.
Given (not necessary convex) sets 
$
  \Omega,
  \Omega_1,
  \Omega_2
  \subseteq 
  \R^n
$
and
$\lambda\in\R,$
define the \textbf{set addition} and \textbf{multiplication}
by a real scalar as 
$
  \Omega_1 
  +
  \Omega_2 
  :=
  \left\{ 
    x_1 + x_2 
    \colon
    x_1 \in \Omega_1
    ,
    x_2 \in \Omega_2
  \right\}
$
and
$
  \lambda \Omega
  :=
  \left\{ 
    \lambda x
    \colon
    x\in\Omega
  \right\}.
$
For convex sets the addition and multiplication by a real scalar are convex.


A mapping 
$
  A: \R^n \to \R^m
$
is called \textbf{affine mapping} if there exist a linear mapping
$
  L: \R^n \to \R^m
$
and a vector $b\in \R^m$
such that
$
  A(x)
  =
  L(x)
  +
  b
  \ 
  \text{for all}\ 
  x \in \R^n
  .
$
The image and inverse image/preimage of convex sets under affine mappings are also convex.

\section{Relative interiors}



\begin{definition}
  \emph{(Affine set and hull)}
  A set 
  $A\subseteq \R^n$
  is called \textbf{affine}, if
  \begin{gather}
    \alpha x + (1-\alpha)y \in A
    \quad
    \text{for all}
    \ 
    x,y \in A
    \ 
    \text{and}
    \ 
    \alpha \in \R.
  \end{gather}
  The \textbf{affine hull} 
  $\mathrm{aff}(\Omega)$
  of a set 
  $\Omega\subseteq \R^n$
  is the smallest affine set that includes $\Omega.$
\end{definition}

\begin{definition}
  \emph{(Relative interior)}
  Let 
  $\Omega\subseteq \R^n.$
  We define the \textbf{relative interior} of $\Omega$ by
  \begin{gather}
    \mathrm{ri}(\Omega)
    :=
    \left\{ 
      x \in \Omega 
      \colon
      \text{there exists}\ 
      \gamma > 0\ 
      \text{such that}\ 
      \mathrm{B}_\gamma(x)
      \cap
      \mathrm{aff}(\Omega)
      \in
      \Omega
    \right\}.
  \end{gather}

\begin{proposition}
  Let $C$ be a non-empty convex set in $\R^n.$ Then we get the representation
  \begin{gather}
    \mathrm{ri}(C)
    =
    \left\{ 
      z \in C
      \colon
      \text{for all}\ 
      x \in C \ 
      \text{there exists}\ 
      t > 0 \ 
      \text{such that}\ 
      z + t (z-x)
      \in C
    \right\}.
  \end{gather}
\end{proposition}
\begin{proof}
  \cite[Theorem~6.4]{Rockafellar1970}
\end{proof}
\begin{proposition}
  Let $C_1\subseteq \R^{n_1}$ and $C_2\subseteq \R^{n_2}$ be two non-empty convex sets. Then it holds
  \begin{gather}
    \mathrm{ri}(C_1\times C_2)
    = 
    \mathrm{ri}(C_1)
    \times
    \mathrm{ri}(C_2).
  \end{gather}
\end{proposition}
\begin{proof}
  Let
  $
  (z_1, z_2)
  \in 
  \mathrm{ri}(C_1\times C_2).
  $
  Then for all 
  $
  (x_1, x_2)
  \in 
  C_1\times C_2
  $
  there exists
  $t>0$
  such that
  \begin{gather}
      z_i + t (z_i-x_i)
      \in C_i
      \qquad
      \text{for}\ 
      i\in \left\{ 1,2 \right\}.
  \end{gather}
  This proves $\subseteq.$
  Suppose 
  $
    z_1 
    \in
    \mathrm{ri}(C_1)
  $
  and
  $
    z_2 
    \in
    \mathrm{ri}(C_2).
  $
  Let
  $
    (x_1,x_2)\in C_1\times C_2
  $
  with corresponding $t_1,t_2 >0$.
  If $t_1=t_2$ everything is clear.
  W.l.o.g.
  assume 
  $t_1<t_2$. Define $\theta:=\frac{t_1}{t_2}\in (0,1).$
  By the convexity of $C_2$ it follows
  \begin{gather}
    z_2 + t_1 (z_2 - x_2)
    =
    \theta
    (
    z_2 + t_2 (z_2 - x_2)
    )
    +
    (1-\theta)z_2
    \in C_2.
  \end{gather}
  Thus
  $
  (z_1,z_2)\in\mathrm{ri}(C_1\times C_2). 
  $
  This proves $\supseteq$ and equality.
\end{proof}
 
\section{Convex Separation}
\begin{definition}
  Let 
  $C_1$ and $C_2$
  be two non-empty convex sets in $\R^n$. 
  A hyperplane $H$ is said to \textbf{separate}
  $C_1$ and $C_2$
  if $C_1$ is contained in one of the closed half-spaces associated with
  $H$ and $C_2$ lies in the opposite closed half-space. It is said to separate 
  $C_1$ and $C_2$
  \textbf{properly} if 
  $C_1$ and $C_2$
  are not \textit{both} actually contained in $H$ itselef.
\end{definition}
\begin{theorem}
  Let $C_1$ and $C_2$ be two non-empty convex sets in $\R^n$. 
  There exists a hyperplane separating
  $C_1$ and $C_2$
  properly 
  if and only if
  there exists a vector $b\in \R^n$ such that
  \begin{gather}
    \sup_{x\in C_2} \inner{x}{b}
    \le
    \inf_{x\in C_1} \inner{x}{b}
    \quad 
    \text{and}
    \quad 
    \inf_{x\in C_2} \inner{x}{b}
    <
    \sup_{x\in C_1} \inner{x}{b}
    .
  \end{gather}
\end{theorem}
\begin{proof}
  \cite[Theorem~11.1]{Rockafellar1970}
\end{proof}
\begin{ftheorem}
  \emph{(Convex separation in finite dimension)}
  Let $C_1$ and $C_2$ be two non-empty convex sets in $\R^n$. 
  Then $C_1$ and $C_2$ can be properly separated if and only if 
  $\mathrm{ri}(C_1)\cap\mathrm{ri}(C_2)=\emptyset.$
\end{ftheorem}
\begin{proof}
  \cite[Theorem~11.3]{Rockafellar1970}
\end{proof}
\end{definition}
\begin{definition}
  \emph{(Support function intersection rule)}
  \emph{(Support function)}
  Given a nonempty subset 
  $\Omega \subseteq \R^n$
  the \textbf{support function} 
  $
  \sigma_\Omega : \R^n \to \overline{\R}
  $
  of $\Omega$
  is defined by
  \begin{gather}
    \sigma_\Omega
    (x^*)
    :=
    \sup_{x \in \Omega}
    \ 
    \inner{x^*}{x}
    \qquad
    \text{for}\ 
    x^* \in \R^n
    .
  \end{gather}
\end{definition}
\begin{ftheorem}
  Let $C_1$ and $C_2$ be two non-empty convex sets in $\R^n$ with
  $\mathrm{ri}(C_1)\cap\mathrm{ri}(C_2)\neq\emptyset.$
  Then the support function of the intersection 
  $
    C_1 \cap C_2
  $
  is represented as
  \begin{gather}
    (\sigma_{
    C_1 \cap C_2
    })
    (x^*)
    =
    (\sigma_{C_1}\square \sigma_{C_2})
    (x^*)
    \qquad
    \text{for all}\ 
    x^* \in \R^n.
  \end{gather}
  Furthermore, for any
  $
  x^*\in \mathrm{dom}
    (\sigma_{
    C_1 \cap C_2
    })
  $
  there exist dual elements 
  $
    x_1^*
    ,
    x_2^*
    \in \R^n
  $ 
  such that 
  $
    x^*
    =
    x_1^*
    +
    x_2^*.
  $
  and
  \begin{gather}
    (\sigma_{
    C_1 \cap C_2
    })
    (x^*)
    =
    \sigma_{C_1}(x_1^*)
    +
    \sigma_{C_2}(x_2^*).
  \end{gather}
\end{ftheorem}
\begin{proof}
  \emph{\cite[Theorem~4.23]{Mordukhovich2022}}
  We define
  \begin{gather}
    \Theta_1
    :=
    C_1 \times [0,\infty)
    \quad
    \text{and}
    \quad
    \Theta_2
    :=
    \left\{ 
      (x,\lambda)\in \R^n
      \colon
      x \in C_2
      \ 
      \text{and}
      \ 
      \lambda
      \le
      \inner{x^*}{x} - \alpha
    \right\}.
  \end{gather}
  Clearly $\Theta_1$ is convex by the convexity of $C_1.$
  Consider the affine function
  \begin{gather}
    \varphi:
    \R^{n+1} \to \R
    ,
    \quad
    (x,\lambda)
    \mapsto
    \alpha - \inner{x^*}{x} - \lambda.
  \end{gather}
  It holds 
  $
    \Theta_2
    =
    \varphi^{-1}((-\infty,0])
    \cap
    (
      C_2\times \R
    )
    .
  $
  Thus, by the convexity of the sets
  $
    \varphi^{-1}((-\infty,0])
  $
  and $C_2$
  it follows that $\Theta_2$ is convex.
  We want to apply convex separation to 
  $\Theta_1$ and $\Theta_2$.
  To this end we show 
  $\mathrm{ri}(\Theta_1)\cap\mathrm{ri}(\Theta_2)=\emptyset.$
  First note that
  \begin{gather}
    \mathrm{ri}(\Theta_1)
    =
    \mathrm{ri}(C_1)
    \times
    \mathrm{ri}([0,\infty))
    \subseteq
    \mathrm{ri}(C_1)
    \times
    (0,\infty).
  \end{gather}
  Indeed, if 
  $
    0\in
    \mathrm{ri}([0,\infty))
  $
  then there exists $t>0$ such that $-tx\ge 0$ for some $x>0.$ A contradition.
  Furthermore
\begin{gather}
  \mathrm{ri}(\Theta_2)
  \subseteq
    \left\{ 
      (x,\lambda)\in \R^n
      \colon
      x \in \mathrm{ri}(C_2) 
      \ 
      \text{and}
      \ 
      \lambda
      <
      \inner{x^*}{x} - \alpha
    \right\}.
\end{gather}
To see this, assume there is 
$
(x,\lambda)
\in \mathrm{ri}(\Theta_2)
$
with
$
      \lambda
      =
      \inner{x^*}{x} - \alpha
      .
$
Then for some 
$
  (y,\mu)
  \in \Theta_2
$
with
$
      \mu    
      <
      \inner{x^*}{y} - \alpha
$
there exists $t>0$ such that 
$
  (x,\lambda)
  +
  t
  (
  (x,\lambda)
  -
  (y,\mu)
  )
  \in \Theta_2.
$
It follows
\begin{align}
  0
  \le
  (1+t)(\inner{x^*}{x}-\alpha -\lambda)
  +
  t
  (
    \mu 
    -\inner{x^*}{y}
    +
    \alpha
  )
  <0
,
\end{align}
a contradiction.
The first inequality is due to 
$
  (x,\lambda)
  +
  t
  (
  (x,\lambda)
  -
  (y,\mu)
  )
  \in \Theta_2
$
and the second inequality due to
$
      \mu    
      <
      \inner{x^*}{y} - \alpha
$
and
$
      \lambda
      =
      \inner{x^*}{x} - \alpha
      .
$
But then 
$
\mathrm{ri}(\Theta_1)\cap\mathrm{ri}(\Theta_2)=\emptyset
.
$
Indeed, suppose that there exists 
$
  (x,\lambda)
  \in
  \mathrm{ri}(\Theta_1)\cap\mathrm{ri}(\Theta_2)
  .
$
Then it holds
$
  \inner{x^*}{x}
  -
  \alpha
  \le
  0
$
and $\lambda>0$
since 
$
 x
 \in
  \mathrm{ri}(C_1)\cap\mathrm{ri}(C_2)
  \subseteq
  C_1\cap C_2.
$
On the other hand
\begin{gather}
  0
  <
  \lambda
  <
  \inner{x^*}{x}
  -
  \alpha
  \le
  0
  ,
\end{gather}
a contradiction.
\end{proof}
\begin{colbox}{F8E0E0}
\textbf{Takeaways} \lipsum[1] % dummy text
\end{colbox}
