
\begin{definition}
  Let 
  $\Omega\subseteq \R^n.$
  We define the \textbf{relative interior} of $\Omega$ by
  \begin{gather}
    \mathrm{ri}(\Omega)
    :=
    \left\{ 
      x \in \Omega 
      \colon
      \text{there exists}\ 
      \gamma > 0\ 
      \text{such that}\ 
      \mathrm{B}_\gamma(x)
      \cap
      \mathrm{aff}(\Omega)
      \in
      \Omega
    \right\}.
  \end{gather}
\end{definition}
Next we collect some useful properties of relative interiors.
\begin{theorem}

\end{theorem}
\begin{ftheorem}
  Let $C$ be a non-empty convex set in $\R^n.$ Then we get the representation
\begin{enumerate}[label={(\roman*)}]
  \item
    $
    \mathrm{ri}(C)
    =
    \left\{ 
      z \in C
      \colon
      \text{for all}\ 
      x \in C \ 
      \text{there exists}\ 
      t > 0 \ 
      \text{such that}\ 
      z + t (z-x)
      \in C
    \right\}.
    $
  \item
    $
      \mathrm{cl}(C)
      \ 
      \text{and}\ 
      \mathrm{ri}(C).
    $
    are convex sets
\end{enumerate}
\end{ftheorem}
