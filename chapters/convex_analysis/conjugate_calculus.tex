When studying different primal problems such as \eqref{primal_weighting_binary} we often turn to the dual instead.
Therefore we need some reliable tools.
Begin able to compute specific convex conjugates is one tool required.


\begin{definition}
  \label{ def_convex_conjugate }
  \emph{(Convex conjugate)}
  Given a function
  $
    f:
    \R^n \to \overline{\R}
  $
  ,
  the 
  \textbf{convex conjugate}
  $
    f^*:
    \R^n \to \overline{\R}
  $
  of $f$ is defined as
  \begin{gather}
    f^*(x^*)
    :=
    \sup_{ x \in \R^n }
    (x^*)^T x - f(x)
  \end{gather}
\end{definition}

Note that $f$ in Definition~\ref{ def_convex_conjugate }
does not have to be convex. On the other hand, the convex conjugate is always convex:

\begin{proposition}
  Let  
  $
    f:
    \R^n \to ( - \infty, \infty ]
  $
  be a proper function. 
  Then its convex conjugate
  $
    f^*:
    \R^n \to ( - \infty, \infty ]
  $
  is convex.
\end{proposition}

\begin{definition}
  Given a nonempty subset 
  $\Omega \subseteq \R^n$
  the \textbf{support function} 
  $
  \sigma_\Omega : \R^n \to \overline{\R}
  $
  of $\Omega$
  is defined by
  \begin{gather}
    \sigma_\Omega
    (x^*)
    :=
    \sup_{x \in \Omega}
    \ 
    \inner{x^*}{x}
    \qquad
    \text{for}\ 
    x^* \in \R^n
    .
  \end{gather}
\end{definition}

\begin{lemma}
  For any proper function
  $
    f:\R^n\to\overline{\R}
  $
  we have
  \begin{gather}
    f^*(x^*) 
    =
    \sigma_{\mathrm{epi}(f)}
    (x^*,-1)
    \qquad
    \text{for}
    \ 
    x^* \in \R^n.
  \end{gather}
\end{lemma}
\begin{proof}
  Let $x^*\in\R^n$
  and
  $
    (x,\lambda)\in \mathrm{epi}(f).
  $
  Then
  $
    x \in \mathrm{dom}(f)
  $
  and
  $
    f(x)\le \lambda.
  $
  Thus
  \begin{gather}
    \inner{x^*}{x} - f(x)
    \ge
    \inner{x^*}{x} - \lambda
    \qquad
    \text{for all}\ 
    (x,\lambda)\in \mathrm{epi}(f).
  \end{gather}
  On the other hand 
  $
    (x,f(x))\in \mathrm{epi}(f)
  $
  for all
  $
    x \in \mathrm{dom}(f).
  $
  It follows
  \begin{gather}
    \inner{x^*}{x} - f(x)
    \le
    \sup_{(x,\lambda)\in\mathrm{epi}(f)}
    \inner{x^*}{x} - \lambda
    \qquad
    \text{for all}\ 
    x \in \mathrm{dom}(f).
  \end{gather}
  Taking the supremum in the last two displays yields
  \begin{align}
    f^*(x^*)
    =
    \sup_{x\in\mathrm{dom}(f)}
    \inner{x^*}{x} - f(x)
    &=
    \sup_{(x,\lambda)\in\mathrm{epi}(f)}
    \inner{x^*}{x} - \lambda
    \\
    &=
    \sup_{(x,\lambda)\in\mathrm{epi}(f)}
    \inner{(x^*,-1)}{(x,\lambda)} 
    =
    \sigma_{\mathrm{epi}(f)}
    (x^*,-1).
  \end{align}
\end{proof}
% conjugate chain rule %
 %%%%%%%%%%%%%%%%%%%%%%
\begin{proposition}

\end{proposition}
\begin{theorem}
  \emph{(Conjugate Chain Rule)}
  \label{cvxa_conjugate_chain_rule}
  Let 
  $
    A:
      \R^m \to \R^n
  $
  be a linear map (matrix)
  and
  $
    g:
      \R^n \to (-\infty, \infty]
  $
  a proper convex function. If
  $
    \text{Im}(A) \cap \text{ri}(\text{dom}(g))
    \neq
    \emptyset
  $
  it follows
  \begin{gather}
    ( g \circ A )^* ( x^* )
    =
    \inf_
          { y^* \in ( A^* )^{ -1 } ( x^* )}
                                          g^*( y^* )
                                          .
  \end{gather}
  Furthermore, 
    for any 
      $
        x^* \in \text{dom}( g \circ A)^*
      $
        there exists
          $
            y^* \in ( A^* )^{ -1 } ( x^* )
          $
            such that
              $
                ( g \circ A)^* ( x^* )
                =
                g^*( y^* )
              $.
\end{theorem}

% conjugate sum rule %
 %%%%%%%%%%%%%%%%%%%%

\begin{definition}
  \emph{(Infimal convolution)}
  Given functions
  $
    f_i:
    \R^n \to (-\infty, \infty]
  $
  for $ i = 1, \ldots, n $
  the \textbf{infimal convolution} of these functions as defined as
  \begin{gather}
    (f_1 \square \ldots \square f_m)(x)
    :=
    \inf_{
    \begin{smallmatrix}
      x_i \in \R^n \\
      \sum_{i = 1}^{m} 
        x_i
      =
      x
    \end{smallmatrix}
    }
    \sum_{i = 1}^{m}
      f_i(x_i)
  \end{gather}
\end{definition}


\begin{theorem}
  Let
  $
    f,g:
    \R^n \to (-\infty, \infty]
  $
  be proper convex functions 
  and
  $
  \text{ri}\left( \text{dom}(f) \right)
  \cap
  \text{ri}\left( \text{dom}(g) \right)
  \neq 
  \emptyset
  .
  $
  Then we have the conjugate sum rule
  \begin{gather}
    ( f + g )^*(x^*)
    =
    ( f^* \square g^*)(x^*)
  \end{gather}
  for all $x^* \in \R^n$.
  Moreover, the infimum in 
  $
    ( f^* \square g^*)(x^*)
  $
  is attained, i.e., for any
  $
    x^* \in \text{dom}(f+g)^*
  $
  there exists vectors $x_1^*, x_2^*$
  for which
  \begin{gather}
    (f+g)^*(x^*)
    =
    f^*(x_1^*)
    +
    g^*(x_2^*),
    \quad
    x^* = x_1^* + x_2^*.
  \end{gather}
\end{theorem}
\begin{proof}
  Let $x^*\in\R^n$ and fix $x_1^*,x_2^*\in\R^n$ such that
  $x^*=x^*_1+x^*_2$.
  We get
  \begin{align*}
    f^*(x^*_1)+g^*(x^*_2)
    &=
    \sup_{x\in\R^n}
    \inner{x^*_1}{x}-f(x)
    +
    \sup_{x\in\R^n}
    \inner{x^*_2}{x}-g(x)
    \\
    &\ge
    \sup_{x\in\R^n}
    \inner{x^*_1}{x}-f(x)
    +
    \inner{x^*_2}{x}-g(x)
    =
    \sup_{x\in\R^n}
    \inner{x^*_1+x^*_2}{x}-(f(x)+g(x))
    \\
    &=
    \sup_{x\in\R^n}
    \inner{x^*}{x}-(f+g)(x)
    =(f+g)^*(x^*)
  \end{align*}
  Taking the infimum over $x_1^*,x_2^*\in\R^n$ in the above display gives 
  $
  (f^*\square g^*)(x^*)
  \ge
  (f+g)^*(x^*).
  $
  Let us prove now $\le$ under the condition
  $
  \text{ri}\left( \text{dom}(f) \right)
  \cap
  \text{ri}\left( \text{dom}(g) \right)
  \neq 
  \emptyset
  .
  $
  The only case we need to consider is
  $
    (f+g)^*(x^*)<\infty.
  $
  Define two convex sets by
  \begin{align}
    \Omega_1
    &:=
    \left\{ 
      (x,\alpha,\beta)\in\R^{n+2}
      \colon
      \alpha\ge f(x)
    \right\}
    =
    \mathrm{epi}(f)\times \R,
    \\
    \Omega_2
    &:=
    \left\{ 
      (x,\alpha,\beta)\in\R^{n+2}
      \colon
      \beta\ge g(x)
    \right\}.
  \end{align}
  Similar to Lemma we get the representation
  \begin{gather}
    (f+g)^*(x^*)
    =
    \sigma_{\Omega_1\cap\Omega_2}
    (x^*,-1,-1).
  \end{gather}
  Indeed, the only thing we need to verify is
  $
    \mathrm{dom}(f)\cap\mathrm{dom}(g)
    =
    \mathrm{dom}(f+g).
  $
  The inclusion $\subseteq$ is clear.
  Assume towards a contradiction that
  $
    (f+g)(x)<\infty
  $
  and
  $
    f(x)=\infty.
  $
  Since $g(x)>-\infty$ it holds
  \begin{gather}
    \infty
    =
    \infty+g(x)
    =f(x)+g(x)
    =(f+g)(x)
    <
    \infty.
  \end{gather}
  This is a contradiction. The same holds for $f$ and $g$ reversed. It follows the inclusion $\supseteq$ and equality.
  By the support function intersection rule there exist triples
  \begin{gather}
    (x^*_1,-\alpha_1,-\beta_1),
    (x^*_2,-\alpha_2,-\beta_2)
    \in \R^{n+2}
    \quad
    \text{such that}
    \quad
    (x^*,-1,-1)
    =
    (x^*_1+x^*_2,-(\alpha_1+\alpha_2),-(\beta_1+\beta_2))
  \end{gather}
  and
  \begin{gather}
    (f+g)^*(x^*)
    =
    \sigma_{\Omega_1\cap\Omega_2}
    (x^*,-1,-1)
    =
    \sigma_{\Omega_1}
    (x^*_1,-\alpha_1,-\beta_1)
    +
    \sigma_{\Omega_2}
    (x^*_2,-\alpha_2,-\beta_2).
  \end{gather}
  Next we show
  $\beta_1=\alpha_2=0.$
  Suppose towards a contradiction that 
  $\beta_1\neq 0.$ 
  We fix 
  $(\overline{x},\overline{\alpha})\in\mathrm{epi}(f).$
  Then
  \begin{gather}
    \sigma_{\Omega_1}
    (x^*_1,-\alpha_1,-\beta_1)
    =
    \sup_{(x,\alpha,\beta)\in \mathrm{epi}(f)\times \R}
    \inner{x^*}{x}-\alpha \alpha_1 -\beta \beta_1
    \ge
    \sup_{\beta\in \R}
    \inner{x^*}{\overline{x}}-\overline{\alpha} \alpha_1 -\beta \beta_1
    =\infty.
  \end{gather}
  This contradicts
  $
    (f+g)^*(x^*)<\infty.
  $
  In a similar fashion we can derive a contradiction for $\alpha_2\neq0.$
  Employing Lemma and taking into account the structures of the sets 
  $\Omega_1$ and $\Omega_2$ this implies
  \begin{align}
    (f+g)^*(x^*)
    &=
    \sigma_{\Omega_1\cap\Omega_2}
    (x^*,-1,-1)
    =
    \sigma_{\Omega_1}
    (x^*_1,-1,0)
    +
    \sigma_{\Omega_2}
    (x^*_2,0,-1)
    \\
    &=
    \sigma_{\mathrm{epi}(f)}(x^*_1,-1)
    +
    \sigma_{\mathrm{epi}(g)}(x^*_2,-1)
    =
    f^*(x^*_1)
    +
    g^*(x^*_2)
    \ge
    (f^*\square g^*)(x^*).
  \end{align}
  This finishes the proof.
\end{proof}
