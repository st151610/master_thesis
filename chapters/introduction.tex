% SAY THAT RESEARCHERS ARE OFTEN LEFT WITH OBSERVATIONAL STUDYS TO ANSWER THEIR QUESTIONS

% SAY SOMETHING ABOUT PS METHODS AS TO DO WITH CAUSAL INFERENCE IN OS

% SAY THAT MODEL DEPENDENCY IS OFTEN A PROBLEM WITH PS

% MENTION DIFFERENT APPROACHES TO ALLEVIATE THIS PROBLEM

% HIGHLIGHT THE METHODS BEING DISCUSSED IN THE THESIS

% OUTLINE STRUCTURE 
%% - WEIGHTING APPROACH BINARY
%%  - MODIFICATIONS IN BINARY SETTING TO ESTIMATE ATE
%%  - EXTENSION TO CONTINUOUS TREATMENT
%% - MATCHING APPROACH BINARY 
%%  - POSSIBLE EXTENSION
%% - SIMULATION STUDY
%% - REAL DATA USECASES


Researchers are often left with observational studies to answer questions about causality.
When Confounders are present the task can become arbitrarily complex.
Propensity Score methods 
\cite{Rosenbaum1983},
e.g. IPW or matching,
are popular methods to adjust for confounders.
They rely heavily on estimates of the true propensity score,
wich are known to suffer from model dependencies and misspecification
\cite{Kang2007}.
This issue becomes more pressing when moving from binary treatment to 
Continuous treatment\cite{Hirano2005}%
.% 
Therefore Methods have been proposed to directly target imbalance in the data.
\cite{Fong2018}\cite{Hainmueller2012}\cite{Zubizarreta2015}.
We take a closer look at \cite{Wang2019} and extend the analysis to settings with Continuous treatment 
\cite{Vegetabile2020}\cite{Tubbicke2020}.
