% SAY THAT RESEARCHERS ARE OFTEN LEFT WITH OBSERVATIONAL STUDYS TO ANSWER THEIR QUESTIONS

% SAY SOMETHING ABOUT PS METHODS AS TO DO WITH CAUSAL INFERENCE IN OS

% SAY THAT MODEL DEPENDENCY IS OFTEN A PROBLEM WITH PS

% MENTION DIFFERENT APPROACHES TO ALLEVIATE THIS PROBLEM

% HIGHLIGHT THE METHODS BEING DISCUSSED IN THE THESIS

% OUTLINE STRUCTURE 
%% - WEIGHTING APPROACH BINARY
%%  - MODIFICATIONS IN BINARY SETTING TO ESTIMATE ATE
%%  - EXTENSION TO CONTINUOUS TREATMENT
%% - MATCHING APPROACH BINARY 
%%  - POSSIBLE EXTENSION
%% - SIMULATION STUDY
%% - REAL DATA USECASES


Researchers are often left with observational studies to answer questions about causality. When confounders are present the task of infering causality can become arbitrarily complex. Propensity score methods \cite{Rosenbaum1983}, e.g. inverse probability weighting or matching, are popular methods to adjust for confounders. Usually these methods rely heavily on estimates of the true propensity score, which are known to suffer from model dependencies and misspecification\cite{Kang2007}. This issue becomes more pressing when moving from binary to continuous treatment\cite{Hirano2005}. Therefore methods have been developed to directly target imbalances in the data\cite{Fong2018}\cite{Hainmueller2012}\cite{Zubizarreta2015}.
We take a closer look at \cite{Wang2019} and extend the analysis to settings with continuous treatment 
\cite{Vegetabile2020}\cite{Tubbicke2020}.
