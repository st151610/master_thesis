We begin by defining convex sets
%

\begin{definition}
  A subset $\Omega\subseteq \R^n$ is called CONVEX if we have $\lambda x+(1-\lambda)y\in \Omega$ for all $x,y\in \Omega$ and $\lambda\in (0,1)$. 
\end{definition}

Clearly, the line segment 
$[a,b]:=\left\{ \lambda a+(1-\lambda)b\,\mid \, \lambda\in [0,1] \right\}$ is contained in $\Omega$ for all $a,b\in \Omega$ if and only if $\Omega$ is a convex set.
%

Next we define convex functions. 
%

The concept of convex functions is closely related to convex sets.
%  
 
The line segment between two points on the graph of a convex function lies on or above and does not intersect the graph.
%

In other words: The area above the graph of a convex function $f$ is a convex set, i.e. the \textit{epigraph}
$\text{epi}(f):=\left\{ (x,\alpha)\in \R^n\times\R\,\mid\, f(x)\le \alpha\right\}$ is a convex set in $\R^{n+1}$.
%

Often an equivalent characterisation of convex functions is more useful.
%

\begin{theorem}
  The convexity of a function $f:\R^n\to \overline{\R}$ on $\R^n$ is equivalent to the following statement:

  For all $x,y\in \R^n$ and $\lambda\in(0,1)$ we have 
    \begin{align}
      f(\lambda x + (1-\lambda)y)\le \lambda f(x)+(1-\lambda)f(y).
    \end{align}
\end{theorem}
