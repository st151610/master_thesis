To control empirical processes - apart from strong theorems - we need the notion of bracketing number and integral (see \cite[page 270]{Vaart2000}). 
Given two functions $\underline{f}\le \overline{f}$,
\begin{gather*}
  \text{
the bracket
  }\quad
[\underline{f},\overline{f}]
\quad 
\text{
is the set of all functions $f$ with 
}\quad 
\underline{f}\ \le\ f \ \le\  \overline{f}
\,.
\end{gather*}
For $\varepsilon>0$
we define a
\begin{gather*}
  \text{
$(\varepsilon, L^{r}(\P))$ -bracket
to be a bracket
  }
  \quad
[\underline{f},\overline{f}]
\quad
\text{with}
\quad
\norm{\overline{f}-\underline{f}}_{ L^r(\P)}
\ <\  \varepsilon
\,.
\end{gather*}
The \textbf{
bracketing number
} 
$
N_{[\,]}(\varepsilon, \mathcal{F}, L^r(\P))
$
is 
the minimum number of 
$(\varepsilon, L^{r}(\P))$-brackets needed to cover $\mathcal{F}$.

For most classes $\mathcal{F}$ the bracketing number grows to infinity for $\varepsilon\to 0$.
To measure the speed of growth we introduce 
for $\delta>0$
the
\textbf{bracketing integral}
\begin{gather*}
     J
    _{[\,]}
    (
    \delta
    ,
    \mathcal{F}
    ,
    L_r(\P)
    )
    \ 
    =
    \ 
  \int_0^{\delta}
      \sqrt{
        \log 
      N_{[\,]}
\left( \varepsilon, \mathcal{F}_N, L^r(\P) \right)
    }
    \,
    d\varepsilon
    \,.
\end{gather*}

Next we give a technical lemma to 
bound the bracketing numbers of products of two function classes, that is,
\begin{gather*}
  \mathcal{F}\cdot \mathcal{G}
  \ 
  :=
  \ 
  \left\{ 
    f\cdot g
    \ 
    \colon
  \ 
    f\in\mathcal{F},
    g\in\mathcal{G}
  \right\}\,.
\end{gather*}
\begin{lemma}
  \label{lem_prod_br}
  Let
  $\mathcal{F}$ and $\mathcal{G}$ be two function classes 
  with envelope functions $F$ and $G$ satisfying
  $\norm{F}_\infty,\norm{G}_\infty\le 1$.
  For all $\varepsilon>0$ and all $r\in [1,\infty)$ it holds
  \begin{gather*}
    N_{[\,]}(2\varepsilon,\mathcal{F}\cdot\mathcal{G},\mathrm{L}_r(\P))
    \
    \le
    \ 
    N_{[\,]}(\varepsilon,\mathcal{F},\mathrm{L}_r(\P))
    \cdot
    N_{[\,]}(\varepsilon,\mathcal{G},\mathrm{L}_r(\P))
    \,.
  \end{gather*}
\end{lemma}
\begin{proof}
  The proof is simple. We omit the details.
%  Let $f\in\mathcal{F}$ and $g\in\mathcal{G}$.
%  We can choose two 
%  $(\varepsilon,L^r(\P))$
%  brackets
%  $[\underline{f},\overline{f}]$
%  and
%  $[\underline{g},\overline{g}]$
%  containing $f$ and $g$ with 
%  $\norm{\underline{f}}_\infty,\norm{\overline{f}}_\infty\le\norm{F}_\infty\le 1$
%  and
%  $\norm{\underline{g}}_\infty,\norm{\overline{g}}_\infty\le\norm{G}_\infty\le 1$.
%  We the get an 
%  $(2\varepsilon,L^r(\P))$
%  $[\underline{h},\overline{h}]$
%  bracket, containing $f\cdot g$, by
\end{proof}

