\begin{lemma}
  \label{aa:mean:r3:lem:conv}
  Let Assumption~\eqref{asu:treatment_asign_str_ing}, Assumption~\ref{asu:feas_dual_sol} and Assumption~\ref{asu:x_finite} hold true.
  Then
  \begin{align*}
  \sup_{z\in\R}\left| R_3(z) \right|
  \ 
  \overset{\P}{\to}
  \ 
  0
  \end{align*}
\end{lemma}
\begin{proof}
  Let
  $z\in\R$
  and let Assumption~\eqref{asu:treatment_asign_str_ing} hold true.
    By Lemma~\ref{lem:f_z}
  it holds 
  \begin{align*}
    &
  f_z(T,X,Y(T))
  \ 
  \in
  \ 
  L^1(\P)\,, 
  \\
  &
  f_z(T,X,Y(T))
  \ 
  \perp
  \ 
  D_N
  \,,
  \\
  \E
  [
  &
  f_z(T,X,Y(T))
    |X
  ]
  \ 
  =
  \ 
  0
  \,.
  \end{align*}
  Thus, 
  it follows from Lemma~\ref{w.Z=0}
  \begin{gather*}
    \E
    \left[
      w_0^\dagger(X)
      \cdot
      f_z(T,X,Y(T))
    \right]
    \ 
    =
    \ 
    0
    \,.
  \end{gather*}
  Since
  \begin{align*}
    \E
    \left[
      \frac{1}{\pi(X)}
      f_z(T,X,Y(T))
    \right]
    &
    \ 
    =
    \ 
    \E
    \left[
      \frac{T}{\pi(X)}
      \left( 
        \mathbf{1}
        _{\left\{  Y(T)\,\le\,z \right\}}
        -
        F_{Y(1)}(z|X)
      \right)
    \right]
    \\
    &
    \ 
    =
    \ 
    \E \left[
        \mathbf{1}
        _{\left\{  Y(1)\,\le\,z \right\}}
    \right]
    -
        F_{Y(1)}(z)
    \  
    =
    \ 
    0
  \end{align*}
  by Lemma~\ref{ps_weights_lemma},
  it follows
  \begin{align*}
    \E
    \left[ 
      \left( 
      w_0^\dagger(X)
      -
      \frac{1}{\pi(X)}
      \right)
     \cdot
     f_z(T,X,Y(T))
    \right]
    \ 
    =
    \ 
  0
  \,.
  \end{align*}
  But then 
  \begin{align*}
    R_3(z)
    &
    \ 
    =
    \ 
  \frac{1}
  {
\sqrt{N}
  }
    \sum_{i=1}^{N} 
    \left[ 
    \left( 
    w_0^\dagger(X_i) 
    -
    \frac{1}{\pi(X_i)}
    \right)
    T_i
    \left( 
    \mathbf{1}{\left\{ Y_i \le z \right\}}
    -
  F_{Y(1)}(z|X_i)
    \right)
    \right]
    \ 
    =
    \ 
    \G_N
    \left( 
      \left( 
      w_0^\dagger
      -
      \frac{1}{\pi}
      \right)
     \cdot
     f_z
    \right)
    \,.
  \end{align*}
Let
  $g^\dagger$ 
  denote the stochastic process \eqref{error_process}, that is,
\begin{align*}
  g^\dagger
  (x)
  \ 
  :=
  \ 
  \mathbf{1}{
    \left\{ 
      \sup_{y\in A_N(x)}
      \left| 
      w_0^\dagger(y)
      -
      \frac{1}{\pi(y)}
      \right|
      \,
      \le
      \,
      \varepsilon_N
    \right\}
  }
  \left( 
    w_0^\dagger(x)
      -
      \frac{1}{\pi(x)}
  \right)
  \cdot
  \mathbf{1}
  \bigcup_{k=1}^N
  \left\{ 
    x=X_k
  \right\}
\end{align*}
for all $x\in\R^d$.
  If
  \begin{gather}
    \label{event}
    T_i
    \cdot
    \left| 
    w_0^\dagger(X_i)
    \ 
    -
    \ 
    \frac{1}{\pi(X_i)} 
    \right|
    \ 
    \le
    \ 
    \varepsilon_N
    \qquad
    \text{for all}\ 
    i\in \left\{ 1,\ldots,N \right\}
    \,,
  \end{gather}
  it holds
  for all $i\in \left\{ 1,\ldots,N \right\}$
  \begin{gather}
    \label{7926}
    g^\dagger
    (X_i)
    \cdot
    f_z
    (T_i,X_i,Y_i(T_i))
    \ 
    =
    \ 
  \left( 
      w_0^\dagger(X_i)
      -
      \frac{1}{\pi(X_i)}
  \right)
    f_z
    (T_i,X_i,Y_i(T_i))
      \,.
  \end{gather}
  Thus, if \eqref{event} holds it follows
\begin{align*}
 R_3(z)
 \ 
 =
 \ 
 \G_N(g^\dagger \cdot f_z)
 \,.
\end{align*}
  It follows
\begin{align*}
    \P
    \left[ 
      \sup_{z\in\R}
     | 
    R_3(z)
    |
      \ge
      \varepsilon
    \right]
    &
    \ 
    \le
    \ 
    \P
    \left[ 
      \sup_{z\in\R}
     | 
    R_3(z)
    |
      \ge
      \varepsilon
      \ 
      \text{and}
      \ 
      T_i\cdot
    | 
    w_0^\dagger(X_i)- 1/\pi(X_i)
    |
    \ 
    \le
    \ 
    \varepsilon_N
    \ \text{for all}\ i\in\{1,\ldots,N\}
    \right]
    \\
    &
    \quad
    +
    \ 
    \P
    \left[ 
      T_i\cdot
    | 
    w_0^\dagger(X_i)- 1/\pi(X_i)
    |
    \ 
    >
    \ 
    \varepsilon_N
    \ 
    \ \text{for some}\ i\in\{1,\ldots,N\}
    \right]
    \\
    &
    \ 
    \le
    \ 
    \P
    \left[ 
      \norm{\G_N}^*_{\mathcal{F}_N\cdot\mathcal{F}}
      \ge
      \varepsilon
    \right]
    \ 
    +
    \ 
    \ 
    \P
    \left[ 
      \max_{i\in \left\{ 1,\ldots,N \right\}}
      T_i\cdot
    | 
    w_0^\dagger(X_i)- 1/\pi(X_i)
    |
    \ 
    >
    \ 
    \varepsilon_N
    \right]
    \\
    &
    \ 
    \to
    \ 
    0
    \,.
\end{align*}
The convergence of the first term follows from
Assumption~\ref{asu:x_finite},
Lemma~\ref{lem:br_n_st},
Lemma~\ref{lem:x_finite}, and
Lemma~\ref{aa:r3:lemma:1}.
The convergence of the second term follows from
Assumption~\ref{asu:feas_dual_sol} and
Theorem~\ref{th:max_weight_cons}.
\end{proof}
\begin{remark}
  There is a similar section \cite[page 27-28]{Wang2019}.
  It contains interesting ideas. But one has to be careful to understand the notation.
  Statements like
  \begin{align*}
    A\ \lesssim\  \E[A] \qquad
    \text{by Markov's inequality}
  \end{align*}
  would be better expressed as
  \begin{align*}
    \P[A\ge \varepsilon]
    \ 
    \le 
    \ 
    \frac
    {\E[A]}
    {\varepsilon}
    \,.
  \end{align*}
  Wang and Zubizarretas argument why a bound on the bracketing numbers of the difference $w-1/\pi$ exist is incomplete. 
  The point is that we should also consider the weights in the difference $w-1/\pi$. 
  We do this in Section~5.1.2., where we profit from the rigorous construction of the weights in Chapter~3.
\end{remark}


