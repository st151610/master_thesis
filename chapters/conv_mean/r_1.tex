\begin{lemma}
  \label{aa:mean:l:r1}
  Let
$\sqrt{N}\norm{\delta}_1\overset{\P}{\to}0$.
Then it holds
$\sup_{z\in\R}|R_1(z)|\overset{\P}{\to}0$.
  \end{lemma}
\begin{proof}
If $s_N\in \mathrm{int}\,\Theta_N$, then 
$
w_1^\dagger(X_1),\cdots,w_N^\dagger(X_N)
$
satisfy the box constraints of Problem~\ref{bw:1:primal} (in the form with the $T_i$ instead of $n$).
  \begin{align}
    \label{R_1:1}
    \begin{split}
    \sup_{z\in\R}
    \left| 
    R_1(z)
    \right|
    &
    \ 
    =
    \ 
  \sqrt{N}
  \sup_{z\in\R}
  \sum_{k=1}^{N} 
  \left[ 
  \frac{1}{N}
  \left( 
    \sum_{i=1}^{N} 
    T_i
    \cdot
    w^\dagger(X_i)
    \cdot
    B_k(X_i)
    \ 
    -
    \ 
    \sum_{i=1}^{N} 
    B_k(X_i)
  \right)
  \cdot
  F_{Y(1)}(z|X_k)
  \right]
    \\
    &
    \ 
    \le
    \ 
  \sqrt{N}
  \sum_{k=1}^{N} 
  \left| 
  \frac{1}{N}
  \left( 
    \sum_{i=1}^{N} 
    T_i
    \cdot
    w^\dagger(X_i)
    \cdot
    B_k(X_i)
    \ 
    -
    \ 
    \sum_{i=1}^{N} 
    B_k(X_i)
  \right)
  \right|
  \cdot
    \sup_{z\in\R}
  F_{Y(1)}(z|X_k)
  \\
    &
    \ 
    \le
    \ 
  \sqrt{N}
  \norm{\delta}_1
    \end{split}
  \end{align}
  The last inequality is due to $F_{Y(1)}\in[0,1]$ and the assumption that $(w_i^\dagger(X_i))$ satisfies the box constraints of Problem~\ref{bw:1:primal}.
  It remains to analyse the case $s_N\notin \mathrm{int}\,\Theta_N$.
  Then $(w_i^\dagger(X_i))$ are not the optimal weights of Problem~\ref{bw:1:primal}
  and we can't employ the box constraints to bound $R_1$.
  But we can use 
  $
  \P \left[ s_N\notin \Theta_N \right]
  \to 0
  $.
  To this end, note 
  \begin{align*}
    \P
    \left[ 
      \frac{
  \P \left[ s_N\notin \Theta_N \right]
      }{\sqrt{N}}
      \sum_{i=1}^{N} 
      \left( 
      \frac{T_i}{\pi(X_i)}
      -
      1
      \right)
      \ge \varepsilon
    \right]
    \le
  \P \left[ s_N\notin \Theta_N \right]
  \frac{\mathbf{Var}[T/\pi(X)]}{\varepsilon^2}
  \to 
  0
  \end{align*}
  for all $\varepsilon>0$.


  Since we assume 
$\sqrt{N}\norm{\delta}_1\overset{\P}{\to}0$
it holds
$\sup_{z\in\R}|R_1(z)|\overset{\P}{\to}0$.
\end{proof}
\begin{remark}
  We want to comment on the box constraints of Problem~\ref{bw:1:primal}, that is,
 \begin{gather*}
      \left| 
      \frac{1}{N} 
      \left( 
      \sum_{i = 1}^{n} 
      w^\dagger(X_i)
      B_k(X_i)
      -
      \sum_{i=1}^{N} 
      B_k(X_i)
      \right)
    \right|
    \ 
    \le 
    \ 
    \delta_k
    \qquad
    \text{for all}\ 
    k \in \left\{ 1, \ldots, N \right\}
    \,.
  \end{gather*}
  Note, that the first sum goes over $\left\{ 1,\ldots,n \right\}$ while the second sum goes over $\left\{ 1,\ldots,N \right\}$.
  A second, equivalent version of the constraints is
  \begin{gather*}
      \left| 
      \frac{1}{N} 
      \left( 
      \sum_{i = 1}^{N} 
      T_i
      w^\dagger(X_i)
      B_k(X_i)
      -
      \sum_{i=1}^{N} 
      B_k(X_i)
      \right)
    \right|
    \ 
    \le 
    \ 
    \delta_k
    \qquad
    \text{for all}\ 
    k \in \left\{ 1, \ldots, N \right\}
    \,.
  \end{gather*}
  Now both sums go over $\left\{ 1,\ldots,N \right\}$ and the
  indicator of treatment $T_i$ takes care that in the first sum only the terms with $i\le n$ are effective. 
  Having this flexibility with the versions helps. I regard the first version as suitable for non-probabilistic computations, although $n$ is of course a random variable. On the other hand, the second version is more honest, exactly telling the dependence on the indicator of treatment. This version is useful in probabilistic computations. 

  Also we want to comment on the assumption on $\norm{\delta}$.
  Playing around with norm equivalences we discover that 
  $\sqrt{N}\norm{\delta}_1\overset{\P}{\to}0$ for $N\to \infty$ is the weakest
  (natural) assumption to
  control $R_1$.
  Indeed, other ways to continue the second row in \eqref{R_1:1} are
  \begin{gather*}
    (\,\cdots)
    \ 
  \le
    \ 
  \sqrt{N}
  \norm{\delta}_2
  \left( 
  \sum_{k=1}^{N} 
  \left( 
    \sup_{z\in\R}
  F_{Y(1)}(z|X_k)
  \right)^2
\right)^{1/2}
\ 
\le
\ 
N
  \norm{\delta}_2\,,
  \end{gather*}
  by the Cauchy-Schwarz inequality and
  $
  F_{Y(1)}\in [0,1]
  $,
or
\begin{gather*}
  (\,\cdots)
  \ 
  \le
  \ 
  \sqrt{N}
  \norm{\delta}_\infty
  \sum_{k=1}^{N} 
    \sup_{z\in\R}
  F_{Y(1)}(z|X_k)
  \ 
  \le
  \ 
  N^{3/2}
  \norm{\delta}_\infty
  \,.
\end{gather*}
Since $\delta\in \R^N$, however, it holds
\begin{gather*}
  \sqrt{N}\norm{\delta}_1
  \ 
  \le
  \ 
  N\norm{\delta}_2
  \ 
  \le
  \ 
  N^{3/2}\norm{\delta}_\infty
  \,.
\end{gather*}
With hindsight, the assumption 
$\sqrt{N}\norm{\delta}_1\overset{\P}{\to}0$ for $N\to \infty$ 
  also 
  suffices 
  to control the second (or first) occurrence of a term, that we control by assumptions on $\norm{\delta}$.
This is the \textbf{second term} of \eqref{c:1}, where we estimate
\begin{gather*}
  \inner{\delta}{\left| \Delta \right|}
  \ 
  =
  \ 
  \sum_{k=1}^{N} 
  \delta_k
  \left| \Delta_k \right|
  \ 
  \le
  \ 
  \norm{\delta}_1
  \norm{\Delta}_\infty
  \ 
  \le
  \ 
  \norm{\delta}_1
  \norm{\Delta}_2
  \ 
  \le
  \ 
  \norm{\delta}_1
  \varepsilon
  \ 
  \overset{\P}{\to}
  \ 
  0
  \quad
  \text{for}\ 
  N\to \infty
  \,.
\end{gather*}

\end{remark}

