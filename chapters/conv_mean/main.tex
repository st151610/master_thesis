Is there a better estimator of the distribution function than the empirical distribution function?
Yes, a weighted empirical distribution function.
Is there a worse estimator of the distribution function than the empirical distribution function?
Yes, a weighted empirical distribution function.
It depends on the weights.

In Chapter~4 we show that the optimal weights of Problem~\ref{bw:1:primal} are consistent estimators of the best possible weights - the inverse propensity score weights.
Now we want to use this to obtain good asymptotic properties of a weighted mean estimator.
To this end, let $Y_1,\ldots,Y_N$ be independent and identically distributed observed outcomes. To do this, we have to drop the order $T_i=1$ for $i\le n$ introduced in Chapter~2. 
Note that $Y_i=Y_i(T_i)$, where $(Y_i(0),Y_i(1))$ are the potential outcomes of unit $i$. 

We mentioned in the introduction to Chapter~2 that the weights work for different outcomes.
If Assumption~\ref{asu:treatment_asign_str_ing} holds, by the law of large numbers and by the central limit theorem it holds
\begin{align*}
  &
  \frac{1}{N}
  \sum_{i=1}^{N} 
  \frac{T_i}{\pi(X_i)}f(Y_i(T_i))
  \ 
  \overset{\P}
  {\to}
  \ 
  \E
  \left[ 
    f(Y(1))
  \right]
  \intertext{and}
  &
  \frac{1}{\sqrt{N}}
  \sum_{i=1}^{N} 
  \frac{T_i}{\pi(X_i)}f(Y_i(T_i))
  \qquad
  \text{converges in distribution}
  \,.
\end{align*}
By the consistency of the weights, we hope to recover the good asymptotic behaviour of the propensity score weights.
To prove this,
we could try the following error decomposition.
\begin{align*}
  &
  \frac{1}{N}
  \sum_{i=1}^{N} 
  T_i\cdot w^\dagger_0(X_i)\cdot f(Y_i(T_i))
  \ 
  - 
  \ 
  \E
  \left[ 
    f(Y(1))
  \right]
  \\
  &
  \ 
  =
  \ 
  \frac{1}{N}
  \sum_{i=1}^{N} 
  T_i
  \left( 
  w^\dagger_0(X_i)
  -
  \frac{T_i}{\pi(X_i)}
  \right)f(Y_i(T_i))
  \ 
  + 
  \ 
  \left( 
  \frac{1}{N}
  \sum_{i=1}^{N} 
  \frac{T_i}{\pi(X_i)}f(Y_i(T_i))
  \ 
  -
  \ 
  \E
  \left[ 
    f(Y(1))
  \right]
  \right)
  \,.
\end{align*}
Clearly, the second term goes to 0 in probability.
Since the difference in the first term goes to 0, by the consistency of the weights, we would expect the first term also to be well behaved.
It turns out that something similar is the case for an estimate of the distribution function of $Y(1)$ (only the argument is much more involved).
The high-level idea remains that the best possible weights, the propensity score weights, are well behaved and the weights of Problem~\ref{bw:1:primal} approximate them (reasonably) well.

Throughout this section we use the following notation.
Let $F_{Y(1)}$ denote the distribution function of $Y(1)$, that is,
\index{$F_{Y(1)}$, distribution function of the potential outcome under treatment}
\begin{gather*}
  F_{Y(1)}
  \ 
  \colon
  \ 
  \R
  \ 
  \to
  \ 
  [0,1]
  \, 
  , 
  \qquad
  z
  \ 
  \mapsto
  \ 
  \P
  [
  Y(1)
  \le
  z
  ]
  \,.
\end{gather*}
Let $F_{Y(1)}(\cdot|x)$ denote the distribution function of $Y(1)$ conditional on $X=x\in\mathcal{X}$, that is,
\index{$F_{Y(1)}(\cdot \lvert x)$, conditional distribution function of the potential outcome under treatment}
\begin{gather}
  \label{3228}
  F_{Y(1)}
  (z|x)
\ 
  =
\ 
  \P
  [
  Y(1)
  \le
  z
  \,
  |
  \,
  X=x
  ]
  \qquad
  \text{for all}\ 
  (z,x)\in\R\times\mathcal{X}
  \,.
\end{gather}

We illustrate the flexibility of 
the weighted mean estimator by 
extending the method of \cite{Wang2019} to
estimates of 
the distribution function of $Y(1)$, that is, $F_{Y(1)}$.
For the asymptotic analysis of estimating the mean $\E[Y(1)]$ see \cite[Proof of Theorem~3]{Wang2019}.
To make this extension, the central observation is, that we can adapt the error decomposition in \cite[page 27]{Wang2019} 
to estimates of the distribution function $F_{Y(1)}$ of $Y(1)$.
We do this in Lemma~\ref{aa:mean:lemma_decomp}.
With this modification, we aim at proving
the convergence of
\begin{gather*}
    \sqrt{N}
    \left( 
  \frac{1}{N}
    \sum_{i=1}^{n} 
    w_0^\dagger(X_i)
    \mathbf{1}\left\{ Y_i(T_i)\,\le\, z \right\}
    \ 
    -
    \ 
    F_{Y(1)}(z)
    \right)
    _{z\in\R}
    \,
  \end{gather*}
  in
  $l^\infty(\R)$
  to a Gaussian process with mean 0 and covariance specified in Theorem~\ref{th:main}.
\section{Tools}
  For the subsequent analysis we need the theory of empirical processes.
For an introduction to empirical processes see \cite[§19]{Vaart2000}. For a thorough treatment see \cite[§2]{vaart2013}. 
\subsection{Empirical Processes - Definition}
  Let 
$
  \left( 
    \Omega,
    \mathcal{A},
    \P
  \right)
$
be a probability space,
$
  \left( 
    \mathcal{Z},
    \Sigma
  \right)
$
a measurable space, and 
\begin{gather*}
  \xi_1,\ldots,\xi_N
  :
  \left( 
    \Omega,
    \mathcal{A},
    \P
  \right)
  \to
  \left( 
    \mathcal{Z},
    \Sigma
  \right)
  \quad
  \text{independent and identically-distributed
  }
\end{gather*}
random variables
with probability distribution $\P_{\!\xi}$.
\index{$\P_{\xi}$, probability distribution of the random variable $\xi$}
Let $\mathcal{F}$ be a class of measurable functions 
\index{$\mathcal{F}$, class of measurable functions}
$
  f:
  \left( 
    \mathcal{Z},
    \Sigma
  \right)
    \to
  \left( 
    \R,
    \mathcal{B}(\R)
  \right)
$, where
$
    \mathcal{B}(\R)
$
is the Borel-$\sigma$-algebra on $\R$.
\index{$\mathcal{B}(\cdot)$, Borel-$\sigma$-algebra}
Then $\mathcal{F}$
induces a stochastic process by
\begin{gather}
  f
  \ 
  \mapsto
  \ 
  \G_N f 
  \ 
  :=
  \ 
  \frac{1}{\sqrt{n}}
  \sum_{i=1}^{N} 
  \left(
    f(\xi_i)
    -
    \E_\xi[f]
  \right)
  \,,
\end{gather}
where
$
    \E_\xi[f]
    :=
    \int_\mathcal{Z}
    f
    \,
    d\P_\xi
$.
We call
$\G_N$ the  \textbf{empirical process} indexed by $\mathcal{F}$.
\index{$\G_N$, empirical process}
The purpose of this construction is to study the behaviour of a centered, scaled arithmetic mean uniformly over $\mathcal{F}$.
To this end, we define the (random) norm
\begin{gather}
  \norm{\G_n}_\mathcal{F}
  :=
  \sup_
        { f \in \mathcal{F}}
        \left|
          \G_N f
        \right|
        .
\end{gather}
\index{
$
  \norm{\G_n}^*_\mathcal{F}
$, (measurable version of the) uniform norm of an empirical process}
We stress that 
$
  \norm{\G_n}_\mathcal{F}
$
often ceases to be measurable, even in simple situations~\cite[page 3]{vaart2013}.
To deal with this, we introduce the notion of \textbf{outer expectation} $\E^*$ (see \cite[page~6]{vaart2013}):
\index{$
  \E^*[Z]
$, outer expectation}
\begin{gather*}
  \E^*[Z]
  \ 
  :=
  \ 
    \inf
  \left\{ 
    \E[U]
  \ 
  \lvert
  \ 
    U\ge Z,
    \ 
    U:
  \left( 
    \Omega,
    \mathcal{A},
    \P
  \right)
  \to 
  \left( 
    \overline{\R},
    \mathcal{B}(\overline{\R})
  \right)
  \text{measurable and}
  \ 
  \E[U]<\infty
  \right\}
\end{gather*}
In our application the technical difficulties halt at this point, because we only consider $Z$ with $\E^*[Z]<\infty$. Then there exists a smallest measurable function $Z^*$ dominating $Z$ with
$\E^*[Z]=\E[Z^*]$ (see \cite[Lemma~1.2.1]{vaart2013}).

An \textbf{envelope function} $F$ of a class $\mathcal{F}$ satisfies 
\index{envelope function}
\begin{align*}
|f(z)|
\ 
\le
\ 
F(z)< \infty 
\qquad
\text{for all}\ 
f\in\mathcal{F}
\ \text{and all}\ 
z\in\mathcal{Z}
\,.
\end{align*}


\subsection{Bracketing Numbers and Integral}
  To control empirical processes - apart from strong theorems - we need the notion of bracketing number and integral (see \cite[page 270]{Vaart2000}). 
Given two functions $\underline{f}\le \overline{f}$,
\begin{gather*}
  \text{
the bracket
  }\quad
[\underline{f},\overline{f}]
\quad 
\text{
is the set of all functions $f$ with 
}\quad 
\underline{f}\ \le\ f \ \le\  \overline{f}
\,.
\end{gather*}
For $\varepsilon>0$
we define a
\begin{gather*}
  \text{
$(\varepsilon, L^{r}(\P))$ -bracket
to be a bracket
  }
  \quad
[\underline{f},\overline{f}]
\quad
\text{with}
\quad
\norm{\overline{f}-\underline{f}}_{ L^r(\P)}
\ <\  \varepsilon
\,.
\end{gather*}
The \textbf{
bracketing number
} 
$
N_{[\,]}(\varepsilon, \mathcal{F}, L^r(\P))
$
is 
the minimum number of 
$(\varepsilon, L^{r}(\P))$-brackets needed to cover $\mathcal{F}$.

For most classes $\mathcal{F}$ the bracketing number grows to infinity for $\varepsilon\to 0$.
To measure the speed of growth we introduce 
for $\delta>0$
the
\textbf{bracketing integral}
\begin{gather*}
     J
    _{[\,]}
    (
    \delta
    ,
    \mathcal{F}
    ,
    L_r(\P)
    )
    \ 
    =
    \ 
  \int_0^{\delta}
      \sqrt{
        \log 
      N_{[\,]}
\left( \varepsilon, \mathcal{F}_N, L^r(\P) \right)
    }
    \,
    d\varepsilon
    \,.
\end{gather*}

Next we give a technical lemma to 
bound the bracketing numbers of products of two function classes, that is,
\begin{gather*}
  \mathcal{F}\cdot \mathcal{G}
  \ 
  :=
  \ 
  \left\{ 
    f\cdot g
    \ 
    \colon
  \ 
    f\in\mathcal{F},
    g\in\mathcal{G}
  \right\}\,.
\end{gather*}
\begin{lemma}
  \label{lem_prod_br}
  Let
  $\mathcal{F}$ and $\mathcal{G}$ be two function classes 
  with envelope functions $F$ and $G$ satisfying
  $\norm{F}_\infty,\norm{G}_\infty\le 1$.
  For all $\varepsilon>0$ and all $r\in [1,\infty)$ it holds
  \begin{gather*}
    N_{[\,]}(2\varepsilon,\mathcal{F}\cdot\mathcal{G},\mathrm{L}_r(\P))
    \
    \le
    \ 
    N_{[\,]}(\varepsilon,\mathcal{F},\mathrm{L}_r(\P))
    \cdot
    N_{[\,]}(\varepsilon,\mathcal{G},\mathrm{L}_r(\P))
    \,.
  \end{gather*}
\end{lemma}
\begin{proof}
  The proof is simple. We omit the details.
%  Let $f\in\mathcal{F}$ and $g\in\mathcal{G}$.
%  We can choose two 
%  $(\varepsilon,L^r(\P))$
%  brackets
%  $[\underline{f},\overline{f}]$
%  and
%  $[\underline{g},\overline{g}]$
%  containing $f$ and $g$ with 
%  $\norm{\underline{f}}_\infty,\norm{\overline{f}}_\infty\le\norm{F}_\infty\le 1$
%  and
%  $\norm{\underline{g}}_\infty,\norm{\overline{g}}_\infty\le\norm{G}_\infty\le 1$.
%  We the get an 
%  $(2\varepsilon,L^r(\P))$
%  $[\underline{h},\overline{h}]$
%  bracket, containing $f\cdot g$, by
\end{proof}


\subsection{Maximal Inequality}
  In our application we need concentration inequalities for 
$
  \norm{\G_n}^*_\mathcal{F}
$.
One easy way to obtain this is, to use a maximal inequality (see Theorem~\ref{th:max_ineq}) to control the expectation,
together with Markov's inequality. There are also Bernstein-like inequalities for empirical processes (see \cite[§2.14.2]{vaart2013}). 
\begin{theorem}
  \label{th:max_ineq}
  \emph{(Maximal inequality)}
  For any class $\mathcal{F}$ of measurable functions with envelope function $F,$
  \begin{gather*}
    \E^*
    \norm{
      \G
      _n
      }
      _\mathcal{F}
    \ 
    \lesssim
    \ 
    J
    _{[\,]}
    (
    \norm{
      F
    }
    _{ L^2(\P)}
    ,
    \mathcal{F}
    ,
    L^2(\P)
    )
    \, 
    .
  \end{gather*}
\end{theorem}
\begin{proof}
  \cite[Corollary~19.35]{Vaart2000}
\end{proof}

\begin{lemma}
  \label{markov_max_lemma}
  Let $(\mathcal{H}_N)$ be a sequence of measurable function classes with envelope functions $(H_N)$.
  If
  \begin{gather*}
    J_{[\, ]}
    \left( 
    \norm{H_N}_{ L^2(\P)}
    ,
    \mathcal{H}_N
    ,
     L^2(\P)
    \right)
    \ 
    \to
    \ 
    0
    \qquad
    \text{for}
    \ 
    N
    \to
    \infty
    \,,
  \end{gather*}
  it holds 
  $
  \norm{\G_N}^*_{\mathcal{H}_N}\overset{\P}{\to}0
  $.
\end{lemma}
\begin{proof}
  By Markov's inequality and Theorem~\ref{th:max_ineq} it holds for all $\varepsilon>0$
  \begin{align*}
    \P
    [
  \norm{\G_N}^*_{\mathcal{H}_N}
  \ge
  \varepsilon
    ]
    &
    \ 
    \le
    \ 
    \varepsilon^{-1}
    \E
    [
  \norm{\G_N}^*_{\mathcal{H}_N}
    ]
    \ 
    =
    \ 
    \varepsilon^{-1}
    \E^*
    [
  \norm{\G_N}_{\mathcal{H}_N}
    ]
    \\
    &
    \ 
    \lesssim
    \ 
    \varepsilon^{-1}
    J_{[\, ]}
    \left( 
    \norm{H_N}_{ L^2(\P)}
    ,
    \mathcal{H}_N
    ,
     L^2(\P)
    \right)
    \\
    &
    \ 
    \to
    \ 
    0
    \qquad
    \text{for}
    \ 
    N\to\infty
    \,.
  \end{align*}
\end{proof}

\begin{lemma}
  \label{aa:r3:lemma:1}
  Let
$(\varepsilon_N)\subset(0,1]$
be 
a decreasing sequence
with $\varepsilon_N\to 0$ for $N\to\infty$ and
$(\mathcal{F}_N)$
a sequence of (measurable) function classes
with envelope functions
$(F_N)$,
satisfying 
for some $k<2$
\begin{gather*}
\norm{F_N}_{L^2(\P)}
\ 
\le
\ 
\varepsilon_N
\quad
\text{and}
\quad
  \log
  N_{[\,]}(\varepsilon,\mathcal{F}_N,\mathrm{L}_2(\P_{X}))
  \ 
  \lesssim
  \ 
  \left( 
  \frac{1}{\varepsilon}
  \right)^k
  \quad
  \text{for all}
  \ 
  N\in\mathbb{N}
  \,.
\end{gather*}
Then
\begin{gather*}
  J_{[\,]}(
\norm{F_N}_{L^2(\P)}
,\mathcal{F}_N\cdot\mathcal{F},\mathrm{L}_2(\P))
  \to 0
  \quad
  \text{and}
  \quad
  \norm{\G_N}^*_{\mathcal{F}_N\cdot\mathcal{F}}\overset{\P}{\to}0
  \qquad
  \text{for}\ 
  N\to\infty
  \,,
\end{gather*}
where $\mathcal{F}$ is defined in \eqref{F_g}.
\end{lemma}
\begin{proof}
  By assumption
  and Lemma~\ref{aa:mean:l:br} it holds
for some $k<2$
\begin{gather*}
\norm{F_N}_{L^2(\P)}
\ 
\le
\ 
\varepsilon_N
\quad
\text{and}
\quad
  \log
  N_{[\,]}(\varepsilon,\mathcal{F}_N,\mathrm{L}_2(\P))
  \ 
  \lesssim
  \ 
  \left( 
  \frac{1}{\varepsilon}
  \right)^k
  \quad
  \text{for all}
  \ 
  N\in\mathbb{N}
  \,,
\end{gather*}
and
  \begin{gather*}
    N_{[\,]}
    (
    \varepsilon
    ,
    \mathcal{F}, L^2(\P))
    \ 
    \lesssim
    \ 
    \left( 
      \frac{1}{\varepsilon}
    \right)^2
    \qquad
    \text{for all}
    \ 
    \varepsilon>0
    \,.
  \end{gather*}
  Since $\mathcal{F}_N$ and $\mathcal{F}$ have envelope functions smaller 1, we can apply Lemma~\ref{lem_prod_br} to get
  \begin{gather*}
  \log
  N_{[\,]}(\varepsilon,\mathcal{F}_N\cdot\mathcal{F},\mathrm{L}_2(\P))
  \ 
  \lesssim
  \ 
  \left( 
  \frac{1}{\varepsilon}
  \right)^k
  +
  \log
  (1/\varepsilon)
  \ 
  \lesssim
  \ 
  \left( 
  \frac{1}{\varepsilon}
  \right)^k
  \quad
  \text{for all}\ 
  \varepsilon>0
  \,.
  \end{gather*}
  Since 
  $k/2\in(0,1)$
  it holds
\begin{align*}
  J_{[\,]}(
\norm{F_N}_{L^2(\P)}
,\mathcal{F}_N\cdot\mathcal{F},\mathrm{L}_2(\P))
  &
  \ 
=
  \ 
\int_0^{
\norm{F_N}_{L^2(\P)}
}
\sqrt{
  \log
  N_{[\,]}(\varepsilon,\mathcal{F}_N\cdot\mathcal{F},\mathrm{L}_2(\P))
}
\,d\varepsilon
\\
&
\ 
\lesssim
\ 
\int_0^{
  \varepsilon_N
}
  \left( 
  \frac{1}{\varepsilon}
\right)^{k/2}
\,d\varepsilon
\\
&
\ 
=
\ 
\frac{
\varepsilon_N^{1-k/2}
}{1-k/2}
\ 
\to 0
\ 
\qquad
\text{for}
\ 
N\to\infty
\,.
\end{align*}
The second statement follows from Lemma~\ref{markov_max_lemma}
for 
$\mathcal{H}_N:=\mathcal{F}_N\cdot\mathcal{F}$ and
$H_N:=F_N$.
\end{proof}





\subsection{Donsker's Theorem}
  There is a powerful theorem --- a central limit theorem for $\G_N$ uniform in $\mathcal{F}$ --- that we now introduce.
\begin{definition}
  We call a class 
  $\mathcal{F}$ of measurable functions 
$\P$-Donsker
if the sequence of processes 
$\left\{ \G_N f \colon f\in\mathcal{F}\right\}$
converges in
$l^\infty(\mathcal{F})$
to a tight limit process.
\end{definition}

\begin{theorem}
  \label{th:donsker}
  Every class $\mathcal{F}$ of measurable functions 
  with
  \begin{gather*}
    J
    _{[\,]}
    (
    1
    ,
    \mathcal{F}
    ,
    L_2(\P)
    )
    <\infty
  \end{gather*}
  is
  $\P$-Donsker.
  Furthermore,
  the sequence of processes 
$\left\{ \G_N f \colon f\in\mathcal{F}\right\}$
  converges 
  in
$l^\infty(\mathcal{F})$
to a Gaussian process with mean 0 and covariance function given by
\begin{gather*}
  \mathbf{Cov}(f,g)
  \ 
  :=
  \ 
  \E[fg]
  \ 
  -
  \ 
  \E[f]\E[g]
  \,.
\end{gather*}
\end{theorem}
\begin{proof}
  \cite[Theorem~19.5]{Vaart2000}
\end{proof}
\begin{lemma}
  \label{lem:G_P_donsker}
  Let Assumption~\eqref{asu:treatment_asign_str_ing} and Assumption~\ref{asu:x_finite} hold true.
Then the function class $\mathcal{G}$ defined in \eqref{F_g} is $\P$-Donsker. 
\end{lemma}
\begin{proof}
  By Theorem~\ref{th:donsker} it suffices to show that the bracketing integral is finite.
Note that by Assumption~\ref{asu:x_finite} ($\mathcal{X}$ is finite) and Assumption~\eqref{asu:treatment_asign_str_ing} ($\pi(X)>0$) it holds $1/\pi(X)\in L^2(\P)$.
  Thus,
  by Lemma~\ref{aa:mean:l:br}
  it holds
  \begin{align*}
    &
  \log
  N_{[\,]}
    (
    \varepsilon
    ,
    \mathcal{G}, L^2(\P))
    \\
    &
    \ 
    \lesssim
    \ 
    \log
    \left(
      \frac
      {
      1+
    \norm{1/\pi(X)}_{ L^2(\P)}
      }
      {\varepsilon}
    \right)
    \ 
    \lesssim
    \ 
      \frac
      {
      1+
    \norm{1/\pi(X)}_{ L^2(\P)}
      }
      {\varepsilon}
    \qquad
    \text{for all}
    \ 
    \varepsilon\in (0,1)
    \,.
  \end{align*}
  Thus
  \begin{gather*}
    J_{[\,]}(1,\mathcal{G},L^2(\P))
    \ 
    \lesssim
    \ 
    \int_0^1
    \sqrt
    {
      \frac
      {
      1+
    \norm{1/\pi(X)}_{ L^2(\P)}
      }
      {\varepsilon}
    }
    \,
    d\varepsilon
    \ 
    \lesssim
    \ 
      1+
    \norm{1/\pi(X)}_{ L^2(\P)}
    \ 
    <
    \ 
    \infty
    \,.
  \end{gather*}
But then $\mathcal{G}$ is $\P$-Donsker.

\end{proof}

\subsection{Propensity Score Weights}
  The next lemma shows what effect the 
\textbf{propensity score weights}
$T/\pi(X)$ have on other functions.
\begin{lemma}
  \label{lem:ps_weights}
  \label{ps_weights_lemma}
  Let
  $
  g_1\colon
  \mathcal{X}\to\R
  $
  and
  $
  g_2\colon
  \mathcal{Y}\to\R
  $
  be a measurable functions.
  \begin{enumerate}[label=(\roman*)]
    \item
  It holds
  \begin{gather*}
    \E
    \left[
    \frac{T}{\pi(X)}
    g_1(X)
    \right]
    \ 
    =
    \ 
    \E
    \left[
    g_1(X)
    \right]
    \,.
  \end{gather*}
  \item
 If Assumption~\ref{aa:assumption:treatment_str_ign} holds true, then
  \begin{gather*}
    \E
    \left[
    \frac{T}{\pi(X)}
    g_2(Y(T))
    \right]
    \ 
    =
    \ 
    \E
    \left[
    f(Y(1))
    \right]
    \,.
  \end{gather*}
  \end{enumerate}
 \end{lemma}




Now we are ready to state the main result.
\newpage
\section{Main Result}
  Before we state the main result we collect all assumptions. Note that in the proofs we refer to the assumptions by their initial location, for example, Assumption~\ref{main_asu}.\textit{(iv)} is Assumption~\ref{asu:feas_dual_sol}.
\begin{assumption}
  \label{main_asu}
  \begin{enumerate}[label=(\roman*)]
 \item
$\sqrt{N}\norm{\delta}_1\overset{\P}{\to}0$ for $N\to\infty$.
\item
  \begin{align*}
    \sqrt{N}
    \sup_{z\in\R}
    \omega
    \left( 
      F_{Y(1)}(z|\cdot)
      ,h_N^d
    \right)
    \ 
    \to 
    \ 
    0
    \qquad
    \text{for}\ 
    N\to\infty
    \,.
  \end{align*}
  \item
\begin{gather*}
  (Y(0),Y(1))\ \perp \ T \,|\,X
  \quad
  \text{and}
  \quad
  0<\pi(X)<1
  \,,
\end{gather*}
\item
  For all $N\in\mathbb{N}$ there exists a non-empty, compact, and deterministic 
  parameter space 
  $
  \Theta_N
  \subset
  \R^{N}_{\ge 0}
  \times
  \R
  \times
  \R^N
  $
  such that the optimal solution 
  $
  \left( \rho^\dagger,\lambda_0^\dagger,\lambda^\dagger \right)
  $
  of Problem~\ref{dual}
  are contained in $\Theta_N$.
     \item 
       $\# J_N\le \# \mathcal{X}<\infty$ for all $N\in\mathbb{N}$, where 
       $
   J_N
   :=
   \left\{ j\in\mathbb{N}\colon
     \P[X\in A_{n,j}]>0
   \right\}
       $
\item
  For all $N\in\mathbb{N}$ there exist $(M_{N,j})_{j\in J_N}$ such that $\infty>M_{N,j}\ge 0$ for all $j\in J_n$, and 
  $\frac{1}{\pi(\cdot)}\in C^\alpha_{M_{N,j}}(\mathrm{cl}\,A_{N,j})$ for all $(j,N)\in J_N\times \mathbb{N}$, with $\alpha>d/2$.
 \end{enumerate} 
\end{assumption}
\begin{ftheorem}
  \label{th:main}
  Let Assumption~\ref{main_asu} hold true.
  Then the stochastic process
\begin{gather}
    \sqrt{N}
    \left( 
  \frac{1}{N}
    \sum_{i=1}^{N} 
    T_i
    \cdot
    w_0^\dagger(X_i)
    \cdot
    \mathbf{1}
    \left\{ Y_i\,\le\, z \right\}
    \ 
    -
    \ 
    F_{Y(1)}(z)
    \right)
    _{z\in\R}
    \,
  \end{gather}
  converges in
  $l^\infty(\R)$
  to a Gaussian process with mean 0 and covariance function
  satisfying for all $z_1,z_2\in\R$
\begin{align}
  \label{cov:lp}
 \begin{split}
  &
  \mathbf{Cov}
  (z_1,z_2)
  \\
  &
  =\ 
  \E
  \left[ 
 \frac{
 F_{Y(1)}(z_1 \land z_2\,|\,X)
}{\pi(X)}
\ 
-
\ 
 \frac{1-\pi(X)}{\pi(X)}
 F_{Y(1)}(z_1|X)
 \cdot
 F_{Y(1)}(z_2|X)
  \right]
  \\
  &
  \qquad 
 -
 \ 
 F_{Y(1)}(z_1)
 \cdot
 F_{Y(1)}(z_2)
 \,.
 \end{split}
\end{align}
\end{ftheorem}



In the introduction to this section we talked about proof strategies. The next section gives an error decomposition that is central to proof.
It consists of four terms that we shall bound consecutively. 
\section{Error Decomposition}
  \begin{lemma}
  \label{aa:mean:lemma_decomp}
  It holds
  \begin{gather}
    \sqrt{N}
    \left( 
  \frac{1}{N}
    \sum_{i=1}^{N} 
    T_i
    \cdot
    w_0^\dagger(X_i)
    \cdot
    \mathbf{1}
    \left\{ Y_i\,\le\, z \right\}
    \ 
    -
    \ 
    F_{Y(1)}(z)
    \right)
    _{z\in\R}
    \ 
    =
    \ 
    R_1
    \ 
    +
    \ 
    R_2
    \ 
    +
    \ 
    R_3
    \ 
    +
    \ 
    R_4
  \end{gather}
  with
\begin{align*}
  R_1
  &
  \ 
  :=
  \ 
  \sqrt{N}
  \sum_{k=1}^{N} 
  \left[ 
  \frac{1}{N}
  \left( 
    \sum_{i=1}^{N} 
    T_i
    \cdot
    w_0^\dagger(X_i)
    \cdot
    B_k(X_i)
    \ 
    -
    \ 
    \sum_{i=1}^{N} 
    B_k(X_i)
  \right)
  \cdot
  F_{Y(1)}(z|X_k)
  \right]
  _{z\in\R}
  \,,
  %%%% 1 %%%%
  \\
  R_2
  &
  \
  :=
  \ 
  \sqrt{N}
    \sum_{i=1}^{N} 
    \frac{1}{N}
    \left[ 
      \left( 
    T_i\cdot w_0^\dagger(X_i) 
    \ 
    -
    \ 
    1 
      \right)
    \left( 
  F_{Y(1)}(z|X_i)
    \ 
    -
    \ 
    \sum_{k=1}^{N} 
    B_k(X_i)
    \cdot
  F_{Y(1)}(z|X_k)
    \right)
    \right]
  _{z\in\R}
  \,,
  %%%  %%%%%%%%%%m
  \\
  R_3
  &
  \
  :=
  \ 
  \sqrt{N}
  \left( 
  \frac{1}{N}
    \sum_{i=1}^{N} 
    \left[ 
    T_i
    \cdot
    \left( 
    w^\dagger_0(X_i) 
    \ 
    -
    \ 
    \frac{1}{\pi(X_i)}
    \right)
    \cdot
    \left( 
    \mathbf{1}{\left\{ Y_i \le z \right\}}
    \ 
    -
    \ 
  F_{Y(1)}(z|X_i)
    \right)
    \,
    \right]
  \right)
  _{z\in\R}
  \,,
  %%%% 3 %%%%
  \\
  R_4
  &
  \
  :=
  \ 
  \sqrt{N}
  \left( 
  \frac{1}{N}
    \sum_{i=1}^{N} 
    \frac{T_i}{\pi(X_i)}
    \left( 
    \mathbf{1}{\left\{ Y_i \le z \right\}}
    -
  F_{Y(1)}(z|X_i)
    \right)
    \ 
    +
    \ 
    \left( 
  F_{Y(1)}(z|X_i)
    -
  F_{Y(1)}(z)
    \right)
  \right)
  _{z\in\R}
  \,.
  \end{align*}
\end{lemma}
\begin{proof}
  We fix $z\in\R$.
  It holds
  \begin{align*}
    &
    \frac{1}{N}
    \sum_{i=1}^{N} 
    w_0^\dagger(X_i)
    \cdot
    T_i
    \cdot
    \mathbf{1}{\left\{ Y_i\, \le\, z \right\}}
    \\
    &
    \ 
    =
    \ 
    \frac{1}{N}
    \sum_{i=1}^{N} 
    \left( 
    w_0^\dagger(X_i)
    \ 
    -
    \ 
    \frac{1}{\pi(X_i)}
    \right)
    T_i
    \cdot
    \mathbf{1}{\left\{ Y_i\, \le\, z \right\}}
    \\
    &
    \quad 
    +
    \ 
    \frac{1}{N}
    \sum_{i=1}^{N} 
    \frac{T_i}{\pi(X_i)}
    \mathbf{1}{\left\{ Y_i\, \le\, z \right\}}
    \\
    &
    \ 
    =
    \ 
    \frac{1}{N}
    \sum_{i=1}^{N} 
    \left( 
    w_0^\dagger(X_i)
    -
    \frac{1}{\pi(X_i)}
    \right)
    T_i
    \left( 
    \mathbf{1}{\left\{ Y_i\, \le\, z \right\}}
    -
    F_{Y(1)}(z|X_i)
    \right)
    \\
    &
    \quad 
    +
    \ 
    \frac{1}{N}
    \sum_{i=1}^{N} 
    \frac{T_i}{\pi(X_i)}
    \left( 
    \mathbf{1}{\left\{ Y_i\, \le\, z \right\}}
    -
    F_{Y(1)}(z|X_i)
    \right)
    \\
    &
    \qquad 
    +
    \ 
    \frac{1}{N}
    \sum_{i=1}^{N} 
    w_0^\dagger(X_i)\cdot T_i\cdot
    F_{Y(1)}(z|X_i)
    \\
    &
    \ 
    =
    \ 
    R_3(z)
    /\sqrt{N}
    \\
    &
    \quad 
    +
    \ 
    \frac{1}{N}
    \sum_{i=1}^{N} 
    \frac{T_i}{\pi(X_i)}
    \left( 
    \mathbf{1}{\left\{ Y_i\, \le\, z \right\}}
    -
    F_{Y(1)}(z|X_i)
    \right)
    +
    \left( 
    F_{Y(1)}(z|X_i)
    -
    F_{Y(1)}(z)
    \right)
    \\
    &
    \qquad
    +
    \ 
    \frac{1}{N}
    \sum_{i=1}^{N} 
    \left( 
    w_0^\dagger(X_i)\cdot T_i
    \ 
    -
    \ 
    1
    \right)
    F_{Y(1)}(z|X_i)
    \\
    &
    \quad\qquad
    +
    \ 
    F_{Y(1)}(z)
    \\
    &
    \ 
    =
    \ 
    R_3(z)
    /\sqrt{N}
    \\
    &
    \quad
    +
    \ 
    R_4(z)/\sqrt{N}
    \\
    &
    \qquad
    +
    \ 
    \frac{1}{N}
    \sum_{i=1}^{N} 
    \left( 
    w_0^\dagger(X_i)\cdot T_i
    \ 
    -
    \ 
    1
    \right)
    \left( 
    F_{Y(1)}(z|X_i)
    -
    \sum_{k=1}^{N} 
    B_k(X_i)
    \cdot
  F_{Y(1)}(z|X_k)
    \right)
    \\
    &
    \quad\qquad
    +
    \ 
    \frac{1}{N}
    \sum_{i=1}^{N} 
    \left( 
    w_0^\dagger(X_i)\cdot T_i
    \ 
    -
    \ 
    1
    \right)
    \sum_{k=1}^{N} 
    B_k(X_i)
    \cdot
  F_{Y(1)}(z|X_k)
    \\
    &
    \qquad\qquad
    +
    \ 
    F_{Y(1)}(z)
\\
    &
    \ 
    =
    \ 
    (
    R_3(z)
    \ 
    +
    R_4(z)
    )
    /\sqrt{N}
    \\
    &
    \qquad
    +
    \ 
    R_2(z)/\sqrt{N}
    \\
    &
    \quad\qquad
    +
    \ 
    \sum_{k=1}^{N} 
    \frac{1}{N}
    \sum_{i=1}^{N} 
    \left( 
    w_0^\dagger(X_i)\cdot T_i
    B_k(X_i)
    \ 
    -
    \ 
    B_k(X_i)
    \right)
    \cdot
  F_{Y(1)}(z|X_k)
    \\
    &
    \qquad\qquad
    +
    \ 
    F_{Y(1)}(z)
    \\
    &
    \ 
    =
    \ 
    \left( 
R_3(z)
    \ 
    +
    \ 
    R_4(z)
    \ 
    +
    \ 
    R_2(z)
    \ 
    +
    \ 
    R_1(z)
    \right)
    /\sqrt{N}
    \ 
    +
    \ 
    F_{Y(1)}(z)
    \,.
  \end{align*}
\end{proof}


\section{Analysis of the Error Terms}
  \subsection{Analysis of $R_1$}
    \begin{lemma}
  \label{aa:mean:l:r1}
  Let
$\sqrt{N}\norm{\delta}_1\overset{\P}{\to}0$.
Then it holds
$\sup_{z\in\R}|R_1(z)|\overset{\P}{\to}0$.
  \end{lemma}
\begin{proof}
If $s_N\in \mathrm{int}\,\Theta_N$, then 
$
w_1^\dagger(X_1),\cdots,w_N^\dagger(X_N)
$
satisfy the box constraints of Problem~\ref{bw:1:primal} (in the form with the $T_i$ instead of $n$).
  \begin{align}
    \label{R_1:1}
    \begin{split}
    \sup_{z\in\R}
    \left| 
    R_1(z)
    \right|
    &
    \ 
    =
    \ 
  \sqrt{N}
  \sup_{z\in\R}
  \sum_{k=1}^{N} 
  \left[ 
  \frac{1}{N}
  \left( 
    \sum_{i=1}^{N} 
    T_i
    \cdot
    w^\dagger(X_i)
    \cdot
    B_k(X_i)
    \ 
    -
    \ 
    \sum_{i=1}^{N} 
    B_k(X_i)
  \right)
  \cdot
  F_{Y(1)}(z|X_k)
  \right]
    \\
    &
    \ 
    \le
    \ 
  \sqrt{N}
  \sum_{k=1}^{N} 
  \left| 
  \frac{1}{N}
  \left( 
    \sum_{i=1}^{N} 
    T_i
    \cdot
    w^\dagger(X_i)
    \cdot
    B_k(X_i)
    \ 
    -
    \ 
    \sum_{i=1}^{N} 
    B_k(X_i)
  \right)
  \right|
  \cdot
    \sup_{z\in\R}
  F_{Y(1)}(z|X_k)
  \\
    &
    \ 
    \le
    \ 
  \sqrt{N}
  \norm{\delta}_1
    \end{split}
  \end{align}
  The last inequality is due to $F_{Y(1)}\in[0,1]$ and the assumption that $(w_i^\dagger(X_i))$ satisfies the box constraints of Problem~\ref{bw:1:primal}.
  It remains to analyse the case $s_N\notin \mathrm{int}\,\Theta_N$.
  Then $(w_i^\dagger(X_i))$ are not the optimal weights of Problem~\ref{bw:1:primal}
  and we can't employ the box constraints to bound $R_1$.
  But we can use 
  $
  \P \left[ s_N\notin \Theta_N \right]
  \to 0
  $.
  To this end, note 
  \begin{align*}
    \P
    \left[ 
      \frac{
  \P \left[ s_N\notin \Theta_N \right]
      }{\sqrt{N}}
      \sum_{i=1}^{N} 
      \left( 
      \frac{T_i}{\pi(X_i)}
      -
      1
      \right)
      \ge \varepsilon
    \right]
    \le
  \P \left[ s_N\notin \Theta_N \right]
  \frac{\mathbf{Var}[T/\pi(X)]}{\varepsilon^2}
  \to 
  0
  \end{align*}
  for all $\varepsilon>0$.


  Since we assume 
$\sqrt{N}\norm{\delta}_1\overset{\P}{\to}0$
it holds
$\sup_{z\in\R}|R_1(z)|\overset{\P}{\to}0$.
\end{proof}
\begin{remark}
  We want to comment on the box constraints of Problem~\ref{bw:1:primal}, that is,
 \begin{gather*}
      \left| 
      \frac{1}{N} 
      \left( 
      \sum_{i = 1}^{n} 
      w^\dagger(X_i)
      B_k(X_i)
      -
      \sum_{i=1}^{N} 
      B_k(X_i)
      \right)
    \right|
    \ 
    \le 
    \ 
    \delta_k
    \qquad
    \text{for all}\ 
    k \in \left\{ 1, \ldots, N \right\}
    \,.
  \end{gather*}
  Note, that the first sum goes over $\left\{ 1,\ldots,n \right\}$ while the second sum goes over $\left\{ 1,\ldots,N \right\}$.
  A second, equivalent version of the constraints is
  \begin{gather*}
      \left| 
      \frac{1}{N} 
      \left( 
      \sum_{i = 1}^{N} 
      T_i
      w^\dagger(X_i)
      B_k(X_i)
      -
      \sum_{i=1}^{N} 
      B_k(X_i)
      \right)
    \right|
    \ 
    \le 
    \ 
    \delta_k
    \qquad
    \text{for all}\ 
    k \in \left\{ 1, \ldots, N \right\}
    \,.
  \end{gather*}
  Now both sums go over $\left\{ 1,\ldots,N \right\}$ and the
  indicator of treatment $T_i$ takes care that in the first sum only the terms with $i\le n$ are effective. 
  Having this flexibility with the versions helps. I regard the first version as suitable for non-probabilistic computations, although $n$ is of course a random variable. On the other hand, the second version is more honest, exactly telling the dependence on the indicator of treatment. This version is useful in probabilistic computations. 

  Also we want to comment on the assumption on $\norm{\delta}$.
  Playing around with norm equivalences we discover that 
  $\sqrt{N}\norm{\delta}_1\overset{\P}{\to}0$ for $N\to \infty$ is the weakest
  (natural) assumption to
  control $R_1$.
  Indeed, other ways to continue the second row in \eqref{R_1:1} are
  \begin{gather*}
    (\,\cdots)
    \ 
  \le
    \ 
  \sqrt{N}
  \norm{\delta}_2
  \left( 
  \sum_{k=1}^{N} 
  \left( 
    \sup_{z\in\R}
  F_{Y(1)}(z|X_k)
  \right)^2
\right)^{1/2}
\ 
\le
\ 
N
  \norm{\delta}_2\,,
  \end{gather*}
  by the Cauchy-Schwarz inequality and
  $
  F_{Y(1)}\in [0,1]
  $,
or
\begin{gather*}
  (\,\cdots)
  \ 
  \le
  \ 
  \sqrt{N}
  \norm{\delta}_\infty
  \sum_{k=1}^{N} 
    \sup_{z\in\R}
  F_{Y(1)}(z|X_k)
  \ 
  \le
  \ 
  N^{3/2}
  \norm{\delta}_\infty
  \,.
\end{gather*}
Since $\delta\in \R^N$, however, it holds
\begin{gather*}
  \sqrt{N}\norm{\delta}_1
  \ 
  \le
  \ 
  N\norm{\delta}_2
  \ 
  \le
  \ 
  N^{3/2}\norm{\delta}_\infty
  \,.
\end{gather*}
With hindsight, the assumption 
$\sqrt{N}\norm{\delta}_1\overset{\P}{\to}0$ for $N\to \infty$ 
  also 
  suffices 
  to control the second (or first) occurrence of a term, that we control by assumptions on $\norm{\delta}$.
This is the \textbf{second term} of \eqref{c:1}, where we estimate
\begin{gather*}
  \inner{\delta}{\left| \Delta \right|}
  \ 
  =
  \ 
  \sum_{k=1}^{N} 
  \delta_k
  \left| \Delta_k \right|
  \ 
  \le
  \ 
  \norm{\delta}_1
  \norm{\Delta}_\infty
  \ 
  \le
  \ 
  \norm{\delta}_1
  \norm{\Delta}_2
  \ 
  \le
  \ 
  \norm{\delta}_1
  \varepsilon
  \ 
  \overset{\P}{\to}
  \ 
  0
  \quad
  \text{for}\ 
  N\to \infty
  \,.
\end{gather*}

\end{remark}


  \subsection{Analysis of $R_2$}
    The convergence of this term is closely related to good approximation properties of $B$ (see Lemme~\ref{lem:basis_2}.\textit{(ii)}). 
\begin{lemma}
\label{aa:mean:l:r2}
  Assume
  \begin{align*}
    \sqrt{N}
    \sup_{z\in\R}
    \omega
    \left( 
      F_{Y(1)}(z|\cdot)
      ,h_N^d
    \right)
    \ 
    \to 
    \ 
    0
    \qquad
    \text{for}\ 
    N\to\infty
    \,.
  \end{align*}
  Then $\sup_{z\in\R}|R_2(z)|\overset{\P}{\to} 0$.
\end{lemma}
\begin{proof}
  \begin{align*}
    \sup_{z\in\R}
    \left| R_2(z) \right|
    &
    \
    =
    \
  \sqrt{N}
  \sup_{z\in\R}
  \left|
    \sum_{i=1}^{N} 
    \frac{1}{N}
    \left[ 
      \left( 
    T_i\cdot w_0^\dagger(X_i) 
    \ 
    -
    \ 
    1 
      \right)
    \left( 
  F_{Y(1)}(z|X_i)
    \ 
    -
    \ 
    \sum_{k=1}^{N} 
    B_k(X_i)
    \cdot
  F_{Y(1)}(z|X_k)
    \right)
    \right]
  \right|
\\
    &
    \  
    \le
    \  
        \sqrt{N}
        \sup_{z\in\R}
        \max_{i\in \left\{ 1,\ldots,N \right\}}
        \sum_{k=1}^{N}
            \left|
        B_k(X_i,X_1,\ldots,X_N)
        \cdot
        F_{Y(1)}(z|X_k)
            \ 
            -
            \ 
        F_{Y(1)}(z|X_i)
            \right|
            \\
            &
            \qquad
            \cdot
            \ 
    \frac{1}{N}
    \sum_{i=1}^{N} 
      \left| 
    T_i\cdot w^\dagger_0(X_i) 
    \ 
    -
    \ 
    1 
      \right|
  \end{align*}
  Note that by Theorem~\ref{th:weights_constr}.\textit{(i)-(ii)}
  it holds
  \begin{align*}
    \frac{1}{N}
    \sum_{i=1}^{N} 
      \left| 
    T_i\cdot w^\dagger_0(X_i) 
    \ 
    -
    \ 
    1 
      \right|
      \ 
    \le
      \ 
    1
    \ 
    +
    \ 
    \frac{1}{N}
    \sum_{i=1}^{N} 
    T_i\cdot w^\dagger_0(X_i) 
    \ 
    =
    \ 
    2
    \,.
  \end{align*}
  The statement follows from Lemma~\ref{lem:basis_2}.\textit{(ii)}
\end{proof}
\begin{remark}
In the original paper \cite{Wang2019} the authors derive concrete learning rates for the weights and employ them in bounding this term. They obtain a multiplied learning rate that is sufficiently fast. Their approach, however, calls for concrete learning rates of the weights. Arguably, the process of deriving such rates is the most complicated part of the paper. 
I found out that with the basis functions of partitioning estimates (or similar basis functions) we don't need concrete rates for the weights. 
Consistency of the weights is enough and gives us an (arbitrarily slow but sufficient) learning rate to establish the results.
We don't even need rates for the weights to control $R_2$.
They only play a role in bounding $R_3$. 
\end{remark}



  \subsection{Analysis of $R_3$}
    \begin{lemma}
  It holds
  \begin{align*}
    \E
    \left[ 
      T
    \left( 
    \mathbf{1}{\left\{ Y(T) \le z \right\}}
    \ 
    -
    \ 
  F_{Y(1)}(z|X)
    \right)
    \right]
    \ 
    =
    \ 
    0
    \qquad
    \text{for all}\ 
    z\in\R
    \,.
  \end{align*}
\end{lemma}
\begin{proof}
  Let $z\in\R$.
  By Lemma~\ref{lem:ps_weights}.\textit{(i)} 
  and the properties of conditional expectation
  it holds
  \begin{align*}
    \E
    \left[ 
      T
      \cdot
    \mathbf{1}{\left\{ Y(T) \le z \right\}}
    \right]
    \ 
    =
    \ 
    \E
    \left[ 
      \pi(X)
      \cdot
    \mathbf{1}{\left\{ Y(1) \le z \right\}}
    \right]
    \ 
    =
    \ 
    \E
    \left[ 
      \pi(X)
      \cdot
      F_{Y(1)}(z|X)
    \right]
    \,.
  \end{align*}
  Furthermore,
  by Lemma~\ref{lem:ps_weights}.\textit{(ii)} 
  it holds
  \begin{align*}
    \E
    \left[ 
      T
      \cdot
  F_{Y(1)}(z|X)
    \right]
    \ 
    =
    \ 
    \E
    \left[ 
      \pi(X)
      \cdot
  F_{Y(1)}(z|X)
    \right]
    \,.
  \end{align*}
  It follows the result.
\end{proof}

  \subsection{Analysis of $R_4$}
    \begin{lemma}
  \label{aa:mean:l:r4}
  Let
  Assumption~\eqref{asu:treatment_asign_str_ing} and Assumption~\ref{asu:x_finite} hold true.
  Then
  $R_4$ 
  converges in
  $l^\infty(\R)$
  to a Gaussian process with mean 0 and covariance
\begin{align*}
  &
  \mathbf{Cov}
  (z_1,z_2)
  \\
  &
  =\ 
  \E
  \left[ 
 \frac{
 F_{Y(1)}(z_1 \land z_2\,|\,X)
}{\pi(X)}
\ 
-
\ 
 \frac{1-\pi(X)}{\pi(X)}
 F_{Y(1)}(z_1|X)
 \cdot
 F_{Y(1)}(z_2|X)
  \right]
  \ 
 -
 \ 
 F_{Y(1)}(z_1)
 \cdot
 F_{Y(1)}(z_2)
\end{align*}

\end{lemma}
\begin{proof}
  By Lemma~\ref{lem:f_z} it holds
  \begin{align*}
    \E
    \left[
      \frac{f_z(T,X,Y(T))}{\pi(X)}
      +
      F_{Y(1)}(z|X)
      -
      F_{Y(1)}(z)
      \right]
      \ 
      =
      \ 
      \E
      \left[
      \frac{1}{\pi(X)}
      \E
      \left[
        f_z(T,X,Y(T))
        |X
      \right]
      \right]
      \ 
      =
      \ 
      0
      \,.
  \end{align*}
  Thus
  \begin{align*}
    R_4(z)
    &
  \
  =
  \ 
  \frac{1}{
  \sqrt{N}
  }
    \sum_{i=1}^{N} 
    \frac{T_i}{\pi(X_i)}
    \left( 
    \mathbf{1}{\left\{ Y_i \le z \right\}}
    -
  F_{Y(1)}(z|X_i)
    \right)
    \ 
    +
    \ 
    \left( 
  F_{Y(1)}(z|X_i)
    -
  F_{Y(1)}(z)
    \right)
    \\
    &
    \ 
  =
    \ 
  \frac{1}{
  \sqrt{N}
  }
    \sum_{i=1}^{N} 
      \frac{f_z(T_i,X_i,Y_i)}{\pi(X_i)}
      +
      \left( 
      F_{Y(1)}(z|X_i)
      -
      F_{Y(1)}(z)
      \right)
      \\
      &
      \ 
      =
      \ 
      \G_N 
      \left(
       \frac{f_z}{\pi(\cdot)}
      +
      F_{Y(1)}(z|\cdot)
      -
      F_{Y(1)}(z)
      \right)
      \,.
  \end{align*}
  By Assumption~\eqref{asu:treatment_asign_str_ing}, Assumption~\ref{asu:x_finite},
and
Lemma~\ref{lem:G_P_donsker}
the function class 
\begin{align*}
  \mathcal{G}
    &
    \ 
    :=
    \ 
    \left\{ 
      \frac{f_z}{\pi(\cdot)}
      +
      F_{Y(1)}(z|\cdot)
      -
      F_{Y(1)}(z)
      \ 
      \colon
      \ 
      z\in\R\ 
    \right\}
\end{align*}
is $\P$-Donsker
Thus, by Theorem~\ref{th:donsker},
the process $R_4$ converges in $l^\infty(\R)$ to a Gaussian process with mean 0.
It remains to calculate the covariance of the limiting process.
We write
\begin{align*}
  &
  \E
  \left[
  \left( 
  f^{z_1}_{1/\pi}
  +
  F_{Y(1)}(z_1|X)
  -
F_{Y(1)}(z_1)
  \right)
  \left( 
  f^{z_2}_{1/\pi}
  +
  F_{Y(1)}(z_2|X)
  -
F_{Y(1)}(z_2)
  \right)
  \right]
  \\
  &
  \ 
  =
  \ 
\E
\left[
  f^{z_1}_{1/\pi}
  \cdot
  f^{z_2}_{1/\pi}
\right]
\\
  &
  \quad
  +
  \ 
  \E
  \left[
  f^{z_1}_{1/\pi}
  \left( 
  F_{Y(1)}(z_2|X)
  -
F_{Y(1)}(z_2)
  \right)
  \right]
  \ 
  +
  \ 
  \E
  \left[
  f^{z_2}_{1/\pi}
  \left( 
  F_{Y(1)}(z_1|X)
  -
F_{Y(1)}(z_1)
  \right)
  \right]
  \\
  &
  \quad
  +
  \ 
  \E
  \left[
  \left( 
  F_{Y(1)}(z_1|X)
  -
F_{Y(1)}(z_1)
  \right)
  \left( 
  F_{Y(1)}(z_2|X)
  -
F_{Y(1)}(z_2)
  \right)
  \right]
  \\
  &
  \ 
  =:
  \ 
  C_0
  \quad 
  +
  \quad 
  C_1
  +
  C_2
  \quad 
  +
  \quad 
  C_3
  \,.
\end{align*}
It holds
by Assumption~\eqref{asu:treatment_asign_str_ing} and Lemma~\ref{lem:ps_weights}
\begin{align*}
  C_0 
  &
  \ 
  =
  \ 
\E
\left[
  f^{z_1}_{1/\pi}
  \cdot
  f^{z_2}_{1/\pi}
\right]
\\
&
\ 
=
\ 
\E
\left[
\frac{1}{\pi(X)}
\frac{T}{\pi(X)}
\left( 
\mathbf{1}{\left\{ Y(T)\,\le\, z_1 \right\}}
-
F_{Y(1)}(z_1|X)
\right)
\left( 
\mathbf{1}{\left\{ Y(T)\,\le\, z_2 \right\}}
-
F_{Y(1)}(z_2|X)
\right)
\right]
\\
&
\ 
=
\ 
\E
\left[
\frac{1}{\pi(X)}
\left( 
\mathbf{1}{\left\{ Y(1)\,\le\, z_1 \right\}}
-
F_{Y(1)}(z_1|X)
\right)
\left( 
\mathbf{1}{\left\{ Y(1)\,\le\, z_2 \right\}}
-
F_{Y(1)}(z_2|X)
\right)
\right]
\\
&
\ 
=
\ 
\E
\left[
\frac{1}{\pi(X)}
\left( 
F_{Y(1)}(z_1\land z_2|X)
\ 
-
\ 
F_{Y(1)}(z_1|X)
\cdot
F_{Y(1)}(z_2|X)
\right)
\right]
\,,
\end{align*}
and
\begin{align*}
  C_1
  &
  \ 
  =
  \ 
 \E
  \left[
  f^{z_1}_{1/\pi}
  \left( 
  F_{Y(1)}(z_2|X)
  -
F_{Y(1)}(z_2)
  \right)
  \right]
  \\
  &
  \ 
  =
  \ 
 \E
  \left[
\frac{T}{\pi(X)}
\left( 
\mathbf{1}{\left\{ Y(T)\,\le\, z_1 \right\}}
-
F_{Y(1)}(z_1|X)
\right)
  \left( 
  F_{Y(1)}(z_2|X)
  -
F_{Y(1)}(z_2)
  \right)
  \right]
  \\
  &
  \ 
  =
  \ 
 \E
  \left[
\left( 
\mathbf{1}{\left\{ Y(1)\,\le\, z_1 \right\}}
-
F_{Y(1)}(z_1|X)
\right)
  \left( 
  F_{Y(1)}(z_2|X)
  -
F_{Y(1)}(z_2)
  \right)
  \right]
  \\
  &
  \ 
  =
  \ 
  0
  \,.
\end{align*}
In the same way we see $C_2=0$. Finally,
\begin{align*}
  C_3
  &
  \ 
  =
  \ 
  \E
  \left[
  \left( 
  F_{Y(1)}(z_1|X)
  -
F_{Y(1)}(z_1)
  \right)
  \left( 
  F_{Y(1)}(z_2|X)
  -
F_{Y(1)}(z_2)
  \right)
  \right]
  \\
  &
  \ 
  =
  \ 
  \E
  \left[
  F_{Y(1)}(z_1|X)
  \cdot
  F_{Y(1)}(z_2|X)
  \right]
  \ 
  -
  \ 
  F_{Y(1)}(z_1)
  \cdot
  F_{Y(1)}(z_2)
  \,.
\end{align*}
Adding up the results gives us \eqref{cov:lp}.
\end{proof}


  \subsection{Proof of Theorem~\ref{th:main}}
    We have gathered all the results to prove Theorem~\ref{th:main}.
\begin{proof}
  \emph{(Theorem~\ref{th:main})}
  We connect the statement of the theorem to the error decomposition by Lemma~\ref{aa:mean:lemma_decomp}.
  By Lemma~\ref{aa:mean:l:r1}, 
  Lemma~\ref{aa:mean:l:r2},
  Lemma~\ref{aa:mean:r3:lem:conv}
   it follows 
   $\sup_{z\in\R}|R_i(z)|\overset{\P}{\to}0$ for $i=1,2,3$.
   Thus, by Slutzky's theorem (cf.\cite[Theorem~13.18]{Klenke2020})
   the behaviour of the limiting process is the one of Lemma~\ref{aa:mean:l:r4}.
\end{proof}

