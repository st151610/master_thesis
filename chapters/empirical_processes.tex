%%%%%%%%%%%%%%%%
% INTRODUCTION %
%%%%%%%%%%%%%%%%

Let 
$
  \left( 
    \Omega,
    \mathcal{A},
    \P
  \right)
$
be a probability space.
Let 
$X_1, \ldots, X_n$ be independent and identically-distributed (i.i.d.)
real random variables
and $\mathcal{F}$ a familiy of measurable functions
$
  f:
  \left( 
    \Omega,
    \mathcal{A},
    \P
  \right)
    \to
  \left( 
    \R,
    \mathcal{B}(\R),
    \lambda
  \right)
$.
Consider the map
\begin{gather}
  f
  \mapsto
  G_n f
  :=
  \sqrt{n}
  \left( 
    \frac{1}{n}
    \sum_{i = 1}^{n}
      f(X_i)
    -
    \E f
  \right).
\end{gather}
We call 
$
  \left( 
    G_n f
  \right)_{f \in \mathcal{F}}
$
the empirical process indexed by $\mathcal{F}$.

\begin{lemma}
  \emph{(Bernstein Inequality for Empirical Processes)}
  For any bounded, measurable function $f$
  it holds for all $t > 0$
  \begin{gather}
    \P 
    \left(
      \left| 
        G_n f
      \right|
      >
      t
    \right)
    \le
    2
    \exp
    \left( 
      - \frac{1}{4}
      \frac{t^2}
      {
        \E(f^2)
        +
        t
        \norm{f}_\infty
        /
        \sqrt{n}
      }
    \right)
  \end{gather}
\end{lemma}
\begin{proof}
  By the Markov inequality it holds for all $\lambda > 0$
  \begin{gather}
    \P
    \left( 
      G_n f 
      > 
      t
    \right)
    \le
    e^{-\lambda t}
    \E
    \exp
    \left( 
      \lambda
      G_n f 
    \right)
  \end{gather}
\end{proof}
