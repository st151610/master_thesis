The formulation of Problem~\ref{bw:1:primal} allows for great 
flexibility.
%
To streamline the analysis, however, we shall restrict it.
%
\begin{definition}
  \label{def:obj_f}
  We define $\varphi$ in Problem~\ref{bw:1:primal} by
  \begin{align*}
    \varphi
    \ 
    \colon
    \ 
    \R
    \ 
    \to
    \ 
    [0,\infty)
    \,,
    \qquad
    x\mapsto (x-1)^2
    \,.
  \end{align*}
\end{definition}
\begin{remark}
  If we plug this choice in Problem~\ref{bw:1:primal},
  we observe
  \begin{align*}
    \sum_{i=1}^{n} 
\varphi(w_i)
\ 
=
\ 
    \sum_{i=1}^{N}
    T_i
    \left( 
      T_i\cdot w_i
      -
      1
    \right)
    ^{2}
\ 
=
\ 
    \sum_{i=1}^{N}
    T_i
    \left( 
      T_i\cdot w_i
      -
      \frac{1}{N}
      \sum_{i=1}^{N} 
      T_i\cdot w_i
    \right)
    ^{2}
    \,.
  \end{align*}
  Thus, Problem~\ref{bw:1:primal} minimizes the sample variance of the weights 
  $(T_i\cdot w_i)$. This is in line with the objective function in \cite{Zubizarreta2015}.
\end{remark}
Next, we derive theoretical properties of $\varphi$ that we will use in the subsequent analysis.
\begin{lemma}
  \label{lem:obj_f}
  The function $\varphi$ of Definition~\ref{def:obj_f} satisfies

  \begin{enumerate}[label=(\roman*)]
    \item $\varphi$ is strictly convex and continuously differentiable on $\R$, with derivative $\varphi^{'}$
    \item
      The inverse of the derivative 
      $(\varphi^{'})^{-1}$
      exists and is continuously differentiable
    \item
      Both $\varphi^{'}$ and
      $(\varphi^{'})^{-1}$
      are uniformly continuous
  \end{enumerate}
\end{lemma}
\begin{proof}
  The proof is easy. We omit the details.
\end{proof}
The next lemma prepares a link to the assumptions of Theorem~\ref{cv:ts:th}.
\begin{lemma}
  \label{1165}
  The convex conjugate of $\varphi$ (see \eqref{def:convex_conjugate}) is
  \begin{gather*}
    \varphi^*
    \colon
    \R
    \ 
    \to
    \ 
    \R
    \,,
    \quad
    x^*
    \ 
    \mapsto
    \ 
    x^*
    \!
    \cdot
    (
    \varphi^{'}
    )^{-1}
    (x^*)
    \ 
    -
    \ 
    \varphi
    \left( 
      (
    \varphi^{'}
    )^{-1}
    (x^*)
    \right)
    \,.
  \end{gather*}
  Furthermore, $\varphi^*$ is strictly convex and continuously differentiable on $\R$.
\end{lemma}
\begin{proof}
We define
\begin{gather*}
 \phi
 \ 
 \colon
 \R
 \times
 \R
 \ 
 \to
 \ 
 \R
 \,,
 \quad
 (x,x^*)
 \ 
 \mapsto
 \ 
 x\cdot x^*
 \ 
 -
 \ 
 \varphi(x)
 \,.
\end{gather*}
Let $x^*\in\R$.
By Lemma~\ref{lem:obj_f}.\textit{(i)},
      $\varphi$ is continuously differentiable on $\R$ with derivative $\varphi^{'}$.
      The same holds for $\phi(\cdot,x^*)$ with derivative
      satisfying 
      \begin{gather*}
        \frac{\partial}{\partial x}
        \phi(x,x^*)
        \ 
        =
 \ 
        x^*
        -
        \varphi^{'}(x)
        \qquad
        \text{for all}\ 
        x\in\R
        \,.
      \end{gather*}
      By Lemma~\ref{lem:obj_f}.\textit{(ii)},
  it holds that 
  \begin{gather*}
    z
    \ 
    :=
    \ 
      (
    \varphi^{'}
    )^{-1}
    (x^*)
  \end{gather*}
  is an extreme point of $\phi(\cdot,x^*)$.
  Since $\varphi$ is strictly convex by Lemma~\ref{lem:obj_f}.\textit{(i)}, 
  $\phi(\cdot,x^*)$
  is strictly concave. 
  Thus,
  $z$ is the unique maximum point
  of
  $\phi(\cdot,x^*)$
  on $\R$.
  Thus
  \begin{align*}
    \varphi^*(x^*)
    &
    \ 
    =
    \ 
    \sup_{x\in\R}
    x\cdot x^* - \varphi(x)
    \ 
    =
    \ 
    \sup_{x\in\R}
    \phi(x,x^*)
    \\
    &
    \ 
    =
    \ 
    \phi(z,x^*)
    \\
    &
    \ 
    =
    \ 
    x^*
    \!
    \cdot
    (
    \varphi^{'}
    )^{-1}
    (x^*)
    \ 
    -
    \ 
    \varphi
    \left( 
      (
    \varphi^{'}
    )^{-1}
    (x^*)
    \right)
    \qquad 
    \text{for all}\ 
    x^*\in\R
    \,.
  \end{align*}
  Now we proof the second statement.
  Since
  $
      (
    \varphi^{'}
    )^{-1}
  $
  is continuously differentiable by Lemma~\ref{lem:obj_f}.\textit{(ii)}, it holds
  \begin{align}
    \label{0098}
    \begin{split}
    \frac{\partial}{\partial x^*}
     \varphi^*(x^*)
    &
    \ 
    =
    \ 
    (
    \varphi^{'}
    )^{-1}
    (x^*)
    \ 
    +
    \ 
    x^*
    \!
    \cdot
    \frac{\partial}{\partial x^*}
    (
    \varphi^{'}
    )^{-1}
    (x^*)
    \ 
    -
    \ 
    \varphi^{'}
    \left( 
      (
    \varphi^{'}
    )^{-1}
    (x^*)
    \right)
    \cdot
    \frac{\partial}{\partial x^*}
    (
    \varphi^{'}
    )^{-1}
    (x^*)
    \\
    &
    \ 
    =
    \ 
    (
    \varphi^{'}
    )^{-1}
    (x^*)
    \ 
    +
    \ 
    x^*
    \!
    \cdot
    \frac{\partial}{\partial x^*}
    (
    \varphi^{'}
    )^{-1}
    (x^*)
    \ 
    -
    \ 
    x^*
    \cdot
    \frac{\partial}{\partial x^*}
    (
    \varphi^{'}
    )^{-1}
    (x^*)
    \\
    &
    \ 
    =
    \ 
    (
    \varphi^{'}
    )^{-1}
    (x^*)
    \qquad
    \text{for all}\ 
    x^*\in\R
    \,.
    \end{split}
  \end{align}
  Since $\varphi$ is strictly convex and continuously differentiable, 
  $\varphi^{'}$ is continuous and strictly non-decreasing.
  Thus 
  $
    (
    \varphi^{'}
    )^{-1}
  $
  is continuous and strictly non-decreasing.
  It follows from \eqref{0098} that $\varphi^*$ is strictly convex and continuously differentiable.
\end{proof}
With Lemma~\ref{1165} we are ready to complete the link.
\begin{lemma}
  \label{9991}
  The function
\begin{gather*}
  \Phi
  \ 
  :
  \ 
  \R^n
  \to
  \ 
  \overline{\R}
  \,
  ,
  \qquad
  [w_1,\ldots,w_n]^\top
  \ 
  \mapsto
  \ 
  \sum_{i=1}^n \varphi(w_i)
  \,,
\end{gather*}
satisfies Assumption~\ref{cv:ts:asu}.
\end{lemma}
\begin{proof}
  By Example~\ref{cv:cc:ex}
  the convex conjugate of $\Phi$ is 
\begin{gather*}
  \Phi^*
  \ 
  :
  \ 
  \R^n
  \to
  \ 
  \R
  \,
  ,
  \qquad
  [\lambda_1,\ldots,\lambda_n]^\top
  \ 
  \mapsto
  \ 
  \sum_{i=1}^n \varphi^*(\lambda_i)
  \,,
\end{gather*}
where $\varphi^*$ is the convex conjugate of $\varphi$.
By Lemma~\ref{lem:obj_f}, $\varphi$ is strictly convex.
Thus,
$\Phi$ is strictly convex. By Lemma~\ref{1165}, $\varphi^*$ continuously differentiable on $\R$. Thus,
$\Phi^*$ is continuously differentiable on $\R^n$.
It follows the statement of Assumption~\ref{cv:ts:asu} for $\Phi$.
\end{proof}
\begin{takeaways}
  The choice of Definition~\ref{def:obj_f} 
  introduces the sample variance to Problem~\ref{bw:1:primal}.
  It has good practical and theoretical properties.
  Among the theoretical are strict convexity that allows linking Problem~\ref{bw:1:primal}
  to the theory of convex analysis.
\end{takeaways}
