In the previous section we have expounded our choice of $\varphi$ --- and with it the objective function of Problem~\ref{bw:1:primal}.
%
Now, we want to apply Theorem~\ref{cv:ts:th} to Problem~\ref{bw:1:primal}.
%
To this end, we provide its proper formulation.
%
For this we need some more notation. 

Let $\mathbf{I}_n$ be the $n$-dimensional unit matrix\index{$\mathbf{I}_N$, the $N$-dimensional unit matrix}, 
$\mathrm{0}_n$ and $\mathrm{1}_n$ the $n$-dimensional vectors containing only zeros or ones.
\index{
$\mathrm{0}_N$ and $\mathrm{1}_N$, the $N$-dimensional vectors containing only zeros or ones.
}
Furthermore, we define the matrix of basis functions \textbf{of the treated} to be
\index{$\mathbf{B}(\mathbf{X})$, matrix of basis functions of the treated}
\begin{gather*}
    \mathbf{B}(\mathbf{X})
    \ 
    :=
    \ 
    \begin{bmatrix}
      B(X_1), \ldots, B(X_n)
    \end{bmatrix}
    \in\R^{N\times n}
    \,.
\end{gather*}
Note that the size of $\mathbf{B}(\mathbf{X})$ depends on the random size $n\in\mathbb{N}$ of the treatment group in the sample.
\begin{lemma}
  \label{matrix_notation}
  A matrix formulation of Problem~\ref{bw:1:primal} is 
\begin{align*}
  %%%% objective %%%%
    &\underset{w \in \R^n}
    {\mathrm{minimize}}
    &&\qquad\qquad
    \Phi(w)
    &&&
    \\
    %%%% Ax >= b %%%%
    \nonumber
    &\mathrm{subject}\ \mathrm{to} 
    &&\qquad\qquad
    \mathbf{U}w
    \ 
    \ge
    \ 
    d
    \,,
    \\
    \nonumber
    &
    &&\qquad\qquad
    \mathbf{A}w
    \ 
    =
    \ 
    a
    \,,
\end{align*}
with objective function
\begin{gather*}
  \Phi
  \ 
  :
  \ 
  \R^n
  \to
  \ 
  \overline{\R}
  \,
  ,
  \qquad
  [w_1,\ldots,w_n]^\top
  \ 
  \mapsto
  \ 
  \sum_{i=1}^n \varphi(w_i)
  \,,
\end{gather*}
inequality matrix and vector
\begin{alignat*}{2}
    \mathbf{U}
    &
    \ 
    :=
    \ 
    \begin{bmatrix}
      \mathbf{I}_n
      \\
      \pm\,\mathbf{B}(\mathbf{X})
    \end{bmatrix}
    \in
    \R^{(n+  2 N)\times n}
        \qquad
    &&
d
    \ 
    :=
    \ 
    \begin{bmatrix}
      0_n
      \\
      -N\cdot\delta 
      \ 
      \pm\ 
      \sum_{i = 1}^{N} B(X_i)
    \end{bmatrix}
    \in
    \R^{n+  2 N}
    \,,
    \intertext{and equality matrix and vector}
    \mathbf{A}
    &
    \ 
    :=
    \ 
      \mathrm{1}_n
      ^\top
      \in\R^{1\times n}
      \qquad
    &&
    a
  \ 
    :=
    \ 
    N
    \in\mathbb{N}
    \,.
\end{alignat*}
\end{lemma}

\begin{proof}
  Recall that the box constraints of Problem~\ref{bw:1:primal} are
  \begin{gather*}
        \left| 
      \frac{1}{N} 
      \left( 
      \sum_{i = 1}^{n} 
      w_i
      B_k(X_i)
      -
      \sum_{i=1}^{N} 
      B_k(X_i)
      \right)
    \right|
    \ 
    \le 
    \ 
    \delta_k
    \qquad
    \text{for all}\ 
    k\in \left\{ 1,\ldots, N \right\}
    \,.
  \end{gather*}
  Put differently, it holds both
  \begin{align*}
    -
      \sum_{i = 1}^{n} 
      w_i
      B_k(X_i)
      \ 
    \ge 
      \ 
    -
    N
    \delta_k
      -
      \sum_{i=1}^{N} 
      B_k(X_i)
      \qquad 
    \text{and}
      \qquad
      \sum_{i = 1}^{n} 
      w_i
      B_k(X_i)
      \ 
    \ge 
      \ 
    -
    N
    \delta_k
      +
      \sum_{i=1}^{N} 
      B_k(X_i)
  \end{align*}
  for all 
  $
    k\in \left\{ 1,\ldots, N \right\}
  $. In matrix notation this is 
  \begin{gather*}
    \pm\mathbf{B}(\mathbf{X})w
    \ 
    \ge
    \ 
    [d_{n+1},\ldots, d_{n+  2 N}]^\top
    \,.
  \end{gather*}
  Proving the rest of the statements is straightforward. We omit the details.
\end{proof}
\begin{remark}
  The inequality constraints of
  Lemma~\ref{matrix_notation} differ from its counterpart
  \cite[Proof of Lemma~1]{Wang2019}.
  We don't transform the variable $w$, but shift to $d$ what prevents us from keeping $w$.
  Note, that the choice of
  \cite[Proof of Lemma~1]{Wang2019} leads to a mistake on page 21.
  The mistake is most obvious in the second display, where the first implication follows from dividing by 0.
  I discussed this with the authors and proposed a version of Lemma\ref{matrix_notation} to solve the problem. I think it's best not to transform variables, because the mistake comes from (wrongly) calculating the convex conjugate of the (more complicated) transformed version of the objective function. The subsequent analysis even simplifies with my version.
  I was surprised to find the (exact) same mistake in the earlier paper 
  \cite[page 35 second display]{Chan2016}. 
\end{remark}
%
In the next lemma we apply Theorem~\ref{cv:ts:th} to Problem~\ref{bw:1:primal}.
%
\begin{lemma}
  Consider the optimization problem
\begin{align}
  \label{9993}
  \begin{split}
  \underset
  {\begin{smallmatrix}
\rho\,,\, \lambda^+,\,\lambda^-\ge 0 \\
\lambda_0\in\R
  \end{smallmatrix}}
  {
    \mathrm{maximize}
  }
  \quad
  &
  -
\sum_{i=1} 
  ^n
    \,
  \varphi^*
  \!
  \left( 
    \rho_i
    +
\lambda_0
+
\inner
{B(X_i)}
{
\lambda^+
-
\lambda^-
}
  \right)
  \\
  &
+
\ 
\sum_{i=1}^{N} 
  \left( 
\lambda_0
+
\inner
{B(X_i)}
{
\lambda^+
-
\lambda^-
}
  \right)
  \,
  \ 
-
\ 
\inner
{\delta}
{
\lambda^+
+
\lambda^-
}
  \,.
  \end{split}
\end{align}
If there exists the optimal solution 
$
(\rho^\dagger,\lambda_0^\dagger,\lambda^{+,\dagger},\lambda^{-,\dagger})
$
then the unique optimal solutions to Problem~\ref{bw:1:primal} are 
\begin{gather*}
  w^\dagger_i
  \ 
  :=
  \ 
  (
  \varphi^{'}
  )^{-1}
  \left(
    \rho^\dagger_i
  \ 
    +
  \ 
\lambda_0^\dagger
  \ 
+
  \ 
\inner
{B(X_i)}
{
  \lambda^{+,\dagger}
-
\lambda^{-,\dagger}
}
  \right)
  \qquad
  \text{for all}\ 
  i\in
  \left\{ 1,\ldots,n \right\}
  \,.
\end{gather*}
\end{lemma}
\begin{proof}
  First, note that by the strict convexity of $\varphi^*$ (see Lemma~\ref{1165}), a solution to Problem~\eqref{9993}
  is unique (if it exists).
  By Lemma~\ref{matrix_notation},
  Problem~\ref{bw:1:primal} has the form required in Theorem~\ref{cv:ts:th}.
  By Lemma~\ref{9991}, the objective function $\Phi$ of Problem~\ref{bw:1:primal}
  satisfies Assumption~\ref{cv:ts:asu}.
  Thus we can apply
  Theorem~\ref{cv:ts:th} to Problem~\ref{bw:1:primal}.
  Calculations yield the result.
\end{proof}
With the next theorem we merge $\lambda^{+}$,$\lambda^{-}\ge 0$ to 
$\lambda=\lambda^{+}-\lambda^{-}\in\R$.
Let $[x]^+:=0\land x$ be the positive part of $x\in\R$.
\index{$[x]^+$, positive part of $x\in\R$}
\begin{ftheorem}
  \label{dual_solution_th}
  Consider the optimization problem
\begin{align}
  \label{dual}
  \begin{split}
  \underset
  {\begin{smallmatrix}
      \rho&\in&&\R^N 
      \\
      \lambda_0 & \in&&\R
      \\
      \lambda&\in&&\R^{N}
  \end{smallmatrix}}
  {
    \mathrm{minimize}
  }
  \quad
  \frac{1}{N}
\sum_{i=1} 
  ^N
  &
  \Big[
  T_i
  \cdot
  \varphi^*
  \!
  \left( 
    \rho_i
    +
\lambda_0
+
\inner
{B(X_i)}
{
\lambda
}
  \right)
  \ 
  -
  \ 
\lambda_0
-
\inner
{B(X_i)}
{
\lambda
}
\Big]
  \ 
+
\ 
\inner
{\delta}
{
  |\lambda|
}
  \,,
  \\
  \mathrm{subject}\ \mathrm{to}
  \quad
  \qquad
  &
  \rho_i \ge 0 
  \quad 
  \mathrm{for}\ \mathrm{all}\ i\le n
  \qquad 
  \\
  \mathrm{and}
  \quad
  \qquad
  &
  \rho_i
  \ 
  =
  \ 
  \left[ 
  \varphi^{-1}
  (0)
  -
  \left( 
  \lambda_0 + \inner{B(X_i)}{\lambda}
  \right)
  \right]^+
  \quad 
  \mathrm{for}\ \mathrm{all}\ i>n
  \,.
\end{split}
\end{align}
If there exists the optimal solution 
$
(\rho^\dagger,\lambda_0^\dagger,\lambda^\dagger)
$
then the unique optimal solutions to Problem~\ref{bw:1:primal} are 
\begin{gather*}
  w^\dagger_i
  \ 
  :=
  \ 
  (
  \varphi^{'}
  )^{-1}
  \left(
    \rho^\dagger_i
  \ 
    +
  \ 
\lambda_0^\dagger
  \ 
+
  \ 
\inner
{B(X_i)}
{
\lambda^{\dagger}
}
  \right)
  \qquad
  \text{for all}\ 
  i\in
  \left\{ 1,\ldots,n \right\}
  \,.
\end{gather*}
\end{ftheorem}

\begin{proof}
  Assume that
$
  (\rho^\dagger,\lambda_0^\dagger,\lambda^{+,\dagger},\lambda^{-,\dagger})
$
is an optimal solution to Problem~\ref{9993}.
We write
\begin{align*}
  G
  (\rho,\lambda_0,\lambda^+,\lambda^-)
  &
  \ 
  :=
  \ 
 -
\sum_{i=1} 
  ^n
    \,
  \varphi^*
  \!
  \left( 
    \rho_i
    +
\lambda_0
+
\inner
{B(X_i)}
{
\lambda^+
-
\lambda^-
}
  \right)
  \\
  &
  \qquad
+
\ 
\sum_{i=1}^{N} 
  \left( 
\lambda_0
+
\inner
{B(X_i)}
{
\lambda^+
-
\lambda^-
}
  \right)
  \,
  \ 
-
\ 
\inner
{\delta}
{
\lambda^+
+
\lambda^-
}
  \,.
\end{align*}
 To eliminate the remaining constraints, 
  we paraphrase \cite[pages~19-20]{Wang2019}.
  We show 
  for all $i \in \left\{ 1,\ldots,N \right\}$
\begin{alignat}{2}
  \notag
  \text{either}
  &
  &&
  \qquad
  \lambda_i^{+,\dagger} > 0
  \\
  \label{9992}
  \text{or}
  &
  &&
  \qquad
  \lambda_i^{-,\dagger} > 0
  \,.
\end{alignat}
Assume towards a contradiction that 
\begin{gather}
  \label{1232}
  \text{
there exists
  } 
  \ 
i \in \left\{ 1,\ldots,N \right\}
\ 
\text{such that}
\qquad
  \lambda_i^{+,\dagger} > 0
  \qquad 
  \text{and}
  \qquad
  \lambda_i^{-,\dagger} > 0
  \,.
\end{gather}
Consider
  \begin{align*}
    \tilde{\lambda}^{+,\dagger}
    &
    \ 
    :=
    \ 
    \begin{bmatrix}
      \ 
      \lambda_1^{+,\dagger}
      \ldots,
      \ 
      \lambda_i^{+,\dagger}
      \!
      \ 
      -
      \ 
      (
      \lambda_i^{+,\dagger}
      \!
      \land
      \lambda_i^{-,\dagger}
      )\,,
      \ 
      \ldots,
      \lambda_{N}^{+,\dagger}
    \end{bmatrix}
    ^\top
    \intertext{and}
    \tilde{\lambda}^{-,\dagger}
    &
    \ 
    :=
    \ 
    \begin{bmatrix}
      \ 
      \lambda_1^{-,\dagger}
      \ldots,
      \ 
      \lambda_i^{-,\dagger}
      \!
      \ 
      -
      \ 
      (
      \lambda_i^{+,\dagger}
      \!
      \land
      \lambda_i^{-,\dagger}
      )\,,
      \ 
      \ldots,
      \lambda_{N}^{-,\dagger}
    \end{bmatrix}
    ^\top
    \,.
  \end{align*}
  Since
  \begin{gather*}
      \lambda_i^{\pm,\dagger}
      \!
      \ 
      -
      \ 
      (
      \lambda_i^{+,\dagger}
      \!
      \land
      \lambda_i^{-,\dagger}
      )
      \ 
      \ge 
      \ 
      0
      \,,
  \end{gather*}
  the perturbed vectors $\tilde{\lambda}^{\pm,\dagger}$ are  in the domain of the 
  optimization problem.
  By Assumption~\eqref{1232} and $\delta>0$ it follows
  \begin{align*}
  G
  \left( 
  \rho^\dagger,\lambda_0^\dagger,\tilde{\lambda}^{+,\dagger},\tilde{\lambda}^{-,\dagger}
  \right)
  \ 
  -
  \ 
  G
  \left( 
  \rho^\dagger,\lambda_0^\dagger,\lambda^{+,\dagger},\lambda^{-,\dagger}
  \right)
  \ 
  =
  \ 
  2
  \cdot
  \delta_i
  \cdot
      (
      \lambda_i^{+,\dagger}
      \!
      \land
      \lambda_i^{-,\dagger}
      )
  \ 
  >
  \ 
  0
  \,,
  \end{align*}
  which contradicts the optimality of
$
  (\rho^\dagger,\lambda_0^\dagger,\lambda^{+,\dagger},\lambda^{-,\dagger})
$
(it is supposed to be a maximum in the domain of the optimization problem)
.
It follows \eqref{9992}.
But then 
$
\lambda^{\pm,\dagger}_i
\ge 0
$
collapses to
$
\lambda_i^\dagger\in \R
$ 
for all
$i\in \left\{ 0,\ldots,N \right\}$, that is, we set
\begin{gather*}
 \lambda_i^\dagger
 \ 
 =
 \ 
 \lambda_i^{+,\dagger}
 \ 
 -
 \ 
 \lambda_i^{-,\dagger}
 \qquad
 \text{and}
 \qquad
|\lambda_i^\dagger|
\ 
=
\ 
\lambda_i^{+,\dagger}
\ 
+
\ 
\lambda_i^{-,\dagger}
\,.
\end{gather*}
Thus, we can extend the domain of Problem~\ref{9993} to $\lambda\in\R^{N}$ and update the objective function in the following way
(without changing the optimal solution).
\begin{align*}
  G
  (\rho,\lambda_0,\lambda)
  &
  \ 
  :=
  \ 
 -
\sum_{i=1} 
  ^n
    \,
  \varphi^*
  \!
  \left( 
    \rho_i
    +
\lambda_0
+
\inner
{B(X_i)}
{
\lambda
}
  \right)
  \\
  &
  \qquad
+
\ 
\sum_{i=1}^{N} 
  \left( 
\lambda_0
+
\inner
{B(X_i)}
{
\lambda
}
  \right)
  \,
  \ 
-
\ 
\inner
{\delta}
{
  |\lambda|
}
  \,.
\end{align*}
Multiplying $G$ with $-1/N$ doesn't change the solution either
(if we search instead for a minimum).
To finish the proof, we choose the notation with $T_i$ instead of $n$. This extends the domain of $\rho$ to $\R^N_{\ge 0}$, but the 
new $\rho_i$ are not effective because of $T_i=0$ for all $i>n$. 
Thus we may set them to an arbitrary value.
\end{proof}
\begin{remark}
  The dual variables $(\rho,\lambda_0,\lambda)$ are connected to the constraints of Problem~\ref{bw:1:primal}, that is,
  \begin{align*}
    \rho&\in\R^N_{\ge 0}
  &&
  \qquad
  \text{to}
  \qquad
  \ 
  T_i\cdot w_i\ge 0
  \qquad
  \text{for all}\ i\in \left\{ 1,\ldots,N \right\}
  \,,
  \\
    \lambda_0&\in\R
  &&
  \qquad
  \text{to}\qquad
  \ 
  \frac{1}{N}
  \sum_{i=1}^{N} 
  T_i\cdot w_i
  -
  1
  \ 
  =
  \ 
  0
  \,,
  \\
    \lambda&\in\R^N 
  &&
  \qquad
  \text{to}\qquad
    \left| 
      \frac{1}{N} 
      \left( 
      \sum_{i = 1}^{n} 
      w_i
      \cdot
      B_k(X_i)
      \ 
      -
      \ 
      \sum_{i=1}^{N} 
      B_k(X_i)
      \right)
    \right|
    \ 
    \le 
    \ 
    \delta_k
    \qquad
    \text{for all}\ 
    k \in \left\{ 1, \ldots, N \right\}
    \,.
  \end{align*}
  Note that we shifted the complexity of the constraints of Problem~\ref{bw:1:primal} to the objective function of Problem~\ref{dual}. In Lemma~\ref{lem:link_conv_p} we derive sufficient conditions for the existence of an optimal solution. The second point of this lemma is that the solution (if it exists) is close to some oracle parameter $\lambda^*$. It turns out that by this we can derive consistency of optimal (dual) solutions for the oracle parameter (see Theorem~\ref{th:cons_dual}). In the end, by the functional relationship of dual and primal solutions (see Theorem~\ref{dual_solution_th}), we can derive consistency of the optimal weights for the inverse propensity score. This is done in Chapter~\ref{ch:consis}.
\end{remark}
As an example for the interested reader and as transition to the next section, we show how $\rho$ is linked to the first constraint of Problem~\ref{bw:1:primal}. 
\begin{lemma}
  \label{lem:simple_weights}
  Let 
$
(\rho^\dagger,\lambda_0^\dagger,\lambda^\dagger)
$
be the optimal solution to Problem~\ref{dual}.
It holds
\begin{align*}
  \rho_i^\dagger
  =
  \left[ 
  \varphi^{'}
  (0)
  -
  \left( 
    \lambda_0^\dagger + \inner{B(X_i)}{\lambda^\dagger}
  \right)
  \right]^+
\qquad
  \text{for all}\ i\in \left\{ 1,\ldots,N \right\}
  \,.
\end{align*}
Furthermore, the unique optimal solutions to Problem~\ref{bw:1:primal} are
\begin{align*}
  w^\dagger_i
  \ 
  :=
  \ 
  \left[
  (
  \varphi^{'}
  )^{-1}
  \left(
\lambda_0^\dagger
  \ 
+
  \ 
\inner
{B(X_i)}
{
\lambda^{\dagger}
}
  \right)
  \right]^+
  \qquad
  \text{for all}\ 
  i\in
  \left\{ 1,\ldots,n \right\}
  \,.
\end{align*}
\end{lemma}
\begin{proof}
  For $i>n$ this is clear.
  Let $i\le n$, that is, $T_i=1$.
  By the complementary slackness (see the proof of Theorem~\ref{cv:ts:th}), if $\rho_i^\dagger>0$ it holds
  \begin{align*}
    w_i^\dagger
    \ 
    =
    \ 
    (
    \varphi^{'}
    )^{-1}
\left( 
  \rho_i^\dagger
  +
    \lambda_0^\dagger + \inner{B(X_i)}{\lambda^\dagger}
\right)
    \ 
=
    \ 
0
\,.
  \end{align*}
  By Lemma~\ref{lem:obj_f} it follows
  \begin{align*}
    \rho_i^\dagger
    \ 
    =
    \ 
    \varphi^{'}
    (0)
    -
    \left( 
    \lambda_0^\dagger + \inner{B(X_i)}{\lambda^\dagger}
    \right)
    \,.
  \end{align*}
  If on the other hand $\rho_i^\dagger=0$, then by the complementary slackness it follows
  \begin{align*}
    w_i^\dagger
    \ 
    =
    \ 
    (
    \varphi^{'}
    )^{-1}
\left( 
    \lambda_0^\dagger + \inner{B(X_i)}{\lambda^\dagger}
\right)
    \ 
\ge 
    \ 
0
\,,
  \end{align*}
  and
  \begin{align*}
    0
    \ 
    \ge
    \ 
    \varphi^{'}
    (0)
    -
    \left( 
    \lambda_0^\dagger + \inner{B(X_i)}{\lambda^\dagger}
    \right)
    \,.
  \end{align*}
\end{proof}
\begin{takeaways}
  We derive a dual formulation of Problem~\ref{bw:1:primal} that is easier to analyse.
  Theorem~\ref{dual_solution_th} provides a functional relationship of optimal dual solutions and 
  optimal weights.
  The dual variables are connected to the constraints of the primal problem.
\end{takeaways}
