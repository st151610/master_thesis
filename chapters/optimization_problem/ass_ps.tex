\begin{definition}
  We define the \textbf{propensity score function} by
  \begin{gather*}
    \pi
    \ 
    \colon
    \R^d
    \ 
    \to
    \ 
    [0,1]
    \,,
    \qquad
    x
    \ 
    \mapsto
    \ 
    \P
    [
    T=1
    |
    X=x
    ]
    \,,
  \end{gather*}
  that is, 
  as the conditional probability of treatment given individual characteristics $x\in\R^d$.
\end{definition}
\begin{remark}
  We define $\pi$ on the whole $\R^d$, although $\mathcal{X}\subset \R^d$ may be a much smaller (possibly finite or countable) subset.
  The reason is that we want to assume continuity.
\end{remark}
\begin{assumption}
  \label{asu:ps}
  The propensity score function $\pi$ satisfies
  \begin{enumerate}[label=(\roman*)]
    \item
      There exists a constant $C_\pi>0$ such that 
      $\pi(x)>C_\pi$ for all $x\in\mathcal{X}$
    \item
      $\pi$ is continuously differentiable on $\R^d$
  \end{enumerate}
\end{assumption}

Next we show two parametric propensity score models and that meet the assumptions.
\begin{example}
  \emph{(logit model)}
  We assume that there exists $(\beta_0,\beta)\in\R^{d+1}$ such that
  \begin{gather*}
    \pi(x)
    \ 
    =
    \ 
    \frac{1}
    {1
      +
      \exp(-\beta_0-\inner{x}{\beta})
    }
    \qquad
    \text{for all}
    \ 
    x\in\R^d
    \,.
  \end{gather*}
  Clearly, the logit model satisfies Assumption~\ref{asu:ps}.
\end{example}

\begin{example}
  \emph{(probit model)}
  We assume that there exists $\gamma\in\R^d$ such that
  \begin{gather*}
    \pi(x)
    \ 
    =
    \ 
    \Phi(\inner{x}{\gamma})
    \qquad
    \text{for all}
    \ 
    x\in\R^d
    \,,
  \end{gather*}
  where $\Phi$ is the distribution function of the standard normal 
  distribution (please do not confuse with the objective function of Problem~\ref{bw:1:primal}), that is
  \begin{gather*}
    \Phi(x)
    :=
    \frac{1}{\sqrt{2\pi}}
    \int_{-\infty}
    ^x
    \exp
    \left( 
      -\frac{s^2}{2}
    \right)
    \,
    ds
    \,.
  \end{gather*}
  It is readily clear that the probit model also satisfies Assumption~\ref{asu:ps}.
\end{example}
