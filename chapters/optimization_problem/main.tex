There are different ways
to generate weights for covariate balance.
%
We discussed this in the introduction.
%
Now, we introduce the balancing weights framework of \cite{Wang2019}.
%
It is a (generic) convex optimization problem
that enforces covariate balance by constraints on the search space.
%
Similar to classical propensity score estimates, it only
extracts from the data information about treatment status and individual characteristics.
%
It ignores the outcome.
%
This gives the additional option to balance covariates before observing outcomes.

%
The primary optimization task is to minimize an objective function
over a predefined search space.
%
From a practical point of view, the objective function
instils additional goodness in the weights, 
for example, 
low sample variance \cite[Introduction]{Zubizarreta2015}. 
%
%
More important, however, are the constraints that enforce covariate balance.
%
Both objective function and design of the constraints distinguish the method.
%

From a mathematical point of view,
we have slightly different requirements.
%
The proofs should be as clear an short as possible.
%
The mathematical objects involved should help with that
(or at least not prevent it).
%
Therefore, we focus on theoretical properties
of the method.
%
As a by-product, we create new ideas that wait for testing in practice.

%
The notion of what to balance is defined by the basis functions of the covariates.
%
It is common that they form a regression basis.
%
We shall use this flexibility to introduce basis functions with random design to this framework.
%
This makes the proofs easier.
\section{Introduction}

Let $(T_1,X_1),\ldots,(T_N,X_N)$ be independent and identically-distributed copies of $T$ and $X$ (see Introduction page). 
We gather them in the (random) data set 
\begin{gather*}
D_N
\ 
:=
\ 
\left\{\, (T_i,X_i)\ \colon\  i\in \left\{ 1,\ldots,N \right\}\, \right\}
\,.
\end{gather*}
Furthermore, let
\begin{gather*}
  n
  \ 
  :=
  \ 
  \# 
  \left\{ 
    i\in \left\{ 1,\ldots,N \right\}
    \ 
    \colon
    \ 
    T_i=1
  \right\}
\end{gather*}
be the number of treated units. This is a random variable. We assume the order $T_i=1$ for all $i\le n$.

Let
$\# A\in\mathbb{N}_0$ be the number of elements of a finite set $A$, let
$\overline{\R}:=\R\cup \left\{ \infty \right\}$ 
be the extended real numbers, 
$\Sigma_\mathcal{X}$ be a $\sigma$-algebra on the covariate space $\mathcal{X}$,
and let $\mathcal{B}(\R)$ denote the Borel-$\sigma$-algebra on the real numbers.

For a convex function $\varphi\colon \R\to\overline{\R}$,
a finite set of measurable (basis-)functions on the covariate space
\begin{gather*}
  \mathfrak{B}
  :
  =
  \left\{ 
B_k\colon (\mathcal{X},\Sigma_\mathcal{X})\to(\R,\mathcal{B}(\R))
\ 
|
\ 
k\in \left\{ 1,\ldots,\# \mathfrak{B} \right\}
  \right\}
\end{gather*}
and a constraints vector $\delta=[\delta_1,\ldots,\delta_{\# \mathfrak{B}}]^\top>0$
we consider the following (random) convex optimization problem.
\newpage
\begin{fproblem}
  \label{bw:1:primal}
\begin{align*}
  %%%% objective %%%%
    &\underset{w_1, \ldots, w_n \in \R}
    {\text{minimize}}
    &&\qquad\qquad
    \sum_{i = 1}^{n} 
    \varphi(w_i)
    &&&
    \\
    %%%% w_i T_i >= 0 %%%%
    &\text{subject to}
    &&\qquad\qquad
    w_i 
    \ge
    0
    &&&
    \qquad
    \text{for all}\ 
    i \in \left\{ 1, \ldots, n \right\}
    \,,
    \\
    %%%% 1/n sum w = 1 %%%%
    & 
    &&\qquad\qquad
    \frac{1}{N}
    \sum_{i=1}^{n} 
    w_i
    =1
    \\
    %%%% box constraints %%%%
    & 
    &&\qquad
    \left| 
      \frac{1}{N} 
      \left( 
      \sum_{i = 1}^{n} 
      w_i
      B_k(X_i)
      -
      \sum_{i=1}^{N} 
      B_k(X_i)
      \right)
    \right|
    \ 
    \le 
    \ 
    \delta_k
    &&&
    \qquad
    \text{for all}\ 
    k \in \left\{ 1, \ldots, \# \mathfrak{B} \right\}
    \,.
\end{align*}
\end{fproblem}
What is random in Problem~\ref{bw:1:primal}?
First, the dimension of the search space $(w\in\R^n)$ depends on the random variable $n$. 
Thus, we only compute weights for the treated units (the ones with $T_i=1$).
Next consider the \textbf{objective function}
\begin{gather*}
    \sum_{i = 1}^{n} 
    \varphi(w_i)
    \,.
\end{gather*}
The number of summands is random (again $n$). We sometimes use the equivalent notation
\begin{gather*}
    \sum_{i = 1}^{N} 
    T_i
    \cdot
    \varphi(w_i)
    \,,
\end{gather*}
where we set the weights of the untreated (the ones with $T_i=0$) to some arbitrary value in the domain of $\varphi$.
Let's consider the constraints. There is no randomness in the first two constraints.
\begin{gather*}
    w_i 
    \ge
    0
    \qquad
    \text{for all}\ 
    i \in \left\{ 1, \ldots, n \right\}
    \quad
    \text{and}
    \quad
    \frac{1}{N}
    \sum_{i=1}^{n} 
    w_i
    =1
    \,.
\end{gather*}
They only make sure, that the weights (divided by $N$) form a convex combination.
If, for example, the outcome space $\mathcal{Y}$ is convex we make sure that a weighted-mean-estimate of $\E[Y(1)]$ satisfies
\begin{gather*}
  \widehat{Y}(1) 
  \ 
  :=
  \ 
  \frac{1}{N}
  \sum_{i=1}^{n} 
  w_i\cdot Y_i
  \ 
  \in
  \ 
  \mathcal{Y}
\end{gather*}
or that a weighted-mean-estimate of the distribution function of $Y(1)$ satisfies
\begin{gather*}
  \widehat{F}_{Y(1)} 
  \ 
  :=
  \ 
  \frac{1}{N}
  \sum_{i=1}^{n} 
  w_i\cdot \mathbf{1}\left\{ Y_i\le z \right\}
  \ 
  \in
  \ 
  [0,1]
  \,.
\end{gather*}
There remain the box constraints
\begin{gather*}
    \left| 
      \frac{1}{N} 
      \left( 
      \sum_{i = 1}^{n} 
      w_i
      B_k(X_i)
      -
      \sum_{i=1}^{N} 
      B_k(X_i)
      \right)
    \right|
    \ 
    \le 
    \ 
    \delta_k
    \qquad
    \text{for all}\ 
    k \in \left\{ 1, \ldots, \# \mathfrak{B} \right\}
    \,.
\end{gather*}
Here the number of summands in
\begin{gather*}
      \sum_{i = 1}^{n} 
      w_i
      B_k(X_i)
\end{gather*}
is again random, and we sometimes use the equivalent notation
\begin{gather*}
      \sum_{i = 1}^{N} 
      T_i
      \cdot
      w_i
      \cdot
      B_k(X_i)
      \,.
\end{gather*}
The (basis-)functions in $\mathfrak{B}$ can be random functions, for example, if they depend on the data $D_N$.
Also the constraint vector $\delta$ can depend on the data (see \cite[Algorithm~1 on page 11]{Wang2019}).

\section{Objective Function}
The formulation of Problem~\ref{bw:1:primal} allows for great 
flexibility.
%
To obtain clear and short proofs, however, we have to restrict it.
%
\begin{definition}
  \label{def:obj_f}
  We define $\varphi$ in Problem~\ref{bw:1:primal} by
  \begin{align*}
    \varphi
    \ 
    \colon
    \ 
    \R
    \ 
    \to
    \ 
    [0,\infty)
    \,,
    \qquad
    x\mapsto (x-1)^2
    \,.
  \end{align*}
\end{definition}
\begin{remark}
  If we plug this choice in Problem~\ref{bw:1:primal},
  we observe
  \begin{align*}
    \sum_{i=1}^{n} 
\varphi(w_i)
\ 
=
\ 
    \sum_{i=1}^{N}
    T_i
    \left( 
      T_i\cdot w_i
      -
      1
    \right)
    ^{2}
\ 
=
\ 
    \sum_{i=1}^{N}
    T_i
    \left( 
      T_i\cdot w_i
      -
      \frac{1}{N}
      \sum_{i=1}^{N} 
      T_i\cdot w_i
    \right)
    ^{2}
    \,.
  \end{align*}
  Thus Problem~\ref{bw:1:primal} minimizes the sample variance of the weights 
  $(T_i\cdot w_i)$. This is in line with the objective function in \cite{Zubizarreta2015}.
\end{remark}
Next, we derive theoretical properties of $\varphi$ that we will use in the subsequent analysis.
\begin{lemma}
  \label{lem:obj_f}
  The function $\varphi$ of Definition~\ref{def:obj_f} satisfies

  \begin{enumerate}[label=(\roman*)]
    \item $\varphi$ is strictly convex and continuously differentiable on $\R$, with derivative $\varphi^{'}$
    \item
      The inverse of the derivative 
      $(\varphi^{'})^{-1}$
      exists and is continuously differentiable
    \item
      Both $\varphi^{'}$ and
      $(\varphi^{'})^{-1}$
      are uniformly continuous
  \end{enumerate}
\end{lemma}
\begin{proof}
  The proof is easy. We omit the details.
\end{proof}
The next lemma prepares a link to the assumptions of Theorem~\ref{cv:ts:th}.
\begin{lemma}
  \label{1165}
  The convex conjugate of $\varphi$ (see \eqref{def:convex_conjugate}) is
  \begin{gather*}
    \varphi^*
    \colon
    \R
    \ 
    \to
    \ 
    \R
    \,,
    \quad
    x^*
    \ 
    \mapsto
    \ 
    x^*
    \!
    \cdot
    (
    \varphi^{'}
    )^{-1}
    (x^*)
    \ 
    -
    \ 
    \varphi
    \left( 
      (
    \varphi^{'}
    )^{-1}
    (x^*)
    \right)
    \,.
  \end{gather*}
  Furthermore, $\varphi^*$ is strictly convex and continuously differentiable on $\R$.
\end{lemma}
\begin{proof}
We define
\begin{gather*}
 \phi
 \ 
 \colon
 \R
 \times
 \R
 \ 
 \to
 \ 
 \R
 \,,
 \quad
 (x,x^*)
 \ 
 \mapsto
 \ 
 x\cdot x^*
 \ 
 -
 \ 
 \varphi(x)
 \,.
\end{gather*}
Let $x^*\in\R$.
By Lemma~\ref{lem:obj_f}.\textit{(i)},
      $\varphi$ is continuously differentiable on $\R$ with derivative $\varphi^{'}$.
      The same holds for $\phi(\cdot,x^*)$ with derivative
      satisfying 
      \begin{gather*}
        \frac{\partial}{\partial x}
        \phi(x,x^*)
        \ 
        =
 \ 
        x^*
        -
        \varphi^{'}(x)
        \qquad
        \text{for all}\ 
        x\in\R
        \,.
      \end{gather*}
      By Lemma~\ref{lem:obj_f}.\textit{(ii)},
  it holds that 
  \begin{gather*}
    z
    \ 
    :=
    \ 
      (
    \varphi^{'}
    )^{-1}
    (x^*)
  \end{gather*}
  is an extreme point of $\phi(\cdot,x^*)$.
  Since $\varphi$ is strictly convex by Lemma~\ref{lem:obj_f}.\textit{(i)}, 
  $\phi(\cdot,x^*)$
  is strictly concave. 
  Thus,
  $z$ is the unique maximum point
  of
  $\phi(\cdot,x^*)$
  on $\R$.
  Thus
  \begin{align*}
    \varphi^*(x^*)
    &
    \ 
    =
    \ 
    \sup_{x\in\R}
    x\cdot x^* - \varphi(x)
    \ 
    =
    \ 
    \sup_{x\in\R}
    \phi(x,x^*)
    \\
    &
    \ 
    =
    \ 
    \phi(z,x^*)
    \\
    &
    \ 
    =
    \ 
    x^*
    \!
    \cdot
    (
    \varphi^{'}
    )^{-1}
    (x^*)
    \ 
    -
    \ 
    \varphi
    \left( 
      (
    \varphi^{'}
    )^{-1}
    (x^*)
    \right)
    \qquad 
    \text{for all}\ 
    x^*\in\R
    \,.
  \end{align*}
  Now we proof the second statement.
  Since
  $
      (
    \varphi^{'}
    )^{-1}
  $
  is continuously differentiable by Lemma~\ref{lem:obj_f}.\textit{(ii)}, it holds
  \begin{align}
    \label{0098}
    \begin{split}
    \frac{\partial}{\partial x^*}
     \varphi^*(x^*)
    &
    \ 
    =
    \ 
    (
    \varphi^{'}
    )^{-1}
    (x^*)
    \ 
    +
    \ 
    x^*
    \!
    \cdot
    \frac{\partial}{\partial x^*}
    (
    \varphi^{'}
    )^{-1}
    (x^*)
    \ 
    -
    \ 
    \varphi^{'}
    \left( 
      (
    \varphi^{'}
    )^{-1}
    (x^*)
    \right)
    \cdot
    \frac{\partial}{\partial x^*}
    (
    \varphi^{'}
    )^{-1}
    (x^*)
    \\
    &
    \ 
    =
    \ 
    (
    \varphi^{'}
    )^{-1}
    (x^*)
    \ 
    +
    \ 
    x^*
    \!
    \cdot
    \frac{\partial}{\partial x^*}
    (
    \varphi^{'}
    )^{-1}
    (x^*)
    \ 
    -
    \ 
    x^*
    \cdot
    \frac{\partial}{\partial x^*}
    (
    \varphi^{'}
    )^{-1}
    (x^*)
    \\
    &
    \ 
    =
    \ 
    (
    \varphi^{'}
    )^{-1}
    (x^*)
    \qquad
    \text{for all}\ 
    x^*\in\R
    \,.
    \end{split}
  \end{align}
  Since $\varphi$ is strictly convex and continuously differentiable, 
  $\varphi^{'}$ is continuous and strictly non-decreasing.
  Thus 
  $
    (
    \varphi^{'}
    )^{-1}
  $
  is continuous and strictly non-decreasing.
  It follows from \eqref{0098} that $\varphi^*$ is strictly convex and continuously differentiable.
\end{proof}
With Lemma~\ref{1165} we are ready to complete the link.
\begin{lemma}
  \label{9991}
  The function
\begin{gather*}
  \Phi
  \ 
  :
  \ 
  \R^n
  \to
  \ 
  \overline{\R}
  \,
  ,
  \qquad
  [w_1,\ldots,w_n]^\top
  \ 
  \mapsto
  \ 
  \sum_{i=1}^n \varphi(w_i)
  \,,
\end{gather*}
satisfies Assumption~\ref{cv:ts:asu}.
\end{lemma}
\begin{proof}
  By Example~\ref{cv:cc:ex}
  the convex conjugate of $\Phi$ is 
\begin{gather*}
  \Phi^*
  \ 
  :
  \ 
  \R^n
  \to
  \ 
  \R
  \,
  ,
  \qquad
  [\lambda_1,\ldots,\lambda_n]^\top
  \ 
  \mapsto
  \ 
  \sum_{i=1}^n \varphi^*(\lambda_i)
  \,,
\end{gather*}
where $\varphi^*$ is the convex conjugate of $\varphi$.
By Lemma~\ref{lem:obj_f}, $\varphi$ is strictly convex.
Thus,
$\Phi$ is strictly convex. By Lemma~\ref{1165}, $\varphi^*$ continuously differentiable on $\R$. Thus,
$\Phi$ is continuously differentiable on $\R^n$.
It follows the statement of Assumption~\ref{cv:ts:asu} for $\Phi$.
\end{proof}
\begin{takeaways}
  The choice of Definition~\ref{def:obj_f} 
  introduces the sample variance to Problem~\ref{bw:1:primal}.
  It has good practical and theoretical properties.
  Among the theoretical are strict convexity that allows linking Problem~\ref{bw:1:primal}
  to the theory of convex analysis.
\end{takeaways}

\section{Dual Problem}
In the previous section we have expounded our choice of $\varphi$ - and with it the objective function of Problem~\ref{bw:1:primal}.
%
Now, we want to apply Theorem~\ref{cv:ts:th} to Problem~\ref{bw:1:primal}.
%
To this end, we provide its proper formulation.
%
\begin{lemma}
  \label{matrix_notation}
  A matrix formulation of Problem~\ref{bw:1:primal} is 
\begin{align}
  \label{cv:ts:primal}
  %%%% objective %%%%
    &\underset{w \in \R^n}
    {\mathrm{minimize}}
    &&\qquad\qquad
    \Phi(w)
    &&&
    \\
    %%%% Ax >= b %%%%
    \nonumber
    &\mathrm{subject}\ \mathrm{to} 
    &&\qquad\qquad
    \mathbf{U}w
    \ 
    \ge
    \ 
    d
    \,,
    \\
    \nonumber
    &
    &&\qquad\qquad
    \mathbf{A}w
    \ 
    =
    \ 
    a
    \,,
\end{align}
with objective function
\begin{gather*}
  \Phi
  \ 
  :
  \ 
  \R^n
  \to
  \ 
  \overline{\R}
  \,
  ,
  \qquad
  [w_1,\ldots,w_n]^\top
  \ 
  \mapsto
  \ 
  \sum_{i=1}^n \varphi(w_i)
  \,,
\end{gather*}
inequality matrix and vector
\begin{alignat*}{2}
    \mathbf{U}
    &
    \ 
    :=
    \ 
    \begin{bmatrix}
      \mathbf{I}_n
      \\
      \pm\,\mathbf{B}(\mathbf{X})
    \end{bmatrix}
    \in
    \R^{(n+  2 N)\times n}
        \qquad
    &&
d
    \ 
    :=
    \ 
    \begin{bmatrix}
      0_n
      \\
      -N\cdot\delta 
      \ 
      \pm\ 
      \sum_{i = 1}^{N} B(X_i)
    \end{bmatrix}
    \in
    \R^{n+  2 N}
    \,,
    \intertext{and equality matrix and vector}
    \mathbf{A}
    &
    \ 
    :=
    \ 
      \mathrm{1}_n
      ^\top
      \in\R^{1\times n}
      \qquad
    &&
    a
  \ 
    :=
    \ 
    N
    \in\mathbb{N}
    \,.
\end{alignat*}
\end{lemma}

\begin{proof}
  Recall that the box constraints of Problem~\ref{bw:1:primal} are
  \begin{gather*}
        \left| 
      \frac{1}{N} 
      \left( 
      \sum_{i = 1}^{n} 
      w_i
      B_k(X_i)
      -
      \sum_{i=1}^{N} 
      B_k(X_i)
      \right)
    \right|
    \ 
    \le 
    \ 
    \delta_k
    \qquad
    \text{for all}\ 
    k\in \left\{ 1,\ldots, N \right\}
    \,.
  \end{gather*}
  Put differently, it holds both
  \begin{align*}
    -
      \sum_{i = 1}^{n} 
      w_i
      B_k(X_i)
    \ge 
    -
    N
    \delta_k
      -
      \sum_{i=1}^{N} 
      B_k(X_i)
      \quad 
    \text{and}
      \quad
      \sum_{i = 1}^{n} 
      w_i
      B_k(X_i)
    \ge 
    -
    N
    \delta_k
      +
      \sum_{i=1}^{N} 
      B_k(X_i)
  \end{align*}
  for all 
  $
    k\in \left\{ 1,\ldots, N \right\}
  $. In matrix notation this is 
  \begin{gather*}
    \pm\mathbf{B}(\mathbf{X})w
    \ 
    \ge
    \ 
    [d_{n+1},\ldots, d_{n+  2 N}]^\top
    \,.
  \end{gather*}
  Proving the rest of the statements is straightforward. We omit the details.
\end{proof}
\begin{remark}
  The inequality constraints of
  Lemma~\ref{matrix_notation} differ from its counterpart
  \cite[Proof of Lemma~1]{Wang2019}.
  We don't transform the variable $w$, but shift to $d$ what prevents us from keeping $w$.
  Note, that the choice of
  \cite[Proof of Lemma~1]{Wang2019} leads to a mistake on page 21.
  The mistake is most obvious in the second display, where the first implication follows from dividing by 0.
  I discussed this with the authors and proposed a version of Lemma\ref{matrix_notation} to solve the problem. I think it's best not to transform variables, because the mistake comes from (wrongly) calculating the convex conjugate of the (more complicated) transformed version of the objective function. The subsequent analysis even simplifies with my version.

  I was surprised to find the (exact) same mistake in the earlier paper 
  \cite[page 35 second display]{Chan2016}. 
  There is no reference in
  \cite[Proof of Lemma~1]{Wang2019} 
  to
  \cite{Chan2016}. Yet the formulation and the mistake are the same.
  Did the authors of \cite{Wang2019} (inadvertently?) plagiarize
  the mathematical analysis of 
  \cite{Chan2016}
  ?
\end{remark}

\begin{lemma}
  Consider the optimization problem
\begin{align}
  \label{9993}
  \begin{split}
  \underset
  {\begin{smallmatrix}
\rho\,,\, \lambda^+,\,\lambda^-\ge 0 \\
\lambda_0\in\R
  \end{smallmatrix}}
  {
    \mathrm{maximize}
  }
  \quad
  &
  -
\sum_{i=1} 
  ^n
    \,
  \varphi^*
  \!
  \left( 
    \rho_i
    +
\lambda_0
+
\inner
{B(X_i)}
{
\lambda^+
-
\lambda^-
}
  \right)
  \\
  &
+
\ 
\sum_{i=1}^{N} 
  \left( 
\lambda_0
+
\inner
{B(X_i)}
{
\lambda^+
-
\lambda^-
}
  \right)
  \,
  \ 
-
\ 
\inner
{\delta}
{
\lambda^+
+
\lambda^-
}
  \,.
  \end{split}
\end{align}
If Assumption~\ref{asu:objective_f} holds true 
and there exists an optimal solution 
$
(\rho^\dagger,\lambda_0^\dagger,\lambda^{+,\dagger},\lambda^{-,\dagger})
$
then the unique optimal solutions to Problem~\ref{bw:1:primal} are 
\begin{gather*}
  w^\dagger_i
  \ 
  :=
  \ 
  (
  \varphi^{'}
  )^{-1}
  \left(
    \rho^\dagger_i
  \ 
    +
  \ 
\lambda_0^\dagger
  \ 
+
  \ 
\inner
{B(X_i)}
{
  \lambda^{+,\dagger}
-
\lambda^{-,\dagger}
}
  \right)
  \qquad
  \text{for all}\ 
  i\in
  \left\{ 1,\ldots,n \right\}
  \,.
\end{gather*}
\end{lemma}
\begin{proof}
  By Lemma~\ref{matrix_notation},
  Problem~\ref{bw:1:primal} has the form required in Theorem~\ref{cv:ts:th}.
  By Assumption~\ref{asu:objective_f} and Lemma~\ref{9991} the objective function $\Phi$ of Problem~\ref{bw:1:primal}
  satisfies Assumption~\ref{cv:ts:asu}.
  Thus we can apply
  Theorem~\ref{cv:ts:th} to Problem~\ref{bw:1:primal}.
  Calculations yield the result.
\end{proof}
The next Theorem aims at simplifying this result. 
\newpage
\begin{ftheorem}
  \label{dual_solution_th}
  Consider the optimization problem
\begin{align}
  \label{dual}
  \begin{split}
  \underset
  {\begin{smallmatrix}
      \rho&\in&&\R^N 
      \\
      \lambda_0 & \in&&\R
      \\
      \lambda&\in&&\R^{N}
  \end{smallmatrix}}
  {
    \mathrm{minimize}
  }
  \quad
  \frac{1}{N}
\sum_{i=1} 
  ^N
  &
  \Big[
  T_i
  \cdot
  \varphi^*
  \!
  \left( 
    \rho_i
    +
\lambda_0
+
\inner
{B(X_i)}
{
\lambda
}
  \right)
  \ 
  -
  \ 
\lambda_0
-
\inner
{B(X_i)}
{
\lambda
}
\Big]
  \ 
+
\ 
\inner
{\delta}
{
  |\lambda|
}
  \,,
  \\
  \mathrm{subject}\ \mathrm{to}
  \quad
  \qquad
  &
  \rho_i \ge 0 
  \quad 
  \mathrm{for}\ \mathrm{all}\ i\le n
  \qquad 
  \mathrm{and}
  \qquad
  \rho_i=0
  \quad 
  \mathrm{for}\ \mathrm{all}\ i>n
  \,.
\end{split}
\end{align}
If Assumption~\ref{asu:objective_f} holds true 
and there exists an optimal solution 
$
(\rho^\dagger,\lambda_0^\dagger,\lambda^\dagger)
$
then the unique optimal solutions to Problem~\ref{bw:1:primal} are 
\begin{gather*}
  w^\dagger_i
  \ 
  :=
  \ 
  (
  \varphi^{'}
  )^{-1}
  \left(
    \rho^\dagger_i
  \ 
    +
  \ 
\lambda_0^\dagger
  \ 
+
  \ 
\inner
{B(X_i)}
{
\lambda^{\dagger}
}
  \right)
  \qquad
  \text{for all}\ 
  i\in
  \left\{ 1,\ldots,n \right\}
  \,.
\end{gather*}
\end{ftheorem}

\begin{proof}
  Assume that
$
  (\rho^\dagger,\lambda_0^\dagger,\lambda^{+,\dagger},\lambda^{-,\dagger})
$
is an optimal solution to Problem~\ref{9993}.
We write
\begin{align*}
  G
  (\rho,\lambda_0,\lambda^+,\lambda^-)
  &
  \ 
  :=
  \ 
 -
\sum_{i=1} 
  ^n
    \,
  \varphi^*
  \!
  \left( 
    \rho_i
    +
\lambda_0
+
\inner
{B(X_i)}
{
\lambda^+
-
\lambda^-
}
  \right)
  \\
  &
  \qquad
+
\ 
\sum_{i=1}^{N} 
  \left( 
\lambda_0
+
\inner
{B(X_i)}
{
\lambda^+
-
\lambda^-
}
  \right)
  \,
  \ 
-
\ 
\inner
{\delta}
{
\lambda^+
+
\lambda^-
}
  \,.
\end{align*}
 To eliminate the remaining constraints, 
  we paraphrase \cite[pages~19-20]{Wang2019}.
  We show 
  for all $i \in \left\{ 1,\ldots,N \right\}$
\begin{alignat}{2}
  \notag
  \text{either}
  &
  &&
  \qquad
  \lambda_i^{+,\dagger} > 0
  \\
  \label{9992}
  \text{or}
  &
  &&
  \qquad
  \lambda_i^{-,\dagger} > 0
  \,.
\end{alignat}
Assume towards a contradiction that 
\begin{gather}
  \label{1232}
  \text{
there exists
  } 
  \ 
i \in \left\{ 1,\ldots,N \right\}
\ 
\text{such that}
\qquad
  \lambda_i^{+,\dagger} > 0
  \qquad 
  \text{and}
  \qquad
  \lambda_i^{-,\dagger} > 0
  \,.
\end{gather}
Consider
  \begin{align*}
    \tilde{\lambda}^{+,\dagger}
    &
    \ 
    :=
    \ 
    \begin{bmatrix}
      \ 
      \lambda_1^{+,\dagger}
      \ldots,
      \ 
      \lambda_i^{+,\dagger}
      \!
      \ 
      -
      \ 
      (
      \lambda_i^{+,\dagger}
      \!
      \land
      \lambda_i^{-,\dagger}
      )\,,
      \ 
      \ldots,
      \lambda_{N}^{+,\dagger}
    \end{bmatrix}
    ^\top
    \intertext{and}
    \tilde{\lambda}^{-,\dagger}
    &
    \ 
    :=
    \ 
    \begin{bmatrix}
      \ 
      \lambda_1^{-,\dagger}
      \ldots,
      \ 
      \lambda_i^{-,\dagger}
      \!
      \ 
      -
      \ 
      (
      \lambda_i^{+,\dagger}
      \!
      \land
      \lambda_i^{-,\dagger}
      )\,,
      \ 
      \ldots,
      \lambda_{N}^{-,\dagger}
    \end{bmatrix}
    ^\top
    \,.
  \end{align*}
  Since
  \begin{gather*}
      \lambda_i^{\pm,\dagger}
      \!
      \ 
      -
      \ 
      (
      \lambda_i^{+,\dagger}
      \!
      \land
      \lambda_i^{-,\dagger}
      )
      \ 
      \ge 
      \ 
      0
      \,,
  \end{gather*}
  the perturbed vectors $\tilde{\lambda}^{\pm,\dagger}$ are  in the domain of the 
  optimization problem.
  By Assumption~\eqref{1232} and $\delta>0$ it follows
  \begin{align*}
  G
  \left( 
  \rho^\dagger,\lambda_0^\dagger,\tilde{\lambda}^{+,\dagger},\tilde{\lambda}^{-,\dagger}
  \right)
  \ 
  -
  \ 
  G
  \left( 
  \rho^\dagger,\lambda_0^\dagger,\lambda^{+,\dagger},\lambda^{-,\dagger}
  \right)
  \ 
  =
  \ 
  2
  \cdot
  \delta_i
  \cdot
      (
      \lambda_i^{+,\dagger}
      \!
      \land
      \lambda_i^{-,\dagger}
      )
  \ 
  >
  \ 
  0
  \,,
  \end{align*}
  which contradicts the optimality of
$
  (\rho^\dagger,\lambda^{+,\dagger},\lambda^{-,\dagger},\lambda_0^\dagger)
$
(it is supposed to be a maximum in the domain of the optimization problem)
.
It follows \eqref{9992}.
But then 
$
\lambda^{\pm,\dagger}_i
\ge 0
$
collapses to
$
\lambda_i^\dagger\in \R
$ 
for all
$i\in \left\{ 0,\ldots,N \right\}$, that is, we set
\begin{gather*}
 \lambda_i^\dagger
 \ 
 =
 \ 
 \lambda_i^{+,\dagger}
 \ 
 -
 \ 
 \lambda_i^{-,\dagger}
 \qquad
 \text{and}
 \qquad
|\lambda_i^\dagger|
\ 
=
\ 
\lambda_i^{+,\dagger}
\ 
+
\ 
\lambda_i^{-,\dagger}
\,.
\end{gather*}
Thus, we can extend the domain of Problem~\ref{9993} to $\lambda\in\R^{N}$ and update the objective function in the following way
(without changing the optimal solution).
\begin{align*}
  G
  (\rho,\lambda_0,\lambda)
  &
  \ 
  :=
  \ 
 -
\sum_{i=1} 
  ^n
    \,
  \varphi^*
  \!
  \left( 
    \rho_i
    +
\lambda_0
+
\inner
{B(X_i)}
{
\lambda
}
  \right)
  \\
  &
  \qquad
+
\ 
\sum_{i=1}^{N} 
  \left( 
\lambda_0
+
\inner
{B(X_i)}
{
\lambda
}
  \right)
  \,
  \ 
-
\ 
\inner
{\delta}
{
  |\lambda|
}
  \,.
\end{align*}
Multiplying $G$ with $-1/N$ doesn't change the solution either
(if we search instead for a minimum).
To finish the proof, we choose the notation with $T_i$ instead of $n$. This extends the domain of $\rho$ to $\R^N_{\ge 0}$, but the 
new $\rho_i$ are not effective because of $T_i=0$ for all $i>n$. 
Thus we may set them to 0.
\end{proof}


%Bring optimal solutions together. with argmax meas. and Proposition for aSsumption and assumption.

%It is useful too define a weights function
%New chapter?

%Before we define the weights function we specify the basis $\mathfrak{B}$.
%\subsection{basis}
%Let $
\left(
\mathcal{P}_N
\right)
$
denote a sequence of countable, $\mathcal{B}$-measurable partitions 
\begin{align*}
\mathcal{P}_N= \left\{
  A_{N,1},
  A_{N,2},
  \ldots
\right\}
\subset \mathcal{B}(\R^d)
\end{align*}
of $\R^d$, that is, 
\begin{align*}
  A_{N,i}\cap A_{N,j}=\emptyset
  \qquad
  \text{if}\ i\neq j
  \qquad
  \text{and}
  \qquad 
  \bigcap_{i\in\mathbb{N}}A_{N,i}
  \ 
  =
  \ 
  \R^d
  \,.
\end{align*}
We define
$ A_N(x) $ to be the cell of $ \mathcal{P}_N $ containing $x$, that is,
\begin{align*}
  A_N
  \colon
  \R^d 
  \ 
  \twoheadrightarrow 
  \ 
  \R^d  
  \,,\qquad
  x
  \ 
  \mapsto
  \ 
  A_N(x)
  \,,
\end{align*}
where $A_N(x)$ is the only cell containing $x$. 

Next, we define the (empirical) basis vector
\begin{align*}
  B\colon
  \R^d\times \R^{d\cdot N}
  \to
  \R
  \,,
  \qquad
  (x,(x_1,\ldots,x_N))\mapsto
  \frac
  {
    \left[
    \mathbf{1}
    _{
      A_N(x)
    }
    (x_k)
    \right]
    _{k\in \left\{
        1,\ldots,N
    \right\}}
  }
  {
    \sum_{j=1}^N
    \mathbf{1}
    _{
      A_N(x)
    }
    (x_j)
    }
  \,,
\end{align*}
where we keep to the convention $"0/0=0"$.
\begin{lemma}
  \quad
  \begin{enumerate}[label=(\roman*)]
\item
  $B(\cdot,(X_1,\ldots,X_N))(\omega)$ is 
  $\left(
    \mathcal{B}(\R^d),\mathcal{B}(\R^N)
  \right)$-measurable
  and
  constant on each cell $A_N\in\mathcal{P}_N$
  for all $\omega\in\Omega$. 
\item
  $B(X,(X_1,\ldots,X_N))$ is $\left(
    \mathcal{A},\mathcal{B}(\R^N)
  \right)$-measurable. 
  \end{enumerate}
\end{lemma}

\begin{proof}
Consider
for $k\in \left\{
  1,\ldots,N
\right\}$
and $\omega\in\Omega$
the indicator function
\begin{align*}
  \mathbf{1}
  _
  {A_N(X_k(\omega))}
  \colon \R^d\to \left\{
    0,1
  \right\}
  \,.
\end{align*}
Since 
$
  {A_N(X_k(\omega))}
  \in\mathcal{B}(\R^d)
$
this is a 
  $\left(
    \mathcal{B}(\R^d),\mathcal{B}(\R^N)
  \right)$-measurable
  function.
  Since it is 1 if $
  x\in
  {A_N(X_k(\omega))}
  $
  and 0 else, it is also constant on each cell. It follows statement (i).
Since
\begin{align*}
  \mathbf{1}
  _
  {A_N(X_k(\omega))}(X(\omega))
  \ 
  =
  \ 
  \mathbf{1}
  \bigcup_{i\in\mathbb{N}}
  \left\{
    X,X_k \in A_{N,i}
  \right\}
  (\omega)
  \qquad
  \text{for all}\ 
  \omega\in\Omega
  \,,
\end{align*}
and
$
  \bigcup_{i\in\mathbb{N}}
  \left\{
    X,X_k \in A_{N,i}
  \right\}
  \in\mathcal{A}
$
it follows statement (ii).
\end{proof}

%\subsection{Weights Function}
%Theorem~\ref{dual_solution_th} tells us that if an optimal solution
$
(\rho^\dagger,\lambda_0^\dagger,\lambda^\dagger)
$
to Problem~\ref{dual} exists,
then the unique optimal solutions to Problem~\ref{bw:1:primal} are 
\begin{gather*}
  w^\dagger_i
  \ 
  :=
  \ 
  (
  \varphi^{'}
  )^{-1}
  \left(
    \rho^\dagger_i
  \ 
    +
  \ 
\lambda_0^\dagger
  \ 
+
  \ 
\inner
{B(X_i)}
{
\lambda^{\dagger}
}
  \right)
  \qquad
  \text{for all}\ 
  i\in
  \left\{ 1,\ldots,n \right\}
  \,.
\end{gather*}

This point of view is sufficient from a practical point of view
(when we are interested in actually computing the optimal weights).

For the subsequent analysis we need to view the weights as random quantities. To this end, we shall introduce the weights process. 

\begin{definition}
  \label{def:weights_function}
  We define the weights process to be a stochastic process indexed 
  over
  $
  \mathcal{X}
  \,
  \times
  \,
  \R^N_{\ge 0}
  \,
  \times
  \,
  \R
  \times
  \,
  \R^{N}
  $
  with values in $\R^N$ such that
\begin{align*}
  w
  \left( 
  x
  ,
  \rho
  ,
  \lambda_0
  ,
  \lambda
  \right)
  \ 
  :=
  \ 
  \left[ 
    (
    \varphi^{'}
    )^{-1}
    \left( 
     \rho_i 
     +
      \lambda_0
      +
      \inner{B(x)}{\lambda}
    \right)
  \right]_{i\in \left\{ 1, \ldots,N \right\}}
    \,.
\end{align*}
\end{definition}
By the measurability of the basis functions (or the basis processes),
the weights processes is also measurable 
(the quantities 
$
  w
  \left( 
  x
  ,
  \rho
  ,
  \lambda_0
  ,
  \lambda
  \right)
$
are random variables).
The next question is if there are (plausible) assumptions such that the weights processes at a random parameter
$
(\rho^\dagger,\lambda_0^\dagger,\lambda^\dagger)
$
is measurable.
To answer this in a good way we need the argmax measurability theorem \cite[Theorem~18.19]{Aliprantis2007}.
\subsection{Argmax Measurability Theorem}
We follow \cite{Aliprantis2007}
A \textbf{correspondence}$ \psi$ from a set $X$ to a set $Y$ assigns to each $x\in X$ a subset $\psi(x)\subset Y$.
To clarify that we map $x$ to a set, we use the double arrow, that is,
$
  \psi
  \colon
  X
  \twoheadrightarrow
  Y
$.
  Let 
  $(S,\Sigma_S)$ be a measurable space and $X$ a topological space.
  We say, that a correspondence 
  $
  \psi
  \colon
  S
  \twoheadrightarrow
  X
  $
  is 
  \textbf{
  weakly measurable
  },
  if
  \begin{gather*}
    \left\{ 
      s\in S
      \ 
      |
      \ 
      \psi(s)
      \cap
      O
      \neq
      \emptyset
    \right\}
    \in
    \Sigma_S
    \qquad
    \text{for all open subsets}
    \ 
    O\subset X
    \,.
  \end{gather*}

A selector from a correspondence $\psi\colon X\twoheadrightarrow Y$ is a function $s\colon X\to Y$ that satisfies 
$
s(x)\in\psi(x)
$
for all $x\in X$.

\begin{definition}
  Let 
  $(S,\Sigma_S)$ be a measurable space, and let $X$ and $Y$  be topological space.
  A function 
  $f\colon S\times X \to Y$
  is a \textbf{Caratheodory function} if
  \begin{align*}
    f(\cdot,x)
    &
    \colon
    S\to Y
    \qquad
    \text{is}\ 
    (\Sigma_{S},\mathcal{B}(Y))-measurable
    \ 
    \text{for all}
    \ 
    x\in X
    \,,
    \intertext{and}
    f(s,\cdot)
    &
    \colon
    X\to Y
    \qquad
    \text{is continuous for all}\ 
    s\in S
    \,.
  \end{align*}
\end{definition}
\begin{theorem}
  Let $X$ be a separable metrizable space and
  $
  (S,\Sigma_S)
  $
  a measurable space.
  Let $\psi\colon S \twoheadrightarrow X$ be a weakly measurable correspondence with non-empty compact values, and suppose
  $f\colon S\times X \to \R$
  is a Caratheodory function. Define the value function 
  $m\colon S\to \R$ by
  \begin{gather*}
    m(s):=\max_{\psi(s)}f(s,x)
    \,,
  \end{gather*}
  and the correspondence 
  $mu\colon S\twoheadrightarrow X$ of maximizers by
  \begin{gather*}
    \mu(s):= \left\{ 
      x\in \psi(s)
      |
      f(s,x)=m(s)
    \right\}
    \,.
  \end{gather*}
  Then the value function $m$ is measurable, 
  the argmax correspondence $\mu$ has non-empty and compact values,
  is measurable and admits a measurable selector.
\end{theorem}
\begin{proof}
  \cite[Theorem~18.19]{Aliprantis2007}
\end{proof}

Our goal is to find (plausible) assumptions under which we can measurably select optimal solutions of Problem~\ref{dual}.

\begin{assumption}
  There exists $\underline{N}\in\mathbb{N}$ such that 
  for all $N\ge \underline{N}$ 
  there exists a compact and deterministic parameter space
  $
  \Theta_N
  \subset
  \R^N_{\ge 0}
  \times
  \R
  \times
  \R^{N}
  $
  such that for all data sets $D_N$
  a sequence of optimal solution
  $(\rho^\dagger,\lambda_0^\dagger,\lambda^\dagger)$
  satisfying the Lemma
exist and it holds
\begin{gather*}
  (
  \rho^\dagger,
  0_{N-n},
  \lambda_0^\dagger,\lambda^\dagger)
  \in
  \Theta_N
  \,.
\end{gather*}
\end{assumption}

With this assumption we can construct the (constant) correspondence
\begin{align*}
  \psi
  \colon
  (\Omega,\mathcal{A},\P)
  \ 
  \twoheadrightarrow
  \ 
  \R^N_{\ge 0}
  \times
  \R
  \times
  \R^{N}
  \,,
  \qquad
  \omega
  \ 
  \to
  \ 
  \Theta_N
  \,.
\end{align*}
This is weakly measurable, because $\Theta_N$ is deterministic and thus
$\Theta_N\cap O$ is either true or false, that is,
  \begin{gather*}
    \left\{ 
      \omega\in \Omega
      \ 
      |
      \ 
      \psi(\omega)
      \cap
      O
      \neq
      \emptyset
    \right\}
    \ 
    \in
    \ 
    \left\{ \Omega,\emptyset \right\}
    \subset
    \Sigma_S
    \qquad
    \text{for all open subsets}
    \ 
    O
    \subset
  \R^N_{\ge 0}
  \times
  \R
  \times
  \R^{N}
    \,.
  \end{gather*}
  Furthermore the objective function is a Caratheodory function.
  Thus, by Theorem there exists the argmax correspondence  of Problem~\ref{dual} and a measurable selector that selects
  $(\rho^\dagger,0_{N-n},\lambda_0^\dagger,\lambda^\dagger)$ 
  of Assumption.
  With this, we can define the optimal weights processes
  \begin{gather*}
    w^\dagger(x)
    \ 
    :=
    \ 
    w
    \left( 
    x,\rho^\dagger,0_{N-n},\lambda_0^\dagger,\lambda^\dagger
    \right)
    \qquad
    \text{indexed over}\ 
    x\in\mathcal{X}\,.
  \end{gather*}
  Due to the definition of the (general) weights processes and the measurability of the argmax selector, the random variables $w^\dagger(x)$ are measurable for all $x\in\mathcal{X}$.
  To end the measurability discussion, note that $w^\dagger(X)$ is a random variable.


\begin{lemma}
  \label{weights_l_inf}
  Let Assumption~\ref{asu:existence_sol}, 
  Assumption~\ref{asu:basis},
  and Assumption~\ref{asu:objective_f} hold true.
  Then it holds 
  \begin{gather*}
    w_i^\dagger(X)
  \ 
  \in
  \ 
  L^\infty(\P)
  \qquad
  \text{for all}
  \ 
  i\in \left\{ 1,\ldots,N \right\}
  \ 
  \text{
    and
  for all 
  }\ 
  N\ge \underline{N}
  \,.
  \end{gather*}
\end{lemma}
\begin{proof}
  Let $N\ge \underline{N}$.
  By Assumption~\ref{asu:basis} it holds 
  $
    \text{for all}\ 
  $ 
  \begin{gather}
    \label{3967}
    \left| 
     \rho_i 
     +
      \lambda_0
      +
      \inner{B(x)}{\lambda}
    \right|
    \ 
    \lesssim
    \ 
    \norm{(\rho,\lambda_0,\lambda)}_2
    \,.
  \end{gather}
  $
    \text{for all}
    \ 
    x\in\mathcal{X}
    \
    \text{and for all}\ 
    (\rho,\lambda,\lambda_0)
    \in
  \R^N_{\ge 0}
  \times
  \R
  \times
  \R^{N}
  $.
  By Assumption~\ref{asu:existence_sol}, 
  there exists a (compact) parameter space
  $\Theta_N$ around the origin, with $\mathrm{diam}\,  \Theta_N<\infty$, 
  such that for all data sets $D_N$ it holds  $(\rho^\dagger,\lambda^\dagger,\lambda_0^\dagger)\in\Theta_N$.
  By Assumption~\ref{asu:objective_f}, $(\varphi^{'})^{-1}$ is non-decreasing and continuous. Thus
    by \eqref{3967} it holds
  \begin{align*}
    \left| 
    w_i^\dagger(x)
    \right|
    \ 
    \lesssim
    \ 
    \left| 
    (\varphi^{'})^{-1}
    \left( 
      \,
      -
    \norm{(\rho^\dagger,\lambda_0^\dagger,\lambda^\dagger)}_2
    \right)
    \right|
    \ 
    +
    \ 
    \left| 
    (\varphi^{'})^{-1}
    \left( 
      \,
    \norm{(\rho^\dagger,\lambda_0^\dagger,\lambda^\dagger)}_2
    \right)
    \right|
    \ 
    \ 
    <
    \ 
    \infty
  \end{align*}
  for all $x\in\mathcal{X}$
  and all $i\in \left\{ 1,\ldots,n \right\}$
  .
\end{proof}
\begin{lemma}
  \label{w.Z=0}
  Let Assumption~\ref{aa:assumption:2} and Assumption~\ref{aa:assumption:3} hold true.
  Furthermore, 
  let
  $N\ge\underline{N}$, and
  let
  $Z\in L^1(\P)$
  be a random variable that is independent of $D_N$ 
  with
  $
\E
\left[
  Z
  \,
  |
  \, 
  X
\right]
= 0
  $
  almost surely.
  It holds
  \begin{gather*}
  \E
  \left[
  w_i(X,\rho^\dagger,\lambda^\dagger,\lambda_0^\dagger)
  \cdot Z
  \right]
  =0
  \qquad
  \text{for all}
  \ 
  i\in \left\{ 1,\ldots,n \right\}
  \,.
  \end{gather*}
\end{lemma}
\begin{proof}
  Let
  $N\ge\underline{N}$.
  We write
  \begin{gather*}
  w_i(X,\rho^\dagger,\lambda^\dagger,\lambda_0^\dagger)
  \ 
  =
  \ 
  w^\dagger(X)
  \end{gather*}
  and ignore the index $i$.
  By Lemma~\ref{weights_l_inf} and 
  $Z\in L^1(\P)$
  it holds
  \begin{gather}
    \label{9876}
    \norm{
  w^\dagger(X)\cdot Z
    }_{L^1(\P)}
    \ 
  \le
    \ 
  \norm{w^\dagger(X)}_{L^\infty(\P)}
  \norm{Z}_{L^1(\P)}
  \ 
  <
  \ 
  \infty
  \,.
  \end{gather}
  By 
  \eqref{9876},
  $Z\perp D_N$
  and
  $
\E
\left[
  Z
  \,
  |
  \, 
  X
\right]
= 0
  $
  almost surely
  it holds 
  \begin{align*}
    \E
  \left[
  w^\dagger(X)
  \cdot
  Z
  \,
  |
  \,
  D_N,X
  \right]
  &
  \ 
  =
  \ 
  w^\dagger(X)
  \cdot
  \E
  \left[
  Z
  \,
  |
  \,
  D_N,X
  \right]
  \\
  &
  \ 
  =
  \ 
  w^\dagger(X)
  \cdot
  \E
  \left[
  Z
  \,
  |
  \,
  X
  \right]
  \
  =
  \ 
  0
  \qquad
  \text{almost surely.}
  \end{align*}
  Thus
  \begin{gather*}
    \E
    \left[
  w^\dagger(X)
  \cdot
  Z
  \,
    \right]
    \ 
    =
    \ 
    \E
    \left[
 \E
  \left[
  w^\dagger(X)
  \cdot
  Z
  \,
  |
  \,
  D_N,X
  \right]
    \right]
    \ 
    =
    \ 
    0
    \,.
     \end{gather*}
\end{proof}



