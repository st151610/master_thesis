Let $(T_1,X_1),\ldots,(T_N,X_N)$ be independent and identically-distributed copies of $T$ and $X$ (see Introduction page). 
We gather them in the (random) data set 
\begin{gather*}
D_N:=\left\{ (T_i,X_i)\colon i\in \left\{ 1,\ldots,N \right\} \right\}
\,.
\end{gather*}
Furthermore, let
\begin{gather*}
  n
  \ 
  :=
  \ 
  \# 
  \left\{ 
    i\in \left\{ 1,\ldots,N \right\}
    \colon
    T_i=1
  \right\}
\end{gather*}
be the number of treated units. This is a random variable. We assume the order $T_i=1$ for all $i\le n$.

Let
$\# A\in\mathbb{N}_0$ be the number of elements of a finite set $A$, let
$\overline{\R}:=\R\cup \left\{ \infty \right\}$ 
be the extended real numbers, 
$\Sigma_\mathcal{X}$ be a $\sigma$-algebra on the covariate space $\mathcal{X}$,
and let $\mathcal{B}(\R)$ denote the Borel-$\sigma$-algebra on the real numbers.

For a convex function $\varphi\colon \R\to\overline{\R}$,
a finite set of measurable (basis-)functions on the covariate space
\begin{gather*}
  \mathfrak{B}
  :
  =
  \left\{ 
B_k\colon (\mathcal{X},\Sigma_\mathcal{X})\to(\R,\mathcal{B}(\R))
\ 
|
\ 
k\in \left\{ 1,\ldots,\# \mathfrak{B} \right\}
  \right\}
\end{gather*}
and a constraints vector $\delta=[\delta_1,\ldots,\delta_{\# \mathfrak{B}}]^\top>0$
we consider the following (random) convex optimization problem.
\newpage
\begin{fproblem}
  \label{bw:1:primal}
\begin{align*}
  %%%% objective %%%%
    &\underset{w_1, \ldots, w_n \in \R}
    {\text{minimize}}
    &&\qquad\qquad
    \sum_{i = 1}^{n} 
    \varphi(w_i)
    &&&
    \\
    %%%% w_i T_i >= 0 %%%%
    &\text{subject to}
    &&\qquad\qquad
    w_i 
    \ge
    0
    &&&
    \qquad
    \text{for all}\ 
    i \in \left\{ 1, \ldots, n \right\}
    \,,
    \\
    %%%% 1/n sum w = 1 %%%%
    & 
    &&\qquad\qquad
    \frac{1}{N}
    \sum_{i=1}^{n} 
    w_i
    =1
    \\
    %%%% box constraints %%%%
    & 
    &&\qquad
    \left| 
      \frac{1}{N} 
      \left( 
      \sum_{i = 1}^{n} 
      w_i
      B_k(X_i)
      -
      \sum_{i=1}^{N} 
      B_k(X_i)
      \right)
    \right|
    \ 
    \le 
    \ 
    \delta_k
    &&&
    \qquad
    \text{for all}\ 
    k \in \left\{ 1, \ldots, \# \mathfrak{B} \right\}
    \,.
\end{align*}
\end{fproblem}
What is random in Problem~\ref{bw:1:primal}?
First, the dimension of the search space $(w\in\R^n)$ depends on the random variable $n$. 
Thus, we only compute weights for the treated units (the ones with $T_i=1$).
Next consider the \textbf{objective function}
\begin{gather*}
    \sum_{i = 1}^{n} 
    \varphi(w_i)
    \,.
\end{gather*}
The number of summands is random (again $n$). We sometimes use the equivalent notation
\begin{gather*}
    \sum_{i = 1}^{N} 
    T_i
    \cdot
    \varphi(w_i)
    \,,
\end{gather*}
where we set the weights of the untreated (the ones with $T_i=0$) to some arbitrary value in the domain of $\varphi$.
Let's consider the constraints. There is no randomness in the first two constraints.
\begin{gather*}
    w_i 
    \ge
    0
    \qquad
    \text{for all}\ 
    i \in \left\{ 1, \ldots, n \right\}
    \quad
    \text{and}
    \quad
    \frac{1}{N}
    \sum_{i=1}^{n} 
    w_i
    =1
    \,.
\end{gather*}
They only make sure, that the weights (divided by $N$) form a convex combination.
If, for example, the outcome space $\mathcal{Y}$ is convex we make sure that a weighted-mean-estimate of $\E[Y(1)]$ satisfies
\begin{gather*}
  \widehat{Y}(1) 
  \ 
  :=
  \ 
  \frac{1}{N}
  \sum_{i=1}^{n} 
  w_i\cdot Y_i
  \ 
  \in
  \ 
  \mathcal{Y}
\end{gather*}
or that a weighted-mean-estimate of the distribution function of $Y(1)$ satisfies
\begin{gather*}
  \widehat{F}_{Y(1)} 
  \ 
  :=
  \ 
  \frac{1}{N}
  \sum_{i=1}^{n} 
  w_i\cdot \mathbf{1}\left\{ Y_i\le z \right\}
  \ 
  \in
  \ 
  [0,1]
  \,.
\end{gather*}
There remain the box constraints
\begin{gather*}
    \left| 
      \frac{1}{N} 
      \left( 
      \sum_{i = 1}^{n} 
      w_i
      B_k(X_i)
      -
      \sum_{i=1}^{N} 
      B_k(X_i)
      \right)
    \right|
    \ 
    \le 
    \ 
    \delta_k
    \qquad
    \text{for all}\ 
    k \in \left\{ 1, \ldots, \# \mathfrak{B} \right\}
    \,.
\end{gather*}
Here the number of summands in
\begin{gather*}
      \sum_{i = 1}^{n} 
      w_i
      B_k(X_i)
\end{gather*}
is again random, and we sometimes use the equivalent notation
\begin{gather*}
      \sum_{i = 1}^{N} 
      T_i
      \cdot
      w_i
      \cdot
      B_k(X_i)
      \,.
\end{gather*}
The (basis-)functions in $\mathfrak{B}$ can be random functions, for example, if they depend on the data $D_N$.
Also the constraint vector $\delta$ can depend on the data (see \cite[Algorithm~1 on page 11]{Wang2019}).


Next, we formulate and discuss assumptions on (the core of) the objective function $\varphi$.
\subsection{Objective Function}
\begin{assumption}
  \label{asu:objective_f}
  The objective function $\varphi\colon \R\to\overline{\R}$ of Problem~\ref{bw:1:primal} 
  satisfies the following conditions
  \begin{itemize}
    \item
      $\varphi(0)=0$ and $\varphi(x)=\infty$ for all $x<0$
    \item
      $\varphi$ is strictly convex and continuously differentiable on $(0,\infty)$ with derivative $\varphi^{'}$
    \item
      $\varphi^{'}(0,\infty)=\R$
    \item
      the inverse of the derivative $(\varphi^{'})^{-1}$ is Lipschitz continuous and continuously differentiable on $\R$.
  \end{itemize}
\end{assumption}
The next lemma provides a link to the assumptions on the objective function in Theorem~\ref{cv:ts:th}.
\newpage
\begin{lemma}
  \label{1065}
  Let Assumption\ref{asu:objective_f} hold true. Then the convex conjugate of $\varphi$ (see \eqref{def:convex_conjugate}) is
  \begin{gather*}
    \varphi^*
    \colon
    \R
    \ 
    \to
    \ 
    \R
    \,,
    \quad
    x^*
    \ 
    \mapsto
    \ 
    x^*
    \!
    \cdot
    (
    \varphi^{'}
    )^{-1}
    (x^*)
    \ 
    -
    \ 
    \varphi
    \left( 
      (
    \varphi^{'}
    )^{-1}
    (x^*)
    \right)
    \,.
  \end{gather*}
  Furthermore, $\varphi^*$ is strictly convex and continuously differentiable on $\R$.
\end{lemma}
\begin{proof}
We define
\begin{gather*}
 \phi
 \ 
 \colon
 [0,\infty)
 \times
 \R
 \ 
 \to
 \ 
 \R
 \,,
 \quad
 (x,x^*)
 \ 
 \mapsto
 \ 
 x\cdot x^*
 \ 
 -
 \ 
 \varphi(x)
 \,.
\end{gather*}
Let $x^*\in\R$.
Since
(by assumption)
      $\varphi$ is continuously differentiable on $(0,\infty)$ with derivative $\varphi^{'}$,
      so is $\phi(\cdot,x^*)$ with derivative
      satisfying 
      \begin{gather*}
        \frac{\partial}{\partial x}
        \phi(x,x^*)
        \ 
        =
 \ 
        x^*
        -
        \varphi^{'}(x)
        \qquad
        \text{for all}\ 
        x\in(0,\infty)
        \,.
      \end{gather*}
  It follows that 
  \begin{gather*}
    z
    \ 
    :=
    \ 
      (
    \varphi^{'}
    )^{-1}
    (x^*)
  \end{gather*}
  is an extreme value of $\phi(\cdot,x^*)$.
  Note, that $x^*\in\R$ is in the domain of 
  $
      (
    \varphi^{'}
    )^{-1}
  $
  by the assumption $\varphi^{'}(0,\infty)=\R$.
  Since $\varphi$ is strictly convex, $\phi(\cdot,x^*)$ is strictly concave. 
  Thus,
  $z>0$ is the unique maximum in $(0,1)$.
  By the continuity of $\phi(\cdot,x^*)$ on $[0,\infty)$ it follows, that $z$ is the unique maximum on $[0,\infty)$.
  Thus
  \begin{align*}
    \varphi^*(x^*)
    &
    \ 
    =
    \ 
    \sup_{x\in\R}
    x\cdot x^* - \varphi(x)
    \ 
    =
    \ 
    \sup_{x\in [0,\infty)}
    x\cdot x^* - \varphi(x)
    \ 
    =
    \ 
    \sup_{x\in [0,\infty)}
    \phi(x,x^*)
    \\
    &
    \ 
    =
    \ 
    \phi(z,x^*)
    \\
    &
    \ 
    =
    \ 
    x^*
    \!
    \cdot
    (
    \varphi^{'}
    )^{-1}
    (x^*)
    \ 
    -
    \ 
    \varphi
    \left( 
      (
    \varphi^{'}
    )^{-1}
    (x^*)
    \right)
    \qquad 
    \text{for all}\ 
    x^*\in\R
    \,.
  \end{align*}
  Now we proof the second statement.
  Since
  $
      (
    \varphi^{'}
    )^{-1}
  $
  is (by assumption) continuously differentiable, it holds
  \begin{align}
    \label{0098}
    \begin{split}
    \frac{\partial}{\partial x^*}
     \varphi^*(x^*)
    &
    \ 
    =
    \ 
    (
    \varphi^{'}
    )^{-1}
    (x^*)
    \ 
    +
    \ 
    x^*
    \!
    \cdot
    \frac{\partial}{\partial x^*}
    (
    \varphi^{'}
    )^{-1}
    (x^*)
    \ 
    -
    \ 
    \varphi^{'}
    \left( 
      (
    \varphi^{'}
    )^{-1}
    (x^*)
    \right)
    \cdot
    \frac{\partial}{\partial x^*}
    (
    \varphi^{'}
    )^{-1}
    (x^*)
    \\
    &
    \ 
    =
    \ 
    (
    \varphi^{'}
    )^{-1}
    (x^*)
    \ 
    +
    \ 
    x^*
    \!
    \cdot
    \frac{\partial}{\partial x^*}
    (
    \varphi^{'}
    )^{-1}
    (x^*)
    \ 
    -
    \ 
    x^*
    \cdot
    \frac{\partial}{\partial x^*}
    (
    \varphi^{'}
    )^{-1}
    (x^*)
    \\
    &
    \ 
    =
    \ 
    (
    \varphi^{'}
    )^{-1}
    (x^*)
    \qquad
    \text{for all}\ 
    x^*\in\R
    \,.
    \end{split}
  \end{align}
  Since $\varphi$ is strictly convex and continuously differentiable, 
  $\varphi^{'}$ is continuous and strictly non-decreasing.
  Thus 
  $
    (
    \varphi^{'}
    )^{-1}
  $
  is continuous and strictly non-decreasing.
  It follows from \eqref{0098} that $\varphi^*$ is strictly convex and continuously differentiable.
\end{proof}
The next lemma completes the link.
\begin{lemma}
  Let Assumption~\ref{asu:objective_f} hold true. Then 
\begin{gather*}
  \Phi
  \ 
  :
  \ 
  \R^n
  \to
  \ 
  \overline{\R}
  \,
  ,
  \qquad
  [w_1,\ldots,w_n]^\top
  \ 
  \mapsto
  \ 
  \sum_{i=1}^n \varphi(w_i)
  \,,
\end{gather*}
satisfies Assumption~\ref{cv:ts:asu}.
\end{lemma}
\begin{proof}
  By Example~\ref{cv:cc:ex}
  the convex conjugate of $\Phi$ is 
\begin{gather*}
  \Phi^*
  \ 
  :
  \ 
  \R^n
  \to
  \ 
  \overline{\R}
  \,
  ,
  \qquad
  [\lambda_1,\ldots,\lambda_n]^\top
  \ 
  \mapsto
  \ 
  \sum_{i=1}^n \varphi^*(\lambda_i)
  \,,
\end{gather*}
where $\varphi^*$ is the convex conjugate of $\varphi$.
By Assumption~\ref{asu:objective_f} $\varphi$ is strictly convex. Thus,
$\Phi$ is strictly convex. By Lemma~\ref{1065}, $\varphi^*$ continuously differentiable on $\R$. Thus,
$\Phi$ is continuously differentiable on $\R^n$.
It follows the statement of Assumption~\ref{cv:ts:asu} for $\Phi$.
\end{proof}


Next we discuss some concrete choices of $\varphi$

\begin{example}
  For two discrete distributions
  \begin{gather*}
    p:=[p_1,\ldots,p_N]
    \qquad
    \text{and}
    \qquad
    q:=[q_1,\ldots,p_N]
  \end{gather*}
  we consider the following distance measure
  \begin{gather*}
   D(p|q)
   \ 
   :=
   \ 
   \sum_{i=1}^{N} 
   p_i
   \cdot
   \log
   \left( 
     \frac{p_i}{q_i}
   \right)
   \,.
  \end{gather*}
  This is known as the Kullback-Leibler-Entropy.
  In \cite[§3.1]{Hainmueller2012} the author connects this concept to a convex optimization problem.
  The idea is, to optimize the Kullback-Leibler-Entropy of the distribution induced by the weights and some base weights.
  For example, if we choose
  \begin{gather*}
    w:=
    \frac{1}{N}[w_1\cdot T_1,\ldots,w_N\cdot T_N]
    \qquad
    \text{and}
    \qquad
    q:=\frac{1}{N}[1,\ldots,1]
  \end{gather*}
  we get
  \begin{align}
    \label{8926}
   D(w|q)
   \ 
   =
   \ 
   \frac{1}{N}
   \sum_{i=1}^{N} 
w_i\cdot T_i
   \cdot
   \log
   \left( 
     \frac{w_i\cdot T_i}{N}\cdot N
   \right)
   \ 
   =
   \ 
   \frac{1}{N}
   \sum_{i=1}^{n} 
w_i
   \cdot
   \log
   \left( 
     w_i
   \right)
   \,,
  \end{align}
  where we set $"0\cdot \log(0)"=0$. Thus, the optimization problem
  \begin{gather*}
    \underset{w_1, \ldots, w_n \in \R}
    {\text{minimize}}
    \qquad\qquad
    \sum_{i = 1}^{n} 
w_i
   \cdot
   \log
   \left( 
     w_i
   \right)
  \end{gather*}
  produces the same optimal solutions as minimizing the Kullback-Leibler-Entropy \eqref{8926} with respect to $w$.
Thus, we consider 
\begin{gather*}
  \varphi
  \ 
  \colon
  \R
  \ 
  \to
  \ 
  \overline{\R}
  \,,
  \qquad
  x
  \ 
  \mapsto
  \ 
  \begin{cases}
    x\cdot \log x\ &\text{if}\ x>0\,, \\
    0 \ &\text{if}\, x=0\,,\\
    \infty\ &\text{if}\ x<0\,.
  \end{cases}
\end{gather*}
We show, that this choices satisfies Assumption~\ref{asu:objective_f}.
By definition it holds \textit{(i)}.
For \textit{(ii)} note, that the second derivative on $(0,\infty)$ is $x\mapsto 1/x$. Thus (by the second derivative test), $\varphi$ is strictly convex.
Clearly it is also twice continuously differentiable.
For the continuity in $0$ note, that $\lim_{x\to 0} x\log x=0$. 
For \textit{(iii)} note, that $x\mapsto \log x$ is continuous and strictly non-decreasing on $\R$. 
Since 
\begin{align*}
  \lim_{x\to 0}\log x\ =\ -\infty
  \qquad
  \text{and}
  \qquad
  \lim_{x\to \infty}\log x\ =\ \infty
  \,,
\end{align*}
and $\varphi^{'}=(x\mapsto \log x +1)$ on $(0,\infty)$, it follows \textit{(iii)}. 
Finally, it holds $(\varphi^{'})^{-1}=(x\mapsto \exp(x-1))$ on $\R$. 
Thus, it follows \textit{(iv)}.
\end{example}
\begin{example}
  In a similar setting, the authors of \cite{Zubizarreta2015} choose the sample variance of the weights as objective function, that is,
\begin{gather*}
  \varphi
  \ 
  \colon
  \R
  \ 
  \to
  \ 
  \overline{\R}
  \,,
  \qquad
  x
  \ 
  \mapsto
  \ 
  \begin{cases}
    \left(
      x-\frac
      {1}
      {n}
    \right)^2\ &\text{if}\ x\ge0\,, \\
    \infty\ &\text{if}\ x<0\,.
  \end{cases}
\end{gather*}
\end{example}


Next we formulate assumptions on the propensity score function.
\subsection{Propensity Score Function}

\begin{definition}
  We define the \textbf{propensity score function} by
  \begin{gather*}
    \pi
    \ 
    \colon
    \R^d
    \ 
    \to
    \ 
    [0,1]
    \,,
    \qquad
    x
    \ 
    \mapsto
    \ 
    \P
    [
    T=1
    |
    X=x
    ]
    \,,
  \end{gather*}
  that is, 
  as the conditional probability of treatment given individual characteristics $x\in\R^d$.
\end{definition}
\begin{remark}
  We define $\pi$ on the whole $\R^d$, although $\mathcal{X}\subset \R^d$ may be a much smaller (possibly finite or countable) subset.
  The reason is that we want to assume continuity.
\end{remark}
\begin{assumption}
  \label{asu:ps}
  The propensity score function $\pi$ satisfies
  \begin{itemize}
    \item
      $\pi(x)\in (0,1)$ for all $x\in\R^d$
    \item
      $\pi$ is continuously differentiable on $\R^d$
  \end{itemize}
\end{assumption}

Next we give a standard example for a propensity score model

\begin{example}
  Logistic regression
\end{example}


Next we formulate and discuss assumptions on the basis functions.
\subsection{Basis Functions}
In our setting basis functions of the covariates are stochastic processes indexed over the covariate space $\mathcal{X}$.
We consider a set $\mathfrak{B}$ of stochastic processes 
\begin{gather*}
  \left\{ 
      B_k(x)
      \ 
      \colon
      (
      \Omega,
      \mathcal{A}
      ,
      \P
      )
      \ 
      \to
      \ 
      (
      \R
      ,
      \mathcal{B}(\R)
      )
      \ 
      |
      \ 
      x\in\mathcal{X}
  \right\}
  \qquad
  \text{for}\ 
k\in \left\{ 1,\ldots,\# \mathfrak{B} \right\}
\end{gather*}
and
denote the vector of the basis functions (also a stochastic processes indexed over $\mathcal{X}$)
as
\begin{align*}
B(x)
  \ 
  :=
  \ 
  [B_1(x),\ldots,B_{\# \mathfrak{B}}(x)]^\top
  \ 
  \colon
  \ 
      (
      \Omega,
      \mathcal{A}
      ,
      \P
      )
      \to
      (
      \R^{\#\mathfrak{B}}
      ,
      \mathcal{B}
      (
      \R^{\#\mathfrak{B}}
      )
      )
    \qquad
  \text{for}\ 
  x\in\mathcal{X}
  \,.
\end{align*}
\begin{gather*}
\end{gather*}
\begin{assumption}
  \label{asu:basis}
  The set of basis functions 
  $
  \mathfrak{B}
  $
  satisfies
  \begin{enumerate}[label=(\roman*)]
    \item
      $\norm{B(x)}_2\lesssim 1$ for all $x\in \mathcal{X}$.
    \item
      There exist random vectors
      $
      \lambda^*_{\varphi^{'}\circ\, 1/\pi}
      $
      and
      $
      \lambda^*_{F_{Y(1)(z|\cdot)}}
      $,
      for all $z\in\R$,
      with values in $\R^{\#\mathfrak{B}}$
      such that
      \begin{align}
        \label{asu:basis:ii:1}
        \frac{1}{N}
        \sum_{i=1}^{N} 
        \left| 
        \inner{B(X_i)}
        {
      \lambda^*_{\varphi^{'}\circ\, 1/\pi}
      }
        -
        \varphi^{'}\left( \frac{1}{\pi(X_i)} \right)
        \right|
        &
        \ 
        \overset{\P}
        {
        \to 
        }
        \ 
        0
        \qquad
        \text{for}
        \ 
        N\to\infty
        \,,
        \intertext{and}
        \label{asu:basis:ii:2}
        \sqrt{N}
        \max_{i\in \left\{ 1,\ldots,N \right\}}
        \sup_{z\in\R}
        \left| 
        \inner{B(X_i)}
        {
      \lambda^*_{F_{Y(1)}(z|\cdot)}
      }
        -
        F_{Y(1)}(z|X_i)
        \right|
        &
        \ 
        \overset{\P}
        {
        \to 
        }
        \ 
        0
        \qquad
        \text{for}
        \ 
        N\to\infty
        \,.
      \end{align}
    \end{enumerate}
\end{assumption}

Next, we give some examples. 
We begin with a histogram basis \cite[§4]{Gyorfi2002}
\begin{example}
We consider a sequence of partitions
$
\left( 
  \mathcal{P}_N
  =
  \left\{ 
    A_{N,1}
    ,
    A_{N,2}
    ,
    \ldots
  \right\}
\right)
$
of $ \R^d $
and define
$ A_N(x) $ to be the cell of $ \mathcal{P}_N $ containing $x$.
We also assume
uniform partition width that decreases to 0, that is,
\begin{gather}
  \label{8881}
  \text{
for all $j\in\mathbb{N}$
it holds
  }
  \qquad
  \lambda^d(A_{N,j})
  \ 
  =
  \ 
  h_N^d
  \ 
  \to
  \ 
  0
  \qquad
  \text{for}
  \ 
  N\to\infty
  \,.
\end{gather}
We define $N$ basis functions $B_k$ of the covariates by
\begin{gather*}
  B_k(x)
  \ 
  :=
  \ 
  \frac{
  \mathbf{1}{\left\{ X_k \in A_N(x) \right\}}
  }{
  \sum_{j=1}^{N} 
  \mathbf{1}{\left\{ X_j \in A_N(x) \right\}}
  }
  \qquad
  \text{for}
  \,
  k\in
  \left\{ 
  1,\ldots,N
  \right\}
  \,,
\end{gather*}
where we keep to the convention $"0/0=0"$.
If at least one $B_k(x)>0$, the basis functions sum to 1. If $B_k(x)=0$ for all basis functions, the sum is 0.
Thus
\begin{align}
  \label{8882}
  \sum_{k=1}^{N}
  B_k(x)
  &
  \ 
  \in
  \ 
  \left\{ 0,1 \right\}
  \qquad
  \text{for all}
  \ 
  x\in\R^d
  \,.
  \notag
  \intertext{
Since $B_i(X_i)>0$, it holds
  }
  \sum_{k=1}^{N}
  B_k(X_i)
  &
  \ 
  =
 \  
  1
  \qquad
  \text{for all}\ 
  i\in
  \left\{ 1,\ldots,N \right\}
  \,.
\end{align}
We check \textit{(i)} in Assumption~\ref{asu:basis}. 
Since 
\begin{gather*}
B_k(x)
\ 
\in
\ 
[0,1]
\qquad
\text{for all}\ 
x\in\R^d
\ 
\text{and for all}\ 
k\in \left\{ 1,\ldots,N \right\}
\,,
\end{gather*}
it holds
\begin{gather*}
  \label{basis_l2_bdd}
  \norm{B(x)}_2^2
  \ 
  =
  \ 
  \sum_{k=1}^{N} 
  B_k(x)
  ^2
  \ 
  \le
  \ 
  \sum_{k=1}^{N} 
  B_k(x)
  \ 
  \in
  \ 
  \left\{ 0,1 \right\}
  \quad
  \text{
    for all
  }
x\in\R^d
\,
\,.
\end{gather*}
Next we check \eqref{asu:basis:ii:1} in \textit{(ii)} in Assumption~\ref{asu:basis}.
To this end we consider
\begin{gather*}
  \lambda^*_{\varphi^{'}\circ\,1/\pi}
  \ 
  :=
  \ 
  \left[ 
    \varphi^{'}
    \left( 
      \frac{1}{\pi(X_1)}
    \right)
    ,
    \ldots
    ,
    \varphi^{'}
    \left( 
      \frac{1}{\pi(X_N)}
    \right)
  \right]
  ^\top
  \,.
\end{gather*}
Since $f$ is continuous and the $X_k$ are random vectors,
$\lambda^*_f$ is also a random vector.
It follows 
\begin{align*}
  &
  \frac{1}{N}
  \sum_{i=1}^{N} 
        \left| 
        \inner{B(X_i)}
        {
  \lambda^*_{\varphi^{'}\circ\,1/\pi}
      }
        -
    \varphi^{'}
    \left( 
      \frac{1}{\pi(X_i)}
    \right)
        \right|
        \\
  &
  \ 
        =
  \ 
  \frac{1}{N}
  \sum_{i=1}^{N} 
        \left| 
        \sum_{k=1}^{N} 
        B_k(X_i)
        \cdot
    \varphi^{'}
    \left( 
      \frac{1}{\pi(X_k)}
    \right)
        -
    \varphi^{'}
    \left( 
      \frac{1}{\pi(X_i)}
    \right)
        \right|
        \\
  &
  \ 
        =
  \ 
  \frac{1}{N}
  \sum_{i=1}^{N} 
        \left| 
        \sum_{k=1}^{N} 
        B_k(X_i)
        \cdot
        \left( 
    \varphi^{'}
    \left( 
      \frac{1}{\pi(X_k)}
    \right)
        -
    \varphi^{'}
    \left( 
      \frac{1}{\pi(X_i)}
    \right)
        \right)
        \right|
\\
  &
  \ 
        =
  \ 
  \frac{1}{N}
  \sum_{i=1}^{N} 
        \left| 
        \sum_{k=1}^{N} 
        B_k(X_i)
        \cdot
        \mathbf{1}
        \left\{ X_k\in A_N(X_i) \right\}
        \cdot
        \left( 
    \varphi^{'}
    \left( 
      \frac{1}{\pi(X_k)}
    \right)
        -
    \varphi^{'}
    \left( 
      \frac{1}{\pi(X_i)}
    \right)
    \right)
        \right|
        \\
  &
  \ 
        \le
  \ 
  \frac{1}{N}
  \sum_{i=1}^{N} 
        \sum_{k=1}^{N} 
        B_k(X_i)
        \cdot
        \omega
        ( 
        \varphi^{'} 
        \circ
        (x\mapsto 1/x)
        \circ
        \pi
        ,
        \lambda^d(A_N(X_i))
        )
        \\
  &
  \ 
        =
  \ 
        \omega
        ( 
        \varphi^{'} 
        \circ
        (x\mapsto 1/x)
        \circ
        \pi
        ,
        h_N^d
        )
        \ 
        \to
        \ 
        0
        \qquad
        \text{for}
        \ 
        N\to\infty
        \,,
\end{align*}
where $\omega$ is the modulus of continuity.
The second equality is due to \eqref{8881}, the third equality follows from the definition of the basis functions, and 
the inequality follows from \eqref{8881}, the convexity of the absolute value and the continuity of $f$.
The convergence in due to \eqref{8882}.

Next, we check \eqref{asu:basis:ii:2} in \textit{(ii)} in Assumption~\ref{asu:basis}.
To this end, we consider
\begin{gather*}
  \lambda^*_{F_{Y(1)}(z|\cdot)}
  \ 
  :=
  \ 
  \left[ 
    F_{Y(1)}(z|X_1)
    ,
    \ldots
    ,
    F_{Y(1)}(z|X_N)
  \right]
  ^\top
  \qquad
  \text{for}\ 
  z\in\R
  \,.
\end{gather*}
To obtain the desired convergence, we need to assume some regularity for 
$
    F_{Y(1)}(z|\cdot)
$.
Let it be continuity for the moment.
With similar arguments as before we get
    \begin{align*}
      &
         \sqrt{N}
        \max_{i\in \left\{ 1,\ldots,N \right\}}
        \sup_{z\in\R}
        \left| 
        \inner{B(X_i)}
        {
      \lambda^*_{F_{Y(1)}(z|\cdot)}
      }
        -
        F_{Y(1)}(z|X_i)
        \right|
        \\
        &
        \ 
        =
        \ 
         \sqrt{N}
        \sup_{z\in\R}
        \omega
        \left( 
        F_{Y(1)}(z|\cdot)
        ,
        h_N^d
        \right)
    \end{align*}
    Thus, if the interplay of $F_{Y(1)}$ and $h_N$ is sufficiently good we get convergence.
    The most abstract assumption would be
    \begin{gather}
      \label{6674}
         \sqrt{N}
        \sup_{z\in\R}
        \omega
        \left( 
        F_{Y(1)}(z|\cdot)
        ,
        h_N^d
        \right)
        \to
        0
        \,.
    \end{gather}
    To be more concrete, if, for example,
$
    F_{Y(1)}(z|\cdot)
    $
    is $\alpha$-Hölder continuous for all $z\in\R$ with 
    $\alpha\in(0,1)$ and $\sqrt{N}h_N^{d\cdot \alpha}\to 0$ it holds
    \eqref{6674}.

\end{example}

%We continue with kernel bases \cite[§5]{Gyorfi2002}.
%\begin{example}
%Let
%$
%  K
%  \colon
%  \R^d
%  \to
%  [0,\infty)
%$
%be a measurable function with $K(0)>0$. 
%We call $K$ a kernel function.
%We assume that there exists
%$R>0$ such that 
%\begin{gather}
%  \label{8884}
%  K(x)
%  \ 
%  \le
%  \ 
%  \mathbf{1}\left\{ \norm{x}_2\le R\right\}
%  \,,
%\end{gather}
%that is, the kernel function has compact support.
%For a decreasing sequence (the bandwith of the kernel) $(h_N)\in (0,\infty)$ with $h_N\to 0$ for $N\to\infty$,
%we define $N$ basis functions $B_k$ of the covariates by
%\begin{gather*}
%  B_k(x)
%  \ 
%  :=
%  \ 
%  \frac{
%    K \left( \frac{\norm{x-X_k}_2}{h_N} \right)
%  }{
%  \sum_{j=1}^{N} 
%    K \left( \frac{\norm{x-X_j}_2}{h_N} \right)
%  }
%  \qquad
%  \text{for}
%  \,
%  k\in
%  \left\{ 
%  1,\ldots,N
%  \right\}
%  \,,
%\end{gather*}
%where (again) we keep to the convention $"0/0=0"$.
%Since $K(0)>0$, as in the previous example, it follows
%\begin{align}
%  \label{8883}
%  \sum_{k=1}^{N}
%  B_k(x)
%  &
%  \ 
%  \in
%  \ 
%  \left\{ 0,1 \right\}
%  \qquad
%  \text{for all}
%  \ 
%  x\in\R^d
%  \,,
%  \notag
%  \\
%  \sum_{k=1}^{N}
%  B_k(X_i)
%  &
%  \ 
%  =
% \  
%  1
%  \qquad
%  \text{for all}\ 
%  i\in
%  \left\{ 1,\ldots,N \right\}
%  \intertext{and}
%  \notag
%  \norm{B(x)}_2^2
%  &
%  \ 
%  \in
%  \ 
%  \left\{ 0,1 \right\}
%  \qquad
%  \text{for all}
%  \ 
%  x\in\R^d
%  \,.
%\end{align}
%This verifies the first condition of Assumption~\ref{asu:basis}.
%To verify the second condition, note, that
%\begin{align}
%  \label{8885}
%  \notag
%  K(x)
%  &
%  \ 
%  =
%  \ 
%  \mathbf{1}\left\{ \norm{x}_2\le R \right\} 
%  K(x)
%  \qquad
%  \text{for all}
%  \ 
%  x\in\R^d
%  \intertext{and thus}
%  B_k(x)
%  &
%  \ 
%  =
%  \ 
%  B_k(x)
%  \cdot
%  \mathbf{1}\left\{ \norm{x-X_k}_2\le R\cdot h_N \right\} 
%  \qquad
%  \text{for all}
%  \ 
%  x\in\R^d
%  \ 
%  \text{and for all}
%  \ 
%  k\in \left\{ 1,\ldots,N \right\}
%  \,.
%\end{align}
%Let 
%$f$ be a continuous function
%and consider (as in the previous example)
%\begin{gather*}
%  \lambda^*_f
%  \ 
%  :=
%  \ 
%  [f(X_1),\ldots,f(X_N)]^\top
%  \,.
%\end{gather*}
%It follows
%\begin{align*}
%  &
%        \left| 
%        \inner{B(X_i)}{\lambda^*_f}
%        -
%        f(X_i)
%        \right|
%        \\
%  &
%  \ 
%        =
%  \ 
%        \left| 
%        \sum_{k=1}^{N} 
%        B_k(X_i)
%        \cdot
%        f(X_k)
%        -
%        f(X_i)
%        \right|
%        \ 
%        =
%        \ 
%        \left| 
%        \sum_{k=1}^{N} 
%        B_k(X_i)
%        \cdot
%        \left( 
%        f(X_k)
%        -
%        f(X_i)
%        \right)
%        \right|
%\\
%  &
%  \ 
%        =
%  \ 
%        \left| 
%        \sum_{k=1}^{N} 
%        B_k(X_i)
%        \cdot
%  \mathbf{1}\left\{ \norm{x-X_k}_2\le R\cdot h_N \right\} 
%        \cdot
%        \left( 
%        f(X_k)
%        -
%        f(X_i)
%        \right)
%        \right|
%        \\
%  &
%  \ 
%        \le
%  \ 
%        \sum_{k=1}^{N} 
%        B_k(X_i)
%        \cdot
%        \omega
%        ( 
%        f,
%        R\cdot h_N
%        )
%        \\
%  &
%  \ 
%        =
%  \ 
%        \omega
%        ( 
%        f,
%        R\cdot h_N
%        )
%        \to
%        0
%        \qquad
%        \text{for}
%        \ 
%        N\to\infty
%        \,,
%\end{align*}
%The convergence in due to $h_N\to 0$ for $N\to \infty$. 
%\end{example}
We now have a large class of possible basis function. We may proceed.


Next, we derive the dual formulation of Problem~\ref{bw:1:primal}.
\subsection{Dual Problem}
\begin{lemma}
  \label{matrix_notation}
  A matrix formulation of Problem~\ref{bw:1:primal} is 
\begin{align}
  \label{cv:ts:primal}
  %%%% objective %%%%
    &\underset{w \in \R^n}
    {\mathrm{minimize}}
    &&\qquad\qquad
    \Phi(w)
    &&&
    \\
    %%%% Ax >= b %%%%
    \nonumber
    &\mathrm{subject}\ \mathrm{to} 
    &&\qquad\qquad
    \mathbf{U}w
    \ 
    \ge
    \ 
    d
    \,,
    \\
    \nonumber
    &
    &&\qquad\qquad
    \mathbf{A}w
    \ 
    =
    \ 
    a
    \,,
\end{align}
with objective function
\begin{gather*}
  \Phi
  \ 
  :
  \ 
  \R^n
  \to
  \ 
  \overline{\R}
  \,
  ,
  \qquad
  [w_1,\ldots,w_n]^\top
  \ 
  \mapsto
  \ 
  \sum_{i=1}^n \varphi(w_i)
  \,,
\end{gather*}
inequality matrix and vector
\begin{alignat*}{2}
    \mathbf{U}
    &
    \ 
    :=
    \ 
    \begin{bmatrix}
      \mathbf{I}_n
      \\
      \pm\,\mathbf{B}(\mathbf{X})
    \end{bmatrix}
    \in
    \R^{(n+  2 N)\times n}
        \qquad
    &&
d
    \ 
    :=
    \ 
    \begin{bmatrix}
      0_n
      \\
      -N\cdot\delta 
      \ 
      \pm\ 
      \sum_{i = 1}^{N} B(X_i)
    \end{bmatrix}
    \in
    \R^{n+  2 N}
    \,,
    \intertext{and equality matrix and vector}
    \mathbf{A}
    &
    \ 
    :=
    \ 
      \mathrm{1}_n
      ^\top
      \in\R^{1\times n}
      \qquad
    &&
    a
  \ 
    :=
    \ 
    N
    \in\mathbb{N}
    \,.
\end{alignat*}
\end{lemma}

\begin{proof}
  Recall that the box constraints of Problem~\ref{bw:1:primal} are
  \begin{gather*}
        \left| 
      \frac{1}{N} 
      \left( 
      \sum_{i = 1}^{n} 
      w_i
      B_k(X_i)
      -
      \sum_{i=1}^{N} 
      B_k(X_i)
      \right)
    \right|
    \ 
    \le 
    \ 
    \delta_k
    \qquad
    \text{for all}\ 
    k\in \left\{ 1,\ldots, N \right\}
    \,.
  \end{gather*}
  Put differently, it holds both
  \begin{align*}
    -
      \sum_{i = 1}^{n} 
      w_i
      B_k(X_i)
    \ge 
    -
    N
    \delta_k
      -
      \sum_{i=1}^{N} 
      B_k(X_i)
      \quad 
    \text{and}
      \quad
      \sum_{i = 1}^{n} 
      w_i
      B_k(X_i)
    \ge 
    -
    N
    \delta_k
      +
      \sum_{i=1}^{N} 
      B_k(X_i)
  \end{align*}
  for all 
  $
    k\in \left\{ 1,\ldots, N \right\}
  $. In matrix notation this is 
  \begin{gather*}
    \pm\mathbf{B}(\mathbf{X})w
    \ 
    \ge
    \ 
    [d_{n+1},\ldots, d_{n+  2 N}]^\top
    \,.
  \end{gather*}
  Proving the rest of the statements is straightforward. We omit the details.
\end{proof}
\begin{remark}
  The inequality constraints of
  Lemma~\ref{matrix_notation} differ from its counterpart
  \cite[Proof of Lemma~1]{Wang2019}.
  We don't transform the variable $w$, but shift to $d$ what prevents us from keeping $w$.
  Note, that the choice of
  \cite[Proof of Lemma~1]{Wang2019} leads to a mistake on page 21.
  The mistake is most obvious in the second display, where the first implication follows from dividing by 0.
  I discussed this with the authors and proposed a version of Lemma\ref{matrix_notation} to solve the problem. I think it's best not to transform variables, because the mistake comes from (wrongly) calculating the convex conjugate of the (more complicated) transformed version of the objective function. The subsequent analysis even simplifies with my version.

  I was surprised to find the (exact) same mistake in the earlier paper 
  \cite[page 35 second display]{Chan2016}. 
  There is no reference in
  \cite[Proof of Lemma~1]{Wang2019} 
  to
  \cite{Chan2016}. Yet the formulation and the mistake are the same.
  Did the authors of \cite{Wang2019} (inadvertently?) plagiarize
  the mathematical analysis of 
  \cite{Chan2016}
  ?
\end{remark}

\begin{lemma}
  Consider the optimization problem
\begin{align}
  \label{9993}
  \begin{split}
  \underset
  {\begin{smallmatrix}
\rho\,,\, \lambda^+,\,\lambda^-\ge 0 \\
\lambda_0\in\R
  \end{smallmatrix}}
  {
    \mathrm{maximize}
  }
  \quad
  &
  -
\sum_{i=1} 
  ^n
    \,
  \varphi^*
  \!
  \left( 
    \rho_i
    +
\lambda_0
+
\inner
{B(X_i)}
{
\lambda^+
-
\lambda^-
}
  \right)
  \\
  &
+
\ 
\sum_{i=1}^{N} 
  \left( 
\lambda_0
+
\inner
{B(X_i)}
{
\lambda^+
-
\lambda^-
}
  \right)
  \,
  \ 
-
\ 
\inner
{\delta}
{
\lambda^+
+
\lambda^-
}
  \,.
  \end{split}
\end{align}
If Assumption~\ref{asu:objective_f} holds true 
and there exists an optimal solution 
$
(\rho^\dagger,\lambda_0^\dagger,\lambda^{+,\dagger},\lambda^{-,\dagger})
$
then the unique optimal solutions to Problem~\ref{bw:1:primal} are 
\begin{gather*}
  w^\dagger_i
  \ 
  :=
  \ 
  (
  \varphi^{'}
  )^{-1}
  \left(
    \rho^\dagger_i
  \ 
    +
  \ 
\lambda_0^\dagger
  \ 
+
  \ 
\inner
{B(X_i)}
{
  \lambda^{+,\dagger}
-
\lambda^{-,\dagger}
}
  \right)
  \qquad
  \text{for all}\ 
  i\in
  \left\{ 1,\ldots,n \right\}
  \,.
\end{gather*}
\end{lemma}
\begin{proof}
  By Lemma~\ref{matrix_notation},
  Problem~\ref{bw:1:primal} has the form required in Theorem~\ref{cv:ts:th}.
  By Assumption~\ref{asu:objective_f} and Lemma~\ref{9991} the objective function $\Phi$ of Problem~\ref{bw:1:primal}
  satisfies Assumption~\ref{cv:ts:asu}.
  Thus we can apply
  Theorem~\ref{cv:ts:th} to Problem~\ref{bw:1:primal}.
  Calculations yield the result.
\end{proof}
The next Theorem aims at simplifying this result. 
\newpage
\begin{ftheorem}
  \label{dual_solution_th}
  Consider the optimization problem
\begin{align}
  \label{dual}
  \begin{split}
  \underset
  {\begin{smallmatrix}
      \rho&\in&&\R^N 
      \\
      \lambda_0 & \in&&\R
      \\
      \lambda&\in&&\R^{N}
  \end{smallmatrix}}
  {
    \mathrm{minimize}
  }
  \quad
  \frac{1}{N}
\sum_{i=1} 
  ^N
  &
  \Big[
  T_i
  \cdot
  \varphi^*
  \!
  \left( 
    \rho_i
    +
\lambda_0
+
\inner
{B(X_i)}
{
\lambda
}
  \right)
  \ 
  -
  \ 
\lambda_0
-
\inner
{B(X_i)}
{
\lambda
}
\Big]
  \ 
+
\ 
\inner
{\delta}
{
  |\lambda|
}
  \,,
  \\
  \mathrm{subject}\ \mathrm{to}
  \quad
  \qquad
  &
  \rho_i \ge 0 
  \quad 
  \mathrm{for}\ \mathrm{all}\ i\le n
  \qquad 
  \mathrm{and}
  \qquad
  \rho_i=0
  \quad 
  \mathrm{for}\ \mathrm{all}\ i>n
  \,.
\end{split}
\end{align}
If Assumption~\ref{asu:objective_f} holds true 
and there exists an optimal solution 
$
(\rho^\dagger,\lambda_0^\dagger,\lambda^\dagger)
$
then the unique optimal solutions to Problem~\ref{bw:1:primal} are 
\begin{gather*}
  w^\dagger_i
  \ 
  :=
  \ 
  (
  \varphi^{'}
  )^{-1}
  \left(
    \rho^\dagger_i
  \ 
    +
  \ 
\lambda_0^\dagger
  \ 
+
  \ 
\inner
{B(X_i)}
{
\lambda^{\dagger}
}
  \right)
  \qquad
  \text{for all}\ 
  i\in
  \left\{ 1,\ldots,n \right\}
  \,.
\end{gather*}
\end{ftheorem}

\begin{proof}
  Assume that
$
  (\rho^\dagger,\lambda_0^\dagger,\lambda^{+,\dagger},\lambda^{-,\dagger})
$
is an optimal solution to Problem~\ref{9993}.
We write
\begin{align*}
  G
  (\rho,\lambda_0,\lambda^+,\lambda^-)
  &
  \ 
  :=
  \ 
 -
\sum_{i=1} 
  ^n
    \,
  \varphi^*
  \!
  \left( 
    \rho_i
    +
\lambda_0
+
\inner
{B(X_i)}
{
\lambda^+
-
\lambda^-
}
  \right)
  \\
  &
  \qquad
+
\ 
\sum_{i=1}^{N} 
  \left( 
\lambda_0
+
\inner
{B(X_i)}
{
\lambda^+
-
\lambda^-
}
  \right)
  \,
  \ 
-
\ 
\inner
{\delta}
{
\lambda^+
+
\lambda^-
}
  \,.
\end{align*}
 To eliminate the remaining constraints, 
  we paraphrase \cite[pages~19-20]{Wang2019}.
  We show 
  for all $i \in \left\{ 1,\ldots,N \right\}$
\begin{alignat}{2}
  \notag
  \text{either}
  &
  &&
  \qquad
  \lambda_i^{+,\dagger} > 0
  \\
  \label{9992}
  \text{or}
  &
  &&
  \qquad
  \lambda_i^{-,\dagger} > 0
  \,.
\end{alignat}
Assume towards a contradiction that 
\begin{gather}
  \label{1232}
  \text{
there exists
  } 
  \ 
i \in \left\{ 1,\ldots,N \right\}
\ 
\text{such that}
\qquad
  \lambda_i^{+,\dagger} > 0
  \qquad 
  \text{and}
  \qquad
  \lambda_i^{-,\dagger} > 0
  \,.
\end{gather}
Consider
  \begin{align*}
    \tilde{\lambda}^{+,\dagger}
    &
    \ 
    :=
    \ 
    \begin{bmatrix}
      \ 
      \lambda_1^{+,\dagger}
      \ldots,
      \ 
      \lambda_i^{+,\dagger}
      \!
      \ 
      -
      \ 
      (
      \lambda_i^{+,\dagger}
      \!
      \land
      \lambda_i^{-,\dagger}
      )\,,
      \ 
      \ldots,
      \lambda_{N}^{+,\dagger}
    \end{bmatrix}
    ^\top
    \intertext{and}
    \tilde{\lambda}^{-,\dagger}
    &
    \ 
    :=
    \ 
    \begin{bmatrix}
      \ 
      \lambda_1^{-,\dagger}
      \ldots,
      \ 
      \lambda_i^{-,\dagger}
      \!
      \ 
      -
      \ 
      (
      \lambda_i^{+,\dagger}
      \!
      \land
      \lambda_i^{-,\dagger}
      )\,,
      \ 
      \ldots,
      \lambda_{N}^{-,\dagger}
    \end{bmatrix}
    ^\top
    \,.
  \end{align*}
  Since
  \begin{gather*}
      \lambda_i^{\pm,\dagger}
      \!
      \ 
      -
      \ 
      (
      \lambda_i^{+,\dagger}
      \!
      \land
      \lambda_i^{-,\dagger}
      )
      \ 
      \ge 
      \ 
      0
      \,,
  \end{gather*}
  the perturbed vectors $\tilde{\lambda}^{\pm,\dagger}$ are  in the domain of the 
  optimization problem.
  By Assumption~\eqref{1232} and $\delta>0$ it follows
  \begin{align*}
  G
  \left( 
  \rho^\dagger,\lambda_0^\dagger,\tilde{\lambda}^{+,\dagger},\tilde{\lambda}^{-,\dagger}
  \right)
  \ 
  -
  \ 
  G
  \left( 
  \rho^\dagger,\lambda_0^\dagger,\lambda^{+,\dagger},\lambda^{-,\dagger}
  \right)
  \ 
  =
  \ 
  2
  \cdot
  \delta_i
  \cdot
      (
      \lambda_i^{+,\dagger}
      \!
      \land
      \lambda_i^{-,\dagger}
      )
  \ 
  >
  \ 
  0
  \,,
  \end{align*}
  which contradicts the optimality of
$
  (\rho^\dagger,\lambda^{+,\dagger},\lambda^{-,\dagger},\lambda_0^\dagger)
$
(it is supposed to be a maximum in the domain of the optimization problem)
.
It follows \eqref{9992}.
But then 
$
\lambda^{\pm,\dagger}_i
\ge 0
$
collapses to
$
\lambda_i^\dagger\in \R
$ 
for all
$i\in \left\{ 0,\ldots,N \right\}$, that is, we set
\begin{gather*}
 \lambda_i^\dagger
 \ 
 =
 \ 
 \lambda_i^{+,\dagger}
 \ 
 -
 \ 
 \lambda_i^{-,\dagger}
 \qquad
 \text{and}
 \qquad
|\lambda_i^\dagger|
\ 
=
\ 
\lambda_i^{+,\dagger}
\ 
+
\ 
\lambda_i^{-,\dagger}
\,.
\end{gather*}
Thus, we can extend the domain of Problem~\ref{9993} to $\lambda\in\R^{N}$ and update the objective function in the following way
(without changing the optimal solution).
\begin{align*}
  G
  (\rho,\lambda_0,\lambda)
  &
  \ 
  :=
  \ 
 -
\sum_{i=1} 
  ^n
    \,
  \varphi^*
  \!
  \left( 
    \rho_i
    +
\lambda_0
+
\inner
{B(X_i)}
{
\lambda
}
  \right)
  \\
  &
  \qquad
+
\ 
\sum_{i=1}^{N} 
  \left( 
\lambda_0
+
\inner
{B(X_i)}
{
\lambda
}
  \right)
  \,
  \ 
-
\ 
\inner
{\delta}
{
  |\lambda|
}
  \,.
\end{align*}
Multiplying $G$ with $-1/N$ doesn't change the solution either
(if we search instead for a minimum).
To finish the proof, we choose the notation with $T_i$ instead of $n$. This extends the domain of $\rho$ to $\R^N_{\ge 0}$, but the 
new $\rho_i$ are not effective because of $T_i=0$ for all $i>n$. 
Thus we may set them to 0.
\end{proof}


