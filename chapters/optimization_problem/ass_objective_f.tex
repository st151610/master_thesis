The formulation of Problem~\ref{bw:1:primal} allows for
great flexibility.
To do the mathematical analysis, however, we have to restrict it.
There are (at least) two good ways to do this.
We may, right from the start,
make concrete choices.
For example, we may choose the Kullback-Leibler-Entropy (see the end of this section) as the objective function of Problem~\ref{bw:1:primal}.
From this we can prove lemmas that we use later.
But we shall do it the other way around.
What should be a lemma in the first approach, shall for us be an assumption.
A concrete choice in the first approach shall for us be an example.
In this way, we hope to preserve more of the initial flexibility.
\newpage
\begin{assumption}
  \label{asu:objective_f}
  The (proper) convex function $\varphi$ in Problem~\ref{bw:1:primal} 
  satisfies the following:
  \begin{enumerate}[label=(\roman*)]
    \item
      $\varphi(x)=\infty$ for all $x<0$
    \item
      $\varphi$ is strictly convex,
      continuous on $[0,\infty)$,
      and continuously differentiable on $(0,\infty)$ with derivative $\varphi^{'}$
    \item
      $\varphi^{'}(0,\infty)=\R$
    \item
      the inverse of the derivative $(\varphi^{'})^{-1}$ is continuously differentiable on $\R$
  \end{enumerate}
\end{assumption}
Now we begin to link
Assumption~\ref{asu:objective_f}
to the assumptions on the objective function in Theorem~\ref{cv:ts:th}.
\begin{lemma}
  \label{1065}
  Let Assumption\ref{asu:objective_f} hold true. Then the convex conjugate of $\varphi$ (see \eqref{def:convex_conjugate}) is
  \begin{gather*}
    \varphi^*
    \colon
    \R
    \ 
    \to
    \ 
    \R
    \,,
    \quad
    x^*
    \ 
    \mapsto
    \ 
    x^*
    \!
    \cdot
    (
    \varphi^{'}
    )^{-1}
    (x^*)
    \ 
    -
    \ 
    \varphi
    \left( 
      (
    \varphi^{'}
    )^{-1}
    (x^*)
    \right)
    \,.
  \end{gather*}
  Furthermore, $\varphi^*$ is strictly convex and continuously differentiable on $\R$.
\end{lemma}
\begin{proof}
We define
\begin{gather*}
 \phi
 \ 
 \colon
 [0,\infty)
 \times
 \R
 \ 
 \to
 \ 
 \R
 \,,
 \quad
 (x,x^*)
 \ 
 \mapsto
 \ 
 x\cdot x^*
 \ 
 -
 \ 
 \varphi(x)
 \,.
\end{gather*}
Let $x^*\in\R$.
Since
(by assumption)
      $\varphi$ is continuously differentiable on $(0,\infty)$ with derivative $\varphi^{'}$,
      so is $\phi(\cdot,x^*)$ with derivative
      satisfying 
      \begin{gather*}
        \frac{\partial}{\partial x}
        \phi(x,x^*)
        \ 
        =
 \ 
        x^*
        -
        \varphi^{'}(x)
        \qquad
        \text{for all}\ 
        x\in(0,\infty)
        \,.
      \end{gather*}
  It follows that 
  \begin{gather*}
    z
    \ 
    :=
    \ 
      (
    \varphi^{'}
    )^{-1}
    (x^*)
  \end{gather*}
  is an extreme value of $\phi(\cdot,x^*)$.
  Note, that $x^*\in\R$ is in the domain of 
  $
      (
    \varphi^{'}
    )^{-1}
  $
  by the assumption $\varphi^{'}(0,\infty)=\R$.
  Since $\varphi$ is strictly convex, $\phi(\cdot,x^*)$ is strictly concave. 
  Thus,
  $z>0$ is the unique maximum in $(0,\infty)$.
  By the continuity of $\phi(\cdot,x^*)$ on $[0,\infty)$ it follows, that $z$ is the unique maximum on $[0,\infty)$.
  Thus
  \begin{align*}
    \varphi^*(x^*)
    &
    \ 
    =
    \ 
    \sup_{x\in\R}
    x\cdot x^* - \varphi(x)
    \ 
    =
    \ 
    \sup_{x\in [0,\infty)}
    x\cdot x^* - \varphi(x)
    \ 
    =
    \ 
    \sup_{x\in [0,\infty)}
    \phi(x,x^*)
    \\
    &
    \ 
    =
    \ 
    \phi(z,x^*)
    \\
    &
    \ 
    =
    \ 
    x^*
    \!
    \cdot
    (
    \varphi^{'}
    )^{-1}
    (x^*)
    \ 
    -
    \ 
    \varphi
    \left( 
      (
    \varphi^{'}
    )^{-1}
    (x^*)
    \right)
    \qquad 
    \text{for all}\ 
    x^*\in\R
    \,.
  \end{align*}
  Now we proof the second statement.
  Since
  $
      (
    \varphi^{'}
    )^{-1}
  $
  is (by assumption) continuously differentiable, it holds
  \begin{align}
    \label{0098}
    \begin{split}
    \frac{\partial}{\partial x^*}
     \varphi^*(x^*)
    &
    \ 
    =
    \ 
    (
    \varphi^{'}
    )^{-1}
    (x^*)
    \ 
    +
    \ 
    x^*
    \!
    \cdot
    \frac{\partial}{\partial x^*}
    (
    \varphi^{'}
    )^{-1}
    (x^*)
    \ 
    -
    \ 
    \varphi^{'}
    \left( 
      (
    \varphi^{'}
    )^{-1}
    (x^*)
    \right)
    \cdot
    \frac{\partial}{\partial x^*}
    (
    \varphi^{'}
    )^{-1}
    (x^*)
    \\
    &
    \ 
    =
    \ 
    (
    \varphi^{'}
    )^{-1}
    (x^*)
    \ 
    +
    \ 
    x^*
    \!
    \cdot
    \frac{\partial}{\partial x^*}
    (
    \varphi^{'}
    )^{-1}
    (x^*)
    \ 
    -
    \ 
    x^*
    \cdot
    \frac{\partial}{\partial x^*}
    (
    \varphi^{'}
    )^{-1}
    (x^*)
    \\
    &
    \ 
    =
    \ 
    (
    \varphi^{'}
    )^{-1}
    (x^*)
    \qquad
    \text{for all}\ 
    x^*\in\R
    \,.
    \end{split}
  \end{align}
  Since $\varphi$ is strictly convex and continuously differentiable, 
  $\varphi^{'}$ is continuous and strictly non-decreasing.
  Thus 
  $
    (
    \varphi^{'}
    )^{-1}
  $
  is continuous and strictly non-decreasing.
  It follows from \eqref{0098} that $\varphi^*$ is strictly convex and continuously differentiable.
\end{proof}
The next lemma completes the link.
\begin{lemma}
  \label{9991}
  Let Assumption~\ref{asu:objective_f} hold true. Then 
\begin{gather*}
  \Phi
  \ 
  :
  \ 
  \R^n
  \to
  \ 
  \overline{\R}
  \,
  ,
  \qquad
  [w_1,\ldots,w_n]^\top
  \ 
  \mapsto
  \ 
  \sum_{i=1}^n \varphi(w_i)
  \,,
\end{gather*}
satisfies Assumption~\ref{cv:ts:asu}.
\end{lemma}
\begin{proof}
  By Example~\ref{cv:cc:ex}
  the convex conjugate of $\Phi$ is 
\begin{gather*}
  \Phi^*
  \ 
  :
  \ 
  \R^n
  \to
  \ 
  \overline{\R}
  \,
  ,
  \qquad
  [\lambda_1,\ldots,\lambda_n]^\top
  \ 
  \mapsto
  \ 
  \sum_{i=1}^n \varphi^*(\lambda_i)
  \,,
\end{gather*}
where $\varphi^*$ is the convex conjugate of $\varphi$.
By Assumption~\ref{asu:objective_f} $\varphi$ is strictly convex. Thus,
$\Phi$ is strictly convex. By Lemma~\ref{1065}, $\varphi^*$ continuously differentiable on $\R$. Thus,
$\Phi$ is continuously differentiable on $\R^n$.
It follows the statement of Assumption~\ref{cv:ts:asu} for $\Phi$.
\end{proof}
