\begin{assumption}
  \label{asu:objective_f}
  The objective function $\varphi\colon \R\to\overline{\R}$ of Problem~\ref{bw:1:primal} 
  satisfies the following conditions
  \begin{itemize}
    \item
      $\varphi(0)=0$ and $\varphi(x)=\infty$ for all $x<0$
    \item
      $\varphi$ is strictly convex and continuously differentiable on $(0,\infty)$ with derivative $\varphi^{'}$
    \item
      $\varphi^{'}(0,\infty)=\R$
    \item
      the inverse of the derivative $(\varphi^{'})^{-1}$ is Lipschitz continuous and continuously differentiable on $\R$.
  \end{itemize}
\end{assumption}
The next lemma provides a link to the assumptions on the objective function in Theorem~\ref{cv:ts:th}.
\newpage
\begin{lemma}
  \label{1065}
  Let Assumption\ref{asu:objective_f} hold true. Then the convex conjugate of $\varphi$ (see \eqref{def:convex_conjugate}) is
  \begin{gather*}
    \varphi^*
    \colon
    \R
    \ 
    \to
    \ 
    \R
    \,,
    \quad
    x^*
    \ 
    \mapsto
    \ 
    x^*
    \!
    \cdot
    (
    \varphi^{'}
    )^{-1}
    (x^*)
    \ 
    -
    \ 
    \varphi
    \left( 
      (
    \varphi^{'}
    )^{-1}
    (x^*)
    \right)
    \,.
  \end{gather*}
  Furthermore, $\varphi^*$ is strictly convex and continuously differentiable on $\R$.
\end{lemma}
\begin{proof}
We define
\begin{gather*}
 \phi
 \ 
 \colon
 [0,\infty)
 \times
 \R
 \ 
 \to
 \ 
 \R
 \,,
 \quad
 (x,x^*)
 \ 
 \mapsto
 \ 
 x\cdot x^*
 \ 
 -
 \ 
 \varphi(x)
 \,.
\end{gather*}
Let $x^*\in\R$.
Since
(by assumption)
      $\varphi$ is continuously differentiable on $(0,\infty)$ with derivative $\varphi^{'}$,
      so is $\phi(\cdot,x^*)$ with derivative
      satisfying 
      \begin{gather*}
        \frac{\partial}{\partial x}
        \phi(x,x^*)
        \ 
        =
 \ 
        x^*
        -
        \varphi^{'}(x)
        \qquad
        \text{for all}\ 
        x\in(0,\infty)
        \,.
      \end{gather*}
  It follows that 
  \begin{gather*}
    z
    \ 
    :=
    \ 
      (
    \varphi^{'}
    )^{-1}
    (x^*)
  \end{gather*}
  is an extreme value of $\phi(\cdot,x^*)$.
  Note, that $x^*\in\R$ is in the domain of 
  $
      (
    \varphi^{'}
    )^{-1}
  $
  by the assumption $\varphi^{'}(0,\infty)=\R$.
  Since $\varphi$ is strictly convex, $\phi(\cdot,x^*)$ is strictly concave. 
  Thus,
  $z>0$ is the unique maximum in $(0,1)$.
  By the continuity of $\phi(\cdot,x^*)$ on $[0,\infty)$ it follows, that $z$ is the unique maximum on $[0,\infty)$.
  Thus
  \begin{align*}
    \varphi^*(x^*)
    &
    \ 
    =
    \ 
    \sup_{x\in\R}
    x\cdot x^* - \varphi(x)
    \ 
    =
    \ 
    \sup_{x\in [0,\infty)}
    x\cdot x^* - \varphi(x)
    \ 
    =
    \ 
    \sup_{x\in [0,\infty)}
    \phi(x,x^*)
    \\
    &
    \ 
    =
    \ 
    \phi(z,x^*)
    \\
    &
    \ 
    =
    \ 
    x^*
    \!
    \cdot
    (
    \varphi^{'}
    )^{-1}
    (x^*)
    \ 
    -
    \ 
    \varphi
    \left( 
      (
    \varphi^{'}
    )^{-1}
    (x^*)
    \right)
    \qquad 
    \text{for all}\ 
    x^*\in\R
    \,.
  \end{align*}
  Now we proof the second statement.
  Since
  $
      (
    \varphi^{'}
    )^{-1}
  $
  is (by assumption) continuously differentiable, it holds
  \begin{align}
    \label{0098}
    \begin{split}
    \frac{\partial}{\partial x^*}
     \varphi^*(x^*)
    &
    \ 
    =
    \ 
    (
    \varphi^{'}
    )^{-1}
    (x^*)
    \ 
    +
    \ 
    x^*
    \!
    \cdot
    \frac{\partial}{\partial x^*}
    (
    \varphi^{'}
    )^{-1}
    (x^*)
    \ 
    -
    \ 
    \varphi^{'}
    \left( 
      (
    \varphi^{'}
    )^{-1}
    (x^*)
    \right)
    \cdot
    \frac{\partial}{\partial x^*}
    (
    \varphi^{'}
    )^{-1}
    (x^*)
    \\
    &
    \ 
    =
    \ 
    (
    \varphi^{'}
    )^{-1}
    (x^*)
    \ 
    +
    \ 
    x^*
    \!
    \cdot
    \frac{\partial}{\partial x^*}
    (
    \varphi^{'}
    )^{-1}
    (x^*)
    \ 
    -
    \ 
    x^*
    \cdot
    \frac{\partial}{\partial x^*}
    (
    \varphi^{'}
    )^{-1}
    (x^*)
    \\
    &
    \ 
    =
    \ 
    (
    \varphi^{'}
    )^{-1}
    (x^*)
    \qquad
    \text{for all}\ 
    x^*\in\R
    \,.
    \end{split}
  \end{align}
  Since $\varphi$ is strictly convex and continuously differentiable, 
  $\varphi^{'}$ is continuous and strictly non-decreasing.
  Thus 
  $
    (
    \varphi^{'}
    )^{-1}
  $
  is continuous and strictly non-decreasing.
  It follows from \eqref{0098} that $\varphi^*$ is strictly convex and continuously differentiable.
\end{proof}
The next lemma completes the link.
\begin{lemma}
  Let Assumption~\ref{asu:objective_f} hold true. Then 
\begin{gather*}
  \Phi
  \ 
  :
  \ 
  \R^n
  \to
  \ 
  \overline{\R}
  \,
  ,
  \qquad
  [w_1,\ldots,w_n]^\top
  \ 
  \mapsto
  \ 
  \sum_{i=1}^n \varphi(w_i)
  \,,
\end{gather*}
satisfies Assumption~\ref{cv:ts:asu}.
\end{lemma}
\begin{proof}
  By Example~\ref{cv:cc:ex}
  the convex conjugate of $\Phi$ is 
\begin{gather*}
  \Phi^*
  \ 
  :
  \ 
  \R^n
  \to
  \ 
  \overline{\R}
  \,
  ,
  \qquad
  [\lambda_1,\ldots,\lambda_n]^\top
  \ 
  \mapsto
  \ 
  \sum_{i=1}^n \varphi^*(\lambda_i)
  \,,
\end{gather*}
where $\varphi^*$ is the convex conjugate of $\varphi$.
By Assumption~\ref{asu:objective_f} $\varphi$ is strictly convex. Thus,
$\Phi$ is strictly convex. By Lemma~\ref{1065}, $\varphi^*$ continuously differentiable on $\R$. Thus,
$\Phi$ is continuously differentiable on $\R^n$.
It follows the statement of Assumption~\ref{cv:ts:asu} for $\Phi$.
\end{proof}
