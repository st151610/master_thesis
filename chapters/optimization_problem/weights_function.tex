Theorem~\ref{dual_solution_th} tells us that if an optimal solution
$
(\rho^\dagger,\lambda_0^\dagger,\lambda^\dagger)
$
to Problem~\ref{dual} exists,
then the unique optimal solutions to Problem~\ref{bw:1:primal} are 
\begin{gather*}
  w^\dagger_i
  \ 
  :=
  \ 
  (
  \varphi^{'}
  )^{-1}
  \left(
    \rho^\dagger_i
  \ 
    +
  \ 
\lambda_0^\dagger
  \ 
+
  \ 
\inner
{B(X_i)}
{
\lambda^{\dagger}
}
  \right)
  \qquad
  \text{for all}\ 
  i\in
  \left\{ 1,\ldots,n \right\}
  \,.
\end{gather*}

This point of view is sufficient from a practical point of view
(when we are interested in actually computing the optimal weights).

For the subsequent analysis we need to view the weights as random quantities. To this end, we shall introduce the weights process. 

\begin{definition}
  \label{def:weights_function}
  We define the weights process to be a stochastic process indexed 
  over
  $
  \mathcal{X}
  \,
  \times
  \,
  \R^N_{\ge 0}
  \,
  \times
  \,
  \R
  \times
  \,
  \R^{N}
  $
  with values in $\R^N$ such that
\begin{align*}
  w
  \left( 
  x
  ,
  \rho
  ,
  \lambda_0
  ,
  \lambda
  \right)
  \ 
  :=
  \ 
  \left[ 
    (
    \varphi^{'}
    )^{-1}
    \left( 
     \rho_i 
     +
      \lambda_0
      +
      \inner{B(x)}{\lambda}
    \right)
  \right]_{i\in \left\{ 1, \ldots,N \right\}}
    \,.
\end{align*}
\end{definition}
By the measurability of the basis functions (or the basis processes),
the weights processes is also measurable 
(the quantities 
$
  w
  \left( 
  x
  ,
  \rho
  ,
  \lambda_0
  ,
  \lambda
  \right)
$
are random variables).
The next question is if there are (plausible) assumptions such that the weights processes at a random parameter
$
(\rho^\dagger,\lambda_0^\dagger,\lambda^\dagger)
$
is measurable.
To answer this in a good way we need the argmax measurability theorem \cite[Theorem~18.19]{Aliprantis2007}.
\subsection{Argmax Measurability Theorem}
We follow \cite{Aliprantis2007}
A \textbf{correspondence}$ \psi$ from a set $X$ to a set $Y$ assigns to each $x\in X$ a subset $\psi(x)\subset Y$.
To clarify that we map $x$ to a set, we use the double arrow, that is,
$
  \psi
  \colon
  X
  \twoheadrightarrow
  Y
$.
  Let 
  $(S,\Sigma_S)$ be a measurable space and $X$ a topological space.
  We say, that a correspondence 
  $
  \psi
  \colon
  S
  \twoheadrightarrow
  X
  $
  is 
  \textbf{
  weakly measurable
  },
  if
  \begin{gather*}
    \left\{ 
      s\in S
      \ 
      |
      \ 
      \psi(s)
      \cap
      O
      \neq
      \emptyset
    \right\}
    \in
    \Sigma_S
    \qquad
    \text{for all open subsets}
    \ 
    O\subset X
    \,.
  \end{gather*}

A selector from a correspondence $\psi\colon X\twoheadrightarrow Y$ is a function $s\colon X\to Y$ that satisfies 
$
s(x)\in\psi(x)
$
for all $x\in X$.

\begin{definition}
  Let 
  $(S,\Sigma_S)$ be a measurable space, and let $X$ and $Y$  be topological space.
  A function 
  $f\colon S\times X \to Y$
  is a \textbf{Caratheodory function} if
  \begin{align*}
    f(\cdot,x)
    &
    \colon
    S\to Y
    \qquad
    \text{is}\ 
    (\Sigma_{S},\mathcal{B}(Y))-measurable
    \ 
    \text{for all}
    \ 
    x\in X
    \,,
    \intertext{and}
    f(s,\cdot)
    &
    \colon
    X\to Y
    \qquad
    \text{is continuous for all}\ 
    s\in S
    \,.
  \end{align*}
\end{definition}
\begin{theorem}
  Let $X$ be a separable metrizable space and
  $
  (S,\Sigma_S)
  $
  a measurable space.
  Let $\psi\colon S \twoheadrightarrow X$ be a weakly measurable correspondence with non-empty compact values, and suppose
  $f\colon S\times X \to \R$
  is a Caratheodory function. Define the value function 
  $m\colon S\to \R$ by
  \begin{gather*}
    m(s):=\max_{\psi(s)}f(s,x)
    \,,
  \end{gather*}
  and the correspondence 
  $mu\colon S\twoheadrightarrow X$ of maximizers by
  \begin{gather*}
    \mu(s):= \left\{ 
      x\in \psi(s)
      |
      f(s,x)=m(s)
    \right\}
    \,.
  \end{gather*}
  Then the value function $m$ is measurable, 
  the argmax correspondence $\mu$ has non-empty and compact values,
  is measurable and admits a measurable selector.
\end{theorem}
\begin{proof}
  \cite[Theorem~18.19]{Aliprantis2007}
\end{proof}

Our goal is to find (plausible) assumptions under which we can measurably select optimal solutions of Problem~\ref{dual}.

\begin{assumption}
  There exists $\underline{N}\in\mathbb{N}$ such that 
  for all $N\ge \underline{N}$ 
  there exists a compact and deterministic parameter space
  $
  \Theta_N
  \subset
  \R^N_{\ge 0}
  \times
  \R
  \times
  \R^{N}
  $
  such that for all data sets $D_N$
  a sequence of optimal solution
  $(\rho^\dagger,\lambda_0^\dagger,\lambda^\dagger)$
  satisfying the Lemma
exist and it holds
\begin{gather*}
  (
  \rho^\dagger,
  0_{N-n},
  \lambda_0^\dagger,\lambda^\dagger)
  \in
  \Theta_N
  \,.
\end{gather*}
\end{assumption}

With this assumption we can construct the (constant) correspondence
\begin{align*}
  \psi
  \colon
  (\Omega,\mathcal{A},\P)
  \ 
  \twoheadrightarrow
  \ 
  \R^N_{\ge 0}
  \times
  \R
  \times
  \R^{N}
  \,,
  \qquad
  \omega
  \ 
  \to
  \ 
  \Theta_N
  \,.
\end{align*}
This is weakly measurable, because $\Theta_N$ is deterministic and thus
$\Theta_N\cap O$ is either true or false, that is,
  \begin{gather*}
    \left\{ 
      \omega\in \Omega
      \ 
      |
      \ 
      \psi(\omega)
      \cap
      O
      \neq
      \emptyset
    \right\}
    \ 
    \in
    \ 
    \left\{ \Omega,\emptyset \right\}
    \subset
    \Sigma_S
    \qquad
    \text{for all open subsets}
    \ 
    O
    \subset
  \R^N_{\ge 0}
  \times
  \R
  \times
  \R^{N}
    \,.
  \end{gather*}
  Furthermore the objective function is a Caratheodory function.
  Thus, by Theorem there exists the argmax correspondence  of Problem~\ref{dual} and a measurable selector that selects
  $(\rho^\dagger,0_{N-n},\lambda_0^\dagger,\lambda^\dagger)$ 
  of Assumption.
  With this, we can define the optimal weights processes
  \begin{gather*}
    w^\dagger(x)
    \ 
    :=
    \ 
    w
    \left( 
    x,\rho^\dagger,0_{N-n},\lambda_0^\dagger,\lambda^\dagger
    \right)
    \qquad
    \text{indexed over}\ 
    x\in\mathcal{X}\,.
  \end{gather*}
  Due to the definition of the (general) weights processes and the measurability of the argmax selector, the random variables $w^\dagger(x)$ are measurable for all $x\in\mathcal{X}$.
  To end the measurability discussion, note that $w^\dagger(X)$ is a random variable.


\begin{lemma}
  \label{weights_l_inf}
  Let Assumption~\ref{asu:existence_sol}, 
  Assumption~\ref{asu:basis},
  and Assumption~\ref{asu:objective_f} hold true.
  Then it holds 
  \begin{gather*}
    w_i^\dagger(X)
  \ 
  \in
  \ 
  L^\infty(\P)
  \qquad
  \text{for all}
  \ 
  i\in \left\{ 1,\ldots,N \right\}
  \ 
  \text{
    and
  for all 
  }\ 
  N\ge \underline{N}
  \,.
  \end{gather*}
\end{lemma}
\begin{proof}
  Let $N\ge \underline{N}$.
  By Assumption~\ref{asu:basis} it holds 
  $
    \text{for all}\ 
  $ 
  \begin{gather}
    \label{3967}
    \left| 
     \rho_i 
     +
      \lambda_0
      +
      \inner{B(x)}{\lambda}
    \right|
    \ 
    \lesssim
    \ 
    \norm{(\rho,\lambda_0,\lambda)}_2
    \,.
  \end{gather}
  $
    \text{for all}
    \ 
    x\in\mathcal{X}
    \
    \text{and for all}\ 
    (\rho,\lambda,\lambda_0)
    \in
  \R^N_{\ge 0}
  \times
  \R
  \times
  \R^{N}
  $.
  By Assumption~\ref{asu:existence_sol}, 
  there exists a (compact) parameter space
  $\Theta_N$ around the origin, with $\mathrm{diam}\,  \Theta_N<\infty$, 
  such that for all data sets $D_N$ it holds  $(\rho^\dagger,\lambda^\dagger,\lambda_0^\dagger)\in\Theta_N$.
  By Assumption~\ref{asu:objective_f}, $(\varphi^{'})^{-1}$ is non-decreasing and continuous. Thus
    by \eqref{3967} it holds
  \begin{align*}
    \left| 
    w_i^\dagger(x)
    \right|
    \ 
    \lesssim
    \ 
    \left| 
    (\varphi^{'})^{-1}
    \left( 
      \,
      -
    \norm{(\rho^\dagger,\lambda_0^\dagger,\lambda^\dagger)}_2
    \right)
    \right|
    \ 
    +
    \ 
    \left| 
    (\varphi^{'})^{-1}
    \left( 
      \,
    \norm{(\rho^\dagger,\lambda_0^\dagger,\lambda^\dagger)}_2
    \right)
    \right|
    \ 
    \ 
    <
    \ 
    \infty
  \end{align*}
  for all $x\in\mathcal{X}$
  and all $i\in \left\{ 1,\ldots,n \right\}$
  .
\end{proof}
\begin{lemma}
  \label{w.Z=0}
  Let Assumption~\ref{aa:assumption:2} and Assumption~\ref{aa:assumption:3} hold true.
  Furthermore, 
  let
  $N\ge\underline{N}$, and
  let
  $Z\in L^1(\P)$
  be a random variable that is independent of $D_N$ 
  with
  $
\E
\left[
  Z
  \,
  |
  \, 
  X
\right]
= 0
  $
  almost surely.
  It holds
  \begin{gather*}
  \E
  \left[
  w_i(X,\rho^\dagger,\lambda^\dagger,\lambda_0^\dagger)
  \cdot Z
  \right]
  =0
  \qquad
  \text{for all}
  \ 
  i\in \left\{ 1,\ldots,n \right\}
  \,.
  \end{gather*}
\end{lemma}
\begin{proof}
  Let
  $N\ge\underline{N}$.
  We write
  \begin{gather*}
  w_i(X,\rho^\dagger,\lambda^\dagger,\lambda_0^\dagger)
  \ 
  =
  \ 
  w^\dagger(X)
  \end{gather*}
  and ignore the index $i$.
  By Lemma~\ref{weights_l_inf} and 
  $Z\in L^1(\P)$
  it holds
  \begin{gather}
    \label{9876}
    \norm{
  w^\dagger(X)\cdot Z
    }_{L^1(\P)}
    \ 
  \le
    \ 
  \norm{w^\dagger(X)}_{L^\infty(\P)}
  \norm{Z}_{L^1(\P)}
  \ 
  <
  \ 
  \infty
  \,.
  \end{gather}
  By 
  \eqref{9876},
  $Z\perp D_N$
  and
  $
\E
\left[
  Z
  \,
  |
  \, 
  X
\right]
= 0
  $
  almost surely
  it holds 
  \begin{align*}
    \E
  \left[
  w^\dagger(X)
  \cdot
  Z
  \,
  |
  \,
  D_N,X
  \right]
  &
  \ 
  =
  \ 
  w^\dagger(X)
  \cdot
  \E
  \left[
  Z
  \,
  |
  \,
  D_N,X
  \right]
  \\
  &
  \ 
  =
  \ 
  w^\dagger(X)
  \cdot
  \E
  \left[
  Z
  \,
  |
  \,
  X
  \right]
  \
  =
  \ 
  0
  \qquad
  \text{almost surely.}
  \end{align*}
  Thus
  \begin{gather*}
    \E
    \left[
  w^\dagger(X)
  \cdot
  Z
  \,
    \right]
    \ 
    =
    \ 
    \E
    \left[
 \E
  \left[
  w^\dagger(X)
  \cdot
  Z
  \,
  |
  \,
  D_N,X
  \right]
    \right]
    \ 
    =
    \ 
    0
    \,.
     \end{gather*}
\end{proof}

