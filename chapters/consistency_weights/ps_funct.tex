We defined the propensity score in \eqref{def:ps}.
By assumption~\eqref{asu:treatment_asign_str_ing} 
the inverse propensity score $1/\pi(X)$ is a well defined random variable that has good balancing properties. 
The next lemma shows what effect the 
\textbf{propensity score weights}
$T/\pi(X)$ have on other functions.
\begin{lemma}
  \label{lem:ps_weights}
  \label{ps_weights_lemma}
  Let
  $
  g_1\colon
  \mathcal{X}\to\R
  $
  and
  $
  g_2\colon
  \mathcal{Y}\to\R
  $
  be a measurable functions such that $g_1(X)\in L^\infty(\P)$ and let \eqref{asu:treatment_asign_str_ing} hold true.
  It holds
  \begin{enumerate}[label=(\roman*)]
    \item
  \begin{gather*}
    \E
    \left[
    \frac{T}{\pi(X)}
    g_1(X)
    \right]
    \ 
    =
    \ 
    \E
    \left[
    g_1(X)
    \right]
    \,.
  \end{gather*}
  \item
  \begin{gather*}
    \E
    \left[
    \frac{T}{\pi(X)}
    g_2(Y(T))
    \right]
    \ 
    =
    \ 
    \E
    \left[
    f(Y(1))
    \right]
    \,.
  \end{gather*}
  \end{enumerate}
 \end{lemma}
 \begin{proof}
   It holds $\pi(X)>0$ by assumption. Thus, $1/\pi(X)$ is a well defined random variable.
Since $g_1(X)\in L^\infty(\P)$
it holds
\begin{align*}
    \E
    \left|
    \frac{T}{\pi(X)}
    g_1(X)
    \right|
    \ 
    \le
    \ 
    \E
    \left[
    \frac{T}{\pi(X)}
    \right]
    \norm{
    g_1(X)
  }_{L^\infty(\P)}
    \ 
  =
    \ 
    \norm{
    g_1(X)
  }_{L^\infty(\P)}
    \ 
  <
    \ 
  \infty
  \,.
\end{align*}
Thus, by the properties of conditional expectation it holds
\begin{align*}
    \E
    \left[
    \frac{T}{\pi(X)}
    g_1(X)
    \right]
    \ 
    =
    \ 
    \E
    \left[
      \E[T\,|\,X]
    \frac{g_1(X)}{\pi(X)}
    \right]
    \ 
    =
    \ 
    \E[g_1(X)]
    \,.
\end{align*}
This proves \textit{(i)}.
For \textit{(ii)}, note that
\begin{align}
  \begin{split}
    \E
    \left[ 
g_2(Y(T))
      \frac{T}{\pi(X)}
    \right]
  &\ =\ 
  \E
  \left[ 
g_2(Y(1))
    \,
    /
    \,
    \pi(X)
    \,
    \vert
    \,
    T=1
  \right]
  \cdot
  \P[T=1]
  \\
  &\ =\ 
  \int_\mathcal{X}
  \E
  \left[ 
g_2(Y(1))
    \,
    \vert
    \,
    X=x
    ,
    T=1
  \right]
  \cdot
  (
  \P[T=1]
  \,
  /
  \,
  \pi(x)
  )
  \,
  \P_{X|T}(dx\,|\,1)
  \\
  &\ =\ 
  \int_\mathcal{X}
  \left[ 
g_2(Y(1))
    \vert
    X=x
  \right]
  \P_X(dx)
  =
  \E[
g_2(Y(1))
  ]
\,.
\end{split}
\end{align}
The first, second and last equality stem from 
$
  T\in \left\{ 0,1 \right\}
$,
and the law of total expectation, applied with $T$ and $X$.
The fourth equality is justified by 
\eqref{asu:treatment_asign_str_ing}.
The density transformation is due to Bayes's Theorem.
 \end{proof}
 Before we go on, we make some assumptions on the inverse propensity score that we will need in Chapter~5.
 To this end, let 
 \begin{align*}
   J_N
   :=
   \left\{ j\in\mathbb{N}\colon
     \P[X\in A_{n,j}]>0
   \right\}
   \qquad
   \text{for all}\ 
   N\in \mathbb{N}
   \,.
 \end{align*}
 Note that we define the function space $C^\alpha_M(\mathcal{Z})$ in
\eqref{def:c_alpha}.
Let $\mathrm{cl}\, A$ denote the closure of a set $A\subset \R^d$.
\index{$\mathrm{cl}(\cdot)$, closure of a set}
 \begin{assumption}
   \label{asu:x_finite}
   It holds
   \begin{enumerate}[label=(\roman*)]
     \item 
       $\# J_N\le \# \mathcal{X}<\infty$ for all $N\in\mathbb{N}$
\item
  For all $N\in\mathbb{N}$ there exist $(M_{N,j})_{j\in J_N}$ such that $\infty>M_{N,j}\ge 0$ for all $j\in J_N$, and 
  $\frac{1}{\pi(\cdot)}\in C^\alpha_{M_{N,j}}(\mathrm{cl}\,A_{N,j})$ for all $(j,N)\in J_N\times \mathbb{N}$, with $\alpha>d/2$.
   \end{enumerate}
 \end{assumption}
 \begin{remark}
\index{$\mathcal{X}$, covariate space}
   Assumption~\ref{asu:x_finite}.\textit{(i)} says that the covariate space $\mathcal{X}\subset \R^d$ is finite.
   We need this to derive bracketing numbers in Lemma~\ref{lem:x_finite}.
   Assumption~\ref{asu:x_finite}.\textit{(ii)} is a regularity condition on the inverse propensity score function restricted to the (finite) partition cells covering $\mathcal{X}$.
   We need this to derive bracketing numbers in Lemma~\ref{lem:br_n_st}.
   Note that by the finiteness of $\mathcal{X}$ condition \textit{(ii)} is met, for example, by a logistic regression model.  
 \end{remark}
 We finish with a lemma about uniform continuity.
 \begin{lemma}
   \label{lem:ips_unif_cont}
   Let Assumption~\eqref{asu:treatment_asign_str_ing} and Assumption~\ref{asu:x_finite} hold true.
   Then the function 
   \begin{align*}
     x\mapsto 
     \varphi^{'}
     \left( \frac{1}{\pi(x)} \right)
   \end{align*}
   is uniformly continuous on all $A_{N,j}$ with $j\in J_N$.
 \end{lemma}
 \begin{proof}
   That the function is well defined follows from \eqref{asu:treatment_asign_str_ing}.
   The uniform continuity follows from the continuity of $1/\pi$ on the bounded 
   and closed sets $\mathrm{cl} A_{N,j}$ and the uniform continuity of 
   $
   \varphi^{'}
   $ (see Lemma~\ref{lem:obj_f}).
 \end{proof}
