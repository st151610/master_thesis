The formulation of the Theorems is always "If the optimal solution" exists.
Next, we want to become independent of this assumption.
With hindsight, we want optimal solutions to estimate an oracle parameter $\lambda^*$ well. 
To this end, we will define this parameter and show that it takes values in a compact set.
We shall extend this set with a 1-tube to obtain a deterministic compact set (or a constant correspondence).
Then Theorem~\ref{th:argmax} gives us a measurable solution of the restricted Problem~\ref{dual}.
If the solution is in the interior, it is a global solution, if it is on the boundary there may be no global solution.
Thus we call it the restricted, or the pseudo solution.
Consider 
the random variable
\begin{align*}
  \left(
  0_{N},0,
  \left[
  \varphi^{'}
  \left(
  \frac
  {1}
  {\pi(X_i)}
  \right)
\right]_{i\in \left\{
  1,\ldots,N
\right\}}
  \right)
\end{align*}
with values in $\R^N_{\ge 0}\times \R\times \R^N$.
If there exists $C_{\pi}>0$, such that $\pi(x)\ge C_\pi$ for all $x\in\mathcal{X}$, 
then 
the random variable has values in a compact set. 
We define $\Theta_N$ to be the 1-tube around this set. 
Clearly
\begin{align*}
  \left(
  0_{N},0,
  \varphi^{'}
  \left(
  \frac
  {1}
  {\pi(X)}
  \right)
  \right)
  \in 
  \Theta_N
  ,.
\end{align*}
Since $\Theta_N$ is deterministic, there exists a measurable solution to Problem~\ref{dual} restricted to $\Theta_N$.
There are two cases. If the solution is on the boundary of $\Theta_N$, we don't now if it is also a global solution.
If it is in the interior, we know, that it is also a global solution.
We will prove later, that with probability going to 1 the latter will be the case.
Furthermore, we will prove, that the (measurable) solution converges to the random variable in probability.
