The formulation of the Theorems is always "If the optimal solution" exists.
Next, we want to become independent of this assumption.
To this end, we will define a compact parameter space $\Theta_N$ and restrict Problem~\ref{dual} to this search space.  
We shall see that a (pseudo) solution in $\Theta_N$ always exists and is measurable. 
In the next sections we will prove convergence to the propensity score. Thus the following choice for $\Theta_N$ is natural. 

Consider 
the random variable
\begin{align*}
  \left(
  0_{N},0,
  \varphi^{'}
  \left(
  \frac
  {1}
  {\pi(X)}
  \right)
  \right)
\end{align*}
If there exists $C_{\pi}>0$, such that $\pi(x)\ge C_\pi$ for all $x\in\mathcal{X}$, 
then 
the random variable has values in a compact set. 
We define $\Theta_N$ to be the 1-tube around this set. 
Clearly
\begin{align*}
  \left(
  0_{N},0,
  \varphi^{'}
  \left(
  \frac
  {1}
  {\pi(X)}
  \right)
  \right)
  \in 
  \Theta_N
  ,.
\end{align*}
Since $\Theta_N$ is deterministic, there exists a measurable solution to Problem~\ref{dual} restricted to $\Theta_N$.
There are two cases. If the solution is on the boundary of $\Theta_N$, we don't now if it is also a global solution.
If it is in the interior, we know, that it is also a global solution.
We will prove later, that with probability going to 1 the latter will be the case.
Furthermore, we will prove, that the (measurable) solution converges to the random variable in probability.
