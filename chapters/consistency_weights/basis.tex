Let $
\left(
\mathcal{P}_N
\right)
$
denote a sequence of countable, $\mathcal{B}$-measurable partitions 
\begin{align*}
\mathcal{P}_N= \left\{
  A_{N,1},
  A_{N,2},
  \ldots
\right\}
\subset \mathcal{B}(\R^d)
\end{align*}
of $\R^d$, that is, 
\begin{align*}
  A_{N,i}\cap A_{N,j}=\emptyset
  \qquad
  \text{if}\ i\neq j
  \qquad
  \text{and}
  \qquad 
  \bigcap_{i\in\mathbb{N}}A_{N,i}
  \ 
  =
  \ 
  \R^d
  \,.
\end{align*}
We define
$ A_N(x) $ to be the cell of $ \mathcal{P}_N $ containing $x$, that is,
\begin{align*}
  A_N
  \colon
  \R^d 
  \ 
  \twoheadrightarrow 
  \ 
  \R^d  
  \,,\qquad
  x
  \ 
  \mapsto
  \ 
  A_N(x)
  \,,
\end{align*}
where $A_N(x)$ is the only cell containing $x$. 

Next, we define the (empirical) basis functions vector
\begin{align}
  \label{def:basis}
  B\colon
  \R^d\times \R^{d\cdot N}
  \ 
  \to
  \ 
  \R
  \,,
  \qquad
  (x,(x_1,\ldots,x_N))
  \ 
  \mapsto
  \ 
  \frac
  {
    \left[
    \mathbf{1}
    _{
      A_N(x)
    }
    (x_k)
    \right]
    _{k\in \left\{
        1,\ldots,N
    \right\}}
  }
  {
    \sum_{j=1}^N
    \mathbf{1}
    _{
      A_N(x)
    }
    (x_j)
    }
  \,,
\end{align}
where we keep to the convention $"0/0=0"$.
We shall extend $B$ to depend on the random vectors
$X,X_1,\ldots,X_N$.
The next lemma studies the measurability of the extensions.
\begin{lemma}
  \quad
  \begin{enumerate}[label=(\roman*)]
\item
  $B(\cdot,(X_1,\ldots,X_N))(\omega)$ is 
  $\left(
    \mathcal{B}(\R^d),\mathcal{B}(\R^N)
  \right)$-measurable
  and
  constant on each cell 
  $A_N\in\mathcal{P}_N$
  for all $\omega\in\Omega$. 
\item
  $B(X,(X_1,\ldots,X_N))$ is $\left(
    \mathcal{A},\mathcal{B}(\R^N)
  \right)$-measurable. 
  \end{enumerate}
\end{lemma}

\begin{proof}
Consider
for $k\in \left\{
  1,\ldots,N
\right\}$
and $\omega\in\Omega$
the indicator function
\begin{align}
  \label{33342}
  \mathbf{1}
  _
  {A_N(X_k(\omega))}
  \colon \R^d\to \left\{
    0,1
  \right\}
  \,.
\end{align}
Since 
$
  {A_N(X_k(\omega))}
  \in\mathcal{B}(\R^d)
$
this is a 
  $\left(
    \mathcal{B}(\R^d),\mathcal{B}(\R)
  \right)$-measurable
  function.
  From the definition of $B$ \eqref{def:basis} it follows the first part of (i).
  Since the indicator function in \eqref{33342} is 1 if $
  x\in
  {A_N(X_k(\omega))}
  $
  and 0 else, it is also constant on each cell
  $A_N\in\mathcal{P}_N$.
  It follows (i).
  To prove (ii), note that
\begin{align*}
  \mathbf{1}
  _
  {A_N(X_k(\omega))}(X(\omega))
  \ 
  =
  \ 
  \mathbf{1}
  \bigcup_{i\in\mathbb{N}}
  \left\{
    X,X_k \in A_{N,i}
  \right\}
  (\omega)
  \qquad
  \text{for all}\ 
  \omega\in\Omega
  \,,
\end{align*}
and
$
  \bigcup_{i\in\mathbb{N}}
  \left\{
    X,X_k \in A_{N,i}
  \right\}
  \in\mathcal{A}
$ by the 
$
\left( 
  \mathcal{A},
  \mathcal{B}(\R^d)
\right)
$-
measurability of $X$ and $X_k$.
\end{proof}
