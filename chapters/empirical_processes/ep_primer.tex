Let 
$
(
\mathbb{D}
,
d
)
$
be a metric space, and let 
$
(\P_n)_{n\in \mathbb{N}}
  \P
$
be (Borel) probability measures
on
$
(
\mathbb{D}
,
\mathcal{D}
)
,
$
where 
$\mathcal{D}$
is the Borel $\sigma$-algebra on $\mathbb{D},$
the smallest $\sigma$-algebra
containing all open sets.
Then the sequence 
$\P_n$
\textbf{converges weakly}
to 
$\P,$
which we denote as $\P_n \rightsquigarrow \P,$
if and only if 
\begin{gather}
  \int_\mathbb{D}
  f
  \text{d}
  \P_n
  \to
  \int_\mathbb{D}
  f
  \text{d}
  \P
  \qquad
  \text{for all}
  \ 
  f
  \in
  C_b(\mathbb{D})
  .
\end{gather}
Here 
$
  C_b(\mathbb{D})
$
denotes the set of all bounded, continuous, real functions on $\mathbb{D}.$
Equivalently, if 
$X_n$ and $X$
are 
$\mathbb{D}$-valued
random variables with distribution 
$\P_n$ and $\P$
respectively, then 
$X_n \to X$
if and only if 
\begin{gather}
  \E[f(X_n)]
  \to
  \E[f(X)]
  \qquad
  \text{for all}
  \ 
  f
  \in
  C_b(\mathbb{D})
  .
\end{gather}

This definitions yield the classical theory of weak convergence.
For a modern treatment see \cite{Klenke2020}.

The classical theory requires that 
$\P_n$
is defined, for each $n\in \mathbb{N},$
on the Borel $\sigma$-algebra $\mathcal{D},$
or, equivalently, that $X_n$ is a Borel measurable map for each $n\in \mathbb{N}.$
If 
$
(
\Omega_n,
\mathcal{A}_n,
\P_n
)
$
are the underlying probability spaces on which the maps 
$X_n$
are defined, this means that
$X_n^{-1}(D)\in \mathcal{A}_n$
for every Borel set $D \in \mathcal{D}.$
This required measurability usually holds when $\mathbb{D}$
is a separable metric space such as $\R^k$
or $C([0,1])$ with the supremum metric.

However, this apparently modest requirement can and does easily fail when the metric space $\mathbb{D}$ is not separable.

\begin{example}
  \emph{\cite[Problem~1.7.3]{vaart2013}}
  Let $\mathbb{D}=D([0,1])$ be the \textbf{Skorohod space} of all right-continuous functions on $[0,1]$
  with left limits endowed with the metric induced by the supremum norm.
  Define 
  $
    X:
    [0,1]
    \to
    \mathbb{D}
    ,\ 
    \omega
    \mapsto
    \mathbf{1}_{[\omega,1]}
    .
  $
  If we equip $[0,1]$ with the Borel $\sigma$-algebra 
  $\mathcal{B}([0,1])$, then 
  $X$ is not measurable. To see this, let $B_s$ be the open ball of radius $1/2$ in $\mathbb{D}$ around the function $\mathbf{1}_{[s,1]}.$
  Now $X(\omega)\in B_s$
  if and only if $\omega=s.$ Indeed, if $\omega\neq s$ there exists an $x$ between $\omega$ and $s$ such that the difference of the indicator functions is 1 at $x$. Conversely, if the distance is greater than
  $1/2$ at a point $x\in [0,1]$, it is because $x$ lies between $\omega$ and $s$ and the indicator functions have difference 1.
  Since arbitrary (even uncountable) unions of open sets are open,
  we get for every $S\subseteq [0,1]$ the open set 
  $
  G
  :=
  \bigcup_{s\in S}
    B_s
    \in \mathcal{D}
    .
  $
  It follows
  $
  X^{-1}(G)=S 
  \ 
  \text{for all}
  \ 
  S \subseteq [0,1]
  .
  $
  Since not all subsets of $[0,1]$ are measurable, we have
  $
  X^{-1}(\mathcal{D})\nsubseteq \mathcal{B}([0,1])
.
$
But then $X$ is not measurable. The $\sigma$-algebra $\mathcal{D}$ is to large.
\end{example}



Let 
$
  \left( 
    \Omega,
    \mathcal{A},
    \P
  \right)
$
be a probability space
and
$
  \left( 
    \mathcal{X},
    \Sigma
  \right)
$
a measurable space.
Let 
$
  X_j
  :
  \left( 
    \Omega,
    \mathcal{A},
    \P
  \right)
  \to
  \left( 
    \mathcal{X},
    \Sigma
  \right)
  ,
  j=1,\ldots, n
$ be independent and identically-distributed (i.i.d.)
random variables
with probability distribution $\P_X$ 
and
$\mathcal{F}$ a family of measurable functions
$
  f:
  \left( 
    \mathcal{X},
    \Sigma
  \right)
    \to
  \left( 
    \R,
    \mathcal{B}(\R)
  \right)
$.
Consider the map
\begin{gather}
  f
  \mapsto
  G_n f
  :=
  \sqrt{n}
  \left( 
    \frac{1}{n}
    \sum_{i = 1}^{n}
      f(X_i)
    -
    \P_X f
  \right),
\end{gather}
where
$
  \P_X f 
  :=
  \int_\mathcal{X} f \text{d}\P_X.
$
We call 
$
  \left( 
    G_n f
  \right)_{f \in \mathcal{F}}
$
the empirical process indexed by $\mathcal{F}$.
Furthermore
\begin{gather}
  \norm{G_n f}_\mathcal{F}
  :=
  \sup_
        { f \in \mathcal{F}}
        \left|
          G_n f
        \right|
        .
\end{gather}


