
Let 
$
  \left( 
    \Omega,
    \mathcal{A},
    \P
  \right)
$
be a probability space,
$
  \left( 
    \mathcal{X},
    \Sigma
  \right)
$
a measurable space, and 
$
  X_1,\ldots,X_n
  :
  \left( 
    \Omega,
    \mathcal{A},
    \P
  \right)
  \to
  \left( 
    \mathcal{X},
    \Sigma
  \right)
$
a sample 
of independent and identically-distributed
random variables
with probability distribution $\P_{\!X}$.
Throughout this section we consider the~\textbf{empirical~measure}
of this sample, that is, the discrete random measure
\begin{gather}
  \P_{\!n}:\Sigma \to [0,1]
  ,
  \quad
  C\mapsto 
  \frac{1}{n}
  \#\left\{ 
1\le i \le n \colon
X_i \in C
  \right\}
  \,.
\end{gather}
A family $\mathcal{F}$ of measurable functions 
$
  f:
  \left( 
    \mathcal{X},
    \Sigma
  \right)
    \to
  \left( 
    \R,
    \mathcal{B}(\R)
  \right)
$
induces a stochastic process by
\begin{gather}
  f\mapsto \P_{\!n} f\,,
\end{gather}
where for a measure $Q$ on 
$
  \left( 
    \mathcal{X},
    \Sigma
  \right)
$
we denote $Q f:= \int_\mathcal{X}f\, Q(dx)$.
In this way we define the $\mathcal{F}\!$-indexed \textbf{empirical process} $\G_n$ by
\begin{gather}
  f
  \ 
  \mapsto
  \ 
  \G_n f 
  \ 
  :=
  \ 
  \sqrt{n}
  (\P_{\!n}-\P)f 
  \ 
  =
  \ 
  \frac{1}{\sqrt{n}}
  \sum_{i=1}^{n} 
  (
    f(X_i)
    -
    \P f
  )
  \,.
\end{gather}
The purpose of this notation is to abstract the behaviour of $\G_n$ ranging over $\mathcal{F}$.
Conforming with this integral viewpoint, we define the (random) norm
\begin{gather}
  \norm{\G_n}_\mathcal{F}
  :=
  \sup_
        { f \in \mathcal{F}}
        \left|
          G_n f
        \right|
        .
\end{gather}
We stress that 
$
  \norm{\G_n}_\mathcal{F}
$
often ceases to be measurable, even in simple situations~\cite[page 3]{vaart2013}.
To deal with this, we introduce the notion of \textbf{outer expectation} $\E^*$, that is,
\begin{gather}
  \E^*[T]
  \ 
  :=
  \ 
    \inf
  \left\{ 
    \E[U]
  \ 
  \lvert
  \ 
    U\ge T,
    \ 
    U:
  \left( 
    \Omega,
    \mathcal{A},
    \P
  \right)
  \to 
  \left( 
    \overline{\R},
    \mathcal{B}(\overline{\R})
  \right)
  \text{measurable and}
  \ 
  \E[U]<\infty
  \right\}
  \,.
\end{gather}
In our application the technical difficulties halt at this point, because we only consider $T$ with $\E^*[T]<\infty$. Then there exists a smallest measurable function $T^*$ dominating $T$ with
$\E^*[T]=\E[T^*]$. Thus, we may assume $T$ to be measurable in this regard.

In our application we need concentration inequalities for 
$
  \norm{\G_n}_\mathcal{F}
$.
One easy way is to use maximal inequalities for the expectation together with Markov's inequality. There are also Bernstein-like inequalities for empirical processes.



