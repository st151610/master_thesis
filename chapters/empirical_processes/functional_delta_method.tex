\begin{definition}
  A map 
  $
  \phi:
  \mathbb{D}_\phi
  \to 
  \mathbb{E}
  ,
  $
  defined on a subset 
  $
  \mathbb{D}_\phi
  $
  of a normed space
  $\mathbb{D}$
  that contains 
  $\theta,$
  is called 
  \textbf{Hadamard diffenertiable}
  at $\theta$
  if there exists a continuous,
  linear map
  $
  \phi_\theta^{'}
    :
    \mathbb{D}
    \to 
    \mathbb{E}
  $
  such that
  \begin{gather}
    \norm{
      \frac{
        \phi(\theta + t h_t)
        -
        \phi(\theta)
      }{
        t
      }
      -
      \phi^{'}_\theta
      (h)
    }_\mathbb{E}
    \to
    0
    \quad
    \text{as}
    \ 
    t\searrow 0
    \ 
    \text{for all}
    \ 
    h_t \to h
  \end{gather}
  $
    \text{such that $\theta + th_t$ is contained in $\mathbb{D}_\phi$ for all small $t>0.$}
  $
\end{definition}


\begin{ftheorem}
  \emph{(Delta Method)}
  Let 
  $
    \mathbb{D}
    \ \text{and}
    \ 
    \mathbb{E}
  $
  be normed linear spaces.
  Let
  $
    \phi
    :
    \mathbb{D}_\phi
    \subseteq
    \mathbb{D}
    \to
    \mathbb{E}
  $
  be Hadamard differentiable it $\theta$
  tangentially to 
  $\mathbb{D}_0.$
  Let
  $
    T_n
    :
    \Omega_n
    \to
    \mathbb{D}_\phi
  $
  be maps such that 
  $
    r_n
    (T_n - \theta)
    \rightsquigarrow
    T
  $
  for some sequence of numbers $r_n \to \infty$
  and a random element $T$
  that takes its values in $\mathbb{D}_0.$
  Then 
  $
    r_n(\phi(T_n)-\phi(\theta))
    \rightsquigarrow
    \phi^{'}
    _\theta
    (T)
    .
  $
  If 
  $
    \phi^{'}
    _\theta
  $
  is defined and continuous on the whole space $\mathbb{D},$
  then we also have 
  $
    r_n
    (
      \phi(T_n)
      -
      \phi(\theta)
    )
    =
    \phi^{'}
    _\theta
    (
    r_n
    (
    T_n
    -
    \theta
    )
    )
    +
    o_\P
    (1)
    .
  $
\end{ftheorem}
\begin{proof}
  \cite[Theorem~20.8]{Vaart1998}
\end{proof}
