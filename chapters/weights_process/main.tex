In the formulation of Theorem~\ref{dual} we encounter “If there exists the optimal solution $(\rho^\dagger,\lambda_0^\dagger,\lambda)$ ... " .
To be able to study asymptotic properties of the weights, we 
shall assume that Problem~\ref{dual} is feasible,
construct a measurable dual solution, and plug it in $(\varphi^{'})^{-1}$.
Before we formulate concrete assumptions, we provide tools from functional analysis
to obtain measurability. Afterwards, we tailor the feasibility assumptions to the capability of this tools.
Then, we interpose a section on basis functions before we construct the weights process - the theoretical analogy of optimal weights.
\section{Argmax Measurability Theorem}
  We follow \cite{Aliprantis2007}.
A \textbf{correspondence} $\psi$ from a set $S_1$ to a set $S_2$ assigns to each $s_1\in S_1$ a subset $\psi(s_1)\subset S_2$.
To clarify that we map $s_1$ to a set, we use the double arrow, that is,
$
  \psi
  \colon
  S_1
  \twoheadrightarrow
  S_2
$.
  Let 
  $(\mathcal{Z},\Sigma_{\mathcal{Z}})$ be a measurable space and $\mathcal{S}$  a topological space.
  We say, that a correspondence 
  $
  \psi
  \colon
  \mathcal{Z}
  \twoheadrightarrow
  \mathcal{S}
  $
  is 
  \textbf{
  weakly measurable
  },
  if
  \begin{gather*}
    \left\{ 
      z\in \mathcal{Z}
      \ 
      |
      \ 
      \psi(z)
      \cap
      O
      \neq
      \emptyset
    \right\}
    \in
    \Sigma_{\mathcal{Z}}
    \qquad
    \text{for all open subsets}
    \ 
    O\subset \mathcal{S}
    \,.
  \end{gather*}
  A \textbf{selector} from a correspondence $\psi\colon \mathcal{Z}\twoheadrightarrow \mathcal{S}$ is a function $s\colon \mathcal{Z}\to \mathcal{S}$ that satisfies 
  \begin{align*}
s(z)\in\psi(z)
\qquad
\text{for all}\ 
z\in\mathcal{Z}
\,.
  \end{align*}
  

\begin{definition}
  Let 
  $(\mathcal{Z},\Sigma_{\mathcal{Z}})$ be a measurable space, and let $\mathcal{S}_1$ and $\mathcal{S}_2$  be topological space.
  A function 
  $f\colon \mathcal{Z}\times \mathcal{S}_1 \to \mathcal{S}_2$
  is a \textbf{Caratheodory function} if
  \begin{align*}
    f(\cdot,s_1)
    &
    \colon
    \mathcal{Z}\to \mathcal{S}_2
    \qquad
    \text{is}\ 
    (\Sigma_{\mathcal{Z}},\mathcal{B}(\mathcal{S}_2))-measurable
    \ 
    \text{for all}
    \ 
    s_1\in \mathcal{S}_1
    \,,
    \intertext{and}
    f(z,\cdot)
    &
    \colon
    \mathcal{Z}\to \mathcal{S}_2
    \qquad
    \text{is continuous for all}\ 
    z\in \mathcal{Z}
    \,.
  \end{align*}
\end{definition}
\begin{theorem}
  \label{th:argmax}
  Let $\mathcal{S}$ be a separable metrizable space and
  $
  (\mathcal{Z},\Sigma_{\mathcal{Z}})
  $
  a measurable space.
  Let $\psi\colon \mathcal{Z} \twoheadrightarrow \mathcal{S}$ be a weakly measurable correspondence with non-empty compact values, and suppose
  $f\colon \mathcal{Z}\times \mathcal{S} \to \R$
  is a Caratheodory function. Define the value function 
  $m\colon \mathcal{Z}\to \R$ by
  \begin{gather*}
    m(z):=\max_{s\in\psi(z)}f(z,s)
    \,,
  \end{gather*}
  and the correspondence 
  $\mu\colon \mathcal{Z}\twoheadrightarrow \mathcal{S}$ of maximizers by
  \begin{gather*}
    \mu(z):= \left\{ 
      s\in \psi(z)
      |
      f(z,s)=m(z)
    \right\}
    \,.
  \end{gather*}
  Then the value function $m$ is measurable, 
  the argmax correspondence $\mu$ has non-empty and compact values,
  is measurable and admits a measurable selector.
\end{theorem}
\begin{proof}
  \cite[Theorem~18.19]{Aliprantis2007}
\end{proof}
\begin{takeaways}
  Solving an optimization problem, that has a Caratheodory objective function, on a weakly-measurable, non-empty and compact search space, allows for measurable optimal solutions.
\end{takeaways}

\section{Measurable Dual Solution}
  Next, we formulate the feasibility assumption. The assumption is (asymptotically) justified by Theorem~\ref{th:cons_dual}.
Note that we assume compactness to be able to apply Theorem~\ref{th:argmax}.
\begin{assumption}
  \label{asu:feas_dual_sol}
  For all $N\in\mathbb{N}$ there exists a non-empty, compact, and deterministic 
  parameter space 
  $
  \Theta_N
  \subset
  \R^{N}_{\ge 0}
  \times
  \R
  \times
  \R^N
  $
  such that the optimal solution 
  $
  \left( \rho^\dagger,\lambda_0^\dagger,\lambda^\dagger \right)
  $
  of Problem~\ref{dual}
  are contained in $\Theta_N$.
\end{assumption}
Based on this assumption it is easy to derive measurability for the dual solutions 
  $
  \left( \rho^\dagger,\lambda_0^\dagger,\lambda^\dagger \right)
  $.
  To this end, we take a closer look at the objective function.
  \begin{definition}
    \label{def:rand_obj_f}
We define the (random) objective function of
Problem~\ref{dual} by
  \begin{align*}
    &
  G
  \ 
  \colon
  \ 
  \left(
  \Omega,
\sigma
(D_N)
  \right)
  \times
  \left(
  \R^N_{\ge 0}
  \times
  \R
  \times
  \R^{N}
  \right)
  \ 
  \to
  \ 
  \overline{\R}
  \intertext{with}
    &
  G(\omega,(\rho,\lambda_0,\lambda))
  \ 
  =
  \ 
  \infty
  \qquad 
  \text{if}\quad 
  \rho_i
  \neq
  \left[ 
  \varphi^{-1}
  (0)
  -
  \left( 
  \lambda_0 + \inner{B(X_i)}{\lambda}
  \right)
  \right]^+
  \
  \text{for some}\ i>n
  \,,
  \\
  \intertext{and else}\qquad
  &
  G(\omega,(\rho,\lambda_0,\lambda))
  \\
  &
  \ 
  =
  \ 
  \frac{1}{N}
\sum_{i=1} 
  ^N
  \Big[
  T_i(\omega)
  \cdot
  \varphi^*
  \!
  \left( 
    \rho_i
    +
\lambda_0
+
\inner
{B(X_i)(\omega)}
{
\lambda
}
  \right)
  \ 
  -
  \ 
\lambda_0
-
\inner
{B(X_i)(\omega)}
{
\lambda
}
\Big]
\\
&
  \qquad 
+
\ 
\inner
{\delta(\omega)}
{
  |\lambda|
}
\,.
  \end{align*}
  \end{definition}
  \begin{lemma}
    \label{lem:caratheo_G}
    The function $G$ of Definition~\ref{def:rand_obj_f}
    is Caratheodory.
  \end{lemma}
  \begin{proof}
    This follows from Lemma~\ref{1165}
    (continuity of $\varphi^*$) and the measurability 
  of all random variables included.
  \end{proof}
  In the proof of the next lemma we gather the arguments and apply Theorem~\ref{th:argmax}.
\begin{lemma}
  \label{lem:meas_dual_sol}
  Let Assumption~\ref{asu:feas_dual_sol} hold true.
  Then,
  for all $N\in\mathbb{N}$ the dual solution
  \begin{align*}
  \left( \rho^\dagger,\lambda_0^\dagger,\lambda^\dagger \right)
    \ 
    \colon
   \  
    \Omega
    \ 
    \to
    \ 
  \R^N_{\ge 0}
  \times
  \R
  \times
  \R^{N}
  \end{align*}
  to
  Problem~\ref{dual} 
  is
  \begin{align*}
  \left(
    \sigma \left( D_N \right),\mathcal{B}
  \left(
  \R^N_{\ge 0}
  \times
  \R
  \times
  \R^{N}
  \right)
  \right)
  -\text{measurable}
  \,.
  \end{align*}
\end{lemma}
\begin{proof}
  Since $\Theta_N$ is deterministic (by Assumption~\ref{asu:feas_dual_sol})
  we can define the (constant) correspondence
  $\omega \mapsto \Theta_N$.
  Clearly, this is weakly-measurable, non-empty and compact.
  Next, we consider the (random) objective function of (the maximize version of) Problem~\ref{dual}, that is, $-G$ (see Definition~\ref{def:rand_obj_f}).
  By Lemme~\ref{lem:caratheo_G}, $-G$  is a Caratheodory function.
  Since $-G$ is also strictly concave, it has a unique argmax in $\Theta_N$.
  By Assumption~\ref{asu:feas_dual_sol} this is 
  $
  \left( \rho^\dagger,\lambda_0^\dagger,\lambda^\dagger \right)
  $.
  By Theorem~\ref{th:argmax} this is
  \begin{align*}
  \left(
    \sigma
    (D_N)
    ,\mathcal{B}
  \left(
  \R^N_{\ge 0}
  \times
  \R
  \times
  \R^{N}
  \right)
  \right)
  -\text{measurable}
  \,.
  \end{align*}
\end{proof}

\begin{takeaways}
  With suitable assumptions on the feasibility of Problem~\ref{dual}, we can construct measurable dual solutions.
  An important tool to obtain measurability is the argmax measurability theorem (Theorem~\ref{th:argmax}).
\end{takeaways}

\section{Basis Functions}
Going back to the functional relationship of optimal dual solution and optimal weights (see Theorem~\ref{dual_solution_th}), we see
that the basis vector of the covariates\index{$B$, vector of basis functions of the covariates}
plays an important role.
%
Now, we present our choice.
%
To the best of our knowledge, this is a novelty in the framework of balancing weights.

  Let $
\left(
\mathcal{P}_N
\right)
$
denote a sequence of countable, $\mathcal{B}$-measurable partitions 
\begin{align*}
\mathcal{P}_N= \left\{
  A_{N,1},
  A_{N,2},
  \ldots
\right\}
\subset \mathcal{B}(\R^d)
\end{align*}
of $\R^d$, that is, 
\begin{align*}
  A_{N,i}\cap A_{N,j}=\emptyset
  \qquad
  \text{if}\ i\neq j
  \qquad
  \text{and}
  \qquad 
  \bigcap_{i\in\mathbb{N}}A_{N,i}
  \ 
  =
  \ 
  \R^d
  \,.
\end{align*}
We define
$ A_N(x) $ to be the cell of $ \mathcal{P}_N $ containing $x$, that is,
\begin{align*}
  A_N
  \colon
  \R^d 
  \ 
  \twoheadrightarrow 
  \ 
  \R^d  
  \,,\qquad
  x
  \ 
  \mapsto
  \ 
  A_N(x)
  \,,
\end{align*}
where $A_N(x)$ is the only cell containing $x$. 

\begin{lemma}
  \label{lem:basis_equiv_r}
  The relation
  \begin{align*}
    x\sim y
    \qquad
    :\Leftrightarrow
    \qquad
    x\in A_N(y)
  \end{align*}
  is an equivalence relation.
\end{lemma}
\begin{proof}
  The proof is simple. We omit it.
\end{proof}
Before we define the basis vector, we assume 
uniform partition width such that
\begin{align*}
  \lambda(A_N)
  \ 
  =:
  \ 
  h_N^d
  \ 
  \to
  \ 
  0 
  \qquad
  \text{for}\ N\to\infty
  \,.
\end{align*}
Next, we define the (empirical) basis functions vector
\begin{align}
  \label{def:basis}
  B\colon
  \R^d\times \R^{d\cdot N}
  \ 
  \to
  \ 
  \R
  \,,
  \qquad
  (x,(x_1,\ldots,x_N))
  \ 
  \mapsto
  \ 
  \frac
  {
    \left[
    \mathbf{1}
    _{
      A_N(x)
    }
    (x_k)
    \right]
    _{k\in \left\{
        1,\ldots,N
    \right\}}
  }
  {
    \sum_{j=1}^N
    \mathbf{1}
    _{
      A_N(x)
    }
    (x_j)
    }
  \,,
\end{align}
where we keep to the convention $"0/0=0"$.
We shall extend $B$ to depend on the random vectors
$X,X_1,\ldots,X_N$.
The next lemma studies the measurability of the extensions.
\begin{lemma}
  \label{lem:basis_meas}
  \quad
  \begin{enumerate}[label=(\roman*)]
\item
  $B(\cdot,(X_1,\ldots,X_N))(\omega)$ is 
  $\left(
    \mathcal{B}(\R^d),\mathcal{B}(\R^N)
  \right)$-measurable
  and
  constant on each cell 
  $A_N\in\mathcal{P}_N$
  for all $\omega\in\Omega$. 
\item
  $B(X,(X_1,\ldots,X_N))$ is $\left(
    \sigma(X,D_N),\mathcal{B}(\R^N)
  \right)$-measurable. 
  \end{enumerate}
\end{lemma}
%
\begin{proof}
Consider
for $k\in \left\{
  1,\ldots,N
\right\}$
and $\omega\in\Omega$
the indicator function
\begin{align}
  \label{33342}
  \mathbf{1}
  _
  {A_N(X_k(\omega))}
  \colon \R^d\to \left\{
    0,1
  \right\}
  \,.
\end{align}
Since 
$
  {A_N(X_k(\omega))}
  \in\mathcal{B}(\R^d)
$
this is a 
  $\left(
    \mathcal{B}(\R^d),\mathcal{B}(\R)
  \right)$-measurable
  function.
  From the definition of $B$ \eqref{def:basis} it follows the first part of (i).
  Since the indicator function in \eqref{33342} is 1 if $
  x\in
  {A_N(X_k(\omega))}
  $
  and 0 else, it is also constant on each cell
  $A_N\in\mathcal{P}_N$.
  It follows (i).
  To prove (ii), note that
\begin{align*}
  \mathbf{1}
  _
  {A_N(X_k(\omega))}(X(\omega))
  \ 
  =
  \ 
  \mathbf{1}
  \bigcup_{i\in\mathbb{N}}
  \left\{
    X,X_k \in A_{N,i}
  \right\}
  (\omega)
  \qquad
  \text{for all}\ 
  \omega\in\Omega
  \,,
\end{align*}
and
$
  \bigcup_{i\in\mathbb{N}}
  \left\{
    X,X_k \in A_{N,i}
  \right\}
  \in\sigma(X,D_N)
  $.
\end{proof}
%
Now we gather some useful properties of the (empirical) basis vector.
\begin{lemma}
  \label{lem:basis_sum}
  Let $(x,x_1,\ldots,x_N)\in\R^{d(N+1)}$.
  \begin{enumerate}[label=(\roman*)]
    \item
      $
      \sum_{k=1}^{N} 
      B_k(x,x_1,\ldots,x_N)
      \in
      \left\{ 0,1 \right\}
      $. 
      In particular,
      $
        x_1,\ldots,x_N\notin A_N(x)
      $
      is equivalent to
      $
      \sum_{k=1}^{N} 
      B_k(x,x_1,\ldots,x_N)
      =0
      $
    \item
      $
      \sum_{k=1}^{N} 
      B_k(x_i,x_1,\ldots,x_N)
      \ 
      =
      \ 
      1
      \qquad
      \text{for all}\ 
      i\in \left\{ 1,\ldots,N \right\}
      $.
      \item
        $
        \norm
        {
      B(x,x_1,\ldots,x_N)
        }_2
        \ 
        \le 1
        \ 
        $
      \item
        $
        B_k(x_i,x_1,\ldots,x_N)
        \ 
        =
        \ 
        B_i(x_k,x_1,\ldots,x_N)
        \qquad
        \text{for all}\ 
        i,k\in \left\{ 1,\ldots,N \right\}
        $
  \end{enumerate}
\end{lemma}
\begin{proof}
  Let $(x,x_1,\ldots,x_N)\in\R^{d(N+1)}$.
  We prove \textit{(i)}.
  Then \textit{(ii)} is a direct consequence of \textit{(i)}.
  If 
      $
        x_1,\ldots,x_N\notin A_N(x)
      $,
  then
  \begin{align*}
    B_k(
        x,x_1,\ldots,x_N
    )
    \ 
    =
    \ 
\frac
  {
    \mathbf{1}
    _{
      A_N(x)
    }
    (x_k)
  }
  {
    \sum_{j=1}^N
    \mathbf{1}
    _{
      A_N(x)
    }
    (x_j)
    }
    \ 
    =
    \ 
    0
    \qquad
    \text{for all}\ 
    k\in \left\{ 1,\ldots,N \right\}
    \,.
  \end{align*}
  On the other hand, if the sum is 0 it holds
  \begin{align*}
    \mathbf{1}
    _{
      A_N(x)
    }
    (x_k)
    \ 
  =
  \ 
  0
    \qquad
    \text{for all}\ 
    k\in \left\{ 1,\ldots,N \right\}
    \,.
  \end{align*}
  It follows the desired equivalence.
  If 
  \begin{align*}
    \mathbf{1}
    _{
      A_N(x)
    }
    (x_k)
    \ 
  =
  \ 
  1
    \qquad
    \text{for some}\ 
    k\in \left\{ 1,\ldots,N \right\}
    \,,
  \end{align*}
  then
  $
    \sum_{j=1}^N
    \mathbf{1}
    _{
      A_N(x)
    }
    (x_j)
    \ge 1
  $
  and thus "$0/0$" doesn't occure. 
  It follows
  \begin{align*}
      \sum_{k=1}^{N} 
      B_k(x,x_1,\ldots,x_N)
      \ 
      =
      \ 
\frac
  {
      \sum_{k=1}^{N} 
    \mathbf{1}
    _{
      A_N(x)
    }
    (x_k)
  }
  {
    \sum_{j=1}^N
    \mathbf{1}
    _{
      A_N(x)
    }
    (x_j)
    }
    \ 
    =
    \ 
    1
    \,.
  \end{align*}
  To prove \textit{(iii)}, note that by \textit{(i)}
  \begin{align*}
        \norm
        {
      B(x,x_1,\ldots,x_N)
        }_2^2
        \ 
        =
        \ 
      \sum_{k=1}^{N} 
      B_k(x,x_1,\ldots,x_N)^2
        \ 
      \le
        \ 
      \sum_{k=1}^{N} 
      B_k(x,x_1,\ldots,x_N)
        \ 
      \le
        \ 
      1
      \,.
  \end{align*}
  To prove \textit{(iv)}, note that by Lemma~\ref{lem:basis_equiv_r}
  and by symmetry and transitivity of the equivalence relation
  $x\in A_N(y)$
  it holds
  \begin{align*}
      B_k(x_i,x_1,\ldots,x_N)
      &
    \ 
      =
    \ 
 \frac
  {
    \mathbf{1}
    \left\{ 
      x_k
      \in
      A_N(x_i)
    \right\}
  }
  {
    \sum_{j=1}^N
    \mathbf{1}
    \left\{ 
      x_j
      ,
      x_k
      \in
      A_N(x_i)
    \right\}
    }
    \ 
    =
    \ 
 \frac
  {
    \mathbf{1}
    \left\{ 
      x_i
      \in
      A_N(x_k)
    \right\}
  }
  {
    \sum_{j=1}^N
    \mathbf{1}
    \left\{ 
      x_j
      \in
      A_N(x_k)
    \right\}
    }
    \\
    &
    \ 
    =
    \ 
      B_i(x_k,x_1,\ldots,x_N)
      \,.
  \end{align*}
\end{proof}
%
Now we show that the basis vector plays well with uniformly continuous functions. The result seems simple, yet the consequence are great. It allows us later on to specify an oracle parameter instead of assuming its existence (see \cite[Assumption~1.6]{Wang2019}). This greatly clarifies the proofs.
\begin{lemma}
  \label{lem:basis_approx_f}
  Let $(x,x_1,\ldots,x_N)\in\R^{d(N+1)}$.
  For all uniformly continuous functions $f\colon \R^d\to \R$ it holds
 \begin{align*}
   \begin{split}
   &
   \left|
  \sum_{k=1}^{N}
    B_k(x_i,x_1,\ldots,x_N)\cdot 
    f(x_k)
    -
    f(x_i)
   \right|
   \ 
   \le
   \ 
   \omega
   \left(
    f,h_N^d
   \right)
   \qquad
   \text{for all}\ 
   i\in \left\{ 1,\ldots,N \right\}
   \,,
   \end{split}
 \end{align*}
 where $\omega(f,\cdot)$ is the uniform modulus of continuity of $f$. 
\end{lemma}
\begin{proof}
  It follows from Lemma~\ref{lem:basis_sum}\textit{(ii)}
  \begin{align*}
   \begin{split}
   &
   \left|
  \sum_{k=1}^{N}
    B_k(x_i,x_1,\ldots,x_N)\cdot 
    f(x_k)
    -
    f(x_i)
   \right|
   \\
   &
   \ 
   \le
   \ 
   \left|
  \sum_{k=1}^{N}
    B_k(x_i,x_1,\ldots,x_N)
    \left(
    f(x_k)
    -
    f(x_i)
    \right)
   \right|
   \\
   &
   \ 
   \le
   \ 
  \sum_{k=1}^{N}
    B_k(x_i,x_1,\ldots,x_N)
    \cdot
    \mathbf{1}\left\{
      x_k\in A_N(x_i)
    \right\}
    \left|
    f(x_k)
    -
    f(x_i)
    \right|
   \\
   &
   \ 
   \le
   \ 
   \omega
   \left(
    f,h_N^d
   \right)
   \,.
   \end{split}
 \end{align*}
\end{proof}
Next, we bring forward the applications of Lemma~\ref{lem:basis_approx_f} that we need.
\begin{lemma}
  \label{lem:basis_2}
  Let $(x,x_1,\ldots,x_N)\in\mathcal{X}^{N+1}$.
  It holds
  for $N\to\infty$
  \begin{enumerate}[label=(\roman*)]
      \item
      \begin{align*}
        \frac
        {1}
        {N}
        \sum_{i,k=1}^{N}
            \left|
        B_k(x_i,x_1,\ldots,x_N)
        \cdot
        \varphi^{'}
            \left(
              \frac
              {1}
              {\pi(x_k)}
            \right)
            \ 
            -
            \ 
            \varphi^{'}
            \left(
              \frac
              {1}
              {\pi(x_i)}
            \right)
            \right|
            \ 
            \to
            \ 
            0
            \,,
          \end{align*}
\item
      \begin{align*}
        \sqrt{N}
        \sup_{z\in\R}
        \max_{i\in \left\{ 1,\ldots,N \right\}}
        \sum_{k=1}^{N}
            \left|
        B_k(x_i,x_1,\ldots,x_N)
        \cdot
        F_{Y(1)}(z|x_k)
            \ 
            -
            \ 
        F_{Y(1)}(z|x_i)
            \right|
            \ 
            \to
            \ 
            0
            \,.
      \end{align*}
\end{enumerate}
\end{lemma}
\begin{proof}
  By Lemma~\ref{lem:basis_approx_f},
  the uniform continuity of $\varphi^{'}$
  it holds
      \begin{align*}
        \frac
        {1}
        {N}
        \sum_{i,k=1}^{N}
            \left|
        B_k(x,x_1,\ldots,x_N)
        \cdot
            \varphi^{'}
            \left(
              \frac
              {1}
              {\pi(x_k)}
            \right)
            \ 
            -
            \ 
            \varphi^{'}
            \left(
              \frac
              {1}
              {\pi(x_i)}
            \right)
            \right|
            \ 
            \le
            \ 
   \omega
   \left(
     \varphi^{'},h_N^d
   \right)
            \ 
            \to
            \ 
            0
          \end{align*}
          for $N\to\infty$.
          Likewise
\begin{align*}
  &
        \sqrt{N}
        \sup_{z\in\R}
        \max_{i\in \left\{ 1,\ldots,N \right\}}
        \sum_{k=1}^{N}
            \left|
        B_k(x_i,x_1,\ldots,x_N)
        \cdot
        F_{Y(1)}(z|x_k)
            \ 
            -
            \ 
        F_{Y(1)}(z|x_i)
          \right|
             \\
            &
            \ 
            \le
            \ 
            \sqrt{N}
            \sup_{z\in\R}
            \omega
            \left(
        F_{Y(1)}(z|\cdot)
        ,
        h_N^d
            \right)
            \ 
            \to
            \ 
            0
            \qquad
            \text{for}
            \ 
            N\to\infty
            \,.
\end{align*}
        \end{proof}
        \begin{remark}
We want to comment on the assumption
\begin{gather*}
  \sqrt{N}
  \sup_{z\in\R}
  \omega
  \left( 
    F_{Y(1)}(z|\cdot)
    ,h_N^d
  \right)
  \to
  0
  \qquad
  \text{for}\ 
  N\to \infty
  \,,
\end{gather*}
I decided to keep this more general (and abstract) assumption, althogh
there are many (more concrete, yet stronger) assumptions on the regularity of
$
    F_{Y(1)}(z|\cdot)
$
and the convergence speed of $h_N$.
If for example 
$
    F_{Y(1)}(z|\cdot)
$
is $\alpha$-Hölder continuous with $\alpha\in(0,1]$ for all $z\in\R$, it suffices $\sqrt{N}h_N^{\alpha\cdot d}\to0$.

        \end{remark}
\begin{takeaways}
  Basis functions of non-parametric
  partitioning estimates are new to the framework of balancing weights.
  They play well with uniformly continuous functions and promise to simplify the analysis. 
  This choice of basis functions waits to be tested in practice.
\end{takeaways}

\section{Weights Process}
  Based on Theorem~\ref{dual_solution_th}
we want to use the dual 
solution 
$
\left( \rho^\dagger,\lambda_0^\dagger,\lambda^\dagger \right)
$
to construct weights.
To this end, we define the (empirical) weights function
\begin{align*}
 w\ \colon\
 &
 \left( 
  \R^d\times \R^{d\cdot N}
 \right)
  \times
  \left( 
\R^N_{\ge 0}\times \R\times \R^N
  \right)
  \to
  \R^N
  \\
 &
  \left( 
  (x,x_1,\ldots,x_N),(\rho,\lambda_0,\lambda)
  \right)
  \ 
  \mapsto
  \ 
  \left[ 
  (\varphi^{'})^{-1}
  \left( 
    \rho_i
    +
    \lambda_0
    +
    \inner
    {B(x,x_1,\ldots,x_N)}
    {\lambda}
  \right)
\right]_{i\in \left\{ 1,\ldots,N \right\}}
\,.
\end{align*}
\begin{definition}
  Let 
  $
\left( \rho^\dagger,\lambda_0^\dagger,\lambda^\dagger \right)
  $
  be the dual solution of Lemma~\ref{lem:meas_dual_sol}.
  We define the weights process 
  $\left\{ w^\dagger(x) | x\in\R^d\right\}$
  by
  \begin{align*}
    w^\dagger(x) 
    \ 
    :=
    \ 
    w
    \left( 
    \left( 
    x,X_1,\ldots,X_N,
    \right)
    ,
\left( \rho^\dagger,\lambda_0^\dagger,\lambda^\dagger \right)
    \right)
    \qquad
    \text{for all}\ 
    x\in\R^d
    \,.
  \end{align*}
\end{definition}
\begin{lemma}
  \label{lem:weights:meas}
  \quad
  \begin{enumerate}[label=(\roman*)]
\item
  $w^\dagger(\cdot)(\omega)$ is 
  $\left(
    \mathcal{B}(\R^d),\mathcal{B}(\R^N)
  \right)$-measurable
  and
  constant on each cell 
  $A_N\in\mathcal{P}_N$
  for all $\omega\in\Omega$. 
\item
  $w^\dagger(X)$ is $\left(
    \sigma(X,D_N),\mathcal{B}(\R^N)
  \right)$-measurable. 
  \end{enumerate}
\end{lemma}
\begin{proof}
  This is a direct consequence of Lemme~\ref{lem:basis_meas}, Lemma~\ref{lem:meas_dual_sol}
  and 
  the (assumed) continuity of $(\varphi^{'})^{-1}$.
\end{proof}


\begin{lemma}
  \label{weights_l_inf}
  It holds
  $w_i^\dagger(X)\in L^\infty(\P)$
  for all $i\in \left\{ 1,\ldots,N \right\}$.
\end{lemma}
\begin{proof}
  By Lemma~\ref{lem:basis_sum}.\textit{(iii)} it holds
  \begin{align*}
  \left| 
    \rho_i^\dagger
    +
    \lambda_0^\dagger
    +
    \inner
    {B(x,x_1,\ldots,x_N)}
    {\lambda^\dagger}
  \right|
  \ 
  \lesssim
  \ 
  \norm{
\left( \rho^\dagger,\lambda_0^\dagger,\lambda^\dagger \right)
  }_2
  \qquad
  \text{for all}\ 
  i \in \left\{ 1,\ldots,N \right\}
  \,,
  \end{align*}
  where $\lesssim$ denotes the lesser-or-equal-up-to-a-uniform-constant order, that is, we choose $C>1$ independent of $N$ large enough, such that $a\lesssim b$ means $a\le C\cdot b$.
  \index{$\lesssim$, 
lesser-or-equal-up-to-a-uniform-constant order
  }
  Since
  $
\left( \rho^\dagger,\lambda_0^\dagger,\lambda^\dagger \right)
  $ is contained in the deterministic and compact parameter space $\Theta_N$,
  it holds
  \begin{align*}
  \norm{
\left( \rho^\dagger,\lambda_0^\dagger,\lambda^\dagger \right)
  }_2
  \in 
  L^{\infty}(\P)
  \,.
  \end{align*}
  By the (assumed) uniform continuity of 
  $
  (\varphi^{'})^{-1}
  $
  on $\R$, it follows 
  $w_i^\dagger(X)\in L^\infty(\P)$
  for all $i\in \left\{ 1,\ldots,N \right\}$.
\end{proof}
Next, we want to simplify the weights process in the spirit of Lemma~\ref{lem:simple_weights}.
In other words, we want to become independent of the index $i$ in $w_i^\dagger$. This will be helpful in the subsequent analysis.
To this end, we define the (empirical) simplified weights function
\begin{align*}
 w_0\ \colon\
 &
 \left( 
  \R^d\times \R^{d\cdot N}
 \right)
  \times
  \left( 
    \R\times \R^N
  \right)
  \to
  [0,\infty)
  \\
 &
  \left( 
  (x,x_1,\ldots,x_N),(\lambda_0,\lambda)
  \right)
  \ 
  \mapsto
  \ 
  \left[ 
  (\varphi^{'})^{-1}
  \left( 
    \lambda_0
    +
    \inner
    {B(x,x_1,\ldots,x_N)}
    {\lambda}
  \right)
\right]^+
\,.
\end{align*}
\begin{definition}
  Let 
  $
\left( \rho^\dagger,\lambda_0^\dagger,\lambda^\dagger \right)
  $
  be the dual solution of Lemma~\ref{lem:meas_dual_sol}.
  We define the simplified weights process 
  $\left\{ w_0^\dagger(x) \,|\, x\in\R^d\right\}$
  by
  \begin{align*}
    w_0^\dagger(x) 
    \ 
    :=
    \ 
    w_0
    \left( 
    \left( 
    x,X_1,\ldots,X_N,
    \right)
    ,
\left( \lambda_0^\dagger,\lambda^\dagger \right)
    \right)
    \qquad
    \text{for all}\ 
    x\in\R^d
    \,.
  \end{align*}
\end{definition}
\begin{lemma}
  \label{lem:weights:meas:0}
  \quad
  \begin{enumerate}[label=(\roman*)]
\item
  $w_0^\dagger(\cdot)(\omega)$ is 
  $\left(
    \mathcal{B}(\R^d),\mathcal{B}(\R^N)
  \right)$-measurable
  and
  constant on each cell 
  $A_N\in\mathcal{P}_N$
  for all $\omega\in\Omega$. 
\item
  $w_0^\dagger(X)$ is $\left(
    \sigma(X,D_N),\mathcal{B}(\R^N)
  \right)$-measurable. 
  \end{enumerate}
\end{lemma}
\begin{proof}
The proof is as that of Lemma~\ref{lem:weights:meas}.
\end{proof}


\begin{lemma}
  \label{weights_0_l_inf}
  It holds $w_0^\dagger(X)\in L^\infty(\P)$.
\end{lemma}
\begin{proof}
  By Lemma~\ref{weights_l_inf},
  the monotonicity of 
  $
  (\varphi^{'})^{-1}
  $
  and $\rho_i\ge 0$ for $i\le n$,
  it holds
  \begin{align*}
    w_0^\dagger(X) 
    &
    \ 
    \le
    \ 
  \left[ 
  (\varphi^{'})^{-1}
  \left( 
    \lambda_0^\dagger
    +
    \inner
    {B(X)}
    {\lambda^\dagger}
  \right)
\right]^+
\\
&
\ 
\le
\ 
  \left[ 
  (\varphi^{'})^{-1}
  \left( 
    \rho_i^\dagger
    +
    \lambda_0^\dagger
    +
    \inner
    {B(X)}
    {\lambda^\dagger}
  \right)
\right]^+
\ 
\le
\ 
\left| 
    w_i^\dagger(X) 
\right|
\ 
\in
\ 
L^\infty(\P)
  \end{align*}
\end{proof}
\begin{lemma}
  \label{w.Z=0}
 Let 
 $Z\in L^1(\P)$
  be a random variable that is independent of $D_N=(T_i,X_i)_{i\in \left\{
    1,\ldots,N
  \right\}}$ 
  with
  $
\E
\left[
  Z
  \,
  |
  \, 
  X
\right]
= 0
  $
  almost surely.
  It holds
  \begin{gather*}
  \E
  \left[
    w_0^\dagger(X)
  \cdot Z
  \right]
  \ 
  =
  \ 
  0
  \,.
  \end{gather*}
\end{lemma}
\begin{proof}
  By Lemma~\ref{weights_0_l_inf} it holds
  \begin{gather}
    \label{9876}
    \norm{
  w_0^\dagger(X)\cdot Z
    }_{L^1(\P)}
    \ 
  \le
    \ 
  \norm{w_0^\dagger(X)}_{L^\infty(\P)}
  \norm{Z}_{L^1(\P)}
  \ 
  <
  \ 
  \infty
  \,.
  \end{gather}
  By 
  \eqref{9876},
  $Z\perp D_N$
  and
  $
\E
\left[
  Z
  \,
  |
  \, 
  X
\right]
= 0
  $
  almost surely
  it holds 
  \begin{align*}
    \E
  \left[
  w_0^\dagger(X)
  \cdot
  Z
  \,
  |
  \,
  D_N,X
  \right]
  &
  \ 
  =
  \ 
  w_0^\dagger(X)
  \cdot
  \E
  \left[
  Z
  \,
  |
  \,
  D_N,X
  \right]
  \\
  &
  \ 
  =
  \ 
  w_0^\dagger(X)
  \cdot
  \E
  \left[
  Z
  \,
  |
  \,
  X
  \right]
  \
  =
  \ 
  0
  \qquad
  \text{almost surely.}
  \end{align*}
  Note, that $w_0^\dagger(X)$ is 
  $
  \left(
  \sigma(D_N,X),\mathcal{B}(\R)
  \right)
  $-measurable.  
  Thus
  \begin{gather*}
    \E
    \left[
  w_0^\dagger(X)
  \cdot
  Z
  \,
    \right]
    \ 
    =
    \ 
    \E
    \left[
 \E
  \left[
  w_0^\dagger(X)
  \cdot
  Z
  \,
  |
  \,
  D_N,X
  \right]
    \right]
    \ 
    =
    \ 
    0
    \,.
     \end{gather*}
\end{proof}

\begin{theorem}
  \label{th:weights_constr}
  The simplified weights process satisfies the constraints
  of Problem~\ref{bw:1:primal}, that is,
  \begin{enumerate}[label=(\roman*)]
    \item
      $
      T_i\cdot w_0^\dagger(X_i)
      \ 
      \ge
      \ 
      0
      \qquad
      \text{for all}\ 
      i\in  \left\{ 1,\ldots,N \right\}
      $
    \item
      $
      \frac{1}{N}
      \sum_{i=1}^{N} 
      T_i\cdot w_0^\dagger(X_i)
      \ 
      =
      \ 
      1
      $
    \item
      For all $k\in \left\{ 1,\ldots,N \right\}$
      it holds
      \begin{align*}
      \left| 
      \frac{1}{N}
      \left( 
        \sum_{i=1}^{N} 
      T_i\cdot w_0^\dagger(X_i)
      \cdot
        B_k(X_i,X_1,\ldots,X_N)
        \
        -
        \
        \sum_{i=1}^{N} 
        B_k(X_i,X_1,\ldots,X_N)
      \right)
      \right|
      \ 
      \le
      \ 
      \delta_k
      \end{align*}
  \end{enumerate}
\end{theorem}
\begin{proof}
  This follows from Theorem~\ref{dual_solution_th},
  Lemma~\ref{lem:simple_weights}
  and the construction of the simplified weights process.
\end{proof}
To avoid notational overload, from now on we write
\begin{align*}
  B(x)
  \ 
  :=
  \ 
  B(x,X_1,\ldots,X_N)
  \qquad
  \text{for all}\ 
  x\in\R^d
  \,.
\end{align*}

