Let $
\left(
\mathcal{P}_N
\right)
$
denote a sequence of countable, $\mathcal{B}$-measurable partitions 
\index{$\mathcal{P}_N$, partition of $\R^d$}
\begin{align*}
\mathcal{P}_N= \left\{
  A_{N,1},
  A_{N,2},
  \ldots
\right\}
\subset \mathcal{B}(\R^d)
\end{align*}
of $\R^d$, that is, 
\begin{align*}
  A_{N,i}\cap A_{N,j}=\emptyset
  \qquad
  \text{if}\ i\neq j
  \qquad
  \text{and}
  \qquad 
  \bigcap_{i\in\mathbb{N}}A_{N,i}
  \ 
  =
  \ 
  \R^d
  \,.
\end{align*}
\index{$A_N$, cell of the partition $\mathcal{P}_N$}
We define
$ A_N(x) $ to be the cell of $ \mathcal{P}_N $ containing $x$, that is,
\begin{align*}
  A_N
  \colon
  \R^d 
  \ 
  \twoheadrightarrow 
  \ 
  \R^d  
  \,,\qquad
  x
  \ 
  \mapsto
  \ 
  A_N(x)
  \,,
\end{align*}
where $A_N(x)$ is the only cell containing $x$. 

\begin{lemma}
  \label{lem:basis_equiv_r}
  The relation
  \begin{align*}
    x\sim y
    \qquad
    :\Leftrightarrow
    \qquad
    x\in A_N(y)
  \end{align*}
  is an equivalence relation.
\end{lemma}
\begin{proof}
  The proof is simple. We omit it.
\end{proof}
Before we define the basis vector, we assume 
uniform partition width such that
\begin{align*}
  \lambda(A_N)
  \ 
  =:
  \ 
  h_N^d
  \ 
  \to
  \ 
  0 
  \qquad
  \text{for}\ N\to\infty
  \,.
\end{align*}
Next, we define the (empirical) basis functions vector
\begin{align}
  \label{def:basis}
  B\colon
  \R^d\times \R^{d\cdot N}
  \ 
  \to
  \ 
  \R
  \,,
  \qquad
  (x,(x_1,\ldots,x_N))
  \ 
  \mapsto
  \ 
  \frac
  {
    \left[
    \mathbf{1}
    _{
      A_N(x)
    }
    (x_k)
    \right]
    _{k\in \left\{
        1,\ldots,N
    \right\}}
  }
  {
    \sum_{j=1}^N
    \mathbf{1}
    _{
      A_N(x)
    }
    (x_j)
    }
  \,,
\end{align}
where we keep to the convention $"0/0=0"$.
We shall extend $B$ to depend on the random vectors
$X,X_1,\ldots,X_N$.
The next lemma studies the measurability of the extensions.
\begin{lemma}
  \label{lem:basis_meas}
  \begin{enumerate}[label=(\roman*)]
\item
  $B(\cdot,(X_1,\ldots,X_N))(\omega)$ is 
  $\left(
    \mathcal{B}(\R^d),\mathcal{B}(\R^N)
  \right)$-measurable
  and
  constant on each cell 
  $A_N\in\mathcal{P}_N$
  for all $\omega\in\Omega$. 
\item
  $B(X,(X_1,\ldots,X_N))$ is $\left(
    \sigma(X,D_N),\mathcal{B}(\R^N)
  \right)$-measurable. 
  \end{enumerate}
\end{lemma}
%
\begin{proof}
Consider,
for $k\in \left\{
  1,\ldots,N
\right\}$
and $\omega\in\Omega$,
the indicator function
\index{$\mathbf{1_A}$, indicator function of the set $A$}
\begin{align}
  \label{33342}
  \mathbf{1}
  _
  {A_N(X_k(\omega))}
  \ 
  \colon
  \ 
  \R^d
  \ 
  \to
  \ 
  \left\{
    0,1
  \right\}
  \,.
\end{align}
Since 
$
  {A_N(X_k(\omega))}
  \in\mathcal{B}(\R^d)
$,
this is a 
  $\left(
    \mathcal{B}(\R^d),\mathcal{B}(\R)
  \right)$-measurable
  function.
  From the definition of $B$ \eqref{def:basis} it follows the first part of \textit{(i)}.
  Since the indicator function in \eqref{33342} is 1 if $
  x\in
  {A_N(X_k(\omega))}
  $
  and 0 else, it is also constant on each cell
  $A_N\in\mathcal{P}_N$.
  It follows \textit{(i)}.
  To prove \textit{(ii)}, note that
\begin{align*}
  \mathbf{1}
  _
  {A_N(X_k(\omega))}(X(\omega))
  \ 
  =
  \ 
  \mathbf{1}
  \bigcup_{i\in\mathbb{N}}
  \left\{
    X,X_k \in A_{N,i}
  \right\}
  (\omega)
  \qquad
  \text{for all}\ 
  \omega\in\Omega
  \,,
\end{align*}
and
$
  \bigcup_{i\in\mathbb{N}}
  \left\{
    X,X_k \in A_{N,i}
  \right\}
  \in\sigma(X,D_N)
  $.
\end{proof}
%
Now we gather some useful properties of the (empirical) basis vector.
\begin{lemma}
  \label{lem:basis_sum}
  Let $(x,x_1,\ldots,x_N)\in\R^{d(N+1)}$.
  \begin{enumerate}[label=(\roman*)]
    \item
      $
      \sum_{k=1}^{N} 
      B_k(x,x_1,\ldots,x_N)
      \in
      \left\{ 0,1 \right\}
      $. 
      In particular,
      $
        x_1,\ldots,x_N\notin A_N(x)
      $
      is equivalent to
      $
      \sum_{k=1}^{N} 
      B_k(x,x_1,\ldots,x_N)
      =0
      $
    \item
      $
      \sum_{k=1}^{N} 
      B_k(x_i,x_1,\ldots,x_N)
      \ 
      =
      \ 
      1
      \qquad
      \text{for all}\ 
      i\in \left\{ 1,\ldots,N \right\}
      $.
      \item
        $
        \norm
        {
      B(x,x_1,\ldots,x_N)
        }_2
        \ 
        \le 1
        \ 
        $
      \item
        $
        B_k(x_i,x_1,\ldots,x_N)
        \ 
        =
        \ 
        B_i(x_k,x_1,\ldots,x_N)
        \qquad
        \text{for all}\ 
        i,k\in \left\{ 1,\ldots,N \right\}
        $
  \end{enumerate}
\end{lemma}
\begin{proof}
  Let $(x,x_1,\ldots,x_N)\in\R^{d(N+1)}$.
  We prove \textit{(i)}.
  Then \textit{(ii)} is a direct consequence of \textit{(i)}.
  If 
      $
        x_1,\ldots,x_N\notin A_N(x)
      $,
  then
  \begin{align*}
    B_k(
        x,x_1,\ldots,x_N
    )
    \ 
    =
    \ 
\frac
  {
    \mathbf{1}
    _{
      A_N(x)
    }
    (x_k)
  }
  {
    \sum_{j=1}^N
    \mathbf{1}
    _{
      A_N(x)
    }
    (x_j)
    }
    \ 
    =
    \ 
    0
    \qquad
    \text{for all}\ 
    k\in \left\{ 1,\ldots,N \right\}
    \,.
  \end{align*}
  On the other hand, if the sum is 0 it holds
  \begin{align*}
    \mathbf{1}
    _{
      A_N(x)
    }
    (x_k)
    \ 
  =
  \ 
  0
    \qquad
    \text{for all}\ 
    k\in \left\{ 1,\ldots,N \right\}
    \,.
  \end{align*}
  It follows the desired equivalence.
  If 
  \begin{align*}
    \mathbf{1}
    _{
      A_N(x)
    }
    (x_k)
    \ 
  =
  \ 
  1
    \qquad
    \text{for some}\ 
    k\in \left\{ 1,\ldots,N \right\}
    \,,
  \end{align*}
  then
  $
    \sum_{j=1}^N
    \mathbf{1}
    _{
      A_N(x)
    }
    (x_j)
    \ge 1
  $
  and thus "$0/0$" doesn't occure. 
  It follows
  \begin{align*}
      \sum_{k=1}^{N} 
      B_k(x,x_1,\ldots,x_N)
      \ 
      =
      \ 
\frac
  {
      \sum_{k=1}^{N} 
    \mathbf{1}
    _{
      A_N(x)
    }
    (x_k)
  }
  {
    \sum_{j=1}^N
    \mathbf{1}
    _{
      A_N(x)
    }
    (x_j)
    }
    \ 
    =
    \ 
    1
    \,.
  \end{align*}
  To prove \textit{(iii)}, note that by \textit{(i)}
  \begin{align*}
        \norm
        {
      B(x,x_1,\ldots,x_N)
        }_2^2
        \ 
        =
        \ 
      \sum_{k=1}^{N} 
      B_k(x,x_1,\ldots,x_N)^2
        \ 
      \le
        \ 
      \sum_{k=1}^{N} 
      B_k(x,x_1,\ldots,x_N)
        \ 
      \le
        \ 
      1
      \,.
  \end{align*}
  To prove \textit{(iv)}, note that by Lemma~\ref{lem:basis_equiv_r}
  and by symmetry and transitivity of the equivalence relation
  $x\in A_N(y)$
  it holds
  \begin{align*}
      B_k(x_i,x_1,\ldots,x_N)
      &
    \ 
      =
    \ 
 \frac
  {
    \mathbf{1}
    \left\{ 
      x_k
      \in
      A_N(x_i)
    \right\}
  }
  {
    \sum_{j=1}^N
    \mathbf{1}
    \left\{ 
      x_j
      ,
      x_k
      \in
      A_N(x_i)
    \right\}
    }
    \ 
    =
    \ 
 \frac
  {
    \mathbf{1}
    \left\{ 
      x_i
      \in
      A_N(x_k)
    \right\}
  }
  {
    \sum_{j=1}^N
    \mathbf{1}
    \left\{ 
      x_j
      \in
      A_N(x_k)
    \right\}
    }
    \\
    &
    \ 
    =
    \ 
      B_i(x_k,x_1,\ldots,x_N)
      \,.
  \end{align*}
\end{proof}
%
Now we show that the basis vector plays well with uniformly continuous functions. The result seems simple, yet the consequence are great. It allows us later on to specify an oracle parameter instead of assuming its existence (see \cite[Assumption~1.6]{Wang2019}). This greatly clarifies the proofs.
\begin{lemma}
  \label{lem:basis_approx_f}
  Let $(x,x_1,\ldots,x_N)\in\R^{d(N+1)}$.
  For all uniformly continuous functions $f\colon \R^d\to \R$ it holds
 \begin{align*}
   \begin{split}
   &
   \left|
  \sum_{k=1}^{N}
    B_k(x_i,x_1,\ldots,x_N)\cdot 
    f(x_k)
    -
    f(x_i)
   \right|
   \ 
   \le
   \ 
   \omega
   \left(
    f,h_N^d
   \right)
   \qquad
   \text{for all}\ 
   i\in \left\{ 1,\ldots,N \right\}
   \,,
   \end{split}
 \end{align*}
 where $\omega(f,\cdot)$ is the uniform modulus of continuity of $f$. 
\end{lemma}
\begin{proof}
  It follows from Lemma~\ref{lem:basis_sum}.\textit{(ii)}
  \begin{align*}
   \begin{split}
   &
   \left|
  \sum_{k=1}^{N}
    B_k(x_i,x_1,\ldots,x_N)\cdot 
    f(x_k)
    -
    f(x_i)
   \right|
   \\
   &
   \ 
   \le
   \ 
   \left|
  \sum_{k=1}^{N}
    B_k(x_i,x_1,\ldots,x_N)
    \left(
    f(x_k)
    -
    f(x_i)
    \right)
   \right|
   \\
   &
   \ 
   \le
   \ 
  \sum_{k=1}^{N}
    B_k(x_i,x_1,\ldots,x_N)
    \cdot
    \mathbf{1}\left\{
      x_k\in A_N(x_i)
    \right\}
    \left|
    f(x_k)
    -
    f(x_i)
    \right|
   \\
   &
   \ 
   \le
   \ 
   \omega
   \left(
    f,h_N^d
   \right)
   \,.
   \end{split}
 \end{align*}
\end{proof}
Next, we apply  Lemma~\ref{lem:basis_approx_f}.
On a high-level, the next lemma says that the basis functions estimate both treatment \textit{(i)} and outcome model \textit{(ii)} well.
This feature is connected to double robustness, discussed in \cite{Zhao2017a}.

In the following,
let $F_{Y(1)}(\cdot|x)$ denote the distribution function of $Y(1)$ conditional on $X=x\in\mathcal{X}$ (see \eqref{3228}).
\newpage
\begin{lemma}
  \label{lem:basis_2}
  Let $(x,x_1,\ldots,x_N)\in\mathcal{X}^{N+1}$.
  It holds
  for $N\to\infty$
  \begin{enumerate}[label=(\roman*)]
      \item
        If Assumption~\ref{asu:treatment_asign_str_ing} and Assumption~\ref{asu:x_finite} hold true
      \begin{align*}
        \frac
        {1}
        {N}
        \sum_{i,k=1}^{N}
            \left|
        B_k(x_i,x_1,\ldots,x_N)
        \cdot
        \varphi^{'}
            \left(
              \frac
              {1}
              {\pi(x_k)}
            \right)
            \ 
            -
            \ 
            \varphi^{'}
            \left(
              \frac
              {1}
              {\pi(x_i)}
            \right)
            \right|
            \ 
            \to
            \ 
            0
            \,,
          \end{align*}
\item
  If 
  $
  \sqrt{N}
  \sup_{z\in\R}
  \omega
  \left( 
    F_{Y(1)}(z|\cdot)
    ,h_N^d
  \right)
  \to
  0
  $, then
      \begin{align*}
        \sqrt{N}
        \sup_{z\in\R}
        \max_{i\in \left\{ 1,\ldots,N \right\}}
        \sum_{k=1}^{N}
            \left|
        B_k(x_i,x_1,\ldots,x_N)
        \cdot
        F_{Y(1)}(z|x_k)
            \ 
            -
            \ 
        F_{Y(1)}(z|x_i)
            \right|
            \ 
            \to
            \ 
            0
            \,.
      \end{align*}
\end{enumerate}
\end{lemma}
\begin{proof}
  By Lemma~\ref{lem:basis_approx_f} (good approximation of uniformly continuous functions) and
  Lemma~\ref{lem:ips_unif_cont} 
  (uniform continuity of $\varphi^{'}\circ (x\mapsto 1/x)\circ \pi$),
  it holds
      \begin{align*}
        \frac
        {1}
        {N}
        \sum_{i,k=1}^{N}
            \left|
        B_k(x,x_1,\ldots,x_N)
        \cdot
            \varphi^{'}
            \left(
              \frac
              {1}
              {\pi(x_k)}
            \right)
            \ 
            -
            \ 
            \varphi^{'}
            \left(
              \frac
              {1}
              {\pi(x_i)}
            \right)
            \right|
            \ 
            \le
            \ 
   \omega
   \left(
     \varphi^{'},h_N^d
   \right)
            \ 
            \to
            \ 
            0
          \end{align*}
          for $N\to\infty$.
          Likewise
\begin{align*}
  &
        \sqrt{N}
        \sup_{z\in\R}
        \max_{i\in \left\{ 1,\ldots,N \right\}}
        \sum_{k=1}^{N}
            \left|
        B_k(x_i,x_1,\ldots,x_N)
        \cdot
        F_{Y(1)}(z|x_k)
            \ 
            -
            \ 
        F_{Y(1)}(z|x_i)
          \right|
             \\
            &
            \ 
            \le
            \ 
            \sqrt{N}
            \sup_{z\in\R}
            \omega
            \left(
        F_{Y(1)}(z|\cdot)
        ,
        h_N^d
            \right)
            \ 
            \to
            \ 
            0
            \qquad
            \text{for}
            \ 
            N\to\infty
            \,.
\end{align*}
        \end{proof}
        \begin{remark}
I want to comment on the assumption
\begin{gather*}
  \sqrt{N}
  \sup_{z\in\R}
  \omega
  \left( 
    F_{Y(1)}(z|\cdot)
    ,h_N^d
  \right)
  \to
  0
  \qquad
  \text{for}\ 
  N\to \infty
  \,.
\end{gather*}
I decided to keep this more general (and abstract) assumption, although
there are many (more concrete, yet stronger) sufficient assumptions on the regularity of
$
    F_{Y(1)}(z|\cdot)
$
and the convergence speed of $h_N$.
If for example 
$
    F_{Y(1)}(z|\cdot)
$
is $\alpha$-Hölder continuous with $\alpha\in(0,1]$ for all $z\in\R$, it suffices $\sqrt{N}h_N^{\alpha\cdot d}\to0$.

        \end{remark}
\begin{takeaways}
  Basis functions of non-parametric
  partitioning estimates are new to the framework of balancing weights.
  They play well with uniformly continuous functions and promise to simplify the analysis. 
  This choice of basis functions waits to be tested in practice.
\end{takeaways}
