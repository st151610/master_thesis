Next, we formulate the feasibility assumption. The assumption is (asymptotically) justified by Theorem~\ref{th:cons_dual}.
Note that we assume compactness to be able to apply Theorem~\ref{th:argmax}.
\begin{assumption}
  \label{asu:feas_dual_sol}
  For all $N\in\mathbb{N}$ there exists a non-empty, compact, and deterministic 
  parameter space 
  $
  \Theta_N
  \subset
  \R^{N}_{\ge 0}
  \times
  \R
  \times
  \R^N
  $
  such that the optimal solution 
  $
  \left( \rho^\dagger,\lambda_0^\dagger,\lambda^\dagger \right)
  $
  of Problem~\ref{dual}
  are contained in $\Theta_N$.
\end{assumption}
Based on this assumption it is easy to derive measurability for the dual solutions 
  $
  \left( \rho^\dagger,\lambda_0^\dagger,\lambda^\dagger \right)
  $.
  To this end, we take a closer look at the objective function.
  \begin{definition}
    \label{def:rand_obj_f}
We define the (random) objective function of
Problem~\ref{dual} by
  \begin{align*}
    &
  G
  \ 
  \colon
  \ 
  \left(
  \Omega,
\sigma
(D_N)
  \right)
  \times
  \left(
  \R^N_{\ge 0}
  \times
  \R
  \times
  \R^{N}
  \right)
  \ 
  \to
  \ 
  \overline{\R}
  \intertext{with}
    &
  G(\omega,(\rho,\lambda_0,\lambda))
  \ 
  =
  \ 
  \infty
  \qquad 
  \text{if}\quad 
  \rho_i
  \neq
  \left[ 
  \varphi^{-1}
  (0)
  -
  \left( 
  \lambda_0 + \inner{B(X_i)}{\lambda}
  \right)
  \right]^+
  \
  \text{for some}\ i>n
  \,,
  \\
  \intertext{and else}\qquad
  &
  G(\omega,(\rho,\lambda_0,\lambda))
  \\
  &
  \ 
  =
  \ 
  \frac{1}{N}
\sum_{i=1} 
  ^N
  \Big[
  T_i(\omega)
  \cdot
  \varphi^*
  \!
  \left( 
    \rho_i
    +
\lambda_0
+
\inner
{B(X_i)(\omega)}
{
\lambda
}
  \right)
  \ 
  -
  \ 
\lambda_0
-
\inner
{B(X_i)(\omega)}
{
\lambda
}
\Big]
\\
&
  \qquad 
+
\ 
\inner
{\delta(\omega)}
{
  |\lambda|
}
\,.
  \end{align*}
  \end{definition}
  \begin{lemma}
    \label{lem:caratheo_G}
    The function $G$ of Definition~\ref{def:rand_obj_f}
    is Caratheodory.
  \end{lemma}
  \begin{proof}
    This follows from Lemma~\ref{1165}
    (continuity of $\varphi^*$) and the measurability 
  of all random variables included.
  \end{proof}
  In the proof of the next lemma we gather the arguments and apply Theorem~\ref{th:argmax}.
\begin{lemma}
  \label{lem:meas_dual_sol}
  Let Assumption~\ref{asu:feas_dual_sol} hold true.
  Then,
  for all $N\in\mathbb{N}$ the dual solution
  \begin{align*}
  \left( \rho^\dagger,\lambda_0^\dagger,\lambda^\dagger \right)
    \ 
    \colon
   \  
    \Omega
    \ 
    \to
    \ 
  \R^N_{\ge 0}
  \times
  \R
  \times
  \R^{N}
  \end{align*}
  to
  Problem~\ref{dual} 
  is
  \begin{align*}
  \left(
    \sigma \left( D_N \right),\mathcal{B}
  \left(
  \R^N_{\ge 0}
  \times
  \R
  \times
  \R^{N}
  \right)
  \right)
  -\text{measurable}
  \,.
  \end{align*}
\end{lemma}
\begin{proof}
  Since $\Theta_N$ is deterministic (by Assumption~\ref{asu:feas_dual_sol})
  we can define the (constant) correspondence
  $\omega \mapsto \Theta_N$.
  Clearly, this is weakly-measurable, non-empty and compact.
  Next, we consider the (random) objective function of (the maximize version of) Problem~\ref{dual}, that is, $-G$ (see Definition~\ref{def:rand_obj_f}).
  By Lemme~\ref{lem:caratheo_G}, $-G$  is a Caratheodory function.
  Since $-G$ is also strictly concave, it has a unique argmax in $\Theta_N$.
  By Assumption~\ref{asu:feas_dual_sol} this is 
  $
  \left( \rho^\dagger,\lambda_0^\dagger,\lambda^\dagger \right)
  $.
  By Theorem~\ref{th:argmax} this is
  \begin{align*}
  \left(
    \sigma
    (D_N)
    ,\mathcal{B}
  \left(
  \R^N_{\ge 0}
  \times
  \R
  \times
  \R^{N}
  \right)
  \right)
  -\text{measurable}
  \,.
  \end{align*}
\end{proof}

\begin{takeaways}
  With suitable assumptions on the feasibility of Problem~\ref{dual}, we can construct measurable dual solutions.
  An important tool to obtain measurability is the argmax measurability theorem (Theorem~\ref{th:argmax}).
\end{takeaways}
