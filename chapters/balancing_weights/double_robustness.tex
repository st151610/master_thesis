%read the paper \cite{Zhao2017a} for discussion of double robustness for balancing weights and \cite{Hahn1998} for semiparametric efficiency bounds.


\subsection{Learning Rates of the weighted mean}

We begin the discussion with an
uncomplicated question:
What is the speed of convergence in the weak law of large numbers?
The next statement gives a clear-cut answer:
The arithmetic mean of independent, identically distributed, square-integrable random variables 
learns with rate $n^{-1/2}$.
Furthermore, the statement is easy to prove using
Bienaymé's formula and Chebyshev's inequality (cf. \cite[Theorem~5.14]{Klenke2020}).


\begin{theorem*}
  Let 
  $
    X_1,X_2,\ldots
  $
  be i.i.d, square-integrable random variables with 
  $
    V:=
    \mathbf{Var}[X_1]
    <\infty
  $.
  Then, for any $\tau \in (0,1]$ and all $n\in\mathbb{N}$, we have
  \begin{gather}
   \P
   \left[
     \left| 
   \frac{1}{n}
   \sum_{i=1}^{n}
   (X_i - \E[X_i])
     \right|
     \le
     \sqrt{V}
     \frac{1}{\sqrt{\tau}}
     \frac{1}{\sqrt{n}}
   \right]
   \ge
   1-\tau\,.
  \end{gather}
\end{theorem*}

Deriving learning rates in this way, we make an implicit assumption.
We assume that
observed outcomes of the treated follow the same distribution as
marginal potential outcomes under treatment.
In other words, we require
\begin{gather}
  Y(1)\,|\,T=1
  \ 
  \sim 
  \ 
  \P_
  {
  Y(1)
  }
  \,.
\end{gather}
Virtually every scenario of interest violates this assumption.
Indeed, any external influence on both $T$ and $Y(1)$ can ruin the 
above assessment simply by imposing 
\begin{gather}
  \E[Y(T)]\ \neq\  \E[Y(1)]
  \,.
\end{gather}
Statisticians have wrestled with this issue for nearly a century.

In experimental studies we usually specify treatment assignment as opposed to merely observing a unit receiving treatment. 

The next statement makes use of the propensity score.

\begin{theorem*}
  Consider the weighted mean estimator with weights
  \begin{gather}
    w_i
    \ 
    =
    \ 
    \frac{1}{n}
    \frac{T_i}{\pi(X_i)}
    \,.
  \end{gather}
  Denote
  $
    V:=
    \E[
    (
      Y(1)
    )^2
    \,/\,
    \pi(X)
    ]
    -
    \E[Y(1)]^2
  $.  
  Assume that \textit{weak unconfoundedness} holds.
  Then, for any $\tau \in (0,1]$ and all $n\in\mathbb{N}$, we have
  \begin{gather}
   \P
   \left[
     \left| 
   \sum_{i=1}^{n}
   w_iY_i - \E[Y(1)]
     \right|
     \le
     \sqrt{V}
     \frac{1}{\sqrt{\tau}}
     \frac{1}{\sqrt{n}}
   \right]
   \ge
   1-\tau\,.
  \end{gather}

\end{theorem*}
\begin{proof}
We want to reinforce the coherent use of the weak law of large numbers.
To this end, we verify 
\begin{align*}
  n\,\E[w(T,X)\,Y(T)]
  \ 
  &=
  \ 
  \E[Y(1)]
  \,,
  \\
  n^2\,
  \mathbf{Var}[w(T,X)\,Y(T)]
  \ 
  &=
  \ 
    \E[
    (
      Y(1)
    )^2
    \,/\,
    \pi(X)
    ]
    -
    \E[Y(1)]^2
    \,.
\end{align*}
Essentially, the random weight $w(T,X)$ acts on $Y(T)$ through $T\,/\,\pi(X)$.
In a second we will see, how it induces independence of observed outcome $Y(T)$ and treatment $T$.
To showcase this detail we added the $n$ and $n^2$ factors in the above display.
The calculations go as follows.
\begin{align}
  \begin{split}
  n\,\E[w(T,X)\,Y(T)]
  &\ 
  =
  \ 
  \E
  \left[ 
    Y(T)
    \cdot
    (T\,/\,\pi(X))
  \right]
  \\
  &\ =\ 
  \E
  \left[ 
    Y(1)
    \,
    /
    \,
    \pi(X)
    \,
    \vert
    \,
    T=1
  \right]
  \cdot
  \P[T=1]
  \\
  &\ =\ 
  \int_\mathcal{X}
  \E
  \left[ 
    Y(1)
    \,
    \vert
    \,
    X=x
    ,
    T=1
  \right]
  \cdot
  (
  \P[T=1]
  \,
  /
  \,
  \pi(x)
  )
  \,
  \P_{X|T}(dx\,|\,1)
  \\
  &\ =\ 
  \int_\mathcal{X}
  \left[ 
    Y(1)
    \vert
    X=x
  \right]
  \P_X(dx)
  =
  \E[Y(1)]
  .
\end{split}
\end{align}
The first equality comes from the definition of the weights.
The second, third and fifth equality stem from 
$
  T\in \left\{ 0,1 \right\}
$,
and the law of total expectation, applied with $T$ and $X$.
The fourth equality is due to the assumption of weak unconfoundedness, i.e., 
\begin{gather}
  (Y(0),Y(1))\perp\!\!\!\perp T\, |\, X\,.
\end{gather}
The density transformation is due to Bayes's Theorem.
With slight modifications in the above argument, it follows
\begin{gather}
  n^2
  \E
  \left[ 
  \Big( 
    Y(T)
    \cdot
    (T\,/\,\pi(X))
  \Big)
  ^2
  \right]
  \ 
    =
    \ 
    \E
    \Big[
    (Y(1))^2
    \,
    /
    \,
    \pi(X)
    \Big]
    \,.
\end{gather}
We omit the details.
Invoking the weak law of large numbers finishes the proof.
\end{proof}

\begin{gather}
  \pi(x)=\P[T=1|X=x]
\end{gather}


We readily calculate the learning rate 
\begin{gather}
  \sqrt{
    \mathbf{V}^*
    \frac{\log\log n}{n}
  }
  .
\end{gather}
