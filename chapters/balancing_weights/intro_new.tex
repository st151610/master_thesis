We consider a study population in which we want to test the effect of a treatment.
We introduce the \textbf{indicator of treatment} $T\in \left\{ 0,1 \right\}$.
For each treatment level there exist the \textbf{marginal potential outcomes}
$(Y(0),Y(1))$. We would like to estimate $\E[Y(1)]$. If we succeed the same technique
shall yield an estimate of $\E[Y(0)]$. We shall compare $\E[Y(1)]$ and $\E[Y(0)]$ and 
find out something about the effect of the treatment in the population.

The data we acquire is independent and identically distributed. But usually
\begin{gather}
  Y(1)|T=1 \nsim Y(1) 
  \,,
\end{gather}
that is, $T=1$ carries more information than observing the outcome under treatment.
We say that $Y(1)|T=1$ is \textbf{confounded}. To extract that plus of information from $T=1$ and put it where it belongs by collecting more data.
We gather it in $X\in \R^d$ and assume
\begin{gather}
  (Y(0),Y(1))
  \perp
  T
  \ 
  |
  \ 
  X
  \,,
\end{gather}
that is, \textbf{conditional unconfoundedness}.
Thus, we end up collecting $N\in \mathbb{N}$ independent and identically distributed copies of 
$(T,X,Y(T))$. For convenience, we assume that the first $n\in \mathbb{N}$ copies have $T=1$.

A natural estimator for $\E[Y(1)]$ is the weighted mean
\begin{gather}
  \frac{1}{n}
  \sum_{i=1}^{n} 
  w_i Y_i
  \,.
\end{gather}
The weights should satisfy (in a broader sense)
\begin{gather}
  w_i\cdot Y_i \to Y(1)
  \qquad 
  \text{for}\ 
  N
  \ 
  \to
  \ 
  \infty
  \,.
\end{gather}
One class of such weights has been recently analyzed in \cite{Wang2019}.
We take ideas and extend.

\subsection*{The algorithm}
\begin{fproblem}
  \label{bw:1:primal}
\begin{align*}
  %%%% objective %%%%
    &\underset{w_1, \ldots, w_n \in \R}
    {\text{minimize}}
    &&\qquad\qquad
    \sum_{i = 1}^{n} 
    f(w_i)
    &&&
    \\
    %%%% w_i T_i >= 0 %%%%
    &\text{subject to}
    &&\qquad\qquad
    w_i 
    \ge
    0
    &&&
    \qquad
    \text{for all}\ 
    i \in \left\{ 1, \ldots, n \right\}
    \,,
    \\
    %%%% 1/n sum w = 1 %%%%
    & 
    &&\qquad\qquad
    \frac{1}{N}
    \sum_{i=1}^{n} 
    w_i
    =1
    \\
    %%%% box constraints %%%%
    & 
    &&\qquad
    \left| 
      \frac{1}{N} 
      \left( 
      \sum_{i = 1}^{n} 
      w_i
      B_k(X_i)
      -
      \sum_{i=1}^{N} 
      B_k(X_i)
      \right)
    \right|
    \ 
    \le 
    \ 
    \delta_k
    &&&
    \qquad
    \text{for all}\ 
    k \in \left\{ 1, \ldots, N \right\}
    \,.
\end{align*}
\end{fproblem}
This is a (convex) optimization problem. We will talk about the \textbf{objective function} $f$ and
the \textbf{equality} and \textbf{inequality constraints}, especially about the 
\textbf{regression basis} $B$.

\section*{Objective Function}
Strictly speaking, we consider the sum
\begin{gather}
  [w_1,\ldots,w_n]^\top
  \ 
  \mapsto
  \ 
  \sum_{i=1}^{n} 
  f(w_i)
\end{gather}
as the objective function. It is natural to consider the dual formulation of the optimization problem. This involves the \textbf{convex conjugate}(cf.Definition~?) of the original objective function. We show in Example that for the sum this is
\begin{gather}
  [\lambda_1,\ldots,\lambda_n]^\top
  \ 
  \mapsto
  \ 
  \sum_{i=1}^{n} 
  f^*(\lambda_i)
\end{gather}
where $f^*$ is the Legendre transformation of $f$.

In the sequel we need $f$ to be strictly convex and its convex conjugate (or Legendre transformation) to be continuously differentiable and strictly non-decreasing.
Two popular choices of $f$ are the \textbf{negative entropy} and the \textbf{sample variance}.
\subsection*{Negative Entropy}
We define the negative entropy to be
\begin{gather}
  f
  \colon
  [0,\infty)
  \to
  \R
  ,\quad
  w
  \mapsto
  \begin{cases}
    0\quad\text{if}\ w=0,\\
    w\log w\quad
    \text{else}.
  \end{cases}
\end{gather}
It is strictly convex. To compute its Legendre transformation we note, that
\begin{gather}
  (f^{'})^{-1}
  =
  \lambda\mapsto
  e^{\lambda-1}
\end{gather}
Thus
  \begin{align*}
  f^*
  (\lambda)
  &
  \ 
  =
  \ 
  \lambda
    \cdot
    (f^{'})^{-1}(\lambda)
  \ 
    -
  \ 
    f
    \left( 
      (f^{'})^{-1}(\lambda)
    \right)
    \\
  &
  \ 
  =
  \ 
  \lambda
    \cdot
  e^{\lambda-1}
  \ 
    -
  \ 
  e^{\lambda-1}
  \log
  \left( 
  e^{\lambda-1}
  \right)
  \\
  &
  \ 
  =
  \ 
  e^{\lambda-1}
  \,.
  \end{align*}
  Thus $f^*$ is smooth and strictly non-decreasing.


  \subsection*{Sample Variance}
We define the sample variance to be
\begin{gather}
  f
  \colon
  \R
  \to
  \R
  ,\quad
  w
  \mapsto
  (w-1/n)^2
\end{gather}
It is strictly convex. To compute its Legendre transformation we note, that
\begin{gather}
  (f^{'})^{-1}
  =
  \lambda\mapsto
  \frac{\lambda}{2}
  +
  \frac{1}{n}
\end{gather}
Thus
  \begin{align*}
  f^*
  (\lambda)
  &
  \ 
  =
  \ 
  \lambda
    \cdot
    \left( 
  \frac{\lambda}{2}
  +
  \frac{1}{n}
    \right)
  \ 
    -
  \ 
    \left( 
    \left( 
  \frac{\lambda}{2}
  +
  \frac{1}{n}
    \right)
    -
    \frac{1}{n}
    \right)
    ^2
    \\
  &
  \ 
  =
  \ 
  \frac{\lambda^2}{4}
  +
  \frac{\lambda}{n}
  \,.
  \end{align*}
  Thus $f^*$ is smooth.
  To eliminate some variables in the optimization problem,
  we need $f^*$ also to be
  strictly non-decreasing. But the sample variance violates this assumption.

\section*{Constraints}
Let's turn our attention to the constraints.
The first constraint makes sure we do not extrapolate from the poputation.
The second constraint norms the weights. 
The third constraint controls the bias of the resulting estimator.
\section*{Regression Basis}
We adopt ideas from \cite{Gyorfi2002}. Another angle would be sieve estimates\cite{Newey1997a} where the number of basis functions can grow slower than $N$.
Their notion of (weak) consistency~\cite[Definitien~1.1]{Gyorfi2002} for noiseless estimands
is
\begin{gather}
  \E
  \left[ 
    \int_\mathcal{X}
    \left| 
    \sum_{k=1}^{N} 
    B_k(x)
    \cdot
    m(X_k)
    -
    m(x)
    \right|
    ^2
    \P_X(dx)
  \right]
  \to
  0
  \qquad
  \text{as}
  \ 
  n\to \infty
  \,.
\end{gather}
Universal consistency in this sense holds, if this is true for all distributions with
$\E[m(X)^2]<\infty$(cf.\cite[Definition~1.3]{Gyorfi2002}).

We adopt a slightly different notion of consistency. The next theorem dose the translation work. 
\begin{theorem}
  Assume
  $\E[m(X)^2]<\infty$
  and the basis function are 
(weak) universal consistency in the sense of 
\cite[Definitien~1.3]{Gyorfi2002}.
Then it holds for all $\varepsilon>0$
\begin{gather}
  \P
  \left[ 
    \left| 
    \sum_{k=1}^{N} 
    B_k(X)
    \cdot
    m(X_k)
    -
    m(X)
    \right|
    \ge
    \varepsilon
  \right]
  \to
  0
  \qquad
  \text{as}
  \ 
  n\to \infty
  \,.
\end{gather}
\end{theorem}
\begin{proof}
  By Markov's inequality it holds
  \begin{align*}
    &
  \P
  \left[ 
    \left| 
    \sum_{k=1}^{N} 
    B_k(X)
    \cdot
    m(X_k)
    -
    m(X)
    \right|
    \ge
    \varepsilon
  \right]
  \\
  &
  \ 
  \le
  \ 
  \frac
  {
  \E
  \left[ 
    \left| 
    \sum_{k=1}^{N} 
    B_k(X)
    \cdot
    m(X_k)
    -
    m(X)
    \right|
    ^2
  \right]
  }
  {\varepsilon^2}
  \\
  &
  \ 
  =
  \ 
  \frac
  {
  \E
  \left[ 
    \E
    \left[ 
    \left| 
    \sum_{k=1}^{N} 
    B_k(X)
    \cdot
    m(X_k)
    -
    m(X)
    \right|
    ^2
    |
    X_1,
    \ldots,
    X_N
    \right]
  \right]
  }
  {\varepsilon^2}
  \\
  &
  \ 
  =
  \ 
  \frac
  {
  \E
  \left[ 
    \int_\mathcal{X}
    \left| 
    \sum_{k=1}^{N} 
    B_k(x)
    \cdot
    m(X_k)
    -
    m(x)
    \right|
    ^2
    \P_X(dx)
  \right]
  }
  {\varepsilon^2}
  \,.
  \end{align*}
  The last equality is due to \cite[(1.2)]{Gyorfi2002}.
  By the weak universal consistency of $B$
  the last expression goes to $0$ as $N\to \infty$.
\end{proof}

Classical choices of the basis functions are \textbf{partitioning estimates} and 
\textbf{kernel estimates}(cf.\cite[§4,§5]{Gyorfi2002}).

\subsection*{Partitioning Estimates}
We consider a partition
$
  \mathcal{P}_N
  =
  \left\{ 
    A_{N,1}
    ,
    A_{N,2}
    ,
    \ldots
  \right\}
$
of $ \R^d $
and define
$ A_N(x) $ to be the cell of $ \mathcal{P}_N $ containing $x$.
We define $N$ basis functions $B_k$ of the covariates by
\begin{gather*}
  B_k(x)
  :=
  \frac{
  \mathbf{1}_{X_k \in A_N(x)}
  }{
  \sum_{j=1}^{N} 
  \mathbf{1}_{X_j \in A_N(x)}
  }
  \,,
  \qquad
  k=
  1,\ldots,N
  \,.
\end{gather*}
The euclidian norm of the basis functions is bounded above by $1$.
\begin{gather*}
  \norm{B(x)}^2
  =
  \sum_{k=1}^{n} 
  \left( 
  \frac{
  \mathbf{1}_{X_k \in A_n(x)}
  }{
  \sum_{j=1}^{n} 
  \mathbf{1}_{X_j \in A_n(x)}
  }
  \right)
  ^2
  \le
  \sum_{k=1}^{n} 
  \frac{
  \mathbf{1}_{X_k \in A_n(x)}
  }{
  \sum_{j=1}^{n} 
  \mathbf{1}_{X_j \in A_n(x)}
  }
  =1
  \,.
\end{gather*}
Under mild conditions, the basis functions are universally consistent.
\begin{theorem}
  \label{bw:i:bf:pe:th:c}
  If for each sphere $S$ centered at the origin 
  \begin{gather}
    \max
    _
    {
      j\colon
      A_{N,j} 
      \cap
      S
      \neq
      \emptyset
    }
    \mathrm{diam}
    \ 
      A_{N,j} 
      \ 
      \to
      \ 
      0
      \qquad
      \text{for}\ 
      N\to \infty 
  \end{gather}
  and
  \begin{gather}
    \frac
    {
    \#
    \left\{  
      j\colon
      A_{N,j} 
      \cap
      S
      \neq
      \emptyset
    \right\}
    }
    {N}
      \ 
      \to
      \ 
      0
      \qquad
      \text{for}\ 
      N\to \infty 
  \end{gather}
  then the partitioning regression function estimate 
  (definition)
  is
  universally consistent (definition).
\end{theorem}
\begin{proof}
  \cite[Theorem~4.2.]{Gyorfi2002}
\end{proof}
\begin{corollary}
Assume
  $\E[m(X)^2]<\infty$
  and the basis functions $B$ belong to a partitioning estimate.
  Furthermore assume that the conditions of 
  Theorem~\ref{bw:i:bf:pe:th:c} are met.
Then it holds for all $\varepsilon>0$
\begin{gather}
  \P
  \left[ 
    \left| 
    \sum_{k=1}^{N} 
    B_k(X)
    \cdot
    m(X_k)
    -
    m(X)
    \right|
    \ge
    \varepsilon
  \right]
  \to
  0
  \qquad
  \text{as}
  \ 
  n\to \infty
  \,.
\end{gather}
\end{corollary}
\subsection*{Kernel Estimates}
Let $K\colon \R^d\to [0,1]$ (bounded kernel) and $h_n>0$ (bandwith).
For examples see \cite[§5.1.]{Gyorfi2002}.
We define
\begin{gather}
  B_k(x)
  :=
  \frac
  {
    K \left( \frac{x-X_k}{h_n} \right)
  }
  {
    \sum_{i=1}^{N} 
    K \left( \frac{x-X_i}{h_n} \right)
  }
  \,.
\end{gather}
By the boundedness of the kernel it follows
$\norm{B(x)}\le 1$.
\begin{theorem}
  \label{bw:i:bf:ke:th:c}
  Assume that there are 
  balls
  $S_{0,r}$ of radius $r$ 
  and
  balls
  $S_{0,R}$ of radius $R$ 
  centered at the origin with $0<r\le R$, and a constant $b>0$ such that
  \begin{gather}
    \mathbf{1}_{\left\{ x\in S_{0,R} \right\}}
    \ge
    K(x)
    \ge
    b
    \cdot
    \mathbf{1}_{\left\{ x\in S_{0,r} \right\}}
  \end{gather}
\end{theorem}
(boxed kernel). Then for bandwiths with $h_n\to0$ and
$n\cdot h_n^d\to\infty$ as $n\to \infty$ the kernel estimate is weakly universally consistent.
\begin{corollary}
Assume
  $\E[m(X)^2]<\infty$
  and the basis functions $B$ belong to a kernel estimate.
  Furthermore assume that the conditions of 
  Theorem~\ref{bw:i:bf:ke:th:c} are met.
Then it holds for all $\varepsilon>0$
\begin{gather}
  \P
  \left[ 
    \left| 
    \sum_{k=1}^{N} 
    B_k(X)
    \cdot
    m(X_k)
    -
    m(X)
    \right|
    \ge
    \varepsilon
  \right]
  \to
  0
  \qquad
  \text{as}
  \ 
  n\to \infty
  \,.
\end{gather}
\end{corollary}

In the sequel we mainly work with the dual problem.

\section*{Dual Problem}
\begin{ftheorem*}
  The dual of 
  Problem~\ref{bw:1:primal}
  is the unconstrained optimization problem 
\begin{gather*}
  \underset
  {\lambda_0,\ldots,\lambda_{N}\in \R}
  {
    \mathrm{minimize}
  }
  \quad
  \frac{1}{N}
\sum_{i=1} 
  ^N
  \left[ 
    \,
  T_i
  \cdot
  f^*
  \!
  \left( 
\lambda_0
+
\inner
{B(X_i)}
{\lambda}
  \right)
  \ 
-
\ 
  \left( 
\lambda_0
+
\inner
{B(X_i)}
{\lambda}
  \right)
  \,
  \right]
  \ 
+
\ 
\inner
{\delta}
{|\lambda|}
  \,.
\end{gather*}
  where
  \begin{gather*}
  f^*
  \,
  \colon
  \, 
  \R
  \ 
  \to
  \ 
  \R
  \,
  ,
  \qquad 
  x^*
  \ 
  \mapsto
  \ 
    x^*
    \!
    \cdot
    (f^{'})^{-1}(x^*)
  \ 
    -
  \ 
    f
    \left( 
      (f^{'})^{-1}(x^*)
    \right)
  \end{gather*}
  is the Legendre transformation of $f$,
  the vector
  $
    B(X_i)
    =
    \left[ 
      B_1(X_i)
      ,
      \ldots
      ,
      B_n(X_i)
    \right]
    ^\top
  $
  denotes 
  the $N$ basis functions of the covariates 
  of unit $i\in \left\{ 1, \ldots, N \right\}$
  and
  $
    \left| \lambda \right|
    =
    \left[ 
      \left| \lambda_1 \right|
      ,
      \ldots
      ,
      \left| \lambda_N \right|
    \right]
    ^\top
    ,
  $
  where $\left| \,\cdot\, \right|$
  is the absolute value of a real-valued scalar.
  Moreover,
  if $\lambda^\dagger$
  is an optimal solution of the above problem 
  then the optimal solution to problem
  Problem~\ref{bw:1:primal}
  is given by
  \begin{gather*}
    w_i^\dagger
    \ 
    =
    \ 
    (f^{'})^{-1}
    \left( 
      \inner
      {B(X_i)}
      {\lambda^\dagger}
      +
      \lambda_0^\dagger
    \right)
    \qquad
    \text{for}\ 
    i\in \left\{ 1\ldots,n \right\}
    \,.
  \end{gather*}
\end{ftheorem*}
\subsection*{Plan of proof}
  We want to apply Theorem~\ref{cv:ts:th}.
  To this end, we find the suitable \textbf{matrix notation}.
  (\cite[p.20-22]{Wang2019} fail to do so. The problem is, that they divide by 0 in the second display on p.21).
  Theorem~\ref{cv:ts:th} covers only parts of the constraints, 
  so we apply the argument in \cite[p.19-20]{Wang2019} to eliminate the remaining \textbf{non-negativity constraints}. 
%%%%%%%%%%%%%%%  
%%%% PROOF %%%%
%%%%%%%%%%%%%%%
\begin{proof}
  \subsubsection*{Matrix notation}
  We consider the vector of basis functions of the covariates 
  of unit $i\in \left\{ 1, \ldots, n \right\}$, that is,
\begin{align*}
    B(X_i)
    &
    \ 
    :=
    \ 
    \left[ 
      B_1(X_i)
      ,
      \ldots
      ,
      B_N(X_i)
    \right]
    ^\top
    \,,
    \intertext{the constraints vectors}
d
    &
    \ 
    :=
    \ 
    \begin{bmatrix}
      0_n
      \\
      -N\cdot\delta 
      \pm\,
      \sum_{i = 1}^{N} B_k(X_i)
    \end{bmatrix}
    \,,
    \\
  a
&
  \ 
    :=
    \ 
    N
    \intertext{
  the matrix of the basis functions of the treated
  }
    \mathbf{B}(\mathbf{X})
    &
    \ 
    :=
    \ 
    \begin{bmatrix}
      B(X_1), \ldots, B(X_n)
    \end{bmatrix}
    \intertext{
      and
  the constraint matrices
    }
    \mathbf{U}
    &
    \ 
    :=
    \ 
    \begin{bmatrix}
      \mathbf{I}_n
      \\
      \pm\,\mathbf{B}(\mathbf{X})
    \end{bmatrix}
        \,.
        \\
    \mathbf{A}
    &
    \ 
    :=
    \ 
      \mathrm{1}_n
  \end{align*}
  By Example~\ref{cv:cc:ex}
the convex conjugate of the objective function 
of 
  Problem~\ref{bw:1:primal} is
\begin{gather*}
  [x^*_1,\ldots,x^*_n]^\top
  \ 
  \mapsto
  \ 
  \sum_{i=1}^n f^*(x^*_i)
  \,,
\end{gather*}
Before we apply Theorem~\ref{cv:ts:th}
we eliminate the non-negativity constraints.
To this end, we consider the objective function $G$ of the dual
problem and update it until we reach its final form.
We write
\begin{gather}
  \lambda_d
  =:
  \begin{bmatrix}
    \rho\\
    \lambda^+\\
    \lambda^-
  \end{bmatrix}
\end{gather}
\begin{align*}
G(\lambda_d,\lambda_0)
&
\ 
=
\ 
G
\left( 
\rho,
  \lambda^+,
  \lambda^-,
\lambda_0
\right)
\\
&
\ 
:=
\ 
    \sum_{i = 1}^{N} 
    -
    f^*
    \left( 
    \rho_i
    +
    \lambda_0
    +
    \inner{B(X_i)}
    {\lambda^+-\lambda^-}
    \right)
    \ 
    +
    \ 
    \left( 
    \lambda_0
    +
    \inner{B(X_i)}
    {\lambda^+-\lambda^-}
    \right)
\\
&
    \qquad 
    -
    \ 
    N
    \cdot
    \inner{\delta}
    {\lambda^++\lambda^-}
\end{align*}
Since we maximize $G$ and $f^*$ is strictly non-decreasing, 
$\rho=0$ is optimal. We update $G$.
\begin{align*}
G
\left( 
  \lambda^+,
  \lambda^-,
\lambda_0
\right)
&
\ 
=
\ 
    \sum_{i = 1}^{N} 
    -
    f^*
    \left( 
    \lambda_0
    +
    \inner{B(X_i)}
    {\lambda^+-\lambda^-}
    \right)
    \ 
    +
    \ 
    \left( 
    \lambda_0
    +
    \inner{B(X_i)}
    {\lambda^+-\lambda^-}
    \right)
\\
&
    \qquad 
    -
    \ 
    N
    \cdot
    \inner{\delta}
    {\lambda^++\lambda^-}
\end{align*}

\subsubsection*{Non-negativity constraints}

  Next we want to remove the non-negativity constraints on $\lambda^\pm$.
  We show 
  for all $i \in \left\{ 1,\ldots,N \right\}$
\begin{alignat*}{2}
  \text{either}
  &
  &&
  \qquad
      \lambda_i^+ > 0
  \\
  \text{or}
  &
  &&
  \qquad
      \lambda_i^- > 0
  \,.
\end{alignat*}
Assume towards a contradiction that 
there exists
$i \in \left\{ 1,\ldots,N \right\}$
such that
$
      \lambda_i^+ > 0
$
and
$
      \lambda_i^- > 0
$ 
and that 
$\lambda^\pm$ is optimal.
Consider
  \begin{gather}
    \tilde{\lambda}
    \ 
    :=
    \ 
    \begin{bmatrix}
      \ 
      \lambda_1^+,
      \ldots,
      \ 
      \lambda_i^+
      \!
      -
      (
      \lambda_i^+
      \!
      \land
      \lambda_i^-
      )\,,
      \ 
      \ldots,
      \lambda_N^+,
      \ 
      \lambda_1^-,
      \ldots,
      \lambda_i^-
      \!
      -
      (
      \lambda_i^+
      \!
      \land
      \lambda_i^-
      )\,,
      \ 
      \ldots,
      \lambda_N^-
      \,,
      \lambda_0
    \end{bmatrix}
    ^\top
    \,.
  \end{gather}
  Since 
  $
      \lambda_i^\pm
      -
      (
      \lambda_i^+
      \!
      \land
      \lambda_i^-
      )
      \ge 
      0
  $,
  the perturbed vector $\tilde{\lambda}$ is in the domain of the 
  optimization problem.
  But 
  \begin{align}
  G(\tilde{\lambda})
  -
  G(\lambda)
  \ 
  =
  \ 
  2
  N
  \cdot
  \delta_i
  \cdot
      (
      \lambda_i^+
      \!
      \land
      \lambda_i^-
      )
  \ 
  >
  \ 
  0
  \,,
  \end{align}
  which contradicts the optimality of $\lambda$.
But then 
$
\lambda^\pm_i
\ge 0
$
collapses to
$
\lambda_i\in \R
$ 
for all
$i\in \left\{ 0,\ldots,N \right\}$, that is,
$ \lambda_i=\lambda_i^+\!-\lambda_i^- $.
Note that
$ |\lambda_i|=\lambda_i^+\!+\lambda_i^- $.

We update the objective function one more time.
Multiplying with $-1/N$ and introducing $T$ we get

\begin{gather*}
  \underset
  {\lambda_0,\ldots,\lambda_{N}\in \R}
  {
    \mathrm{minimize}
  }
  \quad
  \frac{1}{N}
\sum_{i=1} 
  ^N
  \left[ 
    \,
  T_i
  \cdot
  f^*
  \!
  \left( 
\lambda_0
+
\inner
{B(X_i)}
{\lambda}
  \right)
  \ 
-
\ 
  \left( 
\lambda_0
+
\inner
{B(X_i)}
{\lambda}
  \right)
  \,
  \right]
  \ 
+
\ 
\inner
{\delta}
{|\lambda|}
  \,.
\end{gather*}
We apply Theorem~\ref{cv:ts:th} to finish the proof.
\end{proof}
We have gathered all the tools to tackle consistency of the weighted mean.
