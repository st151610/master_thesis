\begin{definition}
  \emph{(Hermitian Dilation)}
  The Hermitian dilation
  \begin{gather*}
    \mathfrak{H} : \C^{d_1 \times d_2} \to \mathbb{H}_{d_1 \times d_2}
  \end{gather*}
  is a map from a general matrix to an Hermitian matrix defined by
  \begin{gather}
    \label{ rmineq_hermitian_dilation } 
    \mathfrak{H}(B)
    :=
    \begin{bmatrix}
      0   & B \\
      B^* & 0 \\
    \end{bmatrix}
  \end{gather}
\end{definition}




%%%%%%%%%%%%%%%%%%%%
% MATRIX BERNSTEIN %
%%%%%%%%%%%%%%%%%%%%
\begin{theorem}
  \emph{(Matrix Bernstein Inequality)}
  \label{rmineq_bernstein}
  Let $A_1, \ldots, A_n$ be a independent, random matrices with dimension 
  $d_1 \times d_2$
  . Assume that
    \begin{gather}
      \label{rmineq_bernstein_cond_1}
      \E(A_k)=0\quad\text{and}\quad\norm{A_k}\le L \quad \text{for each}\quad  k\in \{1,\ldots,n\}.
    \end{gather}
    Introduce the random matrix
      \begin{gather*}
        S:=\sum_{k=1}^n A_k.
      \end{gather*}
    Let $v(S)$ be the matrix variance statistic of the sum:
      \begin{align}
        \label{rmineq_bernstein_cond_2}
        v(S):= \norm{\E[SS^\intercal ]} \lor \norm{\E[S^T S]} 
             = \norm{\sum_{k=1}^n\E[A_kA_k^T]} \lor \norm{\sum_{k=1}^n\E[A_k^T A_k]} .
      \end{align}
    Then
      \begin{align}
        \label{rmineq_bernstein_expectation_bound}
        \E \left[ \norm{S} \right]\le \sqrt{2v(S)\log(d_1+d_2)} + \frac{1}{3}L\log(d_1+d_2).
      \end{align}
    Furthermore, 
      \begin{gather}
        \label{rmineq_bernstein_probability_bound}
        \P[\norm{S}\ge t]\le(d_1+d_2)\exp\left(\frac{-t^2/2}{v(S)+Lt/3} \right)
        \quad
        \text{for all $t \ge 0$}
        .
      \end{gather}
\end{theorem}

%%%%%%%%%%%%%%%%%%%%%%%%%%%
% SCALAR MATRIX BERNSTEIN %
%%%%%%%%%%%%%%%%%%%%%%%%%%%
\begin{corollary}
  \emph{(Scalar Bernstein Inequality)}
  \label{rmineq_bernstein_scalar}
Let
$
  X_1, \ldots , X_n
$
be centered, independent real random variables with
$
  \norm{X_i}_\infty
  \le
  L
$
and
$
  \E(X_1^2)
  \le
  \sigma^2
$
for all $i = 1, \ldots, n$.
Then it holds for all $ t \ge 0 $
\begin{gather}
  \label{rmineq_bernstein_scalar_result}
  \P
  \left( 
    \left| 
      \sum_{i = 1}^{n} 
        X_i
    \right|
    \ge
    t
  \right)
  \le
  2
  \exp
  \left( 
    -
    \frac{1}{2}
    \frac{t^2}
    {
      n\sigma^2 
      +
      Lt/3
    }
  \right)
\end{gather}
\end{corollary}
\begin{proof}
  We verify the scalar versions of
  \eqref{rmineq_bernstein_cond_1}
  and
  \eqref{rmineq_bernstein_cond_2}.
  Since the 2- and $\infty$- norm coincide in one dimension and
  the $X_i$ are centered and independent we get
  \begin{gather}
    v(S)
    =
    \sum_{i = 1}^{n}
        \E(X_i^2)
    \le
    n\sigma^2
  \end{gather}
  We get \eqref{rmineq_bernstein_scalar_result}
  by applying Theorem~\ref{rmineq_bernstein}.
\end{proof}

%%%%%%%%%%%%%%%%%%%%%%%%%%%%%%%
% intrinsic dimension version %
%%%%%%%%%%%%%%%%%%%%%%%%%%%%%%%
\begin{definition}
  \emph{Intrinsic Dimension}
  \label{rmineq_intrinsic_bernstein}
  For a positive-semidefinite matrix $A$,
  the intrinsic dimension is the quantity
  \begin{gather*}
    \text{intdim}
    (A)
    :=
    \frac{\text{tr}A}{\norm{A}_2}
    ,
  \end{gather*}
  where tr is the trace of a matrix.
\end{definition}
\begin{theorem}
  \emph{(Intrinsic Matrix Bernstein)}
  \label{rmineq_bernstein}
  Let $(A_k)_{1\le k \le n}$  be a finite sequence of independent, random matrices with the same size. Assume that
    \begin{gather}
      \E(A_k)=0\quad\text{and}\quad\norm{A_k}\le L \quad \text{for each}\quad  k\in \{1,\ldots,n\}.
    \end{gather}
  Introduce the random matrix
      \begin{gather*}
        S:=\sum_{k=1}^n A_k.
      \end{gather*}
  Let 
  $V_1$ 
  and
  $V_2$ 
  be semidefinite upper bounds for the matrix-valued variances
  $\mathrm{Var}_1(S)$
  and
  $\textbf{Var}_2(S)$:
  \begin{align*}
    V_1 
    &\succcurlyeq 
    \text{Var}_1(S)
    :=
    \E(S)
    =
    \sum_{k = 1}^{n}
    \E(A_k A_k^T),\\
    V_2 
    &\succcurlyeq 
    \text{Var}_2(S)
    :=
    \E(S)
    =
    \sum_{k = 1}^{n}
    \E(A_k^T A_k)
    .
  \end{align*}
  Define an intrinsic dimension bound and a variance bound
  \begin{gather}
    \label{ rmineq_intrinsic_bernstein_intdim_var_bound }
    d
    :=
    \text{intdim}
    \begin{bmatrix}
      V_1 & 0 \\
      0   & V_2
    \end{bmatrix}
    \qquad
    \text{and}
    \qquad
    v
    :=
    \max
    \left\{ \norm{V_1}_2, \norm{V_2}_2 \right\}
    .
  \end{gather}
  Then 
  it holds
      \begin{align}
        \label{rmineq_intrinsic_bernstein_expectation_bound}
        \E\norm{S}\le \text{Const.}\left( \sqrt{v\log(d+1)} + L\log(d+1) \right).
      \end{align}
    Furthermore, for all 
  $
    t
    \ge
    \sqrt{v}
    +
    \frac{L}{3}
  $,
      \begin{gather}
        \label{rmineq_intrinsic_bernstein_probability_bound}
        \P(\norm{S}\ge t)\le 4d \exp\left(\frac{-t^2/2}{v+Lt/3} \right).
      \end{gather}
\end{theorem}

%%%%%%%%%%%%%%%%%%%%%%%%%%%%%%%%%%%%%%%%%%%%%%%%%%%%
% INTRINSIC DIMENSION BERNSTEIN FOR RANDOM VECTORS %
%%%%%%%%%%%%%%%%%%%%%%%%%%%%%%%%%%%%%%%%%%%%%%%%%%%%
It turns out that for random vectors $ d \le 2 $. This is remarkable, because this property of the intrinsic dimension is invarient under the dimension of the vector. This fact motivates the following result:
\begin{corollary}
  \label{rmineq_intrinsic_bernstein_vector_corollary}
  Let $(A_k)_{1\le k \le n}\subseteq\R^{K}$ be a finite sequence of independent, random vectors. Assume that
    \begin{gather}
      \label{rmineq_bernstein_cond_1}
      \E(A_k)=0\quad\text{and}\quad\norm{A_k}\le L \quad \text{for each}\quad  k\in \{1,\ldots,n\}.
    \end{gather}
    Let $v(S)$ be the matrix variance statistic of the sum 
    as defined in
    \eqref{rmineq_bernstein_cond_2}
    .
  Then 
  it holds
      \begin{align}
        \label{rmineq_intrinsic_bernstein_expectation_bound}
        \E\norm{S}\le \text{Const.}\left( \sqrt{v(S)\log(3)} + L\log(3) \right).
      \end{align}
    Furthermore, for all 
  $
    t
    \ge
    \sqrt{v(S)}
    +
    \frac{L}{3}
  $,
      \begin{gather}
        \label{rmineq_intrinsic_bernstein_probability_bound}
        \P(\norm{S}\ge t)\le 8 \exp\left(\frac{-t^2/2}{v(S)+Lt/3} \right).
      \end{gather}
\end{corollary}
\begin{proof}
  First, let's verify 
  \begin{gather}
   \label{rmineq_intrinsic_bernstein_cor_1}
    d
    :=
    \text{intdim}
    \begin{bmatrix}
      \text{Var}_1(S) & 0 \\
      0   & \text{Var}_2(S)
    \end{bmatrix}
    \le
    2
    .
  \end{gather}
  Since
 
  \begin{gather}
  \text{tr} 
  B B^T 
  =
  B^T B
  \quad
  \text{and}
  \quad
  \norm{
    \begin{bmatrix}
      B B^T & 0 \\
      0   & B^T B
    \end{bmatrix}
  }_2
  =
  \max
  \left\{ 
    \norm{B B^T},
    B^T B
  \right\}
  \end{gather}
  $
  \text{
  for all 
  }
  B \in \R^K
  $
  it follows
  \begin{align}
    \text{intdim} 
    \begin{bmatrix}
      B B^T & 0 \\
      0   & B^T B
    \end{bmatrix}
    =
  \frac{
    \text{tr}B B^T
    +
    B^T B
  }
  {
  \max
  \left\{ 
    \norm{B B^T},
    B^T B
  \right\}
  }
  \le
  2  
  \end{align}

$
  \text{
  for all 
  }
  B \in \R^K
$.
By the linearity of the expectation  
it holds
for random vectors 
$X=(X_1, \ldots, X_K)$
with
$\norm{X}_\infty < \infty$

\begin{gather}
  \E(\text{tr} X X^T) 
  =
  \text{tr} \E (X X^T). 
\end{gather}
Indeed
\begin{align*}
  \E(\text{tr} X X^T) 
  &=
  \E \left( 
    \sum_{i = 1}^{n}
      X_i^2
  \right)
  =
  \sum_{i = 1}^{n}
    \E(
      X_i^2
    )  
  =
  \text{tr}\left( 
    E(X_i X_j)
  \right)_{1 \le i,j \le K}
  \\
  &=
  \text{tr}
    \E(X X^T)
\end{align*}
Note that by the linearity of the trace and the scalar product, the above also holds for finite sums of bounded random vectors. 
We have thus established \eqref{rmineq_intrinsic_bernstein_cor_1}.
Applying Theorem~\ref{rmineq_intrinsic_bernstein}
finishes the proof.
\end{proof}
