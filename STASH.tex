\begin{example}
  Consider a proper convex function
  $
    f:
    \R \to ( -\infty, \infty]
  $
  We would like to compute the convex conjugate of the function 
  $h$ defined by 
  $
    h(x) = f \left( x - \frac{1}{n} \right)
  $.
  For this we notice that
  \begin{gather*}
    h = f \circ L
  \end{gather*}
  where 
  \begin{gather*}
   L := x \mapsto x - \frac{1}{n} 
  \end{gather*}
  is a linear map.
  Since 
  $
    \text{Im}(L) = \R
  $
  and 
  $\text{dom}(f) \neq \emptyset$
  we have
  $
    \text{Im}(L) 
    \cap
    \text{ri}(\text{dom}(f) )
    \neq
    \emptyset
  $.
  Furthermore
  \begin{gather}
    (L^*)^{-1}(x^*)
    =
    \left\{ 
     \left( x \mapsto x + \frac{1}{n} \right) (x^*)
    \right\}
    =
    \left\{ x^* + \frac{1}{n} \right\}
  \end{gather}
  Then Theorem~\ref{cvxa_conjugate_chain_rule}
  yields
  \begin{gather}
    h^* = f^* \circ L^{-1} 
        =  f^* \circ  
        \left( x^* \mapsto x^* + \frac{1}{n} \right)
  \end{gather}

\end{example}

%%%%%%%%%%%%%%%%%%%%%%%%%%%%%%%%%%%%%%%%%%%%%%%%%%%%%%%%%%%%%%%%%%%%%%%%
We employ 
Theorem~\ref{cvxa_fenchel_theorem}
together with the box constraints in 
Problem~\eqref{primal_weighting_binary}
to obtain Proposition~\ref{ch_1_dual}.

To prove Proposition~\ref{ch_1_near_oracle}
we employ
Proposition~\ref{syu_1_result}
and 
Corollary~\ref{syu_taylor_corollary}
to get
\begin{align}
  \begin{split}
    & 
    G(\lambda^*_1 + \Delta) 
    -
    G(\lambda^*_1)
\\
    &\ge
    \frac{1}{n}
    \sum_{j = 1}^{n} 
      \left[ 
        -T_j n 
        \rho^{'} 
        \left( 
          B(X_j)^T \lambda^*_1
        \right)
        +
        1
      \right]
      \Delta^T B(X_j)
\\
    & +
    \frac{1}{2}
    \sum_{j = 1}^{n} 
      -T_j  
      \rho^{''} 
      \left( 
        B(X_j)^T 
        (
          \lambda^*_1 + \xi \Delta
        )
      \right)
      \Delta^T
      \left( 
        B(X_j)
        B(X_j)^T
      \right)
      \Delta
\\
    &-
    |\Delta|^T \delta
\\
    &\ge
    - \norm{\Delta}_2
    \left( 
      \norm{
        \frac{1}{n}
        \sum_{j = 1}^{n} 
          \left[ 
            -T_j n 
            \rho^{'} 
            \left( 
              B(X_j)^T \lambda^*_1
            \right)
            +
            1
          \right]
        B(X_j)
      }_2
      +
      \norm{\delta}_2
    \right)
\\
    &+
    n
    \norm{\Delta}^2_2
    \varphi_{\rho^{''}}
    \underline{\varphi_{aa^T}}
  \end{split}
\end{align}

Next we employ Bernstein inequality~\ref{rmineq_bernstein} to bound
\begin{align}
    \norm{
      \frac{1}{n}
      \sum_{j = 1}^{n} 
      \left[ 
        -T_j n 
        \rho^{'} 
        \left( 
          B(X_j)^T \lambda^*_1
        \right)
      +
      1
      \right]
      B(X_j)
    }_2
    \le
    \CP \Ctau \LearnRate
\end{align}
with probability $1 - \tau$.
Then for 
$\norm{\Delta}_2$ large enough it holds
\begin{gather}
  G(\lambda^*_1 + \Delta) 
  -
  G(\lambda^*_1)
  >
  0
\end{gather}
with probability $1 - \tau$.
Thus by Proposition~\ref{syu_1_result}
  \begin{gather}
    \P
    \left( 
      \norm{
        \lambda^\dagger
        -
        \lambda^*_1
      }_2
      \le
      \norm{\Delta}_2
    \right)
    \ge 
    1 - \tau
    .
  \end{gather}


